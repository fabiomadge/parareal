\begin{frame}
    \frametitle{Idea}
    \begin{itemize}[<+->]
        \item Parallel-in-time: Decomposition of \(\left(t_0, T\right] = \bigcup_{i=0}^{P-1} \left(t_i, t_{i+1}\right]\)
        \item Solve the subintervals in parallel
        \item Combination of coarse \(\mathcal{G}\) (fast results) and fine \(\mathcal{F}\) (desired accuracy) solver
        \item Run them over multiple iterations
        \item Start with \(\mathcal{G}\!\left(y, t_{A}, t_{\Omega}\right)\) for initial value:\\\(y_{j}^{0} = \mathcal{G}\!\left(y(t_0), t_0, t_{j}\right), \quad j \in \left\{i \,\middle|\, 0 \leq i < P\right\}\)
        \begin{description}[\(t_{A}\)]
            \item[\(y\)]<.-> Initial value
            \item[\(t_{A}\)]<.-> Start time of simulation
            \item[\(t_{\Omega}\)]<.-> End time of simulation
        \end{description}
        \item Combine multiple results in later iterations
    \end{itemize}
    \end{frame}

\subsection{Corrections}

\begin{frame}
\begin{figure}[ht]
    \centering
        \begin{tikzpicture}[scale=1.7]
            % \documentclass[]{minimal}
% \usepackage{tikz}
% \usetikzlibrary{positioning}
% \usetikzlibrary{arrows}
% \usetikzlibrary{patterns}

% \begin{document}

% \pagestyle{empty}

% \begin{tikzpicture}[scale=2.2]


%legend
\uncover<1->{
\draw[] (3.25,.25) rectangle (4.75, 1.65);

\begin{scope}[anchor=west]
    \node[] at (3.8,1.45) {\(\mathcal{F}\)};
    \node[] at (3.8,1.25) {\(\mathcal{G}\)};
    \node[] at (3.8,.95) {$i_k$};
    \node[] at (3.8,.75) {$i_{k+1}$};
    \node[] at (3.8,.45) {\small correction};
\end{scope}

\draw[-, dotted,very thick] (3.4,1.45) -- (3.7,1.45);
\draw[-, loosely dotted,very thick] (3.4,1.25) -- (3.7,1.25);
\draw[pattern color=blue,pattern=north east lines] (3.4,.9) rectangle (3.7, 1);
\draw[pattern color=orange,pattern=north east lines] (3.4,.7) rectangle (3.7, .8);
\draw[->,red, ultra thick] (3.4,.45) -- (3.7,.45);

% coordinates
\draw[->] (0,0) -- (0,3) node[pos=1,below left] {y};

\draw[-] (0,0) -- (.25,0);
\draw[dotted] (.25,0) -- (.5,0);
\draw[-] (.5,0) -- (5.5,0);
\draw[dotted] (5.5,0) -- (5.75,0);
\draw[->] (5.75,0) -- (6,0) node[pos=1,below left] {x};

% threads
\draw[loosely dashed] (1,0) -- (1,3.3);
\draw[loosely dashed] (3,0) -- (3,3.3);
\draw[loosely dashed] (5,0) -- (5,3.3);
\node[below] at (1,0) {$t_{i-1}$};
\node[below] at (3,0) {$t_{i}$};
\node[below] at (5,0) {$t_{i+1}$};

% exact
\draw[thick] (.5,-.3) parabola  bend +(5,2.6) (5.5,2.3);
}
\begin{scope}[very thick]
\uncover<2->{
% pCoarse
\begin{scope}[loosely dotted, color=blue]
\draw (1,.8) to[out=45,in=215] (3,2.6);
\draw (3,2.15) to[out=39,in=192] (5,2.95);
\end{scope}
}
% pFine
\uncover<3->{
\begin{scope}[densely dotted, color=blue]
\draw (1,.8) to[out=44,in=212] (3,2.35);
\draw (3,2.15) to[out=28,in=186] (5,2.75);
\end{scope}
}
% coarse

\begin{scope}[loosely dotted, color=orange]
\uncover<4->{\draw (1,.5) to[out=45,in=215] (3,2.2);}
\uncover<7->{\draw (3,1.95) to[out=32,in=188] (5,2.7);}
\end{scope}

\end{scope}

% merges
\begin{scope}[red, ultra thick]
\uncover<5->{\draw[->, dashed] (2.96,2.6) -- (2.96,2.2);}
\uncover<6->{\draw[->] (3.04,2.35) -- (3.04,1.95);}
\uncover<8->{
\draw[->, dashed] (4.96,2.95) -- (4.96,2.7);
\draw[->] (5.04,2.75) -- (5.04,2.5);
}
\end{scope}


% \end{tikzpicture}

% \end{document}
        \end{tikzpicture}
    \caption{derivation of the correction}
    \label{fig:merge}
\end{figure}
\uncover<9->{
\begin{equation*}
    y_{j+1}^{k+1} = \mathcal{G}\!\!\left(y_j^{k+1}, t_j, t_{j+1}\right) + \mathcal{F}\!\!\left(y_j^k, t_j, t_{j+1}\right) - \mathcal{G}\!\!\left(y_j^k, t_j, t_{j+1}\right)
\end{equation*}
}
\end{frame}

% \begin{frame}
% \begin{equation*}
% y_{j+1}^{k+1} = \mathcal{G}\!\!\left(y_j^{k+1}, t_j, t_{j+1}\right) + \mathcal{F}\!\!\left(y_j^k, t_j, t_{j+1}\right) - \mathcal{G}\!\!\left(y_j^k, t_j, t_{j+1}\right)
% \end{equation*}
% \end{frame}

\begin{frame}
\begin{figure}[ht]
    \centering
        \scalebox{0.65}{\rotatebox{90}{\documentclass[]{minimal}
\usepackage{tikz}
\usetikzlibrary{positioning}
\usetikzlibrary{arrows}

\begin{document}

\pagestyle{empty}

\begin{tikzpicture}[scale=2]
    \newcommand\Threads{4}
    \newcommand\Threadsmm{3}

    % Y-Axis
    \draw[->] (-.3,.3) -- (-.3,-1*\Threads-1.7) node[pos=1,below left, sloped] {Iteration};
    \foreach \x in {0,...,\Threads} {
        \draw (-.4, -1.1*\x) -- (-.2, -1.1*\x) node [pos=0, left] {\x};
    }

    % X-Axis
    \draw[->] (-.3,.3) -- (.75*\Threads+1.4,.3) node[pos=1,above left] {Thread};
    \foreach \x in {0,...,\Threads} {
        \draw(.75*\x, .2) rectangle (.5+0.75*\x, .4);
        \node at (0.75*\x+.25,.5) {\x};
        }


    % Computatations
    \foreach \y in {0,...,\Threads} {
        \foreach \x in {\y,...,\Threads} {
            \draw[fill=red] (.75*\x,-\y-.1-.1*\x) rectangle (.5+0.75*\x, -\y-.1*\x);
            \draw[fill=green] (.75*\x,-\y-1-.1*\x) rectangle (.5+0.75*\x, -\y-.1-.1*\x);
        }
    }

    % Communication
    % init
    \foreach \x in {0,...,\Threadsmm} {
        \draw[->] (.5+0.75*\x,-.1*\x-.1) -- (.75+0.75*\x,-.1*\x-.1);
    }
    % first of every iteration
    \foreach \x in {0,...,\Threadsmm} {
        \draw[->] (.5+0.75*\x,-1.1*\x-1) -- (.75+0.75*\x,-1.1*\x-1.1);
    }
    % merges
    \foreach \y in {1,...,\Threadsmm} {
        \foreach \x in {\y,...,\Threadsmm} {
        % previous coarse
        \draw[shorten >=0.08cm] (.5+0.75*\x,-\y-.1*\x+.9) -- (.71+0.75*\x,-\y-.1-.1*\x);
        % fine
        \draw[ shorten >=0.09cm] (.5+0.75*\x,-\y-.1*\x) -- (.71+0.75*\x,-\y-.1-.1*\x);
        % coarse
        \draw[-o, shorten >=0.0cm] (.5+0.75*\x,-\y-.1*\x-.1) -- (.75+0.75*\x,-\y-.1-.1*\x);
        }
    }



\end{tikzpicture}


\end{document}}}
    \caption{Utilization and communication of the threads}
    \label{fig:sequence}
\end{figure}
\begin{equation*}
y_{j+1}^{k+1} = \mathcal{G}\!\!\left(y_j^{k+1}, t_j, t_{j+1}\right) + \mathcal{F}\!\!\left(y_j^k, t_j, t_{j+1}\right) - \mathcal{G}\!\!\left(y_j^k, t_j, t_{j+1}\right)
\end{equation*}
\end{frame}

\subsection{Example}

\begin{frame}
\frametitle{Logistic equation}
\begin{equation*}
    y'(t) = k \cdot y(t) \cdot (L - y(t))
\end{equation*}
With \(y(0)= \frac{L}{2}\):
\begin{equation*}
    y(t) = \frac{L}{1+e^{-k L t}}
\end{equation*}
\end{frame}

\begin{frame}
    \begin{figure}[ht]
        \centering
        \scalebox{0.8}{%% Creator: Matplotlib, PGF backend
%%
%% To include the figure in your LaTeX document, write
%%   \input{<filename>.pgf}
%%
%% Make sure the required packages are loaded in your preamble
%%   \usepackage{pgf}
%%
%% Figures using additional raster images can only be included by \input if
%% they are in the same directory as the main LaTeX file. For loading figures
%% from other directories you can use the `import` package
%%   \usepackage{import}
%% and then include the figures with
%%   \import{<path to file>}{<filename>.pgf}
%%
%% Matplotlib used the following preamble
%%   \usepackage{fontspec}
%%   \setmainfont{Times New Roman}
%%   \setsansfont{Lucida Grande}
%%   \setmonofont{Andale Mono}
%%
\begingroup%
\makeatletter%
\begin{pgfpicture}%
\pgfpathrectangle{\pgfpointorigin}{\pgfqpoint{5.000000in}{3.000000in}}%
\pgfusepath{use as bounding box, clip}%
\begin{pgfscope}%
\pgfsetbuttcap%
\pgfsetmiterjoin%
\definecolor{currentfill}{rgb}{1.000000,1.000000,1.000000}%
\pgfsetfillcolor{currentfill}%
\pgfsetlinewidth{0.000000pt}%
\definecolor{currentstroke}{rgb}{1.000000,1.000000,1.000000}%
\pgfsetstrokecolor{currentstroke}%
\pgfsetdash{}{0pt}%
\pgfpathmoveto{\pgfqpoint{0.000000in}{0.000000in}}%
\pgfpathlineto{\pgfqpoint{5.000000in}{0.000000in}}%
\pgfpathlineto{\pgfqpoint{5.000000in}{3.000000in}}%
\pgfpathlineto{\pgfqpoint{0.000000in}{3.000000in}}%
\pgfpathclose%
\pgfusepath{fill}%
\end{pgfscope}%
\begin{pgfscope}%
\pgfsetbuttcap%
\pgfsetmiterjoin%
\definecolor{currentfill}{rgb}{1.000000,1.000000,1.000000}%
\pgfsetfillcolor{currentfill}%
\pgfsetlinewidth{0.000000pt}%
\definecolor{currentstroke}{rgb}{0.000000,0.000000,0.000000}%
\pgfsetstrokecolor{currentstroke}%
\pgfsetstrokeopacity{0.000000}%
\pgfsetdash{}{0pt}%
\pgfpathmoveto{\pgfqpoint{0.467954in}{0.402778in}}%
\pgfpathlineto{\pgfqpoint{4.951389in}{0.402778in}}%
\pgfpathlineto{\pgfqpoint{4.951389in}{2.951389in}}%
\pgfpathlineto{\pgfqpoint{0.467954in}{2.951389in}}%
\pgfpathclose%
\pgfusepath{fill}%
\end{pgfscope}%
\begin{pgfscope}%
\pgfsetbuttcap%
\pgfsetroundjoin%
\definecolor{currentfill}{rgb}{0.000000,0.000000,0.000000}%
\pgfsetfillcolor{currentfill}%
\pgfsetlinewidth{0.803000pt}%
\definecolor{currentstroke}{rgb}{0.000000,0.000000,0.000000}%
\pgfsetstrokecolor{currentstroke}%
\pgfsetdash{}{0pt}%
\pgfsys@defobject{currentmarker}{\pgfqpoint{0.000000in}{-0.048611in}}{\pgfqpoint{0.000000in}{0.000000in}}{%
\pgfpathmoveto{\pgfqpoint{0.000000in}{0.000000in}}%
\pgfpathlineto{\pgfqpoint{0.000000in}{-0.048611in}}%
\pgfusepath{stroke,fill}%
}%
\begin{pgfscope}%
\pgfsys@transformshift{0.671747in}{0.402778in}%
\pgfsys@useobject{currentmarker}{}%
\end{pgfscope}%
\end{pgfscope}%
\begin{pgfscope}%
\pgftext[x=0.671747in,y=0.305556in,,top]{\rmfamily\fontsize{10.000000}{12.000000}\selectfont \(\displaystyle -6\)}%
\end{pgfscope}%
\begin{pgfscope}%
\pgfsetbuttcap%
\pgfsetroundjoin%
\definecolor{currentfill}{rgb}{0.000000,0.000000,0.000000}%
\pgfsetfillcolor{currentfill}%
\pgfsetlinewidth{0.803000pt}%
\definecolor{currentstroke}{rgb}{0.000000,0.000000,0.000000}%
\pgfsetstrokecolor{currentstroke}%
\pgfsetdash{}{0pt}%
\pgfsys@defobject{currentmarker}{\pgfqpoint{0.000000in}{-0.048611in}}{\pgfqpoint{0.000000in}{0.000000in}}{%
\pgfpathmoveto{\pgfqpoint{0.000000in}{0.000000in}}%
\pgfpathlineto{\pgfqpoint{0.000000in}{-0.048611in}}%
\pgfusepath{stroke,fill}%
}%
\begin{pgfscope}%
\pgfsys@transformshift{1.351055in}{0.402778in}%
\pgfsys@useobject{currentmarker}{}%
\end{pgfscope}%
\end{pgfscope}%
\begin{pgfscope}%
\pgftext[x=1.351055in,y=0.305556in,,top]{\rmfamily\fontsize{10.000000}{12.000000}\selectfont \(\displaystyle -4\)}%
\end{pgfscope}%
\begin{pgfscope}%
\pgfsetbuttcap%
\pgfsetroundjoin%
\definecolor{currentfill}{rgb}{0.000000,0.000000,0.000000}%
\pgfsetfillcolor{currentfill}%
\pgfsetlinewidth{0.803000pt}%
\definecolor{currentstroke}{rgb}{0.000000,0.000000,0.000000}%
\pgfsetstrokecolor{currentstroke}%
\pgfsetdash{}{0pt}%
\pgfsys@defobject{currentmarker}{\pgfqpoint{0.000000in}{-0.048611in}}{\pgfqpoint{0.000000in}{0.000000in}}{%
\pgfpathmoveto{\pgfqpoint{0.000000in}{0.000000in}}%
\pgfpathlineto{\pgfqpoint{0.000000in}{-0.048611in}}%
\pgfusepath{stroke,fill}%
}%
\begin{pgfscope}%
\pgfsys@transformshift{2.030363in}{0.402778in}%
\pgfsys@useobject{currentmarker}{}%
\end{pgfscope}%
\end{pgfscope}%
\begin{pgfscope}%
\pgftext[x=2.030363in,y=0.305556in,,top]{\rmfamily\fontsize{10.000000}{12.000000}\selectfont \(\displaystyle -2\)}%
\end{pgfscope}%
\begin{pgfscope}%
\pgfsetbuttcap%
\pgfsetroundjoin%
\definecolor{currentfill}{rgb}{0.000000,0.000000,0.000000}%
\pgfsetfillcolor{currentfill}%
\pgfsetlinewidth{0.803000pt}%
\definecolor{currentstroke}{rgb}{0.000000,0.000000,0.000000}%
\pgfsetstrokecolor{currentstroke}%
\pgfsetdash{}{0pt}%
\pgfsys@defobject{currentmarker}{\pgfqpoint{0.000000in}{-0.048611in}}{\pgfqpoint{0.000000in}{0.000000in}}{%
\pgfpathmoveto{\pgfqpoint{0.000000in}{0.000000in}}%
\pgfpathlineto{\pgfqpoint{0.000000in}{-0.048611in}}%
\pgfusepath{stroke,fill}%
}%
\begin{pgfscope}%
\pgfsys@transformshift{2.709672in}{0.402778in}%
\pgfsys@useobject{currentmarker}{}%
\end{pgfscope}%
\end{pgfscope}%
\begin{pgfscope}%
\pgftext[x=2.709672in,y=0.305556in,,top]{\rmfamily\fontsize{10.000000}{12.000000}\selectfont \(\displaystyle 0\)}%
\end{pgfscope}%
\begin{pgfscope}%
\pgfsetbuttcap%
\pgfsetroundjoin%
\definecolor{currentfill}{rgb}{0.000000,0.000000,0.000000}%
\pgfsetfillcolor{currentfill}%
\pgfsetlinewidth{0.803000pt}%
\definecolor{currentstroke}{rgb}{0.000000,0.000000,0.000000}%
\pgfsetstrokecolor{currentstroke}%
\pgfsetdash{}{0pt}%
\pgfsys@defobject{currentmarker}{\pgfqpoint{0.000000in}{-0.048611in}}{\pgfqpoint{0.000000in}{0.000000in}}{%
\pgfpathmoveto{\pgfqpoint{0.000000in}{0.000000in}}%
\pgfpathlineto{\pgfqpoint{0.000000in}{-0.048611in}}%
\pgfusepath{stroke,fill}%
}%
\begin{pgfscope}%
\pgfsys@transformshift{3.388980in}{0.402778in}%
\pgfsys@useobject{currentmarker}{}%
\end{pgfscope}%
\end{pgfscope}%
\begin{pgfscope}%
\pgftext[x=3.388980in,y=0.305556in,,top]{\rmfamily\fontsize{10.000000}{12.000000}\selectfont \(\displaystyle 2\)}%
\end{pgfscope}%
\begin{pgfscope}%
\pgfsetbuttcap%
\pgfsetroundjoin%
\definecolor{currentfill}{rgb}{0.000000,0.000000,0.000000}%
\pgfsetfillcolor{currentfill}%
\pgfsetlinewidth{0.803000pt}%
\definecolor{currentstroke}{rgb}{0.000000,0.000000,0.000000}%
\pgfsetstrokecolor{currentstroke}%
\pgfsetdash{}{0pt}%
\pgfsys@defobject{currentmarker}{\pgfqpoint{0.000000in}{-0.048611in}}{\pgfqpoint{0.000000in}{0.000000in}}{%
\pgfpathmoveto{\pgfqpoint{0.000000in}{0.000000in}}%
\pgfpathlineto{\pgfqpoint{0.000000in}{-0.048611in}}%
\pgfusepath{stroke,fill}%
}%
\begin{pgfscope}%
\pgfsys@transformshift{4.068288in}{0.402778in}%
\pgfsys@useobject{currentmarker}{}%
\end{pgfscope}%
\end{pgfscope}%
\begin{pgfscope}%
\pgftext[x=4.068288in,y=0.305556in,,top]{\rmfamily\fontsize{10.000000}{12.000000}\selectfont \(\displaystyle 4\)}%
\end{pgfscope}%
\begin{pgfscope}%
\pgfsetbuttcap%
\pgfsetroundjoin%
\definecolor{currentfill}{rgb}{0.000000,0.000000,0.000000}%
\pgfsetfillcolor{currentfill}%
\pgfsetlinewidth{0.803000pt}%
\definecolor{currentstroke}{rgb}{0.000000,0.000000,0.000000}%
\pgfsetstrokecolor{currentstroke}%
\pgfsetdash{}{0pt}%
\pgfsys@defobject{currentmarker}{\pgfqpoint{0.000000in}{-0.048611in}}{\pgfqpoint{0.000000in}{0.000000in}}{%
\pgfpathmoveto{\pgfqpoint{0.000000in}{0.000000in}}%
\pgfpathlineto{\pgfqpoint{0.000000in}{-0.048611in}}%
\pgfusepath{stroke,fill}%
}%
\begin{pgfscope}%
\pgfsys@transformshift{4.747596in}{0.402778in}%
\pgfsys@useobject{currentmarker}{}%
\end{pgfscope}%
\end{pgfscope}%
\begin{pgfscope}%
\pgftext[x=4.747596in,y=0.305556in,,top]{\rmfamily\fontsize{10.000000}{12.000000}\selectfont \(\displaystyle 6\)}%
\end{pgfscope}%
\begin{pgfscope}%
\pgftext[x=2.709672in,y=0.123861in,,top]{\rmfamily\fontsize{10.000000}{12.000000}\selectfont t}%
\end{pgfscope}%
\begin{pgfscope}%
\pgfsetbuttcap%
\pgfsetroundjoin%
\definecolor{currentfill}{rgb}{0.000000,0.000000,0.000000}%
\pgfsetfillcolor{currentfill}%
\pgfsetlinewidth{0.803000pt}%
\definecolor{currentstroke}{rgb}{0.000000,0.000000,0.000000}%
\pgfsetstrokecolor{currentstroke}%
\pgfsetdash{}{0pt}%
\pgfsys@defobject{currentmarker}{\pgfqpoint{-0.048611in}{0.000000in}}{\pgfqpoint{0.000000in}{0.000000in}}{%
\pgfpathmoveto{\pgfqpoint{0.000000in}{0.000000in}}%
\pgfpathlineto{\pgfqpoint{-0.048611in}{0.000000in}}%
\pgfusepath{stroke,fill}%
}%
\begin{pgfscope}%
\pgfsys@transformshift{0.467954in}{0.512870in}%
\pgfsys@useobject{currentmarker}{}%
\end{pgfscope}%
\end{pgfscope}%
\begin{pgfscope}%
\pgftext[x=0.193262in,y=0.464653in,left,base]{\rmfamily\fontsize{10.000000}{12.000000}\selectfont \(\displaystyle 0.0\)}%
\end{pgfscope}%
\begin{pgfscope}%
\pgfsetbuttcap%
\pgfsetroundjoin%
\definecolor{currentfill}{rgb}{0.000000,0.000000,0.000000}%
\pgfsetfillcolor{currentfill}%
\pgfsetlinewidth{0.803000pt}%
\definecolor{currentstroke}{rgb}{0.000000,0.000000,0.000000}%
\pgfsetstrokecolor{currentstroke}%
\pgfsetdash{}{0pt}%
\pgfsys@defobject{currentmarker}{\pgfqpoint{-0.048611in}{0.000000in}}{\pgfqpoint{0.000000in}{0.000000in}}{%
\pgfpathmoveto{\pgfqpoint{0.000000in}{0.000000in}}%
\pgfpathlineto{\pgfqpoint{-0.048611in}{0.000000in}}%
\pgfusepath{stroke,fill}%
}%
\begin{pgfscope}%
\pgfsys@transformshift{0.467954in}{0.978232in}%
\pgfsys@useobject{currentmarker}{}%
\end{pgfscope}%
\end{pgfscope}%
\begin{pgfscope}%
\pgftext[x=0.193262in,y=0.930014in,left,base]{\rmfamily\fontsize{10.000000}{12.000000}\selectfont \(\displaystyle 0.2\)}%
\end{pgfscope}%
\begin{pgfscope}%
\pgfsetbuttcap%
\pgfsetroundjoin%
\definecolor{currentfill}{rgb}{0.000000,0.000000,0.000000}%
\pgfsetfillcolor{currentfill}%
\pgfsetlinewidth{0.803000pt}%
\definecolor{currentstroke}{rgb}{0.000000,0.000000,0.000000}%
\pgfsetstrokecolor{currentstroke}%
\pgfsetdash{}{0pt}%
\pgfsys@defobject{currentmarker}{\pgfqpoint{-0.048611in}{0.000000in}}{\pgfqpoint{0.000000in}{0.000000in}}{%
\pgfpathmoveto{\pgfqpoint{0.000000in}{0.000000in}}%
\pgfpathlineto{\pgfqpoint{-0.048611in}{0.000000in}}%
\pgfusepath{stroke,fill}%
}%
\begin{pgfscope}%
\pgfsys@transformshift{0.467954in}{1.443594in}%
\pgfsys@useobject{currentmarker}{}%
\end{pgfscope}%
\end{pgfscope}%
\begin{pgfscope}%
\pgftext[x=0.193262in,y=1.395376in,left,base]{\rmfamily\fontsize{10.000000}{12.000000}\selectfont \(\displaystyle 0.4\)}%
\end{pgfscope}%
\begin{pgfscope}%
\pgfsetbuttcap%
\pgfsetroundjoin%
\definecolor{currentfill}{rgb}{0.000000,0.000000,0.000000}%
\pgfsetfillcolor{currentfill}%
\pgfsetlinewidth{0.803000pt}%
\definecolor{currentstroke}{rgb}{0.000000,0.000000,0.000000}%
\pgfsetstrokecolor{currentstroke}%
\pgfsetdash{}{0pt}%
\pgfsys@defobject{currentmarker}{\pgfqpoint{-0.048611in}{0.000000in}}{\pgfqpoint{0.000000in}{0.000000in}}{%
\pgfpathmoveto{\pgfqpoint{0.000000in}{0.000000in}}%
\pgfpathlineto{\pgfqpoint{-0.048611in}{0.000000in}}%
\pgfusepath{stroke,fill}%
}%
\begin{pgfscope}%
\pgfsys@transformshift{0.467954in}{1.908956in}%
\pgfsys@useobject{currentmarker}{}%
\end{pgfscope}%
\end{pgfscope}%
\begin{pgfscope}%
\pgftext[x=0.193262in,y=1.860738in,left,base]{\rmfamily\fontsize{10.000000}{12.000000}\selectfont \(\displaystyle 0.6\)}%
\end{pgfscope}%
\begin{pgfscope}%
\pgfsetbuttcap%
\pgfsetroundjoin%
\definecolor{currentfill}{rgb}{0.000000,0.000000,0.000000}%
\pgfsetfillcolor{currentfill}%
\pgfsetlinewidth{0.803000pt}%
\definecolor{currentstroke}{rgb}{0.000000,0.000000,0.000000}%
\pgfsetstrokecolor{currentstroke}%
\pgfsetdash{}{0pt}%
\pgfsys@defobject{currentmarker}{\pgfqpoint{-0.048611in}{0.000000in}}{\pgfqpoint{0.000000in}{0.000000in}}{%
\pgfpathmoveto{\pgfqpoint{0.000000in}{0.000000in}}%
\pgfpathlineto{\pgfqpoint{-0.048611in}{0.000000in}}%
\pgfusepath{stroke,fill}%
}%
\begin{pgfscope}%
\pgfsys@transformshift{0.467954in}{2.374317in}%
\pgfsys@useobject{currentmarker}{}%
\end{pgfscope}%
\end{pgfscope}%
\begin{pgfscope}%
\pgftext[x=0.193262in,y=2.326100in,left,base]{\rmfamily\fontsize{10.000000}{12.000000}\selectfont \(\displaystyle 0.8\)}%
\end{pgfscope}%
\begin{pgfscope}%
\pgfsetbuttcap%
\pgfsetroundjoin%
\definecolor{currentfill}{rgb}{0.000000,0.000000,0.000000}%
\pgfsetfillcolor{currentfill}%
\pgfsetlinewidth{0.803000pt}%
\definecolor{currentstroke}{rgb}{0.000000,0.000000,0.000000}%
\pgfsetstrokecolor{currentstroke}%
\pgfsetdash{}{0pt}%
\pgfsys@defobject{currentmarker}{\pgfqpoint{-0.048611in}{0.000000in}}{\pgfqpoint{0.000000in}{0.000000in}}{%
\pgfpathmoveto{\pgfqpoint{0.000000in}{0.000000in}}%
\pgfpathlineto{\pgfqpoint{-0.048611in}{0.000000in}}%
\pgfusepath{stroke,fill}%
}%
\begin{pgfscope}%
\pgfsys@transformshift{0.467954in}{2.839679in}%
\pgfsys@useobject{currentmarker}{}%
\end{pgfscope}%
\end{pgfscope}%
\begin{pgfscope}%
\pgftext[x=0.193262in,y=2.791461in,left,base]{\rmfamily\fontsize{10.000000}{12.000000}\selectfont \(\displaystyle 1.0\)}%
\end{pgfscope}%
\begin{pgfscope}%
\pgftext[x=0.137707in,y=1.677083in,,bottom,rotate=90.000000]{\rmfamily\fontsize{10.000000}{12.000000}\selectfont y(t)}%
\end{pgfscope}%
\begin{pgfscope}%
\pgfpathrectangle{\pgfqpoint{0.467954in}{0.402778in}}{\pgfqpoint{4.483434in}{2.548611in}}%
\pgfusepath{clip}%
\pgfsetrectcap%
\pgfsetroundjoin%
\pgfsetlinewidth{1.505625pt}%
\definecolor{currentstroke}{rgb}{0.121569,0.466667,0.705882}%
\pgfsetstrokecolor{currentstroke}%
\pgfsetdash{}{0pt}%
\pgfpathmoveto{\pgfqpoint{0.671747in}{0.518624in}}%
\pgfpathlineto{\pgfqpoint{0.894583in}{0.523933in}}%
\pgfpathlineto{\pgfqpoint{1.011318in}{0.528440in}}%
\pgfpathlineto{\pgfqpoint{1.012873in}{0.532894in}}%
\pgfpathlineto{\pgfqpoint{1.139994in}{0.541870in}}%
\pgfpathlineto{\pgfqpoint{1.247463in}{0.552479in}}%
\pgfpathlineto{\pgfqpoint{1.340939in}{0.564748in}}%
\pgfpathlineto{\pgfqpoint{1.350952in}{0.566265in}}%
\pgfpathlineto{\pgfqpoint{1.352506in}{0.580630in}}%
\pgfpathlineto{\pgfqpoint{1.426578in}{0.596549in}}%
\pgfpathlineto{\pgfqpoint{1.494492in}{0.614263in}}%
\pgfpathlineto{\pgfqpoint{1.557555in}{0.633874in}}%
\pgfpathlineto{\pgfqpoint{1.616638in}{0.655455in}}%
\pgfpathlineto{\pgfqpoint{1.672549in}{0.679145in}}%
\pgfpathlineto{\pgfqpoint{1.690585in}{0.687533in}}%
\pgfpathlineto{\pgfqpoint{1.692140in}{0.726096in}}%
\pgfpathlineto{\pgfqpoint{1.742267in}{0.756455in}}%
\pgfpathlineto{\pgfqpoint{1.791150in}{0.789674in}}%
\pgfpathlineto{\pgfqpoint{1.839163in}{0.826046in}}%
\pgfpathlineto{\pgfqpoint{1.886678in}{0.865935in}}%
\pgfpathlineto{\pgfqpoint{1.934069in}{0.909781in}}%
\pgfpathlineto{\pgfqpoint{1.981646in}{0.958043in}}%
\pgfpathlineto{\pgfqpoint{2.029845in}{1.011376in}}%
\pgfpathlineto{\pgfqpoint{2.030218in}{1.011806in}}%
\pgfpathlineto{\pgfqpoint{2.031773in}{1.074884in}}%
\pgfpathlineto{\pgfqpoint{2.083020in}{1.141677in}}%
\pgfpathlineto{\pgfqpoint{2.136941in}{1.217118in}}%
\pgfpathlineto{\pgfqpoint{2.195028in}{1.303820in}}%
\pgfpathlineto{\pgfqpoint{2.259957in}{1.406451in}}%
\pgfpathlineto{\pgfqpoint{2.338880in}{1.537282in}}%
\pgfpathlineto{\pgfqpoint{2.369852in}{1.589818in}}%
\pgfpathlineto{\pgfqpoint{2.371407in}{1.624306in}}%
\pgfpathlineto{\pgfqpoint{2.545546in}{1.918895in}}%
\pgfpathlineto{\pgfqpoint{2.617440in}{2.033707in}}%
\pgfpathlineto{\pgfqpoint{2.679135in}{2.126549in}}%
\pgfpathlineto{\pgfqpoint{2.709485in}{2.169961in}}%
\pgfpathlineto{\pgfqpoint{2.709547in}{2.155912in}}%
\pgfpathlineto{\pgfqpoint{2.711040in}{2.158032in}}%
\pgfpathlineto{\pgfqpoint{2.765894in}{2.233191in}}%
\pgfpathlineto{\pgfqpoint{2.817949in}{2.299357in}}%
\pgfpathlineto{\pgfqpoint{2.868138in}{2.358221in}}%
\pgfpathlineto{\pgfqpoint{2.917146in}{2.410992in}}%
\pgfpathlineto{\pgfqpoint{2.965470in}{2.458530in}}%
\pgfpathlineto{\pgfqpoint{3.013544in}{2.501523in}}%
\pgfpathlineto{\pgfqpoint{3.049118in}{2.530684in}}%
\pgfpathlineto{\pgfqpoint{3.049181in}{2.508546in}}%
\pgfpathlineto{\pgfqpoint{3.050673in}{2.509792in}}%
\pgfpathlineto{\pgfqpoint{3.098935in}{2.548011in}}%
\pgfpathlineto{\pgfqpoint{3.147756in}{2.582749in}}%
\pgfpathlineto{\pgfqpoint{3.197510in}{2.614368in}}%
\pgfpathlineto{\pgfqpoint{3.248570in}{2.643160in}}%
\pgfpathlineto{\pgfqpoint{3.301371in}{2.669383in}}%
\pgfpathlineto{\pgfqpoint{3.356412in}{2.693256in}}%
\pgfpathlineto{\pgfqpoint{3.388752in}{2.705790in}}%
\pgfpathlineto{\pgfqpoint{3.388814in}{2.691892in}}%
\pgfpathlineto{\pgfqpoint{3.390307in}{2.692499in}}%
\pgfpathlineto{\pgfqpoint{3.448021in}{2.714259in}}%
\pgfpathlineto{\pgfqpoint{3.509094in}{2.733962in}}%
\pgfpathlineto{\pgfqpoint{3.574397in}{2.751754in}}%
\pgfpathlineto{\pgfqpoint{3.644923in}{2.767730in}}%
\pgfpathlineto{\pgfqpoint{3.721979in}{2.781972in}}%
\pgfpathlineto{\pgfqpoint{3.728447in}{2.783034in}}%
\pgfpathlineto{\pgfqpoint{3.728510in}{2.776244in}}%
\pgfpathlineto{\pgfqpoint{3.730002in}{2.776514in}}%
\pgfpathlineto{\pgfqpoint{3.811723in}{2.789732in}}%
\pgfpathlineto{\pgfqpoint{3.903022in}{2.801310in}}%
\pgfpathlineto{\pgfqpoint{4.006821in}{2.811290in}}%
\pgfpathlineto{\pgfqpoint{4.068143in}{2.815932in}}%
\pgfpathlineto{\pgfqpoint{4.068205in}{2.812914in}}%
\pgfpathlineto{\pgfqpoint{4.069698in}{2.813030in}}%
\pgfpathlineto{\pgfqpoint{4.194394in}{2.821153in}}%
\pgfpathlineto{\pgfqpoint{4.344900in}{2.827751in}}%
\pgfpathlineto{\pgfqpoint{4.427927in}{2.829108in}}%
\pgfpathlineto{\pgfqpoint{4.630861in}{2.833851in}}%
\pgfpathlineto{\pgfqpoint{4.747596in}{2.835543in}}%
\pgfpathlineto{\pgfqpoint{4.747596in}{2.835543in}}%
\pgfusepath{stroke}%
\end{pgfscope}%
\begin{pgfscope}%
\pgfpathrectangle{\pgfqpoint{0.467954in}{0.402778in}}{\pgfqpoint{4.483434in}{2.548611in}}%
\pgfusepath{clip}%
\pgfsetrectcap%
\pgfsetroundjoin%
\pgfsetlinewidth{1.505625pt}%
\definecolor{currentstroke}{rgb}{1.000000,0.498039,0.054902}%
\pgfsetstrokecolor{currentstroke}%
\pgfsetdash{}{0pt}%
\pgfpathmoveto{\pgfqpoint{0.671747in}{0.518624in}}%
\pgfpathlineto{\pgfqpoint{0.894583in}{0.523933in}}%
\pgfpathlineto{\pgfqpoint{1.060886in}{0.530867in}}%
\pgfpathlineto{\pgfqpoint{1.194289in}{0.539426in}}%
\pgfpathlineto{\pgfqpoint{1.306111in}{0.549616in}}%
\pgfpathlineto{\pgfqpoint{1.350952in}{0.554708in}}%
\pgfpathlineto{\pgfqpoint{1.351014in}{0.551882in}}%
\pgfpathlineto{\pgfqpoint{1.352506in}{0.552051in}}%
\pgfpathlineto{\pgfqpoint{1.446417in}{0.564254in}}%
\pgfpathlineto{\pgfqpoint{1.529755in}{0.578142in}}%
\pgfpathlineto{\pgfqpoint{1.605008in}{0.593768in}}%
\pgfpathlineto{\pgfqpoint{1.673855in}{0.611178in}}%
\pgfpathlineto{\pgfqpoint{1.690585in}{0.615921in}}%
\pgfpathlineto{\pgfqpoint{1.690647in}{0.603281in}}%
\pgfpathlineto{\pgfqpoint{1.692140in}{0.603664in}}%
\pgfpathlineto{\pgfqpoint{1.757939in}{0.622160in}}%
\pgfpathlineto{\pgfqpoint{1.819261in}{0.642580in}}%
\pgfpathlineto{\pgfqpoint{1.876976in}{0.665032in}}%
\pgfpathlineto{\pgfqpoint{1.931768in}{0.689645in}}%
\pgfpathlineto{\pgfqpoint{1.984258in}{0.716602in}}%
\pgfpathlineto{\pgfqpoint{2.030218in}{0.743200in}}%
\pgfpathlineto{\pgfqpoint{2.030280in}{0.727386in}}%
\pgfpathlineto{\pgfqpoint{2.031773in}{0.728243in}}%
\pgfpathlineto{\pgfqpoint{2.081838in}{0.758832in}}%
\pgfpathlineto{\pgfqpoint{2.130659in}{0.792288in}}%
\pgfpathlineto{\pgfqpoint{2.178610in}{0.828900in}}%
\pgfpathlineto{\pgfqpoint{2.226125in}{0.869087in}}%
\pgfpathlineto{\pgfqpoint{2.273515in}{0.913246in}}%
\pgfpathlineto{\pgfqpoint{2.321155in}{0.961897in}}%
\pgfpathlineto{\pgfqpoint{2.369416in}{1.015641in}}%
\pgfpathlineto{\pgfqpoint{2.369852in}{1.016146in}}%
\pgfpathlineto{\pgfqpoint{2.371407in}{1.047363in}}%
\pgfpathlineto{\pgfqpoint{2.421720in}{1.110774in}}%
\pgfpathlineto{\pgfqpoint{2.474273in}{1.182050in}}%
\pgfpathlineto{\pgfqpoint{2.530246in}{1.263272in}}%
\pgfpathlineto{\pgfqpoint{2.591630in}{1.357931in}}%
\pgfpathlineto{\pgfqpoint{2.662778in}{1.473549in}}%
\pgfpathlineto{\pgfqpoint{2.709485in}{1.551927in}}%
\pgfpathlineto{\pgfqpoint{2.711040in}{1.603719in}}%
\pgfpathlineto{\pgfqpoint{2.905329in}{1.932122in}}%
\pgfpathlineto{\pgfqpoint{2.975731in}{2.043916in}}%
\pgfpathlineto{\pgfqpoint{3.036742in}{2.135137in}}%
\pgfpathlineto{\pgfqpoint{3.049118in}{2.152906in}}%
\pgfpathlineto{\pgfqpoint{3.050735in}{2.164216in}}%
\pgfpathlineto{\pgfqpoint{3.105341in}{2.238616in}}%
\pgfpathlineto{\pgfqpoint{3.157209in}{2.304147in}}%
\pgfpathlineto{\pgfqpoint{3.207274in}{2.362490in}}%
\pgfpathlineto{\pgfqpoint{3.256220in}{2.414837in}}%
\pgfpathlineto{\pgfqpoint{3.304481in}{2.461976in}}%
\pgfpathlineto{\pgfqpoint{3.352556in}{2.504648in}}%
\pgfpathlineto{\pgfqpoint{3.388752in}{2.534066in}}%
\pgfpathlineto{\pgfqpoint{3.388814in}{2.525917in}}%
\pgfpathlineto{\pgfqpoint{3.390307in}{2.527108in}}%
\pgfpathlineto{\pgfqpoint{3.438755in}{2.563731in}}%
\pgfpathlineto{\pgfqpoint{3.487949in}{2.597059in}}%
\pgfpathlineto{\pgfqpoint{3.538201in}{2.627381in}}%
\pgfpathlineto{\pgfqpoint{3.589945in}{2.654999in}}%
\pgfpathlineto{\pgfqpoint{3.643679in}{2.680170in}}%
\pgfpathlineto{\pgfqpoint{3.699839in}{2.703051in}}%
\pgfpathlineto{\pgfqpoint{3.728447in}{2.713490in}}%
\pgfpathlineto{\pgfqpoint{3.728510in}{2.706993in}}%
\pgfpathlineto{\pgfqpoint{3.730002in}{2.707542in}}%
\pgfpathlineto{\pgfqpoint{3.789956in}{2.727896in}}%
\pgfpathlineto{\pgfqpoint{3.853828in}{2.746289in}}%
\pgfpathlineto{\pgfqpoint{3.922550in}{2.762831in}}%
\pgfpathlineto{\pgfqpoint{3.997368in}{2.777619in}}%
\pgfpathlineto{\pgfqpoint{4.068143in}{2.789038in}}%
\pgfpathlineto{\pgfqpoint{4.068205in}{2.785924in}}%
\pgfpathlineto{\pgfqpoint{4.069698in}{2.786154in}}%
\pgfpathlineto{\pgfqpoint{4.158011in}{2.798191in}}%
\pgfpathlineto{\pgfqpoint{4.257892in}{2.808620in}}%
\pgfpathlineto{\pgfqpoint{4.373260in}{2.817479in}}%
\pgfpathlineto{\pgfqpoint{4.438873in}{2.820180in}}%
\pgfpathlineto{\pgfqpoint{4.585398in}{2.826975in}}%
\pgfpathlineto{\pgfqpoint{4.747596in}{2.831782in}}%
\pgfpathlineto{\pgfqpoint{4.747596in}{2.831782in}}%
\pgfusepath{stroke}%
\end{pgfscope}%
\begin{pgfscope}%
\pgfpathrectangle{\pgfqpoint{0.467954in}{0.402778in}}{\pgfqpoint{4.483434in}{2.548611in}}%
\pgfusepath{clip}%
\pgfsetrectcap%
\pgfsetroundjoin%
\pgfsetlinewidth{1.505625pt}%
\definecolor{currentstroke}{rgb}{0.172549,0.627451,0.172549}%
\pgfsetstrokecolor{currentstroke}%
\pgfsetdash{}{0pt}%
\pgfpathmoveto{\pgfqpoint{0.671747in}{0.518624in}}%
\pgfpathlineto{\pgfqpoint{0.894583in}{0.523933in}}%
\pgfpathlineto{\pgfqpoint{1.060886in}{0.530867in}}%
\pgfpathlineto{\pgfqpoint{1.194289in}{0.539426in}}%
\pgfpathlineto{\pgfqpoint{1.306111in}{0.549616in}}%
\pgfpathlineto{\pgfqpoint{1.402758in}{0.561458in}}%
\pgfpathlineto{\pgfqpoint{1.488148in}{0.574975in}}%
\pgfpathlineto{\pgfqpoint{1.565018in}{0.590224in}}%
\pgfpathlineto{\pgfqpoint{1.635171in}{0.607251in}}%
\pgfpathlineto{\pgfqpoint{1.690585in}{0.623183in}}%
\pgfpathlineto{\pgfqpoint{1.692202in}{0.625135in}}%
\pgfpathlineto{\pgfqpoint{1.752902in}{0.645847in}}%
\pgfpathlineto{\pgfqpoint{1.810119in}{0.668613in}}%
\pgfpathlineto{\pgfqpoint{1.864537in}{0.693574in}}%
\pgfpathlineto{\pgfqpoint{1.916717in}{0.720897in}}%
\pgfpathlineto{\pgfqpoint{1.967155in}{0.750795in}}%
\pgfpathlineto{\pgfqpoint{2.016225in}{0.783475in}}%
\pgfpathlineto{\pgfqpoint{2.030218in}{0.793483in}}%
\pgfpathlineto{\pgfqpoint{2.031835in}{0.800068in}}%
\pgfpathlineto{\pgfqpoint{2.079661in}{0.837436in}}%
\pgfpathlineto{\pgfqpoint{2.127114in}{0.878455in}}%
\pgfpathlineto{\pgfqpoint{2.174505in}{0.923533in}}%
\pgfpathlineto{\pgfqpoint{2.222207in}{0.973203in}}%
\pgfpathlineto{\pgfqpoint{2.270717in}{1.028223in}}%
\pgfpathlineto{\pgfqpoint{2.320471in}{1.089380in}}%
\pgfpathlineto{\pgfqpoint{2.388012in}{1.178118in}}%
\pgfpathlineto{\pgfqpoint{2.443799in}{1.258799in}}%
\pgfpathlineto{\pgfqpoint{2.504809in}{1.352604in}}%
\pgfpathlineto{\pgfqpoint{2.575274in}{1.466830in}}%
\pgfpathlineto{\pgfqpoint{2.670179in}{1.627090in}}%
\pgfpathlineto{\pgfqpoint{2.709485in}{1.694375in}}%
\pgfpathlineto{\pgfqpoint{2.709547in}{1.682284in}}%
\pgfpathlineto{\pgfqpoint{2.711040in}{1.684841in}}%
\pgfpathlineto{\pgfqpoint{2.842142in}{1.906300in}}%
\pgfpathlineto{\pgfqpoint{2.915467in}{2.024001in}}%
\pgfpathlineto{\pgfqpoint{2.977846in}{2.118433in}}%
\pgfpathlineto{\pgfqpoint{3.034565in}{2.198804in}}%
\pgfpathlineto{\pgfqpoint{3.049118in}{2.218508in}}%
\pgfpathlineto{\pgfqpoint{3.049181in}{2.215073in}}%
\pgfpathlineto{\pgfqpoint{3.050673in}{2.217079in}}%
\pgfpathlineto{\pgfqpoint{3.103288in}{2.285131in}}%
\pgfpathlineto{\pgfqpoint{3.153851in}{2.345544in}}%
\pgfpathlineto{\pgfqpoint{3.203045in}{2.399569in}}%
\pgfpathlineto{\pgfqpoint{3.251493in}{2.448232in}}%
\pgfpathlineto{\pgfqpoint{3.299568in}{2.492179in}}%
\pgfpathlineto{\pgfqpoint{3.347705in}{2.532026in}}%
\pgfpathlineto{\pgfqpoint{3.404300in}{2.574546in}}%
\pgfpathlineto{\pgfqpoint{3.453805in}{2.606908in}}%
\pgfpathlineto{\pgfqpoint{3.504492in}{2.636358in}}%
\pgfpathlineto{\pgfqpoint{3.556796in}{2.663175in}}%
\pgfpathlineto{\pgfqpoint{3.611215in}{2.687597in}}%
\pgfpathlineto{\pgfqpoint{3.668245in}{2.709792in}}%
\pgfpathlineto{\pgfqpoint{3.749469in}{2.737480in}}%
\pgfpathlineto{\pgfqpoint{3.815704in}{2.754927in}}%
\pgfpathlineto{\pgfqpoint{3.887349in}{2.770566in}}%
\pgfpathlineto{\pgfqpoint{3.965774in}{2.784478in}}%
\pgfpathlineto{\pgfqpoint{4.052844in}{2.796730in}}%
\pgfpathlineto{\pgfqpoint{4.174741in}{2.810305in}}%
\pgfpathlineto{\pgfqpoint{4.293404in}{2.818889in}}%
\pgfpathlineto{\pgfqpoint{4.460454in}{2.827477in}}%
\pgfpathlineto{\pgfqpoint{4.648275in}{2.832644in}}%
\pgfpathlineto{\pgfqpoint{4.747596in}{2.834424in}}%
\pgfpathlineto{\pgfqpoint{4.747596in}{2.834424in}}%
\pgfusepath{stroke}%
\end{pgfscope}%
\begin{pgfscope}%
\pgfpathrectangle{\pgfqpoint{0.467954in}{0.402778in}}{\pgfqpoint{4.483434in}{2.548611in}}%
\pgfusepath{clip}%
\pgfsetrectcap%
\pgfsetroundjoin%
\pgfsetlinewidth{1.505625pt}%
\definecolor{currentstroke}{rgb}{0.839216,0.152941,0.156863}%
\pgfsetstrokecolor{currentstroke}%
\pgfsetdash{}{0pt}%
\pgfpathmoveto{\pgfqpoint{0.671747in}{0.518624in}}%
\pgfpathlineto{\pgfqpoint{0.894583in}{0.523933in}}%
\pgfpathlineto{\pgfqpoint{1.060886in}{0.530867in}}%
\pgfpathlineto{\pgfqpoint{1.194289in}{0.539426in}}%
\pgfpathlineto{\pgfqpoint{1.306111in}{0.549616in}}%
\pgfpathlineto{\pgfqpoint{1.402758in}{0.561458in}}%
\pgfpathlineto{\pgfqpoint{1.488148in}{0.574975in}}%
\pgfpathlineto{\pgfqpoint{1.565018in}{0.590224in}}%
\pgfpathlineto{\pgfqpoint{1.635171in}{0.607251in}}%
\pgfpathlineto{\pgfqpoint{1.699976in}{0.626125in}}%
\pgfpathlineto{\pgfqpoint{1.760489in}{0.646939in}}%
\pgfpathlineto{\pgfqpoint{1.817582in}{0.669824in}}%
\pgfpathlineto{\pgfqpoint{1.871876in}{0.694902in}}%
\pgfpathlineto{\pgfqpoint{1.923931in}{0.722337in}}%
\pgfpathlineto{\pgfqpoint{1.974307in}{0.752379in}}%
\pgfpathlineto{\pgfqpoint{2.023377in}{0.785247in}}%
\pgfpathlineto{\pgfqpoint{2.075743in}{0.824216in}}%
\pgfpathlineto{\pgfqpoint{2.123258in}{0.863912in}}%
\pgfpathlineto{\pgfqpoint{2.170649in}{0.907557in}}%
\pgfpathlineto{\pgfqpoint{2.218226in}{0.955610in}}%
\pgfpathlineto{\pgfqpoint{2.266426in}{1.008729in}}%
\pgfpathlineto{\pgfqpoint{2.315682in}{1.067662in}}%
\pgfpathlineto{\pgfqpoint{2.366680in}{1.133565in}}%
\pgfpathlineto{\pgfqpoint{2.414755in}{1.200561in}}%
\pgfpathlineto{\pgfqpoint{2.471785in}{1.284590in}}%
\pgfpathlineto{\pgfqpoint{2.534973in}{1.383349in}}%
\pgfpathlineto{\pgfqpoint{2.609853in}{1.506378in}}%
\pgfpathlineto{\pgfqpoint{2.709485in}{1.675788in}}%
\pgfpathlineto{\pgfqpoint{2.711164in}{1.680286in}}%
\pgfpathlineto{\pgfqpoint{2.844816in}{1.906120in}}%
\pgfpathlineto{\pgfqpoint{2.918203in}{2.023927in}}%
\pgfpathlineto{\pgfqpoint{2.980582in}{2.118363in}}%
\pgfpathlineto{\pgfqpoint{3.037302in}{2.198739in}}%
\pgfpathlineto{\pgfqpoint{3.092280in}{2.271291in}}%
\pgfpathlineto{\pgfqpoint{3.143216in}{2.333216in}}%
\pgfpathlineto{\pgfqpoint{3.192659in}{2.388523in}}%
\pgfpathlineto{\pgfqpoint{3.241231in}{2.438269in}}%
\pgfpathlineto{\pgfqpoint{3.289368in}{2.483184in}}%
\pgfpathlineto{\pgfqpoint{3.337443in}{2.523848in}}%
\pgfpathlineto{\pgfqpoint{3.385829in}{2.560755in}}%
\pgfpathlineto{\pgfqpoint{3.434899in}{2.594319in}}%
\pgfpathlineto{\pgfqpoint{3.485088in}{2.624912in}}%
\pgfpathlineto{\pgfqpoint{3.536708in}{2.652760in}}%
\pgfpathlineto{\pgfqpoint{3.590256in}{2.678131in}}%
\pgfpathlineto{\pgfqpoint{3.646167in}{2.701191in}}%
\pgfpathlineto{\pgfqpoint{3.705063in}{2.722124in}}%
\pgfpathlineto{\pgfqpoint{3.767193in}{2.740917in}}%
\pgfpathlineto{\pgfqpoint{3.834361in}{2.758016in}}%
\pgfpathlineto{\pgfqpoint{3.907189in}{2.773326in}}%
\pgfpathlineto{\pgfqpoint{3.987106in}{2.786922in}}%
\pgfpathlineto{\pgfqpoint{4.073803in}{2.798522in}}%
\pgfpathlineto{\pgfqpoint{4.174119in}{2.808908in}}%
\pgfpathlineto{\pgfqpoint{4.290046in}{2.817722in}}%
\pgfpathlineto{\pgfqpoint{4.423822in}{2.824742in}}%
\pgfpathlineto{\pgfqpoint{4.592364in}{2.830562in}}%
\pgfpathlineto{\pgfqpoint{4.747596in}{2.833898in}}%
\pgfpathlineto{\pgfqpoint{4.747596in}{2.833898in}}%
\pgfusepath{stroke}%
\end{pgfscope}%
\begin{pgfscope}%
\pgfsetrectcap%
\pgfsetmiterjoin%
\pgfsetlinewidth{0.803000pt}%
\definecolor{currentstroke}{rgb}{0.000000,0.000000,0.000000}%
\pgfsetstrokecolor{currentstroke}%
\pgfsetdash{}{0pt}%
\pgfpathmoveto{\pgfqpoint{0.467954in}{0.402778in}}%
\pgfpathlineto{\pgfqpoint{0.467954in}{2.951389in}}%
\pgfusepath{stroke}%
\end{pgfscope}%
\begin{pgfscope}%
\pgfsetrectcap%
\pgfsetmiterjoin%
\pgfsetlinewidth{0.803000pt}%
\definecolor{currentstroke}{rgb}{0.000000,0.000000,0.000000}%
\pgfsetstrokecolor{currentstroke}%
\pgfsetdash{}{0pt}%
\pgfpathmoveto{\pgfqpoint{4.951389in}{0.402778in}}%
\pgfpathlineto{\pgfqpoint{4.951389in}{2.951389in}}%
\pgfusepath{stroke}%
\end{pgfscope}%
\begin{pgfscope}%
\pgfsetrectcap%
\pgfsetmiterjoin%
\pgfsetlinewidth{0.803000pt}%
\definecolor{currentstroke}{rgb}{0.000000,0.000000,0.000000}%
\pgfsetstrokecolor{currentstroke}%
\pgfsetdash{}{0pt}%
\pgfpathmoveto{\pgfqpoint{0.467954in}{0.402778in}}%
\pgfpathlineto{\pgfqpoint{4.951389in}{0.402778in}}%
\pgfusepath{stroke}%
\end{pgfscope}%
\begin{pgfscope}%
\pgfsetrectcap%
\pgfsetmiterjoin%
\pgfsetlinewidth{0.803000pt}%
\definecolor{currentstroke}{rgb}{0.000000,0.000000,0.000000}%
\pgfsetstrokecolor{currentstroke}%
\pgfsetdash{}{0pt}%
\pgfpathmoveto{\pgfqpoint{0.467954in}{2.951389in}}%
\pgfpathlineto{\pgfqpoint{4.951389in}{2.951389in}}%
\pgfusepath{stroke}%
\end{pgfscope}%
\begin{pgfscope}%
\pgfsetbuttcap%
\pgfsetmiterjoin%
\definecolor{currentfill}{rgb}{1.000000,1.000000,1.000000}%
\pgfsetfillcolor{currentfill}%
\pgfsetfillopacity{0.800000}%
\pgfsetlinewidth{1.003750pt}%
\definecolor{currentstroke}{rgb}{0.800000,0.800000,0.800000}%
\pgfsetstrokecolor{currentstroke}%
\pgfsetstrokeopacity{0.800000}%
\pgfsetdash{}{0pt}%
\pgfpathmoveto{\pgfqpoint{0.565177in}{2.054797in}}%
\pgfpathlineto{\pgfqpoint{1.079066in}{2.054797in}}%
\pgfpathquadraticcurveto{\pgfqpoint{1.106843in}{2.054797in}}{\pgfqpoint{1.106843in}{2.082574in}}%
\pgfpathlineto{\pgfqpoint{1.106843in}{2.854167in}}%
\pgfpathquadraticcurveto{\pgfqpoint{1.106843in}{2.881944in}}{\pgfqpoint{1.079066in}{2.881944in}}%
\pgfpathlineto{\pgfqpoint{0.565177in}{2.881944in}}%
\pgfpathquadraticcurveto{\pgfqpoint{0.537399in}{2.881944in}}{\pgfqpoint{0.537399in}{2.854167in}}%
\pgfpathlineto{\pgfqpoint{0.537399in}{2.082574in}}%
\pgfpathquadraticcurveto{\pgfqpoint{0.537399in}{2.054797in}}{\pgfqpoint{0.565177in}{2.054797in}}%
\pgfpathclose%
\pgfusepath{stroke,fill}%
\end{pgfscope}%
\begin{pgfscope}%
\pgfsetrectcap%
\pgfsetroundjoin%
\pgfsetlinewidth{1.505625pt}%
\definecolor{currentstroke}{rgb}{0.121569,0.466667,0.705882}%
\pgfsetstrokecolor{currentstroke}%
\pgfsetdash{}{0pt}%
\pgfpathmoveto{\pgfqpoint{0.592954in}{2.777778in}}%
\pgfpathlineto{\pgfqpoint{0.870732in}{2.777778in}}%
\pgfusepath{stroke}%
\end{pgfscope}%
\begin{pgfscope}%
\pgftext[x=0.981843in,y=2.729167in,left,base]{\rmfamily\fontsize{10.000000}{12.000000}\selectfont 1}%
\end{pgfscope}%
\begin{pgfscope}%
\pgfsetrectcap%
\pgfsetroundjoin%
\pgfsetlinewidth{1.505625pt}%
\definecolor{currentstroke}{rgb}{1.000000,0.498039,0.054902}%
\pgfsetstrokecolor{currentstroke}%
\pgfsetdash{}{0pt}%
\pgfpathmoveto{\pgfqpoint{0.592954in}{2.581407in}}%
\pgfpathlineto{\pgfqpoint{0.870732in}{2.581407in}}%
\pgfusepath{stroke}%
\end{pgfscope}%
\begin{pgfscope}%
\pgftext[x=0.981843in,y=2.532796in,left,base]{\rmfamily\fontsize{10.000000}{12.000000}\selectfont 2}%
\end{pgfscope}%
\begin{pgfscope}%
\pgfsetrectcap%
\pgfsetroundjoin%
\pgfsetlinewidth{1.505625pt}%
\definecolor{currentstroke}{rgb}{0.172549,0.627451,0.172549}%
\pgfsetstrokecolor{currentstroke}%
\pgfsetdash{}{0pt}%
\pgfpathmoveto{\pgfqpoint{0.592954in}{2.385037in}}%
\pgfpathlineto{\pgfqpoint{0.870732in}{2.385037in}}%
\pgfusepath{stroke}%
\end{pgfscope}%
\begin{pgfscope}%
\pgftext[x=0.981843in,y=2.336426in,left,base]{\rmfamily\fontsize{10.000000}{12.000000}\selectfont 3}%
\end{pgfscope}%
\begin{pgfscope}%
\pgfsetrectcap%
\pgfsetroundjoin%
\pgfsetlinewidth{1.505625pt}%
\definecolor{currentstroke}{rgb}{0.839216,0.152941,0.156863}%
\pgfsetstrokecolor{currentstroke}%
\pgfsetdash{}{0pt}%
\pgfpathmoveto{\pgfqpoint{0.592954in}{2.188667in}}%
\pgfpathlineto{\pgfqpoint{0.870732in}{2.188667in}}%
\pgfusepath{stroke}%
\end{pgfscope}%
\begin{pgfscope}%
\pgftext[x=0.981843in,y=2.140056in,left,base]{\rmfamily\fontsize{10.000000}{12.000000}\selectfont 4}%
\end{pgfscope}%
\end{pgfpicture}%
\makeatother%
\endgroup%
}
        \caption{Parareal solution of the logistic equation \(L = k = 1\) stopping after some amount of iterations.}
        \label{fig:iters_log}
    \end{figure}
\end{frame}

\begin{frame}
    \begin{figure}[ht]
        \centering
        \scalebox{0.8}{%% Creator: Matplotlib, PGF backend
%%
%% To include the figure in your LaTeX document, write
%%   \input{<filename>.pgf}
%%
%% Make sure the required packages are loaded in your preamble
%%   \usepackage{pgf}
%%
%% Figures using additional raster images can only be included by \input if
%% they are in the same directory as the main LaTeX file. For loading figures
%% from other directories you can use the `import` package
%%   \usepackage{import}
%% and then include the figures with
%%   \import{<path to file>}{<filename>.pgf}
%%
%% Matplotlib used the following preamble
%%   \usepackage{fontspec}
%%   \setmainfont{Times New Roman}
%%   \setsansfont{Lucida Grande}
%%   \setmonofont{Andale Mono}
%%
\begingroup%
\makeatletter%
\begin{pgfpicture}%
\pgfpathrectangle{\pgfpointorigin}{\pgfqpoint{5.000000in}{3.097222in}}%
\pgfusepath{use as bounding box, clip}%
\begin{pgfscope}%
\pgfsetbuttcap%
\pgfsetmiterjoin%
\definecolor{currentfill}{rgb}{1.000000,1.000000,1.000000}%
\pgfsetfillcolor{currentfill}%
\pgfsetlinewidth{0.000000pt}%
\definecolor{currentstroke}{rgb}{1.000000,1.000000,1.000000}%
\pgfsetstrokecolor{currentstroke}%
\pgfsetdash{}{0pt}%
\pgfpathmoveto{\pgfqpoint{0.000000in}{0.000000in}}%
\pgfpathlineto{\pgfqpoint{5.000000in}{0.000000in}}%
\pgfpathlineto{\pgfqpoint{5.000000in}{3.097222in}}%
\pgfpathlineto{\pgfqpoint{0.000000in}{3.097222in}}%
\pgfpathclose%
\pgfusepath{fill}%
\end{pgfscope}%
\begin{pgfscope}%
\pgfsetbuttcap%
\pgfsetmiterjoin%
\definecolor{currentfill}{rgb}{1.000000,1.000000,1.000000}%
\pgfsetfillcolor{currentfill}%
\pgfsetlinewidth{0.000000pt}%
\definecolor{currentstroke}{rgb}{0.000000,0.000000,0.000000}%
\pgfsetstrokecolor{currentstroke}%
\pgfsetstrokeopacity{0.000000}%
\pgfsetdash{}{0pt}%
\pgfpathmoveto{\pgfqpoint{0.571156in}{0.402778in}}%
\pgfpathlineto{\pgfqpoint{4.951389in}{0.402778in}}%
\pgfpathlineto{\pgfqpoint{4.951389in}{3.048611in}}%
\pgfpathlineto{\pgfqpoint{0.571156in}{3.048611in}}%
\pgfpathclose%
\pgfusepath{fill}%
\end{pgfscope}%
\begin{pgfscope}%
\pgfsetbuttcap%
\pgfsetroundjoin%
\definecolor{currentfill}{rgb}{0.000000,0.000000,0.000000}%
\pgfsetfillcolor{currentfill}%
\pgfsetlinewidth{0.803000pt}%
\definecolor{currentstroke}{rgb}{0.000000,0.000000,0.000000}%
\pgfsetstrokecolor{currentstroke}%
\pgfsetdash{}{0pt}%
\pgfsys@defobject{currentmarker}{\pgfqpoint{0.000000in}{-0.048611in}}{\pgfqpoint{0.000000in}{0.000000in}}{%
\pgfpathmoveto{\pgfqpoint{0.000000in}{0.000000in}}%
\pgfpathlineto{\pgfqpoint{0.000000in}{-0.048611in}}%
\pgfusepath{stroke,fill}%
}%
\begin{pgfscope}%
\pgfsys@transformshift{0.770257in}{0.402778in}%
\pgfsys@useobject{currentmarker}{}%
\end{pgfscope}%
\end{pgfscope}%
\begin{pgfscope}%
\pgftext[x=0.770257in,y=0.305556in,,top]{\rmfamily\fontsize{10.000000}{12.000000}\selectfont \(\displaystyle -6\)}%
\end{pgfscope}%
\begin{pgfscope}%
\pgfsetbuttcap%
\pgfsetroundjoin%
\definecolor{currentfill}{rgb}{0.000000,0.000000,0.000000}%
\pgfsetfillcolor{currentfill}%
\pgfsetlinewidth{0.803000pt}%
\definecolor{currentstroke}{rgb}{0.000000,0.000000,0.000000}%
\pgfsetstrokecolor{currentstroke}%
\pgfsetdash{}{0pt}%
\pgfsys@defobject{currentmarker}{\pgfqpoint{0.000000in}{-0.048611in}}{\pgfqpoint{0.000000in}{0.000000in}}{%
\pgfpathmoveto{\pgfqpoint{0.000000in}{0.000000in}}%
\pgfpathlineto{\pgfqpoint{0.000000in}{-0.048611in}}%
\pgfusepath{stroke,fill}%
}%
\begin{pgfscope}%
\pgfsys@transformshift{1.433929in}{0.402778in}%
\pgfsys@useobject{currentmarker}{}%
\end{pgfscope}%
\end{pgfscope}%
\begin{pgfscope}%
\pgftext[x=1.433929in,y=0.305556in,,top]{\rmfamily\fontsize{10.000000}{12.000000}\selectfont \(\displaystyle -4\)}%
\end{pgfscope}%
\begin{pgfscope}%
\pgfsetbuttcap%
\pgfsetroundjoin%
\definecolor{currentfill}{rgb}{0.000000,0.000000,0.000000}%
\pgfsetfillcolor{currentfill}%
\pgfsetlinewidth{0.803000pt}%
\definecolor{currentstroke}{rgb}{0.000000,0.000000,0.000000}%
\pgfsetstrokecolor{currentstroke}%
\pgfsetdash{}{0pt}%
\pgfsys@defobject{currentmarker}{\pgfqpoint{0.000000in}{-0.048611in}}{\pgfqpoint{0.000000in}{0.000000in}}{%
\pgfpathmoveto{\pgfqpoint{0.000000in}{0.000000in}}%
\pgfpathlineto{\pgfqpoint{0.000000in}{-0.048611in}}%
\pgfusepath{stroke,fill}%
}%
\begin{pgfscope}%
\pgfsys@transformshift{2.097601in}{0.402778in}%
\pgfsys@useobject{currentmarker}{}%
\end{pgfscope}%
\end{pgfscope}%
\begin{pgfscope}%
\pgftext[x=2.097601in,y=0.305556in,,top]{\rmfamily\fontsize{10.000000}{12.000000}\selectfont \(\displaystyle -2\)}%
\end{pgfscope}%
\begin{pgfscope}%
\pgfsetbuttcap%
\pgfsetroundjoin%
\definecolor{currentfill}{rgb}{0.000000,0.000000,0.000000}%
\pgfsetfillcolor{currentfill}%
\pgfsetlinewidth{0.803000pt}%
\definecolor{currentstroke}{rgb}{0.000000,0.000000,0.000000}%
\pgfsetstrokecolor{currentstroke}%
\pgfsetdash{}{0pt}%
\pgfsys@defobject{currentmarker}{\pgfqpoint{0.000000in}{-0.048611in}}{\pgfqpoint{0.000000in}{0.000000in}}{%
\pgfpathmoveto{\pgfqpoint{0.000000in}{0.000000in}}%
\pgfpathlineto{\pgfqpoint{0.000000in}{-0.048611in}}%
\pgfusepath{stroke,fill}%
}%
\begin{pgfscope}%
\pgfsys@transformshift{2.761272in}{0.402778in}%
\pgfsys@useobject{currentmarker}{}%
\end{pgfscope}%
\end{pgfscope}%
\begin{pgfscope}%
\pgftext[x=2.761272in,y=0.305556in,,top]{\rmfamily\fontsize{10.000000}{12.000000}\selectfont \(\displaystyle 0\)}%
\end{pgfscope}%
\begin{pgfscope}%
\pgfsetbuttcap%
\pgfsetroundjoin%
\definecolor{currentfill}{rgb}{0.000000,0.000000,0.000000}%
\pgfsetfillcolor{currentfill}%
\pgfsetlinewidth{0.803000pt}%
\definecolor{currentstroke}{rgb}{0.000000,0.000000,0.000000}%
\pgfsetstrokecolor{currentstroke}%
\pgfsetdash{}{0pt}%
\pgfsys@defobject{currentmarker}{\pgfqpoint{0.000000in}{-0.048611in}}{\pgfqpoint{0.000000in}{0.000000in}}{%
\pgfpathmoveto{\pgfqpoint{0.000000in}{0.000000in}}%
\pgfpathlineto{\pgfqpoint{0.000000in}{-0.048611in}}%
\pgfusepath{stroke,fill}%
}%
\begin{pgfscope}%
\pgfsys@transformshift{3.424944in}{0.402778in}%
\pgfsys@useobject{currentmarker}{}%
\end{pgfscope}%
\end{pgfscope}%
\begin{pgfscope}%
\pgftext[x=3.424944in,y=0.305556in,,top]{\rmfamily\fontsize{10.000000}{12.000000}\selectfont \(\displaystyle 2\)}%
\end{pgfscope}%
\begin{pgfscope}%
\pgfsetbuttcap%
\pgfsetroundjoin%
\definecolor{currentfill}{rgb}{0.000000,0.000000,0.000000}%
\pgfsetfillcolor{currentfill}%
\pgfsetlinewidth{0.803000pt}%
\definecolor{currentstroke}{rgb}{0.000000,0.000000,0.000000}%
\pgfsetstrokecolor{currentstroke}%
\pgfsetdash{}{0pt}%
\pgfsys@defobject{currentmarker}{\pgfqpoint{0.000000in}{-0.048611in}}{\pgfqpoint{0.000000in}{0.000000in}}{%
\pgfpathmoveto{\pgfqpoint{0.000000in}{0.000000in}}%
\pgfpathlineto{\pgfqpoint{0.000000in}{-0.048611in}}%
\pgfusepath{stroke,fill}%
}%
\begin{pgfscope}%
\pgfsys@transformshift{4.088616in}{0.402778in}%
\pgfsys@useobject{currentmarker}{}%
\end{pgfscope}%
\end{pgfscope}%
\begin{pgfscope}%
\pgftext[x=4.088616in,y=0.305556in,,top]{\rmfamily\fontsize{10.000000}{12.000000}\selectfont \(\displaystyle 4\)}%
\end{pgfscope}%
\begin{pgfscope}%
\pgfsetbuttcap%
\pgfsetroundjoin%
\definecolor{currentfill}{rgb}{0.000000,0.000000,0.000000}%
\pgfsetfillcolor{currentfill}%
\pgfsetlinewidth{0.803000pt}%
\definecolor{currentstroke}{rgb}{0.000000,0.000000,0.000000}%
\pgfsetstrokecolor{currentstroke}%
\pgfsetdash{}{0pt}%
\pgfsys@defobject{currentmarker}{\pgfqpoint{0.000000in}{-0.048611in}}{\pgfqpoint{0.000000in}{0.000000in}}{%
\pgfpathmoveto{\pgfqpoint{0.000000in}{0.000000in}}%
\pgfpathlineto{\pgfqpoint{0.000000in}{-0.048611in}}%
\pgfusepath{stroke,fill}%
}%
\begin{pgfscope}%
\pgfsys@transformshift{4.752287in}{0.402778in}%
\pgfsys@useobject{currentmarker}{}%
\end{pgfscope}%
\end{pgfscope}%
\begin{pgfscope}%
\pgftext[x=4.752287in,y=0.305556in,,top]{\rmfamily\fontsize{10.000000}{12.000000}\selectfont \(\displaystyle 6\)}%
\end{pgfscope}%
\begin{pgfscope}%
\pgftext[x=2.761272in,y=0.123861in,,top]{\rmfamily\fontsize{10.000000}{12.000000}\selectfont t}%
\end{pgfscope}%
\begin{pgfscope}%
\pgfsetbuttcap%
\pgfsetroundjoin%
\definecolor{currentfill}{rgb}{0.000000,0.000000,0.000000}%
\pgfsetfillcolor{currentfill}%
\pgfsetlinewidth{0.803000pt}%
\definecolor{currentstroke}{rgb}{0.000000,0.000000,0.000000}%
\pgfsetstrokecolor{currentstroke}%
\pgfsetdash{}{0pt}%
\pgfsys@defobject{currentmarker}{\pgfqpoint{-0.048611in}{0.000000in}}{\pgfqpoint{0.000000in}{0.000000in}}{%
\pgfpathmoveto{\pgfqpoint{0.000000in}{0.000000in}}%
\pgfpathlineto{\pgfqpoint{-0.048611in}{0.000000in}}%
\pgfusepath{stroke,fill}%
}%
\begin{pgfscope}%
\pgfsys@transformshift{0.571156in}{0.418994in}%
\pgfsys@useobject{currentmarker}{}%
\end{pgfscope}%
\end{pgfscope}%
\begin{pgfscope}%
\pgftext[x=0.185931in,y=0.370776in,left,base]{\rmfamily\fontsize{10.000000}{12.000000}\selectfont \(\displaystyle 10^{-7}\)}%
\end{pgfscope}%
\begin{pgfscope}%
\pgfsetbuttcap%
\pgfsetroundjoin%
\definecolor{currentfill}{rgb}{0.000000,0.000000,0.000000}%
\pgfsetfillcolor{currentfill}%
\pgfsetlinewidth{0.803000pt}%
\definecolor{currentstroke}{rgb}{0.000000,0.000000,0.000000}%
\pgfsetstrokecolor{currentstroke}%
\pgfsetdash{}{0pt}%
\pgfsys@defobject{currentmarker}{\pgfqpoint{-0.048611in}{0.000000in}}{\pgfqpoint{0.000000in}{0.000000in}}{%
\pgfpathmoveto{\pgfqpoint{0.000000in}{0.000000in}}%
\pgfpathlineto{\pgfqpoint{-0.048611in}{0.000000in}}%
\pgfusepath{stroke,fill}%
}%
\begin{pgfscope}%
\pgfsys@transformshift{0.571156in}{0.765859in}%
\pgfsys@useobject{currentmarker}{}%
\end{pgfscope}%
\end{pgfscope}%
\begin{pgfscope}%
\pgftext[x=0.185931in,y=0.717641in,left,base]{\rmfamily\fontsize{10.000000}{12.000000}\selectfont \(\displaystyle 10^{-6}\)}%
\end{pgfscope}%
\begin{pgfscope}%
\pgfsetbuttcap%
\pgfsetroundjoin%
\definecolor{currentfill}{rgb}{0.000000,0.000000,0.000000}%
\pgfsetfillcolor{currentfill}%
\pgfsetlinewidth{0.803000pt}%
\definecolor{currentstroke}{rgb}{0.000000,0.000000,0.000000}%
\pgfsetstrokecolor{currentstroke}%
\pgfsetdash{}{0pt}%
\pgfsys@defobject{currentmarker}{\pgfqpoint{-0.048611in}{0.000000in}}{\pgfqpoint{0.000000in}{0.000000in}}{%
\pgfpathmoveto{\pgfqpoint{0.000000in}{0.000000in}}%
\pgfpathlineto{\pgfqpoint{-0.048611in}{0.000000in}}%
\pgfusepath{stroke,fill}%
}%
\begin{pgfscope}%
\pgfsys@transformshift{0.571156in}{1.112723in}%
\pgfsys@useobject{currentmarker}{}%
\end{pgfscope}%
\end{pgfscope}%
\begin{pgfscope}%
\pgftext[x=0.185931in,y=1.064506in,left,base]{\rmfamily\fontsize{10.000000}{12.000000}\selectfont \(\displaystyle 10^{-5}\)}%
\end{pgfscope}%
\begin{pgfscope}%
\pgfsetbuttcap%
\pgfsetroundjoin%
\definecolor{currentfill}{rgb}{0.000000,0.000000,0.000000}%
\pgfsetfillcolor{currentfill}%
\pgfsetlinewidth{0.803000pt}%
\definecolor{currentstroke}{rgb}{0.000000,0.000000,0.000000}%
\pgfsetstrokecolor{currentstroke}%
\pgfsetdash{}{0pt}%
\pgfsys@defobject{currentmarker}{\pgfqpoint{-0.048611in}{0.000000in}}{\pgfqpoint{0.000000in}{0.000000in}}{%
\pgfpathmoveto{\pgfqpoint{0.000000in}{0.000000in}}%
\pgfpathlineto{\pgfqpoint{-0.048611in}{0.000000in}}%
\pgfusepath{stroke,fill}%
}%
\begin{pgfscope}%
\pgfsys@transformshift{0.571156in}{1.459588in}%
\pgfsys@useobject{currentmarker}{}%
\end{pgfscope}%
\end{pgfscope}%
\begin{pgfscope}%
\pgftext[x=0.185931in,y=1.411370in,left,base]{\rmfamily\fontsize{10.000000}{12.000000}\selectfont \(\displaystyle 10^{-4}\)}%
\end{pgfscope}%
\begin{pgfscope}%
\pgfsetbuttcap%
\pgfsetroundjoin%
\definecolor{currentfill}{rgb}{0.000000,0.000000,0.000000}%
\pgfsetfillcolor{currentfill}%
\pgfsetlinewidth{0.803000pt}%
\definecolor{currentstroke}{rgb}{0.000000,0.000000,0.000000}%
\pgfsetstrokecolor{currentstroke}%
\pgfsetdash{}{0pt}%
\pgfsys@defobject{currentmarker}{\pgfqpoint{-0.048611in}{0.000000in}}{\pgfqpoint{0.000000in}{0.000000in}}{%
\pgfpathmoveto{\pgfqpoint{0.000000in}{0.000000in}}%
\pgfpathlineto{\pgfqpoint{-0.048611in}{0.000000in}}%
\pgfusepath{stroke,fill}%
}%
\begin{pgfscope}%
\pgfsys@transformshift{0.571156in}{1.806453in}%
\pgfsys@useobject{currentmarker}{}%
\end{pgfscope}%
\end{pgfscope}%
\begin{pgfscope}%
\pgftext[x=0.185931in,y=1.758235in,left,base]{\rmfamily\fontsize{10.000000}{12.000000}\selectfont \(\displaystyle 10^{-3}\)}%
\end{pgfscope}%
\begin{pgfscope}%
\pgfsetbuttcap%
\pgfsetroundjoin%
\definecolor{currentfill}{rgb}{0.000000,0.000000,0.000000}%
\pgfsetfillcolor{currentfill}%
\pgfsetlinewidth{0.803000pt}%
\definecolor{currentstroke}{rgb}{0.000000,0.000000,0.000000}%
\pgfsetstrokecolor{currentstroke}%
\pgfsetdash{}{0pt}%
\pgfsys@defobject{currentmarker}{\pgfqpoint{-0.048611in}{0.000000in}}{\pgfqpoint{0.000000in}{0.000000in}}{%
\pgfpathmoveto{\pgfqpoint{0.000000in}{0.000000in}}%
\pgfpathlineto{\pgfqpoint{-0.048611in}{0.000000in}}%
\pgfusepath{stroke,fill}%
}%
\begin{pgfscope}%
\pgfsys@transformshift{0.571156in}{2.153317in}%
\pgfsys@useobject{currentmarker}{}%
\end{pgfscope}%
\end{pgfscope}%
\begin{pgfscope}%
\pgftext[x=0.185931in,y=2.105099in,left,base]{\rmfamily\fontsize{10.000000}{12.000000}\selectfont \(\displaystyle 10^{-2}\)}%
\end{pgfscope}%
\begin{pgfscope}%
\pgfsetbuttcap%
\pgfsetroundjoin%
\definecolor{currentfill}{rgb}{0.000000,0.000000,0.000000}%
\pgfsetfillcolor{currentfill}%
\pgfsetlinewidth{0.803000pt}%
\definecolor{currentstroke}{rgb}{0.000000,0.000000,0.000000}%
\pgfsetstrokecolor{currentstroke}%
\pgfsetdash{}{0pt}%
\pgfsys@defobject{currentmarker}{\pgfqpoint{-0.048611in}{0.000000in}}{\pgfqpoint{0.000000in}{0.000000in}}{%
\pgfpathmoveto{\pgfqpoint{0.000000in}{0.000000in}}%
\pgfpathlineto{\pgfqpoint{-0.048611in}{0.000000in}}%
\pgfusepath{stroke,fill}%
}%
\begin{pgfscope}%
\pgfsys@transformshift{0.571156in}{2.500182in}%
\pgfsys@useobject{currentmarker}{}%
\end{pgfscope}%
\end{pgfscope}%
\begin{pgfscope}%
\pgftext[x=0.185931in,y=2.451964in,left,base]{\rmfamily\fontsize{10.000000}{12.000000}\selectfont \(\displaystyle 10^{-1}\)}%
\end{pgfscope}%
\begin{pgfscope}%
\pgfsetbuttcap%
\pgfsetroundjoin%
\definecolor{currentfill}{rgb}{0.000000,0.000000,0.000000}%
\pgfsetfillcolor{currentfill}%
\pgfsetlinewidth{0.803000pt}%
\definecolor{currentstroke}{rgb}{0.000000,0.000000,0.000000}%
\pgfsetstrokecolor{currentstroke}%
\pgfsetdash{}{0pt}%
\pgfsys@defobject{currentmarker}{\pgfqpoint{-0.048611in}{0.000000in}}{\pgfqpoint{0.000000in}{0.000000in}}{%
\pgfpathmoveto{\pgfqpoint{0.000000in}{0.000000in}}%
\pgfpathlineto{\pgfqpoint{-0.048611in}{0.000000in}}%
\pgfusepath{stroke,fill}%
}%
\begin{pgfscope}%
\pgfsys@transformshift{0.571156in}{2.847046in}%
\pgfsys@useobject{currentmarker}{}%
\end{pgfscope}%
\end{pgfscope}%
\begin{pgfscope}%
\pgftext[x=0.272737in,y=2.798828in,left,base]{\rmfamily\fontsize{10.000000}{12.000000}\selectfont \(\displaystyle 10^{0}\)}%
\end{pgfscope}%
\begin{pgfscope}%
\pgfsetbuttcap%
\pgfsetroundjoin%
\definecolor{currentfill}{rgb}{0.000000,0.000000,0.000000}%
\pgfsetfillcolor{currentfill}%
\pgfsetlinewidth{0.602250pt}%
\definecolor{currentstroke}{rgb}{0.000000,0.000000,0.000000}%
\pgfsetstrokecolor{currentstroke}%
\pgfsetdash{}{0pt}%
\pgfsys@defobject{currentmarker}{\pgfqpoint{-0.027778in}{0.000000in}}{\pgfqpoint{0.000000in}{0.000000in}}{%
\pgfpathmoveto{\pgfqpoint{0.000000in}{0.000000in}}%
\pgfpathlineto{\pgfqpoint{-0.027778in}{0.000000in}}%
\pgfusepath{stroke,fill}%
}%
\begin{pgfscope}%
\pgfsys@transformshift{0.571156in}{0.403123in}%
\pgfsys@useobject{currentmarker}{}%
\end{pgfscope}%
\end{pgfscope}%
\begin{pgfscope}%
\pgfsetbuttcap%
\pgfsetroundjoin%
\definecolor{currentfill}{rgb}{0.000000,0.000000,0.000000}%
\pgfsetfillcolor{currentfill}%
\pgfsetlinewidth{0.602250pt}%
\definecolor{currentstroke}{rgb}{0.000000,0.000000,0.000000}%
\pgfsetstrokecolor{currentstroke}%
\pgfsetdash{}{0pt}%
\pgfsys@defobject{currentmarker}{\pgfqpoint{-0.027778in}{0.000000in}}{\pgfqpoint{0.000000in}{0.000000in}}{%
\pgfpathmoveto{\pgfqpoint{0.000000in}{0.000000in}}%
\pgfpathlineto{\pgfqpoint{-0.027778in}{0.000000in}}%
\pgfusepath{stroke,fill}%
}%
\begin{pgfscope}%
\pgfsys@transformshift{0.571156in}{0.523411in}%
\pgfsys@useobject{currentmarker}{}%
\end{pgfscope}%
\end{pgfscope}%
\begin{pgfscope}%
\pgfsetbuttcap%
\pgfsetroundjoin%
\definecolor{currentfill}{rgb}{0.000000,0.000000,0.000000}%
\pgfsetfillcolor{currentfill}%
\pgfsetlinewidth{0.602250pt}%
\definecolor{currentstroke}{rgb}{0.000000,0.000000,0.000000}%
\pgfsetstrokecolor{currentstroke}%
\pgfsetdash{}{0pt}%
\pgfsys@defobject{currentmarker}{\pgfqpoint{-0.027778in}{0.000000in}}{\pgfqpoint{0.000000in}{0.000000in}}{%
\pgfpathmoveto{\pgfqpoint{0.000000in}{0.000000in}}%
\pgfpathlineto{\pgfqpoint{-0.027778in}{0.000000in}}%
\pgfusepath{stroke,fill}%
}%
\begin{pgfscope}%
\pgfsys@transformshift{0.571156in}{0.584491in}%
\pgfsys@useobject{currentmarker}{}%
\end{pgfscope}%
\end{pgfscope}%
\begin{pgfscope}%
\pgfsetbuttcap%
\pgfsetroundjoin%
\definecolor{currentfill}{rgb}{0.000000,0.000000,0.000000}%
\pgfsetfillcolor{currentfill}%
\pgfsetlinewidth{0.602250pt}%
\definecolor{currentstroke}{rgb}{0.000000,0.000000,0.000000}%
\pgfsetstrokecolor{currentstroke}%
\pgfsetdash{}{0pt}%
\pgfsys@defobject{currentmarker}{\pgfqpoint{-0.027778in}{0.000000in}}{\pgfqpoint{0.000000in}{0.000000in}}{%
\pgfpathmoveto{\pgfqpoint{0.000000in}{0.000000in}}%
\pgfpathlineto{\pgfqpoint{-0.027778in}{0.000000in}}%
\pgfusepath{stroke,fill}%
}%
\begin{pgfscope}%
\pgfsys@transformshift{0.571156in}{0.627827in}%
\pgfsys@useobject{currentmarker}{}%
\end{pgfscope}%
\end{pgfscope}%
\begin{pgfscope}%
\pgfsetbuttcap%
\pgfsetroundjoin%
\definecolor{currentfill}{rgb}{0.000000,0.000000,0.000000}%
\pgfsetfillcolor{currentfill}%
\pgfsetlinewidth{0.602250pt}%
\definecolor{currentstroke}{rgb}{0.000000,0.000000,0.000000}%
\pgfsetstrokecolor{currentstroke}%
\pgfsetdash{}{0pt}%
\pgfsys@defobject{currentmarker}{\pgfqpoint{-0.027778in}{0.000000in}}{\pgfqpoint{0.000000in}{0.000000in}}{%
\pgfpathmoveto{\pgfqpoint{0.000000in}{0.000000in}}%
\pgfpathlineto{\pgfqpoint{-0.027778in}{0.000000in}}%
\pgfusepath{stroke,fill}%
}%
\begin{pgfscope}%
\pgfsys@transformshift{0.571156in}{0.661442in}%
\pgfsys@useobject{currentmarker}{}%
\end{pgfscope}%
\end{pgfscope}%
\begin{pgfscope}%
\pgfsetbuttcap%
\pgfsetroundjoin%
\definecolor{currentfill}{rgb}{0.000000,0.000000,0.000000}%
\pgfsetfillcolor{currentfill}%
\pgfsetlinewidth{0.602250pt}%
\definecolor{currentstroke}{rgb}{0.000000,0.000000,0.000000}%
\pgfsetstrokecolor{currentstroke}%
\pgfsetdash{}{0pt}%
\pgfsys@defobject{currentmarker}{\pgfqpoint{-0.027778in}{0.000000in}}{\pgfqpoint{0.000000in}{0.000000in}}{%
\pgfpathmoveto{\pgfqpoint{0.000000in}{0.000000in}}%
\pgfpathlineto{\pgfqpoint{-0.027778in}{0.000000in}}%
\pgfusepath{stroke,fill}%
}%
\begin{pgfscope}%
\pgfsys@transformshift{0.571156in}{0.688907in}%
\pgfsys@useobject{currentmarker}{}%
\end{pgfscope}%
\end{pgfscope}%
\begin{pgfscope}%
\pgfsetbuttcap%
\pgfsetroundjoin%
\definecolor{currentfill}{rgb}{0.000000,0.000000,0.000000}%
\pgfsetfillcolor{currentfill}%
\pgfsetlinewidth{0.602250pt}%
\definecolor{currentstroke}{rgb}{0.000000,0.000000,0.000000}%
\pgfsetstrokecolor{currentstroke}%
\pgfsetdash{}{0pt}%
\pgfsys@defobject{currentmarker}{\pgfqpoint{-0.027778in}{0.000000in}}{\pgfqpoint{0.000000in}{0.000000in}}{%
\pgfpathmoveto{\pgfqpoint{0.000000in}{0.000000in}}%
\pgfpathlineto{\pgfqpoint{-0.027778in}{0.000000in}}%
\pgfusepath{stroke,fill}%
}%
\begin{pgfscope}%
\pgfsys@transformshift{0.571156in}{0.712129in}%
\pgfsys@useobject{currentmarker}{}%
\end{pgfscope}%
\end{pgfscope}%
\begin{pgfscope}%
\pgfsetbuttcap%
\pgfsetroundjoin%
\definecolor{currentfill}{rgb}{0.000000,0.000000,0.000000}%
\pgfsetfillcolor{currentfill}%
\pgfsetlinewidth{0.602250pt}%
\definecolor{currentstroke}{rgb}{0.000000,0.000000,0.000000}%
\pgfsetstrokecolor{currentstroke}%
\pgfsetdash{}{0pt}%
\pgfsys@defobject{currentmarker}{\pgfqpoint{-0.027778in}{0.000000in}}{\pgfqpoint{0.000000in}{0.000000in}}{%
\pgfpathmoveto{\pgfqpoint{0.000000in}{0.000000in}}%
\pgfpathlineto{\pgfqpoint{-0.027778in}{0.000000in}}%
\pgfusepath{stroke,fill}%
}%
\begin{pgfscope}%
\pgfsys@transformshift{0.571156in}{0.732244in}%
\pgfsys@useobject{currentmarker}{}%
\end{pgfscope}%
\end{pgfscope}%
\begin{pgfscope}%
\pgfsetbuttcap%
\pgfsetroundjoin%
\definecolor{currentfill}{rgb}{0.000000,0.000000,0.000000}%
\pgfsetfillcolor{currentfill}%
\pgfsetlinewidth{0.602250pt}%
\definecolor{currentstroke}{rgb}{0.000000,0.000000,0.000000}%
\pgfsetstrokecolor{currentstroke}%
\pgfsetdash{}{0pt}%
\pgfsys@defobject{currentmarker}{\pgfqpoint{-0.027778in}{0.000000in}}{\pgfqpoint{0.000000in}{0.000000in}}{%
\pgfpathmoveto{\pgfqpoint{0.000000in}{0.000000in}}%
\pgfpathlineto{\pgfqpoint{-0.027778in}{0.000000in}}%
\pgfusepath{stroke,fill}%
}%
\begin{pgfscope}%
\pgfsys@transformshift{0.571156in}{0.749987in}%
\pgfsys@useobject{currentmarker}{}%
\end{pgfscope}%
\end{pgfscope}%
\begin{pgfscope}%
\pgfsetbuttcap%
\pgfsetroundjoin%
\definecolor{currentfill}{rgb}{0.000000,0.000000,0.000000}%
\pgfsetfillcolor{currentfill}%
\pgfsetlinewidth{0.602250pt}%
\definecolor{currentstroke}{rgb}{0.000000,0.000000,0.000000}%
\pgfsetstrokecolor{currentstroke}%
\pgfsetdash{}{0pt}%
\pgfsys@defobject{currentmarker}{\pgfqpoint{-0.027778in}{0.000000in}}{\pgfqpoint{0.000000in}{0.000000in}}{%
\pgfpathmoveto{\pgfqpoint{0.000000in}{0.000000in}}%
\pgfpathlineto{\pgfqpoint{-0.027778in}{0.000000in}}%
\pgfusepath{stroke,fill}%
}%
\begin{pgfscope}%
\pgfsys@transformshift{0.571156in}{0.870275in}%
\pgfsys@useobject{currentmarker}{}%
\end{pgfscope}%
\end{pgfscope}%
\begin{pgfscope}%
\pgfsetbuttcap%
\pgfsetroundjoin%
\definecolor{currentfill}{rgb}{0.000000,0.000000,0.000000}%
\pgfsetfillcolor{currentfill}%
\pgfsetlinewidth{0.602250pt}%
\definecolor{currentstroke}{rgb}{0.000000,0.000000,0.000000}%
\pgfsetstrokecolor{currentstroke}%
\pgfsetdash{}{0pt}%
\pgfsys@defobject{currentmarker}{\pgfqpoint{-0.027778in}{0.000000in}}{\pgfqpoint{0.000000in}{0.000000in}}{%
\pgfpathmoveto{\pgfqpoint{0.000000in}{0.000000in}}%
\pgfpathlineto{\pgfqpoint{-0.027778in}{0.000000in}}%
\pgfusepath{stroke,fill}%
}%
\begin{pgfscope}%
\pgfsys@transformshift{0.571156in}{0.931355in}%
\pgfsys@useobject{currentmarker}{}%
\end{pgfscope}%
\end{pgfscope}%
\begin{pgfscope}%
\pgfsetbuttcap%
\pgfsetroundjoin%
\definecolor{currentfill}{rgb}{0.000000,0.000000,0.000000}%
\pgfsetfillcolor{currentfill}%
\pgfsetlinewidth{0.602250pt}%
\definecolor{currentstroke}{rgb}{0.000000,0.000000,0.000000}%
\pgfsetstrokecolor{currentstroke}%
\pgfsetdash{}{0pt}%
\pgfsys@defobject{currentmarker}{\pgfqpoint{-0.027778in}{0.000000in}}{\pgfqpoint{0.000000in}{0.000000in}}{%
\pgfpathmoveto{\pgfqpoint{0.000000in}{0.000000in}}%
\pgfpathlineto{\pgfqpoint{-0.027778in}{0.000000in}}%
\pgfusepath{stroke,fill}%
}%
\begin{pgfscope}%
\pgfsys@transformshift{0.571156in}{0.974692in}%
\pgfsys@useobject{currentmarker}{}%
\end{pgfscope}%
\end{pgfscope}%
\begin{pgfscope}%
\pgfsetbuttcap%
\pgfsetroundjoin%
\definecolor{currentfill}{rgb}{0.000000,0.000000,0.000000}%
\pgfsetfillcolor{currentfill}%
\pgfsetlinewidth{0.602250pt}%
\definecolor{currentstroke}{rgb}{0.000000,0.000000,0.000000}%
\pgfsetstrokecolor{currentstroke}%
\pgfsetdash{}{0pt}%
\pgfsys@defobject{currentmarker}{\pgfqpoint{-0.027778in}{0.000000in}}{\pgfqpoint{0.000000in}{0.000000in}}{%
\pgfpathmoveto{\pgfqpoint{0.000000in}{0.000000in}}%
\pgfpathlineto{\pgfqpoint{-0.027778in}{0.000000in}}%
\pgfusepath{stroke,fill}%
}%
\begin{pgfscope}%
\pgfsys@transformshift{0.571156in}{1.008307in}%
\pgfsys@useobject{currentmarker}{}%
\end{pgfscope}%
\end{pgfscope}%
\begin{pgfscope}%
\pgfsetbuttcap%
\pgfsetroundjoin%
\definecolor{currentfill}{rgb}{0.000000,0.000000,0.000000}%
\pgfsetfillcolor{currentfill}%
\pgfsetlinewidth{0.602250pt}%
\definecolor{currentstroke}{rgb}{0.000000,0.000000,0.000000}%
\pgfsetstrokecolor{currentstroke}%
\pgfsetdash{}{0pt}%
\pgfsys@defobject{currentmarker}{\pgfqpoint{-0.027778in}{0.000000in}}{\pgfqpoint{0.000000in}{0.000000in}}{%
\pgfpathmoveto{\pgfqpoint{0.000000in}{0.000000in}}%
\pgfpathlineto{\pgfqpoint{-0.027778in}{0.000000in}}%
\pgfusepath{stroke,fill}%
}%
\begin{pgfscope}%
\pgfsys@transformshift{0.571156in}{1.035772in}%
\pgfsys@useobject{currentmarker}{}%
\end{pgfscope}%
\end{pgfscope}%
\begin{pgfscope}%
\pgfsetbuttcap%
\pgfsetroundjoin%
\definecolor{currentfill}{rgb}{0.000000,0.000000,0.000000}%
\pgfsetfillcolor{currentfill}%
\pgfsetlinewidth{0.602250pt}%
\definecolor{currentstroke}{rgb}{0.000000,0.000000,0.000000}%
\pgfsetstrokecolor{currentstroke}%
\pgfsetdash{}{0pt}%
\pgfsys@defobject{currentmarker}{\pgfqpoint{-0.027778in}{0.000000in}}{\pgfqpoint{0.000000in}{0.000000in}}{%
\pgfpathmoveto{\pgfqpoint{0.000000in}{0.000000in}}%
\pgfpathlineto{\pgfqpoint{-0.027778in}{0.000000in}}%
\pgfusepath{stroke,fill}%
}%
\begin{pgfscope}%
\pgfsys@transformshift{0.571156in}{1.058993in}%
\pgfsys@useobject{currentmarker}{}%
\end{pgfscope}%
\end{pgfscope}%
\begin{pgfscope}%
\pgfsetbuttcap%
\pgfsetroundjoin%
\definecolor{currentfill}{rgb}{0.000000,0.000000,0.000000}%
\pgfsetfillcolor{currentfill}%
\pgfsetlinewidth{0.602250pt}%
\definecolor{currentstroke}{rgb}{0.000000,0.000000,0.000000}%
\pgfsetstrokecolor{currentstroke}%
\pgfsetdash{}{0pt}%
\pgfsys@defobject{currentmarker}{\pgfqpoint{-0.027778in}{0.000000in}}{\pgfqpoint{0.000000in}{0.000000in}}{%
\pgfpathmoveto{\pgfqpoint{0.000000in}{0.000000in}}%
\pgfpathlineto{\pgfqpoint{-0.027778in}{0.000000in}}%
\pgfusepath{stroke,fill}%
}%
\begin{pgfscope}%
\pgfsys@transformshift{0.571156in}{1.079109in}%
\pgfsys@useobject{currentmarker}{}%
\end{pgfscope}%
\end{pgfscope}%
\begin{pgfscope}%
\pgfsetbuttcap%
\pgfsetroundjoin%
\definecolor{currentfill}{rgb}{0.000000,0.000000,0.000000}%
\pgfsetfillcolor{currentfill}%
\pgfsetlinewidth{0.602250pt}%
\definecolor{currentstroke}{rgb}{0.000000,0.000000,0.000000}%
\pgfsetstrokecolor{currentstroke}%
\pgfsetdash{}{0pt}%
\pgfsys@defobject{currentmarker}{\pgfqpoint{-0.027778in}{0.000000in}}{\pgfqpoint{0.000000in}{0.000000in}}{%
\pgfpathmoveto{\pgfqpoint{0.000000in}{0.000000in}}%
\pgfpathlineto{\pgfqpoint{-0.027778in}{0.000000in}}%
\pgfusepath{stroke,fill}%
}%
\begin{pgfscope}%
\pgfsys@transformshift{0.571156in}{1.096852in}%
\pgfsys@useobject{currentmarker}{}%
\end{pgfscope}%
\end{pgfscope}%
\begin{pgfscope}%
\pgfsetbuttcap%
\pgfsetroundjoin%
\definecolor{currentfill}{rgb}{0.000000,0.000000,0.000000}%
\pgfsetfillcolor{currentfill}%
\pgfsetlinewidth{0.602250pt}%
\definecolor{currentstroke}{rgb}{0.000000,0.000000,0.000000}%
\pgfsetstrokecolor{currentstroke}%
\pgfsetdash{}{0pt}%
\pgfsys@defobject{currentmarker}{\pgfqpoint{-0.027778in}{0.000000in}}{\pgfqpoint{0.000000in}{0.000000in}}{%
\pgfpathmoveto{\pgfqpoint{0.000000in}{0.000000in}}%
\pgfpathlineto{\pgfqpoint{-0.027778in}{0.000000in}}%
\pgfusepath{stroke,fill}%
}%
\begin{pgfscope}%
\pgfsys@transformshift{0.571156in}{1.217140in}%
\pgfsys@useobject{currentmarker}{}%
\end{pgfscope}%
\end{pgfscope}%
\begin{pgfscope}%
\pgfsetbuttcap%
\pgfsetroundjoin%
\definecolor{currentfill}{rgb}{0.000000,0.000000,0.000000}%
\pgfsetfillcolor{currentfill}%
\pgfsetlinewidth{0.602250pt}%
\definecolor{currentstroke}{rgb}{0.000000,0.000000,0.000000}%
\pgfsetstrokecolor{currentstroke}%
\pgfsetdash{}{0pt}%
\pgfsys@defobject{currentmarker}{\pgfqpoint{-0.027778in}{0.000000in}}{\pgfqpoint{0.000000in}{0.000000in}}{%
\pgfpathmoveto{\pgfqpoint{0.000000in}{0.000000in}}%
\pgfpathlineto{\pgfqpoint{-0.027778in}{0.000000in}}%
\pgfusepath{stroke,fill}%
}%
\begin{pgfscope}%
\pgfsys@transformshift{0.571156in}{1.278220in}%
\pgfsys@useobject{currentmarker}{}%
\end{pgfscope}%
\end{pgfscope}%
\begin{pgfscope}%
\pgfsetbuttcap%
\pgfsetroundjoin%
\definecolor{currentfill}{rgb}{0.000000,0.000000,0.000000}%
\pgfsetfillcolor{currentfill}%
\pgfsetlinewidth{0.602250pt}%
\definecolor{currentstroke}{rgb}{0.000000,0.000000,0.000000}%
\pgfsetstrokecolor{currentstroke}%
\pgfsetdash{}{0pt}%
\pgfsys@defobject{currentmarker}{\pgfqpoint{-0.027778in}{0.000000in}}{\pgfqpoint{0.000000in}{0.000000in}}{%
\pgfpathmoveto{\pgfqpoint{0.000000in}{0.000000in}}%
\pgfpathlineto{\pgfqpoint{-0.027778in}{0.000000in}}%
\pgfusepath{stroke,fill}%
}%
\begin{pgfscope}%
\pgfsys@transformshift{0.571156in}{1.321557in}%
\pgfsys@useobject{currentmarker}{}%
\end{pgfscope}%
\end{pgfscope}%
\begin{pgfscope}%
\pgfsetbuttcap%
\pgfsetroundjoin%
\definecolor{currentfill}{rgb}{0.000000,0.000000,0.000000}%
\pgfsetfillcolor{currentfill}%
\pgfsetlinewidth{0.602250pt}%
\definecolor{currentstroke}{rgb}{0.000000,0.000000,0.000000}%
\pgfsetstrokecolor{currentstroke}%
\pgfsetdash{}{0pt}%
\pgfsys@defobject{currentmarker}{\pgfqpoint{-0.027778in}{0.000000in}}{\pgfqpoint{0.000000in}{0.000000in}}{%
\pgfpathmoveto{\pgfqpoint{0.000000in}{0.000000in}}%
\pgfpathlineto{\pgfqpoint{-0.027778in}{0.000000in}}%
\pgfusepath{stroke,fill}%
}%
\begin{pgfscope}%
\pgfsys@transformshift{0.571156in}{1.355171in}%
\pgfsys@useobject{currentmarker}{}%
\end{pgfscope}%
\end{pgfscope}%
\begin{pgfscope}%
\pgfsetbuttcap%
\pgfsetroundjoin%
\definecolor{currentfill}{rgb}{0.000000,0.000000,0.000000}%
\pgfsetfillcolor{currentfill}%
\pgfsetlinewidth{0.602250pt}%
\definecolor{currentstroke}{rgb}{0.000000,0.000000,0.000000}%
\pgfsetstrokecolor{currentstroke}%
\pgfsetdash{}{0pt}%
\pgfsys@defobject{currentmarker}{\pgfqpoint{-0.027778in}{0.000000in}}{\pgfqpoint{0.000000in}{0.000000in}}{%
\pgfpathmoveto{\pgfqpoint{0.000000in}{0.000000in}}%
\pgfpathlineto{\pgfqpoint{-0.027778in}{0.000000in}}%
\pgfusepath{stroke,fill}%
}%
\begin{pgfscope}%
\pgfsys@transformshift{0.571156in}{1.382636in}%
\pgfsys@useobject{currentmarker}{}%
\end{pgfscope}%
\end{pgfscope}%
\begin{pgfscope}%
\pgfsetbuttcap%
\pgfsetroundjoin%
\definecolor{currentfill}{rgb}{0.000000,0.000000,0.000000}%
\pgfsetfillcolor{currentfill}%
\pgfsetlinewidth{0.602250pt}%
\definecolor{currentstroke}{rgb}{0.000000,0.000000,0.000000}%
\pgfsetstrokecolor{currentstroke}%
\pgfsetdash{}{0pt}%
\pgfsys@defobject{currentmarker}{\pgfqpoint{-0.027778in}{0.000000in}}{\pgfqpoint{0.000000in}{0.000000in}}{%
\pgfpathmoveto{\pgfqpoint{0.000000in}{0.000000in}}%
\pgfpathlineto{\pgfqpoint{-0.027778in}{0.000000in}}%
\pgfusepath{stroke,fill}%
}%
\begin{pgfscope}%
\pgfsys@transformshift{0.571156in}{1.405858in}%
\pgfsys@useobject{currentmarker}{}%
\end{pgfscope}%
\end{pgfscope}%
\begin{pgfscope}%
\pgfsetbuttcap%
\pgfsetroundjoin%
\definecolor{currentfill}{rgb}{0.000000,0.000000,0.000000}%
\pgfsetfillcolor{currentfill}%
\pgfsetlinewidth{0.602250pt}%
\definecolor{currentstroke}{rgb}{0.000000,0.000000,0.000000}%
\pgfsetstrokecolor{currentstroke}%
\pgfsetdash{}{0pt}%
\pgfsys@defobject{currentmarker}{\pgfqpoint{-0.027778in}{0.000000in}}{\pgfqpoint{0.000000in}{0.000000in}}{%
\pgfpathmoveto{\pgfqpoint{0.000000in}{0.000000in}}%
\pgfpathlineto{\pgfqpoint{-0.027778in}{0.000000in}}%
\pgfusepath{stroke,fill}%
}%
\begin{pgfscope}%
\pgfsys@transformshift{0.571156in}{1.425973in}%
\pgfsys@useobject{currentmarker}{}%
\end{pgfscope}%
\end{pgfscope}%
\begin{pgfscope}%
\pgfsetbuttcap%
\pgfsetroundjoin%
\definecolor{currentfill}{rgb}{0.000000,0.000000,0.000000}%
\pgfsetfillcolor{currentfill}%
\pgfsetlinewidth{0.602250pt}%
\definecolor{currentstroke}{rgb}{0.000000,0.000000,0.000000}%
\pgfsetstrokecolor{currentstroke}%
\pgfsetdash{}{0pt}%
\pgfsys@defobject{currentmarker}{\pgfqpoint{-0.027778in}{0.000000in}}{\pgfqpoint{0.000000in}{0.000000in}}{%
\pgfpathmoveto{\pgfqpoint{0.000000in}{0.000000in}}%
\pgfpathlineto{\pgfqpoint{-0.027778in}{0.000000in}}%
\pgfusepath{stroke,fill}%
}%
\begin{pgfscope}%
\pgfsys@transformshift{0.571156in}{1.443716in}%
\pgfsys@useobject{currentmarker}{}%
\end{pgfscope}%
\end{pgfscope}%
\begin{pgfscope}%
\pgfsetbuttcap%
\pgfsetroundjoin%
\definecolor{currentfill}{rgb}{0.000000,0.000000,0.000000}%
\pgfsetfillcolor{currentfill}%
\pgfsetlinewidth{0.602250pt}%
\definecolor{currentstroke}{rgb}{0.000000,0.000000,0.000000}%
\pgfsetstrokecolor{currentstroke}%
\pgfsetdash{}{0pt}%
\pgfsys@defobject{currentmarker}{\pgfqpoint{-0.027778in}{0.000000in}}{\pgfqpoint{0.000000in}{0.000000in}}{%
\pgfpathmoveto{\pgfqpoint{0.000000in}{0.000000in}}%
\pgfpathlineto{\pgfqpoint{-0.027778in}{0.000000in}}%
\pgfusepath{stroke,fill}%
}%
\begin{pgfscope}%
\pgfsys@transformshift{0.571156in}{1.564005in}%
\pgfsys@useobject{currentmarker}{}%
\end{pgfscope}%
\end{pgfscope}%
\begin{pgfscope}%
\pgfsetbuttcap%
\pgfsetroundjoin%
\definecolor{currentfill}{rgb}{0.000000,0.000000,0.000000}%
\pgfsetfillcolor{currentfill}%
\pgfsetlinewidth{0.602250pt}%
\definecolor{currentstroke}{rgb}{0.000000,0.000000,0.000000}%
\pgfsetstrokecolor{currentstroke}%
\pgfsetdash{}{0pt}%
\pgfsys@defobject{currentmarker}{\pgfqpoint{-0.027778in}{0.000000in}}{\pgfqpoint{0.000000in}{0.000000in}}{%
\pgfpathmoveto{\pgfqpoint{0.000000in}{0.000000in}}%
\pgfpathlineto{\pgfqpoint{-0.027778in}{0.000000in}}%
\pgfusepath{stroke,fill}%
}%
\begin{pgfscope}%
\pgfsys@transformshift{0.571156in}{1.625084in}%
\pgfsys@useobject{currentmarker}{}%
\end{pgfscope}%
\end{pgfscope}%
\begin{pgfscope}%
\pgfsetbuttcap%
\pgfsetroundjoin%
\definecolor{currentfill}{rgb}{0.000000,0.000000,0.000000}%
\pgfsetfillcolor{currentfill}%
\pgfsetlinewidth{0.602250pt}%
\definecolor{currentstroke}{rgb}{0.000000,0.000000,0.000000}%
\pgfsetstrokecolor{currentstroke}%
\pgfsetdash{}{0pt}%
\pgfsys@defobject{currentmarker}{\pgfqpoint{-0.027778in}{0.000000in}}{\pgfqpoint{0.000000in}{0.000000in}}{%
\pgfpathmoveto{\pgfqpoint{0.000000in}{0.000000in}}%
\pgfpathlineto{\pgfqpoint{-0.027778in}{0.000000in}}%
\pgfusepath{stroke,fill}%
}%
\begin{pgfscope}%
\pgfsys@transformshift{0.571156in}{1.668421in}%
\pgfsys@useobject{currentmarker}{}%
\end{pgfscope}%
\end{pgfscope}%
\begin{pgfscope}%
\pgfsetbuttcap%
\pgfsetroundjoin%
\definecolor{currentfill}{rgb}{0.000000,0.000000,0.000000}%
\pgfsetfillcolor{currentfill}%
\pgfsetlinewidth{0.602250pt}%
\definecolor{currentstroke}{rgb}{0.000000,0.000000,0.000000}%
\pgfsetstrokecolor{currentstroke}%
\pgfsetdash{}{0pt}%
\pgfsys@defobject{currentmarker}{\pgfqpoint{-0.027778in}{0.000000in}}{\pgfqpoint{0.000000in}{0.000000in}}{%
\pgfpathmoveto{\pgfqpoint{0.000000in}{0.000000in}}%
\pgfpathlineto{\pgfqpoint{-0.027778in}{0.000000in}}%
\pgfusepath{stroke,fill}%
}%
\begin{pgfscope}%
\pgfsys@transformshift{0.571156in}{1.702036in}%
\pgfsys@useobject{currentmarker}{}%
\end{pgfscope}%
\end{pgfscope}%
\begin{pgfscope}%
\pgfsetbuttcap%
\pgfsetroundjoin%
\definecolor{currentfill}{rgb}{0.000000,0.000000,0.000000}%
\pgfsetfillcolor{currentfill}%
\pgfsetlinewidth{0.602250pt}%
\definecolor{currentstroke}{rgb}{0.000000,0.000000,0.000000}%
\pgfsetstrokecolor{currentstroke}%
\pgfsetdash{}{0pt}%
\pgfsys@defobject{currentmarker}{\pgfqpoint{-0.027778in}{0.000000in}}{\pgfqpoint{0.000000in}{0.000000in}}{%
\pgfpathmoveto{\pgfqpoint{0.000000in}{0.000000in}}%
\pgfpathlineto{\pgfqpoint{-0.027778in}{0.000000in}}%
\pgfusepath{stroke,fill}%
}%
\begin{pgfscope}%
\pgfsys@transformshift{0.571156in}{1.729501in}%
\pgfsys@useobject{currentmarker}{}%
\end{pgfscope}%
\end{pgfscope}%
\begin{pgfscope}%
\pgfsetbuttcap%
\pgfsetroundjoin%
\definecolor{currentfill}{rgb}{0.000000,0.000000,0.000000}%
\pgfsetfillcolor{currentfill}%
\pgfsetlinewidth{0.602250pt}%
\definecolor{currentstroke}{rgb}{0.000000,0.000000,0.000000}%
\pgfsetstrokecolor{currentstroke}%
\pgfsetdash{}{0pt}%
\pgfsys@defobject{currentmarker}{\pgfqpoint{-0.027778in}{0.000000in}}{\pgfqpoint{0.000000in}{0.000000in}}{%
\pgfpathmoveto{\pgfqpoint{0.000000in}{0.000000in}}%
\pgfpathlineto{\pgfqpoint{-0.027778in}{0.000000in}}%
\pgfusepath{stroke,fill}%
}%
\begin{pgfscope}%
\pgfsys@transformshift{0.571156in}{1.752722in}%
\pgfsys@useobject{currentmarker}{}%
\end{pgfscope}%
\end{pgfscope}%
\begin{pgfscope}%
\pgfsetbuttcap%
\pgfsetroundjoin%
\definecolor{currentfill}{rgb}{0.000000,0.000000,0.000000}%
\pgfsetfillcolor{currentfill}%
\pgfsetlinewidth{0.602250pt}%
\definecolor{currentstroke}{rgb}{0.000000,0.000000,0.000000}%
\pgfsetstrokecolor{currentstroke}%
\pgfsetdash{}{0pt}%
\pgfsys@defobject{currentmarker}{\pgfqpoint{-0.027778in}{0.000000in}}{\pgfqpoint{0.000000in}{0.000000in}}{%
\pgfpathmoveto{\pgfqpoint{0.000000in}{0.000000in}}%
\pgfpathlineto{\pgfqpoint{-0.027778in}{0.000000in}}%
\pgfusepath{stroke,fill}%
}%
\begin{pgfscope}%
\pgfsys@transformshift{0.571156in}{1.772838in}%
\pgfsys@useobject{currentmarker}{}%
\end{pgfscope}%
\end{pgfscope}%
\begin{pgfscope}%
\pgfsetbuttcap%
\pgfsetroundjoin%
\definecolor{currentfill}{rgb}{0.000000,0.000000,0.000000}%
\pgfsetfillcolor{currentfill}%
\pgfsetlinewidth{0.602250pt}%
\definecolor{currentstroke}{rgb}{0.000000,0.000000,0.000000}%
\pgfsetstrokecolor{currentstroke}%
\pgfsetdash{}{0pt}%
\pgfsys@defobject{currentmarker}{\pgfqpoint{-0.027778in}{0.000000in}}{\pgfqpoint{0.000000in}{0.000000in}}{%
\pgfpathmoveto{\pgfqpoint{0.000000in}{0.000000in}}%
\pgfpathlineto{\pgfqpoint{-0.027778in}{0.000000in}}%
\pgfusepath{stroke,fill}%
}%
\begin{pgfscope}%
\pgfsys@transformshift{0.571156in}{1.790581in}%
\pgfsys@useobject{currentmarker}{}%
\end{pgfscope}%
\end{pgfscope}%
\begin{pgfscope}%
\pgfsetbuttcap%
\pgfsetroundjoin%
\definecolor{currentfill}{rgb}{0.000000,0.000000,0.000000}%
\pgfsetfillcolor{currentfill}%
\pgfsetlinewidth{0.602250pt}%
\definecolor{currentstroke}{rgb}{0.000000,0.000000,0.000000}%
\pgfsetstrokecolor{currentstroke}%
\pgfsetdash{}{0pt}%
\pgfsys@defobject{currentmarker}{\pgfqpoint{-0.027778in}{0.000000in}}{\pgfqpoint{0.000000in}{0.000000in}}{%
\pgfpathmoveto{\pgfqpoint{0.000000in}{0.000000in}}%
\pgfpathlineto{\pgfqpoint{-0.027778in}{0.000000in}}%
\pgfusepath{stroke,fill}%
}%
\begin{pgfscope}%
\pgfsys@transformshift{0.571156in}{1.910869in}%
\pgfsys@useobject{currentmarker}{}%
\end{pgfscope}%
\end{pgfscope}%
\begin{pgfscope}%
\pgfsetbuttcap%
\pgfsetroundjoin%
\definecolor{currentfill}{rgb}{0.000000,0.000000,0.000000}%
\pgfsetfillcolor{currentfill}%
\pgfsetlinewidth{0.602250pt}%
\definecolor{currentstroke}{rgb}{0.000000,0.000000,0.000000}%
\pgfsetstrokecolor{currentstroke}%
\pgfsetdash{}{0pt}%
\pgfsys@defobject{currentmarker}{\pgfqpoint{-0.027778in}{0.000000in}}{\pgfqpoint{0.000000in}{0.000000in}}{%
\pgfpathmoveto{\pgfqpoint{0.000000in}{0.000000in}}%
\pgfpathlineto{\pgfqpoint{-0.027778in}{0.000000in}}%
\pgfusepath{stroke,fill}%
}%
\begin{pgfscope}%
\pgfsys@transformshift{0.571156in}{1.971949in}%
\pgfsys@useobject{currentmarker}{}%
\end{pgfscope}%
\end{pgfscope}%
\begin{pgfscope}%
\pgfsetbuttcap%
\pgfsetroundjoin%
\definecolor{currentfill}{rgb}{0.000000,0.000000,0.000000}%
\pgfsetfillcolor{currentfill}%
\pgfsetlinewidth{0.602250pt}%
\definecolor{currentstroke}{rgb}{0.000000,0.000000,0.000000}%
\pgfsetstrokecolor{currentstroke}%
\pgfsetdash{}{0pt}%
\pgfsys@defobject{currentmarker}{\pgfqpoint{-0.027778in}{0.000000in}}{\pgfqpoint{0.000000in}{0.000000in}}{%
\pgfpathmoveto{\pgfqpoint{0.000000in}{0.000000in}}%
\pgfpathlineto{\pgfqpoint{-0.027778in}{0.000000in}}%
\pgfusepath{stroke,fill}%
}%
\begin{pgfscope}%
\pgfsys@transformshift{0.571156in}{2.015286in}%
\pgfsys@useobject{currentmarker}{}%
\end{pgfscope}%
\end{pgfscope}%
\begin{pgfscope}%
\pgfsetbuttcap%
\pgfsetroundjoin%
\definecolor{currentfill}{rgb}{0.000000,0.000000,0.000000}%
\pgfsetfillcolor{currentfill}%
\pgfsetlinewidth{0.602250pt}%
\definecolor{currentstroke}{rgb}{0.000000,0.000000,0.000000}%
\pgfsetstrokecolor{currentstroke}%
\pgfsetdash{}{0pt}%
\pgfsys@defobject{currentmarker}{\pgfqpoint{-0.027778in}{0.000000in}}{\pgfqpoint{0.000000in}{0.000000in}}{%
\pgfpathmoveto{\pgfqpoint{0.000000in}{0.000000in}}%
\pgfpathlineto{\pgfqpoint{-0.027778in}{0.000000in}}%
\pgfusepath{stroke,fill}%
}%
\begin{pgfscope}%
\pgfsys@transformshift{0.571156in}{2.048900in}%
\pgfsys@useobject{currentmarker}{}%
\end{pgfscope}%
\end{pgfscope}%
\begin{pgfscope}%
\pgfsetbuttcap%
\pgfsetroundjoin%
\definecolor{currentfill}{rgb}{0.000000,0.000000,0.000000}%
\pgfsetfillcolor{currentfill}%
\pgfsetlinewidth{0.602250pt}%
\definecolor{currentstroke}{rgb}{0.000000,0.000000,0.000000}%
\pgfsetstrokecolor{currentstroke}%
\pgfsetdash{}{0pt}%
\pgfsys@defobject{currentmarker}{\pgfqpoint{-0.027778in}{0.000000in}}{\pgfqpoint{0.000000in}{0.000000in}}{%
\pgfpathmoveto{\pgfqpoint{0.000000in}{0.000000in}}%
\pgfpathlineto{\pgfqpoint{-0.027778in}{0.000000in}}%
\pgfusepath{stroke,fill}%
}%
\begin{pgfscope}%
\pgfsys@transformshift{0.571156in}{2.076366in}%
\pgfsys@useobject{currentmarker}{}%
\end{pgfscope}%
\end{pgfscope}%
\begin{pgfscope}%
\pgfsetbuttcap%
\pgfsetroundjoin%
\definecolor{currentfill}{rgb}{0.000000,0.000000,0.000000}%
\pgfsetfillcolor{currentfill}%
\pgfsetlinewidth{0.602250pt}%
\definecolor{currentstroke}{rgb}{0.000000,0.000000,0.000000}%
\pgfsetstrokecolor{currentstroke}%
\pgfsetdash{}{0pt}%
\pgfsys@defobject{currentmarker}{\pgfqpoint{-0.027778in}{0.000000in}}{\pgfqpoint{0.000000in}{0.000000in}}{%
\pgfpathmoveto{\pgfqpoint{0.000000in}{0.000000in}}%
\pgfpathlineto{\pgfqpoint{-0.027778in}{0.000000in}}%
\pgfusepath{stroke,fill}%
}%
\begin{pgfscope}%
\pgfsys@transformshift{0.571156in}{2.099587in}%
\pgfsys@useobject{currentmarker}{}%
\end{pgfscope}%
\end{pgfscope}%
\begin{pgfscope}%
\pgfsetbuttcap%
\pgfsetroundjoin%
\definecolor{currentfill}{rgb}{0.000000,0.000000,0.000000}%
\pgfsetfillcolor{currentfill}%
\pgfsetlinewidth{0.602250pt}%
\definecolor{currentstroke}{rgb}{0.000000,0.000000,0.000000}%
\pgfsetstrokecolor{currentstroke}%
\pgfsetdash{}{0pt}%
\pgfsys@defobject{currentmarker}{\pgfqpoint{-0.027778in}{0.000000in}}{\pgfqpoint{0.000000in}{0.000000in}}{%
\pgfpathmoveto{\pgfqpoint{0.000000in}{0.000000in}}%
\pgfpathlineto{\pgfqpoint{-0.027778in}{0.000000in}}%
\pgfusepath{stroke,fill}%
}%
\begin{pgfscope}%
\pgfsys@transformshift{0.571156in}{2.119702in}%
\pgfsys@useobject{currentmarker}{}%
\end{pgfscope}%
\end{pgfscope}%
\begin{pgfscope}%
\pgfsetbuttcap%
\pgfsetroundjoin%
\definecolor{currentfill}{rgb}{0.000000,0.000000,0.000000}%
\pgfsetfillcolor{currentfill}%
\pgfsetlinewidth{0.602250pt}%
\definecolor{currentstroke}{rgb}{0.000000,0.000000,0.000000}%
\pgfsetstrokecolor{currentstroke}%
\pgfsetdash{}{0pt}%
\pgfsys@defobject{currentmarker}{\pgfqpoint{-0.027778in}{0.000000in}}{\pgfqpoint{0.000000in}{0.000000in}}{%
\pgfpathmoveto{\pgfqpoint{0.000000in}{0.000000in}}%
\pgfpathlineto{\pgfqpoint{-0.027778in}{0.000000in}}%
\pgfusepath{stroke,fill}%
}%
\begin{pgfscope}%
\pgfsys@transformshift{0.571156in}{2.137445in}%
\pgfsys@useobject{currentmarker}{}%
\end{pgfscope}%
\end{pgfscope}%
\begin{pgfscope}%
\pgfsetbuttcap%
\pgfsetroundjoin%
\definecolor{currentfill}{rgb}{0.000000,0.000000,0.000000}%
\pgfsetfillcolor{currentfill}%
\pgfsetlinewidth{0.602250pt}%
\definecolor{currentstroke}{rgb}{0.000000,0.000000,0.000000}%
\pgfsetstrokecolor{currentstroke}%
\pgfsetdash{}{0pt}%
\pgfsys@defobject{currentmarker}{\pgfqpoint{-0.027778in}{0.000000in}}{\pgfqpoint{0.000000in}{0.000000in}}{%
\pgfpathmoveto{\pgfqpoint{0.000000in}{0.000000in}}%
\pgfpathlineto{\pgfqpoint{-0.027778in}{0.000000in}}%
\pgfusepath{stroke,fill}%
}%
\begin{pgfscope}%
\pgfsys@transformshift{0.571156in}{2.257734in}%
\pgfsys@useobject{currentmarker}{}%
\end{pgfscope}%
\end{pgfscope}%
\begin{pgfscope}%
\pgfsetbuttcap%
\pgfsetroundjoin%
\definecolor{currentfill}{rgb}{0.000000,0.000000,0.000000}%
\pgfsetfillcolor{currentfill}%
\pgfsetlinewidth{0.602250pt}%
\definecolor{currentstroke}{rgb}{0.000000,0.000000,0.000000}%
\pgfsetstrokecolor{currentstroke}%
\pgfsetdash{}{0pt}%
\pgfsys@defobject{currentmarker}{\pgfqpoint{-0.027778in}{0.000000in}}{\pgfqpoint{0.000000in}{0.000000in}}{%
\pgfpathmoveto{\pgfqpoint{0.000000in}{0.000000in}}%
\pgfpathlineto{\pgfqpoint{-0.027778in}{0.000000in}}%
\pgfusepath{stroke,fill}%
}%
\begin{pgfscope}%
\pgfsys@transformshift{0.571156in}{2.318814in}%
\pgfsys@useobject{currentmarker}{}%
\end{pgfscope}%
\end{pgfscope}%
\begin{pgfscope}%
\pgfsetbuttcap%
\pgfsetroundjoin%
\definecolor{currentfill}{rgb}{0.000000,0.000000,0.000000}%
\pgfsetfillcolor{currentfill}%
\pgfsetlinewidth{0.602250pt}%
\definecolor{currentstroke}{rgb}{0.000000,0.000000,0.000000}%
\pgfsetstrokecolor{currentstroke}%
\pgfsetdash{}{0pt}%
\pgfsys@defobject{currentmarker}{\pgfqpoint{-0.027778in}{0.000000in}}{\pgfqpoint{0.000000in}{0.000000in}}{%
\pgfpathmoveto{\pgfqpoint{0.000000in}{0.000000in}}%
\pgfpathlineto{\pgfqpoint{-0.027778in}{0.000000in}}%
\pgfusepath{stroke,fill}%
}%
\begin{pgfscope}%
\pgfsys@transformshift{0.571156in}{2.362150in}%
\pgfsys@useobject{currentmarker}{}%
\end{pgfscope}%
\end{pgfscope}%
\begin{pgfscope}%
\pgfsetbuttcap%
\pgfsetroundjoin%
\definecolor{currentfill}{rgb}{0.000000,0.000000,0.000000}%
\pgfsetfillcolor{currentfill}%
\pgfsetlinewidth{0.602250pt}%
\definecolor{currentstroke}{rgb}{0.000000,0.000000,0.000000}%
\pgfsetstrokecolor{currentstroke}%
\pgfsetdash{}{0pt}%
\pgfsys@defobject{currentmarker}{\pgfqpoint{-0.027778in}{0.000000in}}{\pgfqpoint{0.000000in}{0.000000in}}{%
\pgfpathmoveto{\pgfqpoint{0.000000in}{0.000000in}}%
\pgfpathlineto{\pgfqpoint{-0.027778in}{0.000000in}}%
\pgfusepath{stroke,fill}%
}%
\begin{pgfscope}%
\pgfsys@transformshift{0.571156in}{2.395765in}%
\pgfsys@useobject{currentmarker}{}%
\end{pgfscope}%
\end{pgfscope}%
\begin{pgfscope}%
\pgfsetbuttcap%
\pgfsetroundjoin%
\definecolor{currentfill}{rgb}{0.000000,0.000000,0.000000}%
\pgfsetfillcolor{currentfill}%
\pgfsetlinewidth{0.602250pt}%
\definecolor{currentstroke}{rgb}{0.000000,0.000000,0.000000}%
\pgfsetstrokecolor{currentstroke}%
\pgfsetdash{}{0pt}%
\pgfsys@defobject{currentmarker}{\pgfqpoint{-0.027778in}{0.000000in}}{\pgfqpoint{0.000000in}{0.000000in}}{%
\pgfpathmoveto{\pgfqpoint{0.000000in}{0.000000in}}%
\pgfpathlineto{\pgfqpoint{-0.027778in}{0.000000in}}%
\pgfusepath{stroke,fill}%
}%
\begin{pgfscope}%
\pgfsys@transformshift{0.571156in}{2.423230in}%
\pgfsys@useobject{currentmarker}{}%
\end{pgfscope}%
\end{pgfscope}%
\begin{pgfscope}%
\pgfsetbuttcap%
\pgfsetroundjoin%
\definecolor{currentfill}{rgb}{0.000000,0.000000,0.000000}%
\pgfsetfillcolor{currentfill}%
\pgfsetlinewidth{0.602250pt}%
\definecolor{currentstroke}{rgb}{0.000000,0.000000,0.000000}%
\pgfsetstrokecolor{currentstroke}%
\pgfsetdash{}{0pt}%
\pgfsys@defobject{currentmarker}{\pgfqpoint{-0.027778in}{0.000000in}}{\pgfqpoint{0.000000in}{0.000000in}}{%
\pgfpathmoveto{\pgfqpoint{0.000000in}{0.000000in}}%
\pgfpathlineto{\pgfqpoint{-0.027778in}{0.000000in}}%
\pgfusepath{stroke,fill}%
}%
\begin{pgfscope}%
\pgfsys@transformshift{0.571156in}{2.446452in}%
\pgfsys@useobject{currentmarker}{}%
\end{pgfscope}%
\end{pgfscope}%
\begin{pgfscope}%
\pgfsetbuttcap%
\pgfsetroundjoin%
\definecolor{currentfill}{rgb}{0.000000,0.000000,0.000000}%
\pgfsetfillcolor{currentfill}%
\pgfsetlinewidth{0.602250pt}%
\definecolor{currentstroke}{rgb}{0.000000,0.000000,0.000000}%
\pgfsetstrokecolor{currentstroke}%
\pgfsetdash{}{0pt}%
\pgfsys@defobject{currentmarker}{\pgfqpoint{-0.027778in}{0.000000in}}{\pgfqpoint{0.000000in}{0.000000in}}{%
\pgfpathmoveto{\pgfqpoint{0.000000in}{0.000000in}}%
\pgfpathlineto{\pgfqpoint{-0.027778in}{0.000000in}}%
\pgfusepath{stroke,fill}%
}%
\begin{pgfscope}%
\pgfsys@transformshift{0.571156in}{2.466567in}%
\pgfsys@useobject{currentmarker}{}%
\end{pgfscope}%
\end{pgfscope}%
\begin{pgfscope}%
\pgfsetbuttcap%
\pgfsetroundjoin%
\definecolor{currentfill}{rgb}{0.000000,0.000000,0.000000}%
\pgfsetfillcolor{currentfill}%
\pgfsetlinewidth{0.602250pt}%
\definecolor{currentstroke}{rgb}{0.000000,0.000000,0.000000}%
\pgfsetstrokecolor{currentstroke}%
\pgfsetdash{}{0pt}%
\pgfsys@defobject{currentmarker}{\pgfqpoint{-0.027778in}{0.000000in}}{\pgfqpoint{0.000000in}{0.000000in}}{%
\pgfpathmoveto{\pgfqpoint{0.000000in}{0.000000in}}%
\pgfpathlineto{\pgfqpoint{-0.027778in}{0.000000in}}%
\pgfusepath{stroke,fill}%
}%
\begin{pgfscope}%
\pgfsys@transformshift{0.571156in}{2.484310in}%
\pgfsys@useobject{currentmarker}{}%
\end{pgfscope}%
\end{pgfscope}%
\begin{pgfscope}%
\pgfsetbuttcap%
\pgfsetroundjoin%
\definecolor{currentfill}{rgb}{0.000000,0.000000,0.000000}%
\pgfsetfillcolor{currentfill}%
\pgfsetlinewidth{0.602250pt}%
\definecolor{currentstroke}{rgb}{0.000000,0.000000,0.000000}%
\pgfsetstrokecolor{currentstroke}%
\pgfsetdash{}{0pt}%
\pgfsys@defobject{currentmarker}{\pgfqpoint{-0.027778in}{0.000000in}}{\pgfqpoint{0.000000in}{0.000000in}}{%
\pgfpathmoveto{\pgfqpoint{0.000000in}{0.000000in}}%
\pgfpathlineto{\pgfqpoint{-0.027778in}{0.000000in}}%
\pgfusepath{stroke,fill}%
}%
\begin{pgfscope}%
\pgfsys@transformshift{0.571156in}{2.604598in}%
\pgfsys@useobject{currentmarker}{}%
\end{pgfscope}%
\end{pgfscope}%
\begin{pgfscope}%
\pgfsetbuttcap%
\pgfsetroundjoin%
\definecolor{currentfill}{rgb}{0.000000,0.000000,0.000000}%
\pgfsetfillcolor{currentfill}%
\pgfsetlinewidth{0.602250pt}%
\definecolor{currentstroke}{rgb}{0.000000,0.000000,0.000000}%
\pgfsetstrokecolor{currentstroke}%
\pgfsetdash{}{0pt}%
\pgfsys@defobject{currentmarker}{\pgfqpoint{-0.027778in}{0.000000in}}{\pgfqpoint{0.000000in}{0.000000in}}{%
\pgfpathmoveto{\pgfqpoint{0.000000in}{0.000000in}}%
\pgfpathlineto{\pgfqpoint{-0.027778in}{0.000000in}}%
\pgfusepath{stroke,fill}%
}%
\begin{pgfscope}%
\pgfsys@transformshift{0.571156in}{2.665678in}%
\pgfsys@useobject{currentmarker}{}%
\end{pgfscope}%
\end{pgfscope}%
\begin{pgfscope}%
\pgfsetbuttcap%
\pgfsetroundjoin%
\definecolor{currentfill}{rgb}{0.000000,0.000000,0.000000}%
\pgfsetfillcolor{currentfill}%
\pgfsetlinewidth{0.602250pt}%
\definecolor{currentstroke}{rgb}{0.000000,0.000000,0.000000}%
\pgfsetstrokecolor{currentstroke}%
\pgfsetdash{}{0pt}%
\pgfsys@defobject{currentmarker}{\pgfqpoint{-0.027778in}{0.000000in}}{\pgfqpoint{0.000000in}{0.000000in}}{%
\pgfpathmoveto{\pgfqpoint{0.000000in}{0.000000in}}%
\pgfpathlineto{\pgfqpoint{-0.027778in}{0.000000in}}%
\pgfusepath{stroke,fill}%
}%
\begin{pgfscope}%
\pgfsys@transformshift{0.571156in}{2.709015in}%
\pgfsys@useobject{currentmarker}{}%
\end{pgfscope}%
\end{pgfscope}%
\begin{pgfscope}%
\pgfsetbuttcap%
\pgfsetroundjoin%
\definecolor{currentfill}{rgb}{0.000000,0.000000,0.000000}%
\pgfsetfillcolor{currentfill}%
\pgfsetlinewidth{0.602250pt}%
\definecolor{currentstroke}{rgb}{0.000000,0.000000,0.000000}%
\pgfsetstrokecolor{currentstroke}%
\pgfsetdash{}{0pt}%
\pgfsys@defobject{currentmarker}{\pgfqpoint{-0.027778in}{0.000000in}}{\pgfqpoint{0.000000in}{0.000000in}}{%
\pgfpathmoveto{\pgfqpoint{0.000000in}{0.000000in}}%
\pgfpathlineto{\pgfqpoint{-0.027778in}{0.000000in}}%
\pgfusepath{stroke,fill}%
}%
\begin{pgfscope}%
\pgfsys@transformshift{0.571156in}{2.742630in}%
\pgfsys@useobject{currentmarker}{}%
\end{pgfscope}%
\end{pgfscope}%
\begin{pgfscope}%
\pgfsetbuttcap%
\pgfsetroundjoin%
\definecolor{currentfill}{rgb}{0.000000,0.000000,0.000000}%
\pgfsetfillcolor{currentfill}%
\pgfsetlinewidth{0.602250pt}%
\definecolor{currentstroke}{rgb}{0.000000,0.000000,0.000000}%
\pgfsetstrokecolor{currentstroke}%
\pgfsetdash{}{0pt}%
\pgfsys@defobject{currentmarker}{\pgfqpoint{-0.027778in}{0.000000in}}{\pgfqpoint{0.000000in}{0.000000in}}{%
\pgfpathmoveto{\pgfqpoint{0.000000in}{0.000000in}}%
\pgfpathlineto{\pgfqpoint{-0.027778in}{0.000000in}}%
\pgfusepath{stroke,fill}%
}%
\begin{pgfscope}%
\pgfsys@transformshift{0.571156in}{2.770095in}%
\pgfsys@useobject{currentmarker}{}%
\end{pgfscope}%
\end{pgfscope}%
\begin{pgfscope}%
\pgfsetbuttcap%
\pgfsetroundjoin%
\definecolor{currentfill}{rgb}{0.000000,0.000000,0.000000}%
\pgfsetfillcolor{currentfill}%
\pgfsetlinewidth{0.602250pt}%
\definecolor{currentstroke}{rgb}{0.000000,0.000000,0.000000}%
\pgfsetstrokecolor{currentstroke}%
\pgfsetdash{}{0pt}%
\pgfsys@defobject{currentmarker}{\pgfqpoint{-0.027778in}{0.000000in}}{\pgfqpoint{0.000000in}{0.000000in}}{%
\pgfpathmoveto{\pgfqpoint{0.000000in}{0.000000in}}%
\pgfpathlineto{\pgfqpoint{-0.027778in}{0.000000in}}%
\pgfusepath{stroke,fill}%
}%
\begin{pgfscope}%
\pgfsys@transformshift{0.571156in}{2.793316in}%
\pgfsys@useobject{currentmarker}{}%
\end{pgfscope}%
\end{pgfscope}%
\begin{pgfscope}%
\pgfsetbuttcap%
\pgfsetroundjoin%
\definecolor{currentfill}{rgb}{0.000000,0.000000,0.000000}%
\pgfsetfillcolor{currentfill}%
\pgfsetlinewidth{0.602250pt}%
\definecolor{currentstroke}{rgb}{0.000000,0.000000,0.000000}%
\pgfsetstrokecolor{currentstroke}%
\pgfsetdash{}{0pt}%
\pgfsys@defobject{currentmarker}{\pgfqpoint{-0.027778in}{0.000000in}}{\pgfqpoint{0.000000in}{0.000000in}}{%
\pgfpathmoveto{\pgfqpoint{0.000000in}{0.000000in}}%
\pgfpathlineto{\pgfqpoint{-0.027778in}{0.000000in}}%
\pgfusepath{stroke,fill}%
}%
\begin{pgfscope}%
\pgfsys@transformshift{0.571156in}{2.813432in}%
\pgfsys@useobject{currentmarker}{}%
\end{pgfscope}%
\end{pgfscope}%
\begin{pgfscope}%
\pgfsetbuttcap%
\pgfsetroundjoin%
\definecolor{currentfill}{rgb}{0.000000,0.000000,0.000000}%
\pgfsetfillcolor{currentfill}%
\pgfsetlinewidth{0.602250pt}%
\definecolor{currentstroke}{rgb}{0.000000,0.000000,0.000000}%
\pgfsetstrokecolor{currentstroke}%
\pgfsetdash{}{0pt}%
\pgfsys@defobject{currentmarker}{\pgfqpoint{-0.027778in}{0.000000in}}{\pgfqpoint{0.000000in}{0.000000in}}{%
\pgfpathmoveto{\pgfqpoint{0.000000in}{0.000000in}}%
\pgfpathlineto{\pgfqpoint{-0.027778in}{0.000000in}}%
\pgfusepath{stroke,fill}%
}%
\begin{pgfscope}%
\pgfsys@transformshift{0.571156in}{2.831175in}%
\pgfsys@useobject{currentmarker}{}%
\end{pgfscope}%
\end{pgfscope}%
\begin{pgfscope}%
\pgfsetbuttcap%
\pgfsetroundjoin%
\definecolor{currentfill}{rgb}{0.000000,0.000000,0.000000}%
\pgfsetfillcolor{currentfill}%
\pgfsetlinewidth{0.602250pt}%
\definecolor{currentstroke}{rgb}{0.000000,0.000000,0.000000}%
\pgfsetstrokecolor{currentstroke}%
\pgfsetdash{}{0pt}%
\pgfsys@defobject{currentmarker}{\pgfqpoint{-0.027778in}{0.000000in}}{\pgfqpoint{0.000000in}{0.000000in}}{%
\pgfpathmoveto{\pgfqpoint{0.000000in}{0.000000in}}%
\pgfpathlineto{\pgfqpoint{-0.027778in}{0.000000in}}%
\pgfusepath{stroke,fill}%
}%
\begin{pgfscope}%
\pgfsys@transformshift{0.571156in}{2.951463in}%
\pgfsys@useobject{currentmarker}{}%
\end{pgfscope}%
\end{pgfscope}%
\begin{pgfscope}%
\pgfsetbuttcap%
\pgfsetroundjoin%
\definecolor{currentfill}{rgb}{0.000000,0.000000,0.000000}%
\pgfsetfillcolor{currentfill}%
\pgfsetlinewidth{0.602250pt}%
\definecolor{currentstroke}{rgb}{0.000000,0.000000,0.000000}%
\pgfsetstrokecolor{currentstroke}%
\pgfsetdash{}{0pt}%
\pgfsys@defobject{currentmarker}{\pgfqpoint{-0.027778in}{0.000000in}}{\pgfqpoint{0.000000in}{0.000000in}}{%
\pgfpathmoveto{\pgfqpoint{0.000000in}{0.000000in}}%
\pgfpathlineto{\pgfqpoint{-0.027778in}{0.000000in}}%
\pgfusepath{stroke,fill}%
}%
\begin{pgfscope}%
\pgfsys@transformshift{0.571156in}{3.012543in}%
\pgfsys@useobject{currentmarker}{}%
\end{pgfscope}%
\end{pgfscope}%
\begin{pgfscope}%
\pgftext[x=0.130375in,y=1.725694in,,bottom,rotate=90.000000]{\rmfamily\fontsize{10.000000}{12.000000}\selectfont E(t)}%
\end{pgfscope}%
\begin{pgfscope}%
\pgfpathrectangle{\pgfqpoint{0.571156in}{0.402778in}}{\pgfqpoint{4.380233in}{2.645833in}}%
\pgfusepath{clip}%
\pgfsetrectcap%
\pgfsetroundjoin%
\pgfsetlinewidth{1.505625pt}%
\definecolor{currentstroke}{rgb}{0.121569,0.466667,0.705882}%
\pgfsetstrokecolor{currentstroke}%
\pgfsetdash{}{0pt}%
\pgfpathmoveto{\pgfqpoint{0.774135in}{0.388889in}}%
\pgfpathlineto{\pgfqpoint{0.774146in}{1.403008in}}%
\pgfpathlineto{\pgfqpoint{0.778035in}{1.446338in}}%
\pgfpathlineto{\pgfqpoint{0.781923in}{1.449970in}}%
\pgfpathlineto{\pgfqpoint{0.785812in}{1.447297in}}%
\pgfpathlineto{\pgfqpoint{0.789701in}{1.447305in}}%
\pgfpathlineto{\pgfqpoint{0.797478in}{1.450373in}}%
\pgfpathlineto{\pgfqpoint{0.816922in}{1.450283in}}%
\pgfpathlineto{\pgfqpoint{1.003579in}{1.449987in}}%
\pgfpathlineto{\pgfqpoint{1.205792in}{1.449361in}}%
\pgfpathlineto{\pgfqpoint{1.209680in}{2.790069in}}%
\pgfpathlineto{\pgfqpoint{1.392449in}{2.788391in}}%
\pgfpathlineto{\pgfqpoint{1.645215in}{2.783869in}}%
\pgfpathlineto{\pgfqpoint{1.649104in}{2.913010in}}%
\pgfpathlineto{\pgfqpoint{1.785208in}{2.906577in}}%
\pgfpathlineto{\pgfqpoint{1.921313in}{2.897349in}}%
\pgfpathlineto{\pgfqpoint{2.041863in}{2.886124in}}%
\pgfpathlineto{\pgfqpoint{2.088527in}{2.880855in}}%
\pgfpathlineto{\pgfqpoint{2.092416in}{2.928346in}}%
\pgfpathlineto{\pgfqpoint{2.189633in}{2.912810in}}%
\pgfpathlineto{\pgfqpoint{2.286851in}{2.894213in}}%
\pgfpathlineto{\pgfqpoint{2.384068in}{2.872418in}}%
\pgfpathlineto{\pgfqpoint{2.481286in}{2.847414in}}%
\pgfpathlineto{\pgfqpoint{2.617390in}{2.808669in}}%
\pgfpathlineto{\pgfqpoint{2.726274in}{2.772531in}}%
\pgfpathlineto{\pgfqpoint{2.846824in}{2.729119in}}%
\pgfpathlineto{\pgfqpoint{2.975151in}{2.679721in}}%
\pgfpathlineto{\pgfqpoint{2.979040in}{2.667579in}}%
\pgfpathlineto{\pgfqpoint{3.138476in}{2.603558in}}%
\pgfpathlineto{\pgfqpoint{3.321245in}{2.526611in}}%
\pgfpathlineto{\pgfqpoint{3.418463in}{2.484617in}}%
\pgfpathlineto{\pgfqpoint{3.422352in}{2.467377in}}%
\pgfpathlineto{\pgfqpoint{3.717893in}{2.337643in}}%
\pgfpathlineto{\pgfqpoint{3.861775in}{2.273433in}}%
\pgfpathlineto{\pgfqpoint{3.865663in}{2.249882in}}%
\pgfpathlineto{\pgfqpoint{4.305087in}{2.052200in}}%
\pgfpathlineto{\pgfqpoint{4.308975in}{2.018293in}}%
\pgfpathlineto{\pgfqpoint{4.752287in}{1.817617in}}%
\pgfpathlineto{\pgfqpoint{4.752287in}{1.817617in}}%
\pgfusepath{stroke}%
\end{pgfscope}%
\begin{pgfscope}%
\pgfpathrectangle{\pgfqpoint{0.571156in}{0.402778in}}{\pgfqpoint{4.380233in}{2.645833in}}%
\pgfusepath{clip}%
\pgfsetrectcap%
\pgfsetroundjoin%
\pgfsetlinewidth{1.505625pt}%
\definecolor{currentstroke}{rgb}{1.000000,0.498039,0.054902}%
\pgfsetstrokecolor{currentstroke}%
\pgfsetdash{}{0pt}%
\pgfpathmoveto{\pgfqpoint{0.774135in}{0.388889in}}%
\pgfpathlineto{\pgfqpoint{0.774146in}{1.403008in}}%
\pgfpathlineto{\pgfqpoint{0.778035in}{1.446338in}}%
\pgfpathlineto{\pgfqpoint{0.781923in}{1.449970in}}%
\pgfpathlineto{\pgfqpoint{0.785812in}{1.447297in}}%
\pgfpathlineto{\pgfqpoint{0.789701in}{1.447305in}}%
\pgfpathlineto{\pgfqpoint{0.797478in}{1.450373in}}%
\pgfpathlineto{\pgfqpoint{0.816922in}{1.450283in}}%
\pgfpathlineto{\pgfqpoint{1.003579in}{1.449987in}}%
\pgfpathlineto{\pgfqpoint{1.209680in}{1.449345in}}%
\pgfpathlineto{\pgfqpoint{1.213569in}{1.531046in}}%
\pgfpathlineto{\pgfqpoint{1.217458in}{1.550705in}}%
\pgfpathlineto{\pgfqpoint{1.221347in}{1.552476in}}%
\pgfpathlineto{\pgfqpoint{1.229124in}{1.551130in}}%
\pgfpathlineto{\pgfqpoint{1.240790in}{1.552740in}}%
\pgfpathlineto{\pgfqpoint{1.306898in}{1.552243in}}%
\pgfpathlineto{\pgfqpoint{1.645215in}{1.548987in}}%
\pgfpathlineto{\pgfqpoint{1.649104in}{2.660614in}}%
\pgfpathlineto{\pgfqpoint{2.053529in}{2.652194in}}%
\pgfpathlineto{\pgfqpoint{2.088527in}{2.650911in}}%
\pgfpathlineto{\pgfqpoint{2.092416in}{2.705103in}}%
\pgfpathlineto{\pgfqpoint{2.294628in}{2.696277in}}%
\pgfpathlineto{\pgfqpoint{2.430733in}{2.687056in}}%
\pgfpathlineto{\pgfqpoint{2.531839in}{2.677873in}}%
\pgfpathlineto{\pgfqpoint{2.535728in}{2.576433in}}%
\pgfpathlineto{\pgfqpoint{2.652389in}{2.559583in}}%
\pgfpathlineto{\pgfqpoint{2.749606in}{2.542325in}}%
\pgfpathlineto{\pgfqpoint{2.846824in}{2.521916in}}%
\pgfpathlineto{\pgfqpoint{2.944041in}{2.498285in}}%
\pgfpathlineto{\pgfqpoint{2.975151in}{2.490050in}}%
\pgfpathlineto{\pgfqpoint{2.979040in}{2.413497in}}%
\pgfpathlineto{\pgfqpoint{3.068480in}{2.387034in}}%
\pgfpathlineto{\pgfqpoint{3.173475in}{2.352939in}}%
\pgfpathlineto{\pgfqpoint{3.290136in}{2.311734in}}%
\pgfpathlineto{\pgfqpoint{3.418463in}{2.263086in}}%
\pgfpathlineto{\pgfqpoint{3.422352in}{2.311339in}}%
\pgfpathlineto{\pgfqpoint{3.519569in}{2.272913in}}%
\pgfpathlineto{\pgfqpoint{3.686783in}{2.203862in}}%
\pgfpathlineto{\pgfqpoint{3.861775in}{2.128877in}}%
\pgfpathlineto{\pgfqpoint{3.865663in}{2.199637in}}%
\pgfpathlineto{\pgfqpoint{4.005657in}{2.138626in}}%
\pgfpathlineto{\pgfqpoint{4.305087in}{2.005574in}}%
\pgfpathlineto{\pgfqpoint{4.308975in}{2.064159in}}%
\pgfpathlineto{\pgfqpoint{4.511188in}{1.973420in}}%
\pgfpathlineto{\pgfqpoint{4.752287in}{1.864602in}}%
\pgfpathlineto{\pgfqpoint{4.752287in}{1.864602in}}%
\pgfusepath{stroke}%
\end{pgfscope}%
\begin{pgfscope}%
\pgfpathrectangle{\pgfqpoint{0.571156in}{0.402778in}}{\pgfqpoint{4.380233in}{2.645833in}}%
\pgfusepath{clip}%
\pgfsetrectcap%
\pgfsetroundjoin%
\pgfsetlinewidth{1.505625pt}%
\definecolor{currentstroke}{rgb}{0.172549,0.627451,0.172549}%
\pgfsetstrokecolor{currentstroke}%
\pgfsetdash{}{0pt}%
\pgfpathmoveto{\pgfqpoint{0.774135in}{0.388889in}}%
\pgfpathlineto{\pgfqpoint{0.774146in}{1.403008in}}%
\pgfpathlineto{\pgfqpoint{0.778035in}{1.446338in}}%
\pgfpathlineto{\pgfqpoint{0.781923in}{1.449970in}}%
\pgfpathlineto{\pgfqpoint{0.785812in}{1.447297in}}%
\pgfpathlineto{\pgfqpoint{0.789701in}{1.447305in}}%
\pgfpathlineto{\pgfqpoint{0.797478in}{1.450373in}}%
\pgfpathlineto{\pgfqpoint{0.816922in}{1.450283in}}%
\pgfpathlineto{\pgfqpoint{1.003579in}{1.449987in}}%
\pgfpathlineto{\pgfqpoint{1.209680in}{1.449345in}}%
\pgfpathlineto{\pgfqpoint{1.213569in}{1.531046in}}%
\pgfpathlineto{\pgfqpoint{1.217458in}{1.550705in}}%
\pgfpathlineto{\pgfqpoint{1.221347in}{1.552476in}}%
\pgfpathlineto{\pgfqpoint{1.229124in}{1.551130in}}%
\pgfpathlineto{\pgfqpoint{1.240790in}{1.552740in}}%
\pgfpathlineto{\pgfqpoint{1.306898in}{1.552243in}}%
\pgfpathlineto{\pgfqpoint{1.649104in}{1.548927in}}%
\pgfpathlineto{\pgfqpoint{1.652992in}{1.593426in}}%
\pgfpathlineto{\pgfqpoint{1.656881in}{1.605690in}}%
\pgfpathlineto{\pgfqpoint{1.660770in}{1.606769in}}%
\pgfpathlineto{\pgfqpoint{1.668547in}{1.605798in}}%
\pgfpathlineto{\pgfqpoint{1.684102in}{1.606553in}}%
\pgfpathlineto{\pgfqpoint{1.851316in}{1.602857in}}%
\pgfpathlineto{\pgfqpoint{2.034085in}{1.596301in}}%
\pgfpathlineto{\pgfqpoint{2.088527in}{1.593624in}}%
\pgfpathlineto{\pgfqpoint{2.092416in}{2.431372in}}%
\pgfpathlineto{\pgfqpoint{2.142969in}{2.428460in}}%
\pgfpathlineto{\pgfqpoint{2.271296in}{2.418640in}}%
\pgfpathlineto{\pgfqpoint{2.387957in}{2.406643in}}%
\pgfpathlineto{\pgfqpoint{2.492952in}{2.392818in}}%
\pgfpathlineto{\pgfqpoint{2.531839in}{2.386881in}}%
\pgfpathlineto{\pgfqpoint{2.535728in}{2.246630in}}%
\pgfpathlineto{\pgfqpoint{2.582392in}{2.239190in}}%
\pgfpathlineto{\pgfqpoint{2.679610in}{2.221096in}}%
\pgfpathlineto{\pgfqpoint{2.776827in}{2.199819in}}%
\pgfpathlineto{\pgfqpoint{2.874045in}{2.175325in}}%
\pgfpathlineto{\pgfqpoint{2.975151in}{2.146550in}}%
\pgfpathlineto{\pgfqpoint{2.979040in}{2.057651in}}%
\pgfpathlineto{\pgfqpoint{3.010149in}{2.048543in}}%
\pgfpathlineto{\pgfqpoint{3.115144in}{2.014745in}}%
\pgfpathlineto{\pgfqpoint{3.231805in}{1.973820in}}%
\pgfpathlineto{\pgfqpoint{3.360132in}{1.925418in}}%
\pgfpathlineto{\pgfqpoint{3.418463in}{1.902456in}}%
\pgfpathlineto{\pgfqpoint{3.422352in}{1.868774in}}%
\pgfpathlineto{\pgfqpoint{3.434018in}{1.864962in}}%
\pgfpathlineto{\pgfqpoint{3.640119in}{1.779835in}}%
\pgfpathlineto{\pgfqpoint{3.857886in}{1.685904in}}%
\pgfpathlineto{\pgfqpoint{3.861775in}{1.684201in}}%
\pgfpathlineto{\pgfqpoint{3.865663in}{1.766294in}}%
\pgfpathlineto{\pgfqpoint{3.885107in}{1.757190in}}%
\pgfpathlineto{\pgfqpoint{4.301198in}{1.571773in}}%
\pgfpathlineto{\pgfqpoint{4.305087in}{1.570024in}}%
\pgfpathlineto{\pgfqpoint{4.308975in}{1.806819in}}%
\pgfpathlineto{\pgfqpoint{4.394527in}{1.768153in}}%
\pgfpathlineto{\pgfqpoint{4.752287in}{1.606386in}}%
\pgfpathlineto{\pgfqpoint{4.752287in}{1.606386in}}%
\pgfusepath{stroke}%
\end{pgfscope}%
\begin{pgfscope}%
\pgfpathrectangle{\pgfqpoint{0.571156in}{0.402778in}}{\pgfqpoint{4.380233in}{2.645833in}}%
\pgfusepath{clip}%
\pgfsetrectcap%
\pgfsetroundjoin%
\pgfsetlinewidth{1.505625pt}%
\definecolor{currentstroke}{rgb}{0.839216,0.152941,0.156863}%
\pgfsetstrokecolor{currentstroke}%
\pgfsetdash{}{0pt}%
\pgfpathmoveto{\pgfqpoint{0.774135in}{0.388889in}}%
\pgfpathlineto{\pgfqpoint{0.774146in}{1.403008in}}%
\pgfpathlineto{\pgfqpoint{0.778035in}{1.446338in}}%
\pgfpathlineto{\pgfqpoint{0.781923in}{1.449970in}}%
\pgfpathlineto{\pgfqpoint{0.785812in}{1.447297in}}%
\pgfpathlineto{\pgfqpoint{0.789701in}{1.447305in}}%
\pgfpathlineto{\pgfqpoint{0.797478in}{1.450373in}}%
\pgfpathlineto{\pgfqpoint{0.816922in}{1.450283in}}%
\pgfpathlineto{\pgfqpoint{1.003579in}{1.449987in}}%
\pgfpathlineto{\pgfqpoint{1.209680in}{1.449345in}}%
\pgfpathlineto{\pgfqpoint{1.213569in}{1.531046in}}%
\pgfpathlineto{\pgfqpoint{1.217458in}{1.550705in}}%
\pgfpathlineto{\pgfqpoint{1.221347in}{1.552476in}}%
\pgfpathlineto{\pgfqpoint{1.229124in}{1.551130in}}%
\pgfpathlineto{\pgfqpoint{1.240790in}{1.552740in}}%
\pgfpathlineto{\pgfqpoint{1.306898in}{1.552243in}}%
\pgfpathlineto{\pgfqpoint{1.649104in}{1.548927in}}%
\pgfpathlineto{\pgfqpoint{1.652992in}{1.593426in}}%
\pgfpathlineto{\pgfqpoint{1.656881in}{1.605690in}}%
\pgfpathlineto{\pgfqpoint{1.660770in}{1.606769in}}%
\pgfpathlineto{\pgfqpoint{1.668547in}{1.605798in}}%
\pgfpathlineto{\pgfqpoint{1.684102in}{1.606553in}}%
\pgfpathlineto{\pgfqpoint{1.851316in}{1.602857in}}%
\pgfpathlineto{\pgfqpoint{2.034085in}{1.596301in}}%
\pgfpathlineto{\pgfqpoint{2.092416in}{1.593418in}}%
\pgfpathlineto{\pgfqpoint{2.096304in}{1.619589in}}%
\pgfpathlineto{\pgfqpoint{2.100193in}{1.627202in}}%
\pgfpathlineto{\pgfqpoint{2.107970in}{1.626928in}}%
\pgfpathlineto{\pgfqpoint{2.255741in}{1.617561in}}%
\pgfpathlineto{\pgfqpoint{2.376291in}{1.606189in}}%
\pgfpathlineto{\pgfqpoint{2.485175in}{1.592834in}}%
\pgfpathlineto{\pgfqpoint{2.531839in}{1.586088in}}%
\pgfpathlineto{\pgfqpoint{2.535728in}{2.068901in}}%
\pgfpathlineto{\pgfqpoint{2.555171in}{2.066370in}}%
\pgfpathlineto{\pgfqpoint{2.648500in}{2.050097in}}%
\pgfpathlineto{\pgfqpoint{2.745718in}{2.030059in}}%
\pgfpathlineto{\pgfqpoint{2.842935in}{2.006802in}}%
\pgfpathlineto{\pgfqpoint{2.940153in}{1.980375in}}%
\pgfpathlineto{\pgfqpoint{2.975151in}{1.970115in}}%
\pgfpathlineto{\pgfqpoint{2.979040in}{2.008315in}}%
\pgfpathlineto{\pgfqpoint{3.006260in}{1.999622in}}%
\pgfpathlineto{\pgfqpoint{3.115144in}{1.964363in}}%
\pgfpathlineto{\pgfqpoint{3.231805in}{1.923192in}}%
\pgfpathlineto{\pgfqpoint{3.360132in}{1.874574in}}%
\pgfpathlineto{\pgfqpoint{3.418463in}{1.851531in}}%
\pgfpathlineto{\pgfqpoint{3.422352in}{1.860717in}}%
\pgfpathlineto{\pgfqpoint{3.437906in}{1.853603in}}%
\pgfpathlineto{\pgfqpoint{3.605120in}{1.784625in}}%
\pgfpathlineto{\pgfqpoint{3.815110in}{1.694478in}}%
\pgfpathlineto{\pgfqpoint{3.861775in}{1.674079in}}%
\pgfpathlineto{\pgfqpoint{3.865663in}{1.715186in}}%
\pgfpathlineto{\pgfqpoint{3.881218in}{1.707558in}}%
\pgfpathlineto{\pgfqpoint{4.211758in}{1.560613in}}%
\pgfpathlineto{\pgfqpoint{4.305087in}{1.518700in}}%
\pgfpathlineto{\pgfqpoint{4.308975in}{1.503699in}}%
\pgfpathlineto{\pgfqpoint{4.320642in}{1.497506in}}%
\pgfpathlineto{\pgfqpoint{4.596739in}{1.372921in}}%
\pgfpathlineto{\pgfqpoint{4.752287in}{1.302527in}}%
\pgfpathlineto{\pgfqpoint{4.752287in}{1.302527in}}%
\pgfusepath{stroke}%
\end{pgfscope}%
\begin{pgfscope}%
\pgfpathrectangle{\pgfqpoint{0.571156in}{0.402778in}}{\pgfqpoint{4.380233in}{2.645833in}}%
\pgfusepath{clip}%
\pgfsetrectcap%
\pgfsetroundjoin%
\pgfsetlinewidth{1.505625pt}%
\definecolor{currentstroke}{rgb}{0.580392,0.403922,0.741176}%
\pgfsetstrokecolor{currentstroke}%
\pgfsetdash{}{0pt}%
\pgfpathmoveto{\pgfqpoint{0.774135in}{0.388889in}}%
\pgfpathlineto{\pgfqpoint{0.774146in}{1.403008in}}%
\pgfpathlineto{\pgfqpoint{0.778035in}{1.446338in}}%
\pgfpathlineto{\pgfqpoint{0.781923in}{1.449970in}}%
\pgfpathlineto{\pgfqpoint{0.785812in}{1.447297in}}%
\pgfpathlineto{\pgfqpoint{0.789701in}{1.447305in}}%
\pgfpathlineto{\pgfqpoint{0.797478in}{1.450373in}}%
\pgfpathlineto{\pgfqpoint{0.816922in}{1.450283in}}%
\pgfpathlineto{\pgfqpoint{1.003579in}{1.449987in}}%
\pgfpathlineto{\pgfqpoint{1.209680in}{1.449345in}}%
\pgfpathlineto{\pgfqpoint{1.213569in}{1.531046in}}%
\pgfpathlineto{\pgfqpoint{1.217458in}{1.550705in}}%
\pgfpathlineto{\pgfqpoint{1.221347in}{1.552476in}}%
\pgfpathlineto{\pgfqpoint{1.229124in}{1.551130in}}%
\pgfpathlineto{\pgfqpoint{1.240790in}{1.552740in}}%
\pgfpathlineto{\pgfqpoint{1.306898in}{1.552243in}}%
\pgfpathlineto{\pgfqpoint{1.649104in}{1.548927in}}%
\pgfpathlineto{\pgfqpoint{1.652992in}{1.593426in}}%
\pgfpathlineto{\pgfqpoint{1.656881in}{1.605690in}}%
\pgfpathlineto{\pgfqpoint{1.660770in}{1.606769in}}%
\pgfpathlineto{\pgfqpoint{1.668547in}{1.605798in}}%
\pgfpathlineto{\pgfqpoint{1.684102in}{1.606553in}}%
\pgfpathlineto{\pgfqpoint{1.851316in}{1.602857in}}%
\pgfpathlineto{\pgfqpoint{2.034085in}{1.596301in}}%
\pgfpathlineto{\pgfqpoint{2.092416in}{1.593418in}}%
\pgfpathlineto{\pgfqpoint{2.096304in}{1.619589in}}%
\pgfpathlineto{\pgfqpoint{2.100193in}{1.627202in}}%
\pgfpathlineto{\pgfqpoint{2.107970in}{1.626928in}}%
\pgfpathlineto{\pgfqpoint{2.255741in}{1.617561in}}%
\pgfpathlineto{\pgfqpoint{2.376291in}{1.606189in}}%
\pgfpathlineto{\pgfqpoint{2.485175in}{1.592834in}}%
\pgfpathlineto{\pgfqpoint{2.535728in}{1.585497in}}%
\pgfpathlineto{\pgfqpoint{2.539616in}{1.594262in}}%
\pgfpathlineto{\pgfqpoint{2.543505in}{1.596592in}}%
\pgfpathlineto{\pgfqpoint{2.570726in}{1.592549in}}%
\pgfpathlineto{\pgfqpoint{2.679610in}{1.572808in}}%
\pgfpathlineto{\pgfqpoint{2.776827in}{1.551834in}}%
\pgfpathlineto{\pgfqpoint{2.874045in}{1.527640in}}%
\pgfpathlineto{\pgfqpoint{2.971262in}{1.500317in}}%
\pgfpathlineto{\pgfqpoint{2.975151in}{1.499161in}}%
\pgfpathlineto{\pgfqpoint{2.979040in}{1.704908in}}%
\pgfpathlineto{\pgfqpoint{2.986817in}{1.699568in}}%
\pgfpathlineto{\pgfqpoint{3.185141in}{1.634138in}}%
\pgfpathlineto{\pgfqpoint{3.309579in}{1.588271in}}%
\pgfpathlineto{\pgfqpoint{3.418463in}{1.545766in}}%
\pgfpathlineto{\pgfqpoint{3.422352in}{1.508567in}}%
\pgfpathlineto{\pgfqpoint{3.426240in}{1.500068in}}%
\pgfpathlineto{\pgfqpoint{3.434018in}{1.494342in}}%
\pgfpathlineto{\pgfqpoint{3.465127in}{1.481755in}}%
\pgfpathlineto{\pgfqpoint{3.644008in}{1.407413in}}%
\pgfpathlineto{\pgfqpoint{3.861775in}{1.313364in}}%
\pgfpathlineto{\pgfqpoint{3.865663in}{1.231274in}}%
\pgfpathlineto{\pgfqpoint{3.869552in}{1.243942in}}%
\pgfpathlineto{\pgfqpoint{3.873441in}{1.246743in}}%
\pgfpathlineto{\pgfqpoint{3.892884in}{1.238686in}}%
\pgfpathlineto{\pgfqpoint{4.032878in}{1.176789in}}%
\pgfpathlineto{\pgfqpoint{4.305087in}{1.055016in}}%
\pgfpathlineto{\pgfqpoint{4.308975in}{1.315650in}}%
\pgfpathlineto{\pgfqpoint{4.312864in}{1.316338in}}%
\pgfpathlineto{\pgfqpoint{4.320642in}{1.313711in}}%
\pgfpathlineto{\pgfqpoint{4.421748in}{1.268187in}}%
\pgfpathlineto{\pgfqpoint{4.752287in}{1.118772in}}%
\pgfpathlineto{\pgfqpoint{4.752287in}{1.118772in}}%
\pgfusepath{stroke}%
\end{pgfscope}%
\begin{pgfscope}%
\pgfpathrectangle{\pgfqpoint{0.571156in}{0.402778in}}{\pgfqpoint{4.380233in}{2.645833in}}%
\pgfusepath{clip}%
\pgfsetrectcap%
\pgfsetroundjoin%
\pgfsetlinewidth{1.505625pt}%
\definecolor{currentstroke}{rgb}{0.549020,0.337255,0.294118}%
\pgfsetstrokecolor{currentstroke}%
\pgfsetdash{}{0pt}%
\pgfpathmoveto{\pgfqpoint{0.774135in}{0.388889in}}%
\pgfpathlineto{\pgfqpoint{0.774146in}{1.403008in}}%
\pgfpathlineto{\pgfqpoint{0.778035in}{1.446338in}}%
\pgfpathlineto{\pgfqpoint{0.781923in}{1.449970in}}%
\pgfpathlineto{\pgfqpoint{0.785812in}{1.447297in}}%
\pgfpathlineto{\pgfqpoint{0.789701in}{1.447305in}}%
\pgfpathlineto{\pgfqpoint{0.797478in}{1.450373in}}%
\pgfpathlineto{\pgfqpoint{0.816922in}{1.450283in}}%
\pgfpathlineto{\pgfqpoint{1.003579in}{1.449987in}}%
\pgfpathlineto{\pgfqpoint{1.209680in}{1.449345in}}%
\pgfpathlineto{\pgfqpoint{1.213569in}{1.531046in}}%
\pgfpathlineto{\pgfqpoint{1.217458in}{1.550705in}}%
\pgfpathlineto{\pgfqpoint{1.221347in}{1.552476in}}%
\pgfpathlineto{\pgfqpoint{1.229124in}{1.551130in}}%
\pgfpathlineto{\pgfqpoint{1.240790in}{1.552740in}}%
\pgfpathlineto{\pgfqpoint{1.306898in}{1.552243in}}%
\pgfpathlineto{\pgfqpoint{1.649104in}{1.548927in}}%
\pgfpathlineto{\pgfqpoint{1.652992in}{1.593426in}}%
\pgfpathlineto{\pgfqpoint{1.656881in}{1.605690in}}%
\pgfpathlineto{\pgfqpoint{1.660770in}{1.606769in}}%
\pgfpathlineto{\pgfqpoint{1.668547in}{1.605798in}}%
\pgfpathlineto{\pgfqpoint{1.684102in}{1.606553in}}%
\pgfpathlineto{\pgfqpoint{1.851316in}{1.602857in}}%
\pgfpathlineto{\pgfqpoint{2.034085in}{1.596301in}}%
\pgfpathlineto{\pgfqpoint{2.092416in}{1.593418in}}%
\pgfpathlineto{\pgfqpoint{2.096304in}{1.619589in}}%
\pgfpathlineto{\pgfqpoint{2.100193in}{1.627202in}}%
\pgfpathlineto{\pgfqpoint{2.107970in}{1.626928in}}%
\pgfpathlineto{\pgfqpoint{2.255741in}{1.617561in}}%
\pgfpathlineto{\pgfqpoint{2.376291in}{1.606189in}}%
\pgfpathlineto{\pgfqpoint{2.485175in}{1.592834in}}%
\pgfpathlineto{\pgfqpoint{2.535728in}{1.585497in}}%
\pgfpathlineto{\pgfqpoint{2.539616in}{1.594262in}}%
\pgfpathlineto{\pgfqpoint{2.543505in}{1.596592in}}%
\pgfpathlineto{\pgfqpoint{2.570726in}{1.592549in}}%
\pgfpathlineto{\pgfqpoint{2.679610in}{1.572808in}}%
\pgfpathlineto{\pgfqpoint{2.776827in}{1.551834in}}%
\pgfpathlineto{\pgfqpoint{2.874045in}{1.527640in}}%
\pgfpathlineto{\pgfqpoint{2.971262in}{1.500317in}}%
\pgfpathlineto{\pgfqpoint{2.979040in}{1.498001in}}%
\pgfpathlineto{\pgfqpoint{2.982928in}{1.487819in}}%
\pgfpathlineto{\pgfqpoint{2.986817in}{1.483451in}}%
\pgfpathlineto{\pgfqpoint{3.010149in}{1.475939in}}%
\pgfpathlineto{\pgfqpoint{3.099589in}{1.447240in}}%
\pgfpathlineto{\pgfqpoint{3.212362in}{1.408012in}}%
\pgfpathlineto{\pgfqpoint{3.340689in}{1.359963in}}%
\pgfpathlineto{\pgfqpoint{3.418463in}{1.329404in}}%
\pgfpathlineto{\pgfqpoint{3.422352in}{1.311998in}}%
\pgfpathlineto{\pgfqpoint{3.426240in}{1.283065in}}%
\pgfpathlineto{\pgfqpoint{3.430129in}{1.271107in}}%
\pgfpathlineto{\pgfqpoint{3.437906in}{1.267894in}}%
\pgfpathlineto{\pgfqpoint{3.659562in}{1.174910in}}%
\pgfpathlineto{\pgfqpoint{3.861775in}{1.087474in}}%
\pgfpathlineto{\pgfqpoint{3.873441in}{0.730325in}}%
\pgfpathlineto{\pgfqpoint{3.877330in}{0.716532in}}%
\pgfpathlineto{\pgfqpoint{3.881218in}{0.726287in}}%
\pgfpathlineto{\pgfqpoint{3.885107in}{0.723950in}}%
\pgfpathlineto{\pgfqpoint{3.892884in}{0.706228in}}%
\pgfpathlineto{\pgfqpoint{3.896773in}{0.703268in}}%
\pgfpathlineto{\pgfqpoint{3.908439in}{0.700217in}}%
\pgfpathlineto{\pgfqpoint{3.939549in}{0.686133in}}%
\pgfpathlineto{\pgfqpoint{4.293421in}{0.528287in}}%
\pgfpathlineto{\pgfqpoint{4.305087in}{0.523043in}}%
\pgfpathlineto{\pgfqpoint{4.308975in}{0.839787in}}%
\pgfpathlineto{\pgfqpoint{4.312864in}{0.765457in}}%
\pgfpathlineto{\pgfqpoint{4.316753in}{0.728883in}}%
\pgfpathlineto{\pgfqpoint{4.320642in}{0.723800in}}%
\pgfpathlineto{\pgfqpoint{4.324530in}{0.725249in}}%
\pgfpathlineto{\pgfqpoint{4.328419in}{0.723309in}}%
\pgfpathlineto{\pgfqpoint{4.336196in}{0.715847in}}%
\pgfpathlineto{\pgfqpoint{4.351751in}{0.709071in}}%
\pgfpathlineto{\pgfqpoint{4.456746in}{0.661645in}}%
\pgfpathlineto{\pgfqpoint{4.752287in}{0.528020in}}%
\pgfpathlineto{\pgfqpoint{4.752287in}{0.528020in}}%
\pgfusepath{stroke}%
\end{pgfscope}%
\begin{pgfscope}%
\pgfpathrectangle{\pgfqpoint{0.571156in}{0.402778in}}{\pgfqpoint{4.380233in}{2.645833in}}%
\pgfusepath{clip}%
\pgfsetrectcap%
\pgfsetroundjoin%
\pgfsetlinewidth{1.505625pt}%
\definecolor{currentstroke}{rgb}{0.890196,0.466667,0.760784}%
\pgfsetstrokecolor{currentstroke}%
\pgfsetdash{}{0pt}%
\pgfpathmoveto{\pgfqpoint{0.774135in}{0.388889in}}%
\pgfpathlineto{\pgfqpoint{0.774146in}{1.403008in}}%
\pgfpathlineto{\pgfqpoint{0.778035in}{1.446338in}}%
\pgfpathlineto{\pgfqpoint{0.781923in}{1.449970in}}%
\pgfpathlineto{\pgfqpoint{0.785812in}{1.447297in}}%
\pgfpathlineto{\pgfqpoint{0.789701in}{1.447305in}}%
\pgfpathlineto{\pgfqpoint{0.797478in}{1.450373in}}%
\pgfpathlineto{\pgfqpoint{0.816922in}{1.450283in}}%
\pgfpathlineto{\pgfqpoint{1.003579in}{1.449987in}}%
\pgfpathlineto{\pgfqpoint{1.209680in}{1.449345in}}%
\pgfpathlineto{\pgfqpoint{1.213569in}{1.531046in}}%
\pgfpathlineto{\pgfqpoint{1.217458in}{1.550705in}}%
\pgfpathlineto{\pgfqpoint{1.221347in}{1.552476in}}%
\pgfpathlineto{\pgfqpoint{1.229124in}{1.551130in}}%
\pgfpathlineto{\pgfqpoint{1.240790in}{1.552740in}}%
\pgfpathlineto{\pgfqpoint{1.306898in}{1.552243in}}%
\pgfpathlineto{\pgfqpoint{1.649104in}{1.548927in}}%
\pgfpathlineto{\pgfqpoint{1.652992in}{1.593426in}}%
\pgfpathlineto{\pgfqpoint{1.656881in}{1.605690in}}%
\pgfpathlineto{\pgfqpoint{1.660770in}{1.606769in}}%
\pgfpathlineto{\pgfqpoint{1.668547in}{1.605798in}}%
\pgfpathlineto{\pgfqpoint{1.684102in}{1.606553in}}%
\pgfpathlineto{\pgfqpoint{1.851316in}{1.602857in}}%
\pgfpathlineto{\pgfqpoint{2.034085in}{1.596301in}}%
\pgfpathlineto{\pgfqpoint{2.092416in}{1.593418in}}%
\pgfpathlineto{\pgfqpoint{2.096304in}{1.619589in}}%
\pgfpathlineto{\pgfqpoint{2.100193in}{1.627202in}}%
\pgfpathlineto{\pgfqpoint{2.107970in}{1.626928in}}%
\pgfpathlineto{\pgfqpoint{2.255741in}{1.617561in}}%
\pgfpathlineto{\pgfqpoint{2.376291in}{1.606189in}}%
\pgfpathlineto{\pgfqpoint{2.485175in}{1.592834in}}%
\pgfpathlineto{\pgfqpoint{2.535728in}{1.585497in}}%
\pgfpathlineto{\pgfqpoint{2.539616in}{1.594262in}}%
\pgfpathlineto{\pgfqpoint{2.543505in}{1.596592in}}%
\pgfpathlineto{\pgfqpoint{2.570726in}{1.592549in}}%
\pgfpathlineto{\pgfqpoint{2.679610in}{1.572808in}}%
\pgfpathlineto{\pgfqpoint{2.776827in}{1.551834in}}%
\pgfpathlineto{\pgfqpoint{2.874045in}{1.527640in}}%
\pgfpathlineto{\pgfqpoint{2.971262in}{1.500317in}}%
\pgfpathlineto{\pgfqpoint{2.979040in}{1.498001in}}%
\pgfpathlineto{\pgfqpoint{2.982928in}{1.487819in}}%
\pgfpathlineto{\pgfqpoint{2.986817in}{1.483451in}}%
\pgfpathlineto{\pgfqpoint{3.010149in}{1.475939in}}%
\pgfpathlineto{\pgfqpoint{3.099589in}{1.447240in}}%
\pgfpathlineto{\pgfqpoint{3.212362in}{1.408012in}}%
\pgfpathlineto{\pgfqpoint{3.340689in}{1.359963in}}%
\pgfpathlineto{\pgfqpoint{3.422352in}{1.327852in}}%
\pgfpathlineto{\pgfqpoint{3.426240in}{1.301889in}}%
\pgfpathlineto{\pgfqpoint{3.430129in}{1.291191in}}%
\pgfpathlineto{\pgfqpoint{3.437906in}{1.287990in}}%
\pgfpathlineto{\pgfqpoint{3.647896in}{1.200097in}}%
\pgfpathlineto{\pgfqpoint{3.865663in}{1.108975in}}%
\pgfpathlineto{\pgfqpoint{3.869552in}{1.068822in}}%
\pgfpathlineto{\pgfqpoint{3.873441in}{1.051742in}}%
\pgfpathlineto{\pgfqpoint{3.877330in}{1.048654in}}%
\pgfpathlineto{\pgfqpoint{3.885107in}{1.046474in}}%
\pgfpathlineto{\pgfqpoint{3.896773in}{1.039601in}}%
\pgfpathlineto{\pgfqpoint{3.955104in}{1.014075in}}%
\pgfpathlineto{\pgfqpoint{4.305087in}{0.857837in}}%
\pgfpathlineto{\pgfqpoint{4.308975in}{0.881614in}}%
\pgfpathlineto{\pgfqpoint{4.312864in}{0.828351in}}%
\pgfpathlineto{\pgfqpoint{4.316753in}{0.804590in}}%
\pgfpathlineto{\pgfqpoint{4.320642in}{0.800833in}}%
\pgfpathlineto{\pgfqpoint{4.324530in}{0.801010in}}%
\pgfpathlineto{\pgfqpoint{4.332308in}{0.796010in}}%
\pgfpathlineto{\pgfqpoint{4.340085in}{0.791315in}}%
\pgfpathlineto{\pgfqpoint{4.382861in}{0.772325in}}%
\pgfpathlineto{\pgfqpoint{4.752287in}{0.605382in}}%
\pgfpathlineto{\pgfqpoint{4.752287in}{0.605382in}}%
\pgfusepath{stroke}%
\end{pgfscope}%
\begin{pgfscope}%
\pgfpathrectangle{\pgfqpoint{0.571156in}{0.402778in}}{\pgfqpoint{4.380233in}{2.645833in}}%
\pgfusepath{clip}%
\pgfsetrectcap%
\pgfsetroundjoin%
\pgfsetlinewidth{1.505625pt}%
\definecolor{currentstroke}{rgb}{0.498039,0.498039,0.498039}%
\pgfsetstrokecolor{currentstroke}%
\pgfsetdash{}{0pt}%
\pgfpathmoveto{\pgfqpoint{0.774135in}{0.388889in}}%
\pgfpathlineto{\pgfqpoint{0.774146in}{1.403008in}}%
\pgfpathlineto{\pgfqpoint{0.778035in}{1.446338in}}%
\pgfpathlineto{\pgfqpoint{0.781923in}{1.449970in}}%
\pgfpathlineto{\pgfqpoint{0.785812in}{1.447297in}}%
\pgfpathlineto{\pgfqpoint{0.789701in}{1.447305in}}%
\pgfpathlineto{\pgfqpoint{0.797478in}{1.450373in}}%
\pgfpathlineto{\pgfqpoint{0.816922in}{1.450283in}}%
\pgfpathlineto{\pgfqpoint{1.003579in}{1.449987in}}%
\pgfpathlineto{\pgfqpoint{1.209680in}{1.449345in}}%
\pgfpathlineto{\pgfqpoint{1.213569in}{1.531046in}}%
\pgfpathlineto{\pgfqpoint{1.217458in}{1.550705in}}%
\pgfpathlineto{\pgfqpoint{1.221347in}{1.552476in}}%
\pgfpathlineto{\pgfqpoint{1.229124in}{1.551130in}}%
\pgfpathlineto{\pgfqpoint{1.240790in}{1.552740in}}%
\pgfpathlineto{\pgfqpoint{1.306898in}{1.552243in}}%
\pgfpathlineto{\pgfqpoint{1.649104in}{1.548927in}}%
\pgfpathlineto{\pgfqpoint{1.652992in}{1.593426in}}%
\pgfpathlineto{\pgfqpoint{1.656881in}{1.605690in}}%
\pgfpathlineto{\pgfqpoint{1.660770in}{1.606769in}}%
\pgfpathlineto{\pgfqpoint{1.668547in}{1.605798in}}%
\pgfpathlineto{\pgfqpoint{1.684102in}{1.606553in}}%
\pgfpathlineto{\pgfqpoint{1.851316in}{1.602857in}}%
\pgfpathlineto{\pgfqpoint{2.034085in}{1.596301in}}%
\pgfpathlineto{\pgfqpoint{2.092416in}{1.593418in}}%
\pgfpathlineto{\pgfqpoint{2.096304in}{1.619589in}}%
\pgfpathlineto{\pgfqpoint{2.100193in}{1.627202in}}%
\pgfpathlineto{\pgfqpoint{2.107970in}{1.626928in}}%
\pgfpathlineto{\pgfqpoint{2.255741in}{1.617561in}}%
\pgfpathlineto{\pgfqpoint{2.376291in}{1.606189in}}%
\pgfpathlineto{\pgfqpoint{2.485175in}{1.592834in}}%
\pgfpathlineto{\pgfqpoint{2.535728in}{1.585497in}}%
\pgfpathlineto{\pgfqpoint{2.539616in}{1.594262in}}%
\pgfpathlineto{\pgfqpoint{2.543505in}{1.596592in}}%
\pgfpathlineto{\pgfqpoint{2.570726in}{1.592549in}}%
\pgfpathlineto{\pgfqpoint{2.679610in}{1.572808in}}%
\pgfpathlineto{\pgfqpoint{2.776827in}{1.551834in}}%
\pgfpathlineto{\pgfqpoint{2.874045in}{1.527640in}}%
\pgfpathlineto{\pgfqpoint{2.971262in}{1.500317in}}%
\pgfpathlineto{\pgfqpoint{2.979040in}{1.498001in}}%
\pgfpathlineto{\pgfqpoint{2.982928in}{1.487819in}}%
\pgfpathlineto{\pgfqpoint{2.986817in}{1.483451in}}%
\pgfpathlineto{\pgfqpoint{3.010149in}{1.475939in}}%
\pgfpathlineto{\pgfqpoint{3.099589in}{1.447240in}}%
\pgfpathlineto{\pgfqpoint{3.212362in}{1.408012in}}%
\pgfpathlineto{\pgfqpoint{3.340689in}{1.359963in}}%
\pgfpathlineto{\pgfqpoint{3.422352in}{1.327852in}}%
\pgfpathlineto{\pgfqpoint{3.426240in}{1.301889in}}%
\pgfpathlineto{\pgfqpoint{3.430129in}{1.291191in}}%
\pgfpathlineto{\pgfqpoint{3.437906in}{1.287990in}}%
\pgfpathlineto{\pgfqpoint{3.647896in}{1.200097in}}%
\pgfpathlineto{\pgfqpoint{3.865663in}{1.105998in}}%
\pgfpathlineto{\pgfqpoint{3.869552in}{1.064968in}}%
\pgfpathlineto{\pgfqpoint{3.873441in}{1.047467in}}%
\pgfpathlineto{\pgfqpoint{3.877330in}{1.044340in}}%
\pgfpathlineto{\pgfqpoint{3.885107in}{1.042195in}}%
\pgfpathlineto{\pgfqpoint{3.896773in}{1.035271in}}%
\pgfpathlineto{\pgfqpoint{3.951215in}{1.011469in}}%
\pgfpathlineto{\pgfqpoint{4.305087in}{0.853512in}}%
\pgfpathlineto{\pgfqpoint{4.308975in}{0.850852in}}%
\pgfpathlineto{\pgfqpoint{4.312864in}{0.782987in}}%
\pgfpathlineto{\pgfqpoint{4.316753in}{0.750657in}}%
\pgfpathlineto{\pgfqpoint{4.320642in}{0.746027in}}%
\pgfpathlineto{\pgfqpoint{4.324530in}{0.747041in}}%
\pgfpathlineto{\pgfqpoint{4.328419in}{0.745127in}}%
\pgfpathlineto{\pgfqpoint{4.336196in}{0.738206in}}%
\pgfpathlineto{\pgfqpoint{4.355640in}{0.729590in}}%
\pgfpathlineto{\pgfqpoint{4.534520in}{0.648870in}}%
\pgfpathlineto{\pgfqpoint{4.752287in}{0.550360in}}%
\pgfpathlineto{\pgfqpoint{4.752287in}{0.550360in}}%
\pgfusepath{stroke}%
\end{pgfscope}%
\begin{pgfscope}%
\pgfpathrectangle{\pgfqpoint{0.571156in}{0.402778in}}{\pgfqpoint{4.380233in}{2.645833in}}%
\pgfusepath{clip}%
\pgfsetrectcap%
\pgfsetroundjoin%
\pgfsetlinewidth{1.505625pt}%
\definecolor{currentstroke}{rgb}{0.737255,0.741176,0.133333}%
\pgfsetstrokecolor{currentstroke}%
\pgfsetdash{}{0pt}%
\pgfpathmoveto{\pgfqpoint{0.774135in}{0.388889in}}%
\pgfpathlineto{\pgfqpoint{0.774146in}{1.403008in}}%
\pgfpathlineto{\pgfqpoint{0.778035in}{1.446338in}}%
\pgfpathlineto{\pgfqpoint{0.781923in}{1.449970in}}%
\pgfpathlineto{\pgfqpoint{0.785812in}{1.447297in}}%
\pgfpathlineto{\pgfqpoint{0.789701in}{1.447305in}}%
\pgfpathlineto{\pgfqpoint{0.797478in}{1.450373in}}%
\pgfpathlineto{\pgfqpoint{0.816922in}{1.450283in}}%
\pgfpathlineto{\pgfqpoint{1.003579in}{1.449987in}}%
\pgfpathlineto{\pgfqpoint{1.209680in}{1.449345in}}%
\pgfpathlineto{\pgfqpoint{1.213569in}{1.531046in}}%
\pgfpathlineto{\pgfqpoint{1.217458in}{1.550705in}}%
\pgfpathlineto{\pgfqpoint{1.221347in}{1.552476in}}%
\pgfpathlineto{\pgfqpoint{1.229124in}{1.551130in}}%
\pgfpathlineto{\pgfqpoint{1.240790in}{1.552740in}}%
\pgfpathlineto{\pgfqpoint{1.306898in}{1.552243in}}%
\pgfpathlineto{\pgfqpoint{1.649104in}{1.548927in}}%
\pgfpathlineto{\pgfqpoint{1.652992in}{1.593426in}}%
\pgfpathlineto{\pgfqpoint{1.656881in}{1.605690in}}%
\pgfpathlineto{\pgfqpoint{1.660770in}{1.606769in}}%
\pgfpathlineto{\pgfqpoint{1.668547in}{1.605798in}}%
\pgfpathlineto{\pgfqpoint{1.684102in}{1.606553in}}%
\pgfpathlineto{\pgfqpoint{1.851316in}{1.602857in}}%
\pgfpathlineto{\pgfqpoint{2.034085in}{1.596301in}}%
\pgfpathlineto{\pgfqpoint{2.092416in}{1.593418in}}%
\pgfpathlineto{\pgfqpoint{2.096304in}{1.619589in}}%
\pgfpathlineto{\pgfqpoint{2.100193in}{1.627202in}}%
\pgfpathlineto{\pgfqpoint{2.107970in}{1.626928in}}%
\pgfpathlineto{\pgfqpoint{2.255741in}{1.617561in}}%
\pgfpathlineto{\pgfqpoint{2.376291in}{1.606189in}}%
\pgfpathlineto{\pgfqpoint{2.485175in}{1.592834in}}%
\pgfpathlineto{\pgfqpoint{2.535728in}{1.585497in}}%
\pgfpathlineto{\pgfqpoint{2.539616in}{1.594262in}}%
\pgfpathlineto{\pgfqpoint{2.543505in}{1.596592in}}%
\pgfpathlineto{\pgfqpoint{2.570726in}{1.592549in}}%
\pgfpathlineto{\pgfqpoint{2.679610in}{1.572808in}}%
\pgfpathlineto{\pgfqpoint{2.776827in}{1.551834in}}%
\pgfpathlineto{\pgfqpoint{2.874045in}{1.527640in}}%
\pgfpathlineto{\pgfqpoint{2.971262in}{1.500317in}}%
\pgfpathlineto{\pgfqpoint{2.979040in}{1.498001in}}%
\pgfpathlineto{\pgfqpoint{2.982928in}{1.487819in}}%
\pgfpathlineto{\pgfqpoint{2.986817in}{1.483451in}}%
\pgfpathlineto{\pgfqpoint{3.010149in}{1.475939in}}%
\pgfpathlineto{\pgfqpoint{3.099589in}{1.447240in}}%
\pgfpathlineto{\pgfqpoint{3.212362in}{1.408012in}}%
\pgfpathlineto{\pgfqpoint{3.340689in}{1.359963in}}%
\pgfpathlineto{\pgfqpoint{3.422352in}{1.327852in}}%
\pgfpathlineto{\pgfqpoint{3.426240in}{1.301889in}}%
\pgfpathlineto{\pgfqpoint{3.430129in}{1.291191in}}%
\pgfpathlineto{\pgfqpoint{3.437906in}{1.287990in}}%
\pgfpathlineto{\pgfqpoint{3.647896in}{1.200097in}}%
\pgfpathlineto{\pgfqpoint{3.865663in}{1.105998in}}%
\pgfpathlineto{\pgfqpoint{3.869552in}{1.064968in}}%
\pgfpathlineto{\pgfqpoint{3.873441in}{1.047467in}}%
\pgfpathlineto{\pgfqpoint{3.877330in}{1.044340in}}%
\pgfpathlineto{\pgfqpoint{3.885107in}{1.042195in}}%
\pgfpathlineto{\pgfqpoint{3.896773in}{1.035271in}}%
\pgfpathlineto{\pgfqpoint{3.951215in}{1.011469in}}%
\pgfpathlineto{\pgfqpoint{4.308975in}{0.851764in}}%
\pgfpathlineto{\pgfqpoint{4.312864in}{0.784398in}}%
\pgfpathlineto{\pgfqpoint{4.316753in}{0.752384in}}%
\pgfpathlineto{\pgfqpoint{4.320642in}{0.747787in}}%
\pgfpathlineto{\pgfqpoint{4.324530in}{0.748769in}}%
\pgfpathlineto{\pgfqpoint{4.328419in}{0.746857in}}%
\pgfpathlineto{\pgfqpoint{4.336196in}{0.739975in}}%
\pgfpathlineto{\pgfqpoint{4.355640in}{0.731358in}}%
\pgfpathlineto{\pgfqpoint{4.538409in}{0.648881in}}%
\pgfpathlineto{\pgfqpoint{4.752287in}{0.552128in}}%
\pgfpathlineto{\pgfqpoint{4.752287in}{0.552128in}}%
\pgfusepath{stroke}%
\end{pgfscope}%
\begin{pgfscope}%
\pgfpathrectangle{\pgfqpoint{0.571156in}{0.402778in}}{\pgfqpoint{4.380233in}{2.645833in}}%
\pgfusepath{clip}%
\pgfsetrectcap%
\pgfsetroundjoin%
\pgfsetlinewidth{1.505625pt}%
\definecolor{currentstroke}{rgb}{0.090196,0.745098,0.811765}%
\pgfsetstrokecolor{currentstroke}%
\pgfsetdash{}{0pt}%
\pgfpathmoveto{\pgfqpoint{0.774135in}{0.388889in}}%
\pgfpathlineto{\pgfqpoint{0.774146in}{1.403008in}}%
\pgfpathlineto{\pgfqpoint{0.778035in}{1.446338in}}%
\pgfpathlineto{\pgfqpoint{0.781923in}{1.449970in}}%
\pgfpathlineto{\pgfqpoint{0.785812in}{1.447297in}}%
\pgfpathlineto{\pgfqpoint{0.789701in}{1.447305in}}%
\pgfpathlineto{\pgfqpoint{0.797478in}{1.450373in}}%
\pgfpathlineto{\pgfqpoint{0.816922in}{1.450283in}}%
\pgfpathlineto{\pgfqpoint{1.003579in}{1.449987in}}%
\pgfpathlineto{\pgfqpoint{1.209680in}{1.449345in}}%
\pgfpathlineto{\pgfqpoint{1.213569in}{1.531046in}}%
\pgfpathlineto{\pgfqpoint{1.217458in}{1.550705in}}%
\pgfpathlineto{\pgfqpoint{1.221347in}{1.552476in}}%
\pgfpathlineto{\pgfqpoint{1.229124in}{1.551130in}}%
\pgfpathlineto{\pgfqpoint{1.240790in}{1.552740in}}%
\pgfpathlineto{\pgfqpoint{1.306898in}{1.552243in}}%
\pgfpathlineto{\pgfqpoint{1.649104in}{1.548927in}}%
\pgfpathlineto{\pgfqpoint{1.652992in}{1.593426in}}%
\pgfpathlineto{\pgfqpoint{1.656881in}{1.605690in}}%
\pgfpathlineto{\pgfqpoint{1.660770in}{1.606769in}}%
\pgfpathlineto{\pgfqpoint{1.668547in}{1.605798in}}%
\pgfpathlineto{\pgfqpoint{1.684102in}{1.606553in}}%
\pgfpathlineto{\pgfqpoint{1.851316in}{1.602857in}}%
\pgfpathlineto{\pgfqpoint{2.034085in}{1.596301in}}%
\pgfpathlineto{\pgfqpoint{2.092416in}{1.593418in}}%
\pgfpathlineto{\pgfqpoint{2.096304in}{1.619589in}}%
\pgfpathlineto{\pgfqpoint{2.100193in}{1.627202in}}%
\pgfpathlineto{\pgfqpoint{2.107970in}{1.626928in}}%
\pgfpathlineto{\pgfqpoint{2.255741in}{1.617561in}}%
\pgfpathlineto{\pgfqpoint{2.376291in}{1.606189in}}%
\pgfpathlineto{\pgfqpoint{2.485175in}{1.592834in}}%
\pgfpathlineto{\pgfqpoint{2.535728in}{1.585497in}}%
\pgfpathlineto{\pgfqpoint{2.539616in}{1.594262in}}%
\pgfpathlineto{\pgfqpoint{2.543505in}{1.596592in}}%
\pgfpathlineto{\pgfqpoint{2.570726in}{1.592549in}}%
\pgfpathlineto{\pgfqpoint{2.679610in}{1.572808in}}%
\pgfpathlineto{\pgfqpoint{2.776827in}{1.551834in}}%
\pgfpathlineto{\pgfqpoint{2.874045in}{1.527640in}}%
\pgfpathlineto{\pgfqpoint{2.971262in}{1.500317in}}%
\pgfpathlineto{\pgfqpoint{2.979040in}{1.498001in}}%
\pgfpathlineto{\pgfqpoint{2.982928in}{1.487819in}}%
\pgfpathlineto{\pgfqpoint{2.986817in}{1.483451in}}%
\pgfpathlineto{\pgfqpoint{3.010149in}{1.475939in}}%
\pgfpathlineto{\pgfqpoint{3.099589in}{1.447240in}}%
\pgfpathlineto{\pgfqpoint{3.212362in}{1.408012in}}%
\pgfpathlineto{\pgfqpoint{3.340689in}{1.359963in}}%
\pgfpathlineto{\pgfqpoint{3.422352in}{1.327852in}}%
\pgfpathlineto{\pgfqpoint{3.426240in}{1.301889in}}%
\pgfpathlineto{\pgfqpoint{3.430129in}{1.291191in}}%
\pgfpathlineto{\pgfqpoint{3.437906in}{1.287990in}}%
\pgfpathlineto{\pgfqpoint{3.647896in}{1.200097in}}%
\pgfpathlineto{\pgfqpoint{3.865663in}{1.105998in}}%
\pgfpathlineto{\pgfqpoint{3.869552in}{1.064968in}}%
\pgfpathlineto{\pgfqpoint{3.873441in}{1.047467in}}%
\pgfpathlineto{\pgfqpoint{3.877330in}{1.044340in}}%
\pgfpathlineto{\pgfqpoint{3.885107in}{1.042195in}}%
\pgfpathlineto{\pgfqpoint{3.896773in}{1.035271in}}%
\pgfpathlineto{\pgfqpoint{3.951215in}{1.011469in}}%
\pgfpathlineto{\pgfqpoint{4.308975in}{0.851764in}}%
\pgfpathlineto{\pgfqpoint{4.312864in}{0.784398in}}%
\pgfpathlineto{\pgfqpoint{4.316753in}{0.752384in}}%
\pgfpathlineto{\pgfqpoint{4.320642in}{0.747787in}}%
\pgfpathlineto{\pgfqpoint{4.324530in}{0.748769in}}%
\pgfpathlineto{\pgfqpoint{4.328419in}{0.746857in}}%
\pgfpathlineto{\pgfqpoint{4.336196in}{0.739975in}}%
\pgfpathlineto{\pgfqpoint{4.355640in}{0.731358in}}%
\pgfpathlineto{\pgfqpoint{4.538409in}{0.648881in}}%
\pgfpathlineto{\pgfqpoint{4.752287in}{0.552128in}}%
\pgfpathlineto{\pgfqpoint{4.752287in}{0.552128in}}%
\pgfusepath{stroke}%
\end{pgfscope}%
\begin{pgfscope}%
\pgfpathrectangle{\pgfqpoint{0.571156in}{0.402778in}}{\pgfqpoint{4.380233in}{2.645833in}}%
\pgfusepath{clip}%
\pgfsetrectcap%
\pgfsetroundjoin%
\pgfsetlinewidth{1.505625pt}%
\definecolor{currentstroke}{rgb}{0.121569,0.466667,0.705882}%
\pgfsetstrokecolor{currentstroke}%
\pgfsetdash{}{0pt}%
\pgfpathmoveto{\pgfqpoint{0.774135in}{0.388889in}}%
\pgfpathlineto{\pgfqpoint{0.774146in}{1.403008in}}%
\pgfpathlineto{\pgfqpoint{0.778035in}{1.446338in}}%
\pgfpathlineto{\pgfqpoint{0.781923in}{1.449970in}}%
\pgfpathlineto{\pgfqpoint{0.785812in}{1.447297in}}%
\pgfpathlineto{\pgfqpoint{0.789701in}{1.447305in}}%
\pgfpathlineto{\pgfqpoint{0.797478in}{1.450373in}}%
\pgfpathlineto{\pgfqpoint{0.816922in}{1.450283in}}%
\pgfpathlineto{\pgfqpoint{1.003579in}{1.449987in}}%
\pgfpathlineto{\pgfqpoint{1.209680in}{1.449345in}}%
\pgfpathlineto{\pgfqpoint{1.213569in}{1.531046in}}%
\pgfpathlineto{\pgfqpoint{1.217458in}{1.550705in}}%
\pgfpathlineto{\pgfqpoint{1.221347in}{1.552476in}}%
\pgfpathlineto{\pgfqpoint{1.229124in}{1.551130in}}%
\pgfpathlineto{\pgfqpoint{1.240790in}{1.552740in}}%
\pgfpathlineto{\pgfqpoint{1.306898in}{1.552243in}}%
\pgfpathlineto{\pgfqpoint{1.649104in}{1.548927in}}%
\pgfpathlineto{\pgfqpoint{1.652992in}{1.593426in}}%
\pgfpathlineto{\pgfqpoint{1.656881in}{1.605690in}}%
\pgfpathlineto{\pgfqpoint{1.660770in}{1.606769in}}%
\pgfpathlineto{\pgfqpoint{1.668547in}{1.605798in}}%
\pgfpathlineto{\pgfqpoint{1.684102in}{1.606553in}}%
\pgfpathlineto{\pgfqpoint{1.851316in}{1.602857in}}%
\pgfpathlineto{\pgfqpoint{2.034085in}{1.596301in}}%
\pgfpathlineto{\pgfqpoint{2.092416in}{1.593418in}}%
\pgfpathlineto{\pgfqpoint{2.096304in}{1.619589in}}%
\pgfpathlineto{\pgfqpoint{2.100193in}{1.627202in}}%
\pgfpathlineto{\pgfqpoint{2.107970in}{1.626928in}}%
\pgfpathlineto{\pgfqpoint{2.255741in}{1.617561in}}%
\pgfpathlineto{\pgfqpoint{2.376291in}{1.606189in}}%
\pgfpathlineto{\pgfqpoint{2.485175in}{1.592834in}}%
\pgfpathlineto{\pgfqpoint{2.535728in}{1.585497in}}%
\pgfpathlineto{\pgfqpoint{2.539616in}{1.594262in}}%
\pgfpathlineto{\pgfqpoint{2.543505in}{1.596592in}}%
\pgfpathlineto{\pgfqpoint{2.570726in}{1.592549in}}%
\pgfpathlineto{\pgfqpoint{2.679610in}{1.572808in}}%
\pgfpathlineto{\pgfqpoint{2.776827in}{1.551834in}}%
\pgfpathlineto{\pgfqpoint{2.874045in}{1.527640in}}%
\pgfpathlineto{\pgfqpoint{2.971262in}{1.500317in}}%
\pgfpathlineto{\pgfqpoint{2.979040in}{1.498001in}}%
\pgfpathlineto{\pgfqpoint{2.982928in}{1.487819in}}%
\pgfpathlineto{\pgfqpoint{2.986817in}{1.483451in}}%
\pgfpathlineto{\pgfqpoint{3.010149in}{1.475939in}}%
\pgfpathlineto{\pgfqpoint{3.099589in}{1.447240in}}%
\pgfpathlineto{\pgfqpoint{3.212362in}{1.408012in}}%
\pgfpathlineto{\pgfqpoint{3.340689in}{1.359963in}}%
\pgfpathlineto{\pgfqpoint{3.422352in}{1.327852in}}%
\pgfpathlineto{\pgfqpoint{3.426240in}{1.301889in}}%
\pgfpathlineto{\pgfqpoint{3.430129in}{1.291191in}}%
\pgfpathlineto{\pgfqpoint{3.437906in}{1.287990in}}%
\pgfpathlineto{\pgfqpoint{3.647896in}{1.200097in}}%
\pgfpathlineto{\pgfqpoint{3.865663in}{1.105998in}}%
\pgfpathlineto{\pgfqpoint{3.869552in}{1.064968in}}%
\pgfpathlineto{\pgfqpoint{3.873441in}{1.047467in}}%
\pgfpathlineto{\pgfqpoint{3.877330in}{1.044340in}}%
\pgfpathlineto{\pgfqpoint{3.885107in}{1.042195in}}%
\pgfpathlineto{\pgfqpoint{3.896773in}{1.035271in}}%
\pgfpathlineto{\pgfqpoint{3.951215in}{1.011469in}}%
\pgfpathlineto{\pgfqpoint{4.308975in}{0.851764in}}%
\pgfpathlineto{\pgfqpoint{4.312864in}{0.784398in}}%
\pgfpathlineto{\pgfqpoint{4.316753in}{0.752384in}}%
\pgfpathlineto{\pgfqpoint{4.320642in}{0.747787in}}%
\pgfpathlineto{\pgfqpoint{4.324530in}{0.748769in}}%
\pgfpathlineto{\pgfqpoint{4.328419in}{0.746857in}}%
\pgfpathlineto{\pgfqpoint{4.336196in}{0.739975in}}%
\pgfpathlineto{\pgfqpoint{4.355640in}{0.731358in}}%
\pgfpathlineto{\pgfqpoint{4.538409in}{0.648881in}}%
\pgfpathlineto{\pgfqpoint{4.752287in}{0.552128in}}%
\pgfpathlineto{\pgfqpoint{4.752287in}{0.552128in}}%
\pgfusepath{stroke}%
\end{pgfscope}%
\begin{pgfscope}%
\pgfpathrectangle{\pgfqpoint{0.571156in}{0.402778in}}{\pgfqpoint{4.380233in}{2.645833in}}%
\pgfusepath{clip}%
\pgfsetrectcap%
\pgfsetroundjoin%
\pgfsetlinewidth{1.505625pt}%
\definecolor{currentstroke}{rgb}{1.000000,0.498039,0.054902}%
\pgfsetstrokecolor{currentstroke}%
\pgfsetdash{}{0pt}%
\pgfpathmoveto{\pgfqpoint{0.774135in}{0.388889in}}%
\pgfpathlineto{\pgfqpoint{0.774146in}{1.403008in}}%
\pgfpathlineto{\pgfqpoint{0.778035in}{1.446338in}}%
\pgfpathlineto{\pgfqpoint{0.781923in}{1.449970in}}%
\pgfpathlineto{\pgfqpoint{0.785812in}{1.447297in}}%
\pgfpathlineto{\pgfqpoint{0.789701in}{1.447305in}}%
\pgfpathlineto{\pgfqpoint{0.797478in}{1.450373in}}%
\pgfpathlineto{\pgfqpoint{0.816922in}{1.450283in}}%
\pgfpathlineto{\pgfqpoint{1.003579in}{1.449987in}}%
\pgfpathlineto{\pgfqpoint{1.209680in}{1.449345in}}%
\pgfpathlineto{\pgfqpoint{1.213569in}{1.531046in}}%
\pgfpathlineto{\pgfqpoint{1.217458in}{1.550705in}}%
\pgfpathlineto{\pgfqpoint{1.221347in}{1.552476in}}%
\pgfpathlineto{\pgfqpoint{1.229124in}{1.551130in}}%
\pgfpathlineto{\pgfqpoint{1.240790in}{1.552740in}}%
\pgfpathlineto{\pgfqpoint{1.306898in}{1.552243in}}%
\pgfpathlineto{\pgfqpoint{1.649104in}{1.548927in}}%
\pgfpathlineto{\pgfqpoint{1.652992in}{1.593426in}}%
\pgfpathlineto{\pgfqpoint{1.656881in}{1.605690in}}%
\pgfpathlineto{\pgfqpoint{1.660770in}{1.606769in}}%
\pgfpathlineto{\pgfqpoint{1.668547in}{1.605798in}}%
\pgfpathlineto{\pgfqpoint{1.684102in}{1.606553in}}%
\pgfpathlineto{\pgfqpoint{1.851316in}{1.602857in}}%
\pgfpathlineto{\pgfqpoint{2.034085in}{1.596301in}}%
\pgfpathlineto{\pgfqpoint{2.092416in}{1.593418in}}%
\pgfpathlineto{\pgfqpoint{2.096304in}{1.619589in}}%
\pgfpathlineto{\pgfqpoint{2.100193in}{1.627202in}}%
\pgfpathlineto{\pgfqpoint{2.107970in}{1.626928in}}%
\pgfpathlineto{\pgfqpoint{2.255741in}{1.617561in}}%
\pgfpathlineto{\pgfqpoint{2.376291in}{1.606189in}}%
\pgfpathlineto{\pgfqpoint{2.485175in}{1.592834in}}%
\pgfpathlineto{\pgfqpoint{2.535728in}{1.585497in}}%
\pgfpathlineto{\pgfqpoint{2.539616in}{1.594262in}}%
\pgfpathlineto{\pgfqpoint{2.543505in}{1.596592in}}%
\pgfpathlineto{\pgfqpoint{2.570726in}{1.592549in}}%
\pgfpathlineto{\pgfqpoint{2.679610in}{1.572808in}}%
\pgfpathlineto{\pgfqpoint{2.776827in}{1.551834in}}%
\pgfpathlineto{\pgfqpoint{2.874045in}{1.527640in}}%
\pgfpathlineto{\pgfqpoint{2.971262in}{1.500317in}}%
\pgfpathlineto{\pgfqpoint{2.979040in}{1.498001in}}%
\pgfpathlineto{\pgfqpoint{2.982928in}{1.487819in}}%
\pgfpathlineto{\pgfqpoint{2.986817in}{1.483451in}}%
\pgfpathlineto{\pgfqpoint{3.010149in}{1.475939in}}%
\pgfpathlineto{\pgfqpoint{3.099589in}{1.447240in}}%
\pgfpathlineto{\pgfqpoint{3.212362in}{1.408012in}}%
\pgfpathlineto{\pgfqpoint{3.340689in}{1.359963in}}%
\pgfpathlineto{\pgfqpoint{3.422352in}{1.327852in}}%
\pgfpathlineto{\pgfqpoint{3.426240in}{1.301889in}}%
\pgfpathlineto{\pgfqpoint{3.430129in}{1.291191in}}%
\pgfpathlineto{\pgfqpoint{3.437906in}{1.287990in}}%
\pgfpathlineto{\pgfqpoint{3.647896in}{1.200097in}}%
\pgfpathlineto{\pgfqpoint{3.865663in}{1.105998in}}%
\pgfpathlineto{\pgfqpoint{3.869552in}{1.064968in}}%
\pgfpathlineto{\pgfqpoint{3.873441in}{1.047467in}}%
\pgfpathlineto{\pgfqpoint{3.877330in}{1.044340in}}%
\pgfpathlineto{\pgfqpoint{3.885107in}{1.042195in}}%
\pgfpathlineto{\pgfqpoint{3.896773in}{1.035271in}}%
\pgfpathlineto{\pgfqpoint{3.951215in}{1.011469in}}%
\pgfpathlineto{\pgfqpoint{4.308975in}{0.851764in}}%
\pgfpathlineto{\pgfqpoint{4.312864in}{0.784398in}}%
\pgfpathlineto{\pgfqpoint{4.316753in}{0.752384in}}%
\pgfpathlineto{\pgfqpoint{4.320642in}{0.747787in}}%
\pgfpathlineto{\pgfqpoint{4.324530in}{0.748769in}}%
\pgfpathlineto{\pgfqpoint{4.328419in}{0.746857in}}%
\pgfpathlineto{\pgfqpoint{4.336196in}{0.739975in}}%
\pgfpathlineto{\pgfqpoint{4.355640in}{0.731358in}}%
\pgfpathlineto{\pgfqpoint{4.538409in}{0.648881in}}%
\pgfpathlineto{\pgfqpoint{4.752287in}{0.552128in}}%
\pgfpathlineto{\pgfqpoint{4.752287in}{0.552128in}}%
\pgfusepath{stroke}%
\end{pgfscope}%
\begin{pgfscope}%
\pgfsetrectcap%
\pgfsetmiterjoin%
\pgfsetlinewidth{0.803000pt}%
\definecolor{currentstroke}{rgb}{0.000000,0.000000,0.000000}%
\pgfsetstrokecolor{currentstroke}%
\pgfsetdash{}{0pt}%
\pgfpathmoveto{\pgfqpoint{0.571156in}{0.402778in}}%
\pgfpathlineto{\pgfqpoint{0.571156in}{3.048611in}}%
\pgfusepath{stroke}%
\end{pgfscope}%
\begin{pgfscope}%
\pgfsetrectcap%
\pgfsetmiterjoin%
\pgfsetlinewidth{0.803000pt}%
\definecolor{currentstroke}{rgb}{0.000000,0.000000,0.000000}%
\pgfsetstrokecolor{currentstroke}%
\pgfsetdash{}{0pt}%
\pgfpathmoveto{\pgfqpoint{4.951389in}{0.402778in}}%
\pgfpathlineto{\pgfqpoint{4.951389in}{3.048611in}}%
\pgfusepath{stroke}%
\end{pgfscope}%
\begin{pgfscope}%
\pgfsetrectcap%
\pgfsetmiterjoin%
\pgfsetlinewidth{0.803000pt}%
\definecolor{currentstroke}{rgb}{0.000000,0.000000,0.000000}%
\pgfsetstrokecolor{currentstroke}%
\pgfsetdash{}{0pt}%
\pgfpathmoveto{\pgfqpoint{0.571156in}{0.402778in}}%
\pgfpathlineto{\pgfqpoint{4.951389in}{0.402778in}}%
\pgfusepath{stroke}%
\end{pgfscope}%
\begin{pgfscope}%
\pgfsetrectcap%
\pgfsetmiterjoin%
\pgfsetlinewidth{0.803000pt}%
\definecolor{currentstroke}{rgb}{0.000000,0.000000,0.000000}%
\pgfsetstrokecolor{currentstroke}%
\pgfsetdash{}{0pt}%
\pgfpathmoveto{\pgfqpoint{0.571156in}{3.048611in}}%
\pgfpathlineto{\pgfqpoint{4.951389in}{3.048611in}}%
\pgfusepath{stroke}%
\end{pgfscope}%
\begin{pgfscope}%
\pgfsetbuttcap%
\pgfsetmiterjoin%
\definecolor{currentfill}{rgb}{1.000000,1.000000,1.000000}%
\pgfsetfillcolor{currentfill}%
\pgfsetfillopacity{0.800000}%
\pgfsetlinewidth{1.003750pt}%
\definecolor{currentstroke}{rgb}{0.800000,0.800000,0.800000}%
\pgfsetstrokecolor{currentstroke}%
\pgfsetstrokeopacity{0.800000}%
\pgfsetdash{}{0pt}%
\pgfpathmoveto{\pgfqpoint{4.270833in}{0.581057in}}%
\pgfpathlineto{\pgfqpoint{4.854167in}{0.581057in}}%
\pgfpathquadraticcurveto{\pgfqpoint{4.881944in}{0.581057in}}{\pgfqpoint{4.881944in}{0.608834in}}%
\pgfpathlineto{\pgfqpoint{4.881944in}{2.951389in}}%
\pgfpathquadraticcurveto{\pgfqpoint{4.881944in}{2.979167in}}{\pgfqpoint{4.854167in}{2.979167in}}%
\pgfpathlineto{\pgfqpoint{4.270833in}{2.979167in}}%
\pgfpathquadraticcurveto{\pgfqpoint{4.243056in}{2.979167in}}{\pgfqpoint{4.243056in}{2.951389in}}%
\pgfpathlineto{\pgfqpoint{4.243056in}{0.608834in}}%
\pgfpathquadraticcurveto{\pgfqpoint{4.243056in}{0.581057in}}{\pgfqpoint{4.270833in}{0.581057in}}%
\pgfpathclose%
\pgfusepath{stroke,fill}%
\end{pgfscope}%
\begin{pgfscope}%
\pgfsetrectcap%
\pgfsetroundjoin%
\pgfsetlinewidth{1.505625pt}%
\definecolor{currentstroke}{rgb}{0.121569,0.466667,0.705882}%
\pgfsetstrokecolor{currentstroke}%
\pgfsetdash{}{0pt}%
\pgfpathmoveto{\pgfqpoint{4.298611in}{2.875000in}}%
\pgfpathlineto{\pgfqpoint{4.576389in}{2.875000in}}%
\pgfusepath{stroke}%
\end{pgfscope}%
\begin{pgfscope}%
\pgftext[x=4.687500in,y=2.826389in,left,base]{\rmfamily\fontsize{10.000000}{12.000000}\selectfont 1}%
\end{pgfscope}%
\begin{pgfscope}%
\pgfsetrectcap%
\pgfsetroundjoin%
\pgfsetlinewidth{1.505625pt}%
\definecolor{currentstroke}{rgb}{1.000000,0.498039,0.054902}%
\pgfsetstrokecolor{currentstroke}%
\pgfsetdash{}{0pt}%
\pgfpathmoveto{\pgfqpoint{4.298611in}{2.678630in}}%
\pgfpathlineto{\pgfqpoint{4.576389in}{2.678630in}}%
\pgfusepath{stroke}%
\end{pgfscope}%
\begin{pgfscope}%
\pgftext[x=4.687500in,y=2.630019in,left,base]{\rmfamily\fontsize{10.000000}{12.000000}\selectfont 2}%
\end{pgfscope}%
\begin{pgfscope}%
\pgfsetrectcap%
\pgfsetroundjoin%
\pgfsetlinewidth{1.505625pt}%
\definecolor{currentstroke}{rgb}{0.172549,0.627451,0.172549}%
\pgfsetstrokecolor{currentstroke}%
\pgfsetdash{}{0pt}%
\pgfpathmoveto{\pgfqpoint{4.298611in}{2.482259in}}%
\pgfpathlineto{\pgfqpoint{4.576389in}{2.482259in}}%
\pgfusepath{stroke}%
\end{pgfscope}%
\begin{pgfscope}%
\pgftext[x=4.687500in,y=2.433648in,left,base]{\rmfamily\fontsize{10.000000}{12.000000}\selectfont 3}%
\end{pgfscope}%
\begin{pgfscope}%
\pgfsetrectcap%
\pgfsetroundjoin%
\pgfsetlinewidth{1.505625pt}%
\definecolor{currentstroke}{rgb}{0.839216,0.152941,0.156863}%
\pgfsetstrokecolor{currentstroke}%
\pgfsetdash{}{0pt}%
\pgfpathmoveto{\pgfqpoint{4.298611in}{2.285889in}}%
\pgfpathlineto{\pgfqpoint{4.576389in}{2.285889in}}%
\pgfusepath{stroke}%
\end{pgfscope}%
\begin{pgfscope}%
\pgftext[x=4.687500in,y=2.237278in,left,base]{\rmfamily\fontsize{10.000000}{12.000000}\selectfont 4}%
\end{pgfscope}%
\begin{pgfscope}%
\pgfsetrectcap%
\pgfsetroundjoin%
\pgfsetlinewidth{1.505625pt}%
\definecolor{currentstroke}{rgb}{0.580392,0.403922,0.741176}%
\pgfsetstrokecolor{currentstroke}%
\pgfsetdash{}{0pt}%
\pgfpathmoveto{\pgfqpoint{4.298611in}{2.089519in}}%
\pgfpathlineto{\pgfqpoint{4.576389in}{2.089519in}}%
\pgfusepath{stroke}%
\end{pgfscope}%
\begin{pgfscope}%
\pgftext[x=4.687500in,y=2.040908in,left,base]{\rmfamily\fontsize{10.000000}{12.000000}\selectfont 5}%
\end{pgfscope}%
\begin{pgfscope}%
\pgfsetrectcap%
\pgfsetroundjoin%
\pgfsetlinewidth{1.505625pt}%
\definecolor{currentstroke}{rgb}{0.549020,0.337255,0.294118}%
\pgfsetstrokecolor{currentstroke}%
\pgfsetdash{}{0pt}%
\pgfpathmoveto{\pgfqpoint{4.298611in}{1.893149in}}%
\pgfpathlineto{\pgfqpoint{4.576389in}{1.893149in}}%
\pgfusepath{stroke}%
\end{pgfscope}%
\begin{pgfscope}%
\pgftext[x=4.687500in,y=1.844537in,left,base]{\rmfamily\fontsize{10.000000}{12.000000}\selectfont 6}%
\end{pgfscope}%
\begin{pgfscope}%
\pgfsetrectcap%
\pgfsetroundjoin%
\pgfsetlinewidth{1.505625pt}%
\definecolor{currentstroke}{rgb}{0.890196,0.466667,0.760784}%
\pgfsetstrokecolor{currentstroke}%
\pgfsetdash{}{0pt}%
\pgfpathmoveto{\pgfqpoint{4.298611in}{1.696778in}}%
\pgfpathlineto{\pgfqpoint{4.576389in}{1.696778in}}%
\pgfusepath{stroke}%
\end{pgfscope}%
\begin{pgfscope}%
\pgftext[x=4.687500in,y=1.648167in,left,base]{\rmfamily\fontsize{10.000000}{12.000000}\selectfont 7}%
\end{pgfscope}%
\begin{pgfscope}%
\pgfsetrectcap%
\pgfsetroundjoin%
\pgfsetlinewidth{1.505625pt}%
\definecolor{currentstroke}{rgb}{0.498039,0.498039,0.498039}%
\pgfsetstrokecolor{currentstroke}%
\pgfsetdash{}{0pt}%
\pgfpathmoveto{\pgfqpoint{4.298611in}{1.500408in}}%
\pgfpathlineto{\pgfqpoint{4.576389in}{1.500408in}}%
\pgfusepath{stroke}%
\end{pgfscope}%
\begin{pgfscope}%
\pgftext[x=4.687500in,y=1.451797in,left,base]{\rmfamily\fontsize{10.000000}{12.000000}\selectfont 8}%
\end{pgfscope}%
\begin{pgfscope}%
\pgfsetrectcap%
\pgfsetroundjoin%
\pgfsetlinewidth{1.505625pt}%
\definecolor{currentstroke}{rgb}{0.737255,0.741176,0.133333}%
\pgfsetstrokecolor{currentstroke}%
\pgfsetdash{}{0pt}%
\pgfpathmoveto{\pgfqpoint{4.298611in}{1.304038in}}%
\pgfpathlineto{\pgfqpoint{4.576389in}{1.304038in}}%
\pgfusepath{stroke}%
\end{pgfscope}%
\begin{pgfscope}%
\pgftext[x=4.687500in,y=1.255427in,left,base]{\rmfamily\fontsize{10.000000}{12.000000}\selectfont 9}%
\end{pgfscope}%
\begin{pgfscope}%
\pgfsetrectcap%
\pgfsetroundjoin%
\pgfsetlinewidth{1.505625pt}%
\definecolor{currentstroke}{rgb}{0.090196,0.745098,0.811765}%
\pgfsetstrokecolor{currentstroke}%
\pgfsetdash{}{0pt}%
\pgfpathmoveto{\pgfqpoint{4.298611in}{1.107668in}}%
\pgfpathlineto{\pgfqpoint{4.576389in}{1.107668in}}%
\pgfusepath{stroke}%
\end{pgfscope}%
\begin{pgfscope}%
\pgftext[x=4.687500in,y=1.059056in,left,base]{\rmfamily\fontsize{10.000000}{12.000000}\selectfont 10}%
\end{pgfscope}%
\begin{pgfscope}%
\pgfsetrectcap%
\pgfsetroundjoin%
\pgfsetlinewidth{1.505625pt}%
\definecolor{currentstroke}{rgb}{0.121569,0.466667,0.705882}%
\pgfsetstrokecolor{currentstroke}%
\pgfsetdash{}{0pt}%
\pgfpathmoveto{\pgfqpoint{4.298611in}{0.911297in}}%
\pgfpathlineto{\pgfqpoint{4.576389in}{0.911297in}}%
\pgfusepath{stroke}%
\end{pgfscope}%
\begin{pgfscope}%
\pgftext[x=4.687500in,y=0.862686in,left,base]{\rmfamily\fontsize{10.000000}{12.000000}\selectfont 11}%
\end{pgfscope}%
\begin{pgfscope}%
\pgfsetrectcap%
\pgfsetroundjoin%
\pgfsetlinewidth{1.505625pt}%
\definecolor{currentstroke}{rgb}{1.000000,0.498039,0.054902}%
\pgfsetstrokecolor{currentstroke}%
\pgfsetdash{}{0pt}%
\pgfpathmoveto{\pgfqpoint{4.298611in}{0.714927in}}%
\pgfpathlineto{\pgfqpoint{4.576389in}{0.714927in}}%
\pgfusepath{stroke}%
\end{pgfscope}%
\begin{pgfscope}%
\pgftext[x=4.687500in,y=0.666316in,left,base]{\rmfamily\fontsize{10.000000}{12.000000}\selectfont 12}%
\end{pgfscope}%
\end{pgfpicture}%
\makeatother%
\endgroup%
}
        \caption{Local relative error of the parareal solutions.}
        \label{fig:iters_log}
    \end{figure}
    \begin{equation*}
        E(t) = \frac{\left|\mathcal{P}(y_0,t_0,T,k)(t)-y(t)\right|}{\left|y(t)\right|}
    \end{equation*}
    \begin{description}[]
        \item[\(\mathcal{P}\!\left(y, t_{A}, t_{\Omega}, k\right)\)]<.-> Parareal solver up to iteration \(k\)
    \end{description}
\end{frame}