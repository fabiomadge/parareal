Um das tatsächliche Verhalten der Implementierung zu untersuchen, werden zwei Schaltungen simuliert. Der belastete Kondensator aus Abbildung~\ref{fig:cap} ist das einfache Beispiel. Das komplexere Beispiel ist ein Invertierer. Dabei handelt es sich sich um eine digitale Schaltung, welche das Eingangssignal negiert. Die konkrete Schaltung besteht aus neun Elementen, wobei zwei dynamisch sind. Abbildung~\ref{fig:sol_sim} zeigt jeweils das Ergebnis der Simulation nach der ersten Iteration und ein möglichst exaktes. Um die Ergebnisse besser beurteilen zu können, ist in Abbildung~\ref{fig:error_local_sim} der lokale absolute Fehler \(E(t) = \left|\mathcal{P}(y_0,t_0,T,k)(t)-y(t)\right|\) gezeigt. Abschließend findet sich in Abbildung~\ref{fig:error_global_sim} der jeweilige globale Fehler \(E(k) = \int_{t_0}^{T}E(t) dt = \int_{t_0}^{T}\left|\mathcal{P}(y_0,t_0,T,k)(t)-y(t)\right| dt\) über die Iterationen.

\begin{figure}[ht]
    \centering
        %% Creator: Matplotlib, PGF backend
%%
%% To include the figure in your LaTeX document, write
%%   \input{<filename>.pgf}
%%
%% Make sure the required packages are loaded in your preamble
%%   \usepackage{pgf}
%%
%% Figures using additional raster images can only be included by \input if
%% they are in the same directory as the main LaTeX file. For loading figures
%% from other directories you can use the `import` package
%%   \usepackage{import}
%% and then include the figures with
%%   \import{<path to file>}{<filename>.pgf}
%%
%% Matplotlib used the following preamble
%%   \usepackage{fontspec}
%%   \setmainfont{Times New Roman}
%%   \setsansfont{Lucida Grande}
%%   \setmonofont{Andale Mono}
%%
\begingroup%
\makeatletter%
\begin{pgfpicture}%
\pgfpathrectangle{\pgfpointorigin}{\pgfqpoint{5.000000in}{3.083333in}}%
\pgfusepath{use as bounding box, clip}%
\begin{pgfscope}%
\pgfsetbuttcap%
\pgfsetmiterjoin%
\definecolor{currentfill}{rgb}{1.000000,1.000000,1.000000}%
\pgfsetfillcolor{currentfill}%
\pgfsetlinewidth{0.000000pt}%
\definecolor{currentstroke}{rgb}{1.000000,1.000000,1.000000}%
\pgfsetstrokecolor{currentstroke}%
\pgfsetdash{}{0pt}%
\pgfpathmoveto{\pgfqpoint{0.000000in}{0.000000in}}%
\pgfpathlineto{\pgfqpoint{5.000000in}{0.000000in}}%
\pgfpathlineto{\pgfqpoint{5.000000in}{3.083333in}}%
\pgfpathlineto{\pgfqpoint{0.000000in}{3.083333in}}%
\pgfpathclose%
\pgfusepath{fill}%
\end{pgfscope}%
\begin{pgfscope}%
\pgfsetbuttcap%
\pgfsetmiterjoin%
\definecolor{currentfill}{rgb}{1.000000,1.000000,1.000000}%
\pgfsetfillcolor{currentfill}%
\pgfsetlinewidth{0.000000pt}%
\definecolor{currentstroke}{rgb}{0.000000,0.000000,0.000000}%
\pgfsetstrokecolor{currentstroke}%
\pgfsetstrokeopacity{0.000000}%
\pgfsetdash{}{0pt}%
\pgfpathmoveto{\pgfqpoint{0.467954in}{0.402778in}}%
\pgfpathlineto{\pgfqpoint{4.880181in}{0.402778in}}%
\pgfpathlineto{\pgfqpoint{4.880181in}{3.034722in}}%
\pgfpathlineto{\pgfqpoint{0.467954in}{3.034722in}}%
\pgfpathclose%
\pgfusepath{fill}%
\end{pgfscope}%
\begin{pgfscope}%
\pgfsetbuttcap%
\pgfsetroundjoin%
\definecolor{currentfill}{rgb}{0.000000,0.000000,0.000000}%
\pgfsetfillcolor{currentfill}%
\pgfsetlinewidth{0.803000pt}%
\definecolor{currentstroke}{rgb}{0.000000,0.000000,0.000000}%
\pgfsetstrokecolor{currentstroke}%
\pgfsetdash{}{0pt}%
\pgfsys@defobject{currentmarker}{\pgfqpoint{0.000000in}{-0.048611in}}{\pgfqpoint{0.000000in}{0.000000in}}{%
\pgfpathmoveto{\pgfqpoint{0.000000in}{0.000000in}}%
\pgfpathlineto{\pgfqpoint{0.000000in}{-0.048611in}}%
\pgfusepath{stroke,fill}%
}%
\begin{pgfscope}%
\pgfsys@transformshift{0.668510in}{0.402778in}%
\pgfsys@useobject{currentmarker}{}%
\end{pgfscope}%
\end{pgfscope}%
\begin{pgfscope}%
\pgftext[x=0.668510in,y=0.305556in,,top]{\rmfamily\fontsize{10.000000}{12.000000}\selectfont \(\displaystyle 0.0000000\)}%
\end{pgfscope}%
\begin{pgfscope}%
\pgfsetbuttcap%
\pgfsetroundjoin%
\definecolor{currentfill}{rgb}{0.000000,0.000000,0.000000}%
\pgfsetfillcolor{currentfill}%
\pgfsetlinewidth{0.803000pt}%
\definecolor{currentstroke}{rgb}{0.000000,0.000000,0.000000}%
\pgfsetstrokecolor{currentstroke}%
\pgfsetdash{}{0pt}%
\pgfsys@defobject{currentmarker}{\pgfqpoint{0.000000in}{-0.048611in}}{\pgfqpoint{0.000000in}{0.000000in}}{%
\pgfpathmoveto{\pgfqpoint{0.000000in}{0.000000in}}%
\pgfpathlineto{\pgfqpoint{0.000000in}{-0.048611in}}%
\pgfusepath{stroke,fill}%
}%
\begin{pgfscope}%
\pgfsys@transformshift{1.470732in}{0.402778in}%
\pgfsys@useobject{currentmarker}{}%
\end{pgfscope}%
\end{pgfscope}%
\begin{pgfscope}%
\pgftext[x=1.470732in,y=0.305556in,,top]{\rmfamily\fontsize{10.000000}{12.000000}\selectfont \(\displaystyle 0.0000002\)}%
\end{pgfscope}%
\begin{pgfscope}%
\pgfsetbuttcap%
\pgfsetroundjoin%
\definecolor{currentfill}{rgb}{0.000000,0.000000,0.000000}%
\pgfsetfillcolor{currentfill}%
\pgfsetlinewidth{0.803000pt}%
\definecolor{currentstroke}{rgb}{0.000000,0.000000,0.000000}%
\pgfsetstrokecolor{currentstroke}%
\pgfsetdash{}{0pt}%
\pgfsys@defobject{currentmarker}{\pgfqpoint{0.000000in}{-0.048611in}}{\pgfqpoint{0.000000in}{0.000000in}}{%
\pgfpathmoveto{\pgfqpoint{0.000000in}{0.000000in}}%
\pgfpathlineto{\pgfqpoint{0.000000in}{-0.048611in}}%
\pgfusepath{stroke,fill}%
}%
\begin{pgfscope}%
\pgfsys@transformshift{2.272955in}{0.402778in}%
\pgfsys@useobject{currentmarker}{}%
\end{pgfscope}%
\end{pgfscope}%
\begin{pgfscope}%
\pgftext[x=2.272955in,y=0.305556in,,top]{\rmfamily\fontsize{10.000000}{12.000000}\selectfont \(\displaystyle 0.0000004\)}%
\end{pgfscope}%
\begin{pgfscope}%
\pgfsetbuttcap%
\pgfsetroundjoin%
\definecolor{currentfill}{rgb}{0.000000,0.000000,0.000000}%
\pgfsetfillcolor{currentfill}%
\pgfsetlinewidth{0.803000pt}%
\definecolor{currentstroke}{rgb}{0.000000,0.000000,0.000000}%
\pgfsetstrokecolor{currentstroke}%
\pgfsetdash{}{0pt}%
\pgfsys@defobject{currentmarker}{\pgfqpoint{0.000000in}{-0.048611in}}{\pgfqpoint{0.000000in}{0.000000in}}{%
\pgfpathmoveto{\pgfqpoint{0.000000in}{0.000000in}}%
\pgfpathlineto{\pgfqpoint{0.000000in}{-0.048611in}}%
\pgfusepath{stroke,fill}%
}%
\begin{pgfscope}%
\pgfsys@transformshift{3.075177in}{0.402778in}%
\pgfsys@useobject{currentmarker}{}%
\end{pgfscope}%
\end{pgfscope}%
\begin{pgfscope}%
\pgftext[x=3.075177in,y=0.305556in,,top]{\rmfamily\fontsize{10.000000}{12.000000}\selectfont \(\displaystyle 0.0000006\)}%
\end{pgfscope}%
\begin{pgfscope}%
\pgfsetbuttcap%
\pgfsetroundjoin%
\definecolor{currentfill}{rgb}{0.000000,0.000000,0.000000}%
\pgfsetfillcolor{currentfill}%
\pgfsetlinewidth{0.803000pt}%
\definecolor{currentstroke}{rgb}{0.000000,0.000000,0.000000}%
\pgfsetstrokecolor{currentstroke}%
\pgfsetdash{}{0pt}%
\pgfsys@defobject{currentmarker}{\pgfqpoint{0.000000in}{-0.048611in}}{\pgfqpoint{0.000000in}{0.000000in}}{%
\pgfpathmoveto{\pgfqpoint{0.000000in}{0.000000in}}%
\pgfpathlineto{\pgfqpoint{0.000000in}{-0.048611in}}%
\pgfusepath{stroke,fill}%
}%
\begin{pgfscope}%
\pgfsys@transformshift{3.877399in}{0.402778in}%
\pgfsys@useobject{currentmarker}{}%
\end{pgfscope}%
\end{pgfscope}%
\begin{pgfscope}%
\pgftext[x=3.877399in,y=0.305556in,,top]{\rmfamily\fontsize{10.000000}{12.000000}\selectfont \(\displaystyle 0.0000008\)}%
\end{pgfscope}%
\begin{pgfscope}%
\pgfsetbuttcap%
\pgfsetroundjoin%
\definecolor{currentfill}{rgb}{0.000000,0.000000,0.000000}%
\pgfsetfillcolor{currentfill}%
\pgfsetlinewidth{0.803000pt}%
\definecolor{currentstroke}{rgb}{0.000000,0.000000,0.000000}%
\pgfsetstrokecolor{currentstroke}%
\pgfsetdash{}{0pt}%
\pgfsys@defobject{currentmarker}{\pgfqpoint{0.000000in}{-0.048611in}}{\pgfqpoint{0.000000in}{0.000000in}}{%
\pgfpathmoveto{\pgfqpoint{0.000000in}{0.000000in}}%
\pgfpathlineto{\pgfqpoint{0.000000in}{-0.048611in}}%
\pgfusepath{stroke,fill}%
}%
\begin{pgfscope}%
\pgfsys@transformshift{4.679621in}{0.402778in}%
\pgfsys@useobject{currentmarker}{}%
\end{pgfscope}%
\end{pgfscope}%
\begin{pgfscope}%
\pgftext[x=4.679621in,y=0.305556in,,top]{\rmfamily\fontsize{10.000000}{12.000000}\selectfont \(\displaystyle 0.0000010\)}%
\end{pgfscope}%
\begin{pgfscope}%
\pgftext[x=2.674068in,y=0.123861in,,top]{\rmfamily\fontsize{10.000000}{12.000000}\selectfont t}%
\end{pgfscope}%
\begin{pgfscope}%
\pgfsetbuttcap%
\pgfsetroundjoin%
\definecolor{currentfill}{rgb}{0.000000,0.000000,0.000000}%
\pgfsetfillcolor{currentfill}%
\pgfsetlinewidth{0.803000pt}%
\definecolor{currentstroke}{rgb}{0.000000,0.000000,0.000000}%
\pgfsetstrokecolor{currentstroke}%
\pgfsetdash{}{0pt}%
\pgfsys@defobject{currentmarker}{\pgfqpoint{-0.048611in}{0.000000in}}{\pgfqpoint{0.000000in}{0.000000in}}{%
\pgfpathmoveto{\pgfqpoint{0.000000in}{0.000000in}}%
\pgfpathlineto{\pgfqpoint{-0.048611in}{0.000000in}}%
\pgfusepath{stroke,fill}%
}%
\begin{pgfscope}%
\pgfsys@transformshift{0.467954in}{0.522303in}%
\pgfsys@useobject{currentmarker}{}%
\end{pgfscope}%
\end{pgfscope}%
\begin{pgfscope}%
\pgftext[x=0.193262in,y=0.474085in,left,base]{\rmfamily\fontsize{10.000000}{12.000000}\selectfont \(\displaystyle 0.0\)}%
\end{pgfscope}%
\begin{pgfscope}%
\pgfsetbuttcap%
\pgfsetroundjoin%
\definecolor{currentfill}{rgb}{0.000000,0.000000,0.000000}%
\pgfsetfillcolor{currentfill}%
\pgfsetlinewidth{0.803000pt}%
\definecolor{currentstroke}{rgb}{0.000000,0.000000,0.000000}%
\pgfsetstrokecolor{currentstroke}%
\pgfsetdash{}{0pt}%
\pgfsys@defobject{currentmarker}{\pgfqpoint{-0.048611in}{0.000000in}}{\pgfqpoint{0.000000in}{0.000000in}}{%
\pgfpathmoveto{\pgfqpoint{0.000000in}{0.000000in}}%
\pgfpathlineto{\pgfqpoint{-0.048611in}{0.000000in}}%
\pgfusepath{stroke,fill}%
}%
\begin{pgfscope}%
\pgfsys@transformshift{0.467954in}{1.000860in}%
\pgfsys@useobject{currentmarker}{}%
\end{pgfscope}%
\end{pgfscope}%
\begin{pgfscope}%
\pgftext[x=0.193262in,y=0.952642in,left,base]{\rmfamily\fontsize{10.000000}{12.000000}\selectfont \(\displaystyle 0.2\)}%
\end{pgfscope}%
\begin{pgfscope}%
\pgfsetbuttcap%
\pgfsetroundjoin%
\definecolor{currentfill}{rgb}{0.000000,0.000000,0.000000}%
\pgfsetfillcolor{currentfill}%
\pgfsetlinewidth{0.803000pt}%
\definecolor{currentstroke}{rgb}{0.000000,0.000000,0.000000}%
\pgfsetstrokecolor{currentstroke}%
\pgfsetdash{}{0pt}%
\pgfsys@defobject{currentmarker}{\pgfqpoint{-0.048611in}{0.000000in}}{\pgfqpoint{0.000000in}{0.000000in}}{%
\pgfpathmoveto{\pgfqpoint{0.000000in}{0.000000in}}%
\pgfpathlineto{\pgfqpoint{-0.048611in}{0.000000in}}%
\pgfusepath{stroke,fill}%
}%
\begin{pgfscope}%
\pgfsys@transformshift{0.467954in}{1.479417in}%
\pgfsys@useobject{currentmarker}{}%
\end{pgfscope}%
\end{pgfscope}%
\begin{pgfscope}%
\pgftext[x=0.193262in,y=1.431199in,left,base]{\rmfamily\fontsize{10.000000}{12.000000}\selectfont \(\displaystyle 0.4\)}%
\end{pgfscope}%
\begin{pgfscope}%
\pgfsetbuttcap%
\pgfsetroundjoin%
\definecolor{currentfill}{rgb}{0.000000,0.000000,0.000000}%
\pgfsetfillcolor{currentfill}%
\pgfsetlinewidth{0.803000pt}%
\definecolor{currentstroke}{rgb}{0.000000,0.000000,0.000000}%
\pgfsetstrokecolor{currentstroke}%
\pgfsetdash{}{0pt}%
\pgfsys@defobject{currentmarker}{\pgfqpoint{-0.048611in}{0.000000in}}{\pgfqpoint{0.000000in}{0.000000in}}{%
\pgfpathmoveto{\pgfqpoint{0.000000in}{0.000000in}}%
\pgfpathlineto{\pgfqpoint{-0.048611in}{0.000000in}}%
\pgfusepath{stroke,fill}%
}%
\begin{pgfscope}%
\pgfsys@transformshift{0.467954in}{1.957974in}%
\pgfsys@useobject{currentmarker}{}%
\end{pgfscope}%
\end{pgfscope}%
\begin{pgfscope}%
\pgftext[x=0.193262in,y=1.909756in,left,base]{\rmfamily\fontsize{10.000000}{12.000000}\selectfont \(\displaystyle 0.6\)}%
\end{pgfscope}%
\begin{pgfscope}%
\pgfsetbuttcap%
\pgfsetroundjoin%
\definecolor{currentfill}{rgb}{0.000000,0.000000,0.000000}%
\pgfsetfillcolor{currentfill}%
\pgfsetlinewidth{0.803000pt}%
\definecolor{currentstroke}{rgb}{0.000000,0.000000,0.000000}%
\pgfsetstrokecolor{currentstroke}%
\pgfsetdash{}{0pt}%
\pgfsys@defobject{currentmarker}{\pgfqpoint{-0.048611in}{0.000000in}}{\pgfqpoint{0.000000in}{0.000000in}}{%
\pgfpathmoveto{\pgfqpoint{0.000000in}{0.000000in}}%
\pgfpathlineto{\pgfqpoint{-0.048611in}{0.000000in}}%
\pgfusepath{stroke,fill}%
}%
\begin{pgfscope}%
\pgfsys@transformshift{0.467954in}{2.436531in}%
\pgfsys@useobject{currentmarker}{}%
\end{pgfscope}%
\end{pgfscope}%
\begin{pgfscope}%
\pgftext[x=0.193262in,y=2.388314in,left,base]{\rmfamily\fontsize{10.000000}{12.000000}\selectfont \(\displaystyle 0.8\)}%
\end{pgfscope}%
\begin{pgfscope}%
\pgfsetbuttcap%
\pgfsetroundjoin%
\definecolor{currentfill}{rgb}{0.000000,0.000000,0.000000}%
\pgfsetfillcolor{currentfill}%
\pgfsetlinewidth{0.803000pt}%
\definecolor{currentstroke}{rgb}{0.000000,0.000000,0.000000}%
\pgfsetstrokecolor{currentstroke}%
\pgfsetdash{}{0pt}%
\pgfsys@defobject{currentmarker}{\pgfqpoint{-0.048611in}{0.000000in}}{\pgfqpoint{0.000000in}{0.000000in}}{%
\pgfpathmoveto{\pgfqpoint{0.000000in}{0.000000in}}%
\pgfpathlineto{\pgfqpoint{-0.048611in}{0.000000in}}%
\pgfusepath{stroke,fill}%
}%
\begin{pgfscope}%
\pgfsys@transformshift{0.467954in}{2.915088in}%
\pgfsys@useobject{currentmarker}{}%
\end{pgfscope}%
\end{pgfscope}%
\begin{pgfscope}%
\pgftext[x=0.193262in,y=2.866871in,left,base]{\rmfamily\fontsize{10.000000}{12.000000}\selectfont \(\displaystyle 1.0\)}%
\end{pgfscope}%
\begin{pgfscope}%
\pgftext[x=0.137707in,y=1.718750in,,bottom,rotate=90.000000]{\rmfamily\fontsize{10.000000}{12.000000}\selectfont y(t)}%
\end{pgfscope}%
\begin{pgfscope}%
\pgfpathrectangle{\pgfqpoint{0.467954in}{0.402778in}}{\pgfqpoint{4.412226in}{2.631944in}}%
\pgfusepath{clip}%
\pgfsetrectcap%
\pgfsetroundjoin%
\pgfsetlinewidth{1.505625pt}%
\definecolor{currentstroke}{rgb}{0.121569,0.466667,0.705882}%
\pgfsetstrokecolor{currentstroke}%
\pgfsetdash{}{0pt}%
\pgfpathmoveto{\pgfqpoint{0.668510in}{2.915088in}}%
\pgfpathlineto{\pgfqpoint{0.704614in}{2.709124in}}%
\pgfpathlineto{\pgfqpoint{0.740717in}{2.520889in}}%
\pgfpathlineto{\pgfqpoint{0.776821in}{2.348856in}}%
\pgfpathlineto{\pgfqpoint{0.812925in}{2.191632in}}%
\pgfpathlineto{\pgfqpoint{0.849028in}{2.047941in}}%
\pgfpathlineto{\pgfqpoint{0.885533in}{1.915225in}}%
\pgfpathlineto{\pgfqpoint{0.922038in}{1.794053in}}%
\pgfpathlineto{\pgfqpoint{0.958543in}{1.683423in}}%
\pgfpathlineto{\pgfqpoint{0.995449in}{1.581357in}}%
\pgfpathlineto{\pgfqpoint{1.032355in}{1.488262in}}%
\pgfpathlineto{\pgfqpoint{1.069662in}{1.402471in}}%
\pgfpathlineto{\pgfqpoint{1.106969in}{1.324298in}}%
\pgfpathlineto{\pgfqpoint{1.144677in}{1.252339in}}%
\pgfpathlineto{\pgfqpoint{1.182787in}{1.186171in}}%
\pgfpathlineto{\pgfqpoint{1.221297in}{1.125398in}}%
\pgfpathlineto{\pgfqpoint{1.260610in}{1.069093in}}%
\pgfpathlineto{\pgfqpoint{1.300324in}{1.017549in}}%
\pgfpathlineto{\pgfqpoint{1.340840in}{0.969968in}}%
\pgfpathlineto{\pgfqpoint{1.382159in}{0.926149in}}%
\pgfpathlineto{\pgfqpoint{1.424681in}{0.885528in}}%
\pgfpathlineto{\pgfqpoint{1.468407in}{0.848014in}}%
\pgfpathlineto{\pgfqpoint{1.513737in}{0.813209in}}%
\pgfpathlineto{\pgfqpoint{1.560671in}{0.781086in}}%
\pgfpathlineto{\pgfqpoint{1.609612in}{0.751361in}}%
\pgfpathlineto{\pgfqpoint{1.660558in}{0.724040in}}%
\pgfpathlineto{\pgfqpoint{1.714312in}{0.698738in}}%
\pgfpathlineto{\pgfqpoint{1.771276in}{0.675379in}}%
\pgfpathlineto{\pgfqpoint{1.831850in}{0.653923in}}%
\pgfpathlineto{\pgfqpoint{1.896435in}{0.634348in}}%
\pgfpathlineto{\pgfqpoint{1.966236in}{0.616453in}}%
\pgfpathlineto{\pgfqpoint{2.042053in}{0.600237in}}%
\pgfpathlineto{\pgfqpoint{2.125493in}{0.585601in}}%
\pgfpathlineto{\pgfqpoint{2.218159in}{0.572544in}}%
\pgfpathlineto{\pgfqpoint{2.322458in}{0.561040in}}%
\pgfpathlineto{\pgfqpoint{2.441600in}{0.551085in}}%
\pgfpathlineto{\pgfqpoint{2.580800in}{0.542646in}}%
\pgfpathlineto{\pgfqpoint{2.748080in}{0.535709in}}%
\pgfpathlineto{\pgfqpoint{2.957481in}{0.530257in}}%
\pgfpathlineto{\pgfqpoint{3.236282in}{0.526272in}}%
\pgfpathlineto{\pgfqpoint{3.650671in}{0.523716in}}%
\pgfpathlineto{\pgfqpoint{4.433318in}{0.522504in}}%
\pgfpathlineto{\pgfqpoint{4.679625in}{0.522412in}}%
\pgfpathlineto{\pgfqpoint{4.679625in}{0.522412in}}%
\pgfusepath{stroke}%
\end{pgfscope}%
\begin{pgfscope}%
\pgfpathrectangle{\pgfqpoint{0.467954in}{0.402778in}}{\pgfqpoint{4.412226in}{2.631944in}}%
\pgfusepath{clip}%
\pgfsetrectcap%
\pgfsetroundjoin%
\pgfsetlinewidth{1.505625pt}%
\definecolor{currentstroke}{rgb}{1.000000,0.498039,0.054902}%
\pgfsetstrokecolor{currentstroke}%
\pgfsetdash{}{0pt}%
\pgfpathmoveto{\pgfqpoint{0.668510in}{2.915088in}}%
\pgfpathlineto{\pgfqpoint{0.704350in}{2.710615in}}%
\pgfpathlineto{\pgfqpoint{0.740352in}{2.522804in}}%
\pgfpathlineto{\pgfqpoint{0.776441in}{2.350716in}}%
\pgfpathlineto{\pgfqpoint{0.812555in}{2.193327in}}%
\pgfpathlineto{\pgfqpoint{0.848868in}{2.048729in}}%
\pgfpathlineto{\pgfqpoint{0.885184in}{1.916632in}}%
\pgfpathlineto{\pgfqpoint{0.921692in}{1.795357in}}%
\pgfpathlineto{\pgfqpoint{0.958193in}{1.684653in}}%
\pgfpathlineto{\pgfqpoint{0.995062in}{1.582600in}}%
\pgfpathlineto{\pgfqpoint{1.002773in}{1.562416in}}%
\pgfpathlineto{\pgfqpoint{1.002773in}{1.677242in}}%
\pgfpathlineto{\pgfqpoint{1.004298in}{1.672864in}}%
\pgfpathlineto{\pgfqpoint{1.041099in}{1.572024in}}%
\pgfpathlineto{\pgfqpoint{1.077978in}{1.479837in}}%
\pgfpathlineto{\pgfqpoint{1.115159in}{1.395090in}}%
\pgfpathlineto{\pgfqpoint{1.152636in}{1.317254in}}%
\pgfpathlineto{\pgfqpoint{1.190471in}{1.245716in}}%
\pgfpathlineto{\pgfqpoint{1.228668in}{1.180022in}}%
\pgfpathlineto{\pgfqpoint{1.267322in}{1.119612in}}%
\pgfpathlineto{\pgfqpoint{1.306489in}{1.064058in}}%
\pgfpathlineto{\pgfqpoint{1.337033in}{1.024346in}}%
\pgfpathlineto{\pgfqpoint{1.337033in}{1.079432in}}%
\pgfpathlineto{\pgfqpoint{1.338557in}{1.077320in}}%
\pgfpathlineto{\pgfqpoint{1.378315in}{1.024958in}}%
\pgfpathlineto{\pgfqpoint{1.418867in}{0.976636in}}%
\pgfpathlineto{\pgfqpoint{1.460244in}{0.932116in}}%
\pgfpathlineto{\pgfqpoint{1.502503in}{0.891147in}}%
\pgfpathlineto{\pgfqpoint{1.546158in}{0.853121in}}%
\pgfpathlineto{\pgfqpoint{1.591063in}{0.818091in}}%
\pgfpathlineto{\pgfqpoint{1.637729in}{0.785613in}}%
\pgfpathlineto{\pgfqpoint{1.671292in}{0.764483in}}%
\pgfpathlineto{\pgfqpoint{1.671292in}{0.790254in}}%
\pgfpathlineto{\pgfqpoint{1.672816in}{0.789238in}}%
\pgfpathlineto{\pgfqpoint{1.721058in}{0.758997in}}%
\pgfpathlineto{\pgfqpoint{1.771465in}{0.731052in}}%
\pgfpathlineto{\pgfqpoint{1.824410in}{0.705245in}}%
\pgfpathlineto{\pgfqpoint{1.880188in}{0.681501in}}%
\pgfpathlineto{\pgfqpoint{1.939659in}{0.659569in}}%
\pgfpathlineto{\pgfqpoint{2.003127in}{0.639485in}}%
\pgfpathlineto{\pgfqpoint{2.005551in}{0.638779in}}%
\pgfpathlineto{\pgfqpoint{2.005551in}{0.650641in}}%
\pgfpathlineto{\pgfqpoint{2.007075in}{0.650154in}}%
\pgfpathlineto{\pgfqpoint{2.072717in}{0.630859in}}%
\pgfpathlineto{\pgfqpoint{2.143685in}{0.613259in}}%
\pgfpathlineto{\pgfqpoint{2.220848in}{0.597345in}}%
\pgfpathlineto{\pgfqpoint{2.305854in}{0.583017in}}%
\pgfpathlineto{\pgfqpoint{2.339810in}{0.578090in}}%
\pgfpathlineto{\pgfqpoint{2.339810in}{0.583772in}}%
\pgfpathlineto{\pgfqpoint{2.341335in}{0.583539in}}%
\pgfpathlineto{\pgfqpoint{2.435428in}{0.570737in}}%
\pgfpathlineto{\pgfqpoint{2.541668in}{0.559470in}}%
\pgfpathlineto{\pgfqpoint{2.663351in}{0.549746in}}%
\pgfpathlineto{\pgfqpoint{2.674070in}{0.549023in}}%
\pgfpathlineto{\pgfqpoint{2.674070in}{0.551744in}}%
\pgfpathlineto{\pgfqpoint{2.675594in}{0.551632in}}%
\pgfpathlineto{\pgfqpoint{2.813403in}{0.543106in}}%
\pgfpathlineto{\pgfqpoint{2.978749in}{0.536080in}}%
\pgfpathlineto{\pgfqpoint{3.089488in}{0.533822in}}%
\pgfpathlineto{\pgfqpoint{3.316935in}{0.528837in}}%
\pgfpathlineto{\pgfqpoint{3.587835in}{0.525968in}}%
\pgfpathlineto{\pgfqpoint{3.999636in}{0.523750in}}%
\pgfpathlineto{\pgfqpoint{4.679625in}{0.522626in}}%
\pgfpathlineto{\pgfqpoint{4.679625in}{0.522626in}}%
\pgfusepath{stroke}%
\end{pgfscope}%
\begin{pgfscope}%
\pgfsetrectcap%
\pgfsetmiterjoin%
\pgfsetlinewidth{0.803000pt}%
\definecolor{currentstroke}{rgb}{0.000000,0.000000,0.000000}%
\pgfsetstrokecolor{currentstroke}%
\pgfsetdash{}{0pt}%
\pgfpathmoveto{\pgfqpoint{0.467954in}{0.402778in}}%
\pgfpathlineto{\pgfqpoint{0.467954in}{3.034722in}}%
\pgfusepath{stroke}%
\end{pgfscope}%
\begin{pgfscope}%
\pgfsetrectcap%
\pgfsetmiterjoin%
\pgfsetlinewidth{0.803000pt}%
\definecolor{currentstroke}{rgb}{0.000000,0.000000,0.000000}%
\pgfsetstrokecolor{currentstroke}%
\pgfsetdash{}{0pt}%
\pgfpathmoveto{\pgfqpoint{4.880181in}{0.402778in}}%
\pgfpathlineto{\pgfqpoint{4.880181in}{3.034722in}}%
\pgfusepath{stroke}%
\end{pgfscope}%
\begin{pgfscope}%
\pgfsetrectcap%
\pgfsetmiterjoin%
\pgfsetlinewidth{0.803000pt}%
\definecolor{currentstroke}{rgb}{0.000000,0.000000,0.000000}%
\pgfsetstrokecolor{currentstroke}%
\pgfsetdash{}{0pt}%
\pgfpathmoveto{\pgfqpoint{0.467954in}{0.402778in}}%
\pgfpathlineto{\pgfqpoint{4.880181in}{0.402778in}}%
\pgfusepath{stroke}%
\end{pgfscope}%
\begin{pgfscope}%
\pgfsetrectcap%
\pgfsetmiterjoin%
\pgfsetlinewidth{0.803000pt}%
\definecolor{currentstroke}{rgb}{0.000000,0.000000,0.000000}%
\pgfsetstrokecolor{currentstroke}%
\pgfsetdash{}{0pt}%
\pgfpathmoveto{\pgfqpoint{0.467954in}{3.034722in}}%
\pgfpathlineto{\pgfqpoint{4.880181in}{3.034722in}}%
\pgfusepath{stroke}%
\end{pgfscope}%
\begin{pgfscope}%
\pgfsetbuttcap%
\pgfsetmiterjoin%
\definecolor{currentfill}{rgb}{1.000000,1.000000,1.000000}%
\pgfsetfillcolor{currentfill}%
\pgfsetfillopacity{0.800000}%
\pgfsetlinewidth{1.003750pt}%
\definecolor{currentstroke}{rgb}{0.800000,0.800000,0.800000}%
\pgfsetstrokecolor{currentstroke}%
\pgfsetstrokeopacity{0.800000}%
\pgfsetdash{}{0pt}%
\pgfpathmoveto{\pgfqpoint{4.037747in}{2.530871in}}%
\pgfpathlineto{\pgfqpoint{4.782959in}{2.530871in}}%
\pgfpathquadraticcurveto{\pgfqpoint{4.810736in}{2.530871in}}{\pgfqpoint{4.810736in}{2.558648in}}%
\pgfpathlineto{\pgfqpoint{4.810736in}{2.937500in}}%
\pgfpathquadraticcurveto{\pgfqpoint{4.810736in}{2.965278in}}{\pgfqpoint{4.782959in}{2.965278in}}%
\pgfpathlineto{\pgfqpoint{4.037747in}{2.965278in}}%
\pgfpathquadraticcurveto{\pgfqpoint{4.009969in}{2.965278in}}{\pgfqpoint{4.009969in}{2.937500in}}%
\pgfpathlineto{\pgfqpoint{4.009969in}{2.558648in}}%
\pgfpathquadraticcurveto{\pgfqpoint{4.009969in}{2.530871in}}{\pgfqpoint{4.037747in}{2.530871in}}%
\pgfpathclose%
\pgfusepath{stroke,fill}%
\end{pgfscope}%
\begin{pgfscope}%
\pgfsetrectcap%
\pgfsetroundjoin%
\pgfsetlinewidth{1.505625pt}%
\definecolor{currentstroke}{rgb}{0.121569,0.466667,0.705882}%
\pgfsetstrokecolor{currentstroke}%
\pgfsetdash{}{0pt}%
\pgfpathmoveto{\pgfqpoint{4.065524in}{2.861111in}}%
\pgfpathlineto{\pgfqpoint{4.343302in}{2.861111in}}%
\pgfusepath{stroke}%
\end{pgfscope}%
\begin{pgfscope}%
\pgftext[x=4.454413in,y=2.812500in,left,base]{\rmfamily\fontsize{10.000000}{12.000000}\selectfont exakt}%
\end{pgfscope}%
\begin{pgfscope}%
\pgfsetrectcap%
\pgfsetroundjoin%
\pgfsetlinewidth{1.505625pt}%
\definecolor{currentstroke}{rgb}{1.000000,0.498039,0.054902}%
\pgfsetstrokecolor{currentstroke}%
\pgfsetdash{}{0pt}%
\pgfpathmoveto{\pgfqpoint{4.065524in}{2.664741in}}%
\pgfpathlineto{\pgfqpoint{4.343302in}{2.664741in}}%
\pgfusepath{stroke}%
\end{pgfscope}%
\begin{pgfscope}%
\pgftext[x=4.454413in,y=2.616130in,left,base]{\rmfamily\fontsize{10.000000}{12.000000}\selectfont k=1}%
\end{pgfscope}%
\end{pgfpicture}%
\makeatother%
\endgroup%

        %% Creator: Matplotlib, PGF backend
%%
%% To include the figure in your LaTeX document, write
%%   \input{<filename>.pgf}
%%
%% Make sure the required packages are loaded in your preamble
%%   \usepackage{pgf}
%%
%% Figures using additional raster images can only be included by \input if
%% they are in the same directory as the main LaTeX file. For loading figures
%% from other directories you can use the `import` package
%%   \usepackage{import}
%% and then include the figures with
%%   \import{<path to file>}{<filename>.pgf}
%%
%% Matplotlib used the following preamble
%%   \usepackage{fontspec}
%%   \setmainfont{Times New Roman}
%%   \setsansfont{Lucida Grande}
%%   \setmonofont{Andale Mono}
%%
\begingroup%
\makeatletter%
\begin{pgfpicture}%
\pgfpathrectangle{\pgfpointorigin}{\pgfqpoint{5.000000in}{3.083333in}}%
\pgfusepath{use as bounding box, clip}%
\begin{pgfscope}%
\pgfsetbuttcap%
\pgfsetmiterjoin%
\definecolor{currentfill}{rgb}{1.000000,1.000000,1.000000}%
\pgfsetfillcolor{currentfill}%
\pgfsetlinewidth{0.000000pt}%
\definecolor{currentstroke}{rgb}{1.000000,1.000000,1.000000}%
\pgfsetstrokecolor{currentstroke}%
\pgfsetdash{}{0pt}%
\pgfpathmoveto{\pgfqpoint{0.000000in}{0.000000in}}%
\pgfpathlineto{\pgfqpoint{5.000000in}{0.000000in}}%
\pgfpathlineto{\pgfqpoint{5.000000in}{3.083333in}}%
\pgfpathlineto{\pgfqpoint{0.000000in}{3.083333in}}%
\pgfpathclose%
\pgfusepath{fill}%
\end{pgfscope}%
\begin{pgfscope}%
\pgfsetbuttcap%
\pgfsetmiterjoin%
\definecolor{currentfill}{rgb}{1.000000,1.000000,1.000000}%
\pgfsetfillcolor{currentfill}%
\pgfsetlinewidth{0.000000pt}%
\definecolor{currentstroke}{rgb}{0.000000,0.000000,0.000000}%
\pgfsetstrokecolor{currentstroke}%
\pgfsetstrokeopacity{0.000000}%
\pgfsetdash{}{0pt}%
\pgfpathmoveto{\pgfqpoint{0.537140in}{0.402778in}}%
\pgfpathlineto{\pgfqpoint{4.951389in}{0.402778in}}%
\pgfpathlineto{\pgfqpoint{4.951389in}{3.034722in}}%
\pgfpathlineto{\pgfqpoint{0.537140in}{3.034722in}}%
\pgfpathclose%
\pgfusepath{fill}%
\end{pgfscope}%
\begin{pgfscope}%
\pgfsetbuttcap%
\pgfsetroundjoin%
\definecolor{currentfill}{rgb}{0.000000,0.000000,0.000000}%
\pgfsetfillcolor{currentfill}%
\pgfsetlinewidth{0.803000pt}%
\definecolor{currentstroke}{rgb}{0.000000,0.000000,0.000000}%
\pgfsetstrokecolor{currentstroke}%
\pgfsetdash{}{0pt}%
\pgfsys@defobject{currentmarker}{\pgfqpoint{0.000000in}{-0.048611in}}{\pgfqpoint{0.000000in}{0.000000in}}{%
\pgfpathmoveto{\pgfqpoint{0.000000in}{0.000000in}}%
\pgfpathlineto{\pgfqpoint{0.000000in}{-0.048611in}}%
\pgfusepath{stroke,fill}%
}%
\begin{pgfscope}%
\pgfsys@transformshift{0.737787in}{0.402778in}%
\pgfsys@useobject{currentmarker}{}%
\end{pgfscope}%
\end{pgfscope}%
\begin{pgfscope}%
\pgftext[x=0.737787in,y=0.305556in,,top]{\rmfamily\fontsize{10.000000}{12.000000}\selectfont \(\displaystyle 0.00\)}%
\end{pgfscope}%
\begin{pgfscope}%
\pgfsetbuttcap%
\pgfsetroundjoin%
\definecolor{currentfill}{rgb}{0.000000,0.000000,0.000000}%
\pgfsetfillcolor{currentfill}%
\pgfsetlinewidth{0.803000pt}%
\definecolor{currentstroke}{rgb}{0.000000,0.000000,0.000000}%
\pgfsetstrokecolor{currentstroke}%
\pgfsetdash{}{0pt}%
\pgfsys@defobject{currentmarker}{\pgfqpoint{0.000000in}{-0.048611in}}{\pgfqpoint{0.000000in}{0.000000in}}{%
\pgfpathmoveto{\pgfqpoint{0.000000in}{0.000000in}}%
\pgfpathlineto{\pgfqpoint{0.000000in}{-0.048611in}}%
\pgfusepath{stroke,fill}%
}%
\begin{pgfscope}%
\pgfsys@transformshift{1.239404in}{0.402778in}%
\pgfsys@useobject{currentmarker}{}%
\end{pgfscope}%
\end{pgfscope}%
\begin{pgfscope}%
\pgftext[x=1.239404in,y=0.305556in,,top]{\rmfamily\fontsize{10.000000}{12.000000}\selectfont \(\displaystyle 0.25\)}%
\end{pgfscope}%
\begin{pgfscope}%
\pgfsetbuttcap%
\pgfsetroundjoin%
\definecolor{currentfill}{rgb}{0.000000,0.000000,0.000000}%
\pgfsetfillcolor{currentfill}%
\pgfsetlinewidth{0.803000pt}%
\definecolor{currentstroke}{rgb}{0.000000,0.000000,0.000000}%
\pgfsetstrokecolor{currentstroke}%
\pgfsetdash{}{0pt}%
\pgfsys@defobject{currentmarker}{\pgfqpoint{0.000000in}{-0.048611in}}{\pgfqpoint{0.000000in}{0.000000in}}{%
\pgfpathmoveto{\pgfqpoint{0.000000in}{0.000000in}}%
\pgfpathlineto{\pgfqpoint{0.000000in}{-0.048611in}}%
\pgfusepath{stroke,fill}%
}%
\begin{pgfscope}%
\pgfsys@transformshift{1.741021in}{0.402778in}%
\pgfsys@useobject{currentmarker}{}%
\end{pgfscope}%
\end{pgfscope}%
\begin{pgfscope}%
\pgftext[x=1.741021in,y=0.305556in,,top]{\rmfamily\fontsize{10.000000}{12.000000}\selectfont \(\displaystyle 0.50\)}%
\end{pgfscope}%
\begin{pgfscope}%
\pgfsetbuttcap%
\pgfsetroundjoin%
\definecolor{currentfill}{rgb}{0.000000,0.000000,0.000000}%
\pgfsetfillcolor{currentfill}%
\pgfsetlinewidth{0.803000pt}%
\definecolor{currentstroke}{rgb}{0.000000,0.000000,0.000000}%
\pgfsetstrokecolor{currentstroke}%
\pgfsetdash{}{0pt}%
\pgfsys@defobject{currentmarker}{\pgfqpoint{0.000000in}{-0.048611in}}{\pgfqpoint{0.000000in}{0.000000in}}{%
\pgfpathmoveto{\pgfqpoint{0.000000in}{0.000000in}}%
\pgfpathlineto{\pgfqpoint{0.000000in}{-0.048611in}}%
\pgfusepath{stroke,fill}%
}%
\begin{pgfscope}%
\pgfsys@transformshift{2.242638in}{0.402778in}%
\pgfsys@useobject{currentmarker}{}%
\end{pgfscope}%
\end{pgfscope}%
\begin{pgfscope}%
\pgftext[x=2.242638in,y=0.305556in,,top]{\rmfamily\fontsize{10.000000}{12.000000}\selectfont \(\displaystyle 0.75\)}%
\end{pgfscope}%
\begin{pgfscope}%
\pgfsetbuttcap%
\pgfsetroundjoin%
\definecolor{currentfill}{rgb}{0.000000,0.000000,0.000000}%
\pgfsetfillcolor{currentfill}%
\pgfsetlinewidth{0.803000pt}%
\definecolor{currentstroke}{rgb}{0.000000,0.000000,0.000000}%
\pgfsetstrokecolor{currentstroke}%
\pgfsetdash{}{0pt}%
\pgfsys@defobject{currentmarker}{\pgfqpoint{0.000000in}{-0.048611in}}{\pgfqpoint{0.000000in}{0.000000in}}{%
\pgfpathmoveto{\pgfqpoint{0.000000in}{0.000000in}}%
\pgfpathlineto{\pgfqpoint{0.000000in}{-0.048611in}}%
\pgfusepath{stroke,fill}%
}%
\begin{pgfscope}%
\pgfsys@transformshift{2.744254in}{0.402778in}%
\pgfsys@useobject{currentmarker}{}%
\end{pgfscope}%
\end{pgfscope}%
\begin{pgfscope}%
\pgftext[x=2.744254in,y=0.305556in,,top]{\rmfamily\fontsize{10.000000}{12.000000}\selectfont \(\displaystyle 1.00\)}%
\end{pgfscope}%
\begin{pgfscope}%
\pgfsetbuttcap%
\pgfsetroundjoin%
\definecolor{currentfill}{rgb}{0.000000,0.000000,0.000000}%
\pgfsetfillcolor{currentfill}%
\pgfsetlinewidth{0.803000pt}%
\definecolor{currentstroke}{rgb}{0.000000,0.000000,0.000000}%
\pgfsetstrokecolor{currentstroke}%
\pgfsetdash{}{0pt}%
\pgfsys@defobject{currentmarker}{\pgfqpoint{0.000000in}{-0.048611in}}{\pgfqpoint{0.000000in}{0.000000in}}{%
\pgfpathmoveto{\pgfqpoint{0.000000in}{0.000000in}}%
\pgfpathlineto{\pgfqpoint{0.000000in}{-0.048611in}}%
\pgfusepath{stroke,fill}%
}%
\begin{pgfscope}%
\pgfsys@transformshift{3.245871in}{0.402778in}%
\pgfsys@useobject{currentmarker}{}%
\end{pgfscope}%
\end{pgfscope}%
\begin{pgfscope}%
\pgftext[x=3.245871in,y=0.305556in,,top]{\rmfamily\fontsize{10.000000}{12.000000}\selectfont \(\displaystyle 1.25\)}%
\end{pgfscope}%
\begin{pgfscope}%
\pgfsetbuttcap%
\pgfsetroundjoin%
\definecolor{currentfill}{rgb}{0.000000,0.000000,0.000000}%
\pgfsetfillcolor{currentfill}%
\pgfsetlinewidth{0.803000pt}%
\definecolor{currentstroke}{rgb}{0.000000,0.000000,0.000000}%
\pgfsetstrokecolor{currentstroke}%
\pgfsetdash{}{0pt}%
\pgfsys@defobject{currentmarker}{\pgfqpoint{0.000000in}{-0.048611in}}{\pgfqpoint{0.000000in}{0.000000in}}{%
\pgfpathmoveto{\pgfqpoint{0.000000in}{0.000000in}}%
\pgfpathlineto{\pgfqpoint{0.000000in}{-0.048611in}}%
\pgfusepath{stroke,fill}%
}%
\begin{pgfscope}%
\pgfsys@transformshift{3.747488in}{0.402778in}%
\pgfsys@useobject{currentmarker}{}%
\end{pgfscope}%
\end{pgfscope}%
\begin{pgfscope}%
\pgftext[x=3.747488in,y=0.305556in,,top]{\rmfamily\fontsize{10.000000}{12.000000}\selectfont \(\displaystyle 1.50\)}%
\end{pgfscope}%
\begin{pgfscope}%
\pgfsetbuttcap%
\pgfsetroundjoin%
\definecolor{currentfill}{rgb}{0.000000,0.000000,0.000000}%
\pgfsetfillcolor{currentfill}%
\pgfsetlinewidth{0.803000pt}%
\definecolor{currentstroke}{rgb}{0.000000,0.000000,0.000000}%
\pgfsetstrokecolor{currentstroke}%
\pgfsetdash{}{0pt}%
\pgfsys@defobject{currentmarker}{\pgfqpoint{0.000000in}{-0.048611in}}{\pgfqpoint{0.000000in}{0.000000in}}{%
\pgfpathmoveto{\pgfqpoint{0.000000in}{0.000000in}}%
\pgfpathlineto{\pgfqpoint{0.000000in}{-0.048611in}}%
\pgfusepath{stroke,fill}%
}%
\begin{pgfscope}%
\pgfsys@transformshift{4.249104in}{0.402778in}%
\pgfsys@useobject{currentmarker}{}%
\end{pgfscope}%
\end{pgfscope}%
\begin{pgfscope}%
\pgftext[x=4.249104in,y=0.305556in,,top]{\rmfamily\fontsize{10.000000}{12.000000}\selectfont \(\displaystyle 1.75\)}%
\end{pgfscope}%
\begin{pgfscope}%
\pgfsetbuttcap%
\pgfsetroundjoin%
\definecolor{currentfill}{rgb}{0.000000,0.000000,0.000000}%
\pgfsetfillcolor{currentfill}%
\pgfsetlinewidth{0.803000pt}%
\definecolor{currentstroke}{rgb}{0.000000,0.000000,0.000000}%
\pgfsetstrokecolor{currentstroke}%
\pgfsetdash{}{0pt}%
\pgfsys@defobject{currentmarker}{\pgfqpoint{0.000000in}{-0.048611in}}{\pgfqpoint{0.000000in}{0.000000in}}{%
\pgfpathmoveto{\pgfqpoint{0.000000in}{0.000000in}}%
\pgfpathlineto{\pgfqpoint{0.000000in}{-0.048611in}}%
\pgfusepath{stroke,fill}%
}%
\begin{pgfscope}%
\pgfsys@transformshift{4.750721in}{0.402778in}%
\pgfsys@useobject{currentmarker}{}%
\end{pgfscope}%
\end{pgfscope}%
\begin{pgfscope}%
\pgftext[x=4.750721in,y=0.305556in,,top]{\rmfamily\fontsize{10.000000}{12.000000}\selectfont \(\displaystyle 2.00\)}%
\end{pgfscope}%
\begin{pgfscope}%
\pgftext[x=2.744264in,y=0.123861in,,top]{\rmfamily\fontsize{10.000000}{12.000000}\selectfont t}%
\end{pgfscope}%
\begin{pgfscope}%
\pgftext[x=4.951389in,y=0.137750in,right,top]{\rmfamily\fontsize{10.000000}{12.000000}\selectfont \(\displaystyle \times10^{-7}\)}%
\end{pgfscope}%
\begin{pgfscope}%
\pgfsetbuttcap%
\pgfsetroundjoin%
\definecolor{currentfill}{rgb}{0.000000,0.000000,0.000000}%
\pgfsetfillcolor{currentfill}%
\pgfsetlinewidth{0.803000pt}%
\definecolor{currentstroke}{rgb}{0.000000,0.000000,0.000000}%
\pgfsetstrokecolor{currentstroke}%
\pgfsetdash{}{0pt}%
\pgfsys@defobject{currentmarker}{\pgfqpoint{-0.048611in}{0.000000in}}{\pgfqpoint{0.000000in}{0.000000in}}{%
\pgfpathmoveto{\pgfqpoint{0.000000in}{0.000000in}}%
\pgfpathlineto{\pgfqpoint{-0.048611in}{0.000000in}}%
\pgfusepath{stroke,fill}%
}%
\begin{pgfscope}%
\pgfsys@transformshift{0.537140in}{0.522412in}%
\pgfsys@useobject{currentmarker}{}%
\end{pgfscope}%
\end{pgfscope}%
\begin{pgfscope}%
\pgftext[x=0.262448in,y=0.474194in,left,base]{\rmfamily\fontsize{10.000000}{12.000000}\selectfont \(\displaystyle 0.0\)}%
\end{pgfscope}%
\begin{pgfscope}%
\pgfsetbuttcap%
\pgfsetroundjoin%
\definecolor{currentfill}{rgb}{0.000000,0.000000,0.000000}%
\pgfsetfillcolor{currentfill}%
\pgfsetlinewidth{0.803000pt}%
\definecolor{currentstroke}{rgb}{0.000000,0.000000,0.000000}%
\pgfsetstrokecolor{currentstroke}%
\pgfsetdash{}{0pt}%
\pgfsys@defobject{currentmarker}{\pgfqpoint{-0.048611in}{0.000000in}}{\pgfqpoint{0.000000in}{0.000000in}}{%
\pgfpathmoveto{\pgfqpoint{0.000000in}{0.000000in}}%
\pgfpathlineto{\pgfqpoint{-0.048611in}{0.000000in}}%
\pgfusepath{stroke,fill}%
}%
\begin{pgfscope}%
\pgfsys@transformshift{0.537140in}{0.841435in}%
\pgfsys@useobject{currentmarker}{}%
\end{pgfscope}%
\end{pgfscope}%
\begin{pgfscope}%
\pgftext[x=0.262448in,y=0.793217in,left,base]{\rmfamily\fontsize{10.000000}{12.000000}\selectfont \(\displaystyle 0.2\)}%
\end{pgfscope}%
\begin{pgfscope}%
\pgfsetbuttcap%
\pgfsetroundjoin%
\definecolor{currentfill}{rgb}{0.000000,0.000000,0.000000}%
\pgfsetfillcolor{currentfill}%
\pgfsetlinewidth{0.803000pt}%
\definecolor{currentstroke}{rgb}{0.000000,0.000000,0.000000}%
\pgfsetstrokecolor{currentstroke}%
\pgfsetdash{}{0pt}%
\pgfsys@defobject{currentmarker}{\pgfqpoint{-0.048611in}{0.000000in}}{\pgfqpoint{0.000000in}{0.000000in}}{%
\pgfpathmoveto{\pgfqpoint{0.000000in}{0.000000in}}%
\pgfpathlineto{\pgfqpoint{-0.048611in}{0.000000in}}%
\pgfusepath{stroke,fill}%
}%
\begin{pgfscope}%
\pgfsys@transformshift{0.537140in}{1.160459in}%
\pgfsys@useobject{currentmarker}{}%
\end{pgfscope}%
\end{pgfscope}%
\begin{pgfscope}%
\pgftext[x=0.262448in,y=1.112241in,left,base]{\rmfamily\fontsize{10.000000}{12.000000}\selectfont \(\displaystyle 0.4\)}%
\end{pgfscope}%
\begin{pgfscope}%
\pgfsetbuttcap%
\pgfsetroundjoin%
\definecolor{currentfill}{rgb}{0.000000,0.000000,0.000000}%
\pgfsetfillcolor{currentfill}%
\pgfsetlinewidth{0.803000pt}%
\definecolor{currentstroke}{rgb}{0.000000,0.000000,0.000000}%
\pgfsetstrokecolor{currentstroke}%
\pgfsetdash{}{0pt}%
\pgfsys@defobject{currentmarker}{\pgfqpoint{-0.048611in}{0.000000in}}{\pgfqpoint{0.000000in}{0.000000in}}{%
\pgfpathmoveto{\pgfqpoint{0.000000in}{0.000000in}}%
\pgfpathlineto{\pgfqpoint{-0.048611in}{0.000000in}}%
\pgfusepath{stroke,fill}%
}%
\begin{pgfscope}%
\pgfsys@transformshift{0.537140in}{1.479482in}%
\pgfsys@useobject{currentmarker}{}%
\end{pgfscope}%
\end{pgfscope}%
\begin{pgfscope}%
\pgftext[x=0.262448in,y=1.431265in,left,base]{\rmfamily\fontsize{10.000000}{12.000000}\selectfont \(\displaystyle 0.6\)}%
\end{pgfscope}%
\begin{pgfscope}%
\pgfsetbuttcap%
\pgfsetroundjoin%
\definecolor{currentfill}{rgb}{0.000000,0.000000,0.000000}%
\pgfsetfillcolor{currentfill}%
\pgfsetlinewidth{0.803000pt}%
\definecolor{currentstroke}{rgb}{0.000000,0.000000,0.000000}%
\pgfsetstrokecolor{currentstroke}%
\pgfsetdash{}{0pt}%
\pgfsys@defobject{currentmarker}{\pgfqpoint{-0.048611in}{0.000000in}}{\pgfqpoint{0.000000in}{0.000000in}}{%
\pgfpathmoveto{\pgfqpoint{0.000000in}{0.000000in}}%
\pgfpathlineto{\pgfqpoint{-0.048611in}{0.000000in}}%
\pgfusepath{stroke,fill}%
}%
\begin{pgfscope}%
\pgfsys@transformshift{0.537140in}{1.798506in}%
\pgfsys@useobject{currentmarker}{}%
\end{pgfscope}%
\end{pgfscope}%
\begin{pgfscope}%
\pgftext[x=0.262448in,y=1.750288in,left,base]{\rmfamily\fontsize{10.000000}{12.000000}\selectfont \(\displaystyle 0.8\)}%
\end{pgfscope}%
\begin{pgfscope}%
\pgfsetbuttcap%
\pgfsetroundjoin%
\definecolor{currentfill}{rgb}{0.000000,0.000000,0.000000}%
\pgfsetfillcolor{currentfill}%
\pgfsetlinewidth{0.803000pt}%
\definecolor{currentstroke}{rgb}{0.000000,0.000000,0.000000}%
\pgfsetstrokecolor{currentstroke}%
\pgfsetdash{}{0pt}%
\pgfsys@defobject{currentmarker}{\pgfqpoint{-0.048611in}{0.000000in}}{\pgfqpoint{0.000000in}{0.000000in}}{%
\pgfpathmoveto{\pgfqpoint{0.000000in}{0.000000in}}%
\pgfpathlineto{\pgfqpoint{-0.048611in}{0.000000in}}%
\pgfusepath{stroke,fill}%
}%
\begin{pgfscope}%
\pgfsys@transformshift{0.537140in}{2.117529in}%
\pgfsys@useobject{currentmarker}{}%
\end{pgfscope}%
\end{pgfscope}%
\begin{pgfscope}%
\pgftext[x=0.262448in,y=2.069312in,left,base]{\rmfamily\fontsize{10.000000}{12.000000}\selectfont \(\displaystyle 1.0\)}%
\end{pgfscope}%
\begin{pgfscope}%
\pgfsetbuttcap%
\pgfsetroundjoin%
\definecolor{currentfill}{rgb}{0.000000,0.000000,0.000000}%
\pgfsetfillcolor{currentfill}%
\pgfsetlinewidth{0.803000pt}%
\definecolor{currentstroke}{rgb}{0.000000,0.000000,0.000000}%
\pgfsetstrokecolor{currentstroke}%
\pgfsetdash{}{0pt}%
\pgfsys@defobject{currentmarker}{\pgfqpoint{-0.048611in}{0.000000in}}{\pgfqpoint{0.000000in}{0.000000in}}{%
\pgfpathmoveto{\pgfqpoint{0.000000in}{0.000000in}}%
\pgfpathlineto{\pgfqpoint{-0.048611in}{0.000000in}}%
\pgfusepath{stroke,fill}%
}%
\begin{pgfscope}%
\pgfsys@transformshift{0.537140in}{2.436553in}%
\pgfsys@useobject{currentmarker}{}%
\end{pgfscope}%
\end{pgfscope}%
\begin{pgfscope}%
\pgftext[x=0.262448in,y=2.388335in,left,base]{\rmfamily\fontsize{10.000000}{12.000000}\selectfont \(\displaystyle 1.2\)}%
\end{pgfscope}%
\begin{pgfscope}%
\pgfsetbuttcap%
\pgfsetroundjoin%
\definecolor{currentfill}{rgb}{0.000000,0.000000,0.000000}%
\pgfsetfillcolor{currentfill}%
\pgfsetlinewidth{0.803000pt}%
\definecolor{currentstroke}{rgb}{0.000000,0.000000,0.000000}%
\pgfsetstrokecolor{currentstroke}%
\pgfsetdash{}{0pt}%
\pgfsys@defobject{currentmarker}{\pgfqpoint{-0.048611in}{0.000000in}}{\pgfqpoint{0.000000in}{0.000000in}}{%
\pgfpathmoveto{\pgfqpoint{0.000000in}{0.000000in}}%
\pgfpathlineto{\pgfqpoint{-0.048611in}{0.000000in}}%
\pgfusepath{stroke,fill}%
}%
\begin{pgfscope}%
\pgfsys@transformshift{0.537140in}{2.755577in}%
\pgfsys@useobject{currentmarker}{}%
\end{pgfscope}%
\end{pgfscope}%
\begin{pgfscope}%
\pgftext[x=0.262448in,y=2.707359in,left,base]{\rmfamily\fontsize{10.000000}{12.000000}\selectfont \(\displaystyle 1.4\)}%
\end{pgfscope}%
\begin{pgfscope}%
\pgftext[x=0.206892in,y=1.718750in,,bottom,rotate=90.000000]{\rmfamily\fontsize{10.000000}{12.000000}\selectfont y(t)}%
\end{pgfscope}%
\begin{pgfscope}%
\pgfpathrectangle{\pgfqpoint{0.537140in}{0.402778in}}{\pgfqpoint{4.414249in}{2.631944in}}%
\pgfusepath{clip}%
\pgfsetrectcap%
\pgfsetroundjoin%
\pgfsetlinewidth{1.505625pt}%
\definecolor{currentstroke}{rgb}{0.121569,0.466667,0.705882}%
\pgfsetstrokecolor{currentstroke}%
\pgfsetdash{}{0pt}%
\pgfpathmoveto{\pgfqpoint{0.737787in}{2.915088in}}%
\pgfpathlineto{\pgfqpoint{0.796382in}{2.915088in}}%
\pgfpathlineto{\pgfqpoint{0.797586in}{2.843228in}}%
\pgfpathlineto{\pgfqpoint{0.812436in}{1.839457in}}%
\pgfpathlineto{\pgfqpoint{0.822469in}{1.346067in}}%
\pgfpathlineto{\pgfqpoint{0.830496in}{1.095860in}}%
\pgfpathlineto{\pgfqpoint{0.838924in}{0.922480in}}%
\pgfpathlineto{\pgfqpoint{0.847753in}{0.800990in}}%
\pgfpathlineto{\pgfqpoint{0.856984in}{0.715418in}}%
\pgfpathlineto{\pgfqpoint{0.866215in}{0.657330in}}%
\pgfpathlineto{\pgfqpoint{0.875847in}{0.615787in}}%
\pgfpathlineto{\pgfqpoint{0.885479in}{0.587350in}}%
\pgfpathlineto{\pgfqpoint{0.895512in}{0.567057in}}%
\pgfpathlineto{\pgfqpoint{0.905947in}{0.552751in}}%
\pgfpathlineto{\pgfqpoint{0.917184in}{0.542478in}}%
\pgfpathlineto{\pgfqpoint{0.930027in}{0.534899in}}%
\pgfpathlineto{\pgfqpoint{0.945679in}{0.529422in}}%
\pgfpathlineto{\pgfqpoint{0.966549in}{0.525664in}}%
\pgfpathlineto{\pgfqpoint{0.998254in}{0.523423in}}%
\pgfpathlineto{\pgfqpoint{1.064876in}{0.522498in}}%
\pgfpathlineto{\pgfqpoint{1.338185in}{0.522412in}}%
\pgfpathlineto{\pgfqpoint{1.339791in}{0.628433in}}%
\pgfpathlineto{\pgfqpoint{1.354640in}{1.624508in}}%
\pgfpathlineto{\pgfqpoint{1.364673in}{2.109690in}}%
\pgfpathlineto{\pgfqpoint{1.372700in}{2.353611in}}%
\pgfpathlineto{\pgfqpoint{1.381128in}{2.522989in}}%
\pgfpathlineto{\pgfqpoint{1.389958in}{2.641862in}}%
\pgfpathlineto{\pgfqpoint{1.399188in}{2.725689in}}%
\pgfpathlineto{\pgfqpoint{1.408419in}{2.782705in}}%
\pgfpathlineto{\pgfqpoint{1.417650in}{2.822045in}}%
\pgfpathlineto{\pgfqpoint{1.427282in}{2.850379in}}%
\pgfpathlineto{\pgfqpoint{1.437315in}{2.870596in}}%
\pgfpathlineto{\pgfqpoint{1.447750in}{2.884853in}}%
\pgfpathlineto{\pgfqpoint{1.458987in}{2.895087in}}%
\pgfpathlineto{\pgfqpoint{1.471830in}{2.902646in}}%
\pgfpathlineto{\pgfqpoint{1.487482in}{2.908101in}}%
\pgfpathlineto{\pgfqpoint{1.508352in}{2.911836in}}%
\pgfpathlineto{\pgfqpoint{1.540458in}{2.914081in}}%
\pgfpathlineto{\pgfqpoint{1.607481in}{2.914990in}}%
\pgfpathlineto{\pgfqpoint{1.799721in}{2.915087in}}%
\pgfpathlineto{\pgfqpoint{1.802530in}{2.715754in}}%
\pgfpathlineto{\pgfqpoint{1.816176in}{1.810336in}}%
\pgfpathlineto{\pgfqpoint{1.826209in}{1.326388in}}%
\pgfpathlineto{\pgfqpoint{1.834236in}{1.082984in}}%
\pgfpathlineto{\pgfqpoint{1.842664in}{0.913907in}}%
\pgfpathlineto{\pgfqpoint{1.851493in}{0.795231in}}%
\pgfpathlineto{\pgfqpoint{1.860724in}{0.711537in}}%
\pgfpathlineto{\pgfqpoint{1.869955in}{0.654642in}}%
\pgfpathlineto{\pgfqpoint{1.879587in}{0.613954in}}%
\pgfpathlineto{\pgfqpoint{1.889219in}{0.586088in}}%
\pgfpathlineto{\pgfqpoint{1.899252in}{0.566198in}}%
\pgfpathlineto{\pgfqpoint{1.909687in}{0.552170in}}%
\pgfpathlineto{\pgfqpoint{1.920924in}{0.542097in}}%
\pgfpathlineto{\pgfqpoint{1.933767in}{0.534661in}}%
\pgfpathlineto{\pgfqpoint{1.949419in}{0.529289in}}%
\pgfpathlineto{\pgfqpoint{1.970289in}{0.525602in}}%
\pgfpathlineto{\pgfqpoint{2.002396in}{0.523389in}}%
\pgfpathlineto{\pgfqpoint{2.070623in}{0.522491in}}%
\pgfpathlineto{\pgfqpoint{2.341524in}{0.522412in}}%
\pgfpathlineto{\pgfqpoint{2.346741in}{0.903061in}}%
\pgfpathlineto{\pgfqpoint{2.359584in}{1.720078in}}%
\pgfpathlineto{\pgfqpoint{2.368815in}{2.142839in}}%
\pgfpathlineto{\pgfqpoint{2.377243in}{2.384743in}}%
\pgfpathlineto{\pgfqpoint{2.385671in}{2.543786in}}%
\pgfpathlineto{\pgfqpoint{2.394500in}{2.655881in}}%
\pgfpathlineto{\pgfqpoint{2.403731in}{2.735168in}}%
\pgfpathlineto{\pgfqpoint{2.412962in}{2.789149in}}%
\pgfpathlineto{\pgfqpoint{2.422594in}{2.827835in}}%
\pgfpathlineto{\pgfqpoint{2.432226in}{2.854364in}}%
\pgfpathlineto{\pgfqpoint{2.442259in}{2.873312in}}%
\pgfpathlineto{\pgfqpoint{2.452694in}{2.886686in}}%
\pgfpathlineto{\pgfqpoint{2.464333in}{2.896581in}}%
\pgfpathlineto{\pgfqpoint{2.477577in}{2.903744in}}%
\pgfpathlineto{\pgfqpoint{2.493630in}{2.908807in}}%
\pgfpathlineto{\pgfqpoint{2.515704in}{2.912289in}}%
\pgfpathlineto{\pgfqpoint{2.551021in}{2.914317in}}%
\pgfpathlineto{\pgfqpoint{2.633295in}{2.915040in}}%
\pgfpathlineto{\pgfqpoint{2.803060in}{2.915087in}}%
\pgfpathlineto{\pgfqpoint{2.826337in}{1.453548in}}%
\pgfpathlineto{\pgfqpoint{2.834364in}{1.165514in}}%
\pgfpathlineto{\pgfqpoint{2.842792in}{0.968600in}}%
\pgfpathlineto{\pgfqpoint{2.851622in}{0.831869in}}%
\pgfpathlineto{\pgfqpoint{2.860451in}{0.739626in}}%
\pgfpathlineto{\pgfqpoint{2.869682in}{0.673810in}}%
\pgfpathlineto{\pgfqpoint{2.878912in}{0.628621in}}%
\pgfpathlineto{\pgfqpoint{2.888544in}{0.596164in}}%
\pgfpathlineto{\pgfqpoint{2.898176in}{0.573824in}}%
\pgfpathlineto{\pgfqpoint{2.908210in}{0.557832in}}%
\pgfpathlineto{\pgfqpoint{2.919046in}{0.546168in}}%
\pgfpathlineto{\pgfqpoint{2.931086in}{0.537655in}}%
\pgfpathlineto{\pgfqpoint{2.945133in}{0.531486in}}%
\pgfpathlineto{\pgfqpoint{2.963193in}{0.527078in}}%
\pgfpathlineto{\pgfqpoint{2.989280in}{0.524198in}}%
\pgfpathlineto{\pgfqpoint{3.035032in}{0.522742in}}%
\pgfpathlineto{\pgfqpoint{3.189546in}{0.522413in}}%
\pgfpathlineto{\pgfqpoint{3.344863in}{0.522423in}}%
\pgfpathlineto{\pgfqpoint{3.361318in}{1.636191in}}%
\pgfpathlineto{\pgfqpoint{3.371351in}{2.117409in}}%
\pgfpathlineto{\pgfqpoint{3.379378in}{2.358677in}}%
\pgfpathlineto{\pgfqpoint{3.387806in}{2.526368in}}%
\pgfpathlineto{\pgfqpoint{3.396635in}{2.644135in}}%
\pgfpathlineto{\pgfqpoint{3.405866in}{2.727227in}}%
\pgfpathlineto{\pgfqpoint{3.415097in}{2.783734in}}%
\pgfpathlineto{\pgfqpoint{3.424729in}{2.824141in}}%
\pgfpathlineto{\pgfqpoint{3.434361in}{2.851822in}}%
\pgfpathlineto{\pgfqpoint{3.444394in}{2.871579in}}%
\pgfpathlineto{\pgfqpoint{3.454829in}{2.885516in}}%
\pgfpathlineto{\pgfqpoint{3.466066in}{2.895527in}}%
\pgfpathlineto{\pgfqpoint{3.478909in}{2.902920in}}%
\pgfpathlineto{\pgfqpoint{3.494561in}{2.908254in}}%
\pgfpathlineto{\pgfqpoint{3.515430in}{2.911907in}}%
\pgfpathlineto{\pgfqpoint{3.547939in}{2.914118in}}%
\pgfpathlineto{\pgfqpoint{3.616968in}{2.915001in}}%
\pgfpathlineto{\pgfqpoint{3.805997in}{2.915087in}}%
\pgfpathlineto{\pgfqpoint{3.806399in}{2.910870in}}%
\pgfpathlineto{\pgfqpoint{3.821248in}{1.891264in}}%
\pgfpathlineto{\pgfqpoint{3.831683in}{1.364444in}}%
\pgfpathlineto{\pgfqpoint{3.839710in}{1.107917in}}%
\pgfpathlineto{\pgfqpoint{3.848138in}{0.930494in}}%
\pgfpathlineto{\pgfqpoint{3.856967in}{0.806391in}}%
\pgfpathlineto{\pgfqpoint{3.866198in}{0.719060in}}%
\pgfpathlineto{\pgfqpoint{3.875428in}{0.659801in}}%
\pgfpathlineto{\pgfqpoint{3.884659in}{0.618932in}}%
\pgfpathlineto{\pgfqpoint{3.894291in}{0.589511in}}%
\pgfpathlineto{\pgfqpoint{3.904325in}{0.568529in}}%
\pgfpathlineto{\pgfqpoint{3.914759in}{0.553744in}}%
\pgfpathlineto{\pgfqpoint{3.925997in}{0.543136in}}%
\pgfpathlineto{\pgfqpoint{3.938839in}{0.535305in}}%
\pgfpathlineto{\pgfqpoint{3.954090in}{0.529757in}}%
\pgfpathlineto{\pgfqpoint{3.974157in}{0.525922in}}%
\pgfpathlineto{\pgfqpoint{4.004658in}{0.523551in}}%
\pgfpathlineto{\pgfqpoint{4.066063in}{0.522530in}}%
\pgfpathlineto{\pgfqpoint{4.347800in}{0.522412in}}%
\pgfpathlineto{\pgfqpoint{4.348603in}{0.561421in}}%
\pgfpathlineto{\pgfqpoint{4.363452in}{1.573312in}}%
\pgfpathlineto{\pgfqpoint{4.373887in}{2.091362in}}%
\pgfpathlineto{\pgfqpoint{4.381914in}{2.341569in}}%
\pgfpathlineto{\pgfqpoint{4.390342in}{2.514972in}}%
\pgfpathlineto{\pgfqpoint{4.399171in}{2.636473in}}%
\pgfpathlineto{\pgfqpoint{4.408402in}{2.722059in}}%
\pgfpathlineto{\pgfqpoint{4.417633in}{2.780197in}}%
\pgfpathlineto{\pgfqpoint{4.427265in}{2.821731in}}%
\pgfpathlineto{\pgfqpoint{4.436897in}{2.850163in}}%
\pgfpathlineto{\pgfqpoint{4.446930in}{2.870449in}}%
\pgfpathlineto{\pgfqpoint{4.457365in}{2.884754in}}%
\pgfpathlineto{\pgfqpoint{4.468602in}{2.895021in}}%
\pgfpathlineto{\pgfqpoint{4.481445in}{2.902605in}}%
\pgfpathlineto{\pgfqpoint{4.497097in}{2.908078in}}%
\pgfpathlineto{\pgfqpoint{4.517967in}{2.911826in}}%
\pgfpathlineto{\pgfqpoint{4.550073in}{2.914078in}}%
\pgfpathlineto{\pgfqpoint{4.617096in}{2.914991in}}%
\pgfpathlineto{\pgfqpoint{4.750741in}{2.915085in}}%
\pgfpathlineto{\pgfqpoint{4.750741in}{2.915085in}}%
\pgfusepath{stroke}%
\end{pgfscope}%
\begin{pgfscope}%
\pgfpathrectangle{\pgfqpoint{0.537140in}{0.402778in}}{\pgfqpoint{4.414249in}{2.631944in}}%
\pgfusepath{clip}%
\pgfsetrectcap%
\pgfsetroundjoin%
\pgfsetlinewidth{1.505625pt}%
\definecolor{currentstroke}{rgb}{1.000000,0.498039,0.054902}%
\pgfsetstrokecolor{currentstroke}%
\pgfsetdash{}{0pt}%
\pgfpathmoveto{\pgfqpoint{0.737787in}{2.915088in}}%
\pgfpathlineto{\pgfqpoint{0.796657in}{2.914053in}}%
\pgfpathlineto{\pgfqpoint{0.810389in}{1.970587in}}%
\pgfpathlineto{\pgfqpoint{0.821574in}{1.393797in}}%
\pgfpathlineto{\pgfqpoint{0.829617in}{1.126358in}}%
\pgfpathlineto{\pgfqpoint{0.838044in}{0.942802in}}%
\pgfpathlineto{\pgfqpoint{0.846787in}{0.815712in}}%
\pgfpathlineto{\pgfqpoint{0.855755in}{0.727612in}}%
\pgfpathlineto{\pgfqpoint{0.864795in}{0.667036in}}%
\pgfpathlineto{\pgfqpoint{0.874034in}{0.624128in}}%
\pgfpathlineto{\pgfqpoint{0.883453in}{0.593791in}}%
\pgfpathlineto{\pgfqpoint{0.893189in}{0.572009in}}%
\pgfpathlineto{\pgfqpoint{0.903568in}{0.556161in}}%
\pgfpathlineto{\pgfqpoint{0.914626in}{0.544802in}}%
\pgfpathlineto{\pgfqpoint{0.926779in}{0.536702in}}%
\pgfpathlineto{\pgfqpoint{0.940984in}{0.530903in}}%
\pgfpathlineto{\pgfqpoint{0.960239in}{0.526615in}}%
\pgfpathlineto{\pgfqpoint{0.987779in}{0.523949in}}%
\pgfpathlineto{\pgfqpoint{1.038084in}{0.522657in}}%
\pgfpathlineto{\pgfqpoint{1.183689in}{0.522413in}}%
\pgfpathlineto{\pgfqpoint{1.183689in}{0.688792in}}%
\pgfpathlineto{\pgfqpoint{1.185206in}{0.679329in}}%
\pgfpathlineto{\pgfqpoint{1.194453in}{0.632646in}}%
\pgfpathlineto{\pgfqpoint{1.203913in}{0.599546in}}%
\pgfpathlineto{\pgfqpoint{1.213616in}{0.576062in}}%
\pgfpathlineto{\pgfqpoint{1.223611in}{0.559413in}}%
\pgfpathlineto{\pgfqpoint{1.234398in}{0.547217in}}%
\pgfpathlineto{\pgfqpoint{1.246179in}{0.538453in}}%
\pgfpathlineto{\pgfqpoint{1.260158in}{0.532030in}}%
\pgfpathlineto{\pgfqpoint{1.277983in}{0.527431in}}%
\pgfpathlineto{\pgfqpoint{1.303197in}{0.524410in}}%
\pgfpathlineto{\pgfqpoint{1.338389in}{0.523151in}}%
\pgfpathlineto{\pgfqpoint{1.338392in}{0.523321in}}%
\pgfpathlineto{\pgfqpoint{1.352222in}{1.473383in}}%
\pgfpathlineto{\pgfqpoint{1.363373in}{2.051562in}}%
\pgfpathlineto{\pgfqpoint{1.371425in}{2.316443in}}%
\pgfpathlineto{\pgfqpoint{1.379858in}{2.498302in}}%
\pgfpathlineto{\pgfqpoint{1.388605in}{2.624249in}}%
\pgfpathlineto{\pgfqpoint{1.397594in}{2.711821in}}%
\pgfpathlineto{\pgfqpoint{1.406668in}{2.771988in}}%
\pgfpathlineto{\pgfqpoint{1.415968in}{2.814668in}}%
\pgfpathlineto{\pgfqpoint{1.425392in}{2.844616in}}%
\pgfpathlineto{\pgfqpoint{1.435242in}{2.866329in}}%
\pgfpathlineto{\pgfqpoint{1.445492in}{2.881743in}}%
\pgfpathlineto{\pgfqpoint{1.456299in}{2.892716in}}%
\pgfpathlineto{\pgfqpoint{1.469223in}{2.901192in}}%
\pgfpathlineto{\pgfqpoint{1.483623in}{2.906848in}}%
\pgfpathlineto{\pgfqpoint{1.502734in}{2.910887in}}%
\pgfpathlineto{\pgfqpoint{1.525991in}{2.913072in}}%
\pgfpathlineto{\pgfqpoint{1.588137in}{2.914772in}}%
\pgfpathlineto{\pgfqpoint{1.629570in}{2.914983in}}%
\pgfpathlineto{\pgfqpoint{1.629571in}{2.756192in}}%
\pgfpathlineto{\pgfqpoint{1.631066in}{2.765079in}}%
\pgfpathlineto{\pgfqpoint{1.640312in}{2.809650in}}%
\pgfpathlineto{\pgfqpoint{1.649796in}{2.841326in}}%
\pgfpathlineto{\pgfqpoint{1.659621in}{2.864007in}}%
\pgfpathlineto{\pgfqpoint{1.669666in}{2.879908in}}%
\pgfpathlineto{\pgfqpoint{1.680601in}{2.891638in}}%
\pgfpathlineto{\pgfqpoint{1.692443in}{2.899947in}}%
\pgfpathlineto{\pgfqpoint{1.707009in}{2.906204in}}%
\pgfpathlineto{\pgfqpoint{1.724630in}{2.910341in}}%
\pgfpathlineto{\pgfqpoint{1.749538in}{2.913014in}}%
\pgfpathlineto{\pgfqpoint{1.795947in}{2.914478in}}%
\pgfpathlineto{\pgfqpoint{1.799891in}{2.913450in}}%
\pgfpathlineto{\pgfqpoint{1.813834in}{1.957154in}}%
\pgfpathlineto{\pgfqpoint{1.824670in}{1.399161in}}%
\pgfpathlineto{\pgfqpoint{1.832704in}{1.130119in}}%
\pgfpathlineto{\pgfqpoint{1.841125in}{0.945407in}}%
\pgfpathlineto{\pgfqpoint{1.849855in}{0.817653in}}%
\pgfpathlineto{\pgfqpoint{1.858819in}{0.728970in}}%
\pgfpathlineto{\pgfqpoint{1.867859in}{0.667968in}}%
\pgfpathlineto{\pgfqpoint{1.877135in}{0.624636in}}%
\pgfpathlineto{\pgfqpoint{1.886696in}{0.593760in}}%
\pgfpathlineto{\pgfqpoint{1.896434in}{0.571986in}}%
\pgfpathlineto{\pgfqpoint{1.906815in}{0.556142in}}%
\pgfpathlineto{\pgfqpoint{1.917844in}{0.544816in}}%
\pgfpathlineto{\pgfqpoint{1.930006in}{0.536706in}}%
\pgfpathlineto{\pgfqpoint{1.944464in}{0.530832in}}%
\pgfpathlineto{\pgfqpoint{1.963155in}{0.526666in}}%
\pgfpathlineto{\pgfqpoint{1.990564in}{0.523976in}}%
\pgfpathlineto{\pgfqpoint{2.041290in}{0.522658in}}%
\pgfpathlineto{\pgfqpoint{2.075452in}{0.522483in}}%
\pgfpathlineto{\pgfqpoint{2.075452in}{0.689044in}}%
\pgfpathlineto{\pgfqpoint{2.076992in}{0.679438in}}%
\pgfpathlineto{\pgfqpoint{2.086238in}{0.632722in}}%
\pgfpathlineto{\pgfqpoint{2.095689in}{0.599627in}}%
\pgfpathlineto{\pgfqpoint{2.105390in}{0.576121in}}%
\pgfpathlineto{\pgfqpoint{2.115382in}{0.559458in}}%
\pgfpathlineto{\pgfqpoint{2.126176in}{0.547241in}}%
\pgfpathlineto{\pgfqpoint{2.138023in}{0.538433in}}%
\pgfpathlineto{\pgfqpoint{2.151816in}{0.532065in}}%
\pgfpathlineto{\pgfqpoint{2.169215in}{0.527527in}}%
\pgfpathlineto{\pgfqpoint{2.193884in}{0.524489in}}%
\pgfpathlineto{\pgfqpoint{2.236244in}{0.522855in}}%
\pgfpathlineto{\pgfqpoint{2.341625in}{0.522455in}}%
\pgfpathlineto{\pgfqpoint{2.341627in}{0.522594in}}%
\pgfpathlineto{\pgfqpoint{2.355511in}{1.476596in}}%
\pgfpathlineto{\pgfqpoint{2.366820in}{2.059983in}}%
\pgfpathlineto{\pgfqpoint{2.374882in}{2.322197in}}%
\pgfpathlineto{\pgfqpoint{2.383329in}{2.502370in}}%
\pgfpathlineto{\pgfqpoint{2.392081in}{2.627028in}}%
\pgfpathlineto{\pgfqpoint{2.401057in}{2.713513in}}%
\pgfpathlineto{\pgfqpoint{2.410117in}{2.773081in}}%
\pgfpathlineto{\pgfqpoint{2.419399in}{2.815359in}}%
\pgfpathlineto{\pgfqpoint{2.428791in}{2.845010in}}%
\pgfpathlineto{\pgfqpoint{2.438511in}{2.866366in}}%
\pgfpathlineto{\pgfqpoint{2.448763in}{2.881771in}}%
\pgfpathlineto{\pgfqpoint{2.459912in}{2.893063in}}%
\pgfpathlineto{\pgfqpoint{2.472120in}{2.901043in}}%
\pgfpathlineto{\pgfqpoint{2.486485in}{2.906774in}}%
\pgfpathlineto{\pgfqpoint{2.504127in}{2.910610in}}%
\pgfpathlineto{\pgfqpoint{2.521334in}{2.912593in}}%
\pgfpathlineto{\pgfqpoint{2.521334in}{2.615281in}}%
\pgfpathlineto{\pgfqpoint{2.522876in}{2.633399in}}%
\pgfpathlineto{\pgfqpoint{2.531828in}{2.717593in}}%
\pgfpathlineto{\pgfqpoint{2.540896in}{2.775936in}}%
\pgfpathlineto{\pgfqpoint{2.550248in}{2.817608in}}%
\pgfpathlineto{\pgfqpoint{2.559671in}{2.846639in}}%
\pgfpathlineto{\pgfqpoint{2.569496in}{2.867681in}}%
\pgfpathlineto{\pgfqpoint{2.579547in}{2.882425in}}%
\pgfpathlineto{\pgfqpoint{2.591167in}{2.893869in}}%
\pgfpathlineto{\pgfqpoint{2.603584in}{2.901657in}}%
\pgfpathlineto{\pgfqpoint{2.618622in}{2.907321in}}%
\pgfpathlineto{\pgfqpoint{2.638083in}{2.911173in}}%
\pgfpathlineto{\pgfqpoint{2.669324in}{2.913653in}}%
\pgfpathlineto{\pgfqpoint{2.731842in}{2.914826in}}%
\pgfpathlineto{\pgfqpoint{2.803106in}{2.915028in}}%
\pgfpathlineto{\pgfqpoint{2.805999in}{2.698565in}}%
\pgfpathlineto{\pgfqpoint{2.818931in}{1.848481in}}%
\pgfpathlineto{\pgfqpoint{2.829086in}{1.351755in}}%
\pgfpathlineto{\pgfqpoint{2.837186in}{1.097586in}}%
\pgfpathlineto{\pgfqpoint{2.845657in}{0.922985in}}%
\pgfpathlineto{\pgfqpoint{2.854434in}{0.802003in}}%
\pgfpathlineto{\pgfqpoint{2.863396in}{0.718409in}}%
\pgfpathlineto{\pgfqpoint{2.872474in}{0.660474in}}%
\pgfpathlineto{\pgfqpoint{2.881812in}{0.619188in}}%
\pgfpathlineto{\pgfqpoint{2.891376in}{0.590013in}}%
\pgfpathlineto{\pgfqpoint{2.901197in}{0.569243in}}%
\pgfpathlineto{\pgfqpoint{2.911285in}{0.554638in}}%
\pgfpathlineto{\pgfqpoint{2.922956in}{0.543313in}}%
\pgfpathlineto{\pgfqpoint{2.935407in}{0.535612in}}%
\pgfpathlineto{\pgfqpoint{2.950298in}{0.530061in}}%
\pgfpathlineto{\pgfqpoint{2.967215in}{0.526534in}}%
\pgfpathlineto{\pgfqpoint{2.967215in}{0.799366in}}%
\pgfpathlineto{\pgfqpoint{2.968755in}{0.782755in}}%
\pgfpathlineto{\pgfqpoint{2.977721in}{0.705230in}}%
\pgfpathlineto{\pgfqpoint{2.986911in}{0.650798in}}%
\pgfpathlineto{\pgfqpoint{2.996224in}{0.612553in}}%
\pgfpathlineto{\pgfqpoint{3.005838in}{0.585291in}}%
\pgfpathlineto{\pgfqpoint{3.015825in}{0.565737in}}%
\pgfpathlineto{\pgfqpoint{3.026108in}{0.552015in}}%
\pgfpathlineto{\pgfqpoint{3.037231in}{0.542009in}}%
\pgfpathlineto{\pgfqpoint{3.050081in}{0.534617in}}%
\pgfpathlineto{\pgfqpoint{3.065101in}{0.529487in}}%
\pgfpathlineto{\pgfqpoint{3.087402in}{0.525546in}}%
\pgfpathlineto{\pgfqpoint{3.121612in}{0.523313in}}%
\pgfpathlineto{\pgfqpoint{3.193577in}{0.522477in}}%
\pgfpathlineto{\pgfqpoint{3.344870in}{0.523447in}}%
\pgfpathlineto{\pgfqpoint{3.358691in}{1.473454in}}%
\pgfpathlineto{\pgfqpoint{3.370014in}{2.058360in}}%
\pgfpathlineto{\pgfqpoint{3.378076in}{2.321159in}}%
\pgfpathlineto{\pgfqpoint{3.386516in}{2.501548in}}%
\pgfpathlineto{\pgfqpoint{3.395265in}{2.626424in}}%
\pgfpathlineto{\pgfqpoint{3.404224in}{2.712943in}}%
\pgfpathlineto{\pgfqpoint{3.413097in}{2.771662in}}%
\pgfpathlineto{\pgfqpoint{3.413097in}{2.588215in}}%
\pgfpathlineto{\pgfqpoint{3.414638in}{2.608090in}}%
\pgfpathlineto{\pgfqpoint{3.423582in}{2.700388in}}%
\pgfpathlineto{\pgfqpoint{3.424773in}{2.710183in}}%
\pgfpathlineto{\pgfqpoint{3.433849in}{2.770871in}}%
\pgfpathlineto{\pgfqpoint{3.443157in}{2.813923in}}%
\pgfpathlineto{\pgfqpoint{3.452574in}{2.844085in}}%
\pgfpathlineto{\pgfqpoint{3.462323in}{2.865777in}}%
\pgfpathlineto{\pgfqpoint{3.472567in}{2.881364in}}%
\pgfpathlineto{\pgfqpoint{3.484077in}{2.893086in}}%
\pgfpathlineto{\pgfqpoint{3.496137in}{2.900989in}}%
\pgfpathlineto{\pgfqpoint{3.510722in}{2.906783in}}%
\pgfpathlineto{\pgfqpoint{3.529529in}{2.910830in}}%
\pgfpathlineto{\pgfqpoint{3.557143in}{2.913348in}}%
\pgfpathlineto{\pgfqpoint{3.606987in}{2.914686in}}%
\pgfpathlineto{\pgfqpoint{3.806345in}{2.915018in}}%
\pgfpathlineto{\pgfqpoint{3.820105in}{1.969600in}}%
\pgfpathlineto{\pgfqpoint{3.831458in}{1.382649in}}%
\pgfpathlineto{\pgfqpoint{3.839514in}{1.118793in}}%
\pgfpathlineto{\pgfqpoint{3.847952in}{0.937632in}}%
\pgfpathlineto{\pgfqpoint{3.856699in}{0.812229in}}%
\pgfpathlineto{\pgfqpoint{3.858978in}{0.786818in}}%
\pgfpathlineto{\pgfqpoint{3.858978in}{0.844496in}}%
\pgfpathlineto{\pgfqpoint{3.860521in}{0.824911in}}%
\pgfpathlineto{\pgfqpoint{3.869455in}{0.734170in}}%
\pgfpathlineto{\pgfqpoint{3.878538in}{0.671308in}}%
\pgfpathlineto{\pgfqpoint{3.887744in}{0.627237in}}%
\pgfpathlineto{\pgfqpoint{3.897117in}{0.596056in}}%
\pgfpathlineto{\pgfqpoint{3.906873in}{0.573542in}}%
\pgfpathlineto{\pgfqpoint{3.917198in}{0.557263in}}%
\pgfpathlineto{\pgfqpoint{3.928159in}{0.545624in}}%
\pgfpathlineto{\pgfqpoint{3.940334in}{0.537207in}}%
\pgfpathlineto{\pgfqpoint{3.954557in}{0.531195in}}%
\pgfpathlineto{\pgfqpoint{3.973081in}{0.526878in}}%
\pgfpathlineto{\pgfqpoint{3.999625in}{0.524107in}}%
\pgfpathlineto{\pgfqpoint{4.047859in}{0.522704in}}%
\pgfpathlineto{\pgfqpoint{4.219814in}{0.522412in}}%
\pgfpathlineto{\pgfqpoint{4.304860in}{0.522412in}}%
\pgfpathlineto{\pgfqpoint{4.304860in}{0.538738in}}%
\pgfpathlineto{\pgfqpoint{4.306261in}{0.537928in}}%
\pgfpathlineto{\pgfqpoint{4.320544in}{0.531601in}}%
\pgfpathlineto{\pgfqpoint{4.337235in}{0.527662in}}%
\pgfpathlineto{\pgfqpoint{4.348089in}{0.526900in}}%
\pgfpathlineto{\pgfqpoint{4.348097in}{0.527439in}}%
\pgfpathlineto{\pgfqpoint{4.362108in}{1.487500in}}%
\pgfpathlineto{\pgfqpoint{4.373209in}{2.058405in}}%
\pgfpathlineto{\pgfqpoint{4.381270in}{2.321124in}}%
\pgfpathlineto{\pgfqpoint{4.389713in}{2.501581in}}%
\pgfpathlineto{\pgfqpoint{4.398463in}{2.626498in}}%
\pgfpathlineto{\pgfqpoint{4.407440in}{2.713182in}}%
\pgfpathlineto{\pgfqpoint{4.416541in}{2.773069in}}%
\pgfpathlineto{\pgfqpoint{4.425822in}{2.815350in}}%
\pgfpathlineto{\pgfqpoint{4.435215in}{2.845004in}}%
\pgfpathlineto{\pgfqpoint{4.444934in}{2.866361in}}%
\pgfpathlineto{\pgfqpoint{4.455186in}{2.881767in}}%
\pgfpathlineto{\pgfqpoint{4.466375in}{2.893089in}}%
\pgfpathlineto{\pgfqpoint{4.478691in}{2.901119in}}%
\pgfpathlineto{\pgfqpoint{4.493173in}{2.906840in}}%
\pgfpathlineto{\pgfqpoint{4.511697in}{2.910787in}}%
\pgfpathlineto{\pgfqpoint{4.538190in}{2.913201in}}%
\pgfpathlineto{\pgfqpoint{4.597933in}{2.914761in}}%
\pgfpathlineto{\pgfqpoint{4.750741in}{2.915074in}}%
\pgfpathlineto{\pgfqpoint{4.750741in}{2.915074in}}%
\pgfusepath{stroke}%
\end{pgfscope}%
\begin{pgfscope}%
\pgfpathrectangle{\pgfqpoint{0.537140in}{0.402778in}}{\pgfqpoint{4.414249in}{2.631944in}}%
\pgfusepath{clip}%
\pgfsetrectcap%
\pgfsetroundjoin%
\pgfsetlinewidth{1.505625pt}%
\definecolor{currentstroke}{rgb}{0.172549,0.627451,0.172549}%
\pgfsetstrokecolor{currentstroke}%
\pgfsetdash{}{0pt}%
\pgfpathmoveto{\pgfqpoint{0.737787in}{0.522412in}}%
\pgfpathlineto{\pgfqpoint{0.777921in}{0.522663in}}%
\pgfpathlineto{\pgfqpoint{0.819258in}{2.915088in}}%
\pgfpathlineto{\pgfqpoint{1.319323in}{2.915088in}}%
\pgfpathlineto{\pgfqpoint{1.319724in}{2.911445in}}%
\pgfpathlineto{\pgfqpoint{1.361061in}{0.522412in}}%
\pgfpathlineto{\pgfqpoint{1.780858in}{0.522412in}}%
\pgfpathlineto{\pgfqpoint{1.781661in}{0.552874in}}%
\pgfpathlineto{\pgfqpoint{1.822597in}{2.915088in}}%
\pgfpathlineto{\pgfqpoint{2.322661in}{2.915088in}}%
\pgfpathlineto{\pgfqpoint{2.323865in}{2.857305in}}%
\pgfpathlineto{\pgfqpoint{2.364400in}{0.522412in}}%
\pgfpathlineto{\pgfqpoint{2.784197in}{0.522412in}}%
\pgfpathlineto{\pgfqpoint{2.786204in}{0.630943in}}%
\pgfpathlineto{\pgfqpoint{2.825936in}{2.915088in}}%
\pgfpathlineto{\pgfqpoint{3.326000in}{2.915088in}}%
\pgfpathlineto{\pgfqpoint{3.329612in}{2.707448in}}%
\pgfpathlineto{\pgfqpoint{3.367739in}{0.522412in}}%
\pgfpathlineto{\pgfqpoint{3.787536in}{0.522412in}}%
\pgfpathlineto{\pgfqpoint{3.793957in}{0.900446in}}%
\pgfpathlineto{\pgfqpoint{3.829275in}{2.915088in}}%
\pgfpathlineto{\pgfqpoint{4.329339in}{2.915088in}}%
\pgfpathlineto{\pgfqpoint{4.353419in}{1.480773in}}%
\pgfpathlineto{\pgfqpoint{4.370676in}{0.522412in}}%
\pgfpathlineto{\pgfqpoint{4.750741in}{0.522412in}}%
\pgfpathlineto{\pgfqpoint{4.750741in}{0.522412in}}%
\pgfusepath{stroke}%
\end{pgfscope}%
\begin{pgfscope}%
\pgfsetrectcap%
\pgfsetmiterjoin%
\pgfsetlinewidth{0.803000pt}%
\definecolor{currentstroke}{rgb}{0.000000,0.000000,0.000000}%
\pgfsetstrokecolor{currentstroke}%
\pgfsetdash{}{0pt}%
\pgfpathmoveto{\pgfqpoint{0.537140in}{0.402778in}}%
\pgfpathlineto{\pgfqpoint{0.537140in}{3.034722in}}%
\pgfusepath{stroke}%
\end{pgfscope}%
\begin{pgfscope}%
\pgfsetrectcap%
\pgfsetmiterjoin%
\pgfsetlinewidth{0.803000pt}%
\definecolor{currentstroke}{rgb}{0.000000,0.000000,0.000000}%
\pgfsetstrokecolor{currentstroke}%
\pgfsetdash{}{0pt}%
\pgfpathmoveto{\pgfqpoint{4.951389in}{0.402778in}}%
\pgfpathlineto{\pgfqpoint{4.951389in}{3.034722in}}%
\pgfusepath{stroke}%
\end{pgfscope}%
\begin{pgfscope}%
\pgfsetrectcap%
\pgfsetmiterjoin%
\pgfsetlinewidth{0.803000pt}%
\definecolor{currentstroke}{rgb}{0.000000,0.000000,0.000000}%
\pgfsetstrokecolor{currentstroke}%
\pgfsetdash{}{0pt}%
\pgfpathmoveto{\pgfqpoint{0.537140in}{0.402778in}}%
\pgfpathlineto{\pgfqpoint{4.951389in}{0.402778in}}%
\pgfusepath{stroke}%
\end{pgfscope}%
\begin{pgfscope}%
\pgfsetrectcap%
\pgfsetmiterjoin%
\pgfsetlinewidth{0.803000pt}%
\definecolor{currentstroke}{rgb}{0.000000,0.000000,0.000000}%
\pgfsetstrokecolor{currentstroke}%
\pgfsetdash{}{0pt}%
\pgfpathmoveto{\pgfqpoint{0.537140in}{3.034722in}}%
\pgfpathlineto{\pgfqpoint{4.951389in}{3.034722in}}%
\pgfusepath{stroke}%
\end{pgfscope}%
\begin{pgfscope}%
\pgfsetbuttcap%
\pgfsetmiterjoin%
\definecolor{currentfill}{rgb}{1.000000,1.000000,1.000000}%
\pgfsetfillcolor{currentfill}%
\pgfsetfillopacity{0.800000}%
\pgfsetlinewidth{1.003750pt}%
\definecolor{currentstroke}{rgb}{0.800000,0.800000,0.800000}%
\pgfsetstrokecolor{currentstroke}%
\pgfsetstrokeopacity{0.800000}%
\pgfsetdash{}{0pt}%
\pgfpathmoveto{\pgfqpoint{3.946872in}{0.472222in}}%
\pgfpathlineto{\pgfqpoint{4.854167in}{0.472222in}}%
\pgfpathquadraticcurveto{\pgfqpoint{4.881944in}{0.472222in}}{\pgfqpoint{4.881944in}{0.500000in}}%
\pgfpathlineto{\pgfqpoint{4.881944in}{1.075493in}}%
\pgfpathquadraticcurveto{\pgfqpoint{4.881944in}{1.103271in}}{\pgfqpoint{4.854167in}{1.103271in}}%
\pgfpathlineto{\pgfqpoint{3.946872in}{1.103271in}}%
\pgfpathquadraticcurveto{\pgfqpoint{3.919094in}{1.103271in}}{\pgfqpoint{3.919094in}{1.075493in}}%
\pgfpathlineto{\pgfqpoint{3.919094in}{0.500000in}}%
\pgfpathquadraticcurveto{\pgfqpoint{3.919094in}{0.472222in}}{\pgfqpoint{3.946872in}{0.472222in}}%
\pgfpathclose%
\pgfusepath{stroke,fill}%
\end{pgfscope}%
\begin{pgfscope}%
\pgfsetrectcap%
\pgfsetroundjoin%
\pgfsetlinewidth{1.505625pt}%
\definecolor{currentstroke}{rgb}{0.121569,0.466667,0.705882}%
\pgfsetstrokecolor{currentstroke}%
\pgfsetdash{}{0pt}%
\pgfpathmoveto{\pgfqpoint{3.974650in}{0.999104in}}%
\pgfpathlineto{\pgfqpoint{4.252428in}{0.999104in}}%
\pgfusepath{stroke}%
\end{pgfscope}%
\begin{pgfscope}%
\pgftext[x=4.363539in,y=0.950493in,left,base]{\rmfamily\fontsize{10.000000}{12.000000}\selectfont exakt}%
\end{pgfscope}%
\begin{pgfscope}%
\pgfsetrectcap%
\pgfsetroundjoin%
\pgfsetlinewidth{1.505625pt}%
\definecolor{currentstroke}{rgb}{1.000000,0.498039,0.054902}%
\pgfsetstrokecolor{currentstroke}%
\pgfsetdash{}{0pt}%
\pgfpathmoveto{\pgfqpoint{3.974650in}{0.802734in}}%
\pgfpathlineto{\pgfqpoint{4.252428in}{0.802734in}}%
\pgfusepath{stroke}%
\end{pgfscope}%
\begin{pgfscope}%
\pgftext[x=4.363539in,y=0.754123in,left,base]{\rmfamily\fontsize{10.000000}{12.000000}\selectfont k=1}%
\end{pgfscope}%
\begin{pgfscope}%
\pgfsetrectcap%
\pgfsetroundjoin%
\pgfsetlinewidth{1.505625pt}%
\definecolor{currentstroke}{rgb}{0.172549,0.627451,0.172549}%
\pgfsetstrokecolor{currentstroke}%
\pgfsetdash{}{0pt}%
\pgfpathmoveto{\pgfqpoint{3.974650in}{0.606364in}}%
\pgfpathlineto{\pgfqpoint{4.252428in}{0.606364in}}%
\pgfusepath{stroke}%
\end{pgfscope}%
\begin{pgfscope}%
\pgftext[x=4.363539in,y=0.557753in,left,base]{\rmfamily\fontsize{10.000000}{12.000000}\selectfont Eingang}%
\end{pgfscope}%
\end{pgfpicture}%
\makeatother%
\endgroup%

    \caption{Lösung des Schaltungssimulators nach einer Iteration für den Kondensator(oben) und den Invertierer(unten).}
    \label{fig:sol_sim}
\end{figure}
\begin{figure}[ht]
    \centering
        %% Creator: Matplotlib, PGF backend
%%
%% To include the figure in your LaTeX document, write
%%   \input{<filename>.pgf}
%%
%% Make sure the required packages are loaded in your preamble
%%   \usepackage{pgf}
%%
%% Figures using additional raster images can only be included by \input if
%% they are in the same directory as the main LaTeX file. For loading figures
%% from other directories you can use the `import` package
%%   \usepackage{import}
%% and then include the figures with
%%   \import{<path to file>}{<filename>.pgf}
%%
%% Matplotlib used the following preamble
%%   \usepackage{fontspec}
%%   \setmainfont{Times New Roman}
%%   \setsansfont{Lucida Grande}
%%   \setmonofont{Andale Mono}
%%
\begingroup%
\makeatletter%
\begin{pgfpicture}%
\pgfpathrectangle{\pgfpointorigin}{\pgfqpoint{5.000000in}{3.388889in}}%
\pgfusepath{use as bounding box, clip}%
\begin{pgfscope}%
\pgfsetbuttcap%
\pgfsetmiterjoin%
\definecolor{currentfill}{rgb}{1.000000,1.000000,1.000000}%
\pgfsetfillcolor{currentfill}%
\pgfsetlinewidth{0.000000pt}%
\definecolor{currentstroke}{rgb}{1.000000,1.000000,1.000000}%
\pgfsetstrokecolor{currentstroke}%
\pgfsetdash{}{0pt}%
\pgfpathmoveto{\pgfqpoint{0.000000in}{0.000000in}}%
\pgfpathlineto{\pgfqpoint{5.000000in}{0.000000in}}%
\pgfpathlineto{\pgfqpoint{5.000000in}{3.388889in}}%
\pgfpathlineto{\pgfqpoint{0.000000in}{3.388889in}}%
\pgfpathclose%
\pgfusepath{fill}%
\end{pgfscope}%
\begin{pgfscope}%
\pgfsetbuttcap%
\pgfsetmiterjoin%
\definecolor{currentfill}{rgb}{1.000000,1.000000,1.000000}%
\pgfsetfillcolor{currentfill}%
\pgfsetlinewidth{0.000000pt}%
\definecolor{currentstroke}{rgb}{0.000000,0.000000,0.000000}%
\pgfsetstrokecolor{currentstroke}%
\pgfsetstrokeopacity{0.000000}%
\pgfsetdash{}{0pt}%
\pgfpathmoveto{\pgfqpoint{0.626312in}{0.402778in}}%
\pgfpathlineto{\pgfqpoint{4.880181in}{0.402778in}}%
\pgfpathlineto{\pgfqpoint{4.880181in}{3.340278in}}%
\pgfpathlineto{\pgfqpoint{0.626312in}{3.340278in}}%
\pgfpathclose%
\pgfusepath{fill}%
\end{pgfscope}%
\begin{pgfscope}%
\pgfsetbuttcap%
\pgfsetroundjoin%
\definecolor{currentfill}{rgb}{0.000000,0.000000,0.000000}%
\pgfsetfillcolor{currentfill}%
\pgfsetlinewidth{0.803000pt}%
\definecolor{currentstroke}{rgb}{0.000000,0.000000,0.000000}%
\pgfsetstrokecolor{currentstroke}%
\pgfsetdash{}{0pt}%
\pgfsys@defobject{currentmarker}{\pgfqpoint{0.000000in}{-0.048611in}}{\pgfqpoint{0.000000in}{0.000000in}}{%
\pgfpathmoveto{\pgfqpoint{0.000000in}{0.000000in}}%
\pgfpathlineto{\pgfqpoint{0.000000in}{-0.048611in}}%
\pgfusepath{stroke,fill}%
}%
\begin{pgfscope}%
\pgfsys@transformshift{0.819670in}{0.402778in}%
\pgfsys@useobject{currentmarker}{}%
\end{pgfscope}%
\end{pgfscope}%
\begin{pgfscope}%
\pgftext[x=0.819670in,y=0.305556in,,top]{\rmfamily\fontsize{10.000000}{12.000000}\selectfont \(\displaystyle 0.0000000\)}%
\end{pgfscope}%
\begin{pgfscope}%
\pgfsetbuttcap%
\pgfsetroundjoin%
\definecolor{currentfill}{rgb}{0.000000,0.000000,0.000000}%
\pgfsetfillcolor{currentfill}%
\pgfsetlinewidth{0.803000pt}%
\definecolor{currentstroke}{rgb}{0.000000,0.000000,0.000000}%
\pgfsetstrokecolor{currentstroke}%
\pgfsetdash{}{0pt}%
\pgfsys@defobject{currentmarker}{\pgfqpoint{0.000000in}{-0.048611in}}{\pgfqpoint{0.000000in}{0.000000in}}{%
\pgfpathmoveto{\pgfqpoint{0.000000in}{0.000000in}}%
\pgfpathlineto{\pgfqpoint{0.000000in}{-0.048611in}}%
\pgfusepath{stroke,fill}%
}%
\begin{pgfscope}%
\pgfsys@transformshift{1.593099in}{0.402778in}%
\pgfsys@useobject{currentmarker}{}%
\end{pgfscope}%
\end{pgfscope}%
\begin{pgfscope}%
\pgftext[x=1.593099in,y=0.305556in,,top]{\rmfamily\fontsize{10.000000}{12.000000}\selectfont \(\displaystyle 0.0000002\)}%
\end{pgfscope}%
\begin{pgfscope}%
\pgfsetbuttcap%
\pgfsetroundjoin%
\definecolor{currentfill}{rgb}{0.000000,0.000000,0.000000}%
\pgfsetfillcolor{currentfill}%
\pgfsetlinewidth{0.803000pt}%
\definecolor{currentstroke}{rgb}{0.000000,0.000000,0.000000}%
\pgfsetstrokecolor{currentstroke}%
\pgfsetdash{}{0pt}%
\pgfsys@defobject{currentmarker}{\pgfqpoint{0.000000in}{-0.048611in}}{\pgfqpoint{0.000000in}{0.000000in}}{%
\pgfpathmoveto{\pgfqpoint{0.000000in}{0.000000in}}%
\pgfpathlineto{\pgfqpoint{0.000000in}{-0.048611in}}%
\pgfusepath{stroke,fill}%
}%
\begin{pgfscope}%
\pgfsys@transformshift{2.366529in}{0.402778in}%
\pgfsys@useobject{currentmarker}{}%
\end{pgfscope}%
\end{pgfscope}%
\begin{pgfscope}%
\pgftext[x=2.366529in,y=0.305556in,,top]{\rmfamily\fontsize{10.000000}{12.000000}\selectfont \(\displaystyle 0.0000004\)}%
\end{pgfscope}%
\begin{pgfscope}%
\pgfsetbuttcap%
\pgfsetroundjoin%
\definecolor{currentfill}{rgb}{0.000000,0.000000,0.000000}%
\pgfsetfillcolor{currentfill}%
\pgfsetlinewidth{0.803000pt}%
\definecolor{currentstroke}{rgb}{0.000000,0.000000,0.000000}%
\pgfsetstrokecolor{currentstroke}%
\pgfsetdash{}{0pt}%
\pgfsys@defobject{currentmarker}{\pgfqpoint{0.000000in}{-0.048611in}}{\pgfqpoint{0.000000in}{0.000000in}}{%
\pgfpathmoveto{\pgfqpoint{0.000000in}{0.000000in}}%
\pgfpathlineto{\pgfqpoint{0.000000in}{-0.048611in}}%
\pgfusepath{stroke,fill}%
}%
\begin{pgfscope}%
\pgfsys@transformshift{3.139959in}{0.402778in}%
\pgfsys@useobject{currentmarker}{}%
\end{pgfscope}%
\end{pgfscope}%
\begin{pgfscope}%
\pgftext[x=3.139959in,y=0.305556in,,top]{\rmfamily\fontsize{10.000000}{12.000000}\selectfont \(\displaystyle 0.0000006\)}%
\end{pgfscope}%
\begin{pgfscope}%
\pgfsetbuttcap%
\pgfsetroundjoin%
\definecolor{currentfill}{rgb}{0.000000,0.000000,0.000000}%
\pgfsetfillcolor{currentfill}%
\pgfsetlinewidth{0.803000pt}%
\definecolor{currentstroke}{rgb}{0.000000,0.000000,0.000000}%
\pgfsetstrokecolor{currentstroke}%
\pgfsetdash{}{0pt}%
\pgfsys@defobject{currentmarker}{\pgfqpoint{0.000000in}{-0.048611in}}{\pgfqpoint{0.000000in}{0.000000in}}{%
\pgfpathmoveto{\pgfqpoint{0.000000in}{0.000000in}}%
\pgfpathlineto{\pgfqpoint{0.000000in}{-0.048611in}}%
\pgfusepath{stroke,fill}%
}%
\begin{pgfscope}%
\pgfsys@transformshift{3.913389in}{0.402778in}%
\pgfsys@useobject{currentmarker}{}%
\end{pgfscope}%
\end{pgfscope}%
\begin{pgfscope}%
\pgftext[x=3.913389in,y=0.305556in,,top]{\rmfamily\fontsize{10.000000}{12.000000}\selectfont \(\displaystyle 0.0000008\)}%
\end{pgfscope}%
\begin{pgfscope}%
\pgfsetbuttcap%
\pgfsetroundjoin%
\definecolor{currentfill}{rgb}{0.000000,0.000000,0.000000}%
\pgfsetfillcolor{currentfill}%
\pgfsetlinewidth{0.803000pt}%
\definecolor{currentstroke}{rgb}{0.000000,0.000000,0.000000}%
\pgfsetstrokecolor{currentstroke}%
\pgfsetdash{}{0pt}%
\pgfsys@defobject{currentmarker}{\pgfqpoint{0.000000in}{-0.048611in}}{\pgfqpoint{0.000000in}{0.000000in}}{%
\pgfpathmoveto{\pgfqpoint{0.000000in}{0.000000in}}%
\pgfpathlineto{\pgfqpoint{0.000000in}{-0.048611in}}%
\pgfusepath{stroke,fill}%
}%
\begin{pgfscope}%
\pgfsys@transformshift{4.686819in}{0.402778in}%
\pgfsys@useobject{currentmarker}{}%
\end{pgfscope}%
\end{pgfscope}%
\begin{pgfscope}%
\pgftext[x=4.686819in,y=0.305556in,,top]{\rmfamily\fontsize{10.000000}{12.000000}\selectfont \(\displaystyle 0.0000010\)}%
\end{pgfscope}%
\begin{pgfscope}%
\pgftext[x=2.753246in,y=0.123861in,,top]{\rmfamily\fontsize{10.000000}{12.000000}\selectfont t}%
\end{pgfscope}%
\begin{pgfscope}%
\pgfsetbuttcap%
\pgfsetroundjoin%
\definecolor{currentfill}{rgb}{0.000000,0.000000,0.000000}%
\pgfsetfillcolor{currentfill}%
\pgfsetlinewidth{0.803000pt}%
\definecolor{currentstroke}{rgb}{0.000000,0.000000,0.000000}%
\pgfsetstrokecolor{currentstroke}%
\pgfsetdash{}{0pt}%
\pgfsys@defobject{currentmarker}{\pgfqpoint{-0.048611in}{0.000000in}}{\pgfqpoint{0.000000in}{0.000000in}}{%
\pgfpathmoveto{\pgfqpoint{0.000000in}{0.000000in}}%
\pgfpathlineto{\pgfqpoint{-0.048611in}{0.000000in}}%
\pgfusepath{stroke,fill}%
}%
\begin{pgfscope}%
\pgfsys@transformshift{0.626312in}{0.803972in}%
\pgfsys@useobject{currentmarker}{}%
\end{pgfscope}%
\end{pgfscope}%
\begin{pgfscope}%
\pgftext[x=0.185724in,y=0.755754in,left,base]{\rmfamily\fontsize{10.000000}{12.000000}\selectfont \(\displaystyle 10^{-13}\)}%
\end{pgfscope}%
\begin{pgfscope}%
\pgfsetbuttcap%
\pgfsetroundjoin%
\definecolor{currentfill}{rgb}{0.000000,0.000000,0.000000}%
\pgfsetfillcolor{currentfill}%
\pgfsetlinewidth{0.803000pt}%
\definecolor{currentstroke}{rgb}{0.000000,0.000000,0.000000}%
\pgfsetstrokecolor{currentstroke}%
\pgfsetdash{}{0pt}%
\pgfsys@defobject{currentmarker}{\pgfqpoint{-0.048611in}{0.000000in}}{\pgfqpoint{0.000000in}{0.000000in}}{%
\pgfpathmoveto{\pgfqpoint{0.000000in}{0.000000in}}%
\pgfpathlineto{\pgfqpoint{-0.048611in}{0.000000in}}%
\pgfusepath{stroke,fill}%
}%
\begin{pgfscope}%
\pgfsys@transformshift{0.626312in}{1.215337in}%
\pgfsys@useobject{currentmarker}{}%
\end{pgfscope}%
\end{pgfscope}%
\begin{pgfscope}%
\pgftext[x=0.185724in,y=1.167120in,left,base]{\rmfamily\fontsize{10.000000}{12.000000}\selectfont \(\displaystyle 10^{-11}\)}%
\end{pgfscope}%
\begin{pgfscope}%
\pgfsetbuttcap%
\pgfsetroundjoin%
\definecolor{currentfill}{rgb}{0.000000,0.000000,0.000000}%
\pgfsetfillcolor{currentfill}%
\pgfsetlinewidth{0.803000pt}%
\definecolor{currentstroke}{rgb}{0.000000,0.000000,0.000000}%
\pgfsetstrokecolor{currentstroke}%
\pgfsetdash{}{0pt}%
\pgfsys@defobject{currentmarker}{\pgfqpoint{-0.048611in}{0.000000in}}{\pgfqpoint{0.000000in}{0.000000in}}{%
\pgfpathmoveto{\pgfqpoint{0.000000in}{0.000000in}}%
\pgfpathlineto{\pgfqpoint{-0.048611in}{0.000000in}}%
\pgfusepath{stroke,fill}%
}%
\begin{pgfscope}%
\pgfsys@transformshift{0.626312in}{1.626703in}%
\pgfsys@useobject{currentmarker}{}%
\end{pgfscope}%
\end{pgfscope}%
\begin{pgfscope}%
\pgftext[x=0.241087in,y=1.578485in,left,base]{\rmfamily\fontsize{10.000000}{12.000000}\selectfont \(\displaystyle 10^{-9}\)}%
\end{pgfscope}%
\begin{pgfscope}%
\pgfsetbuttcap%
\pgfsetroundjoin%
\definecolor{currentfill}{rgb}{0.000000,0.000000,0.000000}%
\pgfsetfillcolor{currentfill}%
\pgfsetlinewidth{0.803000pt}%
\definecolor{currentstroke}{rgb}{0.000000,0.000000,0.000000}%
\pgfsetstrokecolor{currentstroke}%
\pgfsetdash{}{0pt}%
\pgfsys@defobject{currentmarker}{\pgfqpoint{-0.048611in}{0.000000in}}{\pgfqpoint{0.000000in}{0.000000in}}{%
\pgfpathmoveto{\pgfqpoint{0.000000in}{0.000000in}}%
\pgfpathlineto{\pgfqpoint{-0.048611in}{0.000000in}}%
\pgfusepath{stroke,fill}%
}%
\begin{pgfscope}%
\pgfsys@transformshift{0.626312in}{2.038069in}%
\pgfsys@useobject{currentmarker}{}%
\end{pgfscope}%
\end{pgfscope}%
\begin{pgfscope}%
\pgftext[x=0.241087in,y=1.989851in,left,base]{\rmfamily\fontsize{10.000000}{12.000000}\selectfont \(\displaystyle 10^{-7}\)}%
\end{pgfscope}%
\begin{pgfscope}%
\pgfsetbuttcap%
\pgfsetroundjoin%
\definecolor{currentfill}{rgb}{0.000000,0.000000,0.000000}%
\pgfsetfillcolor{currentfill}%
\pgfsetlinewidth{0.803000pt}%
\definecolor{currentstroke}{rgb}{0.000000,0.000000,0.000000}%
\pgfsetstrokecolor{currentstroke}%
\pgfsetdash{}{0pt}%
\pgfsys@defobject{currentmarker}{\pgfqpoint{-0.048611in}{0.000000in}}{\pgfqpoint{0.000000in}{0.000000in}}{%
\pgfpathmoveto{\pgfqpoint{0.000000in}{0.000000in}}%
\pgfpathlineto{\pgfqpoint{-0.048611in}{0.000000in}}%
\pgfusepath{stroke,fill}%
}%
\begin{pgfscope}%
\pgfsys@transformshift{0.626312in}{2.449435in}%
\pgfsys@useobject{currentmarker}{}%
\end{pgfscope}%
\end{pgfscope}%
\begin{pgfscope}%
\pgftext[x=0.241087in,y=2.401217in,left,base]{\rmfamily\fontsize{10.000000}{12.000000}\selectfont \(\displaystyle 10^{-5}\)}%
\end{pgfscope}%
\begin{pgfscope}%
\pgfsetbuttcap%
\pgfsetroundjoin%
\definecolor{currentfill}{rgb}{0.000000,0.000000,0.000000}%
\pgfsetfillcolor{currentfill}%
\pgfsetlinewidth{0.803000pt}%
\definecolor{currentstroke}{rgb}{0.000000,0.000000,0.000000}%
\pgfsetstrokecolor{currentstroke}%
\pgfsetdash{}{0pt}%
\pgfsys@defobject{currentmarker}{\pgfqpoint{-0.048611in}{0.000000in}}{\pgfqpoint{0.000000in}{0.000000in}}{%
\pgfpathmoveto{\pgfqpoint{0.000000in}{0.000000in}}%
\pgfpathlineto{\pgfqpoint{-0.048611in}{0.000000in}}%
\pgfusepath{stroke,fill}%
}%
\begin{pgfscope}%
\pgfsys@transformshift{0.626312in}{2.860801in}%
\pgfsys@useobject{currentmarker}{}%
\end{pgfscope}%
\end{pgfscope}%
\begin{pgfscope}%
\pgftext[x=0.241087in,y=2.812583in,left,base]{\rmfamily\fontsize{10.000000}{12.000000}\selectfont \(\displaystyle 10^{-3}\)}%
\end{pgfscope}%
\begin{pgfscope}%
\pgfsetbuttcap%
\pgfsetroundjoin%
\definecolor{currentfill}{rgb}{0.000000,0.000000,0.000000}%
\pgfsetfillcolor{currentfill}%
\pgfsetlinewidth{0.803000pt}%
\definecolor{currentstroke}{rgb}{0.000000,0.000000,0.000000}%
\pgfsetstrokecolor{currentstroke}%
\pgfsetdash{}{0pt}%
\pgfsys@defobject{currentmarker}{\pgfqpoint{-0.048611in}{0.000000in}}{\pgfqpoint{0.000000in}{0.000000in}}{%
\pgfpathmoveto{\pgfqpoint{0.000000in}{0.000000in}}%
\pgfpathlineto{\pgfqpoint{-0.048611in}{0.000000in}}%
\pgfusepath{stroke,fill}%
}%
\begin{pgfscope}%
\pgfsys@transformshift{0.626312in}{3.272166in}%
\pgfsys@useobject{currentmarker}{}%
\end{pgfscope}%
\end{pgfscope}%
\begin{pgfscope}%
\pgftext[x=0.241087in,y=3.223949in,left,base]{\rmfamily\fontsize{10.000000}{12.000000}\selectfont \(\displaystyle 10^{-1}\)}%
\end{pgfscope}%
\begin{pgfscope}%
\pgftext[x=0.130169in,y=1.871528in,,bottom,rotate=90.000000]{\rmfamily\fontsize{10.000000}{12.000000}\selectfont E(t)}%
\end{pgfscope}%
\begin{pgfscope}%
\pgfpathrectangle{\pgfqpoint{0.626312in}{0.402778in}}{\pgfqpoint{4.253869in}{2.937500in}}%
\pgfusepath{clip}%
\pgfsetrectcap%
\pgfsetroundjoin%
\pgfsetlinewidth{1.505625pt}%
\definecolor{currentstroke}{rgb}{0.121569,0.466667,0.705882}%
\pgfsetstrokecolor{currentstroke}%
\pgfsetdash{}{0pt}%
\pgfpathmoveto{\pgfqpoint{0.819670in}{0.388889in}}%
\pgfpathlineto{\pgfqpoint{0.821208in}{2.299693in}}%
\pgfpathlineto{\pgfqpoint{0.826207in}{2.389202in}}%
\pgfpathlineto{\pgfqpoint{0.837261in}{2.465437in}}%
\pgfpathlineto{\pgfqpoint{0.855773in}{2.521284in}}%
\pgfpathlineto{\pgfqpoint{0.876637in}{2.557001in}}%
\pgfpathlineto{\pgfqpoint{0.881591in}{2.563052in}}%
\pgfpathlineto{\pgfqpoint{0.884738in}{2.566799in}}%
\pgfpathlineto{\pgfqpoint{0.893617in}{2.575637in}}%
\pgfpathlineto{\pgfqpoint{0.897605in}{2.579282in}}%
\pgfpathlineto{\pgfqpoint{0.910090in}{2.589478in}}%
\pgfpathlineto{\pgfqpoint{0.913430in}{2.591840in}}%
\pgfpathlineto{\pgfqpoint{0.923807in}{2.598764in}}%
\pgfpathlineto{\pgfqpoint{0.932040in}{2.603502in}}%
\pgfpathlineto{\pgfqpoint{0.941691in}{2.608713in}}%
\pgfpathlineto{\pgfqpoint{0.969837in}{2.620648in}}%
\pgfpathlineto{\pgfqpoint{0.976664in}{2.623096in}}%
\pgfpathlineto{\pgfqpoint{0.981650in}{2.624531in}}%
\pgfpathlineto{\pgfqpoint{1.012143in}{2.632851in}}%
\pgfpathlineto{\pgfqpoint{1.040651in}{2.638514in}}%
\pgfpathlineto{\pgfqpoint{1.090076in}{2.645214in}}%
\pgfpathlineto{\pgfqpoint{1.136757in}{2.648822in}}%
\pgfpathlineto{\pgfqpoint{1.141936in}{2.649032in}}%
\pgfpathlineto{\pgfqpoint{1.141936in}{3.206755in}}%
\pgfpathlineto{\pgfqpoint{1.143474in}{3.206401in}}%
\pgfpathlineto{\pgfqpoint{1.464198in}{3.132510in}}%
\pgfpathlineto{\pgfqpoint{1.464198in}{3.198756in}}%
\pgfpathlineto{\pgfqpoint{1.465737in}{3.198402in}}%
\pgfpathlineto{\pgfqpoint{1.786461in}{3.124419in}}%
\pgfpathlineto{\pgfqpoint{1.786461in}{3.164317in}}%
\pgfpathlineto{\pgfqpoint{1.787999in}{3.163962in}}%
\pgfpathlineto{\pgfqpoint{2.108723in}{3.089951in}}%
\pgfpathlineto{\pgfqpoint{2.108723in}{3.118798in}}%
\pgfpathlineto{\pgfqpoint{2.110262in}{3.118443in}}%
\pgfpathlineto{\pgfqpoint{2.430986in}{3.044417in}}%
\pgfpathlineto{\pgfqpoint{2.430986in}{3.068125in}}%
\pgfpathlineto{\pgfqpoint{2.432524in}{3.067770in}}%
\pgfpathlineto{\pgfqpoint{2.753248in}{2.993735in}}%
\pgfpathlineto{\pgfqpoint{2.753248in}{3.014149in}}%
\pgfpathlineto{\pgfqpoint{2.754787in}{3.013794in}}%
\pgfpathlineto{\pgfqpoint{3.075511in}{2.939753in}}%
\pgfpathlineto{\pgfqpoint{3.075511in}{2.957885in}}%
\pgfpathlineto{\pgfqpoint{3.077049in}{2.957530in}}%
\pgfpathlineto{\pgfqpoint{3.397773in}{2.883484in}}%
\pgfpathlineto{\pgfqpoint{3.397773in}{2.899947in}}%
\pgfpathlineto{\pgfqpoint{3.399312in}{2.899592in}}%
\pgfpathlineto{\pgfqpoint{3.720036in}{2.825543in}}%
\pgfpathlineto{\pgfqpoint{3.720036in}{2.840737in}}%
\pgfpathlineto{\pgfqpoint{3.721574in}{2.840382in}}%
\pgfpathlineto{\pgfqpoint{4.042298in}{2.766330in}}%
\pgfpathlineto{\pgfqpoint{4.042298in}{2.780532in}}%
\pgfpathlineto{\pgfqpoint{4.043837in}{2.780177in}}%
\pgfpathlineto{\pgfqpoint{4.364561in}{2.706124in}}%
\pgfpathlineto{\pgfqpoint{4.364561in}{2.719530in}}%
\pgfpathlineto{\pgfqpoint{4.366099in}{2.719175in}}%
\pgfpathlineto{\pgfqpoint{4.686823in}{2.645121in}}%
\pgfpathlineto{\pgfqpoint{4.686823in}{2.645121in}}%
\pgfusepath{stroke}%
\end{pgfscope}%
\begin{pgfscope}%
\pgfpathrectangle{\pgfqpoint{0.626312in}{0.402778in}}{\pgfqpoint{4.253869in}{2.937500in}}%
\pgfusepath{clip}%
\pgfsetrectcap%
\pgfsetroundjoin%
\pgfsetlinewidth{1.505625pt}%
\definecolor{currentstroke}{rgb}{1.000000,0.498039,0.054902}%
\pgfsetstrokecolor{currentstroke}%
\pgfsetdash{}{0pt}%
\pgfpathmoveto{\pgfqpoint{0.819670in}{0.388889in}}%
\pgfpathlineto{\pgfqpoint{0.821208in}{2.299693in}}%
\pgfpathlineto{\pgfqpoint{0.826207in}{2.389202in}}%
\pgfpathlineto{\pgfqpoint{0.837261in}{2.465437in}}%
\pgfpathlineto{\pgfqpoint{0.855773in}{2.521284in}}%
\pgfpathlineto{\pgfqpoint{0.876637in}{2.557001in}}%
\pgfpathlineto{\pgfqpoint{0.881591in}{2.563052in}}%
\pgfpathlineto{\pgfqpoint{0.884738in}{2.566799in}}%
\pgfpathlineto{\pgfqpoint{0.893617in}{2.575637in}}%
\pgfpathlineto{\pgfqpoint{0.897605in}{2.579282in}}%
\pgfpathlineto{\pgfqpoint{0.910090in}{2.589478in}}%
\pgfpathlineto{\pgfqpoint{0.913430in}{2.591840in}}%
\pgfpathlineto{\pgfqpoint{0.923807in}{2.598764in}}%
\pgfpathlineto{\pgfqpoint{0.932040in}{2.603502in}}%
\pgfpathlineto{\pgfqpoint{0.941691in}{2.608713in}}%
\pgfpathlineto{\pgfqpoint{0.969837in}{2.620648in}}%
\pgfpathlineto{\pgfqpoint{0.976664in}{2.623096in}}%
\pgfpathlineto{\pgfqpoint{0.981650in}{2.624531in}}%
\pgfpathlineto{\pgfqpoint{1.012143in}{2.632851in}}%
\pgfpathlineto{\pgfqpoint{1.040651in}{2.638514in}}%
\pgfpathlineto{\pgfqpoint{1.090076in}{2.645214in}}%
\pgfpathlineto{\pgfqpoint{1.136757in}{2.648822in}}%
\pgfpathlineto{\pgfqpoint{1.164455in}{2.649873in}}%
\pgfpathlineto{\pgfqpoint{1.174693in}{2.650132in}}%
\pgfpathlineto{\pgfqpoint{1.205042in}{2.650274in}}%
\pgfpathlineto{\pgfqpoint{1.298537in}{2.648199in}}%
\pgfpathlineto{\pgfqpoint{1.324351in}{2.646931in}}%
\pgfpathlineto{\pgfqpoint{1.335857in}{2.646282in}}%
\pgfpathlineto{\pgfqpoint{1.355664in}{2.645129in}}%
\pgfpathlineto{\pgfqpoint{1.368251in}{2.644263in}}%
\pgfpathlineto{\pgfqpoint{1.392922in}{2.642463in}}%
\pgfpathlineto{\pgfqpoint{1.447799in}{2.638033in}}%
\pgfpathlineto{\pgfqpoint{1.463909in}{2.636608in}}%
\pgfpathlineto{\pgfqpoint{1.464198in}{2.636580in}}%
\pgfpathlineto{\pgfqpoint{1.464199in}{2.950892in}}%
\pgfpathlineto{\pgfqpoint{1.465731in}{2.950526in}}%
\pgfpathlineto{\pgfqpoint{1.786461in}{2.875126in}}%
\pgfpathlineto{\pgfqpoint{1.786461in}{2.961025in}}%
\pgfpathlineto{\pgfqpoint{1.787952in}{2.960672in}}%
\pgfpathlineto{\pgfqpoint{2.108723in}{2.886084in}}%
\pgfpathlineto{\pgfqpoint{2.108724in}{2.942666in}}%
\pgfpathlineto{\pgfqpoint{2.110239in}{2.942314in}}%
\pgfpathlineto{\pgfqpoint{2.430986in}{2.867969in}}%
\pgfpathlineto{\pgfqpoint{2.430986in}{2.916587in}}%
\pgfpathlineto{\pgfqpoint{2.432526in}{2.916230in}}%
\pgfpathlineto{\pgfqpoint{2.753248in}{2.842007in}}%
\pgfpathlineto{\pgfqpoint{2.753248in}{2.884083in}}%
\pgfpathlineto{\pgfqpoint{2.754760in}{2.883733in}}%
\pgfpathlineto{\pgfqpoint{3.075511in}{2.809563in}}%
\pgfpathlineto{\pgfqpoint{3.075511in}{2.846029in}}%
\pgfpathlineto{\pgfqpoint{3.077020in}{2.845680in}}%
\pgfpathlineto{\pgfqpoint{3.397773in}{2.771542in}}%
\pgfpathlineto{\pgfqpoint{3.397773in}{2.804008in}}%
\pgfpathlineto{\pgfqpoint{3.399312in}{2.803652in}}%
\pgfpathlineto{\pgfqpoint{3.720036in}{2.729542in}}%
\pgfpathlineto{\pgfqpoint{3.720036in}{2.759172in}}%
\pgfpathlineto{\pgfqpoint{3.721561in}{2.758820in}}%
\pgfpathlineto{\pgfqpoint{4.042298in}{2.684719in}}%
\pgfpathlineto{\pgfqpoint{4.042298in}{2.711487in}}%
\pgfpathlineto{\pgfqpoint{4.043796in}{2.711141in}}%
\pgfpathlineto{\pgfqpoint{4.364561in}{2.637042in}}%
\pgfpathlineto{\pgfqpoint{4.364626in}{2.661825in}}%
\pgfpathlineto{\pgfqpoint{4.366165in}{2.661470in}}%
\pgfpathlineto{\pgfqpoint{4.686823in}{2.587402in}}%
\pgfpathlineto{\pgfqpoint{4.686823in}{2.587402in}}%
\pgfusepath{stroke}%
\end{pgfscope}%
\begin{pgfscope}%
\pgfpathrectangle{\pgfqpoint{0.626312in}{0.402778in}}{\pgfqpoint{4.253869in}{2.937500in}}%
\pgfusepath{clip}%
\pgfsetrectcap%
\pgfsetroundjoin%
\pgfsetlinewidth{1.505625pt}%
\definecolor{currentstroke}{rgb}{0.172549,0.627451,0.172549}%
\pgfsetstrokecolor{currentstroke}%
\pgfsetdash{}{0pt}%
\pgfpathmoveto{\pgfqpoint{0.819670in}{0.388889in}}%
\pgfpathlineto{\pgfqpoint{0.821208in}{2.299693in}}%
\pgfpathlineto{\pgfqpoint{0.826207in}{2.389202in}}%
\pgfpathlineto{\pgfqpoint{0.837261in}{2.465437in}}%
\pgfpathlineto{\pgfqpoint{0.855773in}{2.521284in}}%
\pgfpathlineto{\pgfqpoint{0.876637in}{2.557001in}}%
\pgfpathlineto{\pgfqpoint{0.881591in}{2.563052in}}%
\pgfpathlineto{\pgfqpoint{0.884738in}{2.566799in}}%
\pgfpathlineto{\pgfqpoint{0.893617in}{2.575637in}}%
\pgfpathlineto{\pgfqpoint{0.897605in}{2.579282in}}%
\pgfpathlineto{\pgfqpoint{0.910090in}{2.589478in}}%
\pgfpathlineto{\pgfqpoint{0.913430in}{2.591840in}}%
\pgfpathlineto{\pgfqpoint{0.923807in}{2.598764in}}%
\pgfpathlineto{\pgfqpoint{0.932040in}{2.603502in}}%
\pgfpathlineto{\pgfqpoint{0.941691in}{2.608713in}}%
\pgfpathlineto{\pgfqpoint{0.969837in}{2.620648in}}%
\pgfpathlineto{\pgfqpoint{0.976664in}{2.623096in}}%
\pgfpathlineto{\pgfqpoint{0.981650in}{2.624531in}}%
\pgfpathlineto{\pgfqpoint{1.012143in}{2.632851in}}%
\pgfpathlineto{\pgfqpoint{1.040651in}{2.638514in}}%
\pgfpathlineto{\pgfqpoint{1.090076in}{2.645214in}}%
\pgfpathlineto{\pgfqpoint{1.136757in}{2.648822in}}%
\pgfpathlineto{\pgfqpoint{1.164455in}{2.649873in}}%
\pgfpathlineto{\pgfqpoint{1.174693in}{2.650132in}}%
\pgfpathlineto{\pgfqpoint{1.205042in}{2.650274in}}%
\pgfpathlineto{\pgfqpoint{1.298537in}{2.648199in}}%
\pgfpathlineto{\pgfqpoint{1.324351in}{2.646931in}}%
\pgfpathlineto{\pgfqpoint{1.335857in}{2.646282in}}%
\pgfpathlineto{\pgfqpoint{1.355664in}{2.645129in}}%
\pgfpathlineto{\pgfqpoint{1.368251in}{2.644263in}}%
\pgfpathlineto{\pgfqpoint{1.392922in}{2.642463in}}%
\pgfpathlineto{\pgfqpoint{1.447799in}{2.638033in}}%
\pgfpathlineto{\pgfqpoint{1.463909in}{2.636608in}}%
\pgfpathlineto{\pgfqpoint{1.493718in}{2.633686in}}%
\pgfpathlineto{\pgfqpoint{1.580587in}{2.624524in}}%
\pgfpathlineto{\pgfqpoint{1.588862in}{2.623609in}}%
\pgfpathlineto{\pgfqpoint{1.786461in}{2.598457in}}%
\pgfpathlineto{\pgfqpoint{1.786461in}{2.677009in}}%
\pgfpathlineto{\pgfqpoint{1.787992in}{2.676812in}}%
\pgfpathlineto{\pgfqpoint{2.105830in}{2.614815in}}%
\pgfpathlineto{\pgfqpoint{2.108723in}{2.614243in}}%
\pgfpathlineto{\pgfqpoint{2.108724in}{2.723711in}}%
\pgfpathlineto{\pgfqpoint{2.110244in}{2.723383in}}%
\pgfpathlineto{\pgfqpoint{2.430986in}{2.652426in}}%
\pgfpathlineto{\pgfqpoint{2.430986in}{2.727106in}}%
\pgfpathlineto{\pgfqpoint{2.432521in}{2.726763in}}%
\pgfpathlineto{\pgfqpoint{2.753248in}{2.654017in}}%
\pgfpathlineto{\pgfqpoint{2.753249in}{2.708410in}}%
\pgfpathlineto{\pgfqpoint{2.754771in}{2.708068in}}%
\pgfpathlineto{\pgfqpoint{3.075511in}{2.634705in}}%
\pgfpathlineto{\pgfqpoint{3.075511in}{2.687942in}}%
\pgfpathlineto{\pgfqpoint{3.077037in}{2.687593in}}%
\pgfpathlineto{\pgfqpoint{3.397773in}{2.613914in}}%
\pgfpathlineto{\pgfqpoint{3.397774in}{2.664129in}}%
\pgfpathlineto{\pgfqpoint{3.399270in}{2.663786in}}%
\pgfpathlineto{\pgfqpoint{3.720036in}{2.589931in}}%
\pgfpathlineto{\pgfqpoint{3.720036in}{2.635128in}}%
\pgfpathlineto{\pgfqpoint{3.721527in}{2.634786in}}%
\pgfpathlineto{\pgfqpoint{4.042298in}{2.560843in}}%
\pgfpathlineto{\pgfqpoint{4.042298in}{2.600133in}}%
\pgfpathlineto{\pgfqpoint{4.043818in}{2.599784in}}%
\pgfpathlineto{\pgfqpoint{4.364561in}{2.525800in}}%
\pgfpathlineto{\pgfqpoint{4.364561in}{2.561538in}}%
\pgfpathlineto{\pgfqpoint{4.366100in}{2.561183in}}%
\pgfpathlineto{\pgfqpoint{4.686823in}{2.487175in}}%
\pgfpathlineto{\pgfqpoint{4.686823in}{2.487175in}}%
\pgfusepath{stroke}%
\end{pgfscope}%
\begin{pgfscope}%
\pgfpathrectangle{\pgfqpoint{0.626312in}{0.402778in}}{\pgfqpoint{4.253869in}{2.937500in}}%
\pgfusepath{clip}%
\pgfsetrectcap%
\pgfsetroundjoin%
\pgfsetlinewidth{1.505625pt}%
\definecolor{currentstroke}{rgb}{0.839216,0.152941,0.156863}%
\pgfsetstrokecolor{currentstroke}%
\pgfsetdash{}{0pt}%
\pgfpathmoveto{\pgfqpoint{0.819670in}{0.388889in}}%
\pgfpathlineto{\pgfqpoint{0.821208in}{2.299693in}}%
\pgfpathlineto{\pgfqpoint{0.826207in}{2.389202in}}%
\pgfpathlineto{\pgfqpoint{0.837261in}{2.465437in}}%
\pgfpathlineto{\pgfqpoint{0.855773in}{2.521284in}}%
\pgfpathlineto{\pgfqpoint{0.876637in}{2.557001in}}%
\pgfpathlineto{\pgfqpoint{0.881591in}{2.563052in}}%
\pgfpathlineto{\pgfqpoint{0.884738in}{2.566799in}}%
\pgfpathlineto{\pgfqpoint{0.893617in}{2.575637in}}%
\pgfpathlineto{\pgfqpoint{0.897605in}{2.579282in}}%
\pgfpathlineto{\pgfqpoint{0.910090in}{2.589478in}}%
\pgfpathlineto{\pgfqpoint{0.913430in}{2.591840in}}%
\pgfpathlineto{\pgfqpoint{0.923807in}{2.598764in}}%
\pgfpathlineto{\pgfqpoint{0.932040in}{2.603502in}}%
\pgfpathlineto{\pgfqpoint{0.941691in}{2.608713in}}%
\pgfpathlineto{\pgfqpoint{0.969837in}{2.620648in}}%
\pgfpathlineto{\pgfqpoint{0.976664in}{2.623096in}}%
\pgfpathlineto{\pgfqpoint{0.981650in}{2.624531in}}%
\pgfpathlineto{\pgfqpoint{1.012143in}{2.632851in}}%
\pgfpathlineto{\pgfqpoint{1.040651in}{2.638514in}}%
\pgfpathlineto{\pgfqpoint{1.090076in}{2.645214in}}%
\pgfpathlineto{\pgfqpoint{1.136757in}{2.648822in}}%
\pgfpathlineto{\pgfqpoint{1.164455in}{2.649873in}}%
\pgfpathlineto{\pgfqpoint{1.174693in}{2.650132in}}%
\pgfpathlineto{\pgfqpoint{1.205042in}{2.650274in}}%
\pgfpathlineto{\pgfqpoint{1.298537in}{2.648199in}}%
\pgfpathlineto{\pgfqpoint{1.324351in}{2.646931in}}%
\pgfpathlineto{\pgfqpoint{1.335857in}{2.646282in}}%
\pgfpathlineto{\pgfqpoint{1.355664in}{2.645129in}}%
\pgfpathlineto{\pgfqpoint{1.368251in}{2.644263in}}%
\pgfpathlineto{\pgfqpoint{1.392922in}{2.642463in}}%
\pgfpathlineto{\pgfqpoint{1.447799in}{2.638033in}}%
\pgfpathlineto{\pgfqpoint{1.463909in}{2.636608in}}%
\pgfpathlineto{\pgfqpoint{1.493718in}{2.633686in}}%
\pgfpathlineto{\pgfqpoint{1.580587in}{2.624524in}}%
\pgfpathlineto{\pgfqpoint{1.588862in}{2.623609in}}%
\pgfpathlineto{\pgfqpoint{1.954872in}{2.573843in}}%
\pgfpathlineto{\pgfqpoint{2.108723in}{2.549817in}}%
\pgfpathlineto{\pgfqpoint{2.108961in}{2.539626in}}%
\pgfpathlineto{\pgfqpoint{2.110264in}{2.539680in}}%
\pgfpathlineto{\pgfqpoint{2.159085in}{2.532093in}}%
\pgfpathlineto{\pgfqpoint{2.190122in}{2.527212in}}%
\pgfpathlineto{\pgfqpoint{2.245277in}{2.518378in}}%
\pgfpathlineto{\pgfqpoint{2.312860in}{2.507342in}}%
\pgfpathlineto{\pgfqpoint{2.430986in}{2.487473in}}%
\pgfpathlineto{\pgfqpoint{2.430988in}{2.526595in}}%
\pgfpathlineto{\pgfqpoint{2.432460in}{2.526102in}}%
\pgfpathlineto{\pgfqpoint{2.481076in}{2.512946in}}%
\pgfpathlineto{\pgfqpoint{2.565454in}{2.490004in}}%
\pgfpathlineto{\pgfqpoint{2.753248in}{2.438507in}}%
\pgfpathlineto{\pgfqpoint{2.754743in}{2.393448in}}%
\pgfpathlineto{\pgfqpoint{2.836844in}{2.368113in}}%
\pgfpathlineto{\pgfqpoint{3.075511in}{2.291334in}}%
\pgfpathlineto{\pgfqpoint{3.075511in}{2.475862in}}%
\pgfpathlineto{\pgfqpoint{3.077005in}{2.475447in}}%
\pgfpathlineto{\pgfqpoint{3.397773in}{2.397093in}}%
\pgfpathlineto{\pgfqpoint{3.397774in}{2.474664in}}%
\pgfpathlineto{\pgfqpoint{3.399265in}{2.474303in}}%
\pgfpathlineto{\pgfqpoint{3.720036in}{2.398364in}}%
\pgfpathlineto{\pgfqpoint{3.720036in}{2.460722in}}%
\pgfpathlineto{\pgfqpoint{3.721555in}{2.460364in}}%
\pgfpathlineto{\pgfqpoint{4.042298in}{2.385355in}}%
\pgfpathlineto{\pgfqpoint{4.042299in}{2.448527in}}%
\pgfpathlineto{\pgfqpoint{4.043830in}{2.448170in}}%
\pgfpathlineto{\pgfqpoint{4.364561in}{2.373635in}}%
\pgfpathlineto{\pgfqpoint{4.364561in}{2.421194in}}%
\pgfpathlineto{\pgfqpoint{4.366066in}{2.420842in}}%
\pgfpathlineto{\pgfqpoint{4.686823in}{2.346496in}}%
\pgfpathlineto{\pgfqpoint{4.686823in}{2.346496in}}%
\pgfusepath{stroke}%
\end{pgfscope}%
\begin{pgfscope}%
\pgfpathrectangle{\pgfqpoint{0.626312in}{0.402778in}}{\pgfqpoint{4.253869in}{2.937500in}}%
\pgfusepath{clip}%
\pgfsetrectcap%
\pgfsetroundjoin%
\pgfsetlinewidth{1.505625pt}%
\definecolor{currentstroke}{rgb}{0.580392,0.403922,0.741176}%
\pgfsetstrokecolor{currentstroke}%
\pgfsetdash{}{0pt}%
\pgfpathmoveto{\pgfqpoint{0.819670in}{0.388889in}}%
\pgfpathlineto{\pgfqpoint{0.821208in}{2.299693in}}%
\pgfpathlineto{\pgfqpoint{0.826207in}{2.389202in}}%
\pgfpathlineto{\pgfqpoint{0.837261in}{2.465437in}}%
\pgfpathlineto{\pgfqpoint{0.855773in}{2.521284in}}%
\pgfpathlineto{\pgfqpoint{0.876637in}{2.557001in}}%
\pgfpathlineto{\pgfqpoint{0.881591in}{2.563052in}}%
\pgfpathlineto{\pgfqpoint{0.884738in}{2.566799in}}%
\pgfpathlineto{\pgfqpoint{0.893617in}{2.575637in}}%
\pgfpathlineto{\pgfqpoint{0.897605in}{2.579282in}}%
\pgfpathlineto{\pgfqpoint{0.910090in}{2.589478in}}%
\pgfpathlineto{\pgfqpoint{0.913430in}{2.591840in}}%
\pgfpathlineto{\pgfqpoint{0.923807in}{2.598764in}}%
\pgfpathlineto{\pgfqpoint{0.932040in}{2.603502in}}%
\pgfpathlineto{\pgfqpoint{0.941691in}{2.608713in}}%
\pgfpathlineto{\pgfqpoint{0.969837in}{2.620648in}}%
\pgfpathlineto{\pgfqpoint{0.976664in}{2.623096in}}%
\pgfpathlineto{\pgfqpoint{0.981650in}{2.624531in}}%
\pgfpathlineto{\pgfqpoint{1.012143in}{2.632851in}}%
\pgfpathlineto{\pgfqpoint{1.040651in}{2.638514in}}%
\pgfpathlineto{\pgfqpoint{1.090076in}{2.645214in}}%
\pgfpathlineto{\pgfqpoint{1.136757in}{2.648822in}}%
\pgfpathlineto{\pgfqpoint{1.164455in}{2.649873in}}%
\pgfpathlineto{\pgfqpoint{1.174693in}{2.650132in}}%
\pgfpathlineto{\pgfqpoint{1.205042in}{2.650274in}}%
\pgfpathlineto{\pgfqpoint{1.298537in}{2.648199in}}%
\pgfpathlineto{\pgfqpoint{1.324351in}{2.646931in}}%
\pgfpathlineto{\pgfqpoint{1.335857in}{2.646282in}}%
\pgfpathlineto{\pgfqpoint{1.355664in}{2.645129in}}%
\pgfpathlineto{\pgfqpoint{1.368251in}{2.644263in}}%
\pgfpathlineto{\pgfqpoint{1.392922in}{2.642463in}}%
\pgfpathlineto{\pgfqpoint{1.447799in}{2.638033in}}%
\pgfpathlineto{\pgfqpoint{1.463909in}{2.636608in}}%
\pgfpathlineto{\pgfqpoint{1.493718in}{2.633686in}}%
\pgfpathlineto{\pgfqpoint{1.580587in}{2.624524in}}%
\pgfpathlineto{\pgfqpoint{1.588862in}{2.623609in}}%
\pgfpathlineto{\pgfqpoint{1.954872in}{2.573843in}}%
\pgfpathlineto{\pgfqpoint{2.282552in}{2.520901in}}%
\pgfpathlineto{\pgfqpoint{2.382416in}{2.503755in}}%
\pgfpathlineto{\pgfqpoint{2.430986in}{2.495307in}}%
\pgfpathlineto{\pgfqpoint{2.430991in}{2.600080in}}%
\pgfpathlineto{\pgfqpoint{2.432492in}{2.599778in}}%
\pgfpathlineto{\pgfqpoint{2.753248in}{2.531045in}}%
\pgfpathlineto{\pgfqpoint{2.753249in}{2.551487in}}%
\pgfpathlineto{\pgfqpoint{2.754767in}{2.551171in}}%
\pgfpathlineto{\pgfqpoint{3.075511in}{2.481143in}}%
\pgfpathlineto{\pgfqpoint{3.075515in}{2.501920in}}%
\pgfpathlineto{\pgfqpoint{3.077015in}{2.501606in}}%
\pgfpathlineto{\pgfqpoint{3.397773in}{2.430602in}}%
\pgfpathlineto{\pgfqpoint{3.397774in}{2.453169in}}%
\pgfpathlineto{\pgfqpoint{3.399281in}{2.452838in}}%
\pgfpathlineto{\pgfqpoint{3.720036in}{2.381053in}}%
\pgfpathlineto{\pgfqpoint{3.720036in}{2.398195in}}%
\pgfpathlineto{\pgfqpoint{3.721541in}{2.397863in}}%
\pgfpathlineto{\pgfqpoint{4.042298in}{2.325662in}}%
\pgfpathlineto{\pgfqpoint{4.042298in}{2.366344in}}%
\pgfpathlineto{\pgfqpoint{4.043818in}{2.366009in}}%
\pgfpathlineto{\pgfqpoint{4.364561in}{2.293103in}}%
\pgfpathlineto{\pgfqpoint{4.364561in}{2.306096in}}%
\pgfpathlineto{\pgfqpoint{4.366036in}{2.305763in}}%
\pgfpathlineto{\pgfqpoint{4.686823in}{2.232678in}}%
\pgfpathlineto{\pgfqpoint{4.686823in}{2.232678in}}%
\pgfusepath{stroke}%
\end{pgfscope}%
\begin{pgfscope}%
\pgfpathrectangle{\pgfqpoint{0.626312in}{0.402778in}}{\pgfqpoint{4.253869in}{2.937500in}}%
\pgfusepath{clip}%
\pgfsetrectcap%
\pgfsetroundjoin%
\pgfsetlinewidth{1.505625pt}%
\definecolor{currentstroke}{rgb}{0.549020,0.337255,0.294118}%
\pgfsetstrokecolor{currentstroke}%
\pgfsetdash{}{0pt}%
\pgfpathmoveto{\pgfqpoint{0.819670in}{0.388889in}}%
\pgfpathlineto{\pgfqpoint{0.821208in}{2.299693in}}%
\pgfpathlineto{\pgfqpoint{0.826207in}{2.389202in}}%
\pgfpathlineto{\pgfqpoint{0.837261in}{2.465437in}}%
\pgfpathlineto{\pgfqpoint{0.855773in}{2.521284in}}%
\pgfpathlineto{\pgfqpoint{0.876637in}{2.557001in}}%
\pgfpathlineto{\pgfqpoint{0.881591in}{2.563052in}}%
\pgfpathlineto{\pgfqpoint{0.884738in}{2.566799in}}%
\pgfpathlineto{\pgfqpoint{0.893617in}{2.575637in}}%
\pgfpathlineto{\pgfqpoint{0.897605in}{2.579282in}}%
\pgfpathlineto{\pgfqpoint{0.910090in}{2.589478in}}%
\pgfpathlineto{\pgfqpoint{0.913430in}{2.591840in}}%
\pgfpathlineto{\pgfqpoint{0.923807in}{2.598764in}}%
\pgfpathlineto{\pgfqpoint{0.932040in}{2.603502in}}%
\pgfpathlineto{\pgfqpoint{0.941691in}{2.608713in}}%
\pgfpathlineto{\pgfqpoint{0.969837in}{2.620648in}}%
\pgfpathlineto{\pgfqpoint{0.976664in}{2.623096in}}%
\pgfpathlineto{\pgfqpoint{0.981650in}{2.624531in}}%
\pgfpathlineto{\pgfqpoint{1.012143in}{2.632851in}}%
\pgfpathlineto{\pgfqpoint{1.040651in}{2.638514in}}%
\pgfpathlineto{\pgfqpoint{1.090076in}{2.645214in}}%
\pgfpathlineto{\pgfqpoint{1.136757in}{2.648822in}}%
\pgfpathlineto{\pgfqpoint{1.164455in}{2.649873in}}%
\pgfpathlineto{\pgfqpoint{1.174693in}{2.650132in}}%
\pgfpathlineto{\pgfqpoint{1.205042in}{2.650274in}}%
\pgfpathlineto{\pgfqpoint{1.298537in}{2.648199in}}%
\pgfpathlineto{\pgfqpoint{1.324351in}{2.646931in}}%
\pgfpathlineto{\pgfqpoint{1.335857in}{2.646282in}}%
\pgfpathlineto{\pgfqpoint{1.355664in}{2.645129in}}%
\pgfpathlineto{\pgfqpoint{1.368251in}{2.644263in}}%
\pgfpathlineto{\pgfqpoint{1.392922in}{2.642463in}}%
\pgfpathlineto{\pgfqpoint{1.447799in}{2.638033in}}%
\pgfpathlineto{\pgfqpoint{1.463909in}{2.636608in}}%
\pgfpathlineto{\pgfqpoint{1.493718in}{2.633686in}}%
\pgfpathlineto{\pgfqpoint{1.580587in}{2.624524in}}%
\pgfpathlineto{\pgfqpoint{1.588862in}{2.623609in}}%
\pgfpathlineto{\pgfqpoint{1.954872in}{2.573843in}}%
\pgfpathlineto{\pgfqpoint{2.282552in}{2.520901in}}%
\pgfpathlineto{\pgfqpoint{2.382416in}{2.503755in}}%
\pgfpathlineto{\pgfqpoint{2.456000in}{2.490920in}}%
\pgfpathlineto{\pgfqpoint{2.540002in}{2.476007in}}%
\pgfpathlineto{\pgfqpoint{2.584057in}{2.468117in}}%
\pgfpathlineto{\pgfqpoint{2.662526in}{2.453888in}}%
\pgfpathlineto{\pgfqpoint{2.694549in}{2.448049in}}%
\pgfpathlineto{\pgfqpoint{2.753248in}{2.437211in}}%
\pgfpathlineto{\pgfqpoint{2.753273in}{2.166624in}}%
\pgfpathlineto{\pgfqpoint{2.754794in}{2.168927in}}%
\pgfpathlineto{\pgfqpoint{2.778960in}{2.183228in}}%
\pgfpathlineto{\pgfqpoint{2.782002in}{2.184797in}}%
\pgfpathlineto{\pgfqpoint{2.789198in}{2.188369in}}%
\pgfpathlineto{\pgfqpoint{2.800319in}{2.193217in}}%
\pgfpathlineto{\pgfqpoint{2.807032in}{2.195899in}}%
\pgfpathlineto{\pgfqpoint{2.857831in}{2.210464in}}%
\pgfpathlineto{\pgfqpoint{2.873470in}{2.213478in}}%
\pgfpathlineto{\pgfqpoint{2.883949in}{2.215241in}}%
\pgfpathlineto{\pgfqpoint{2.889651in}{2.216067in}}%
\pgfpathlineto{\pgfqpoint{2.960754in}{2.223247in}}%
\pgfpathlineto{\pgfqpoint{2.986419in}{2.224476in}}%
\pgfpathlineto{\pgfqpoint{2.989845in}{2.224732in}}%
\pgfpathlineto{\pgfqpoint{3.026205in}{2.225755in}}%
\pgfpathlineto{\pgfqpoint{3.033060in}{2.225907in}}%
\pgfpathlineto{\pgfqpoint{3.054105in}{2.225882in}}%
\pgfpathlineto{\pgfqpoint{3.065690in}{2.225898in}}%
\pgfpathlineto{\pgfqpoint{3.075511in}{2.225751in}}%
\pgfpathlineto{\pgfqpoint{3.075511in}{2.333739in}}%
\pgfpathlineto{\pgfqpoint{3.076991in}{2.333240in}}%
\pgfpathlineto{\pgfqpoint{3.397773in}{2.235739in}}%
\pgfpathlineto{\pgfqpoint{3.397774in}{2.398991in}}%
\pgfpathlineto{\pgfqpoint{3.399261in}{2.398595in}}%
\pgfpathlineto{\pgfqpoint{3.720036in}{2.320141in}}%
\pgfpathlineto{\pgfqpoint{3.720037in}{2.365263in}}%
\pgfpathlineto{\pgfqpoint{3.721570in}{2.364869in}}%
\pgfpathlineto{\pgfqpoint{4.042298in}{2.288028in}}%
\pgfpathlineto{\pgfqpoint{4.042299in}{2.337711in}}%
\pgfpathlineto{\pgfqpoint{4.043822in}{2.337332in}}%
\pgfpathlineto{\pgfqpoint{4.364561in}{2.261634in}}%
\pgfpathlineto{\pgfqpoint{4.366064in}{2.255694in}}%
\pgfpathlineto{\pgfqpoint{4.686823in}{2.179853in}}%
\pgfpathlineto{\pgfqpoint{4.686823in}{2.179853in}}%
\pgfusepath{stroke}%
\end{pgfscope}%
\begin{pgfscope}%
\pgfpathrectangle{\pgfqpoint{0.626312in}{0.402778in}}{\pgfqpoint{4.253869in}{2.937500in}}%
\pgfusepath{clip}%
\pgfsetrectcap%
\pgfsetroundjoin%
\pgfsetlinewidth{1.505625pt}%
\definecolor{currentstroke}{rgb}{0.890196,0.466667,0.760784}%
\pgfsetstrokecolor{currentstroke}%
\pgfsetdash{}{0pt}%
\pgfpathmoveto{\pgfqpoint{0.819670in}{0.388889in}}%
\pgfpathlineto{\pgfqpoint{0.821208in}{2.299693in}}%
\pgfpathlineto{\pgfqpoint{0.826207in}{2.389202in}}%
\pgfpathlineto{\pgfqpoint{0.837261in}{2.465437in}}%
\pgfpathlineto{\pgfqpoint{0.855773in}{2.521284in}}%
\pgfpathlineto{\pgfqpoint{0.876637in}{2.557001in}}%
\pgfpathlineto{\pgfqpoint{0.881591in}{2.563052in}}%
\pgfpathlineto{\pgfqpoint{0.884738in}{2.566799in}}%
\pgfpathlineto{\pgfqpoint{0.893617in}{2.575637in}}%
\pgfpathlineto{\pgfqpoint{0.897605in}{2.579282in}}%
\pgfpathlineto{\pgfqpoint{0.910090in}{2.589478in}}%
\pgfpathlineto{\pgfqpoint{0.913430in}{2.591840in}}%
\pgfpathlineto{\pgfqpoint{0.923807in}{2.598764in}}%
\pgfpathlineto{\pgfqpoint{0.932040in}{2.603502in}}%
\pgfpathlineto{\pgfqpoint{0.941691in}{2.608713in}}%
\pgfpathlineto{\pgfqpoint{0.969837in}{2.620648in}}%
\pgfpathlineto{\pgfqpoint{0.976664in}{2.623096in}}%
\pgfpathlineto{\pgfqpoint{0.981650in}{2.624531in}}%
\pgfpathlineto{\pgfqpoint{1.012143in}{2.632851in}}%
\pgfpathlineto{\pgfqpoint{1.040651in}{2.638514in}}%
\pgfpathlineto{\pgfqpoint{1.090076in}{2.645214in}}%
\pgfpathlineto{\pgfqpoint{1.136757in}{2.648822in}}%
\pgfpathlineto{\pgfqpoint{1.164455in}{2.649873in}}%
\pgfpathlineto{\pgfqpoint{1.174693in}{2.650132in}}%
\pgfpathlineto{\pgfqpoint{1.205042in}{2.650274in}}%
\pgfpathlineto{\pgfqpoint{1.298537in}{2.648199in}}%
\pgfpathlineto{\pgfqpoint{1.324351in}{2.646931in}}%
\pgfpathlineto{\pgfqpoint{1.335857in}{2.646282in}}%
\pgfpathlineto{\pgfqpoint{1.355664in}{2.645129in}}%
\pgfpathlineto{\pgfqpoint{1.368251in}{2.644263in}}%
\pgfpathlineto{\pgfqpoint{1.392922in}{2.642463in}}%
\pgfpathlineto{\pgfqpoint{1.447799in}{2.638033in}}%
\pgfpathlineto{\pgfqpoint{1.463909in}{2.636608in}}%
\pgfpathlineto{\pgfqpoint{1.493718in}{2.633686in}}%
\pgfpathlineto{\pgfqpoint{1.580587in}{2.624524in}}%
\pgfpathlineto{\pgfqpoint{1.588862in}{2.623609in}}%
\pgfpathlineto{\pgfqpoint{1.954872in}{2.573843in}}%
\pgfpathlineto{\pgfqpoint{2.282552in}{2.520901in}}%
\pgfpathlineto{\pgfqpoint{2.382416in}{2.503755in}}%
\pgfpathlineto{\pgfqpoint{2.456000in}{2.490920in}}%
\pgfpathlineto{\pgfqpoint{2.540002in}{2.476007in}}%
\pgfpathlineto{\pgfqpoint{2.584057in}{2.468117in}}%
\pgfpathlineto{\pgfqpoint{2.662526in}{2.453888in}}%
\pgfpathlineto{\pgfqpoint{2.694549in}{2.448049in}}%
\pgfpathlineto{\pgfqpoint{2.834227in}{2.422137in}}%
\pgfpathlineto{\pgfqpoint{2.864945in}{2.416437in}}%
\pgfpathlineto{\pgfqpoint{2.915495in}{2.406915in}}%
\pgfpathlineto{\pgfqpoint{3.075511in}{2.376532in}}%
\pgfpathlineto{\pgfqpoint{3.075517in}{2.409938in}}%
\pgfpathlineto{\pgfqpoint{3.077029in}{2.409652in}}%
\pgfpathlineto{\pgfqpoint{3.397773in}{2.343916in}}%
\pgfpathlineto{\pgfqpoint{3.399289in}{2.335605in}}%
\pgfpathlineto{\pgfqpoint{3.720036in}{2.269820in}}%
\pgfpathlineto{\pgfqpoint{3.720036in}{2.324875in}}%
\pgfpathlineto{\pgfqpoint{3.721572in}{2.324567in}}%
\pgfpathlineto{\pgfqpoint{4.042298in}{2.254685in}}%
\pgfpathlineto{\pgfqpoint{4.042299in}{2.317199in}}%
\pgfpathlineto{\pgfqpoint{4.043820in}{2.316883in}}%
\pgfpathlineto{\pgfqpoint{4.364561in}{2.244812in}}%
\pgfpathlineto{\pgfqpoint{4.366060in}{2.236337in}}%
\pgfpathlineto{\pgfqpoint{4.686823in}{2.164411in}}%
\pgfpathlineto{\pgfqpoint{4.686823in}{2.164411in}}%
\pgfusepath{stroke}%
\end{pgfscope}%
\begin{pgfscope}%
\pgfpathrectangle{\pgfqpoint{0.626312in}{0.402778in}}{\pgfqpoint{4.253869in}{2.937500in}}%
\pgfusepath{clip}%
\pgfsetrectcap%
\pgfsetroundjoin%
\pgfsetlinewidth{1.505625pt}%
\definecolor{currentstroke}{rgb}{0.498039,0.498039,0.498039}%
\pgfsetstrokecolor{currentstroke}%
\pgfsetdash{}{0pt}%
\pgfpathmoveto{\pgfqpoint{0.819670in}{0.388889in}}%
\pgfpathlineto{\pgfqpoint{0.821208in}{2.299693in}}%
\pgfpathlineto{\pgfqpoint{0.826207in}{2.389202in}}%
\pgfpathlineto{\pgfqpoint{0.837261in}{2.465437in}}%
\pgfpathlineto{\pgfqpoint{0.855773in}{2.521284in}}%
\pgfpathlineto{\pgfqpoint{0.876637in}{2.557001in}}%
\pgfpathlineto{\pgfqpoint{0.881591in}{2.563052in}}%
\pgfpathlineto{\pgfqpoint{0.884738in}{2.566799in}}%
\pgfpathlineto{\pgfqpoint{0.893617in}{2.575637in}}%
\pgfpathlineto{\pgfqpoint{0.897605in}{2.579282in}}%
\pgfpathlineto{\pgfqpoint{0.910090in}{2.589478in}}%
\pgfpathlineto{\pgfqpoint{0.913430in}{2.591840in}}%
\pgfpathlineto{\pgfqpoint{0.923807in}{2.598764in}}%
\pgfpathlineto{\pgfqpoint{0.932040in}{2.603502in}}%
\pgfpathlineto{\pgfqpoint{0.941691in}{2.608713in}}%
\pgfpathlineto{\pgfqpoint{0.969837in}{2.620648in}}%
\pgfpathlineto{\pgfqpoint{0.976664in}{2.623096in}}%
\pgfpathlineto{\pgfqpoint{0.981650in}{2.624531in}}%
\pgfpathlineto{\pgfqpoint{1.012143in}{2.632851in}}%
\pgfpathlineto{\pgfqpoint{1.040651in}{2.638514in}}%
\pgfpathlineto{\pgfqpoint{1.090076in}{2.645214in}}%
\pgfpathlineto{\pgfqpoint{1.136757in}{2.648822in}}%
\pgfpathlineto{\pgfqpoint{1.164455in}{2.649873in}}%
\pgfpathlineto{\pgfqpoint{1.174693in}{2.650132in}}%
\pgfpathlineto{\pgfqpoint{1.205042in}{2.650274in}}%
\pgfpathlineto{\pgfqpoint{1.298537in}{2.648199in}}%
\pgfpathlineto{\pgfqpoint{1.324351in}{2.646931in}}%
\pgfpathlineto{\pgfqpoint{1.335857in}{2.646282in}}%
\pgfpathlineto{\pgfqpoint{1.355664in}{2.645129in}}%
\pgfpathlineto{\pgfqpoint{1.368251in}{2.644263in}}%
\pgfpathlineto{\pgfqpoint{1.392922in}{2.642463in}}%
\pgfpathlineto{\pgfqpoint{1.447799in}{2.638033in}}%
\pgfpathlineto{\pgfqpoint{1.463909in}{2.636608in}}%
\pgfpathlineto{\pgfqpoint{1.493718in}{2.633686in}}%
\pgfpathlineto{\pgfqpoint{1.580587in}{2.624524in}}%
\pgfpathlineto{\pgfqpoint{1.588862in}{2.623609in}}%
\pgfpathlineto{\pgfqpoint{1.954872in}{2.573843in}}%
\pgfpathlineto{\pgfqpoint{2.282552in}{2.520901in}}%
\pgfpathlineto{\pgfqpoint{2.382416in}{2.503755in}}%
\pgfpathlineto{\pgfqpoint{2.456000in}{2.490920in}}%
\pgfpathlineto{\pgfqpoint{2.540002in}{2.476007in}}%
\pgfpathlineto{\pgfqpoint{2.584057in}{2.468117in}}%
\pgfpathlineto{\pgfqpoint{2.662526in}{2.453888in}}%
\pgfpathlineto{\pgfqpoint{2.694549in}{2.448049in}}%
\pgfpathlineto{\pgfqpoint{2.834227in}{2.422137in}}%
\pgfpathlineto{\pgfqpoint{2.864945in}{2.416437in}}%
\pgfpathlineto{\pgfqpoint{2.915495in}{2.406915in}}%
\pgfpathlineto{\pgfqpoint{3.303893in}{2.332369in}}%
\pgfpathlineto{\pgfqpoint{3.397773in}{2.314020in}}%
\pgfpathlineto{\pgfqpoint{3.397780in}{2.371440in}}%
\pgfpathlineto{\pgfqpoint{3.399322in}{2.371129in}}%
\pgfpathlineto{\pgfqpoint{3.720036in}{2.302759in}}%
\pgfpathlineto{\pgfqpoint{3.721480in}{2.238330in}}%
\pgfpathlineto{\pgfqpoint{3.786990in}{2.225529in}}%
\pgfpathlineto{\pgfqpoint{3.996214in}{2.184106in}}%
\pgfpathlineto{\pgfqpoint{4.042298in}{2.174921in}}%
\pgfpathlineto{\pgfqpoint{4.043826in}{2.072511in}}%
\pgfpathlineto{\pgfqpoint{4.095259in}{2.055417in}}%
\pgfpathlineto{\pgfqpoint{4.162945in}{2.032523in}}%
\pgfpathlineto{\pgfqpoint{4.185292in}{2.024807in}}%
\pgfpathlineto{\pgfqpoint{4.268896in}{1.995287in}}%
\pgfpathlineto{\pgfqpoint{4.279467in}{1.991526in}}%
\pgfpathlineto{\pgfqpoint{4.300096in}{1.983957in}}%
\pgfpathlineto{\pgfqpoint{4.329991in}{1.972909in}}%
\pgfpathlineto{\pgfqpoint{4.361902in}{1.960927in}}%
\pgfpathlineto{\pgfqpoint{4.364561in}{1.959913in}}%
\pgfpathlineto{\pgfqpoint{4.364561in}{2.108723in}}%
\pgfpathlineto{\pgfqpoint{4.366105in}{2.108225in}}%
\pgfpathlineto{\pgfqpoint{4.504687in}{2.072250in}}%
\pgfpathlineto{\pgfqpoint{4.686823in}{2.024589in}}%
\pgfpathlineto{\pgfqpoint{4.686823in}{2.024589in}}%
\pgfusepath{stroke}%
\end{pgfscope}%
\begin{pgfscope}%
\pgfpathrectangle{\pgfqpoint{0.626312in}{0.402778in}}{\pgfqpoint{4.253869in}{2.937500in}}%
\pgfusepath{clip}%
\pgfsetrectcap%
\pgfsetroundjoin%
\pgfsetlinewidth{1.505625pt}%
\definecolor{currentstroke}{rgb}{0.737255,0.741176,0.133333}%
\pgfsetstrokecolor{currentstroke}%
\pgfsetdash{}{0pt}%
\pgfpathmoveto{\pgfqpoint{0.819670in}{0.388889in}}%
\pgfpathlineto{\pgfqpoint{0.821208in}{2.299693in}}%
\pgfpathlineto{\pgfqpoint{0.826207in}{2.389202in}}%
\pgfpathlineto{\pgfqpoint{0.837261in}{2.465437in}}%
\pgfpathlineto{\pgfqpoint{0.855773in}{2.521284in}}%
\pgfpathlineto{\pgfqpoint{0.876637in}{2.557001in}}%
\pgfpathlineto{\pgfqpoint{0.881591in}{2.563052in}}%
\pgfpathlineto{\pgfqpoint{0.884738in}{2.566799in}}%
\pgfpathlineto{\pgfqpoint{0.893617in}{2.575637in}}%
\pgfpathlineto{\pgfqpoint{0.897605in}{2.579282in}}%
\pgfpathlineto{\pgfqpoint{0.910090in}{2.589478in}}%
\pgfpathlineto{\pgfqpoint{0.913430in}{2.591840in}}%
\pgfpathlineto{\pgfqpoint{0.923807in}{2.598764in}}%
\pgfpathlineto{\pgfqpoint{0.932040in}{2.603502in}}%
\pgfpathlineto{\pgfqpoint{0.941691in}{2.608713in}}%
\pgfpathlineto{\pgfqpoint{0.969837in}{2.620648in}}%
\pgfpathlineto{\pgfqpoint{0.976664in}{2.623096in}}%
\pgfpathlineto{\pgfqpoint{0.981650in}{2.624531in}}%
\pgfpathlineto{\pgfqpoint{1.012143in}{2.632851in}}%
\pgfpathlineto{\pgfqpoint{1.040651in}{2.638514in}}%
\pgfpathlineto{\pgfqpoint{1.090076in}{2.645214in}}%
\pgfpathlineto{\pgfqpoint{1.136757in}{2.648822in}}%
\pgfpathlineto{\pgfqpoint{1.164455in}{2.649873in}}%
\pgfpathlineto{\pgfqpoint{1.174693in}{2.650132in}}%
\pgfpathlineto{\pgfqpoint{1.205042in}{2.650274in}}%
\pgfpathlineto{\pgfqpoint{1.298537in}{2.648199in}}%
\pgfpathlineto{\pgfqpoint{1.324351in}{2.646931in}}%
\pgfpathlineto{\pgfqpoint{1.335857in}{2.646282in}}%
\pgfpathlineto{\pgfqpoint{1.355664in}{2.645129in}}%
\pgfpathlineto{\pgfqpoint{1.368251in}{2.644263in}}%
\pgfpathlineto{\pgfqpoint{1.392922in}{2.642463in}}%
\pgfpathlineto{\pgfqpoint{1.447799in}{2.638033in}}%
\pgfpathlineto{\pgfqpoint{1.463909in}{2.636608in}}%
\pgfpathlineto{\pgfqpoint{1.493718in}{2.633686in}}%
\pgfpathlineto{\pgfqpoint{1.580587in}{2.624524in}}%
\pgfpathlineto{\pgfqpoint{1.588862in}{2.623609in}}%
\pgfpathlineto{\pgfqpoint{1.954872in}{2.573843in}}%
\pgfpathlineto{\pgfqpoint{2.282552in}{2.520901in}}%
\pgfpathlineto{\pgfqpoint{2.382416in}{2.503755in}}%
\pgfpathlineto{\pgfqpoint{2.456000in}{2.490920in}}%
\pgfpathlineto{\pgfqpoint{2.540002in}{2.476007in}}%
\pgfpathlineto{\pgfqpoint{2.584057in}{2.468117in}}%
\pgfpathlineto{\pgfqpoint{2.662526in}{2.453888in}}%
\pgfpathlineto{\pgfqpoint{2.694549in}{2.448049in}}%
\pgfpathlineto{\pgfqpoint{2.834227in}{2.422137in}}%
\pgfpathlineto{\pgfqpoint{2.864945in}{2.416437in}}%
\pgfpathlineto{\pgfqpoint{2.915495in}{2.406915in}}%
\pgfpathlineto{\pgfqpoint{3.303893in}{2.332369in}}%
\pgfpathlineto{\pgfqpoint{3.653687in}{2.263385in}}%
\pgfpathlineto{\pgfqpoint{3.720036in}{2.250117in}}%
\pgfpathlineto{\pgfqpoint{3.720042in}{2.307792in}}%
\pgfpathlineto{\pgfqpoint{3.721528in}{2.307472in}}%
\pgfpathlineto{\pgfqpoint{3.918853in}{2.265007in}}%
\pgfpathlineto{\pgfqpoint{4.042298in}{2.238426in}}%
\pgfpathlineto{\pgfqpoint{4.042300in}{2.241058in}}%
\pgfpathlineto{\pgfqpoint{4.043790in}{2.240762in}}%
\pgfpathlineto{\pgfqpoint{4.364561in}{2.171316in}}%
\pgfpathlineto{\pgfqpoint{4.364562in}{2.229850in}}%
\pgfpathlineto{\pgfqpoint{4.366096in}{2.229516in}}%
\pgfpathlineto{\pgfqpoint{4.686823in}{2.157762in}}%
\pgfpathlineto{\pgfqpoint{4.686823in}{2.157762in}}%
\pgfusepath{stroke}%
\end{pgfscope}%
\begin{pgfscope}%
\pgfpathrectangle{\pgfqpoint{0.626312in}{0.402778in}}{\pgfqpoint{4.253869in}{2.937500in}}%
\pgfusepath{clip}%
\pgfsetrectcap%
\pgfsetroundjoin%
\pgfsetlinewidth{1.505625pt}%
\definecolor{currentstroke}{rgb}{0.090196,0.745098,0.811765}%
\pgfsetstrokecolor{currentstroke}%
\pgfsetdash{}{0pt}%
\pgfpathmoveto{\pgfqpoint{0.819670in}{0.388889in}}%
\pgfpathlineto{\pgfqpoint{0.821208in}{2.299693in}}%
\pgfpathlineto{\pgfqpoint{0.826207in}{2.389202in}}%
\pgfpathlineto{\pgfqpoint{0.837261in}{2.465437in}}%
\pgfpathlineto{\pgfqpoint{0.855773in}{2.521284in}}%
\pgfpathlineto{\pgfqpoint{0.876637in}{2.557001in}}%
\pgfpathlineto{\pgfqpoint{0.881591in}{2.563052in}}%
\pgfpathlineto{\pgfqpoint{0.884738in}{2.566799in}}%
\pgfpathlineto{\pgfqpoint{0.893617in}{2.575637in}}%
\pgfpathlineto{\pgfqpoint{0.897605in}{2.579282in}}%
\pgfpathlineto{\pgfqpoint{0.910090in}{2.589478in}}%
\pgfpathlineto{\pgfqpoint{0.913430in}{2.591840in}}%
\pgfpathlineto{\pgfqpoint{0.923807in}{2.598764in}}%
\pgfpathlineto{\pgfqpoint{0.932040in}{2.603502in}}%
\pgfpathlineto{\pgfqpoint{0.941691in}{2.608713in}}%
\pgfpathlineto{\pgfqpoint{0.969837in}{2.620648in}}%
\pgfpathlineto{\pgfqpoint{0.976664in}{2.623096in}}%
\pgfpathlineto{\pgfqpoint{0.981650in}{2.624531in}}%
\pgfpathlineto{\pgfqpoint{1.012143in}{2.632851in}}%
\pgfpathlineto{\pgfqpoint{1.040651in}{2.638514in}}%
\pgfpathlineto{\pgfqpoint{1.090076in}{2.645214in}}%
\pgfpathlineto{\pgfqpoint{1.136757in}{2.648822in}}%
\pgfpathlineto{\pgfqpoint{1.164455in}{2.649873in}}%
\pgfpathlineto{\pgfqpoint{1.174693in}{2.650132in}}%
\pgfpathlineto{\pgfqpoint{1.205042in}{2.650274in}}%
\pgfpathlineto{\pgfqpoint{1.298537in}{2.648199in}}%
\pgfpathlineto{\pgfqpoint{1.324351in}{2.646931in}}%
\pgfpathlineto{\pgfqpoint{1.335857in}{2.646282in}}%
\pgfpathlineto{\pgfqpoint{1.355664in}{2.645129in}}%
\pgfpathlineto{\pgfqpoint{1.368251in}{2.644263in}}%
\pgfpathlineto{\pgfqpoint{1.392922in}{2.642463in}}%
\pgfpathlineto{\pgfqpoint{1.447799in}{2.638033in}}%
\pgfpathlineto{\pgfqpoint{1.463909in}{2.636608in}}%
\pgfpathlineto{\pgfqpoint{1.493718in}{2.633686in}}%
\pgfpathlineto{\pgfqpoint{1.580587in}{2.624524in}}%
\pgfpathlineto{\pgfqpoint{1.588862in}{2.623609in}}%
\pgfpathlineto{\pgfqpoint{1.954872in}{2.573843in}}%
\pgfpathlineto{\pgfqpoint{2.282552in}{2.520901in}}%
\pgfpathlineto{\pgfqpoint{2.382416in}{2.503755in}}%
\pgfpathlineto{\pgfqpoint{2.456000in}{2.490920in}}%
\pgfpathlineto{\pgfqpoint{2.540002in}{2.476007in}}%
\pgfpathlineto{\pgfqpoint{2.584057in}{2.468117in}}%
\pgfpathlineto{\pgfqpoint{2.662526in}{2.453888in}}%
\pgfpathlineto{\pgfqpoint{2.694549in}{2.448049in}}%
\pgfpathlineto{\pgfqpoint{2.834227in}{2.422137in}}%
\pgfpathlineto{\pgfqpoint{2.864945in}{2.416437in}}%
\pgfpathlineto{\pgfqpoint{2.915495in}{2.406915in}}%
\pgfpathlineto{\pgfqpoint{3.303893in}{2.332369in}}%
\pgfpathlineto{\pgfqpoint{3.653687in}{2.263385in}}%
\pgfpathlineto{\pgfqpoint{4.042298in}{2.185086in}}%
\pgfpathlineto{\pgfqpoint{4.042361in}{2.237096in}}%
\pgfpathlineto{\pgfqpoint{4.043862in}{2.236772in}}%
\pgfpathlineto{\pgfqpoint{4.364561in}{2.167564in}}%
\pgfpathlineto{\pgfqpoint{4.366059in}{2.119280in}}%
\pgfpathlineto{\pgfqpoint{4.686823in}{2.052873in}}%
\pgfpathlineto{\pgfqpoint{4.686823in}{2.052873in}}%
\pgfusepath{stroke}%
\end{pgfscope}%
\begin{pgfscope}%
\pgfpathrectangle{\pgfqpoint{0.626312in}{0.402778in}}{\pgfqpoint{4.253869in}{2.937500in}}%
\pgfusepath{clip}%
\pgfsetrectcap%
\pgfsetroundjoin%
\pgfsetlinewidth{1.505625pt}%
\definecolor{currentstroke}{rgb}{0.121569,0.466667,0.705882}%
\pgfsetstrokecolor{currentstroke}%
\pgfsetdash{}{0pt}%
\pgfpathmoveto{\pgfqpoint{0.819670in}{0.388889in}}%
\pgfpathlineto{\pgfqpoint{0.821208in}{2.299693in}}%
\pgfpathlineto{\pgfqpoint{0.826207in}{2.389202in}}%
\pgfpathlineto{\pgfqpoint{0.837261in}{2.465437in}}%
\pgfpathlineto{\pgfqpoint{0.855773in}{2.521284in}}%
\pgfpathlineto{\pgfqpoint{0.876637in}{2.557001in}}%
\pgfpathlineto{\pgfqpoint{0.881591in}{2.563052in}}%
\pgfpathlineto{\pgfqpoint{0.884738in}{2.566799in}}%
\pgfpathlineto{\pgfqpoint{0.893617in}{2.575637in}}%
\pgfpathlineto{\pgfqpoint{0.897605in}{2.579282in}}%
\pgfpathlineto{\pgfqpoint{0.910090in}{2.589478in}}%
\pgfpathlineto{\pgfqpoint{0.913430in}{2.591840in}}%
\pgfpathlineto{\pgfqpoint{0.923807in}{2.598764in}}%
\pgfpathlineto{\pgfqpoint{0.932040in}{2.603502in}}%
\pgfpathlineto{\pgfqpoint{0.941691in}{2.608713in}}%
\pgfpathlineto{\pgfqpoint{0.969837in}{2.620648in}}%
\pgfpathlineto{\pgfqpoint{0.976664in}{2.623096in}}%
\pgfpathlineto{\pgfqpoint{0.981650in}{2.624531in}}%
\pgfpathlineto{\pgfqpoint{1.012143in}{2.632851in}}%
\pgfpathlineto{\pgfqpoint{1.040651in}{2.638514in}}%
\pgfpathlineto{\pgfqpoint{1.090076in}{2.645214in}}%
\pgfpathlineto{\pgfqpoint{1.136757in}{2.648822in}}%
\pgfpathlineto{\pgfqpoint{1.164455in}{2.649873in}}%
\pgfpathlineto{\pgfqpoint{1.174693in}{2.650132in}}%
\pgfpathlineto{\pgfqpoint{1.205042in}{2.650274in}}%
\pgfpathlineto{\pgfqpoint{1.298537in}{2.648199in}}%
\pgfpathlineto{\pgfqpoint{1.324351in}{2.646931in}}%
\pgfpathlineto{\pgfqpoint{1.335857in}{2.646282in}}%
\pgfpathlineto{\pgfqpoint{1.355664in}{2.645129in}}%
\pgfpathlineto{\pgfqpoint{1.368251in}{2.644263in}}%
\pgfpathlineto{\pgfqpoint{1.392922in}{2.642463in}}%
\pgfpathlineto{\pgfqpoint{1.447799in}{2.638033in}}%
\pgfpathlineto{\pgfqpoint{1.463909in}{2.636608in}}%
\pgfpathlineto{\pgfqpoint{1.493718in}{2.633686in}}%
\pgfpathlineto{\pgfqpoint{1.580587in}{2.624524in}}%
\pgfpathlineto{\pgfqpoint{1.588862in}{2.623609in}}%
\pgfpathlineto{\pgfqpoint{1.954872in}{2.573843in}}%
\pgfpathlineto{\pgfqpoint{2.282552in}{2.520901in}}%
\pgfpathlineto{\pgfqpoint{2.382416in}{2.503755in}}%
\pgfpathlineto{\pgfqpoint{2.456000in}{2.490920in}}%
\pgfpathlineto{\pgfqpoint{2.540002in}{2.476007in}}%
\pgfpathlineto{\pgfqpoint{2.584057in}{2.468117in}}%
\pgfpathlineto{\pgfqpoint{2.662526in}{2.453888in}}%
\pgfpathlineto{\pgfqpoint{2.694549in}{2.448049in}}%
\pgfpathlineto{\pgfqpoint{2.834227in}{2.422137in}}%
\pgfpathlineto{\pgfqpoint{2.864945in}{2.416437in}}%
\pgfpathlineto{\pgfqpoint{2.915495in}{2.406915in}}%
\pgfpathlineto{\pgfqpoint{3.303893in}{2.332369in}}%
\pgfpathlineto{\pgfqpoint{3.653687in}{2.263385in}}%
\pgfpathlineto{\pgfqpoint{4.364561in}{2.119276in}}%
\pgfpathlineto{\pgfqpoint{4.364566in}{2.130323in}}%
\pgfpathlineto{\pgfqpoint{4.366084in}{2.130073in}}%
\pgfpathlineto{\pgfqpoint{4.529423in}{2.095911in}}%
\pgfpathlineto{\pgfqpoint{4.686823in}{2.062847in}}%
\pgfpathlineto{\pgfqpoint{4.686823in}{2.062847in}}%
\pgfusepath{stroke}%
\end{pgfscope}%
\begin{pgfscope}%
\pgfpathrectangle{\pgfqpoint{0.626312in}{0.402778in}}{\pgfqpoint{4.253869in}{2.937500in}}%
\pgfusepath{clip}%
\pgfsetrectcap%
\pgfsetroundjoin%
\pgfsetlinewidth{1.505625pt}%
\definecolor{currentstroke}{rgb}{1.000000,0.498039,0.054902}%
\pgfsetstrokecolor{currentstroke}%
\pgfsetdash{}{0pt}%
\pgfpathmoveto{\pgfqpoint{0.819670in}{0.388889in}}%
\pgfpathlineto{\pgfqpoint{0.821208in}{2.299693in}}%
\pgfpathlineto{\pgfqpoint{0.826207in}{2.389202in}}%
\pgfpathlineto{\pgfqpoint{0.837261in}{2.465437in}}%
\pgfpathlineto{\pgfqpoint{0.855773in}{2.521284in}}%
\pgfpathlineto{\pgfqpoint{0.876637in}{2.557001in}}%
\pgfpathlineto{\pgfqpoint{0.881591in}{2.563052in}}%
\pgfpathlineto{\pgfqpoint{0.884738in}{2.566799in}}%
\pgfpathlineto{\pgfqpoint{0.893617in}{2.575637in}}%
\pgfpathlineto{\pgfqpoint{0.897605in}{2.579282in}}%
\pgfpathlineto{\pgfqpoint{0.910090in}{2.589478in}}%
\pgfpathlineto{\pgfqpoint{0.913430in}{2.591840in}}%
\pgfpathlineto{\pgfqpoint{0.923807in}{2.598764in}}%
\pgfpathlineto{\pgfqpoint{0.932040in}{2.603502in}}%
\pgfpathlineto{\pgfqpoint{0.941691in}{2.608713in}}%
\pgfpathlineto{\pgfqpoint{0.969837in}{2.620648in}}%
\pgfpathlineto{\pgfqpoint{0.976664in}{2.623096in}}%
\pgfpathlineto{\pgfqpoint{0.981650in}{2.624531in}}%
\pgfpathlineto{\pgfqpoint{1.012143in}{2.632851in}}%
\pgfpathlineto{\pgfqpoint{1.040651in}{2.638514in}}%
\pgfpathlineto{\pgfqpoint{1.090076in}{2.645214in}}%
\pgfpathlineto{\pgfqpoint{1.136757in}{2.648822in}}%
\pgfpathlineto{\pgfqpoint{1.164455in}{2.649873in}}%
\pgfpathlineto{\pgfqpoint{1.174693in}{2.650132in}}%
\pgfpathlineto{\pgfqpoint{1.205042in}{2.650274in}}%
\pgfpathlineto{\pgfqpoint{1.298537in}{2.648199in}}%
\pgfpathlineto{\pgfqpoint{1.324351in}{2.646931in}}%
\pgfpathlineto{\pgfqpoint{1.335857in}{2.646282in}}%
\pgfpathlineto{\pgfqpoint{1.355664in}{2.645129in}}%
\pgfpathlineto{\pgfqpoint{1.368251in}{2.644263in}}%
\pgfpathlineto{\pgfqpoint{1.392922in}{2.642463in}}%
\pgfpathlineto{\pgfqpoint{1.447799in}{2.638033in}}%
\pgfpathlineto{\pgfqpoint{1.463909in}{2.636608in}}%
\pgfpathlineto{\pgfqpoint{1.493718in}{2.633686in}}%
\pgfpathlineto{\pgfqpoint{1.580587in}{2.624524in}}%
\pgfpathlineto{\pgfqpoint{1.588862in}{2.623609in}}%
\pgfpathlineto{\pgfqpoint{1.954872in}{2.573843in}}%
\pgfpathlineto{\pgfqpoint{2.282552in}{2.520901in}}%
\pgfpathlineto{\pgfqpoint{2.382416in}{2.503755in}}%
\pgfpathlineto{\pgfqpoint{2.456000in}{2.490920in}}%
\pgfpathlineto{\pgfqpoint{2.540002in}{2.476007in}}%
\pgfpathlineto{\pgfqpoint{2.584057in}{2.468117in}}%
\pgfpathlineto{\pgfqpoint{2.662526in}{2.453888in}}%
\pgfpathlineto{\pgfqpoint{2.694549in}{2.448049in}}%
\pgfpathlineto{\pgfqpoint{2.834227in}{2.422137in}}%
\pgfpathlineto{\pgfqpoint{2.864945in}{2.416437in}}%
\pgfpathlineto{\pgfqpoint{2.915495in}{2.406915in}}%
\pgfpathlineto{\pgfqpoint{3.303893in}{2.332369in}}%
\pgfpathlineto{\pgfqpoint{3.653687in}{2.263385in}}%
\pgfpathlineto{\pgfqpoint{4.686823in}{2.052643in}}%
\pgfpathlineto{\pgfqpoint{4.686823in}{2.052643in}}%
\pgfusepath{stroke}%
\end{pgfscope}%
\begin{pgfscope}%
\pgfsetrectcap%
\pgfsetmiterjoin%
\pgfsetlinewidth{0.803000pt}%
\definecolor{currentstroke}{rgb}{0.000000,0.000000,0.000000}%
\pgfsetstrokecolor{currentstroke}%
\pgfsetdash{}{0pt}%
\pgfpathmoveto{\pgfqpoint{0.626312in}{0.402778in}}%
\pgfpathlineto{\pgfqpoint{0.626312in}{3.340278in}}%
\pgfusepath{stroke}%
\end{pgfscope}%
\begin{pgfscope}%
\pgfsetrectcap%
\pgfsetmiterjoin%
\pgfsetlinewidth{0.803000pt}%
\definecolor{currentstroke}{rgb}{0.000000,0.000000,0.000000}%
\pgfsetstrokecolor{currentstroke}%
\pgfsetdash{}{0pt}%
\pgfpathmoveto{\pgfqpoint{4.880181in}{0.402778in}}%
\pgfpathlineto{\pgfqpoint{4.880181in}{3.340278in}}%
\pgfusepath{stroke}%
\end{pgfscope}%
\begin{pgfscope}%
\pgfsetrectcap%
\pgfsetmiterjoin%
\pgfsetlinewidth{0.803000pt}%
\definecolor{currentstroke}{rgb}{0.000000,0.000000,0.000000}%
\pgfsetstrokecolor{currentstroke}%
\pgfsetdash{}{0pt}%
\pgfpathmoveto{\pgfqpoint{0.626312in}{0.402778in}}%
\pgfpathlineto{\pgfqpoint{4.880181in}{0.402778in}}%
\pgfusepath{stroke}%
\end{pgfscope}%
\begin{pgfscope}%
\pgfsetrectcap%
\pgfsetmiterjoin%
\pgfsetlinewidth{0.803000pt}%
\definecolor{currentstroke}{rgb}{0.000000,0.000000,0.000000}%
\pgfsetstrokecolor{currentstroke}%
\pgfsetdash{}{0pt}%
\pgfpathmoveto{\pgfqpoint{0.626312in}{3.340278in}}%
\pgfpathlineto{\pgfqpoint{4.880181in}{3.340278in}}%
\pgfusepath{stroke}%
\end{pgfscope}%
\begin{pgfscope}%
\pgfsetbuttcap%
\pgfsetmiterjoin%
\definecolor{currentfill}{rgb}{1.000000,1.000000,1.000000}%
\pgfsetfillcolor{currentfill}%
\pgfsetfillopacity{0.800000}%
\pgfsetlinewidth{1.003750pt}%
\definecolor{currentstroke}{rgb}{0.800000,0.800000,0.800000}%
\pgfsetstrokecolor{currentstroke}%
\pgfsetstrokeopacity{0.800000}%
\pgfsetdash{}{0pt}%
\pgfpathmoveto{\pgfqpoint{0.723534in}{0.672473in}}%
\pgfpathlineto{\pgfqpoint{1.306867in}{0.672473in}}%
\pgfpathquadraticcurveto{\pgfqpoint{1.334645in}{0.672473in}}{\pgfqpoint{1.334645in}{0.700251in}}%
\pgfpathlineto{\pgfqpoint{1.334645in}{3.042805in}}%
\pgfpathquadraticcurveto{\pgfqpoint{1.334645in}{3.070583in}}{\pgfqpoint{1.306867in}{3.070583in}}%
\pgfpathlineto{\pgfqpoint{0.723534in}{3.070583in}}%
\pgfpathquadraticcurveto{\pgfqpoint{0.695756in}{3.070583in}}{\pgfqpoint{0.695756in}{3.042805in}}%
\pgfpathlineto{\pgfqpoint{0.695756in}{0.700251in}}%
\pgfpathquadraticcurveto{\pgfqpoint{0.695756in}{0.672473in}}{\pgfqpoint{0.723534in}{0.672473in}}%
\pgfpathclose%
\pgfusepath{stroke,fill}%
\end{pgfscope}%
\begin{pgfscope}%
\pgfsetrectcap%
\pgfsetroundjoin%
\pgfsetlinewidth{1.505625pt}%
\definecolor{currentstroke}{rgb}{0.121569,0.466667,0.705882}%
\pgfsetstrokecolor{currentstroke}%
\pgfsetdash{}{0pt}%
\pgfpathmoveto{\pgfqpoint{0.751312in}{2.966416in}}%
\pgfpathlineto{\pgfqpoint{1.029090in}{2.966416in}}%
\pgfusepath{stroke}%
\end{pgfscope}%
\begin{pgfscope}%
\pgftext[x=1.140201in,y=2.917805in,left,base]{\rmfamily\fontsize{10.000000}{12.000000}\selectfont 1}%
\end{pgfscope}%
\begin{pgfscope}%
\pgfsetrectcap%
\pgfsetroundjoin%
\pgfsetlinewidth{1.505625pt}%
\definecolor{currentstroke}{rgb}{1.000000,0.498039,0.054902}%
\pgfsetstrokecolor{currentstroke}%
\pgfsetdash{}{0pt}%
\pgfpathmoveto{\pgfqpoint{0.751312in}{2.770046in}}%
\pgfpathlineto{\pgfqpoint{1.029090in}{2.770046in}}%
\pgfusepath{stroke}%
\end{pgfscope}%
\begin{pgfscope}%
\pgftext[x=1.140201in,y=2.721435in,left,base]{\rmfamily\fontsize{10.000000}{12.000000}\selectfont 2}%
\end{pgfscope}%
\begin{pgfscope}%
\pgfsetrectcap%
\pgfsetroundjoin%
\pgfsetlinewidth{1.505625pt}%
\definecolor{currentstroke}{rgb}{0.172549,0.627451,0.172549}%
\pgfsetstrokecolor{currentstroke}%
\pgfsetdash{}{0pt}%
\pgfpathmoveto{\pgfqpoint{0.751312in}{2.573676in}}%
\pgfpathlineto{\pgfqpoint{1.029090in}{2.573676in}}%
\pgfusepath{stroke}%
\end{pgfscope}%
\begin{pgfscope}%
\pgftext[x=1.140201in,y=2.525064in,left,base]{\rmfamily\fontsize{10.000000}{12.000000}\selectfont 3}%
\end{pgfscope}%
\begin{pgfscope}%
\pgfsetrectcap%
\pgfsetroundjoin%
\pgfsetlinewidth{1.505625pt}%
\definecolor{currentstroke}{rgb}{0.839216,0.152941,0.156863}%
\pgfsetstrokecolor{currentstroke}%
\pgfsetdash{}{0pt}%
\pgfpathmoveto{\pgfqpoint{0.751312in}{2.377305in}}%
\pgfpathlineto{\pgfqpoint{1.029090in}{2.377305in}}%
\pgfusepath{stroke}%
\end{pgfscope}%
\begin{pgfscope}%
\pgftext[x=1.140201in,y=2.328694in,left,base]{\rmfamily\fontsize{10.000000}{12.000000}\selectfont 4}%
\end{pgfscope}%
\begin{pgfscope}%
\pgfsetrectcap%
\pgfsetroundjoin%
\pgfsetlinewidth{1.505625pt}%
\definecolor{currentstroke}{rgb}{0.580392,0.403922,0.741176}%
\pgfsetstrokecolor{currentstroke}%
\pgfsetdash{}{0pt}%
\pgfpathmoveto{\pgfqpoint{0.751312in}{2.180935in}}%
\pgfpathlineto{\pgfqpoint{1.029090in}{2.180935in}}%
\pgfusepath{stroke}%
\end{pgfscope}%
\begin{pgfscope}%
\pgftext[x=1.140201in,y=2.132324in,left,base]{\rmfamily\fontsize{10.000000}{12.000000}\selectfont 5}%
\end{pgfscope}%
\begin{pgfscope}%
\pgfsetrectcap%
\pgfsetroundjoin%
\pgfsetlinewidth{1.505625pt}%
\definecolor{currentstroke}{rgb}{0.549020,0.337255,0.294118}%
\pgfsetstrokecolor{currentstroke}%
\pgfsetdash{}{0pt}%
\pgfpathmoveto{\pgfqpoint{0.751312in}{1.984565in}}%
\pgfpathlineto{\pgfqpoint{1.029090in}{1.984565in}}%
\pgfusepath{stroke}%
\end{pgfscope}%
\begin{pgfscope}%
\pgftext[x=1.140201in,y=1.935954in,left,base]{\rmfamily\fontsize{10.000000}{12.000000}\selectfont 6}%
\end{pgfscope}%
\begin{pgfscope}%
\pgfsetrectcap%
\pgfsetroundjoin%
\pgfsetlinewidth{1.505625pt}%
\definecolor{currentstroke}{rgb}{0.890196,0.466667,0.760784}%
\pgfsetstrokecolor{currentstroke}%
\pgfsetdash{}{0pt}%
\pgfpathmoveto{\pgfqpoint{0.751312in}{1.788194in}}%
\pgfpathlineto{\pgfqpoint{1.029090in}{1.788194in}}%
\pgfusepath{stroke}%
\end{pgfscope}%
\begin{pgfscope}%
\pgftext[x=1.140201in,y=1.739583in,left,base]{\rmfamily\fontsize{10.000000}{12.000000}\selectfont 7}%
\end{pgfscope}%
\begin{pgfscope}%
\pgfsetrectcap%
\pgfsetroundjoin%
\pgfsetlinewidth{1.505625pt}%
\definecolor{currentstroke}{rgb}{0.498039,0.498039,0.498039}%
\pgfsetstrokecolor{currentstroke}%
\pgfsetdash{}{0pt}%
\pgfpathmoveto{\pgfqpoint{0.751312in}{1.591824in}}%
\pgfpathlineto{\pgfqpoint{1.029090in}{1.591824in}}%
\pgfusepath{stroke}%
\end{pgfscope}%
\begin{pgfscope}%
\pgftext[x=1.140201in,y=1.543213in,left,base]{\rmfamily\fontsize{10.000000}{12.000000}\selectfont 8}%
\end{pgfscope}%
\begin{pgfscope}%
\pgfsetrectcap%
\pgfsetroundjoin%
\pgfsetlinewidth{1.505625pt}%
\definecolor{currentstroke}{rgb}{0.737255,0.741176,0.133333}%
\pgfsetstrokecolor{currentstroke}%
\pgfsetdash{}{0pt}%
\pgfpathmoveto{\pgfqpoint{0.751312in}{1.395454in}}%
\pgfpathlineto{\pgfqpoint{1.029090in}{1.395454in}}%
\pgfusepath{stroke}%
\end{pgfscope}%
\begin{pgfscope}%
\pgftext[x=1.140201in,y=1.346843in,left,base]{\rmfamily\fontsize{10.000000}{12.000000}\selectfont 9}%
\end{pgfscope}%
\begin{pgfscope}%
\pgfsetrectcap%
\pgfsetroundjoin%
\pgfsetlinewidth{1.505625pt}%
\definecolor{currentstroke}{rgb}{0.090196,0.745098,0.811765}%
\pgfsetstrokecolor{currentstroke}%
\pgfsetdash{}{0pt}%
\pgfpathmoveto{\pgfqpoint{0.751312in}{1.199084in}}%
\pgfpathlineto{\pgfqpoint{1.029090in}{1.199084in}}%
\pgfusepath{stroke}%
\end{pgfscope}%
\begin{pgfscope}%
\pgftext[x=1.140201in,y=1.150472in,left,base]{\rmfamily\fontsize{10.000000}{12.000000}\selectfont 10}%
\end{pgfscope}%
\begin{pgfscope}%
\pgfsetrectcap%
\pgfsetroundjoin%
\pgfsetlinewidth{1.505625pt}%
\definecolor{currentstroke}{rgb}{0.121569,0.466667,0.705882}%
\pgfsetstrokecolor{currentstroke}%
\pgfsetdash{}{0pt}%
\pgfpathmoveto{\pgfqpoint{0.751312in}{1.002713in}}%
\pgfpathlineto{\pgfqpoint{1.029090in}{1.002713in}}%
\pgfusepath{stroke}%
\end{pgfscope}%
\begin{pgfscope}%
\pgftext[x=1.140201in,y=0.954102in,left,base]{\rmfamily\fontsize{10.000000}{12.000000}\selectfont 11}%
\end{pgfscope}%
\begin{pgfscope}%
\pgfsetrectcap%
\pgfsetroundjoin%
\pgfsetlinewidth{1.505625pt}%
\definecolor{currentstroke}{rgb}{1.000000,0.498039,0.054902}%
\pgfsetstrokecolor{currentstroke}%
\pgfsetdash{}{0pt}%
\pgfpathmoveto{\pgfqpoint{0.751312in}{0.806343in}}%
\pgfpathlineto{\pgfqpoint{1.029090in}{0.806343in}}%
\pgfusepath{stroke}%
\end{pgfscope}%
\begin{pgfscope}%
\pgftext[x=1.140201in,y=0.757732in,left,base]{\rmfamily\fontsize{10.000000}{12.000000}\selectfont 12}%
\end{pgfscope}%
\end{pgfpicture}%
\makeatother%
\endgroup%

        %% Creator: Matplotlib, PGF backend
%%
%% To include the figure in your LaTeX document, write
%%   \input{<filename>.pgf}
%%
%% Make sure the required packages are loaded in your preamble
%%   \usepackage{pgf}
%%
%% Figures using additional raster images can only be included by \input if
%% they are in the same directory as the main LaTeX file. For loading figures
%% from other directories you can use the `import` package
%%   \usepackage{import}
%% and then include the figures with
%%   \import{<path to file>}{<filename>.pgf}
%%
%% Matplotlib used the following preamble
%%   \usepackage{fontspec}
%%   \setmainfont{Times New Roman}
%%   \setsansfont{Lucida Grande}
%%   \setmonofont{Andale Mono}
%%
\begingroup%
\makeatletter%
\begin{pgfpicture}%
\pgfpathrectangle{\pgfpointorigin}{\pgfqpoint{5.000000in}{3.388889in}}%
\pgfusepath{use as bounding box, clip}%
\begin{pgfscope}%
\pgfsetbuttcap%
\pgfsetmiterjoin%
\definecolor{currentfill}{rgb}{1.000000,1.000000,1.000000}%
\pgfsetfillcolor{currentfill}%
\pgfsetlinewidth{0.000000pt}%
\definecolor{currentstroke}{rgb}{1.000000,1.000000,1.000000}%
\pgfsetstrokecolor{currentstroke}%
\pgfsetdash{}{0pt}%
\pgfpathmoveto{\pgfqpoint{0.000000in}{0.000000in}}%
\pgfpathlineto{\pgfqpoint{5.000000in}{0.000000in}}%
\pgfpathlineto{\pgfqpoint{5.000000in}{3.388889in}}%
\pgfpathlineto{\pgfqpoint{0.000000in}{3.388889in}}%
\pgfpathclose%
\pgfusepath{fill}%
\end{pgfscope}%
\begin{pgfscope}%
\pgfsetbuttcap%
\pgfsetmiterjoin%
\definecolor{currentfill}{rgb}{1.000000,1.000000,1.000000}%
\pgfsetfillcolor{currentfill}%
\pgfsetlinewidth{0.000000pt}%
\definecolor{currentstroke}{rgb}{0.000000,0.000000,0.000000}%
\pgfsetstrokecolor{currentstroke}%
\pgfsetstrokeopacity{0.000000}%
\pgfsetdash{}{0pt}%
\pgfpathmoveto{\pgfqpoint{0.626312in}{0.402778in}}%
\pgfpathlineto{\pgfqpoint{4.951389in}{0.402778in}}%
\pgfpathlineto{\pgfqpoint{4.951389in}{3.340278in}}%
\pgfpathlineto{\pgfqpoint{0.626312in}{3.340278in}}%
\pgfpathclose%
\pgfusepath{fill}%
\end{pgfscope}%
\begin{pgfscope}%
\pgfsetbuttcap%
\pgfsetroundjoin%
\definecolor{currentfill}{rgb}{0.000000,0.000000,0.000000}%
\pgfsetfillcolor{currentfill}%
\pgfsetlinewidth{0.803000pt}%
\definecolor{currentstroke}{rgb}{0.000000,0.000000,0.000000}%
\pgfsetstrokecolor{currentstroke}%
\pgfsetdash{}{0pt}%
\pgfsys@defobject{currentmarker}{\pgfqpoint{0.000000in}{-0.048611in}}{\pgfqpoint{0.000000in}{0.000000in}}{%
\pgfpathmoveto{\pgfqpoint{0.000000in}{0.000000in}}%
\pgfpathlineto{\pgfqpoint{0.000000in}{-0.048611in}}%
\pgfusepath{stroke,fill}%
}%
\begin{pgfscope}%
\pgfsys@transformshift{0.822906in}{0.402778in}%
\pgfsys@useobject{currentmarker}{}%
\end{pgfscope}%
\end{pgfscope}%
\begin{pgfscope}%
\pgftext[x=0.822906in,y=0.305556in,,top]{\rmfamily\fontsize{10.000000}{12.000000}\selectfont \(\displaystyle 0.00\)}%
\end{pgfscope}%
\begin{pgfscope}%
\pgfsetbuttcap%
\pgfsetroundjoin%
\definecolor{currentfill}{rgb}{0.000000,0.000000,0.000000}%
\pgfsetfillcolor{currentfill}%
\pgfsetlinewidth{0.803000pt}%
\definecolor{currentstroke}{rgb}{0.000000,0.000000,0.000000}%
\pgfsetstrokecolor{currentstroke}%
\pgfsetdash{}{0pt}%
\pgfsys@defobject{currentmarker}{\pgfqpoint{0.000000in}{-0.048611in}}{\pgfqpoint{0.000000in}{0.000000in}}{%
\pgfpathmoveto{\pgfqpoint{0.000000in}{0.000000in}}%
\pgfpathlineto{\pgfqpoint{0.000000in}{-0.048611in}}%
\pgfusepath{stroke,fill}%
}%
\begin{pgfscope}%
\pgfsys@transformshift{1.314390in}{0.402778in}%
\pgfsys@useobject{currentmarker}{}%
\end{pgfscope}%
\end{pgfscope}%
\begin{pgfscope}%
\pgftext[x=1.314390in,y=0.305556in,,top]{\rmfamily\fontsize{10.000000}{12.000000}\selectfont \(\displaystyle 0.25\)}%
\end{pgfscope}%
\begin{pgfscope}%
\pgfsetbuttcap%
\pgfsetroundjoin%
\definecolor{currentfill}{rgb}{0.000000,0.000000,0.000000}%
\pgfsetfillcolor{currentfill}%
\pgfsetlinewidth{0.803000pt}%
\definecolor{currentstroke}{rgb}{0.000000,0.000000,0.000000}%
\pgfsetstrokecolor{currentstroke}%
\pgfsetdash{}{0pt}%
\pgfsys@defobject{currentmarker}{\pgfqpoint{0.000000in}{-0.048611in}}{\pgfqpoint{0.000000in}{0.000000in}}{%
\pgfpathmoveto{\pgfqpoint{0.000000in}{0.000000in}}%
\pgfpathlineto{\pgfqpoint{0.000000in}{-0.048611in}}%
\pgfusepath{stroke,fill}%
}%
\begin{pgfscope}%
\pgfsys@transformshift{1.805873in}{0.402778in}%
\pgfsys@useobject{currentmarker}{}%
\end{pgfscope}%
\end{pgfscope}%
\begin{pgfscope}%
\pgftext[x=1.805873in,y=0.305556in,,top]{\rmfamily\fontsize{10.000000}{12.000000}\selectfont \(\displaystyle 0.50\)}%
\end{pgfscope}%
\begin{pgfscope}%
\pgfsetbuttcap%
\pgfsetroundjoin%
\definecolor{currentfill}{rgb}{0.000000,0.000000,0.000000}%
\pgfsetfillcolor{currentfill}%
\pgfsetlinewidth{0.803000pt}%
\definecolor{currentstroke}{rgb}{0.000000,0.000000,0.000000}%
\pgfsetstrokecolor{currentstroke}%
\pgfsetdash{}{0pt}%
\pgfsys@defobject{currentmarker}{\pgfqpoint{0.000000in}{-0.048611in}}{\pgfqpoint{0.000000in}{0.000000in}}{%
\pgfpathmoveto{\pgfqpoint{0.000000in}{0.000000in}}%
\pgfpathlineto{\pgfqpoint{0.000000in}{-0.048611in}}%
\pgfusepath{stroke,fill}%
}%
\begin{pgfscope}%
\pgfsys@transformshift{2.297357in}{0.402778in}%
\pgfsys@useobject{currentmarker}{}%
\end{pgfscope}%
\end{pgfscope}%
\begin{pgfscope}%
\pgftext[x=2.297357in,y=0.305556in,,top]{\rmfamily\fontsize{10.000000}{12.000000}\selectfont \(\displaystyle 0.75\)}%
\end{pgfscope}%
\begin{pgfscope}%
\pgfsetbuttcap%
\pgfsetroundjoin%
\definecolor{currentfill}{rgb}{0.000000,0.000000,0.000000}%
\pgfsetfillcolor{currentfill}%
\pgfsetlinewidth{0.803000pt}%
\definecolor{currentstroke}{rgb}{0.000000,0.000000,0.000000}%
\pgfsetstrokecolor{currentstroke}%
\pgfsetdash{}{0pt}%
\pgfsys@defobject{currentmarker}{\pgfqpoint{0.000000in}{-0.048611in}}{\pgfqpoint{0.000000in}{0.000000in}}{%
\pgfpathmoveto{\pgfqpoint{0.000000in}{0.000000in}}%
\pgfpathlineto{\pgfqpoint{0.000000in}{-0.048611in}}%
\pgfusepath{stroke,fill}%
}%
\begin{pgfscope}%
\pgfsys@transformshift{2.788841in}{0.402778in}%
\pgfsys@useobject{currentmarker}{}%
\end{pgfscope}%
\end{pgfscope}%
\begin{pgfscope}%
\pgftext[x=2.788841in,y=0.305556in,,top]{\rmfamily\fontsize{10.000000}{12.000000}\selectfont \(\displaystyle 1.00\)}%
\end{pgfscope}%
\begin{pgfscope}%
\pgfsetbuttcap%
\pgfsetroundjoin%
\definecolor{currentfill}{rgb}{0.000000,0.000000,0.000000}%
\pgfsetfillcolor{currentfill}%
\pgfsetlinewidth{0.803000pt}%
\definecolor{currentstroke}{rgb}{0.000000,0.000000,0.000000}%
\pgfsetstrokecolor{currentstroke}%
\pgfsetdash{}{0pt}%
\pgfsys@defobject{currentmarker}{\pgfqpoint{0.000000in}{-0.048611in}}{\pgfqpoint{0.000000in}{0.000000in}}{%
\pgfpathmoveto{\pgfqpoint{0.000000in}{0.000000in}}%
\pgfpathlineto{\pgfqpoint{0.000000in}{-0.048611in}}%
\pgfusepath{stroke,fill}%
}%
\begin{pgfscope}%
\pgfsys@transformshift{3.280324in}{0.402778in}%
\pgfsys@useobject{currentmarker}{}%
\end{pgfscope}%
\end{pgfscope}%
\begin{pgfscope}%
\pgftext[x=3.280324in,y=0.305556in,,top]{\rmfamily\fontsize{10.000000}{12.000000}\selectfont \(\displaystyle 1.25\)}%
\end{pgfscope}%
\begin{pgfscope}%
\pgfsetbuttcap%
\pgfsetroundjoin%
\definecolor{currentfill}{rgb}{0.000000,0.000000,0.000000}%
\pgfsetfillcolor{currentfill}%
\pgfsetlinewidth{0.803000pt}%
\definecolor{currentstroke}{rgb}{0.000000,0.000000,0.000000}%
\pgfsetstrokecolor{currentstroke}%
\pgfsetdash{}{0pt}%
\pgfsys@defobject{currentmarker}{\pgfqpoint{0.000000in}{-0.048611in}}{\pgfqpoint{0.000000in}{0.000000in}}{%
\pgfpathmoveto{\pgfqpoint{0.000000in}{0.000000in}}%
\pgfpathlineto{\pgfqpoint{0.000000in}{-0.048611in}}%
\pgfusepath{stroke,fill}%
}%
\begin{pgfscope}%
\pgfsys@transformshift{3.771808in}{0.402778in}%
\pgfsys@useobject{currentmarker}{}%
\end{pgfscope}%
\end{pgfscope}%
\begin{pgfscope}%
\pgftext[x=3.771808in,y=0.305556in,,top]{\rmfamily\fontsize{10.000000}{12.000000}\selectfont \(\displaystyle 1.50\)}%
\end{pgfscope}%
\begin{pgfscope}%
\pgfsetbuttcap%
\pgfsetroundjoin%
\definecolor{currentfill}{rgb}{0.000000,0.000000,0.000000}%
\pgfsetfillcolor{currentfill}%
\pgfsetlinewidth{0.803000pt}%
\definecolor{currentstroke}{rgb}{0.000000,0.000000,0.000000}%
\pgfsetstrokecolor{currentstroke}%
\pgfsetdash{}{0pt}%
\pgfsys@defobject{currentmarker}{\pgfqpoint{0.000000in}{-0.048611in}}{\pgfqpoint{0.000000in}{0.000000in}}{%
\pgfpathmoveto{\pgfqpoint{0.000000in}{0.000000in}}%
\pgfpathlineto{\pgfqpoint{0.000000in}{-0.048611in}}%
\pgfusepath{stroke,fill}%
}%
\begin{pgfscope}%
\pgfsys@transformshift{4.263291in}{0.402778in}%
\pgfsys@useobject{currentmarker}{}%
\end{pgfscope}%
\end{pgfscope}%
\begin{pgfscope}%
\pgftext[x=4.263291in,y=0.305556in,,top]{\rmfamily\fontsize{10.000000}{12.000000}\selectfont \(\displaystyle 1.75\)}%
\end{pgfscope}%
\begin{pgfscope}%
\pgfsetbuttcap%
\pgfsetroundjoin%
\definecolor{currentfill}{rgb}{0.000000,0.000000,0.000000}%
\pgfsetfillcolor{currentfill}%
\pgfsetlinewidth{0.803000pt}%
\definecolor{currentstroke}{rgb}{0.000000,0.000000,0.000000}%
\pgfsetstrokecolor{currentstroke}%
\pgfsetdash{}{0pt}%
\pgfsys@defobject{currentmarker}{\pgfqpoint{0.000000in}{-0.048611in}}{\pgfqpoint{0.000000in}{0.000000in}}{%
\pgfpathmoveto{\pgfqpoint{0.000000in}{0.000000in}}%
\pgfpathlineto{\pgfqpoint{0.000000in}{-0.048611in}}%
\pgfusepath{stroke,fill}%
}%
\begin{pgfscope}%
\pgfsys@transformshift{4.754775in}{0.402778in}%
\pgfsys@useobject{currentmarker}{}%
\end{pgfscope}%
\end{pgfscope}%
\begin{pgfscope}%
\pgftext[x=4.754775in,y=0.305556in,,top]{\rmfamily\fontsize{10.000000}{12.000000}\selectfont \(\displaystyle 2.00\)}%
\end{pgfscope}%
\begin{pgfscope}%
\pgftext[x=2.788850in,y=0.123861in,,top]{\rmfamily\fontsize{10.000000}{12.000000}\selectfont t}%
\end{pgfscope}%
\begin{pgfscope}%
\pgftext[x=4.951389in,y=0.137750in,right,top]{\rmfamily\fontsize{10.000000}{12.000000}\selectfont \(\displaystyle \times10^{-7}\)}%
\end{pgfscope}%
\begin{pgfscope}%
\pgfsetbuttcap%
\pgfsetroundjoin%
\definecolor{currentfill}{rgb}{0.000000,0.000000,0.000000}%
\pgfsetfillcolor{currentfill}%
\pgfsetlinewidth{0.803000pt}%
\definecolor{currentstroke}{rgb}{0.000000,0.000000,0.000000}%
\pgfsetstrokecolor{currentstroke}%
\pgfsetdash{}{0pt}%
\pgfsys@defobject{currentmarker}{\pgfqpoint{-0.048611in}{0.000000in}}{\pgfqpoint{0.000000in}{0.000000in}}{%
\pgfpathmoveto{\pgfqpoint{0.000000in}{0.000000in}}%
\pgfpathlineto{\pgfqpoint{-0.048611in}{0.000000in}}%
\pgfusepath{stroke,fill}%
}%
\begin{pgfscope}%
\pgfsys@transformshift{0.626312in}{0.794932in}%
\pgfsys@useobject{currentmarker}{}%
\end{pgfscope}%
\end{pgfscope}%
\begin{pgfscope}%
\pgftext[x=0.185724in,y=0.746714in,left,base]{\rmfamily\fontsize{10.000000}{12.000000}\selectfont \(\displaystyle 10^{-11}\)}%
\end{pgfscope}%
\begin{pgfscope}%
\pgfsetbuttcap%
\pgfsetroundjoin%
\definecolor{currentfill}{rgb}{0.000000,0.000000,0.000000}%
\pgfsetfillcolor{currentfill}%
\pgfsetlinewidth{0.803000pt}%
\definecolor{currentstroke}{rgb}{0.000000,0.000000,0.000000}%
\pgfsetstrokecolor{currentstroke}%
\pgfsetdash{}{0pt}%
\pgfsys@defobject{currentmarker}{\pgfqpoint{-0.048611in}{0.000000in}}{\pgfqpoint{0.000000in}{0.000000in}}{%
\pgfpathmoveto{\pgfqpoint{0.000000in}{0.000000in}}%
\pgfpathlineto{\pgfqpoint{-0.048611in}{0.000000in}}%
\pgfusepath{stroke,fill}%
}%
\begin{pgfscope}%
\pgfsys@transformshift{0.626312in}{1.264581in}%
\pgfsys@useobject{currentmarker}{}%
\end{pgfscope}%
\end{pgfscope}%
\begin{pgfscope}%
\pgftext[x=0.241087in,y=1.216363in,left,base]{\rmfamily\fontsize{10.000000}{12.000000}\selectfont \(\displaystyle 10^{-9}\)}%
\end{pgfscope}%
\begin{pgfscope}%
\pgfsetbuttcap%
\pgfsetroundjoin%
\definecolor{currentfill}{rgb}{0.000000,0.000000,0.000000}%
\pgfsetfillcolor{currentfill}%
\pgfsetlinewidth{0.803000pt}%
\definecolor{currentstroke}{rgb}{0.000000,0.000000,0.000000}%
\pgfsetstrokecolor{currentstroke}%
\pgfsetdash{}{0pt}%
\pgfsys@defobject{currentmarker}{\pgfqpoint{-0.048611in}{0.000000in}}{\pgfqpoint{0.000000in}{0.000000in}}{%
\pgfpathmoveto{\pgfqpoint{0.000000in}{0.000000in}}%
\pgfpathlineto{\pgfqpoint{-0.048611in}{0.000000in}}%
\pgfusepath{stroke,fill}%
}%
\begin{pgfscope}%
\pgfsys@transformshift{0.626312in}{1.734231in}%
\pgfsys@useobject{currentmarker}{}%
\end{pgfscope}%
\end{pgfscope}%
\begin{pgfscope}%
\pgftext[x=0.241087in,y=1.686013in,left,base]{\rmfamily\fontsize{10.000000}{12.000000}\selectfont \(\displaystyle 10^{-7}\)}%
\end{pgfscope}%
\begin{pgfscope}%
\pgfsetbuttcap%
\pgfsetroundjoin%
\definecolor{currentfill}{rgb}{0.000000,0.000000,0.000000}%
\pgfsetfillcolor{currentfill}%
\pgfsetlinewidth{0.803000pt}%
\definecolor{currentstroke}{rgb}{0.000000,0.000000,0.000000}%
\pgfsetstrokecolor{currentstroke}%
\pgfsetdash{}{0pt}%
\pgfsys@defobject{currentmarker}{\pgfqpoint{-0.048611in}{0.000000in}}{\pgfqpoint{0.000000in}{0.000000in}}{%
\pgfpathmoveto{\pgfqpoint{0.000000in}{0.000000in}}%
\pgfpathlineto{\pgfqpoint{-0.048611in}{0.000000in}}%
\pgfusepath{stroke,fill}%
}%
\begin{pgfscope}%
\pgfsys@transformshift{0.626312in}{2.203881in}%
\pgfsys@useobject{currentmarker}{}%
\end{pgfscope}%
\end{pgfscope}%
\begin{pgfscope}%
\pgftext[x=0.241087in,y=2.155663in,left,base]{\rmfamily\fontsize{10.000000}{12.000000}\selectfont \(\displaystyle 10^{-5}\)}%
\end{pgfscope}%
\begin{pgfscope}%
\pgfsetbuttcap%
\pgfsetroundjoin%
\definecolor{currentfill}{rgb}{0.000000,0.000000,0.000000}%
\pgfsetfillcolor{currentfill}%
\pgfsetlinewidth{0.803000pt}%
\definecolor{currentstroke}{rgb}{0.000000,0.000000,0.000000}%
\pgfsetstrokecolor{currentstroke}%
\pgfsetdash{}{0pt}%
\pgfsys@defobject{currentmarker}{\pgfqpoint{-0.048611in}{0.000000in}}{\pgfqpoint{0.000000in}{0.000000in}}{%
\pgfpathmoveto{\pgfqpoint{0.000000in}{0.000000in}}%
\pgfpathlineto{\pgfqpoint{-0.048611in}{0.000000in}}%
\pgfusepath{stroke,fill}%
}%
\begin{pgfscope}%
\pgfsys@transformshift{0.626312in}{2.673530in}%
\pgfsys@useobject{currentmarker}{}%
\end{pgfscope}%
\end{pgfscope}%
\begin{pgfscope}%
\pgftext[x=0.241087in,y=2.625313in,left,base]{\rmfamily\fontsize{10.000000}{12.000000}\selectfont \(\displaystyle 10^{-3}\)}%
\end{pgfscope}%
\begin{pgfscope}%
\pgfsetbuttcap%
\pgfsetroundjoin%
\definecolor{currentfill}{rgb}{0.000000,0.000000,0.000000}%
\pgfsetfillcolor{currentfill}%
\pgfsetlinewidth{0.803000pt}%
\definecolor{currentstroke}{rgb}{0.000000,0.000000,0.000000}%
\pgfsetstrokecolor{currentstroke}%
\pgfsetdash{}{0pt}%
\pgfsys@defobject{currentmarker}{\pgfqpoint{-0.048611in}{0.000000in}}{\pgfqpoint{0.000000in}{0.000000in}}{%
\pgfpathmoveto{\pgfqpoint{0.000000in}{0.000000in}}%
\pgfpathlineto{\pgfqpoint{-0.048611in}{0.000000in}}%
\pgfusepath{stroke,fill}%
}%
\begin{pgfscope}%
\pgfsys@transformshift{0.626312in}{3.143180in}%
\pgfsys@useobject{currentmarker}{}%
\end{pgfscope}%
\end{pgfscope}%
\begin{pgfscope}%
\pgftext[x=0.241087in,y=3.094962in,left,base]{\rmfamily\fontsize{10.000000}{12.000000}\selectfont \(\displaystyle 10^{-1}\)}%
\end{pgfscope}%
\begin{pgfscope}%
\pgftext[x=0.130169in,y=1.871528in,,bottom,rotate=90.000000]{\rmfamily\fontsize{10.000000}{12.000000}\selectfont E(t)}%
\end{pgfscope}%
\begin{pgfscope}%
\pgfpathrectangle{\pgfqpoint{0.626312in}{0.402778in}}{\pgfqpoint{4.325077in}{2.937500in}}%
\pgfusepath{clip}%
\pgfsetrectcap%
\pgfsetroundjoin%
\pgfsetlinewidth{1.505625pt}%
\definecolor{currentstroke}{rgb}{0.121569,0.466667,0.705882}%
\pgfsetstrokecolor{currentstroke}%
\pgfsetdash{}{0pt}%
\pgfpathmoveto{\pgfqpoint{0.822906in}{0.649986in}}%
\pgfpathlineto{\pgfqpoint{0.823711in}{0.707948in}}%
\pgfpathlineto{\pgfqpoint{0.824517in}{0.573982in}}%
\pgfpathlineto{\pgfqpoint{0.826127in}{0.715701in}}%
\pgfpathlineto{\pgfqpoint{0.829348in}{0.711309in}}%
\pgfpathlineto{\pgfqpoint{0.835790in}{0.536301in}}%
\pgfpathlineto{\pgfqpoint{0.848674in}{0.636924in}}%
\pgfpathlineto{\pgfqpoint{0.862246in}{0.646736in}}%
\pgfpathlineto{\pgfqpoint{0.862431in}{0.653381in}}%
\pgfpathlineto{\pgfqpoint{0.862630in}{0.649516in}}%
\pgfpathlineto{\pgfqpoint{0.863026in}{0.596597in}}%
\pgfpathlineto{\pgfqpoint{0.863818in}{0.689047in}}%
\pgfpathlineto{\pgfqpoint{0.865403in}{0.709551in}}%
\pgfpathlineto{\pgfqpoint{0.868572in}{0.646270in}}%
\pgfpathlineto{\pgfqpoint{0.874911in}{0.707448in}}%
\pgfpathlineto{\pgfqpoint{0.876179in}{0.815750in}}%
\pgfpathlineto{\pgfqpoint{0.878714in}{0.752954in}}%
\pgfpathlineto{\pgfqpoint{0.879221in}{0.819063in}}%
\pgfpathlineto{\pgfqpoint{0.880562in}{0.793288in}}%
\pgfpathlineto{\pgfqpoint{0.880574in}{0.785349in}}%
\pgfpathlineto{\pgfqpoint{0.881976in}{2.623259in}}%
\pgfpathlineto{\pgfqpoint{0.885971in}{2.784899in}}%
\pgfpathlineto{\pgfqpoint{0.889163in}{2.825893in}}%
\pgfpathlineto{\pgfqpoint{0.890533in}{2.829876in}}%
\pgfpathlineto{\pgfqpoint{0.892986in}{2.853735in}}%
\pgfpathlineto{\pgfqpoint{0.894887in}{2.862345in}}%
\pgfpathlineto{\pgfqpoint{0.895418in}{2.861902in}}%
\pgfpathlineto{\pgfqpoint{0.896288in}{2.861729in}}%
\pgfpathlineto{\pgfqpoint{0.896646in}{2.862180in}}%
\pgfpathlineto{\pgfqpoint{0.898781in}{2.868170in}}%
\pgfpathlineto{\pgfqpoint{0.900218in}{2.866193in}}%
\pgfpathlineto{\pgfqpoint{0.900728in}{2.865649in}}%
\pgfpathlineto{\pgfqpoint{0.901191in}{2.867219in}}%
\pgfpathlineto{\pgfqpoint{0.901544in}{2.866599in}}%
\pgfpathlineto{\pgfqpoint{0.901662in}{2.904528in}}%
\pgfpathlineto{\pgfqpoint{0.903176in}{2.895187in}}%
\pgfpathlineto{\pgfqpoint{0.920384in}{2.799845in}}%
\pgfpathlineto{\pgfqpoint{0.935432in}{2.733622in}}%
\pgfpathlineto{\pgfqpoint{0.935449in}{2.733652in}}%
\pgfpathlineto{\pgfqpoint{0.939676in}{2.721851in}}%
\pgfpathlineto{\pgfqpoint{0.939677in}{2.721939in}}%
\pgfpathlineto{\pgfqpoint{0.941205in}{2.719332in}}%
\pgfpathlineto{\pgfqpoint{0.941206in}{2.719507in}}%
\pgfpathlineto{\pgfqpoint{0.944031in}{2.713813in}}%
\pgfpathlineto{\pgfqpoint{0.944032in}{2.713999in}}%
\pgfpathlineto{\pgfqpoint{0.945522in}{2.708663in}}%
\pgfpathlineto{\pgfqpoint{0.952469in}{2.690475in}}%
\pgfpathlineto{\pgfqpoint{0.952470in}{2.690664in}}%
\pgfpathlineto{\pgfqpoint{0.953963in}{2.686893in}}%
\pgfpathlineto{\pgfqpoint{0.960807in}{2.664586in}}%
\pgfpathlineto{\pgfqpoint{0.960808in}{2.664943in}}%
\pgfpathlineto{\pgfqpoint{0.962242in}{2.661231in}}%
\pgfpathlineto{\pgfqpoint{0.962295in}{2.661351in}}%
\pgfpathlineto{\pgfqpoint{0.965732in}{2.653010in}}%
\pgfpathlineto{\pgfqpoint{0.965734in}{2.653397in}}%
\pgfpathlineto{\pgfqpoint{0.967219in}{2.648997in}}%
\pgfpathlineto{\pgfqpoint{0.978852in}{2.603065in}}%
\pgfpathlineto{\pgfqpoint{0.987597in}{2.568874in}}%
\pgfpathlineto{\pgfqpoint{0.992347in}{2.540622in}}%
\pgfpathlineto{\pgfqpoint{0.996697in}{2.513985in}}%
\pgfpathlineto{\pgfqpoint{1.001280in}{2.491157in}}%
\pgfpathlineto{\pgfqpoint{1.002746in}{2.485708in}}%
\pgfpathlineto{\pgfqpoint{1.021470in}{2.437668in}}%
\pgfpathlineto{\pgfqpoint{1.022377in}{2.436644in}}%
\pgfpathlineto{\pgfqpoint{1.022943in}{2.437561in}}%
\pgfpathlineto{\pgfqpoint{1.023459in}{2.438012in}}%
\pgfpathlineto{\pgfqpoint{1.024026in}{2.437125in}}%
\pgfpathlineto{\pgfqpoint{1.036548in}{2.404490in}}%
\pgfpathlineto{\pgfqpoint{1.051353in}{2.364787in}}%
\pgfpathlineto{\pgfqpoint{1.091721in}{2.253122in}}%
\pgfpathlineto{\pgfqpoint{1.105858in}{2.211113in}}%
\pgfpathlineto{\pgfqpoint{1.118742in}{2.171987in}}%
\pgfpathlineto{\pgfqpoint{1.205255in}{1.897846in}}%
\pgfpathlineto{\pgfqpoint{1.259800in}{1.725398in}}%
\pgfpathlineto{\pgfqpoint{1.259800in}{3.147479in}}%
\pgfpathlineto{\pgfqpoint{1.261343in}{3.141286in}}%
\pgfpathlineto{\pgfqpoint{1.308004in}{2.959110in}}%
\pgfpathlineto{\pgfqpoint{1.395778in}{2.625426in}}%
\pgfpathlineto{\pgfqpoint{1.403521in}{2.604815in}}%
\pgfpathlineto{\pgfqpoint{1.409611in}{2.596254in}}%
\pgfpathlineto{\pgfqpoint{1.411375in}{2.593599in}}%
\pgfpathlineto{\pgfqpoint{1.411389in}{2.649711in}}%
\pgfpathlineto{\pgfqpoint{1.412778in}{2.318410in}}%
\pgfpathlineto{\pgfqpoint{1.414994in}{2.638013in}}%
\pgfpathlineto{\pgfqpoint{1.418101in}{2.813983in}}%
\pgfpathlineto{\pgfqpoint{1.422302in}{2.840706in}}%
\pgfpathlineto{\pgfqpoint{1.422647in}{2.841512in}}%
\pgfpathlineto{\pgfqpoint{1.424215in}{2.849241in}}%
\pgfpathlineto{\pgfqpoint{1.424667in}{2.848767in}}%
\pgfpathlineto{\pgfqpoint{1.425259in}{2.849243in}}%
\pgfpathlineto{\pgfqpoint{1.428886in}{2.860853in}}%
\pgfpathlineto{\pgfqpoint{1.430141in}{2.858620in}}%
\pgfpathlineto{\pgfqpoint{1.431060in}{2.859354in}}%
\pgfpathlineto{\pgfqpoint{1.431475in}{2.860480in}}%
\pgfpathlineto{\pgfqpoint{1.432181in}{2.858608in}}%
\pgfpathlineto{\pgfqpoint{1.433611in}{2.850422in}}%
\pgfpathlineto{\pgfqpoint{1.451900in}{2.748190in}}%
\pgfpathlineto{\pgfqpoint{1.451903in}{2.748265in}}%
\pgfpathlineto{\pgfqpoint{1.463807in}{2.697978in}}%
\pgfpathlineto{\pgfqpoint{1.463829in}{2.698244in}}%
\pgfpathlineto{\pgfqpoint{1.465330in}{2.692146in}}%
\pgfpathlineto{\pgfqpoint{1.470754in}{2.673388in}}%
\pgfpathlineto{\pgfqpoint{1.471138in}{2.673566in}}%
\pgfpathlineto{\pgfqpoint{1.472522in}{2.674657in}}%
\pgfpathlineto{\pgfqpoint{1.471332in}{2.673555in}}%
\pgfpathlineto{\pgfqpoint{1.472671in}{2.674255in}}%
\pgfpathlineto{\pgfqpoint{1.472927in}{2.674806in}}%
\pgfpathlineto{\pgfqpoint{1.475241in}{2.671356in}}%
\pgfpathlineto{\pgfqpoint{1.475242in}{2.671634in}}%
\pgfpathlineto{\pgfqpoint{1.476680in}{2.670063in}}%
\pgfpathlineto{\pgfqpoint{1.476732in}{2.670138in}}%
\pgfpathlineto{\pgfqpoint{1.476784in}{2.670253in}}%
\pgfpathlineto{\pgfqpoint{1.481105in}{2.661595in}}%
\pgfpathlineto{\pgfqpoint{1.481105in}{2.661854in}}%
\pgfpathlineto{\pgfqpoint{1.482598in}{2.659585in}}%
\pgfpathlineto{\pgfqpoint{1.482650in}{2.659639in}}%
\pgfpathlineto{\pgfqpoint{1.485020in}{2.656144in}}%
\pgfpathlineto{\pgfqpoint{1.485021in}{2.656406in}}%
\pgfpathlineto{\pgfqpoint{1.486470in}{2.651393in}}%
\pgfpathlineto{\pgfqpoint{1.495079in}{2.629102in}}%
\pgfpathlineto{\pgfqpoint{1.495081in}{2.629598in}}%
\pgfpathlineto{\pgfqpoint{1.496517in}{2.626421in}}%
\pgfpathlineto{\pgfqpoint{1.496570in}{2.626660in}}%
\pgfpathlineto{\pgfqpoint{1.504763in}{2.595629in}}%
\pgfpathlineto{\pgfqpoint{1.508319in}{2.581572in}}%
\pgfpathlineto{\pgfqpoint{1.516868in}{2.548894in}}%
\pgfpathlineto{\pgfqpoint{1.517910in}{2.544091in}}%
\pgfpathlineto{\pgfqpoint{1.524220in}{2.509662in}}%
\pgfpathlineto{\pgfqpoint{1.526500in}{2.503964in}}%
\pgfpathlineto{\pgfqpoint{1.526513in}{2.503985in}}%
\pgfpathlineto{\pgfqpoint{1.528184in}{2.499794in}}%
\pgfpathlineto{\pgfqpoint{1.542050in}{2.454260in}}%
\pgfpathlineto{\pgfqpoint{1.542524in}{2.455198in}}%
\pgfpathlineto{\pgfqpoint{1.543107in}{2.455000in}}%
\pgfpathlineto{\pgfqpoint{1.543397in}{2.454245in}}%
\pgfpathlineto{\pgfqpoint{1.544544in}{2.451533in}}%
\pgfpathlineto{\pgfqpoint{1.545127in}{2.453265in}}%
\pgfpathlineto{\pgfqpoint{1.546593in}{2.456074in}}%
\pgfpathlineto{\pgfqpoint{1.547005in}{2.454907in}}%
\pgfpathlineto{\pgfqpoint{1.549660in}{2.450463in}}%
\pgfpathlineto{\pgfqpoint{1.552458in}{2.458731in}}%
\pgfpathlineto{\pgfqpoint{1.552912in}{2.457682in}}%
\pgfpathlineto{\pgfqpoint{1.554825in}{2.456193in}}%
\pgfpathlineto{\pgfqpoint{1.556155in}{2.454840in}}%
\pgfpathlineto{\pgfqpoint{1.559780in}{2.462206in}}%
\pgfpathlineto{\pgfqpoint{1.560734in}{2.460784in}}%
\pgfpathlineto{\pgfqpoint{1.561949in}{2.458265in}}%
\pgfpathlineto{\pgfqpoint{1.562905in}{2.459519in}}%
\pgfpathlineto{\pgfqpoint{1.565820in}{2.462481in}}%
\pgfpathlineto{\pgfqpoint{1.569283in}{2.458249in}}%
\pgfpathlineto{\pgfqpoint{1.577671in}{2.474295in}}%
\pgfpathlineto{\pgfqpoint{1.583525in}{2.479301in}}%
\pgfpathlineto{\pgfqpoint{1.589469in}{2.488385in}}%
\pgfpathlineto{\pgfqpoint{1.595187in}{2.505832in}}%
\pgfpathlineto{\pgfqpoint{1.601873in}{2.517912in}}%
\pgfpathlineto{\pgfqpoint{1.609534in}{2.504216in}}%
\pgfpathlineto{\pgfqpoint{1.615517in}{2.494361in}}%
\pgfpathlineto{\pgfqpoint{1.644498in}{2.438156in}}%
\pgfpathlineto{\pgfqpoint{1.651471in}{2.428881in}}%
\pgfpathlineto{\pgfqpoint{1.679101in}{2.367270in}}%
\pgfpathlineto{\pgfqpoint{1.689287in}{2.349975in}}%
\pgfpathlineto{\pgfqpoint{1.696674in}{2.331694in}}%
\pgfpathlineto{\pgfqpoint{1.696675in}{3.142754in}}%
\pgfpathlineto{\pgfqpoint{1.698218in}{3.136574in}}%
\pgfpathlineto{\pgfqpoint{1.747927in}{2.942759in}}%
\pgfpathlineto{\pgfqpoint{1.796070in}{2.762717in}}%
\pgfpathlineto{\pgfqpoint{1.823794in}{2.667719in}}%
\pgfpathlineto{\pgfqpoint{1.850631in}{2.585079in}}%
\pgfpathlineto{\pgfqpoint{1.855386in}{2.577021in}}%
\pgfpathlineto{\pgfqpoint{1.863540in}{2.573428in}}%
\pgfpathlineto{\pgfqpoint{1.864347in}{2.459576in}}%
\pgfpathlineto{\pgfqpoint{1.864943in}{2.545001in}}%
\pgfpathlineto{\pgfqpoint{1.867743in}{2.706341in}}%
\pgfpathlineto{\pgfqpoint{1.871083in}{2.817212in}}%
\pgfpathlineto{\pgfqpoint{1.880065in}{2.862670in}}%
\pgfpathlineto{\pgfqpoint{1.880523in}{2.862036in}}%
\pgfpathlineto{\pgfqpoint{1.881952in}{2.860439in}}%
\pgfpathlineto{\pgfqpoint{1.882273in}{2.860648in}}%
\pgfpathlineto{\pgfqpoint{1.883855in}{2.865071in}}%
\pgfpathlineto{\pgfqpoint{1.884423in}{2.863727in}}%
\pgfpathlineto{\pgfqpoint{1.884511in}{2.863321in}}%
\pgfpathlineto{\pgfqpoint{1.884629in}{2.906911in}}%
\pgfpathlineto{\pgfqpoint{1.886147in}{2.897541in}}%
\pgfpathlineto{\pgfqpoint{1.903569in}{2.801078in}}%
\pgfpathlineto{\pgfqpoint{1.918264in}{2.736946in}}%
\pgfpathlineto{\pgfqpoint{1.922555in}{2.724956in}}%
\pgfpathlineto{\pgfqpoint{1.925214in}{2.720742in}}%
\pgfpathlineto{\pgfqpoint{1.925215in}{2.721107in}}%
\pgfpathlineto{\pgfqpoint{1.926731in}{2.716987in}}%
\pgfpathlineto{\pgfqpoint{1.929994in}{2.707948in}}%
\pgfpathlineto{\pgfqpoint{1.929995in}{2.708129in}}%
\pgfpathlineto{\pgfqpoint{1.931486in}{2.703120in}}%
\pgfpathlineto{\pgfqpoint{1.935347in}{2.694125in}}%
\pgfpathlineto{\pgfqpoint{1.935348in}{2.694308in}}%
\pgfpathlineto{\pgfqpoint{1.936842in}{2.690616in}}%
\pgfpathlineto{\pgfqpoint{1.947887in}{2.657944in}}%
\pgfpathlineto{\pgfqpoint{1.947889in}{2.658314in}}%
\pgfpathlineto{\pgfqpoint{1.949325in}{2.654473in}}%
\pgfpathlineto{\pgfqpoint{1.949378in}{2.654602in}}%
\pgfpathlineto{\pgfqpoint{1.961019in}{2.609163in}}%
\pgfpathlineto{\pgfqpoint{1.970199in}{2.573844in}}%
\pgfpathlineto{\pgfqpoint{1.973970in}{2.552767in}}%
\pgfpathlineto{\pgfqpoint{1.977164in}{2.533563in}}%
\pgfpathlineto{\pgfqpoint{1.986081in}{2.487076in}}%
\pgfpathlineto{\pgfqpoint{1.994386in}{2.463097in}}%
\pgfpathlineto{\pgfqpoint{1.996902in}{2.457790in}}%
\pgfpathlineto{\pgfqpoint{2.005577in}{2.441321in}}%
\pgfpathlineto{\pgfqpoint{2.034374in}{2.369452in}}%
\pgfpathlineto{\pgfqpoint{2.050375in}{2.323965in}}%
\pgfpathlineto{\pgfqpoint{2.056035in}{2.310890in}}%
\pgfpathlineto{\pgfqpoint{2.064006in}{2.290326in}}%
\pgfpathlineto{\pgfqpoint{2.087945in}{2.220954in}}%
\pgfpathlineto{\pgfqpoint{2.089588in}{2.218118in}}%
\pgfpathlineto{\pgfqpoint{2.119284in}{2.126516in}}%
\pgfpathlineto{\pgfqpoint{2.122836in}{2.115563in}}%
\pgfpathlineto{\pgfqpoint{2.133549in}{2.082448in}}%
\pgfpathlineto{\pgfqpoint{2.133549in}{3.147594in}}%
\pgfpathlineto{\pgfqpoint{2.135058in}{3.141538in}}%
\pgfpathlineto{\pgfqpoint{2.184695in}{2.947858in}}%
\pgfpathlineto{\pgfqpoint{2.366056in}{2.257962in}}%
\pgfpathlineto{\pgfqpoint{2.384995in}{2.193313in}}%
\pgfpathlineto{\pgfqpoint{2.391388in}{2.182761in}}%
\pgfpathlineto{\pgfqpoint{2.394342in}{2.177952in}}%
\pgfpathlineto{\pgfqpoint{2.395746in}{2.599826in}}%
\pgfpathlineto{\pgfqpoint{2.398105in}{2.695395in}}%
\pgfpathlineto{\pgfqpoint{2.401289in}{2.809966in}}%
\pgfpathlineto{\pgfqpoint{2.405855in}{2.835466in}}%
\pgfpathlineto{\pgfqpoint{2.412403in}{2.865930in}}%
\pgfpathlineto{\pgfqpoint{2.415694in}{2.858491in}}%
\pgfpathlineto{\pgfqpoint{2.433261in}{2.758899in}}%
\pgfpathlineto{\pgfqpoint{2.433262in}{2.758954in}}%
\pgfpathlineto{\pgfqpoint{2.434766in}{2.752095in}}%
\pgfpathlineto{\pgfqpoint{2.446890in}{2.701054in}}%
\pgfpathlineto{\pgfqpoint{2.446890in}{2.701340in}}%
\pgfpathlineto{\pgfqpoint{2.448415in}{2.695220in}}%
\pgfpathlineto{\pgfqpoint{2.452718in}{2.683205in}}%
\pgfpathlineto{\pgfqpoint{2.452718in}{2.683338in}}%
\pgfpathlineto{\pgfqpoint{2.454135in}{2.681248in}}%
\pgfpathlineto{\pgfqpoint{2.454242in}{2.681571in}}%
\pgfpathlineto{\pgfqpoint{2.455520in}{2.682043in}}%
\pgfpathlineto{\pgfqpoint{2.455721in}{2.681529in}}%
\pgfpathlineto{\pgfqpoint{2.455925in}{2.682150in}}%
\pgfpathlineto{\pgfqpoint{2.457211in}{2.679924in}}%
\pgfpathlineto{\pgfqpoint{2.457238in}{2.680082in}}%
\pgfpathlineto{\pgfqpoint{2.460295in}{2.672246in}}%
\pgfpathlineto{\pgfqpoint{2.460296in}{2.672506in}}%
\pgfpathlineto{\pgfqpoint{2.461787in}{2.667558in}}%
\pgfpathlineto{\pgfqpoint{2.463485in}{2.663552in}}%
\pgfpathlineto{\pgfqpoint{2.463511in}{2.663707in}}%
\pgfpathlineto{\pgfqpoint{2.466162in}{2.659644in}}%
\pgfpathlineto{\pgfqpoint{2.466163in}{2.659901in}}%
\pgfpathlineto{\pgfqpoint{2.467605in}{2.657189in}}%
\pgfpathlineto{\pgfqpoint{2.467657in}{2.657207in}}%
\pgfpathlineto{\pgfqpoint{2.467709in}{2.657257in}}%
\pgfpathlineto{\pgfqpoint{2.474387in}{2.635963in}}%
\pgfpathlineto{\pgfqpoint{2.474389in}{2.636439in}}%
\pgfpathlineto{\pgfqpoint{2.475822in}{2.633395in}}%
\pgfpathlineto{\pgfqpoint{2.475875in}{2.633620in}}%
\pgfpathlineto{\pgfqpoint{2.478490in}{2.628549in}}%
\pgfpathlineto{\pgfqpoint{2.478492in}{2.629045in}}%
\pgfpathlineto{\pgfqpoint{2.479928in}{2.625847in}}%
\pgfpathlineto{\pgfqpoint{2.479981in}{2.626086in}}%
\pgfpathlineto{\pgfqpoint{2.489374in}{2.589430in}}%
\pgfpathlineto{\pgfqpoint{2.495288in}{2.565781in}}%
\pgfpathlineto{\pgfqpoint{2.496916in}{2.559989in}}%
\pgfpathlineto{\pgfqpoint{2.496916in}{2.559994in}}%
\pgfpathlineto{\pgfqpoint{2.506079in}{2.510492in}}%
\pgfpathlineto{\pgfqpoint{2.509259in}{2.488031in}}%
\pgfpathlineto{\pgfqpoint{2.519318in}{2.446214in}}%
\pgfpathlineto{\pgfqpoint{2.520617in}{2.443295in}}%
\pgfpathlineto{\pgfqpoint{2.521062in}{2.444778in}}%
\pgfpathlineto{\pgfqpoint{2.521421in}{2.445300in}}%
\pgfpathlineto{\pgfqpoint{2.522204in}{2.443738in}}%
\pgfpathlineto{\pgfqpoint{2.525217in}{2.438511in}}%
\pgfpathlineto{\pgfqpoint{2.525957in}{2.439491in}}%
\pgfpathlineto{\pgfqpoint{2.526626in}{2.438099in}}%
\pgfpathlineto{\pgfqpoint{2.529512in}{2.433350in}}%
\pgfpathlineto{\pgfqpoint{2.533424in}{2.433551in}}%
\pgfpathlineto{\pgfqpoint{2.537404in}{2.429143in}}%
\pgfpathlineto{\pgfqpoint{2.539244in}{2.427350in}}%
\pgfpathlineto{\pgfqpoint{2.540291in}{2.425833in}}%
\pgfpathlineto{\pgfqpoint{2.540931in}{2.426803in}}%
\pgfpathlineto{\pgfqpoint{2.548719in}{2.454767in}}%
\pgfpathlineto{\pgfqpoint{2.551840in}{2.457772in}}%
\pgfpathlineto{\pgfqpoint{2.554857in}{2.464780in}}%
\pgfpathlineto{\pgfqpoint{2.559877in}{2.462441in}}%
\pgfpathlineto{\pgfqpoint{2.563383in}{2.466600in}}%
\pgfpathlineto{\pgfqpoint{2.566295in}{2.465772in}}%
\pgfpathlineto{\pgfqpoint{2.568866in}{2.470455in}}%
\pgfpathlineto{\pgfqpoint{2.570423in}{2.469207in}}%
\pgfpathlineto{\pgfqpoint{2.570423in}{3.206755in}}%
\pgfpathlineto{\pgfqpoint{2.571966in}{3.200259in}}%
\pgfpathlineto{\pgfqpoint{2.606393in}{3.062240in}}%
\pgfpathlineto{\pgfqpoint{2.675946in}{2.796088in}}%
\pgfpathlineto{\pgfqpoint{2.715424in}{2.661481in}}%
\pgfpathlineto{\pgfqpoint{2.734548in}{2.600877in}}%
\pgfpathlineto{\pgfqpoint{2.776679in}{2.487452in}}%
\pgfpathlineto{\pgfqpoint{2.799228in}{2.427287in}}%
\pgfpathlineto{\pgfqpoint{2.814059in}{2.386820in}}%
\pgfpathlineto{\pgfqpoint{2.837422in}{2.336780in}}%
\pgfpathlineto{\pgfqpoint{2.845832in}{2.338435in}}%
\pgfpathlineto{\pgfqpoint{2.846503in}{2.338177in}}%
\pgfpathlineto{\pgfqpoint{2.846517in}{2.669014in}}%
\pgfpathlineto{\pgfqpoint{2.847446in}{2.620434in}}%
\pgfpathlineto{\pgfqpoint{2.848524in}{2.428330in}}%
\pgfpathlineto{\pgfqpoint{2.849338in}{2.509867in}}%
\pgfpathlineto{\pgfqpoint{2.850524in}{2.649922in}}%
\pgfpathlineto{\pgfqpoint{2.853672in}{2.795814in}}%
\pgfpathlineto{\pgfqpoint{2.857851in}{2.838757in}}%
\pgfpathlineto{\pgfqpoint{2.861021in}{2.868855in}}%
\pgfpathlineto{\pgfqpoint{2.865282in}{2.863267in}}%
\pgfpathlineto{\pgfqpoint{2.865631in}{2.863654in}}%
\pgfpathlineto{\pgfqpoint{2.866170in}{2.862294in}}%
\pgfpathlineto{\pgfqpoint{2.866988in}{2.860786in}}%
\pgfpathlineto{\pgfqpoint{2.867478in}{2.862883in}}%
\pgfpathlineto{\pgfqpoint{2.867596in}{2.901943in}}%
\pgfpathlineto{\pgfqpoint{2.869111in}{2.892617in}}%
\pgfpathlineto{\pgfqpoint{2.885398in}{2.802729in}}%
\pgfpathlineto{\pgfqpoint{2.885398in}{2.802816in}}%
\pgfpathlineto{\pgfqpoint{2.886904in}{2.795617in}}%
\pgfpathlineto{\pgfqpoint{2.901010in}{2.734436in}}%
\pgfpathlineto{\pgfqpoint{2.904784in}{2.724675in}}%
\pgfpathlineto{\pgfqpoint{2.904784in}{2.724763in}}%
\pgfpathlineto{\pgfqpoint{2.906308in}{2.721327in}}%
\pgfpathlineto{\pgfqpoint{2.906310in}{2.721495in}}%
\pgfpathlineto{\pgfqpoint{2.908866in}{2.717201in}}%
\pgfpathlineto{\pgfqpoint{2.908867in}{2.717363in}}%
\pgfpathlineto{\pgfqpoint{2.910357in}{2.713158in}}%
\pgfpathlineto{\pgfqpoint{2.916374in}{2.698113in}}%
\pgfpathlineto{\pgfqpoint{2.916375in}{2.698294in}}%
\pgfpathlineto{\pgfqpoint{2.917868in}{2.694874in}}%
\pgfpathlineto{\pgfqpoint{2.928629in}{2.663213in}}%
\pgfpathlineto{\pgfqpoint{2.928631in}{2.663571in}}%
\pgfpathlineto{\pgfqpoint{2.930065in}{2.659762in}}%
\pgfpathlineto{\pgfqpoint{2.930118in}{2.659881in}}%
\pgfpathlineto{\pgfqpoint{2.937994in}{2.631786in}}%
\pgfpathlineto{\pgfqpoint{2.942638in}{2.613914in}}%
\pgfpathlineto{\pgfqpoint{2.946764in}{2.597375in}}%
\pgfpathlineto{\pgfqpoint{2.962430in}{2.518461in}}%
\pgfpathlineto{\pgfqpoint{2.965373in}{2.502573in}}%
\pgfpathlineto{\pgfqpoint{2.987860in}{2.437473in}}%
\pgfpathlineto{\pgfqpoint{3.002341in}{2.402110in}}%
\pgfpathlineto{\pgfqpoint{3.007297in}{2.388113in}}%
\pgfpathlineto{\pgfqpoint{3.007298in}{3.197954in}}%
\pgfpathlineto{\pgfqpoint{3.008841in}{3.191496in}}%
\pgfpathlineto{\pgfqpoint{3.044478in}{3.049245in}}%
\pgfpathlineto{\pgfqpoint{3.117297in}{2.770393in}}%
\pgfpathlineto{\pgfqpoint{3.308808in}{2.044667in}}%
\pgfpathlineto{\pgfqpoint{3.363694in}{1.842740in}}%
\pgfpathlineto{\pgfqpoint{3.370413in}{1.827893in}}%
\pgfpathlineto{\pgfqpoint{3.375317in}{1.822596in}}%
\pgfpathlineto{\pgfqpoint{3.377309in}{1.821444in}}%
\pgfpathlineto{\pgfqpoint{3.377336in}{2.653341in}}%
\pgfpathlineto{\pgfqpoint{3.378713in}{2.380755in}}%
\pgfpathlineto{\pgfqpoint{3.381071in}{2.648664in}}%
\pgfpathlineto{\pgfqpoint{3.384257in}{2.799443in}}%
\pgfpathlineto{\pgfqpoint{3.388812in}{2.829272in}}%
\pgfpathlineto{\pgfqpoint{3.393658in}{2.851397in}}%
\pgfpathlineto{\pgfqpoint{3.394037in}{2.852392in}}%
\pgfpathlineto{\pgfqpoint{3.395165in}{2.861174in}}%
\pgfpathlineto{\pgfqpoint{3.396073in}{2.859978in}}%
\pgfpathlineto{\pgfqpoint{3.397363in}{2.859589in}}%
\pgfpathlineto{\pgfqpoint{3.397753in}{2.860483in}}%
\pgfpathlineto{\pgfqpoint{3.398429in}{2.858174in}}%
\pgfpathlineto{\pgfqpoint{3.416279in}{2.756370in}}%
\pgfpathlineto{\pgfqpoint{3.427575in}{2.708282in}}%
\pgfpathlineto{\pgfqpoint{3.434207in}{2.686021in}}%
\pgfpathlineto{\pgfqpoint{3.434208in}{2.686155in}}%
\pgfpathlineto{\pgfqpoint{3.435711in}{2.683720in}}%
\pgfpathlineto{\pgfqpoint{3.435737in}{2.683745in}}%
\pgfpathlineto{\pgfqpoint{3.435763in}{2.683783in}}%
\pgfpathlineto{\pgfqpoint{3.437715in}{2.681916in}}%
\pgfpathlineto{\pgfqpoint{3.438717in}{2.682396in}}%
\pgfpathlineto{\pgfqpoint{3.439253in}{2.681614in}}%
\pgfpathlineto{\pgfqpoint{3.444172in}{2.670911in}}%
\pgfpathlineto{\pgfqpoint{3.444172in}{3.158299in}}%
\pgfpathlineto{\pgfqpoint{3.445714in}{3.151496in}}%
\pgfpathlineto{\pgfqpoint{3.473192in}{3.037460in}}%
\pgfpathlineto{\pgfqpoint{3.521597in}{2.848481in}}%
\pgfpathlineto{\pgfqpoint{3.556730in}{2.718421in}}%
\pgfpathlineto{\pgfqpoint{3.578070in}{2.650638in}}%
\pgfpathlineto{\pgfqpoint{3.595930in}{2.597594in}}%
\pgfpathlineto{\pgfqpoint{3.616166in}{2.540703in}}%
\pgfpathlineto{\pgfqpoint{3.630236in}{2.508061in}}%
\pgfpathlineto{\pgfqpoint{3.634146in}{2.496208in}}%
\pgfpathlineto{\pgfqpoint{3.646565in}{2.464510in}}%
\pgfpathlineto{\pgfqpoint{3.656641in}{2.434850in}}%
\pgfpathlineto{\pgfqpoint{3.693083in}{2.346130in}}%
\pgfpathlineto{\pgfqpoint{3.716559in}{2.313016in}}%
\pgfpathlineto{\pgfqpoint{3.745813in}{2.252699in}}%
\pgfpathlineto{\pgfqpoint{3.769616in}{2.215460in}}%
\pgfpathlineto{\pgfqpoint{3.803110in}{2.173776in}}%
\pgfpathlineto{\pgfqpoint{3.815846in}{2.138216in}}%
\pgfpathlineto{\pgfqpoint{3.823381in}{2.127942in}}%
\pgfpathlineto{\pgfqpoint{3.829475in}{2.126417in}}%
\pgfpathlineto{\pgfqpoint{3.830877in}{2.623713in}}%
\pgfpathlineto{\pgfqpoint{3.834872in}{2.786275in}}%
\pgfpathlineto{\pgfqpoint{3.838065in}{2.826700in}}%
\pgfpathlineto{\pgfqpoint{3.839435in}{2.830558in}}%
\pgfpathlineto{\pgfqpoint{3.841881in}{2.853991in}}%
\pgfpathlineto{\pgfqpoint{3.843797in}{2.862731in}}%
\pgfpathlineto{\pgfqpoint{3.844353in}{2.862430in}}%
\pgfpathlineto{\pgfqpoint{3.845635in}{2.863050in}}%
\pgfpathlineto{\pgfqpoint{3.846599in}{2.866754in}}%
\pgfpathlineto{\pgfqpoint{3.848998in}{2.883875in}}%
\pgfpathlineto{\pgfqpoint{3.849238in}{2.883371in}}%
\pgfpathlineto{\pgfqpoint{3.851487in}{2.870118in}}%
\pgfpathlineto{\pgfqpoint{3.868965in}{2.772134in}}%
\pgfpathlineto{\pgfqpoint{3.881046in}{2.719445in}}%
\pgfpathlineto{\pgfqpoint{3.881046in}{3.043775in}}%
\pgfpathlineto{\pgfqpoint{3.882589in}{3.036797in}}%
\pgfpathlineto{\pgfqpoint{3.904374in}{2.945296in}}%
\pgfpathlineto{\pgfqpoint{3.939917in}{2.805135in}}%
\pgfpathlineto{\pgfqpoint{3.996578in}{2.591285in}}%
\pgfpathlineto{\pgfqpoint{4.216396in}{1.800692in}}%
\pgfpathlineto{\pgfqpoint{4.222269in}{1.789664in}}%
\pgfpathlineto{\pgfqpoint{4.250742in}{1.702319in}}%
\pgfpathlineto{\pgfqpoint{4.257978in}{1.696119in}}%
\pgfpathlineto{\pgfqpoint{4.268949in}{1.668179in}}%
\pgfpathlineto{\pgfqpoint{4.280447in}{1.635969in}}%
\pgfpathlineto{\pgfqpoint{4.317920in}{1.554332in}}%
\pgfpathlineto{\pgfqpoint{4.317920in}{2.910728in}}%
\pgfpathlineto{\pgfqpoint{4.319294in}{2.905539in}}%
\pgfpathlineto{\pgfqpoint{4.350615in}{2.792695in}}%
\pgfpathlineto{\pgfqpoint{4.357170in}{2.781527in}}%
\pgfpathlineto{\pgfqpoint{4.360277in}{2.778638in}}%
\pgfpathlineto{\pgfqpoint{4.360278in}{2.780279in}}%
\pgfpathlineto{\pgfqpoint{4.360284in}{2.785777in}}%
\pgfpathlineto{\pgfqpoint{4.361682in}{2.746476in}}%
\pgfpathlineto{\pgfqpoint{4.365257in}{2.596198in}}%
\pgfpathlineto{\pgfqpoint{4.367787in}{2.755965in}}%
\pgfpathlineto{\pgfqpoint{4.370464in}{2.789935in}}%
\pgfpathlineto{\pgfqpoint{4.376489in}{2.827370in}}%
\pgfpathlineto{\pgfqpoint{4.378106in}{2.839430in}}%
\pgfpathlineto{\pgfqpoint{4.378707in}{2.838764in}}%
\pgfpathlineto{\pgfqpoint{4.378938in}{2.838689in}}%
\pgfpathlineto{\pgfqpoint{4.379302in}{2.839971in}}%
\pgfpathlineto{\pgfqpoint{4.380422in}{2.841876in}}%
\pgfpathlineto{\pgfqpoint{4.381274in}{2.839817in}}%
\pgfpathlineto{\pgfqpoint{4.399896in}{2.733267in}}%
\pgfpathlineto{\pgfqpoint{4.412951in}{2.678169in}}%
\pgfpathlineto{\pgfqpoint{4.412952in}{2.678396in}}%
\pgfpathlineto{\pgfqpoint{4.414476in}{2.672054in}}%
\pgfpathlineto{\pgfqpoint{4.418039in}{2.663351in}}%
\pgfpathlineto{\pgfqpoint{4.418053in}{2.663459in}}%
\pgfpathlineto{\pgfqpoint{4.419643in}{2.661886in}}%
\pgfpathlineto{\pgfqpoint{4.419927in}{2.662279in}}%
\pgfpathlineto{\pgfqpoint{4.421412in}{2.664240in}}%
\pgfpathlineto{\pgfqpoint{4.420027in}{2.662211in}}%
\pgfpathlineto{\pgfqpoint{4.421460in}{2.664055in}}%
\pgfpathlineto{\pgfqpoint{4.421817in}{2.664743in}}%
\pgfpathlineto{\pgfqpoint{4.423411in}{2.662921in}}%
\pgfpathlineto{\pgfqpoint{4.423412in}{2.663231in}}%
\pgfpathlineto{\pgfqpoint{4.424902in}{2.661955in}}%
\pgfpathlineto{\pgfqpoint{4.424954in}{2.662105in}}%
\pgfpathlineto{\pgfqpoint{4.430200in}{2.651288in}}%
\pgfpathlineto{\pgfqpoint{4.430201in}{2.651574in}}%
\pgfpathlineto{\pgfqpoint{4.431642in}{2.649474in}}%
\pgfpathlineto{\pgfqpoint{4.431694in}{2.649565in}}%
\pgfpathlineto{\pgfqpoint{4.441304in}{2.624929in}}%
\pgfpathlineto{\pgfqpoint{4.441306in}{2.625457in}}%
\pgfpathlineto{\pgfqpoint{4.442740in}{2.622682in}}%
\pgfpathlineto{\pgfqpoint{4.442793in}{2.622950in}}%
\pgfpathlineto{\pgfqpoint{4.445203in}{2.618585in}}%
\pgfpathlineto{\pgfqpoint{4.445307in}{2.618625in}}%
\pgfpathlineto{\pgfqpoint{4.445411in}{2.618781in}}%
\pgfpathlineto{\pgfqpoint{4.468735in}{2.523405in}}%
\pgfpathlineto{\pgfqpoint{4.481777in}{2.445726in}}%
\pgfpathlineto{\pgfqpoint{4.485172in}{2.437473in}}%
\pgfpathlineto{\pgfqpoint{4.485524in}{2.438193in}}%
\pgfpathlineto{\pgfqpoint{4.486259in}{2.436681in}}%
\pgfpathlineto{\pgfqpoint{4.490982in}{2.427897in}}%
\pgfpathlineto{\pgfqpoint{4.491349in}{2.428555in}}%
\pgfpathlineto{\pgfqpoint{4.492120in}{2.427161in}}%
\pgfpathlineto{\pgfqpoint{4.493079in}{2.425981in}}%
\pgfpathlineto{\pgfqpoint{4.494006in}{2.424849in}}%
\pgfpathlineto{\pgfqpoint{4.494332in}{2.425083in}}%
\pgfpathlineto{\pgfqpoint{4.495233in}{2.427693in}}%
\pgfpathlineto{\pgfqpoint{4.496248in}{2.426799in}}%
\pgfpathlineto{\pgfqpoint{4.497600in}{2.430039in}}%
\pgfpathlineto{\pgfqpoint{4.498439in}{2.428123in}}%
\pgfpathlineto{\pgfqpoint{4.498737in}{2.427920in}}%
\pgfpathlineto{\pgfqpoint{4.501068in}{2.434658in}}%
\pgfpathlineto{\pgfqpoint{4.501878in}{2.432992in}}%
\pgfpathlineto{\pgfqpoint{4.503768in}{2.432400in}}%
\pgfpathlineto{\pgfqpoint{4.505435in}{2.429414in}}%
\pgfpathlineto{\pgfqpoint{4.512755in}{2.444349in}}%
\pgfpathlineto{\pgfqpoint{4.517245in}{2.448670in}}%
\pgfpathlineto{\pgfqpoint{4.524382in}{2.470752in}}%
\pgfpathlineto{\pgfqpoint{4.528725in}{2.479536in}}%
\pgfpathlineto{\pgfqpoint{4.540239in}{2.495387in}}%
\pgfpathlineto{\pgfqpoint{4.546537in}{2.512643in}}%
\pgfpathlineto{\pgfqpoint{4.551918in}{2.515836in}}%
\pgfpathlineto{\pgfqpoint{4.591215in}{2.449123in}}%
\pgfpathlineto{\pgfqpoint{4.598941in}{2.439848in}}%
\pgfpathlineto{\pgfqpoint{4.610433in}{2.416518in}}%
\pgfpathlineto{\pgfqpoint{4.621155in}{2.403062in}}%
\pgfpathlineto{\pgfqpoint{4.634347in}{2.371548in}}%
\pgfpathlineto{\pgfqpoint{4.645160in}{2.351823in}}%
\pgfpathlineto{\pgfqpoint{4.662255in}{2.309443in}}%
\pgfpathlineto{\pgfqpoint{4.678564in}{2.283956in}}%
\pgfpathlineto{\pgfqpoint{4.707746in}{2.227273in}}%
\pgfpathlineto{\pgfqpoint{4.730059in}{2.193010in}}%
\pgfpathlineto{\pgfqpoint{4.754794in}{2.161053in}}%
\pgfpathlineto{\pgfqpoint{4.754794in}{2.161053in}}%
\pgfusepath{stroke}%
\end{pgfscope}%
\begin{pgfscope}%
\pgfpathrectangle{\pgfqpoint{0.626312in}{0.402778in}}{\pgfqpoint{4.325077in}{2.937500in}}%
\pgfusepath{clip}%
\pgfsetrectcap%
\pgfsetroundjoin%
\pgfsetlinewidth{1.505625pt}%
\definecolor{currentstroke}{rgb}{1.000000,0.498039,0.054902}%
\pgfsetstrokecolor{currentstroke}%
\pgfsetdash{}{0pt}%
\pgfpathmoveto{\pgfqpoint{0.822906in}{0.649986in}}%
\pgfpathlineto{\pgfqpoint{0.823711in}{0.707948in}}%
\pgfpathlineto{\pgfqpoint{0.824517in}{0.573982in}}%
\pgfpathlineto{\pgfqpoint{0.826127in}{0.715701in}}%
\pgfpathlineto{\pgfqpoint{0.829348in}{0.711309in}}%
\pgfpathlineto{\pgfqpoint{0.835790in}{0.536301in}}%
\pgfpathlineto{\pgfqpoint{0.848674in}{0.636924in}}%
\pgfpathlineto{\pgfqpoint{0.862246in}{0.646736in}}%
\pgfpathlineto{\pgfqpoint{0.862431in}{0.653381in}}%
\pgfpathlineto{\pgfqpoint{0.862630in}{0.649516in}}%
\pgfpathlineto{\pgfqpoint{0.863026in}{0.596597in}}%
\pgfpathlineto{\pgfqpoint{0.863818in}{0.689047in}}%
\pgfpathlineto{\pgfqpoint{0.865403in}{0.709551in}}%
\pgfpathlineto{\pgfqpoint{0.868572in}{0.646270in}}%
\pgfpathlineto{\pgfqpoint{0.874911in}{0.707448in}}%
\pgfpathlineto{\pgfqpoint{0.876179in}{0.815750in}}%
\pgfpathlineto{\pgfqpoint{0.878714in}{0.752954in}}%
\pgfpathlineto{\pgfqpoint{0.879221in}{0.819063in}}%
\pgfpathlineto{\pgfqpoint{0.880562in}{0.793288in}}%
\pgfpathlineto{\pgfqpoint{0.880574in}{0.785349in}}%
\pgfpathlineto{\pgfqpoint{0.881976in}{2.623259in}}%
\pgfpathlineto{\pgfqpoint{0.885971in}{2.784899in}}%
\pgfpathlineto{\pgfqpoint{0.889163in}{2.825893in}}%
\pgfpathlineto{\pgfqpoint{0.890533in}{2.829876in}}%
\pgfpathlineto{\pgfqpoint{0.892986in}{2.853735in}}%
\pgfpathlineto{\pgfqpoint{0.894887in}{2.862345in}}%
\pgfpathlineto{\pgfqpoint{0.895418in}{2.861902in}}%
\pgfpathlineto{\pgfqpoint{0.896288in}{2.861729in}}%
\pgfpathlineto{\pgfqpoint{0.896646in}{2.862180in}}%
\pgfpathlineto{\pgfqpoint{0.898781in}{2.868170in}}%
\pgfpathlineto{\pgfqpoint{0.900218in}{2.866193in}}%
\pgfpathlineto{\pgfqpoint{0.900728in}{2.865649in}}%
\pgfpathlineto{\pgfqpoint{0.901191in}{2.867219in}}%
\pgfpathlineto{\pgfqpoint{0.901544in}{2.866599in}}%
\pgfpathlineto{\pgfqpoint{0.901662in}{2.904528in}}%
\pgfpathlineto{\pgfqpoint{0.903176in}{2.895187in}}%
\pgfpathlineto{\pgfqpoint{0.920384in}{2.799845in}}%
\pgfpathlineto{\pgfqpoint{0.935432in}{2.733622in}}%
\pgfpathlineto{\pgfqpoint{0.935449in}{2.733652in}}%
\pgfpathlineto{\pgfqpoint{0.939676in}{2.721851in}}%
\pgfpathlineto{\pgfqpoint{0.939677in}{2.721939in}}%
\pgfpathlineto{\pgfqpoint{0.941205in}{2.719332in}}%
\pgfpathlineto{\pgfqpoint{0.941206in}{2.719507in}}%
\pgfpathlineto{\pgfqpoint{0.944031in}{2.713813in}}%
\pgfpathlineto{\pgfqpoint{0.944032in}{2.713999in}}%
\pgfpathlineto{\pgfqpoint{0.945522in}{2.708663in}}%
\pgfpathlineto{\pgfqpoint{0.952469in}{2.690475in}}%
\pgfpathlineto{\pgfqpoint{0.952470in}{2.690664in}}%
\pgfpathlineto{\pgfqpoint{0.953963in}{2.686893in}}%
\pgfpathlineto{\pgfqpoint{0.960807in}{2.664586in}}%
\pgfpathlineto{\pgfqpoint{0.960808in}{2.664943in}}%
\pgfpathlineto{\pgfqpoint{0.962242in}{2.661231in}}%
\pgfpathlineto{\pgfqpoint{0.962295in}{2.661351in}}%
\pgfpathlineto{\pgfqpoint{0.965732in}{2.653010in}}%
\pgfpathlineto{\pgfqpoint{0.965734in}{2.653397in}}%
\pgfpathlineto{\pgfqpoint{0.967219in}{2.648997in}}%
\pgfpathlineto{\pgfqpoint{0.978852in}{2.603065in}}%
\pgfpathlineto{\pgfqpoint{0.987597in}{2.568874in}}%
\pgfpathlineto{\pgfqpoint{0.992347in}{2.540622in}}%
\pgfpathlineto{\pgfqpoint{0.996697in}{2.513985in}}%
\pgfpathlineto{\pgfqpoint{1.001280in}{2.491157in}}%
\pgfpathlineto{\pgfqpoint{1.002746in}{2.485708in}}%
\pgfpathlineto{\pgfqpoint{1.021470in}{2.437668in}}%
\pgfpathlineto{\pgfqpoint{1.022377in}{2.436644in}}%
\pgfpathlineto{\pgfqpoint{1.022943in}{2.437561in}}%
\pgfpathlineto{\pgfqpoint{1.023459in}{2.438012in}}%
\pgfpathlineto{\pgfqpoint{1.024026in}{2.437125in}}%
\pgfpathlineto{\pgfqpoint{1.036548in}{2.404490in}}%
\pgfpathlineto{\pgfqpoint{1.051353in}{2.364787in}}%
\pgfpathlineto{\pgfqpoint{1.091721in}{2.253122in}}%
\pgfpathlineto{\pgfqpoint{1.105858in}{2.211113in}}%
\pgfpathlineto{\pgfqpoint{1.118742in}{2.171987in}}%
\pgfpathlineto{\pgfqpoint{1.205255in}{1.897846in}}%
\pgfpathlineto{\pgfqpoint{1.264587in}{1.709755in}}%
\pgfpathlineto{\pgfqpoint{1.291008in}{1.634865in}}%
\pgfpathlineto{\pgfqpoint{1.300100in}{1.613141in}}%
\pgfpathlineto{\pgfqpoint{1.305491in}{1.600575in}}%
\pgfpathlineto{\pgfqpoint{1.313848in}{1.586244in}}%
\pgfpathlineto{\pgfqpoint{1.331193in}{1.550908in}}%
\pgfpathlineto{\pgfqpoint{1.335927in}{1.546245in}}%
\pgfpathlineto{\pgfqpoint{1.355988in}{1.511703in}}%
\pgfpathlineto{\pgfqpoint{1.363902in}{1.503823in}}%
\pgfpathlineto{\pgfqpoint{1.377565in}{1.481599in}}%
\pgfpathlineto{\pgfqpoint{1.390481in}{1.469128in}}%
\pgfpathlineto{\pgfqpoint{1.396678in}{1.451088in}}%
\pgfpathlineto{\pgfqpoint{1.400213in}{1.448224in}}%
\pgfpathlineto{\pgfqpoint{1.405137in}{1.450269in}}%
\pgfpathlineto{\pgfqpoint{1.411362in}{1.460086in}}%
\pgfpathlineto{\pgfqpoint{1.412778in}{2.577992in}}%
\pgfpathlineto{\pgfqpoint{1.415357in}{2.704638in}}%
\pgfpathlineto{\pgfqpoint{1.418495in}{2.810626in}}%
\pgfpathlineto{\pgfqpoint{1.422104in}{2.843796in}}%
\pgfpathlineto{\pgfqpoint{1.425325in}{2.872547in}}%
\pgfpathlineto{\pgfqpoint{1.427099in}{2.871591in}}%
\pgfpathlineto{\pgfqpoint{1.427489in}{2.872105in}}%
\pgfpathlineto{\pgfqpoint{1.429339in}{2.879085in}}%
\pgfpathlineto{\pgfqpoint{1.429727in}{2.878006in}}%
\pgfpathlineto{\pgfqpoint{1.434550in}{2.859802in}}%
\pgfpathlineto{\pgfqpoint{1.453263in}{2.756556in}}%
\pgfpathlineto{\pgfqpoint{1.466596in}{2.700202in}}%
\pgfpathlineto{\pgfqpoint{1.466596in}{2.700289in}}%
\pgfpathlineto{\pgfqpoint{1.468109in}{2.697490in}}%
\pgfpathlineto{\pgfqpoint{1.468123in}{2.697553in}}%
\pgfpathlineto{\pgfqpoint{1.470802in}{2.692474in}}%
\pgfpathlineto{\pgfqpoint{1.470893in}{2.692545in}}%
\pgfpathlineto{\pgfqpoint{1.470994in}{2.692828in}}%
\pgfpathlineto{\pgfqpoint{1.472367in}{2.691849in}}%
\pgfpathlineto{\pgfqpoint{1.472417in}{2.692107in}}%
\pgfpathlineto{\pgfqpoint{1.472570in}{2.692280in}}%
\pgfpathlineto{\pgfqpoint{1.475494in}{2.687384in}}%
\pgfpathlineto{\pgfqpoint{1.475495in}{2.687621in}}%
\pgfpathlineto{\pgfqpoint{1.476934in}{2.685455in}}%
\pgfpathlineto{\pgfqpoint{1.476985in}{2.685487in}}%
\pgfpathlineto{\pgfqpoint{1.477037in}{2.685554in}}%
\pgfpathlineto{\pgfqpoint{1.481976in}{2.674498in}}%
\pgfpathlineto{\pgfqpoint{1.481977in}{2.674721in}}%
\pgfpathlineto{\pgfqpoint{1.483418in}{2.672071in}}%
\pgfpathlineto{\pgfqpoint{1.483470in}{2.672100in}}%
\pgfpathlineto{\pgfqpoint{1.483522in}{2.672159in}}%
\pgfpathlineto{\pgfqpoint{1.493896in}{2.642987in}}%
\pgfpathlineto{\pgfqpoint{1.493898in}{2.643423in}}%
\pgfpathlineto{\pgfqpoint{1.495333in}{2.640009in}}%
\pgfpathlineto{\pgfqpoint{1.495386in}{2.640196in}}%
\pgfpathlineto{\pgfqpoint{1.502470in}{2.615103in}}%
\pgfpathlineto{\pgfqpoint{1.507964in}{2.594054in}}%
\pgfpathlineto{\pgfqpoint{1.525468in}{2.508003in}}%
\pgfpathlineto{\pgfqpoint{1.532338in}{2.469715in}}%
\pgfpathlineto{\pgfqpoint{1.544124in}{2.440564in}}%
\pgfpathlineto{\pgfqpoint{1.545407in}{2.446309in}}%
\pgfpathlineto{\pgfqpoint{1.546409in}{2.444912in}}%
\pgfpathlineto{\pgfqpoint{1.548287in}{2.441031in}}%
\pgfpathlineto{\pgfqpoint{1.548796in}{2.441933in}}%
\pgfpathlineto{\pgfqpoint{1.550947in}{2.450893in}}%
\pgfpathlineto{\pgfqpoint{1.551611in}{2.450210in}}%
\pgfpathlineto{\pgfqpoint{1.553326in}{2.447629in}}%
\pgfpathlineto{\pgfqpoint{1.553917in}{2.448420in}}%
\pgfpathlineto{\pgfqpoint{1.556035in}{2.446419in}}%
\pgfpathlineto{\pgfqpoint{1.559604in}{2.441403in}}%
\pgfpathlineto{\pgfqpoint{1.564399in}{2.451633in}}%
\pgfpathlineto{\pgfqpoint{1.565433in}{2.450186in}}%
\pgfpathlineto{\pgfqpoint{1.566615in}{2.448939in}}%
\pgfpathlineto{\pgfqpoint{1.568970in}{2.453578in}}%
\pgfpathlineto{\pgfqpoint{1.578276in}{2.489221in}}%
\pgfpathlineto{\pgfqpoint{1.581571in}{2.488519in}}%
\pgfpathlineto{\pgfqpoint{1.586146in}{2.484630in}}%
\pgfpathlineto{\pgfqpoint{1.591672in}{2.488606in}}%
\pgfpathlineto{\pgfqpoint{1.600101in}{2.487015in}}%
\pgfpathlineto{\pgfqpoint{1.603736in}{2.486803in}}%
\pgfpathlineto{\pgfqpoint{1.614055in}{2.478437in}}%
\pgfpathlineto{\pgfqpoint{1.630778in}{2.483435in}}%
\pgfpathlineto{\pgfqpoint{1.643248in}{2.457005in}}%
\pgfpathlineto{\pgfqpoint{1.648493in}{2.446519in}}%
\pgfpathlineto{\pgfqpoint{1.658984in}{2.436722in}}%
\pgfpathlineto{\pgfqpoint{1.670241in}{2.420444in}}%
\pgfpathlineto{\pgfqpoint{1.689718in}{2.373539in}}%
\pgfpathlineto{\pgfqpoint{1.696674in}{2.355415in}}%
\pgfpathlineto{\pgfqpoint{1.713635in}{2.285128in}}%
\pgfpathlineto{\pgfqpoint{1.729219in}{2.240323in}}%
\pgfpathlineto{\pgfqpoint{1.739204in}{2.213619in}}%
\pgfpathlineto{\pgfqpoint{1.752858in}{2.188771in}}%
\pgfpathlineto{\pgfqpoint{1.779868in}{2.165226in}}%
\pgfpathlineto{\pgfqpoint{1.828988in}{2.087694in}}%
\pgfpathlineto{\pgfqpoint{1.846956in}{2.051855in}}%
\pgfpathlineto{\pgfqpoint{1.852223in}{2.040736in}}%
\pgfpathlineto{\pgfqpoint{1.859246in}{2.042083in}}%
\pgfpathlineto{\pgfqpoint{1.860650in}{2.039528in}}%
\pgfpathlineto{\pgfqpoint{1.863540in}{2.041018in}}%
\pgfpathlineto{\pgfqpoint{1.863544in}{2.020653in}}%
\pgfpathlineto{\pgfqpoint{1.864943in}{2.624238in}}%
\pgfpathlineto{\pgfqpoint{1.868938in}{2.785850in}}%
\pgfpathlineto{\pgfqpoint{1.872130in}{2.826341in}}%
\pgfpathlineto{\pgfqpoint{1.873501in}{2.830300in}}%
\pgfpathlineto{\pgfqpoint{1.875947in}{2.853903in}}%
\pgfpathlineto{\pgfqpoint{1.877862in}{2.862662in}}%
\pgfpathlineto{\pgfqpoint{1.878417in}{2.862354in}}%
\pgfpathlineto{\pgfqpoint{1.879697in}{2.862967in}}%
\pgfpathlineto{\pgfqpoint{1.880687in}{2.867119in}}%
\pgfpathlineto{\pgfqpoint{1.881410in}{2.872932in}}%
\pgfpathlineto{\pgfqpoint{1.884629in}{2.933957in}}%
\pgfpathlineto{\pgfqpoint{1.885987in}{2.925552in}}%
\pgfpathlineto{\pgfqpoint{1.903185in}{2.830008in}}%
\pgfpathlineto{\pgfqpoint{1.919224in}{2.758923in}}%
\pgfpathlineto{\pgfqpoint{1.919225in}{2.759054in}}%
\pgfpathlineto{\pgfqpoint{1.920723in}{2.752551in}}%
\pgfpathlineto{\pgfqpoint{1.924750in}{2.742588in}}%
\pgfpathlineto{\pgfqpoint{1.924751in}{2.742857in}}%
\pgfpathlineto{\pgfqpoint{1.926233in}{2.739081in}}%
\pgfpathlineto{\pgfqpoint{1.926260in}{2.739122in}}%
\pgfpathlineto{\pgfqpoint{1.933124in}{2.718220in}}%
\pgfpathlineto{\pgfqpoint{1.933125in}{2.718369in}}%
\pgfpathlineto{\pgfqpoint{1.934617in}{2.714367in}}%
\pgfpathlineto{\pgfqpoint{1.946894in}{2.676237in}}%
\pgfpathlineto{\pgfqpoint{1.946895in}{2.676548in}}%
\pgfpathlineto{\pgfqpoint{1.948331in}{2.672447in}}%
\pgfpathlineto{\pgfqpoint{1.948384in}{2.672524in}}%
\pgfpathlineto{\pgfqpoint{1.959216in}{2.631457in}}%
\pgfpathlineto{\pgfqpoint{1.963524in}{2.614546in}}%
\pgfpathlineto{\pgfqpoint{1.966224in}{2.604077in}}%
\pgfpathlineto{\pgfqpoint{1.976480in}{2.554898in}}%
\pgfpathlineto{\pgfqpoint{1.988599in}{2.498091in}}%
\pgfpathlineto{\pgfqpoint{1.997956in}{2.470996in}}%
\pgfpathlineto{\pgfqpoint{1.999860in}{2.470372in}}%
\pgfpathlineto{\pgfqpoint{2.002452in}{2.463965in}}%
\pgfpathlineto{\pgfqpoint{2.013970in}{2.434671in}}%
\pgfpathlineto{\pgfqpoint{2.019103in}{2.422638in}}%
\pgfpathlineto{\pgfqpoint{2.088094in}{2.221815in}}%
\pgfpathlineto{\pgfqpoint{2.118146in}{2.128815in}}%
\pgfpathlineto{\pgfqpoint{2.133549in}{2.080017in}}%
\pgfpathlineto{\pgfqpoint{2.133797in}{2.280970in}}%
\pgfpathlineto{\pgfqpoint{2.135178in}{2.275984in}}%
\pgfpathlineto{\pgfqpoint{2.292922in}{1.708141in}}%
\pgfpathlineto{\pgfqpoint{2.317742in}{1.634296in}}%
\pgfpathlineto{\pgfqpoint{2.344695in}{1.562835in}}%
\pgfpathlineto{\pgfqpoint{2.359179in}{1.534037in}}%
\pgfpathlineto{\pgfqpoint{2.365242in}{1.519939in}}%
\pgfpathlineto{\pgfqpoint{2.377393in}{1.496525in}}%
\pgfpathlineto{\pgfqpoint{2.381560in}{1.488364in}}%
\pgfpathlineto{\pgfqpoint{2.385595in}{1.485992in}}%
\pgfpathlineto{\pgfqpoint{2.390683in}{1.486204in}}%
\pgfpathlineto{\pgfqpoint{2.394342in}{1.485032in}}%
\pgfpathlineto{\pgfqpoint{2.395745in}{2.591993in}}%
\pgfpathlineto{\pgfqpoint{2.398104in}{2.692750in}}%
\pgfpathlineto{\pgfqpoint{2.401290in}{2.809319in}}%
\pgfpathlineto{\pgfqpoint{2.405845in}{2.834901in}}%
\pgfpathlineto{\pgfqpoint{2.410691in}{2.854909in}}%
\pgfpathlineto{\pgfqpoint{2.411070in}{2.855792in}}%
\pgfpathlineto{\pgfqpoint{2.412199in}{2.864065in}}%
\pgfpathlineto{\pgfqpoint{2.413108in}{2.862753in}}%
\pgfpathlineto{\pgfqpoint{2.414396in}{2.862148in}}%
\pgfpathlineto{\pgfqpoint{2.414787in}{2.862965in}}%
\pgfpathlineto{\pgfqpoint{2.415441in}{2.860756in}}%
\pgfpathlineto{\pgfqpoint{2.433136in}{2.760238in}}%
\pgfpathlineto{\pgfqpoint{2.446322in}{2.702876in}}%
\pgfpathlineto{\pgfqpoint{2.452797in}{2.681050in}}%
\pgfpathlineto{\pgfqpoint{2.452798in}{2.681185in}}%
\pgfpathlineto{\pgfqpoint{2.454305in}{2.679366in}}%
\pgfpathlineto{\pgfqpoint{2.454330in}{2.679739in}}%
\pgfpathlineto{\pgfqpoint{2.454894in}{2.679500in}}%
\pgfpathlineto{\pgfqpoint{2.455495in}{2.680236in}}%
\pgfpathlineto{\pgfqpoint{2.455898in}{2.679824in}}%
\pgfpathlineto{\pgfqpoint{2.455900in}{2.680372in}}%
\pgfpathlineto{\pgfqpoint{2.457391in}{2.677968in}}%
\pgfpathlineto{\pgfqpoint{2.457418in}{2.678130in}}%
\pgfpathlineto{\pgfqpoint{2.463356in}{2.662077in}}%
\pgfpathlineto{\pgfqpoint{2.463357in}{2.662341in}}%
\pgfpathlineto{\pgfqpoint{2.464798in}{2.660231in}}%
\pgfpathlineto{\pgfqpoint{2.464850in}{2.660294in}}%
\pgfpathlineto{\pgfqpoint{2.464902in}{2.660358in}}%
\pgfpathlineto{\pgfqpoint{2.467580in}{2.655795in}}%
\pgfpathlineto{\pgfqpoint{2.467581in}{2.656059in}}%
\pgfpathlineto{\pgfqpoint{2.469071in}{2.651824in}}%
\pgfpathlineto{\pgfqpoint{2.476409in}{2.631085in}}%
\pgfpathlineto{\pgfqpoint{2.476411in}{2.631576in}}%
\pgfpathlineto{\pgfqpoint{2.477846in}{2.628485in}}%
\pgfpathlineto{\pgfqpoint{2.477898in}{2.628723in}}%
\pgfpathlineto{\pgfqpoint{2.482754in}{2.613584in}}%
\pgfpathlineto{\pgfqpoint{2.488383in}{2.592343in}}%
\pgfpathlineto{\pgfqpoint{2.492203in}{2.577062in}}%
\pgfpathlineto{\pgfqpoint{2.499485in}{2.549566in}}%
\pgfpathlineto{\pgfqpoint{2.505221in}{2.515787in}}%
\pgfpathlineto{\pgfqpoint{2.508161in}{2.496423in}}%
\pgfpathlineto{\pgfqpoint{2.509964in}{2.484356in}}%
\pgfpathlineto{\pgfqpoint{2.511253in}{2.476497in}}%
\pgfpathlineto{\pgfqpoint{2.515915in}{2.455553in}}%
\pgfpathlineto{\pgfqpoint{2.523805in}{2.437147in}}%
\pgfpathlineto{\pgfqpoint{2.524731in}{2.437591in}}%
\pgfpathlineto{\pgfqpoint{2.525188in}{2.438892in}}%
\pgfpathlineto{\pgfqpoint{2.525952in}{2.437096in}}%
\pgfpathlineto{\pgfqpoint{2.527621in}{2.435129in}}%
\pgfpathlineto{\pgfqpoint{2.530982in}{2.430212in}}%
\pgfpathlineto{\pgfqpoint{2.531359in}{2.430478in}}%
\pgfpathlineto{\pgfqpoint{2.533561in}{2.428153in}}%
\pgfpathlineto{\pgfqpoint{2.535831in}{2.425849in}}%
\pgfpathlineto{\pgfqpoint{2.536334in}{2.426666in}}%
\pgfpathlineto{\pgfqpoint{2.540030in}{2.444116in}}%
\pgfpathlineto{\pgfqpoint{2.540476in}{2.443666in}}%
\pgfpathlineto{\pgfqpoint{2.542869in}{2.440533in}}%
\pgfpathlineto{\pgfqpoint{2.546111in}{2.438478in}}%
\pgfpathlineto{\pgfqpoint{2.549453in}{2.447956in}}%
\pgfpathlineto{\pgfqpoint{2.555115in}{2.447329in}}%
\pgfpathlineto{\pgfqpoint{2.557731in}{2.450986in}}%
\pgfpathlineto{\pgfqpoint{2.561844in}{2.449873in}}%
\pgfpathlineto{\pgfqpoint{2.563166in}{2.450212in}}%
\pgfpathlineto{\pgfqpoint{2.567074in}{2.456340in}}%
\pgfpathlineto{\pgfqpoint{2.570423in}{2.456452in}}%
\pgfpathlineto{\pgfqpoint{2.572774in}{2.536792in}}%
\pgfpathlineto{\pgfqpoint{2.573985in}{2.532486in}}%
\pgfpathlineto{\pgfqpoint{2.577575in}{2.526201in}}%
\pgfpathlineto{\pgfqpoint{2.584226in}{2.514936in}}%
\pgfpathlineto{\pgfqpoint{2.591736in}{2.502005in}}%
\pgfpathlineto{\pgfqpoint{2.596864in}{2.499679in}}%
\pgfpathlineto{\pgfqpoint{2.607122in}{2.506285in}}%
\pgfpathlineto{\pgfqpoint{2.614762in}{2.490143in}}%
\pgfpathlineto{\pgfqpoint{2.626669in}{2.462859in}}%
\pgfpathlineto{\pgfqpoint{2.633421in}{2.451841in}}%
\pgfpathlineto{\pgfqpoint{2.644113in}{2.439788in}}%
\pgfpathlineto{\pgfqpoint{2.692629in}{2.335057in}}%
\pgfpathlineto{\pgfqpoint{2.707726in}{2.309157in}}%
\pgfpathlineto{\pgfqpoint{2.733449in}{2.255641in}}%
\pgfpathlineto{\pgfqpoint{2.772971in}{2.189058in}}%
\pgfpathlineto{\pgfqpoint{2.797189in}{2.152468in}}%
\pgfpathlineto{\pgfqpoint{2.829137in}{2.110352in}}%
\pgfpathlineto{\pgfqpoint{2.831751in}{2.102781in}}%
\pgfpathlineto{\pgfqpoint{2.835237in}{2.096583in}}%
\pgfpathlineto{\pgfqpoint{2.842207in}{2.097049in}}%
\pgfpathlineto{\pgfqpoint{2.843601in}{2.094869in}}%
\pgfpathlineto{\pgfqpoint{2.846508in}{2.095884in}}%
\pgfpathlineto{\pgfqpoint{2.847910in}{2.617572in}}%
\pgfpathlineto{\pgfqpoint{2.850710in}{2.722809in}}%
\pgfpathlineto{\pgfqpoint{2.854052in}{2.822652in}}%
\pgfpathlineto{\pgfqpoint{2.858908in}{2.853471in}}%
\pgfpathlineto{\pgfqpoint{2.860846in}{2.862529in}}%
\pgfpathlineto{\pgfqpoint{2.861430in}{2.862360in}}%
\pgfpathlineto{\pgfqpoint{2.862754in}{2.863269in}}%
\pgfpathlineto{\pgfqpoint{2.863712in}{2.867698in}}%
\pgfpathlineto{\pgfqpoint{2.864569in}{2.866031in}}%
\pgfpathlineto{\pgfqpoint{2.865912in}{2.865487in}}%
\pgfpathlineto{\pgfqpoint{2.866331in}{2.866518in}}%
\pgfpathlineto{\pgfqpoint{2.867037in}{2.864573in}}%
\pgfpathlineto{\pgfqpoint{2.867478in}{2.863332in}}%
\pgfpathlineto{\pgfqpoint{2.867577in}{2.895370in}}%
\pgfpathlineto{\pgfqpoint{2.869090in}{2.886023in}}%
\pgfpathlineto{\pgfqpoint{2.885631in}{2.794906in}}%
\pgfpathlineto{\pgfqpoint{2.900088in}{2.731235in}}%
\pgfpathlineto{\pgfqpoint{2.900089in}{2.731340in}}%
\pgfpathlineto{\pgfqpoint{2.901600in}{2.725106in}}%
\pgfpathlineto{\pgfqpoint{2.904983in}{2.717351in}}%
\pgfpathlineto{\pgfqpoint{2.907707in}{2.712956in}}%
\pgfpathlineto{\pgfqpoint{2.907708in}{2.713146in}}%
\pgfpathlineto{\pgfqpoint{2.909196in}{2.710349in}}%
\pgfpathlineto{\pgfqpoint{2.909222in}{2.710403in}}%
\pgfpathlineto{\pgfqpoint{2.910943in}{2.705950in}}%
\pgfpathlineto{\pgfqpoint{2.910944in}{2.706145in}}%
\pgfpathlineto{\pgfqpoint{2.912434in}{2.702598in}}%
\pgfpathlineto{\pgfqpoint{2.912486in}{2.702619in}}%
\pgfpathlineto{\pgfqpoint{2.917940in}{2.689324in}}%
\pgfpathlineto{\pgfqpoint{2.917941in}{2.689516in}}%
\pgfpathlineto{\pgfqpoint{2.919435in}{2.686625in}}%
\pgfpathlineto{\pgfqpoint{2.919487in}{2.686646in}}%
\pgfpathlineto{\pgfqpoint{2.930783in}{2.653199in}}%
\pgfpathlineto{\pgfqpoint{2.930785in}{2.653588in}}%
\pgfpathlineto{\pgfqpoint{2.932221in}{2.649885in}}%
\pgfpathlineto{\pgfqpoint{2.932274in}{2.650030in}}%
\pgfpathlineto{\pgfqpoint{2.944124in}{2.603386in}}%
\pgfpathlineto{\pgfqpoint{2.947186in}{2.591243in}}%
\pgfpathlineto{\pgfqpoint{2.958682in}{2.535484in}}%
\pgfpathlineto{\pgfqpoint{2.965711in}{2.493869in}}%
\pgfpathlineto{\pgfqpoint{2.969868in}{2.476698in}}%
\pgfpathlineto{\pgfqpoint{2.977439in}{2.453238in}}%
\pgfpathlineto{\pgfqpoint{2.980583in}{2.444299in}}%
\pgfpathlineto{\pgfqpoint{2.982779in}{2.439201in}}%
\pgfpathlineto{\pgfqpoint{2.983351in}{2.440863in}}%
\pgfpathlineto{\pgfqpoint{2.984297in}{2.439735in}}%
\pgfpathlineto{\pgfqpoint{2.987325in}{2.435481in}}%
\pgfpathlineto{\pgfqpoint{2.991184in}{2.429127in}}%
\pgfpathlineto{\pgfqpoint{2.993920in}{2.427156in}}%
\pgfpathlineto{\pgfqpoint{2.999979in}{2.412391in}}%
\pgfpathlineto{\pgfqpoint{3.007297in}{2.396021in}}%
\pgfpathlineto{\pgfqpoint{3.007466in}{2.405929in}}%
\pgfpathlineto{\pgfqpoint{3.008877in}{2.403546in}}%
\pgfpathlineto{\pgfqpoint{3.014688in}{2.389261in}}%
\pgfpathlineto{\pgfqpoint{3.025449in}{2.359774in}}%
\pgfpathlineto{\pgfqpoint{3.069536in}{2.231703in}}%
\pgfpathlineto{\pgfqpoint{3.100582in}{2.135317in}}%
\pgfpathlineto{\pgfqpoint{3.115618in}{2.087869in}}%
\pgfpathlineto{\pgfqpoint{3.180675in}{1.875956in}}%
\pgfpathlineto{\pgfqpoint{3.208771in}{1.784555in}}%
\pgfpathlineto{\pgfqpoint{3.260115in}{1.638984in}}%
\pgfpathlineto{\pgfqpoint{3.279561in}{1.597873in}}%
\pgfpathlineto{\pgfqpoint{3.282948in}{1.593585in}}%
\pgfpathlineto{\pgfqpoint{3.289723in}{1.594103in}}%
\pgfpathlineto{\pgfqpoint{3.296772in}{1.590908in}}%
\pgfpathlineto{\pgfqpoint{3.327650in}{1.536607in}}%
\pgfpathlineto{\pgfqpoint{3.344598in}{1.509722in}}%
\pgfpathlineto{\pgfqpoint{3.359100in}{1.479247in}}%
\pgfpathlineto{\pgfqpoint{3.363071in}{1.468880in}}%
\pgfpathlineto{\pgfqpoint{3.366678in}{1.465821in}}%
\pgfpathlineto{\pgfqpoint{3.371261in}{1.466492in}}%
\pgfpathlineto{\pgfqpoint{3.377309in}{1.474478in}}%
\pgfpathlineto{\pgfqpoint{3.377336in}{2.654557in}}%
\pgfpathlineto{\pgfqpoint{3.378712in}{2.395721in}}%
\pgfpathlineto{\pgfqpoint{3.381292in}{2.669847in}}%
\pgfpathlineto{\pgfqpoint{3.384429in}{2.801597in}}%
\pgfpathlineto{\pgfqpoint{3.388021in}{2.838115in}}%
\pgfpathlineto{\pgfqpoint{3.391212in}{2.867300in}}%
\pgfpathlineto{\pgfqpoint{3.392821in}{2.867593in}}%
\pgfpathlineto{\pgfqpoint{3.393893in}{2.873089in}}%
\pgfpathlineto{\pgfqpoint{3.394856in}{2.871759in}}%
\pgfpathlineto{\pgfqpoint{3.396062in}{2.869932in}}%
\pgfpathlineto{\pgfqpoint{3.396457in}{2.870578in}}%
\pgfpathlineto{\pgfqpoint{3.397836in}{2.871351in}}%
\pgfpathlineto{\pgfqpoint{3.398299in}{2.870454in}}%
\pgfpathlineto{\pgfqpoint{3.416369in}{2.767197in}}%
\pgfpathlineto{\pgfqpoint{3.430125in}{2.707106in}}%
\pgfpathlineto{\pgfqpoint{3.430126in}{2.707158in}}%
\pgfpathlineto{\pgfqpoint{3.431630in}{2.700690in}}%
\pgfpathlineto{\pgfqpoint{3.435882in}{2.688983in}}%
\pgfpathlineto{\pgfqpoint{3.435882in}{2.689107in}}%
\pgfpathlineto{\pgfqpoint{3.437339in}{2.687070in}}%
\pgfpathlineto{\pgfqpoint{3.437386in}{2.687323in}}%
\pgfpathlineto{\pgfqpoint{3.437535in}{2.687506in}}%
\pgfpathlineto{\pgfqpoint{3.439399in}{2.685994in}}%
\pgfpathlineto{\pgfqpoint{3.439400in}{2.686183in}}%
\pgfpathlineto{\pgfqpoint{3.440838in}{2.683466in}}%
\pgfpathlineto{\pgfqpoint{3.440889in}{2.683505in}}%
\pgfpathlineto{\pgfqpoint{3.444128in}{2.675312in}}%
\pgfpathlineto{\pgfqpoint{3.444172in}{2.675354in}}%
\pgfpathlineto{\pgfqpoint{3.444187in}{2.675544in}}%
\pgfpathlineto{\pgfqpoint{3.445523in}{2.673071in}}%
\pgfpathlineto{\pgfqpoint{3.445537in}{2.673196in}}%
\pgfpathlineto{\pgfqpoint{3.448406in}{2.667735in}}%
\pgfpathlineto{\pgfqpoint{3.448407in}{2.667973in}}%
\pgfpathlineto{\pgfqpoint{3.449849in}{2.665449in}}%
\pgfpathlineto{\pgfqpoint{3.449900in}{2.665494in}}%
\pgfpathlineto{\pgfqpoint{3.449953in}{2.665526in}}%
\pgfpathlineto{\pgfqpoint{3.456939in}{2.643142in}}%
\pgfpathlineto{\pgfqpoint{3.456941in}{2.643587in}}%
\pgfpathlineto{\pgfqpoint{3.458374in}{2.640395in}}%
\pgfpathlineto{\pgfqpoint{3.458427in}{2.640593in}}%
\pgfpathlineto{\pgfqpoint{3.461247in}{2.634825in}}%
\pgfpathlineto{\pgfqpoint{3.461249in}{2.635292in}}%
\pgfpathlineto{\pgfqpoint{3.462684in}{2.631960in}}%
\pgfpathlineto{\pgfqpoint{3.462737in}{2.632175in}}%
\pgfpathlineto{\pgfqpoint{3.472334in}{2.595078in}}%
\pgfpathlineto{\pgfqpoint{3.478444in}{2.570398in}}%
\pgfpathlineto{\pgfqpoint{3.484999in}{2.543111in}}%
\pgfpathlineto{\pgfqpoint{3.495982in}{2.471360in}}%
\pgfpathlineto{\pgfqpoint{3.499995in}{2.459250in}}%
\pgfpathlineto{\pgfqpoint{3.500929in}{2.456822in}}%
\pgfpathlineto{\pgfqpoint{3.509553in}{2.436288in}}%
\pgfpathlineto{\pgfqpoint{3.509951in}{2.435662in}}%
\pgfpathlineto{\pgfqpoint{3.510419in}{2.436846in}}%
\pgfpathlineto{\pgfqpoint{3.511606in}{2.438769in}}%
\pgfpathlineto{\pgfqpoint{3.512031in}{2.437458in}}%
\pgfpathlineto{\pgfqpoint{3.514468in}{2.434002in}}%
\pgfpathlineto{\pgfqpoint{3.518144in}{2.429784in}}%
\pgfpathlineto{\pgfqpoint{3.520359in}{2.430983in}}%
\pgfpathlineto{\pgfqpoint{3.523903in}{2.430082in}}%
\pgfpathlineto{\pgfqpoint{3.525792in}{2.431797in}}%
\pgfpathlineto{\pgfqpoint{3.528440in}{2.438660in}}%
\pgfpathlineto{\pgfqpoint{3.528983in}{2.438195in}}%
\pgfpathlineto{\pgfqpoint{3.531438in}{2.440927in}}%
\pgfpathlineto{\pgfqpoint{3.533368in}{2.439750in}}%
\pgfpathlineto{\pgfqpoint{3.535578in}{2.443868in}}%
\pgfpathlineto{\pgfqpoint{3.536264in}{2.443335in}}%
\pgfpathlineto{\pgfqpoint{3.537589in}{2.441970in}}%
\pgfpathlineto{\pgfqpoint{3.539695in}{2.443361in}}%
\pgfpathlineto{\pgfqpoint{3.543897in}{2.449318in}}%
\pgfpathlineto{\pgfqpoint{3.545260in}{2.450102in}}%
\pgfpathlineto{\pgfqpoint{3.549706in}{2.461495in}}%
\pgfpathlineto{\pgfqpoint{3.552114in}{2.458820in}}%
\pgfpathlineto{\pgfqpoint{3.554724in}{2.456235in}}%
\pgfpathlineto{\pgfqpoint{3.558937in}{2.458458in}}%
\pgfpathlineto{\pgfqpoint{3.561990in}{2.462760in}}%
\pgfpathlineto{\pgfqpoint{3.568095in}{2.483151in}}%
\pgfpathlineto{\pgfqpoint{3.574450in}{2.492717in}}%
\pgfpathlineto{\pgfqpoint{3.585364in}{2.473477in}}%
\pgfpathlineto{\pgfqpoint{3.588933in}{2.467335in}}%
\pgfpathlineto{\pgfqpoint{3.595490in}{2.464415in}}%
\pgfpathlineto{\pgfqpoint{3.604707in}{2.462375in}}%
\pgfpathlineto{\pgfqpoint{3.625754in}{2.421697in}}%
\pgfpathlineto{\pgfqpoint{3.633030in}{2.408553in}}%
\pgfpathlineto{\pgfqpoint{3.638417in}{2.395484in}}%
\pgfpathlineto{\pgfqpoint{3.647036in}{2.379881in}}%
\pgfpathlineto{\pgfqpoint{3.660910in}{2.345515in}}%
\pgfpathlineto{\pgfqpoint{3.674660in}{2.324198in}}%
\pgfpathlineto{\pgfqpoint{3.688553in}{2.297485in}}%
\pgfpathlineto{\pgfqpoint{3.698962in}{2.273529in}}%
\pgfpathlineto{\pgfqpoint{3.749666in}{2.182707in}}%
\pgfpathlineto{\pgfqpoint{3.813037in}{2.101018in}}%
\pgfpathlineto{\pgfqpoint{3.818448in}{2.089927in}}%
\pgfpathlineto{\pgfqpoint{3.825663in}{2.091588in}}%
\pgfpathlineto{\pgfqpoint{3.829475in}{2.088148in}}%
\pgfpathlineto{\pgfqpoint{3.830877in}{2.623013in}}%
\pgfpathlineto{\pgfqpoint{3.834872in}{2.786162in}}%
\pgfpathlineto{\pgfqpoint{3.838065in}{2.826634in}}%
\pgfpathlineto{\pgfqpoint{3.839435in}{2.830502in}}%
\pgfpathlineto{\pgfqpoint{3.841876in}{2.853794in}}%
\pgfpathlineto{\pgfqpoint{3.843809in}{2.862765in}}%
\pgfpathlineto{\pgfqpoint{3.844386in}{2.862568in}}%
\pgfpathlineto{\pgfqpoint{3.845698in}{2.863386in}}%
\pgfpathlineto{\pgfqpoint{3.846605in}{2.866749in}}%
\pgfpathlineto{\pgfqpoint{3.847557in}{2.865700in}}%
\pgfpathlineto{\pgfqpoint{3.848425in}{2.864127in}}%
\pgfpathlineto{\pgfqpoint{3.848929in}{2.866297in}}%
\pgfpathlineto{\pgfqpoint{3.850296in}{2.864755in}}%
\pgfpathlineto{\pgfqpoint{3.850445in}{2.864284in}}%
\pgfpathlineto{\pgfqpoint{3.850564in}{2.907303in}}%
\pgfpathlineto{\pgfqpoint{3.852081in}{2.897971in}}%
\pgfpathlineto{\pgfqpoint{3.869364in}{2.802062in}}%
\pgfpathlineto{\pgfqpoint{3.869365in}{2.802084in}}%
\pgfpathlineto{\pgfqpoint{3.870814in}{2.795266in}}%
\pgfpathlineto{\pgfqpoint{3.870825in}{2.795315in}}%
\pgfpathlineto{\pgfqpoint{3.881046in}{2.749897in}}%
\pgfpathlineto{\pgfqpoint{3.882591in}{2.711902in}}%
\pgfpathlineto{\pgfqpoint{3.886461in}{2.701805in}}%
\pgfpathlineto{\pgfqpoint{3.886462in}{2.701919in}}%
\pgfpathlineto{\pgfqpoint{3.887965in}{2.699048in}}%
\pgfpathlineto{\pgfqpoint{3.887991in}{2.699053in}}%
\pgfpathlineto{\pgfqpoint{3.890114in}{2.696557in}}%
\pgfpathlineto{\pgfqpoint{3.888017in}{2.699070in}}%
\pgfpathlineto{\pgfqpoint{3.890211in}{2.696627in}}%
\pgfpathlineto{\pgfqpoint{3.890315in}{2.696905in}}%
\pgfpathlineto{\pgfqpoint{3.891688in}{2.694378in}}%
\pgfpathlineto{\pgfqpoint{3.891740in}{2.694400in}}%
\pgfpathlineto{\pgfqpoint{3.891791in}{2.694465in}}%
\pgfpathlineto{\pgfqpoint{3.898475in}{2.677972in}}%
\pgfpathlineto{\pgfqpoint{3.898476in}{2.678198in}}%
\pgfpathlineto{\pgfqpoint{3.899916in}{2.675805in}}%
\pgfpathlineto{\pgfqpoint{3.899968in}{2.675828in}}%
\pgfpathlineto{\pgfqpoint{3.900020in}{2.675882in}}%
\pgfpathlineto{\pgfqpoint{3.910395in}{2.646758in}}%
\pgfpathlineto{\pgfqpoint{3.910397in}{2.647183in}}%
\pgfpathlineto{\pgfqpoint{3.911830in}{2.643802in}}%
\pgfpathlineto{\pgfqpoint{3.911883in}{2.643981in}}%
\pgfpathlineto{\pgfqpoint{3.914910in}{2.637410in}}%
\pgfpathlineto{\pgfqpoint{3.914912in}{2.637860in}}%
\pgfpathlineto{\pgfqpoint{3.916394in}{2.632729in}}%
\pgfpathlineto{\pgfqpoint{3.926205in}{2.594243in}}%
\pgfpathlineto{\pgfqpoint{3.932307in}{2.570069in}}%
\pgfpathlineto{\pgfqpoint{3.937742in}{2.545630in}}%
\pgfpathlineto{\pgfqpoint{3.940511in}{2.528220in}}%
\pgfpathlineto{\pgfqpoint{3.943953in}{2.506309in}}%
\pgfpathlineto{\pgfqpoint{3.947635in}{2.483130in}}%
\pgfpathlineto{\pgfqpoint{3.958359in}{2.447263in}}%
\pgfpathlineto{\pgfqpoint{3.963803in}{2.437303in}}%
\pgfpathlineto{\pgfqpoint{3.967747in}{2.436019in}}%
\pgfpathlineto{\pgfqpoint{3.970409in}{2.430645in}}%
\pgfpathlineto{\pgfqpoint{3.973740in}{2.426644in}}%
\pgfpathlineto{\pgfqpoint{3.979190in}{2.414472in}}%
\pgfpathlineto{\pgfqpoint{4.018157in}{2.308467in}}%
\pgfpathlineto{\pgfqpoint{4.023591in}{2.293966in}}%
\pgfpathlineto{\pgfqpoint{4.077119in}{2.144505in}}%
\pgfpathlineto{\pgfqpoint{4.200560in}{1.750912in}}%
\pgfpathlineto{\pgfqpoint{4.215287in}{1.707737in}}%
\pgfpathlineto{\pgfqpoint{4.231113in}{1.663317in}}%
\pgfpathlineto{\pgfqpoint{4.234879in}{1.655673in}}%
\pgfpathlineto{\pgfqpoint{4.239418in}{1.643950in}}%
\pgfpathlineto{\pgfqpoint{4.247701in}{1.625601in}}%
\pgfpathlineto{\pgfqpoint{4.253003in}{1.614898in}}%
\pgfpathlineto{\pgfqpoint{4.255819in}{1.611307in}}%
\pgfpathlineto{\pgfqpoint{4.260769in}{1.610685in}}%
\pgfpathlineto{\pgfqpoint{4.269310in}{1.616289in}}%
\pgfpathlineto{\pgfqpoint{4.277412in}{1.602736in}}%
\pgfpathlineto{\pgfqpoint{4.285383in}{1.587321in}}%
\pgfpathlineto{\pgfqpoint{4.297185in}{1.562021in}}%
\pgfpathlineto{\pgfqpoint{4.301535in}{1.552605in}}%
\pgfpathlineto{\pgfqpoint{4.316772in}{1.529186in}}%
\pgfpathlineto{\pgfqpoint{4.317920in}{1.525224in}}%
\pgfpathlineto{\pgfqpoint{4.329035in}{2.162177in}}%
\pgfpathlineto{\pgfqpoint{4.330450in}{2.156864in}}%
\pgfpathlineto{\pgfqpoint{4.350551in}{2.087239in}}%
\pgfpathlineto{\pgfqpoint{4.357002in}{2.076287in}}%
\pgfpathlineto{\pgfqpoint{4.360266in}{2.073684in}}%
\pgfpathlineto{\pgfqpoint{4.360272in}{1.986304in}}%
\pgfpathlineto{\pgfqpoint{4.361681in}{2.615438in}}%
\pgfpathlineto{\pgfqpoint{4.365095in}{2.763960in}}%
\pgfpathlineto{\pgfqpoint{4.368412in}{2.826167in}}%
\pgfpathlineto{\pgfqpoint{4.371392in}{2.835028in}}%
\pgfpathlineto{\pgfqpoint{4.378106in}{2.867507in}}%
\pgfpathlineto{\pgfqpoint{4.380024in}{2.865104in}}%
\pgfpathlineto{\pgfqpoint{4.380731in}{2.866453in}}%
\pgfpathlineto{\pgfqpoint{4.382243in}{2.859178in}}%
\pgfpathlineto{\pgfqpoint{4.400202in}{2.758352in}}%
\pgfpathlineto{\pgfqpoint{4.400203in}{2.758378in}}%
\pgfpathlineto{\pgfqpoint{4.401678in}{2.751858in}}%
\pgfpathlineto{\pgfqpoint{4.401682in}{2.751875in}}%
\pgfpathlineto{\pgfqpoint{4.414071in}{2.700183in}}%
\pgfpathlineto{\pgfqpoint{4.414072in}{2.700288in}}%
\pgfpathlineto{\pgfqpoint{4.415566in}{2.695449in}}%
\pgfpathlineto{\pgfqpoint{4.415591in}{2.695465in}}%
\pgfpathlineto{\pgfqpoint{4.415616in}{2.695492in}}%
\pgfpathlineto{\pgfqpoint{4.418378in}{2.691025in}}%
\pgfpathlineto{\pgfqpoint{4.418379in}{2.691148in}}%
\pgfpathlineto{\pgfqpoint{4.419723in}{2.688678in}}%
\pgfpathlineto{\pgfqpoint{4.419866in}{2.688754in}}%
\pgfpathlineto{\pgfqpoint{4.419916in}{2.689024in}}%
\pgfpathlineto{\pgfqpoint{4.421491in}{2.688340in}}%
\pgfpathlineto{\pgfqpoint{4.421899in}{2.688847in}}%
\pgfpathlineto{\pgfqpoint{4.422979in}{2.686836in}}%
\pgfpathlineto{\pgfqpoint{4.423005in}{2.686950in}}%
\pgfpathlineto{\pgfqpoint{4.427091in}{2.677714in}}%
\pgfpathlineto{\pgfqpoint{4.427092in}{2.677955in}}%
\pgfpathlineto{\pgfqpoint{4.428583in}{2.674144in}}%
\pgfpathlineto{\pgfqpoint{4.431208in}{2.669266in}}%
\pgfpathlineto{\pgfqpoint{4.428635in}{2.674156in}}%
\pgfpathlineto{\pgfqpoint{4.431222in}{2.669447in}}%
\pgfpathlineto{\pgfqpoint{4.434094in}{2.664227in}}%
\pgfpathlineto{\pgfqpoint{4.434095in}{2.664470in}}%
\pgfpathlineto{\pgfqpoint{4.435583in}{2.659051in}}%
\pgfpathlineto{\pgfqpoint{4.442413in}{2.639599in}}%
\pgfpathlineto{\pgfqpoint{4.442415in}{2.640051in}}%
\pgfpathlineto{\pgfqpoint{4.443850in}{2.636742in}}%
\pgfpathlineto{\pgfqpoint{4.443903in}{2.636944in}}%
\pgfpathlineto{\pgfqpoint{4.449503in}{2.618592in}}%
\pgfpathlineto{\pgfqpoint{4.455252in}{2.596533in}}%
\pgfpathlineto{\pgfqpoint{4.465855in}{2.555627in}}%
\pgfpathlineto{\pgfqpoint{4.466995in}{2.550287in}}%
\pgfpathlineto{\pgfqpoint{4.470152in}{2.531015in}}%
\pgfpathlineto{\pgfqpoint{4.472753in}{2.513806in}}%
\pgfpathlineto{\pgfqpoint{4.474877in}{2.499826in}}%
\pgfpathlineto{\pgfqpoint{4.480196in}{2.470286in}}%
\pgfpathlineto{\pgfqpoint{4.486761in}{2.449199in}}%
\pgfpathlineto{\pgfqpoint{4.487085in}{2.449434in}}%
\pgfpathlineto{\pgfqpoint{4.487617in}{2.447985in}}%
\pgfpathlineto{\pgfqpoint{4.491576in}{2.440540in}}%
\pgfpathlineto{\pgfqpoint{4.493194in}{2.443178in}}%
\pgfpathlineto{\pgfqpoint{4.493683in}{2.441675in}}%
\pgfpathlineto{\pgfqpoint{4.496532in}{2.437835in}}%
\pgfpathlineto{\pgfqpoint{4.499847in}{2.435227in}}%
\pgfpathlineto{\pgfqpoint{4.501170in}{2.439613in}}%
\pgfpathlineto{\pgfqpoint{4.501961in}{2.438347in}}%
\pgfpathlineto{\pgfqpoint{4.505898in}{2.437608in}}%
\pgfpathlineto{\pgfqpoint{4.508183in}{2.435777in}}%
\pgfpathlineto{\pgfqpoint{4.510573in}{2.440105in}}%
\pgfpathlineto{\pgfqpoint{4.511175in}{2.439881in}}%
\pgfpathlineto{\pgfqpoint{4.514902in}{2.438960in}}%
\pgfpathlineto{\pgfqpoint{4.517542in}{2.441266in}}%
\pgfpathlineto{\pgfqpoint{4.519982in}{2.441231in}}%
\pgfpathlineto{\pgfqpoint{4.522790in}{2.447638in}}%
\pgfpathlineto{\pgfqpoint{4.524745in}{2.447208in}}%
\pgfpathlineto{\pgfqpoint{4.527894in}{2.452671in}}%
\pgfpathlineto{\pgfqpoint{4.531945in}{2.448780in}}%
\pgfpathlineto{\pgfqpoint{4.533807in}{2.450926in}}%
\pgfpathlineto{\pgfqpoint{4.540142in}{2.468947in}}%
\pgfpathlineto{\pgfqpoint{4.544641in}{2.468699in}}%
\pgfpathlineto{\pgfqpoint{4.549481in}{2.467144in}}%
\pgfpathlineto{\pgfqpoint{4.554689in}{2.476106in}}%
\pgfpathlineto{\pgfqpoint{4.558232in}{2.474573in}}%
\pgfpathlineto{\pgfqpoint{4.569096in}{2.457617in}}%
\pgfpathlineto{\pgfqpoint{4.572987in}{2.453176in}}%
\pgfpathlineto{\pgfqpoint{4.579705in}{2.452412in}}%
\pgfpathlineto{\pgfqpoint{4.590002in}{2.454089in}}%
\pgfpathlineto{\pgfqpoint{4.611328in}{2.411565in}}%
\pgfpathlineto{\pgfqpoint{4.624566in}{2.398712in}}%
\pgfpathlineto{\pgfqpoint{4.651224in}{2.343110in}}%
\pgfpathlineto{\pgfqpoint{4.670383in}{2.297551in}}%
\pgfpathlineto{\pgfqpoint{4.708609in}{2.239164in}}%
\pgfpathlineto{\pgfqpoint{4.723660in}{2.207552in}}%
\pgfpathlineto{\pgfqpoint{4.747035in}{2.174557in}}%
\pgfpathlineto{\pgfqpoint{4.754794in}{2.153659in}}%
\pgfpathlineto{\pgfqpoint{4.754794in}{2.153659in}}%
\pgfusepath{stroke}%
\end{pgfscope}%
\begin{pgfscope}%
\pgfpathrectangle{\pgfqpoint{0.626312in}{0.402778in}}{\pgfqpoint{4.325077in}{2.937500in}}%
\pgfusepath{clip}%
\pgfsetrectcap%
\pgfsetroundjoin%
\pgfsetlinewidth{1.505625pt}%
\definecolor{currentstroke}{rgb}{0.172549,0.627451,0.172549}%
\pgfsetstrokecolor{currentstroke}%
\pgfsetdash{}{0pt}%
\pgfpathmoveto{\pgfqpoint{0.822906in}{0.649986in}}%
\pgfpathlineto{\pgfqpoint{0.823711in}{0.707948in}}%
\pgfpathlineto{\pgfqpoint{0.824517in}{0.573982in}}%
\pgfpathlineto{\pgfqpoint{0.826127in}{0.715701in}}%
\pgfpathlineto{\pgfqpoint{0.829348in}{0.711309in}}%
\pgfpathlineto{\pgfqpoint{0.835790in}{0.536301in}}%
\pgfpathlineto{\pgfqpoint{0.848674in}{0.636924in}}%
\pgfpathlineto{\pgfqpoint{0.862246in}{0.646736in}}%
\pgfpathlineto{\pgfqpoint{0.862431in}{0.653381in}}%
\pgfpathlineto{\pgfqpoint{0.862630in}{0.649516in}}%
\pgfpathlineto{\pgfqpoint{0.863026in}{0.596597in}}%
\pgfpathlineto{\pgfqpoint{0.863818in}{0.689047in}}%
\pgfpathlineto{\pgfqpoint{0.865403in}{0.709551in}}%
\pgfpathlineto{\pgfqpoint{0.868572in}{0.646270in}}%
\pgfpathlineto{\pgfqpoint{0.874911in}{0.707448in}}%
\pgfpathlineto{\pgfqpoint{0.876179in}{0.815750in}}%
\pgfpathlineto{\pgfqpoint{0.878714in}{0.752954in}}%
\pgfpathlineto{\pgfqpoint{0.879221in}{0.819063in}}%
\pgfpathlineto{\pgfqpoint{0.880562in}{0.793288in}}%
\pgfpathlineto{\pgfqpoint{0.880574in}{0.785349in}}%
\pgfpathlineto{\pgfqpoint{0.881976in}{2.623259in}}%
\pgfpathlineto{\pgfqpoint{0.885971in}{2.784899in}}%
\pgfpathlineto{\pgfqpoint{0.889163in}{2.825893in}}%
\pgfpathlineto{\pgfqpoint{0.890533in}{2.829876in}}%
\pgfpathlineto{\pgfqpoint{0.892986in}{2.853735in}}%
\pgfpathlineto{\pgfqpoint{0.894887in}{2.862345in}}%
\pgfpathlineto{\pgfqpoint{0.895418in}{2.861902in}}%
\pgfpathlineto{\pgfqpoint{0.896288in}{2.861729in}}%
\pgfpathlineto{\pgfqpoint{0.896646in}{2.862180in}}%
\pgfpathlineto{\pgfqpoint{0.898781in}{2.868170in}}%
\pgfpathlineto{\pgfqpoint{0.900218in}{2.866193in}}%
\pgfpathlineto{\pgfqpoint{0.900728in}{2.865649in}}%
\pgfpathlineto{\pgfqpoint{0.901191in}{2.867219in}}%
\pgfpathlineto{\pgfqpoint{0.901544in}{2.866599in}}%
\pgfpathlineto{\pgfqpoint{0.901662in}{2.904528in}}%
\pgfpathlineto{\pgfqpoint{0.903176in}{2.895187in}}%
\pgfpathlineto{\pgfqpoint{0.920384in}{2.799845in}}%
\pgfpathlineto{\pgfqpoint{0.935432in}{2.733622in}}%
\pgfpathlineto{\pgfqpoint{0.935449in}{2.733652in}}%
\pgfpathlineto{\pgfqpoint{0.939676in}{2.721851in}}%
\pgfpathlineto{\pgfqpoint{0.939677in}{2.721939in}}%
\pgfpathlineto{\pgfqpoint{0.941205in}{2.719332in}}%
\pgfpathlineto{\pgfqpoint{0.941206in}{2.719507in}}%
\pgfpathlineto{\pgfqpoint{0.944031in}{2.713813in}}%
\pgfpathlineto{\pgfqpoint{0.944032in}{2.713999in}}%
\pgfpathlineto{\pgfqpoint{0.945522in}{2.708663in}}%
\pgfpathlineto{\pgfqpoint{0.952469in}{2.690475in}}%
\pgfpathlineto{\pgfqpoint{0.952470in}{2.690664in}}%
\pgfpathlineto{\pgfqpoint{0.953963in}{2.686893in}}%
\pgfpathlineto{\pgfqpoint{0.960807in}{2.664586in}}%
\pgfpathlineto{\pgfqpoint{0.960808in}{2.664943in}}%
\pgfpathlineto{\pgfqpoint{0.962242in}{2.661231in}}%
\pgfpathlineto{\pgfqpoint{0.962295in}{2.661351in}}%
\pgfpathlineto{\pgfqpoint{0.965732in}{2.653010in}}%
\pgfpathlineto{\pgfqpoint{0.965734in}{2.653397in}}%
\pgfpathlineto{\pgfqpoint{0.967219in}{2.648997in}}%
\pgfpathlineto{\pgfqpoint{0.978852in}{2.603065in}}%
\pgfpathlineto{\pgfqpoint{0.987597in}{2.568874in}}%
\pgfpathlineto{\pgfqpoint{0.992347in}{2.540622in}}%
\pgfpathlineto{\pgfqpoint{0.996697in}{2.513985in}}%
\pgfpathlineto{\pgfqpoint{1.001280in}{2.491157in}}%
\pgfpathlineto{\pgfqpoint{1.002746in}{2.485708in}}%
\pgfpathlineto{\pgfqpoint{1.021470in}{2.437668in}}%
\pgfpathlineto{\pgfqpoint{1.022377in}{2.436644in}}%
\pgfpathlineto{\pgfqpoint{1.022943in}{2.437561in}}%
\pgfpathlineto{\pgfqpoint{1.023459in}{2.438012in}}%
\pgfpathlineto{\pgfqpoint{1.024026in}{2.437125in}}%
\pgfpathlineto{\pgfqpoint{1.036548in}{2.404490in}}%
\pgfpathlineto{\pgfqpoint{1.051353in}{2.364787in}}%
\pgfpathlineto{\pgfqpoint{1.091721in}{2.253122in}}%
\pgfpathlineto{\pgfqpoint{1.105858in}{2.211113in}}%
\pgfpathlineto{\pgfqpoint{1.118742in}{2.171987in}}%
\pgfpathlineto{\pgfqpoint{1.205255in}{1.897846in}}%
\pgfpathlineto{\pgfqpoint{1.264587in}{1.709755in}}%
\pgfpathlineto{\pgfqpoint{1.291008in}{1.634865in}}%
\pgfpathlineto{\pgfqpoint{1.300100in}{1.613141in}}%
\pgfpathlineto{\pgfqpoint{1.305491in}{1.600575in}}%
\pgfpathlineto{\pgfqpoint{1.313848in}{1.586244in}}%
\pgfpathlineto{\pgfqpoint{1.331193in}{1.550908in}}%
\pgfpathlineto{\pgfqpoint{1.335927in}{1.546245in}}%
\pgfpathlineto{\pgfqpoint{1.355988in}{1.511703in}}%
\pgfpathlineto{\pgfqpoint{1.363902in}{1.503823in}}%
\pgfpathlineto{\pgfqpoint{1.377565in}{1.481599in}}%
\pgfpathlineto{\pgfqpoint{1.390481in}{1.469128in}}%
\pgfpathlineto{\pgfqpoint{1.396678in}{1.451088in}}%
\pgfpathlineto{\pgfqpoint{1.400213in}{1.448224in}}%
\pgfpathlineto{\pgfqpoint{1.405137in}{1.450269in}}%
\pgfpathlineto{\pgfqpoint{1.411362in}{1.460086in}}%
\pgfpathlineto{\pgfqpoint{1.412778in}{2.577992in}}%
\pgfpathlineto{\pgfqpoint{1.415357in}{2.704638in}}%
\pgfpathlineto{\pgfqpoint{1.418495in}{2.810626in}}%
\pgfpathlineto{\pgfqpoint{1.422104in}{2.843796in}}%
\pgfpathlineto{\pgfqpoint{1.425325in}{2.872547in}}%
\pgfpathlineto{\pgfqpoint{1.427099in}{2.871591in}}%
\pgfpathlineto{\pgfqpoint{1.427489in}{2.872105in}}%
\pgfpathlineto{\pgfqpoint{1.429339in}{2.879085in}}%
\pgfpathlineto{\pgfqpoint{1.429727in}{2.878006in}}%
\pgfpathlineto{\pgfqpoint{1.434550in}{2.859802in}}%
\pgfpathlineto{\pgfqpoint{1.453263in}{2.756556in}}%
\pgfpathlineto{\pgfqpoint{1.466596in}{2.700202in}}%
\pgfpathlineto{\pgfqpoint{1.466596in}{2.700289in}}%
\pgfpathlineto{\pgfqpoint{1.468109in}{2.697490in}}%
\pgfpathlineto{\pgfqpoint{1.468123in}{2.697553in}}%
\pgfpathlineto{\pgfqpoint{1.470802in}{2.692474in}}%
\pgfpathlineto{\pgfqpoint{1.470893in}{2.692545in}}%
\pgfpathlineto{\pgfqpoint{1.470994in}{2.692828in}}%
\pgfpathlineto{\pgfqpoint{1.472367in}{2.691849in}}%
\pgfpathlineto{\pgfqpoint{1.472417in}{2.692107in}}%
\pgfpathlineto{\pgfqpoint{1.472570in}{2.692280in}}%
\pgfpathlineto{\pgfqpoint{1.475494in}{2.687384in}}%
\pgfpathlineto{\pgfqpoint{1.475495in}{2.687621in}}%
\pgfpathlineto{\pgfqpoint{1.476934in}{2.685455in}}%
\pgfpathlineto{\pgfqpoint{1.476985in}{2.685487in}}%
\pgfpathlineto{\pgfqpoint{1.477037in}{2.685554in}}%
\pgfpathlineto{\pgfqpoint{1.481976in}{2.674498in}}%
\pgfpathlineto{\pgfqpoint{1.481977in}{2.674721in}}%
\pgfpathlineto{\pgfqpoint{1.483418in}{2.672071in}}%
\pgfpathlineto{\pgfqpoint{1.483470in}{2.672100in}}%
\pgfpathlineto{\pgfqpoint{1.483522in}{2.672159in}}%
\pgfpathlineto{\pgfqpoint{1.493896in}{2.642987in}}%
\pgfpathlineto{\pgfqpoint{1.493898in}{2.643423in}}%
\pgfpathlineto{\pgfqpoint{1.495333in}{2.640009in}}%
\pgfpathlineto{\pgfqpoint{1.495386in}{2.640196in}}%
\pgfpathlineto{\pgfqpoint{1.502470in}{2.615103in}}%
\pgfpathlineto{\pgfqpoint{1.507964in}{2.594054in}}%
\pgfpathlineto{\pgfqpoint{1.525468in}{2.508003in}}%
\pgfpathlineto{\pgfqpoint{1.532338in}{2.469715in}}%
\pgfpathlineto{\pgfqpoint{1.544124in}{2.440564in}}%
\pgfpathlineto{\pgfqpoint{1.545407in}{2.446309in}}%
\pgfpathlineto{\pgfqpoint{1.546409in}{2.444912in}}%
\pgfpathlineto{\pgfqpoint{1.548287in}{2.441031in}}%
\pgfpathlineto{\pgfqpoint{1.548796in}{2.441933in}}%
\pgfpathlineto{\pgfqpoint{1.550947in}{2.450893in}}%
\pgfpathlineto{\pgfqpoint{1.551611in}{2.450210in}}%
\pgfpathlineto{\pgfqpoint{1.553326in}{2.447629in}}%
\pgfpathlineto{\pgfqpoint{1.553917in}{2.448420in}}%
\pgfpathlineto{\pgfqpoint{1.556035in}{2.446419in}}%
\pgfpathlineto{\pgfqpoint{1.559604in}{2.441403in}}%
\pgfpathlineto{\pgfqpoint{1.564399in}{2.451633in}}%
\pgfpathlineto{\pgfqpoint{1.565433in}{2.450186in}}%
\pgfpathlineto{\pgfqpoint{1.566615in}{2.448939in}}%
\pgfpathlineto{\pgfqpoint{1.568970in}{2.453578in}}%
\pgfpathlineto{\pgfqpoint{1.578276in}{2.489221in}}%
\pgfpathlineto{\pgfqpoint{1.581571in}{2.488519in}}%
\pgfpathlineto{\pgfqpoint{1.586146in}{2.484630in}}%
\pgfpathlineto{\pgfqpoint{1.591672in}{2.488606in}}%
\pgfpathlineto{\pgfqpoint{1.600101in}{2.487015in}}%
\pgfpathlineto{\pgfqpoint{1.603736in}{2.486803in}}%
\pgfpathlineto{\pgfqpoint{1.614055in}{2.478437in}}%
\pgfpathlineto{\pgfqpoint{1.630778in}{2.483435in}}%
\pgfpathlineto{\pgfqpoint{1.643248in}{2.457005in}}%
\pgfpathlineto{\pgfqpoint{1.648493in}{2.446519in}}%
\pgfpathlineto{\pgfqpoint{1.658984in}{2.436722in}}%
\pgfpathlineto{\pgfqpoint{1.670241in}{2.420444in}}%
\pgfpathlineto{\pgfqpoint{1.689718in}{2.373539in}}%
\pgfpathlineto{\pgfqpoint{1.696674in}{2.355415in}}%
\pgfpathlineto{\pgfqpoint{1.709507in}{2.311079in}}%
\pgfpathlineto{\pgfqpoint{1.736664in}{2.240710in}}%
\pgfpathlineto{\pgfqpoint{1.750375in}{2.215071in}}%
\pgfpathlineto{\pgfqpoint{1.777798in}{2.188876in}}%
\pgfpathlineto{\pgfqpoint{1.846218in}{2.074992in}}%
\pgfpathlineto{\pgfqpoint{1.849274in}{2.066464in}}%
\pgfpathlineto{\pgfqpoint{1.853349in}{2.060139in}}%
\pgfpathlineto{\pgfqpoint{1.861500in}{2.067434in}}%
\pgfpathlineto{\pgfqpoint{1.863538in}{2.065582in}}%
\pgfpathlineto{\pgfqpoint{1.865107in}{2.639466in}}%
\pgfpathlineto{\pgfqpoint{1.869411in}{2.796390in}}%
\pgfpathlineto{\pgfqpoint{1.873025in}{2.824815in}}%
\pgfpathlineto{\pgfqpoint{1.879914in}{2.833210in}}%
\pgfpathlineto{\pgfqpoint{1.884511in}{2.831120in}}%
\pgfpathlineto{\pgfqpoint{1.884629in}{2.887097in}}%
\pgfpathlineto{\pgfqpoint{1.886152in}{2.877700in}}%
\pgfpathlineto{\pgfqpoint{1.904310in}{2.776566in}}%
\pgfpathlineto{\pgfqpoint{1.918735in}{2.714307in}}%
\pgfpathlineto{\pgfqpoint{1.918736in}{2.714405in}}%
\pgfpathlineto{\pgfqpoint{1.920244in}{2.711163in}}%
\pgfpathlineto{\pgfqpoint{1.920257in}{2.711211in}}%
\pgfpathlineto{\pgfqpoint{1.923271in}{2.705307in}}%
\pgfpathlineto{\pgfqpoint{1.923272in}{2.705508in}}%
\pgfpathlineto{\pgfqpoint{1.924752in}{2.704100in}}%
\pgfpathlineto{\pgfqpoint{1.924802in}{2.704315in}}%
\pgfpathlineto{\pgfqpoint{1.928090in}{2.696712in}}%
\pgfpathlineto{\pgfqpoint{1.928091in}{2.696925in}}%
\pgfpathlineto{\pgfqpoint{1.929529in}{2.692897in}}%
\pgfpathlineto{\pgfqpoint{1.929581in}{2.692909in}}%
\pgfpathlineto{\pgfqpoint{1.929633in}{2.692953in}}%
\pgfpathlineto{\pgfqpoint{1.934881in}{2.680611in}}%
\pgfpathlineto{\pgfqpoint{1.934882in}{2.680820in}}%
\pgfpathlineto{\pgfqpoint{1.936376in}{2.677599in}}%
\pgfpathlineto{\pgfqpoint{1.943318in}{2.655442in}}%
\pgfpathlineto{\pgfqpoint{1.943319in}{2.655835in}}%
\pgfpathlineto{\pgfqpoint{1.944753in}{2.652342in}}%
\pgfpathlineto{\pgfqpoint{1.944805in}{2.652494in}}%
\pgfpathlineto{\pgfqpoint{1.948036in}{2.645203in}}%
\pgfpathlineto{\pgfqpoint{1.948038in}{2.645622in}}%
\pgfpathlineto{\pgfqpoint{1.949474in}{2.642022in}}%
\pgfpathlineto{\pgfqpoint{1.949527in}{2.642195in}}%
\pgfpathlineto{\pgfqpoint{1.960557in}{2.598856in}}%
\pgfpathlineto{\pgfqpoint{1.968212in}{2.569379in}}%
\pgfpathlineto{\pgfqpoint{1.972364in}{2.549510in}}%
\pgfpathlineto{\pgfqpoint{1.978632in}{2.511309in}}%
\pgfpathlineto{\pgfqpoint{1.982856in}{2.489052in}}%
\pgfpathlineto{\pgfqpoint{1.984035in}{2.483813in}}%
\pgfpathlineto{\pgfqpoint{1.986833in}{2.473009in}}%
\pgfpathlineto{\pgfqpoint{1.992129in}{2.457611in}}%
\pgfpathlineto{\pgfqpoint{1.995322in}{2.449202in}}%
\pgfpathlineto{\pgfqpoint{1.996192in}{2.450497in}}%
\pgfpathlineto{\pgfqpoint{1.997250in}{2.449482in}}%
\pgfpathlineto{\pgfqpoint{2.009236in}{2.427021in}}%
\pgfpathlineto{\pgfqpoint{2.011827in}{2.424473in}}%
\pgfpathlineto{\pgfqpoint{2.053749in}{2.310489in}}%
\pgfpathlineto{\pgfqpoint{2.055328in}{2.307898in}}%
\pgfpathlineto{\pgfqpoint{2.062437in}{2.289940in}}%
\pgfpathlineto{\pgfqpoint{2.064298in}{2.287775in}}%
\pgfpathlineto{\pgfqpoint{2.066128in}{2.286912in}}%
\pgfpathlineto{\pgfqpoint{2.072449in}{2.269517in}}%
\pgfpathlineto{\pgfqpoint{2.084053in}{2.234668in}}%
\pgfpathlineto{\pgfqpoint{2.086489in}{2.228188in}}%
\pgfpathlineto{\pgfqpoint{2.108980in}{2.159586in}}%
\pgfpathlineto{\pgfqpoint{2.133549in}{2.080760in}}%
\pgfpathlineto{\pgfqpoint{2.133720in}{2.088828in}}%
\pgfpathlineto{\pgfqpoint{2.134871in}{2.085508in}}%
\pgfpathlineto{\pgfqpoint{2.142848in}{2.059134in}}%
\pgfpathlineto{\pgfqpoint{2.161824in}{1.997261in}}%
\pgfpathlineto{\pgfqpoint{2.166064in}{1.983759in}}%
\pgfpathlineto{\pgfqpoint{2.237009in}{1.753070in}}%
\pgfpathlineto{\pgfqpoint{2.271083in}{1.654130in}}%
\pgfpathlineto{\pgfqpoint{2.306311in}{1.573390in}}%
\pgfpathlineto{\pgfqpoint{2.312523in}{1.563127in}}%
\pgfpathlineto{\pgfqpoint{2.323514in}{1.539823in}}%
\pgfpathlineto{\pgfqpoint{2.329188in}{1.533686in}}%
\pgfpathlineto{\pgfqpoint{2.338227in}{1.517706in}}%
\pgfpathlineto{\pgfqpoint{2.346459in}{1.509610in}}%
\pgfpathlineto{\pgfqpoint{2.359664in}{1.486996in}}%
\pgfpathlineto{\pgfqpoint{2.372155in}{1.474098in}}%
\pgfpathlineto{\pgfqpoint{2.382273in}{1.450625in}}%
\pgfpathlineto{\pgfqpoint{2.386936in}{1.451283in}}%
\pgfpathlineto{\pgfqpoint{2.392386in}{1.458269in}}%
\pgfpathlineto{\pgfqpoint{2.393397in}{1.457286in}}%
\pgfpathlineto{\pgfqpoint{2.394342in}{1.456950in}}%
\pgfpathlineto{\pgfqpoint{2.395746in}{2.593939in}}%
\pgfpathlineto{\pgfqpoint{2.398104in}{2.693391in}}%
\pgfpathlineto{\pgfqpoint{2.401290in}{2.809475in}}%
\pgfpathlineto{\pgfqpoint{2.405847in}{2.835032in}}%
\pgfpathlineto{\pgfqpoint{2.412288in}{2.865592in}}%
\pgfpathlineto{\pgfqpoint{2.412528in}{2.865252in}}%
\pgfpathlineto{\pgfqpoint{2.413431in}{2.863945in}}%
\pgfpathlineto{\pgfqpoint{2.416655in}{2.854535in}}%
\pgfpathlineto{\pgfqpoint{2.434729in}{2.753677in}}%
\pgfpathlineto{\pgfqpoint{2.449691in}{2.689087in}}%
\pgfpathlineto{\pgfqpoint{2.453537in}{2.679430in}}%
\pgfpathlineto{\pgfqpoint{2.454909in}{2.679662in}}%
\pgfpathlineto{\pgfqpoint{2.454319in}{2.679028in}}%
\pgfpathlineto{\pgfqpoint{2.455056in}{2.679399in}}%
\pgfpathlineto{\pgfqpoint{2.455914in}{2.680064in}}%
\pgfpathlineto{\pgfqpoint{2.457097in}{2.677993in}}%
\pgfpathlineto{\pgfqpoint{2.457098in}{2.678263in}}%
\pgfpathlineto{\pgfqpoint{2.458536in}{2.675632in}}%
\pgfpathlineto{\pgfqpoint{2.458588in}{2.675700in}}%
\pgfpathlineto{\pgfqpoint{2.458639in}{2.675808in}}%
\pgfpathlineto{\pgfqpoint{2.465739in}{2.658708in}}%
\pgfpathlineto{\pgfqpoint{2.465740in}{2.658968in}}%
\pgfpathlineto{\pgfqpoint{2.467182in}{2.656303in}}%
\pgfpathlineto{\pgfqpoint{2.467233in}{2.656324in}}%
\pgfpathlineto{\pgfqpoint{2.467286in}{2.656379in}}%
\pgfpathlineto{\pgfqpoint{2.473965in}{2.635129in}}%
\pgfpathlineto{\pgfqpoint{2.473967in}{2.635611in}}%
\pgfpathlineto{\pgfqpoint{2.475400in}{2.632625in}}%
\pgfpathlineto{\pgfqpoint{2.475453in}{2.632855in}}%
\pgfpathlineto{\pgfqpoint{2.478067in}{2.627893in}}%
\pgfpathlineto{\pgfqpoint{2.478069in}{2.628394in}}%
\pgfpathlineto{\pgfqpoint{2.479505in}{2.625246in}}%
\pgfpathlineto{\pgfqpoint{2.479558in}{2.625491in}}%
\pgfpathlineto{\pgfqpoint{2.487752in}{2.594469in}}%
\pgfpathlineto{\pgfqpoint{2.491410in}{2.579934in}}%
\pgfpathlineto{\pgfqpoint{2.497004in}{2.558046in}}%
\pgfpathlineto{\pgfqpoint{2.510543in}{2.496865in}}%
\pgfpathlineto{\pgfqpoint{2.511075in}{2.497022in}}%
\pgfpathlineto{\pgfqpoint{2.511311in}{2.496165in}}%
\pgfpathlineto{\pgfqpoint{2.517527in}{2.470646in}}%
\pgfpathlineto{\pgfqpoint{2.526928in}{2.446358in}}%
\pgfpathlineto{\pgfqpoint{2.528431in}{2.447577in}}%
\pgfpathlineto{\pgfqpoint{2.529002in}{2.445987in}}%
\pgfpathlineto{\pgfqpoint{2.531669in}{2.441933in}}%
\pgfpathlineto{\pgfqpoint{2.532757in}{2.444299in}}%
\pgfpathlineto{\pgfqpoint{2.533612in}{2.443446in}}%
\pgfpathlineto{\pgfqpoint{2.534742in}{2.443277in}}%
\pgfpathlineto{\pgfqpoint{2.534947in}{2.442774in}}%
\pgfpathlineto{\pgfqpoint{2.535823in}{2.440719in}}%
\pgfpathlineto{\pgfqpoint{2.536379in}{2.441537in}}%
\pgfpathlineto{\pgfqpoint{2.538930in}{2.442653in}}%
\pgfpathlineto{\pgfqpoint{2.539270in}{2.442088in}}%
\pgfpathlineto{\pgfqpoint{2.541077in}{2.440905in}}%
\pgfpathlineto{\pgfqpoint{2.541720in}{2.441396in}}%
\pgfpathlineto{\pgfqpoint{2.542104in}{2.440772in}}%
\pgfpathlineto{\pgfqpoint{2.543637in}{2.437701in}}%
\pgfpathlineto{\pgfqpoint{2.546027in}{2.440832in}}%
\pgfpathlineto{\pgfqpoint{2.549196in}{2.439688in}}%
\pgfpathlineto{\pgfqpoint{2.553117in}{2.450934in}}%
\pgfpathlineto{\pgfqpoint{2.561677in}{2.479534in}}%
\pgfpathlineto{\pgfqpoint{2.565500in}{2.481579in}}%
\pgfpathlineto{\pgfqpoint{2.567414in}{2.481733in}}%
\pgfpathlineto{\pgfqpoint{2.570059in}{2.484381in}}%
\pgfpathlineto{\pgfqpoint{2.570423in}{2.483087in}}%
\pgfpathlineto{\pgfqpoint{2.573390in}{2.539349in}}%
\pgfpathlineto{\pgfqpoint{2.574635in}{2.534968in}}%
\pgfpathlineto{\pgfqpoint{2.578275in}{2.528307in}}%
\pgfpathlineto{\pgfqpoint{2.587153in}{2.512192in}}%
\pgfpathlineto{\pgfqpoint{2.592246in}{2.502994in}}%
\pgfpathlineto{\pgfqpoint{2.596426in}{2.498895in}}%
\pgfpathlineto{\pgfqpoint{2.602472in}{2.496339in}}%
\pgfpathlineto{\pgfqpoint{2.614970in}{2.475104in}}%
\pgfpathlineto{\pgfqpoint{2.635962in}{2.430177in}}%
\pgfpathlineto{\pgfqpoint{2.650375in}{2.404580in}}%
\pgfpathlineto{\pgfqpoint{2.664609in}{2.374010in}}%
\pgfpathlineto{\pgfqpoint{2.695684in}{2.313280in}}%
\pgfpathlineto{\pgfqpoint{2.711497in}{2.287985in}}%
\pgfpathlineto{\pgfqpoint{2.732535in}{2.258590in}}%
\pgfpathlineto{\pgfqpoint{2.769489in}{2.193506in}}%
\pgfpathlineto{\pgfqpoint{2.808772in}{2.155469in}}%
\pgfpathlineto{\pgfqpoint{2.830114in}{2.113588in}}%
\pgfpathlineto{\pgfqpoint{2.835955in}{2.102259in}}%
\pgfpathlineto{\pgfqpoint{2.843743in}{2.106318in}}%
\pgfpathlineto{\pgfqpoint{2.846508in}{2.103925in}}%
\pgfpathlineto{\pgfqpoint{2.847910in}{2.619613in}}%
\pgfpathlineto{\pgfqpoint{2.851905in}{2.785399in}}%
\pgfpathlineto{\pgfqpoint{2.855098in}{2.826212in}}%
\pgfpathlineto{\pgfqpoint{2.856468in}{2.830160in}}%
\pgfpathlineto{\pgfqpoint{2.858035in}{2.841524in}}%
\pgfpathlineto{\pgfqpoint{2.860821in}{2.862562in}}%
\pgfpathlineto{\pgfqpoint{2.862221in}{2.861909in}}%
\pgfpathlineto{\pgfqpoint{2.862580in}{2.862364in}}%
\pgfpathlineto{\pgfqpoint{2.864713in}{2.868349in}}%
\pgfpathlineto{\pgfqpoint{2.866152in}{2.866352in}}%
\pgfpathlineto{\pgfqpoint{2.866659in}{2.865775in}}%
\pgfpathlineto{\pgfqpoint{2.867123in}{2.867329in}}%
\pgfpathlineto{\pgfqpoint{2.867478in}{2.866666in}}%
\pgfpathlineto{\pgfqpoint{2.867596in}{2.904538in}}%
\pgfpathlineto{\pgfqpoint{2.869110in}{2.895192in}}%
\pgfpathlineto{\pgfqpoint{2.885680in}{2.803820in}}%
\pgfpathlineto{\pgfqpoint{2.885682in}{2.803821in}}%
\pgfpathlineto{\pgfqpoint{2.900376in}{2.739195in}}%
\pgfpathlineto{\pgfqpoint{2.900377in}{2.739292in}}%
\pgfpathlineto{\pgfqpoint{2.901878in}{2.733705in}}%
\pgfpathlineto{\pgfqpoint{2.901891in}{2.733723in}}%
\pgfpathlineto{\pgfqpoint{2.905725in}{2.724296in}}%
\pgfpathlineto{\pgfqpoint{2.905726in}{2.724382in}}%
\pgfpathlineto{\pgfqpoint{2.907200in}{2.721737in}}%
\pgfpathlineto{\pgfqpoint{2.907248in}{2.721867in}}%
\pgfpathlineto{\pgfqpoint{2.912061in}{2.710685in}}%
\pgfpathlineto{\pgfqpoint{2.912062in}{2.710865in}}%
\pgfpathlineto{\pgfqpoint{2.913553in}{2.707103in}}%
\pgfpathlineto{\pgfqpoint{2.918855in}{2.694312in}}%
\pgfpathlineto{\pgfqpoint{2.918856in}{2.694493in}}%
\pgfpathlineto{\pgfqpoint{2.920350in}{2.690968in}}%
\pgfpathlineto{\pgfqpoint{2.931919in}{2.656262in}}%
\pgfpathlineto{\pgfqpoint{2.931921in}{2.656636in}}%
\pgfpathlineto{\pgfqpoint{2.933402in}{2.651417in}}%
\pgfpathlineto{\pgfqpoint{2.945651in}{2.602854in}}%
\pgfpathlineto{\pgfqpoint{2.954149in}{2.568855in}}%
\pgfpathlineto{\pgfqpoint{2.959342in}{2.539106in}}%
\pgfpathlineto{\pgfqpoint{2.962657in}{2.518897in}}%
\pgfpathlineto{\pgfqpoint{2.971762in}{2.477662in}}%
\pgfpathlineto{\pgfqpoint{2.976879in}{2.462995in}}%
\pgfpathlineto{\pgfqpoint{2.988543in}{2.439415in}}%
\pgfpathlineto{\pgfqpoint{2.989668in}{2.437983in}}%
\pgfpathlineto{\pgfqpoint{2.991070in}{2.435766in}}%
\pgfpathlineto{\pgfqpoint{2.991573in}{2.436038in}}%
\pgfpathlineto{\pgfqpoint{2.992513in}{2.436124in}}%
\pgfpathlineto{\pgfqpoint{2.992823in}{2.435630in}}%
\pgfpathlineto{\pgfqpoint{2.996424in}{2.427894in}}%
\pgfpathlineto{\pgfqpoint{3.007297in}{2.398749in}}%
\pgfpathlineto{\pgfqpoint{3.007369in}{2.405935in}}%
\pgfpathlineto{\pgfqpoint{3.008782in}{2.403605in}}%
\pgfpathlineto{\pgfqpoint{3.016045in}{2.385531in}}%
\pgfpathlineto{\pgfqpoint{3.052595in}{2.278167in}}%
\pgfpathlineto{\pgfqpoint{3.055075in}{2.274502in}}%
\pgfpathlineto{\pgfqpoint{3.063499in}{2.250106in}}%
\pgfpathlineto{\pgfqpoint{3.074090in}{2.216996in}}%
\pgfpathlineto{\pgfqpoint{3.095139in}{2.150998in}}%
\pgfpathlineto{\pgfqpoint{3.194113in}{1.832819in}}%
\pgfpathlineto{\pgfqpoint{3.233038in}{1.713927in}}%
\pgfpathlineto{\pgfqpoint{3.237523in}{1.701499in}}%
\pgfpathlineto{\pgfqpoint{3.244017in}{1.686830in}}%
\pgfpathlineto{\pgfqpoint{3.282160in}{1.590506in}}%
\pgfpathlineto{\pgfqpoint{3.294094in}{1.566310in}}%
\pgfpathlineto{\pgfqpoint{3.301150in}{1.554773in}}%
\pgfpathlineto{\pgfqpoint{3.303653in}{1.548910in}}%
\pgfpathlineto{\pgfqpoint{3.311921in}{1.535064in}}%
\pgfpathlineto{\pgfqpoint{3.315244in}{1.527775in}}%
\pgfpathlineto{\pgfqpoint{3.321890in}{1.520149in}}%
\pgfpathlineto{\pgfqpoint{3.332621in}{1.500803in}}%
\pgfpathlineto{\pgfqpoint{3.342671in}{1.490312in}}%
\pgfpathlineto{\pgfqpoint{3.352404in}{1.476538in}}%
\pgfpathlineto{\pgfqpoint{3.363181in}{1.451438in}}%
\pgfpathlineto{\pgfqpoint{3.366961in}{1.449180in}}%
\pgfpathlineto{\pgfqpoint{3.372514in}{1.450826in}}%
\pgfpathlineto{\pgfqpoint{3.377301in}{1.448747in}}%
\pgfpathlineto{\pgfqpoint{3.377336in}{2.655147in}}%
\pgfpathlineto{\pgfqpoint{3.378712in}{2.402313in}}%
\pgfpathlineto{\pgfqpoint{3.381292in}{2.669444in}}%
\pgfpathlineto{\pgfqpoint{3.384429in}{2.801508in}}%
\pgfpathlineto{\pgfqpoint{3.388042in}{2.838683in}}%
\pgfpathlineto{\pgfqpoint{3.391281in}{2.869974in}}%
\pgfpathlineto{\pgfqpoint{3.392996in}{2.868859in}}%
\pgfpathlineto{\pgfqpoint{3.393399in}{2.869557in}}%
\pgfpathlineto{\pgfqpoint{3.395713in}{2.873327in}}%
\pgfpathlineto{\pgfqpoint{3.396324in}{2.871861in}}%
\pgfpathlineto{\pgfqpoint{3.400840in}{2.854843in}}%
\pgfpathlineto{\pgfqpoint{3.402446in}{2.843793in}}%
\pgfpathlineto{\pgfqpoint{3.419173in}{2.755403in}}%
\pgfpathlineto{\pgfqpoint{3.431710in}{2.702463in}}%
\pgfpathlineto{\pgfqpoint{3.431710in}{2.702559in}}%
\pgfpathlineto{\pgfqpoint{3.433213in}{2.697900in}}%
\pgfpathlineto{\pgfqpoint{3.433227in}{2.697961in}}%
\pgfpathlineto{\pgfqpoint{3.435999in}{2.693106in}}%
\pgfpathlineto{\pgfqpoint{3.435999in}{2.693225in}}%
\pgfpathlineto{\pgfqpoint{3.437352in}{2.691195in}}%
\pgfpathlineto{\pgfqpoint{3.437498in}{2.691316in}}%
\pgfpathlineto{\pgfqpoint{3.437549in}{2.691582in}}%
\pgfpathlineto{\pgfqpoint{3.439640in}{2.689630in}}%
\pgfpathlineto{\pgfqpoint{3.439640in}{2.689871in}}%
\pgfpathlineto{\pgfqpoint{3.441078in}{2.686253in}}%
\pgfpathlineto{\pgfqpoint{3.441130in}{2.686296in}}%
\pgfpathlineto{\pgfqpoint{3.444163in}{2.678948in}}%
\pgfpathlineto{\pgfqpoint{3.441182in}{2.686375in}}%
\pgfpathlineto{\pgfqpoint{3.444172in}{2.679180in}}%
\pgfpathlineto{\pgfqpoint{3.444192in}{2.682267in}}%
\pgfpathlineto{\pgfqpoint{3.445684in}{2.679261in}}%
\pgfpathlineto{\pgfqpoint{3.449030in}{2.672642in}}%
\pgfpathlineto{\pgfqpoint{3.449031in}{2.672867in}}%
\pgfpathlineto{\pgfqpoint{3.450524in}{2.669987in}}%
\pgfpathlineto{\pgfqpoint{3.450577in}{2.670007in}}%
\pgfpathlineto{\pgfqpoint{3.457270in}{2.648399in}}%
\pgfpathlineto{\pgfqpoint{3.457272in}{2.648820in}}%
\pgfpathlineto{\pgfqpoint{3.458705in}{2.645477in}}%
\pgfpathlineto{\pgfqpoint{3.458758in}{2.645653in}}%
\pgfpathlineto{\pgfqpoint{3.461784in}{2.639099in}}%
\pgfpathlineto{\pgfqpoint{3.461786in}{2.639545in}}%
\pgfpathlineto{\pgfqpoint{3.463222in}{2.636070in}}%
\pgfpathlineto{\pgfqpoint{3.463275in}{2.636266in}}%
\pgfpathlineto{\pgfqpoint{3.473901in}{2.594330in}}%
\pgfpathlineto{\pgfqpoint{3.479489in}{2.571860in}}%
\pgfpathlineto{\pgfqpoint{3.492475in}{2.501383in}}%
\pgfpathlineto{\pgfqpoint{3.496809in}{2.476995in}}%
\pgfpathlineto{\pgfqpoint{3.502820in}{2.456434in}}%
\pgfpathlineto{\pgfqpoint{3.504519in}{2.453272in}}%
\pgfpathlineto{\pgfqpoint{3.510795in}{2.441127in}}%
\pgfpathlineto{\pgfqpoint{3.516727in}{2.434302in}}%
\pgfpathlineto{\pgfqpoint{3.517689in}{2.435765in}}%
\pgfpathlineto{\pgfqpoint{3.518227in}{2.434373in}}%
\pgfpathlineto{\pgfqpoint{3.518500in}{2.433983in}}%
\pgfpathlineto{\pgfqpoint{3.519034in}{2.434807in}}%
\pgfpathlineto{\pgfqpoint{3.520886in}{2.438748in}}%
\pgfpathlineto{\pgfqpoint{3.521713in}{2.437223in}}%
\pgfpathlineto{\pgfqpoint{3.522354in}{2.435487in}}%
\pgfpathlineto{\pgfqpoint{3.522986in}{2.436269in}}%
\pgfpathlineto{\pgfqpoint{3.525932in}{2.441374in}}%
\pgfpathlineto{\pgfqpoint{3.526314in}{2.440677in}}%
\pgfpathlineto{\pgfqpoint{3.527743in}{2.438199in}}%
\pgfpathlineto{\pgfqpoint{3.535022in}{2.459526in}}%
\pgfpathlineto{\pgfqpoint{3.540340in}{2.460857in}}%
\pgfpathlineto{\pgfqpoint{3.543725in}{2.466132in}}%
\pgfpathlineto{\pgfqpoint{3.546533in}{2.466148in}}%
\pgfpathlineto{\pgfqpoint{3.550713in}{2.470915in}}%
\pgfpathlineto{\pgfqpoint{3.557245in}{2.467318in}}%
\pgfpathlineto{\pgfqpoint{3.559914in}{2.469385in}}%
\pgfpathlineto{\pgfqpoint{3.568064in}{2.481057in}}%
\pgfpathlineto{\pgfqpoint{3.589815in}{2.463860in}}%
\pgfpathlineto{\pgfqpoint{3.593562in}{2.457570in}}%
\pgfpathlineto{\pgfqpoint{3.600412in}{2.453895in}}%
\pgfpathlineto{\pgfqpoint{3.622509in}{2.408959in}}%
\pgfpathlineto{\pgfqpoint{3.631975in}{2.398505in}}%
\pgfpathlineto{\pgfqpoint{3.657025in}{2.352641in}}%
\pgfpathlineto{\pgfqpoint{3.685349in}{2.290705in}}%
\pgfpathlineto{\pgfqpoint{3.700285in}{2.264742in}}%
\pgfpathlineto{\pgfqpoint{3.712909in}{2.237856in}}%
\pgfpathlineto{\pgfqpoint{3.730204in}{2.210032in}}%
\pgfpathlineto{\pgfqpoint{3.757302in}{2.174322in}}%
\pgfpathlineto{\pgfqpoint{3.812737in}{2.093774in}}%
\pgfpathlineto{\pgfqpoint{3.817568in}{2.083172in}}%
\pgfpathlineto{\pgfqpoint{3.829475in}{2.078277in}}%
\pgfpathlineto{\pgfqpoint{3.829526in}{1.675985in}}%
\pgfpathlineto{\pgfqpoint{3.830877in}{2.624528in}}%
\pgfpathlineto{\pgfqpoint{3.834872in}{2.786406in}}%
\pgfpathlineto{\pgfqpoint{3.838065in}{2.826775in}}%
\pgfpathlineto{\pgfqpoint{3.839435in}{2.830623in}}%
\pgfpathlineto{\pgfqpoint{3.841886in}{2.854221in}}%
\pgfpathlineto{\pgfqpoint{3.843789in}{2.862778in}}%
\pgfpathlineto{\pgfqpoint{3.844325in}{2.862372in}}%
\pgfpathlineto{\pgfqpoint{3.845204in}{2.862262in}}%
\pgfpathlineto{\pgfqpoint{3.845565in}{2.862728in}}%
\pgfpathlineto{\pgfqpoint{3.847719in}{2.869025in}}%
\pgfpathlineto{\pgfqpoint{3.849081in}{2.866660in}}%
\pgfpathlineto{\pgfqpoint{3.849838in}{2.867956in}}%
\pgfpathlineto{\pgfqpoint{3.850392in}{2.870701in}}%
\pgfpathlineto{\pgfqpoint{3.850445in}{2.870419in}}%
\pgfpathlineto{\pgfqpoint{3.850550in}{2.907137in}}%
\pgfpathlineto{\pgfqpoint{3.852066in}{2.897791in}}%
\pgfpathlineto{\pgfqpoint{3.869485in}{2.801220in}}%
\pgfpathlineto{\pgfqpoint{3.883953in}{2.738310in}}%
\pgfpathlineto{\pgfqpoint{3.883960in}{2.738374in}}%
\pgfpathlineto{\pgfqpoint{3.888362in}{2.726115in}}%
\pgfpathlineto{\pgfqpoint{3.890975in}{2.721869in}}%
\pgfpathlineto{\pgfqpoint{3.890976in}{2.722222in}}%
\pgfpathlineto{\pgfqpoint{3.892487in}{2.718245in}}%
\pgfpathlineto{\pgfqpoint{3.892501in}{2.718358in}}%
\pgfpathlineto{\pgfqpoint{3.898143in}{2.701678in}}%
\pgfpathlineto{\pgfqpoint{3.898144in}{2.701858in}}%
\pgfpathlineto{\pgfqpoint{3.899636in}{2.698666in}}%
\pgfpathlineto{\pgfqpoint{3.910694in}{2.665989in}}%
\pgfpathlineto{\pgfqpoint{3.910696in}{2.666339in}}%
\pgfpathlineto{\pgfqpoint{3.912130in}{2.662532in}}%
\pgfpathlineto{\pgfqpoint{3.912183in}{2.662645in}}%
\pgfpathlineto{\pgfqpoint{3.915621in}{2.654128in}}%
\pgfpathlineto{\pgfqpoint{3.915624in}{2.654507in}}%
\pgfpathlineto{\pgfqpoint{3.917099in}{2.648077in}}%
\pgfpathlineto{\pgfqpoint{3.927724in}{2.606637in}}%
\pgfpathlineto{\pgfqpoint{3.927724in}{2.606679in}}%
\pgfpathlineto{\pgfqpoint{3.929145in}{2.600775in}}%
\pgfpathlineto{\pgfqpoint{3.937106in}{2.568367in}}%
\pgfpathlineto{\pgfqpoint{3.952297in}{2.490011in}}%
\pgfpathlineto{\pgfqpoint{3.953551in}{2.486883in}}%
\pgfpathlineto{\pgfqpoint{3.961314in}{2.462655in}}%
\pgfpathlineto{\pgfqpoint{3.969064in}{2.443836in}}%
\pgfpathlineto{\pgfqpoint{3.975812in}{2.432734in}}%
\pgfpathlineto{\pgfqpoint{3.980796in}{2.423724in}}%
\pgfpathlineto{\pgfqpoint{3.982032in}{2.421175in}}%
\pgfpathlineto{\pgfqpoint{3.994628in}{2.387524in}}%
\pgfpathlineto{\pgfqpoint{3.997788in}{2.382593in}}%
\pgfpathlineto{\pgfqpoint{4.000839in}{2.375257in}}%
\pgfpathlineto{\pgfqpoint{4.010427in}{2.349225in}}%
\pgfpathlineto{\pgfqpoint{4.012876in}{2.343029in}}%
\pgfpathlineto{\pgfqpoint{4.077043in}{2.151821in}}%
\pgfpathlineto{\pgfqpoint{4.197483in}{1.765493in}}%
\pgfpathlineto{\pgfqpoint{4.202268in}{1.751846in}}%
\pgfpathlineto{\pgfqpoint{4.218324in}{1.703886in}}%
\pgfpathlineto{\pgfqpoint{4.260236in}{1.606041in}}%
\pgfpathlineto{\pgfqpoint{4.265814in}{1.605264in}}%
\pgfpathlineto{\pgfqpoint{4.270046in}{1.600222in}}%
\pgfpathlineto{\pgfqpoint{4.292680in}{1.553964in}}%
\pgfpathlineto{\pgfqpoint{4.307257in}{1.536397in}}%
\pgfpathlineto{\pgfqpoint{4.317920in}{1.514693in}}%
\pgfpathlineto{\pgfqpoint{4.320217in}{1.555000in}}%
\pgfpathlineto{\pgfqpoint{4.321023in}{1.551952in}}%
\pgfpathlineto{\pgfqpoint{4.333102in}{1.509643in}}%
\pgfpathlineto{\pgfqpoint{4.346909in}{1.465359in}}%
\pgfpathlineto{\pgfqpoint{4.354656in}{1.450462in}}%
\pgfpathlineto{\pgfqpoint{4.360269in}{1.445444in}}%
\pgfpathlineto{\pgfqpoint{4.361680in}{2.610757in}}%
\pgfpathlineto{\pgfqpoint{4.365095in}{2.763079in}}%
\pgfpathlineto{\pgfqpoint{4.368409in}{2.825691in}}%
\pgfpathlineto{\pgfqpoint{4.371389in}{2.834679in}}%
\pgfpathlineto{\pgfqpoint{4.379136in}{2.868079in}}%
\pgfpathlineto{\pgfqpoint{4.380503in}{2.865792in}}%
\pgfpathlineto{\pgfqpoint{4.381135in}{2.865946in}}%
\pgfpathlineto{\pgfqpoint{4.381390in}{2.864531in}}%
\pgfpathlineto{\pgfqpoint{4.399033in}{2.764125in}}%
\pgfpathlineto{\pgfqpoint{4.412120in}{2.707175in}}%
\pgfpathlineto{\pgfqpoint{4.412121in}{2.707442in}}%
\pgfpathlineto{\pgfqpoint{4.413646in}{2.701227in}}%
\pgfpathlineto{\pgfqpoint{4.417772in}{2.689808in}}%
\pgfpathlineto{\pgfqpoint{4.417773in}{2.689934in}}%
\pgfpathlineto{\pgfqpoint{4.419302in}{2.687301in}}%
\pgfpathlineto{\pgfqpoint{4.419328in}{2.687332in}}%
\pgfpathlineto{\pgfqpoint{4.421053in}{2.686409in}}%
\pgfpathlineto{\pgfqpoint{4.421861in}{2.686946in}}%
\pgfpathlineto{\pgfqpoint{4.422530in}{2.685437in}}%
\pgfpathlineto{\pgfqpoint{4.422582in}{2.685479in}}%
\pgfpathlineto{\pgfqpoint{4.422633in}{2.685575in}}%
\pgfpathlineto{\pgfqpoint{4.427362in}{2.675654in}}%
\pgfpathlineto{\pgfqpoint{4.427363in}{2.675897in}}%
\pgfpathlineto{\pgfqpoint{4.428803in}{2.671925in}}%
\pgfpathlineto{\pgfqpoint{4.428854in}{2.671970in}}%
\pgfpathlineto{\pgfqpoint{4.428906in}{2.672050in}}%
\pgfpathlineto{\pgfqpoint{4.432407in}{2.665721in}}%
\pgfpathlineto{\pgfqpoint{4.432408in}{2.665962in}}%
\pgfpathlineto{\pgfqpoint{4.433850in}{2.663221in}}%
\pgfpathlineto{\pgfqpoint{4.433902in}{2.663224in}}%
\pgfpathlineto{\pgfqpoint{4.433954in}{2.663258in}}%
\pgfpathlineto{\pgfqpoint{4.440731in}{2.641489in}}%
\pgfpathlineto{\pgfqpoint{4.440733in}{2.641939in}}%
\pgfpathlineto{\pgfqpoint{4.442167in}{2.638722in}}%
\pgfpathlineto{\pgfqpoint{4.442219in}{2.638923in}}%
\pgfpathlineto{\pgfqpoint{4.445040in}{2.633064in}}%
\pgfpathlineto{\pgfqpoint{4.445042in}{2.633536in}}%
\pgfpathlineto{\pgfqpoint{4.446530in}{2.629913in}}%
\pgfpathlineto{\pgfqpoint{4.456950in}{2.588486in}}%
\pgfpathlineto{\pgfqpoint{4.462228in}{2.567248in}}%
\pgfpathlineto{\pgfqpoint{4.468189in}{2.541573in}}%
\pgfpathlineto{\pgfqpoint{4.470647in}{2.526129in}}%
\pgfpathlineto{\pgfqpoint{4.473543in}{2.507263in}}%
\pgfpathlineto{\pgfqpoint{4.476081in}{2.490065in}}%
\pgfpathlineto{\pgfqpoint{4.483061in}{2.461772in}}%
\pgfpathlineto{\pgfqpoint{4.483461in}{2.462637in}}%
\pgfpathlineto{\pgfqpoint{4.484167in}{2.461052in}}%
\pgfpathlineto{\pgfqpoint{4.486888in}{2.453368in}}%
\pgfpathlineto{\pgfqpoint{4.487365in}{2.454826in}}%
\pgfpathlineto{\pgfqpoint{4.487786in}{2.455653in}}%
\pgfpathlineto{\pgfqpoint{4.488522in}{2.454106in}}%
\pgfpathlineto{\pgfqpoint{4.492555in}{2.445512in}}%
\pgfpathlineto{\pgfqpoint{4.495362in}{2.443003in}}%
\pgfpathlineto{\pgfqpoint{4.497750in}{2.439443in}}%
\pgfpathlineto{\pgfqpoint{4.498890in}{2.442606in}}%
\pgfpathlineto{\pgfqpoint{4.499452in}{2.441373in}}%
\pgfpathlineto{\pgfqpoint{4.503309in}{2.438978in}}%
\pgfpathlineto{\pgfqpoint{4.504069in}{2.440968in}}%
\pgfpathlineto{\pgfqpoint{4.504839in}{2.439310in}}%
\pgfpathlineto{\pgfqpoint{4.505439in}{2.437652in}}%
\pgfpathlineto{\pgfqpoint{4.506018in}{2.437930in}}%
\pgfpathlineto{\pgfqpoint{4.508869in}{2.442293in}}%
\pgfpathlineto{\pgfqpoint{4.509518in}{2.440750in}}%
\pgfpathlineto{\pgfqpoint{4.510132in}{2.439364in}}%
\pgfpathlineto{\pgfqpoint{4.510833in}{2.439695in}}%
\pgfpathlineto{\pgfqpoint{4.513944in}{2.447317in}}%
\pgfpathlineto{\pgfqpoint{4.514715in}{2.447384in}}%
\pgfpathlineto{\pgfqpoint{4.517055in}{2.452279in}}%
\pgfpathlineto{\pgfqpoint{4.517790in}{2.451814in}}%
\pgfpathlineto{\pgfqpoint{4.518428in}{2.450999in}}%
\pgfpathlineto{\pgfqpoint{4.520708in}{2.453452in}}%
\pgfpathlineto{\pgfqpoint{4.524131in}{2.451522in}}%
\pgfpathlineto{\pgfqpoint{4.525259in}{2.451641in}}%
\pgfpathlineto{\pgfqpoint{4.528533in}{2.456423in}}%
\pgfpathlineto{\pgfqpoint{4.531473in}{2.457080in}}%
\pgfpathlineto{\pgfqpoint{4.535966in}{2.463472in}}%
\pgfpathlineto{\pgfqpoint{4.539380in}{2.462568in}}%
\pgfpathlineto{\pgfqpoint{4.546921in}{2.465919in}}%
\pgfpathlineto{\pgfqpoint{4.553395in}{2.467777in}}%
\pgfpathlineto{\pgfqpoint{4.561903in}{2.469527in}}%
\pgfpathlineto{\pgfqpoint{4.570728in}{2.464092in}}%
\pgfpathlineto{\pgfqpoint{4.579848in}{2.468775in}}%
\pgfpathlineto{\pgfqpoint{4.585728in}{2.461375in}}%
\pgfpathlineto{\pgfqpoint{4.589462in}{2.452967in}}%
\pgfpathlineto{\pgfqpoint{4.596241in}{2.443668in}}%
\pgfpathlineto{\pgfqpoint{4.608788in}{2.413145in}}%
\pgfpathlineto{\pgfqpoint{4.635355in}{2.368260in}}%
\pgfpathlineto{\pgfqpoint{4.652932in}{2.329333in}}%
\pgfpathlineto{\pgfqpoint{4.669766in}{2.303633in}}%
\pgfpathlineto{\pgfqpoint{4.696065in}{2.248847in}}%
\pgfpathlineto{\pgfqpoint{4.721359in}{2.217178in}}%
\pgfpathlineto{\pgfqpoint{4.754794in}{2.181670in}}%
\pgfpathlineto{\pgfqpoint{4.754794in}{2.181670in}}%
\pgfusepath{stroke}%
\end{pgfscope}%
\begin{pgfscope}%
\pgfsetrectcap%
\pgfsetmiterjoin%
\pgfsetlinewidth{0.803000pt}%
\definecolor{currentstroke}{rgb}{0.000000,0.000000,0.000000}%
\pgfsetstrokecolor{currentstroke}%
\pgfsetdash{}{0pt}%
\pgfpathmoveto{\pgfqpoint{0.626312in}{0.402778in}}%
\pgfpathlineto{\pgfqpoint{0.626312in}{3.340278in}}%
\pgfusepath{stroke}%
\end{pgfscope}%
\begin{pgfscope}%
\pgfsetrectcap%
\pgfsetmiterjoin%
\pgfsetlinewidth{0.803000pt}%
\definecolor{currentstroke}{rgb}{0.000000,0.000000,0.000000}%
\pgfsetstrokecolor{currentstroke}%
\pgfsetdash{}{0pt}%
\pgfpathmoveto{\pgfqpoint{4.951389in}{0.402778in}}%
\pgfpathlineto{\pgfqpoint{4.951389in}{3.340278in}}%
\pgfusepath{stroke}%
\end{pgfscope}%
\begin{pgfscope}%
\pgfsetrectcap%
\pgfsetmiterjoin%
\pgfsetlinewidth{0.803000pt}%
\definecolor{currentstroke}{rgb}{0.000000,0.000000,0.000000}%
\pgfsetstrokecolor{currentstroke}%
\pgfsetdash{}{0pt}%
\pgfpathmoveto{\pgfqpoint{0.626312in}{0.402778in}}%
\pgfpathlineto{\pgfqpoint{4.951389in}{0.402778in}}%
\pgfusepath{stroke}%
\end{pgfscope}%
\begin{pgfscope}%
\pgfsetrectcap%
\pgfsetmiterjoin%
\pgfsetlinewidth{0.803000pt}%
\definecolor{currentstroke}{rgb}{0.000000,0.000000,0.000000}%
\pgfsetstrokecolor{currentstroke}%
\pgfsetdash{}{0pt}%
\pgfpathmoveto{\pgfqpoint{0.626312in}{3.340278in}}%
\pgfpathlineto{\pgfqpoint{4.951389in}{3.340278in}}%
\pgfusepath{stroke}%
\end{pgfscope}%
\begin{pgfscope}%
\pgfsetbuttcap%
\pgfsetmiterjoin%
\definecolor{currentfill}{rgb}{1.000000,1.000000,1.000000}%
\pgfsetfillcolor{currentfill}%
\pgfsetfillopacity{0.800000}%
\pgfsetlinewidth{1.003750pt}%
\definecolor{currentstroke}{rgb}{0.800000,0.800000,0.800000}%
\pgfsetstrokecolor{currentstroke}%
\pgfsetstrokeopacity{0.800000}%
\pgfsetdash{}{0pt}%
\pgfpathmoveto{\pgfqpoint{4.340278in}{0.472222in}}%
\pgfpathlineto{\pgfqpoint{4.854167in}{0.472222in}}%
\pgfpathquadraticcurveto{\pgfqpoint{4.881944in}{0.472222in}}{\pgfqpoint{4.881944in}{0.500000in}}%
\pgfpathlineto{\pgfqpoint{4.881944in}{1.075222in}}%
\pgfpathquadraticcurveto{\pgfqpoint{4.881944in}{1.103000in}}{\pgfqpoint{4.854167in}{1.103000in}}%
\pgfpathlineto{\pgfqpoint{4.340278in}{1.103000in}}%
\pgfpathquadraticcurveto{\pgfqpoint{4.312500in}{1.103000in}}{\pgfqpoint{4.312500in}{1.075222in}}%
\pgfpathlineto{\pgfqpoint{4.312500in}{0.500000in}}%
\pgfpathquadraticcurveto{\pgfqpoint{4.312500in}{0.472222in}}{\pgfqpoint{4.340278in}{0.472222in}}%
\pgfpathclose%
\pgfusepath{stroke,fill}%
\end{pgfscope}%
\begin{pgfscope}%
\pgfsetrectcap%
\pgfsetroundjoin%
\pgfsetlinewidth{1.505625pt}%
\definecolor{currentstroke}{rgb}{0.121569,0.466667,0.705882}%
\pgfsetstrokecolor{currentstroke}%
\pgfsetdash{}{0pt}%
\pgfpathmoveto{\pgfqpoint{4.368056in}{0.998833in}}%
\pgfpathlineto{\pgfqpoint{4.645833in}{0.998833in}}%
\pgfusepath{stroke}%
\end{pgfscope}%
\begin{pgfscope}%
\pgftext[x=4.756944in,y=0.950222in,left,base]{\rmfamily\fontsize{10.000000}{12.000000}\selectfont 1}%
\end{pgfscope}%
\begin{pgfscope}%
\pgfsetrectcap%
\pgfsetroundjoin%
\pgfsetlinewidth{1.505625pt}%
\definecolor{currentstroke}{rgb}{1.000000,0.498039,0.054902}%
\pgfsetstrokecolor{currentstroke}%
\pgfsetdash{}{0pt}%
\pgfpathmoveto{\pgfqpoint{4.368056in}{0.802463in}}%
\pgfpathlineto{\pgfqpoint{4.645833in}{0.802463in}}%
\pgfusepath{stroke}%
\end{pgfscope}%
\begin{pgfscope}%
\pgftext[x=4.756944in,y=0.753852in,left,base]{\rmfamily\fontsize{10.000000}{12.000000}\selectfont 2}%
\end{pgfscope}%
\begin{pgfscope}%
\pgfsetrectcap%
\pgfsetroundjoin%
\pgfsetlinewidth{1.505625pt}%
\definecolor{currentstroke}{rgb}{0.172549,0.627451,0.172549}%
\pgfsetstrokecolor{currentstroke}%
\pgfsetdash{}{0pt}%
\pgfpathmoveto{\pgfqpoint{4.368056in}{0.606092in}}%
\pgfpathlineto{\pgfqpoint{4.645833in}{0.606092in}}%
\pgfusepath{stroke}%
\end{pgfscope}%
\begin{pgfscope}%
\pgftext[x=4.756944in,y=0.557481in,left,base]{\rmfamily\fontsize{10.000000}{12.000000}\selectfont 3}%
\end{pgfscope}%
\end{pgfpicture}%
\makeatother%
\endgroup%

    \caption{Lokaler Fehler nach verschiedenen Iterationen für den Kondensator(oben) und den Invertierer(unten).}
    \label{fig:error_local_sim}
\end{figure}
\begin{figure}[ht]
    \centering
        %% Creator: Matplotlib, PGF backend
%%
%% To include the figure in your LaTeX document, write
%%   \input{<filename>.pgf}
%%
%% Make sure the required packages are loaded in your preamble
%%   \usepackage{pgf}
%%
%% Figures using additional raster images can only be included by \input if
%% they are in the same directory as the main LaTeX file. For loading figures
%% from other directories you can use the `import` package
%%   \usepackage{import}
%% and then include the figures with
%%   \import{<path to file>}{<filename>.pgf}
%%
%% Matplotlib used the following preamble
%%   \usepackage{fontspec}
%%   \setmainfont{Times New Roman}
%%   \setsansfont{Lucida Grande}
%%   \setmonofont{Andale Mono}
%%
\begingroup%
\makeatletter%
\begin{pgfpicture}%
\pgfpathrectangle{\pgfpointorigin}{\pgfqpoint{5.000000in}{3.083333in}}%
\pgfusepath{use as bounding box, clip}%
\begin{pgfscope}%
\pgfsetbuttcap%
\pgfsetmiterjoin%
\definecolor{currentfill}{rgb}{1.000000,1.000000,1.000000}%
\pgfsetfillcolor{currentfill}%
\pgfsetlinewidth{0.000000pt}%
\definecolor{currentstroke}{rgb}{1.000000,1.000000,1.000000}%
\pgfsetstrokecolor{currentstroke}%
\pgfsetdash{}{0pt}%
\pgfpathmoveto{\pgfqpoint{0.000000in}{0.000000in}}%
\pgfpathlineto{\pgfqpoint{5.000000in}{0.000000in}}%
\pgfpathlineto{\pgfqpoint{5.000000in}{3.083333in}}%
\pgfpathlineto{\pgfqpoint{0.000000in}{3.083333in}}%
\pgfpathclose%
\pgfusepath{fill}%
\end{pgfscope}%
\begin{pgfscope}%
\pgfsetbuttcap%
\pgfsetmiterjoin%
\definecolor{currentfill}{rgb}{1.000000,1.000000,1.000000}%
\pgfsetfillcolor{currentfill}%
\pgfsetlinewidth{0.000000pt}%
\definecolor{currentstroke}{rgb}{0.000000,0.000000,0.000000}%
\pgfsetstrokecolor{currentstroke}%
\pgfsetstrokeopacity{0.000000}%
\pgfsetdash{}{0pt}%
\pgfpathmoveto{\pgfqpoint{0.612942in}{0.402778in}}%
\pgfpathlineto{\pgfqpoint{4.951389in}{0.402778in}}%
\pgfpathlineto{\pgfqpoint{4.951389in}{3.034722in}}%
\pgfpathlineto{\pgfqpoint{0.612942in}{3.034722in}}%
\pgfpathclose%
\pgfusepath{fill}%
\end{pgfscope}%
\begin{pgfscope}%
\pgfsetbuttcap%
\pgfsetroundjoin%
\definecolor{currentfill}{rgb}{0.000000,0.000000,0.000000}%
\pgfsetfillcolor{currentfill}%
\pgfsetlinewidth{0.803000pt}%
\definecolor{currentstroke}{rgb}{0.000000,0.000000,0.000000}%
\pgfsetstrokecolor{currentstroke}%
\pgfsetdash{}{0pt}%
\pgfsys@defobject{currentmarker}{\pgfqpoint{0.000000in}{-0.048611in}}{\pgfqpoint{0.000000in}{0.000000in}}{%
\pgfpathmoveto{\pgfqpoint{0.000000in}{0.000000in}}%
\pgfpathlineto{\pgfqpoint{0.000000in}{-0.048611in}}%
\pgfusepath{stroke,fill}%
}%
\begin{pgfscope}%
\pgfsys@transformshift{1.204548in}{0.402778in}%
\pgfsys@useobject{currentmarker}{}%
\end{pgfscope}%
\end{pgfscope}%
\begin{pgfscope}%
\pgftext[x=1.204548in,y=0.305556in,,top]{\rmfamily\fontsize{10.000000}{12.000000}\selectfont \(\displaystyle 2\)}%
\end{pgfscope}%
\begin{pgfscope}%
\pgfsetbuttcap%
\pgfsetroundjoin%
\definecolor{currentfill}{rgb}{0.000000,0.000000,0.000000}%
\pgfsetfillcolor{currentfill}%
\pgfsetlinewidth{0.803000pt}%
\definecolor{currentstroke}{rgb}{0.000000,0.000000,0.000000}%
\pgfsetstrokecolor{currentstroke}%
\pgfsetdash{}{0pt}%
\pgfsys@defobject{currentmarker}{\pgfqpoint{0.000000in}{-0.048611in}}{\pgfqpoint{0.000000in}{0.000000in}}{%
\pgfpathmoveto{\pgfqpoint{0.000000in}{0.000000in}}%
\pgfpathlineto{\pgfqpoint{0.000000in}{-0.048611in}}%
\pgfusepath{stroke,fill}%
}%
\begin{pgfscope}%
\pgfsys@transformshift{1.993357in}{0.402778in}%
\pgfsys@useobject{currentmarker}{}%
\end{pgfscope}%
\end{pgfscope}%
\begin{pgfscope}%
\pgftext[x=1.993357in,y=0.305556in,,top]{\rmfamily\fontsize{10.000000}{12.000000}\selectfont \(\displaystyle 4\)}%
\end{pgfscope}%
\begin{pgfscope}%
\pgfsetbuttcap%
\pgfsetroundjoin%
\definecolor{currentfill}{rgb}{0.000000,0.000000,0.000000}%
\pgfsetfillcolor{currentfill}%
\pgfsetlinewidth{0.803000pt}%
\definecolor{currentstroke}{rgb}{0.000000,0.000000,0.000000}%
\pgfsetstrokecolor{currentstroke}%
\pgfsetdash{}{0pt}%
\pgfsys@defobject{currentmarker}{\pgfqpoint{0.000000in}{-0.048611in}}{\pgfqpoint{0.000000in}{0.000000in}}{%
\pgfpathmoveto{\pgfqpoint{0.000000in}{0.000000in}}%
\pgfpathlineto{\pgfqpoint{0.000000in}{-0.048611in}}%
\pgfusepath{stroke,fill}%
}%
\begin{pgfscope}%
\pgfsys@transformshift{2.782165in}{0.402778in}%
\pgfsys@useobject{currentmarker}{}%
\end{pgfscope}%
\end{pgfscope}%
\begin{pgfscope}%
\pgftext[x=2.782165in,y=0.305556in,,top]{\rmfamily\fontsize{10.000000}{12.000000}\selectfont \(\displaystyle 6\)}%
\end{pgfscope}%
\begin{pgfscope}%
\pgfsetbuttcap%
\pgfsetroundjoin%
\definecolor{currentfill}{rgb}{0.000000,0.000000,0.000000}%
\pgfsetfillcolor{currentfill}%
\pgfsetlinewidth{0.803000pt}%
\definecolor{currentstroke}{rgb}{0.000000,0.000000,0.000000}%
\pgfsetstrokecolor{currentstroke}%
\pgfsetdash{}{0pt}%
\pgfsys@defobject{currentmarker}{\pgfqpoint{0.000000in}{-0.048611in}}{\pgfqpoint{0.000000in}{0.000000in}}{%
\pgfpathmoveto{\pgfqpoint{0.000000in}{0.000000in}}%
\pgfpathlineto{\pgfqpoint{0.000000in}{-0.048611in}}%
\pgfusepath{stroke,fill}%
}%
\begin{pgfscope}%
\pgfsys@transformshift{3.570974in}{0.402778in}%
\pgfsys@useobject{currentmarker}{}%
\end{pgfscope}%
\end{pgfscope}%
\begin{pgfscope}%
\pgftext[x=3.570974in,y=0.305556in,,top]{\rmfamily\fontsize{10.000000}{12.000000}\selectfont \(\displaystyle 8\)}%
\end{pgfscope}%
\begin{pgfscope}%
\pgfsetbuttcap%
\pgfsetroundjoin%
\definecolor{currentfill}{rgb}{0.000000,0.000000,0.000000}%
\pgfsetfillcolor{currentfill}%
\pgfsetlinewidth{0.803000pt}%
\definecolor{currentstroke}{rgb}{0.000000,0.000000,0.000000}%
\pgfsetstrokecolor{currentstroke}%
\pgfsetdash{}{0pt}%
\pgfsys@defobject{currentmarker}{\pgfqpoint{0.000000in}{-0.048611in}}{\pgfqpoint{0.000000in}{0.000000in}}{%
\pgfpathmoveto{\pgfqpoint{0.000000in}{0.000000in}}%
\pgfpathlineto{\pgfqpoint{0.000000in}{-0.048611in}}%
\pgfusepath{stroke,fill}%
}%
\begin{pgfscope}%
\pgfsys@transformshift{4.359782in}{0.402778in}%
\pgfsys@useobject{currentmarker}{}%
\end{pgfscope}%
\end{pgfscope}%
\begin{pgfscope}%
\pgftext[x=4.359782in,y=0.305556in,,top]{\rmfamily\fontsize{10.000000}{12.000000}\selectfont \(\displaystyle 10\)}%
\end{pgfscope}%
\begin{pgfscope}%
\pgftext[x=2.782165in,y=0.123861in,,top]{\rmfamily\fontsize{10.000000}{12.000000}\selectfont $k$}%
\end{pgfscope}%
\begin{pgfscope}%
\pgfsetbuttcap%
\pgfsetroundjoin%
\definecolor{currentfill}{rgb}{0.000000,0.000000,0.000000}%
\pgfsetfillcolor{currentfill}%
\pgfsetlinewidth{0.803000pt}%
\definecolor{currentstroke}{rgb}{0.000000,0.000000,0.000000}%
\pgfsetstrokecolor{currentstroke}%
\pgfsetdash{}{0pt}%
\pgfsys@defobject{currentmarker}{\pgfqpoint{-0.048611in}{0.000000in}}{\pgfqpoint{0.000000in}{0.000000in}}{%
\pgfpathmoveto{\pgfqpoint{0.000000in}{0.000000in}}%
\pgfpathlineto{\pgfqpoint{-0.048611in}{0.000000in}}%
\pgfusepath{stroke,fill}%
}%
\begin{pgfscope}%
\pgfsys@transformshift{0.612942in}{1.075343in}%
\pgfsys@useobject{currentmarker}{}%
\end{pgfscope}%
\end{pgfscope}%
\begin{pgfscope}%
\pgftext[x=0.172354in,y=1.027125in,left,base]{\rmfamily\fontsize{10.000000}{12.000000}\selectfont \(\displaystyle 10^{-10}\)}%
\end{pgfscope}%
\begin{pgfscope}%
\pgfsetbuttcap%
\pgfsetroundjoin%
\definecolor{currentfill}{rgb}{0.000000,0.000000,0.000000}%
\pgfsetfillcolor{currentfill}%
\pgfsetlinewidth{0.803000pt}%
\definecolor{currentstroke}{rgb}{0.000000,0.000000,0.000000}%
\pgfsetstrokecolor{currentstroke}%
\pgfsetdash{}{0pt}%
\pgfsys@defobject{currentmarker}{\pgfqpoint{-0.048611in}{0.000000in}}{\pgfqpoint{0.000000in}{0.000000in}}{%
\pgfpathmoveto{\pgfqpoint{0.000000in}{0.000000in}}%
\pgfpathlineto{\pgfqpoint{-0.048611in}{0.000000in}}%
\pgfusepath{stroke,fill}%
}%
\begin{pgfscope}%
\pgfsys@transformshift{0.612942in}{2.013396in}%
\pgfsys@useobject{currentmarker}{}%
\end{pgfscope}%
\end{pgfscope}%
\begin{pgfscope}%
\pgftext[x=0.227717in,y=1.965178in,left,base]{\rmfamily\fontsize{10.000000}{12.000000}\selectfont \(\displaystyle 10^{-9}\)}%
\end{pgfscope}%
\begin{pgfscope}%
\pgfsetbuttcap%
\pgfsetroundjoin%
\definecolor{currentfill}{rgb}{0.000000,0.000000,0.000000}%
\pgfsetfillcolor{currentfill}%
\pgfsetlinewidth{0.803000pt}%
\definecolor{currentstroke}{rgb}{0.000000,0.000000,0.000000}%
\pgfsetstrokecolor{currentstroke}%
\pgfsetdash{}{0pt}%
\pgfsys@defobject{currentmarker}{\pgfqpoint{-0.048611in}{0.000000in}}{\pgfqpoint{0.000000in}{0.000000in}}{%
\pgfpathmoveto{\pgfqpoint{0.000000in}{0.000000in}}%
\pgfpathlineto{\pgfqpoint{-0.048611in}{0.000000in}}%
\pgfusepath{stroke,fill}%
}%
\begin{pgfscope}%
\pgfsys@transformshift{0.612942in}{2.951449in}%
\pgfsys@useobject{currentmarker}{}%
\end{pgfscope}%
\end{pgfscope}%
\begin{pgfscope}%
\pgftext[x=0.227717in,y=2.903231in,left,base]{\rmfamily\fontsize{10.000000}{12.000000}\selectfont \(\displaystyle 10^{-8}\)}%
\end{pgfscope}%
\begin{pgfscope}%
\pgfsetbuttcap%
\pgfsetroundjoin%
\definecolor{currentfill}{rgb}{0.000000,0.000000,0.000000}%
\pgfsetfillcolor{currentfill}%
\pgfsetlinewidth{0.602250pt}%
\definecolor{currentstroke}{rgb}{0.000000,0.000000,0.000000}%
\pgfsetstrokecolor{currentstroke}%
\pgfsetdash{}{0pt}%
\pgfsys@defobject{currentmarker}{\pgfqpoint{-0.027778in}{0.000000in}}{\pgfqpoint{0.000000in}{0.000000in}}{%
\pgfpathmoveto{\pgfqpoint{0.000000in}{0.000000in}}%
\pgfpathlineto{\pgfqpoint{-0.027778in}{0.000000in}}%
\pgfusepath{stroke,fill}%
}%
\begin{pgfscope}%
\pgfsys@transformshift{0.612942in}{0.419672in}%
\pgfsys@useobject{currentmarker}{}%
\end{pgfscope}%
\end{pgfscope}%
\begin{pgfscope}%
\pgfsetbuttcap%
\pgfsetroundjoin%
\definecolor{currentfill}{rgb}{0.000000,0.000000,0.000000}%
\pgfsetfillcolor{currentfill}%
\pgfsetlinewidth{0.602250pt}%
\definecolor{currentstroke}{rgb}{0.000000,0.000000,0.000000}%
\pgfsetstrokecolor{currentstroke}%
\pgfsetdash{}{0pt}%
\pgfsys@defobject{currentmarker}{\pgfqpoint{-0.027778in}{0.000000in}}{\pgfqpoint{0.000000in}{0.000000in}}{%
\pgfpathmoveto{\pgfqpoint{0.000000in}{0.000000in}}%
\pgfpathlineto{\pgfqpoint{-0.027778in}{0.000000in}}%
\pgfusepath{stroke,fill}%
}%
\begin{pgfscope}%
\pgfsys@transformshift{0.612942in}{0.584855in}%
\pgfsys@useobject{currentmarker}{}%
\end{pgfscope}%
\end{pgfscope}%
\begin{pgfscope}%
\pgfsetbuttcap%
\pgfsetroundjoin%
\definecolor{currentfill}{rgb}{0.000000,0.000000,0.000000}%
\pgfsetfillcolor{currentfill}%
\pgfsetlinewidth{0.602250pt}%
\definecolor{currentstroke}{rgb}{0.000000,0.000000,0.000000}%
\pgfsetstrokecolor{currentstroke}%
\pgfsetdash{}{0pt}%
\pgfsys@defobject{currentmarker}{\pgfqpoint{-0.027778in}{0.000000in}}{\pgfqpoint{0.000000in}{0.000000in}}{%
\pgfpathmoveto{\pgfqpoint{0.000000in}{0.000000in}}%
\pgfpathlineto{\pgfqpoint{-0.027778in}{0.000000in}}%
\pgfusepath{stroke,fill}%
}%
\begin{pgfscope}%
\pgfsys@transformshift{0.612942in}{0.702054in}%
\pgfsys@useobject{currentmarker}{}%
\end{pgfscope}%
\end{pgfscope}%
\begin{pgfscope}%
\pgfsetbuttcap%
\pgfsetroundjoin%
\definecolor{currentfill}{rgb}{0.000000,0.000000,0.000000}%
\pgfsetfillcolor{currentfill}%
\pgfsetlinewidth{0.602250pt}%
\definecolor{currentstroke}{rgb}{0.000000,0.000000,0.000000}%
\pgfsetstrokecolor{currentstroke}%
\pgfsetdash{}{0pt}%
\pgfsys@defobject{currentmarker}{\pgfqpoint{-0.027778in}{0.000000in}}{\pgfqpoint{0.000000in}{0.000000in}}{%
\pgfpathmoveto{\pgfqpoint{0.000000in}{0.000000in}}%
\pgfpathlineto{\pgfqpoint{-0.027778in}{0.000000in}}%
\pgfusepath{stroke,fill}%
}%
\begin{pgfscope}%
\pgfsys@transformshift{0.612942in}{0.792961in}%
\pgfsys@useobject{currentmarker}{}%
\end{pgfscope}%
\end{pgfscope}%
\begin{pgfscope}%
\pgfsetbuttcap%
\pgfsetroundjoin%
\definecolor{currentfill}{rgb}{0.000000,0.000000,0.000000}%
\pgfsetfillcolor{currentfill}%
\pgfsetlinewidth{0.602250pt}%
\definecolor{currentstroke}{rgb}{0.000000,0.000000,0.000000}%
\pgfsetstrokecolor{currentstroke}%
\pgfsetdash{}{0pt}%
\pgfsys@defobject{currentmarker}{\pgfqpoint{-0.027778in}{0.000000in}}{\pgfqpoint{0.000000in}{0.000000in}}{%
\pgfpathmoveto{\pgfqpoint{0.000000in}{0.000000in}}%
\pgfpathlineto{\pgfqpoint{-0.027778in}{0.000000in}}%
\pgfusepath{stroke,fill}%
}%
\begin{pgfscope}%
\pgfsys@transformshift{0.612942in}{0.867237in}%
\pgfsys@useobject{currentmarker}{}%
\end{pgfscope}%
\end{pgfscope}%
\begin{pgfscope}%
\pgfsetbuttcap%
\pgfsetroundjoin%
\definecolor{currentfill}{rgb}{0.000000,0.000000,0.000000}%
\pgfsetfillcolor{currentfill}%
\pgfsetlinewidth{0.602250pt}%
\definecolor{currentstroke}{rgb}{0.000000,0.000000,0.000000}%
\pgfsetstrokecolor{currentstroke}%
\pgfsetdash{}{0pt}%
\pgfsys@defobject{currentmarker}{\pgfqpoint{-0.027778in}{0.000000in}}{\pgfqpoint{0.000000in}{0.000000in}}{%
\pgfpathmoveto{\pgfqpoint{0.000000in}{0.000000in}}%
\pgfpathlineto{\pgfqpoint{-0.027778in}{0.000000in}}%
\pgfusepath{stroke,fill}%
}%
\begin{pgfscope}%
\pgfsys@transformshift{0.612942in}{0.930036in}%
\pgfsys@useobject{currentmarker}{}%
\end{pgfscope}%
\end{pgfscope}%
\begin{pgfscope}%
\pgfsetbuttcap%
\pgfsetroundjoin%
\definecolor{currentfill}{rgb}{0.000000,0.000000,0.000000}%
\pgfsetfillcolor{currentfill}%
\pgfsetlinewidth{0.602250pt}%
\definecolor{currentstroke}{rgb}{0.000000,0.000000,0.000000}%
\pgfsetstrokecolor{currentstroke}%
\pgfsetdash{}{0pt}%
\pgfsys@defobject{currentmarker}{\pgfqpoint{-0.027778in}{0.000000in}}{\pgfqpoint{0.000000in}{0.000000in}}{%
\pgfpathmoveto{\pgfqpoint{0.000000in}{0.000000in}}%
\pgfpathlineto{\pgfqpoint{-0.027778in}{0.000000in}}%
\pgfusepath{stroke,fill}%
}%
\begin{pgfscope}%
\pgfsys@transformshift{0.612942in}{0.984436in}%
\pgfsys@useobject{currentmarker}{}%
\end{pgfscope}%
\end{pgfscope}%
\begin{pgfscope}%
\pgfsetbuttcap%
\pgfsetroundjoin%
\definecolor{currentfill}{rgb}{0.000000,0.000000,0.000000}%
\pgfsetfillcolor{currentfill}%
\pgfsetlinewidth{0.602250pt}%
\definecolor{currentstroke}{rgb}{0.000000,0.000000,0.000000}%
\pgfsetstrokecolor{currentstroke}%
\pgfsetdash{}{0pt}%
\pgfsys@defobject{currentmarker}{\pgfqpoint{-0.027778in}{0.000000in}}{\pgfqpoint{0.000000in}{0.000000in}}{%
\pgfpathmoveto{\pgfqpoint{0.000000in}{0.000000in}}%
\pgfpathlineto{\pgfqpoint{-0.027778in}{0.000000in}}%
\pgfusepath{stroke,fill}%
}%
\begin{pgfscope}%
\pgfsys@transformshift{0.612942in}{1.032420in}%
\pgfsys@useobject{currentmarker}{}%
\end{pgfscope}%
\end{pgfscope}%
\begin{pgfscope}%
\pgfsetbuttcap%
\pgfsetroundjoin%
\definecolor{currentfill}{rgb}{0.000000,0.000000,0.000000}%
\pgfsetfillcolor{currentfill}%
\pgfsetlinewidth{0.602250pt}%
\definecolor{currentstroke}{rgb}{0.000000,0.000000,0.000000}%
\pgfsetstrokecolor{currentstroke}%
\pgfsetdash{}{0pt}%
\pgfsys@defobject{currentmarker}{\pgfqpoint{-0.027778in}{0.000000in}}{\pgfqpoint{0.000000in}{0.000000in}}{%
\pgfpathmoveto{\pgfqpoint{0.000000in}{0.000000in}}%
\pgfpathlineto{\pgfqpoint{-0.027778in}{0.000000in}}%
\pgfusepath{stroke,fill}%
}%
\begin{pgfscope}%
\pgfsys@transformshift{0.612942in}{1.357725in}%
\pgfsys@useobject{currentmarker}{}%
\end{pgfscope}%
\end{pgfscope}%
\begin{pgfscope}%
\pgfsetbuttcap%
\pgfsetroundjoin%
\definecolor{currentfill}{rgb}{0.000000,0.000000,0.000000}%
\pgfsetfillcolor{currentfill}%
\pgfsetlinewidth{0.602250pt}%
\definecolor{currentstroke}{rgb}{0.000000,0.000000,0.000000}%
\pgfsetstrokecolor{currentstroke}%
\pgfsetdash{}{0pt}%
\pgfsys@defobject{currentmarker}{\pgfqpoint{-0.027778in}{0.000000in}}{\pgfqpoint{0.000000in}{0.000000in}}{%
\pgfpathmoveto{\pgfqpoint{0.000000in}{0.000000in}}%
\pgfpathlineto{\pgfqpoint{-0.027778in}{0.000000in}}%
\pgfusepath{stroke,fill}%
}%
\begin{pgfscope}%
\pgfsys@transformshift{0.612942in}{1.522908in}%
\pgfsys@useobject{currentmarker}{}%
\end{pgfscope}%
\end{pgfscope}%
\begin{pgfscope}%
\pgfsetbuttcap%
\pgfsetroundjoin%
\definecolor{currentfill}{rgb}{0.000000,0.000000,0.000000}%
\pgfsetfillcolor{currentfill}%
\pgfsetlinewidth{0.602250pt}%
\definecolor{currentstroke}{rgb}{0.000000,0.000000,0.000000}%
\pgfsetstrokecolor{currentstroke}%
\pgfsetdash{}{0pt}%
\pgfsys@defobject{currentmarker}{\pgfqpoint{-0.027778in}{0.000000in}}{\pgfqpoint{0.000000in}{0.000000in}}{%
\pgfpathmoveto{\pgfqpoint{0.000000in}{0.000000in}}%
\pgfpathlineto{\pgfqpoint{-0.027778in}{0.000000in}}%
\pgfusepath{stroke,fill}%
}%
\begin{pgfscope}%
\pgfsys@transformshift{0.612942in}{1.640107in}%
\pgfsys@useobject{currentmarker}{}%
\end{pgfscope}%
\end{pgfscope}%
\begin{pgfscope}%
\pgfsetbuttcap%
\pgfsetroundjoin%
\definecolor{currentfill}{rgb}{0.000000,0.000000,0.000000}%
\pgfsetfillcolor{currentfill}%
\pgfsetlinewidth{0.602250pt}%
\definecolor{currentstroke}{rgb}{0.000000,0.000000,0.000000}%
\pgfsetstrokecolor{currentstroke}%
\pgfsetdash{}{0pt}%
\pgfsys@defobject{currentmarker}{\pgfqpoint{-0.027778in}{0.000000in}}{\pgfqpoint{0.000000in}{0.000000in}}{%
\pgfpathmoveto{\pgfqpoint{0.000000in}{0.000000in}}%
\pgfpathlineto{\pgfqpoint{-0.027778in}{0.000000in}}%
\pgfusepath{stroke,fill}%
}%
\begin{pgfscope}%
\pgfsys@transformshift{0.612942in}{1.731014in}%
\pgfsys@useobject{currentmarker}{}%
\end{pgfscope}%
\end{pgfscope}%
\begin{pgfscope}%
\pgfsetbuttcap%
\pgfsetroundjoin%
\definecolor{currentfill}{rgb}{0.000000,0.000000,0.000000}%
\pgfsetfillcolor{currentfill}%
\pgfsetlinewidth{0.602250pt}%
\definecolor{currentstroke}{rgb}{0.000000,0.000000,0.000000}%
\pgfsetstrokecolor{currentstroke}%
\pgfsetdash{}{0pt}%
\pgfsys@defobject{currentmarker}{\pgfqpoint{-0.027778in}{0.000000in}}{\pgfqpoint{0.000000in}{0.000000in}}{%
\pgfpathmoveto{\pgfqpoint{0.000000in}{0.000000in}}%
\pgfpathlineto{\pgfqpoint{-0.027778in}{0.000000in}}%
\pgfusepath{stroke,fill}%
}%
\begin{pgfscope}%
\pgfsys@transformshift{0.612942in}{1.805290in}%
\pgfsys@useobject{currentmarker}{}%
\end{pgfscope}%
\end{pgfscope}%
\begin{pgfscope}%
\pgfsetbuttcap%
\pgfsetroundjoin%
\definecolor{currentfill}{rgb}{0.000000,0.000000,0.000000}%
\pgfsetfillcolor{currentfill}%
\pgfsetlinewidth{0.602250pt}%
\definecolor{currentstroke}{rgb}{0.000000,0.000000,0.000000}%
\pgfsetstrokecolor{currentstroke}%
\pgfsetdash{}{0pt}%
\pgfsys@defobject{currentmarker}{\pgfqpoint{-0.027778in}{0.000000in}}{\pgfqpoint{0.000000in}{0.000000in}}{%
\pgfpathmoveto{\pgfqpoint{0.000000in}{0.000000in}}%
\pgfpathlineto{\pgfqpoint{-0.027778in}{0.000000in}}%
\pgfusepath{stroke,fill}%
}%
\begin{pgfscope}%
\pgfsys@transformshift{0.612942in}{1.868090in}%
\pgfsys@useobject{currentmarker}{}%
\end{pgfscope}%
\end{pgfscope}%
\begin{pgfscope}%
\pgfsetbuttcap%
\pgfsetroundjoin%
\definecolor{currentfill}{rgb}{0.000000,0.000000,0.000000}%
\pgfsetfillcolor{currentfill}%
\pgfsetlinewidth{0.602250pt}%
\definecolor{currentstroke}{rgb}{0.000000,0.000000,0.000000}%
\pgfsetstrokecolor{currentstroke}%
\pgfsetdash{}{0pt}%
\pgfsys@defobject{currentmarker}{\pgfqpoint{-0.027778in}{0.000000in}}{\pgfqpoint{0.000000in}{0.000000in}}{%
\pgfpathmoveto{\pgfqpoint{0.000000in}{0.000000in}}%
\pgfpathlineto{\pgfqpoint{-0.027778in}{0.000000in}}%
\pgfusepath{stroke,fill}%
}%
\begin{pgfscope}%
\pgfsys@transformshift{0.612942in}{1.922489in}%
\pgfsys@useobject{currentmarker}{}%
\end{pgfscope}%
\end{pgfscope}%
\begin{pgfscope}%
\pgfsetbuttcap%
\pgfsetroundjoin%
\definecolor{currentfill}{rgb}{0.000000,0.000000,0.000000}%
\pgfsetfillcolor{currentfill}%
\pgfsetlinewidth{0.602250pt}%
\definecolor{currentstroke}{rgb}{0.000000,0.000000,0.000000}%
\pgfsetstrokecolor{currentstroke}%
\pgfsetdash{}{0pt}%
\pgfsys@defobject{currentmarker}{\pgfqpoint{-0.027778in}{0.000000in}}{\pgfqpoint{0.000000in}{0.000000in}}{%
\pgfpathmoveto{\pgfqpoint{0.000000in}{0.000000in}}%
\pgfpathlineto{\pgfqpoint{-0.027778in}{0.000000in}}%
\pgfusepath{stroke,fill}%
}%
\begin{pgfscope}%
\pgfsys@transformshift{0.612942in}{1.970473in}%
\pgfsys@useobject{currentmarker}{}%
\end{pgfscope}%
\end{pgfscope}%
\begin{pgfscope}%
\pgfsetbuttcap%
\pgfsetroundjoin%
\definecolor{currentfill}{rgb}{0.000000,0.000000,0.000000}%
\pgfsetfillcolor{currentfill}%
\pgfsetlinewidth{0.602250pt}%
\definecolor{currentstroke}{rgb}{0.000000,0.000000,0.000000}%
\pgfsetstrokecolor{currentstroke}%
\pgfsetdash{}{0pt}%
\pgfsys@defobject{currentmarker}{\pgfqpoint{-0.027778in}{0.000000in}}{\pgfqpoint{0.000000in}{0.000000in}}{%
\pgfpathmoveto{\pgfqpoint{0.000000in}{0.000000in}}%
\pgfpathlineto{\pgfqpoint{-0.027778in}{0.000000in}}%
\pgfusepath{stroke,fill}%
}%
\begin{pgfscope}%
\pgfsys@transformshift{0.612942in}{2.295778in}%
\pgfsys@useobject{currentmarker}{}%
\end{pgfscope}%
\end{pgfscope}%
\begin{pgfscope}%
\pgfsetbuttcap%
\pgfsetroundjoin%
\definecolor{currentfill}{rgb}{0.000000,0.000000,0.000000}%
\pgfsetfillcolor{currentfill}%
\pgfsetlinewidth{0.602250pt}%
\definecolor{currentstroke}{rgb}{0.000000,0.000000,0.000000}%
\pgfsetstrokecolor{currentstroke}%
\pgfsetdash{}{0pt}%
\pgfsys@defobject{currentmarker}{\pgfqpoint{-0.027778in}{0.000000in}}{\pgfqpoint{0.000000in}{0.000000in}}{%
\pgfpathmoveto{\pgfqpoint{0.000000in}{0.000000in}}%
\pgfpathlineto{\pgfqpoint{-0.027778in}{0.000000in}}%
\pgfusepath{stroke,fill}%
}%
\begin{pgfscope}%
\pgfsys@transformshift{0.612942in}{2.460961in}%
\pgfsys@useobject{currentmarker}{}%
\end{pgfscope}%
\end{pgfscope}%
\begin{pgfscope}%
\pgfsetbuttcap%
\pgfsetroundjoin%
\definecolor{currentfill}{rgb}{0.000000,0.000000,0.000000}%
\pgfsetfillcolor{currentfill}%
\pgfsetlinewidth{0.602250pt}%
\definecolor{currentstroke}{rgb}{0.000000,0.000000,0.000000}%
\pgfsetstrokecolor{currentstroke}%
\pgfsetdash{}{0pt}%
\pgfsys@defobject{currentmarker}{\pgfqpoint{-0.027778in}{0.000000in}}{\pgfqpoint{0.000000in}{0.000000in}}{%
\pgfpathmoveto{\pgfqpoint{0.000000in}{0.000000in}}%
\pgfpathlineto{\pgfqpoint{-0.027778in}{0.000000in}}%
\pgfusepath{stroke,fill}%
}%
\begin{pgfscope}%
\pgfsys@transformshift{0.612942in}{2.578160in}%
\pgfsys@useobject{currentmarker}{}%
\end{pgfscope}%
\end{pgfscope}%
\begin{pgfscope}%
\pgfsetbuttcap%
\pgfsetroundjoin%
\definecolor{currentfill}{rgb}{0.000000,0.000000,0.000000}%
\pgfsetfillcolor{currentfill}%
\pgfsetlinewidth{0.602250pt}%
\definecolor{currentstroke}{rgb}{0.000000,0.000000,0.000000}%
\pgfsetstrokecolor{currentstroke}%
\pgfsetdash{}{0pt}%
\pgfsys@defobject{currentmarker}{\pgfqpoint{-0.027778in}{0.000000in}}{\pgfqpoint{0.000000in}{0.000000in}}{%
\pgfpathmoveto{\pgfqpoint{0.000000in}{0.000000in}}%
\pgfpathlineto{\pgfqpoint{-0.027778in}{0.000000in}}%
\pgfusepath{stroke,fill}%
}%
\begin{pgfscope}%
\pgfsys@transformshift{0.612942in}{2.669067in}%
\pgfsys@useobject{currentmarker}{}%
\end{pgfscope}%
\end{pgfscope}%
\begin{pgfscope}%
\pgfsetbuttcap%
\pgfsetroundjoin%
\definecolor{currentfill}{rgb}{0.000000,0.000000,0.000000}%
\pgfsetfillcolor{currentfill}%
\pgfsetlinewidth{0.602250pt}%
\definecolor{currentstroke}{rgb}{0.000000,0.000000,0.000000}%
\pgfsetstrokecolor{currentstroke}%
\pgfsetdash{}{0pt}%
\pgfsys@defobject{currentmarker}{\pgfqpoint{-0.027778in}{0.000000in}}{\pgfqpoint{0.000000in}{0.000000in}}{%
\pgfpathmoveto{\pgfqpoint{0.000000in}{0.000000in}}%
\pgfpathlineto{\pgfqpoint{-0.027778in}{0.000000in}}%
\pgfusepath{stroke,fill}%
}%
\begin{pgfscope}%
\pgfsys@transformshift{0.612942in}{2.743343in}%
\pgfsys@useobject{currentmarker}{}%
\end{pgfscope}%
\end{pgfscope}%
\begin{pgfscope}%
\pgfsetbuttcap%
\pgfsetroundjoin%
\definecolor{currentfill}{rgb}{0.000000,0.000000,0.000000}%
\pgfsetfillcolor{currentfill}%
\pgfsetlinewidth{0.602250pt}%
\definecolor{currentstroke}{rgb}{0.000000,0.000000,0.000000}%
\pgfsetstrokecolor{currentstroke}%
\pgfsetdash{}{0pt}%
\pgfsys@defobject{currentmarker}{\pgfqpoint{-0.027778in}{0.000000in}}{\pgfqpoint{0.000000in}{0.000000in}}{%
\pgfpathmoveto{\pgfqpoint{0.000000in}{0.000000in}}%
\pgfpathlineto{\pgfqpoint{-0.027778in}{0.000000in}}%
\pgfusepath{stroke,fill}%
}%
\begin{pgfscope}%
\pgfsys@transformshift{0.612942in}{2.806143in}%
\pgfsys@useobject{currentmarker}{}%
\end{pgfscope}%
\end{pgfscope}%
\begin{pgfscope}%
\pgfsetbuttcap%
\pgfsetroundjoin%
\definecolor{currentfill}{rgb}{0.000000,0.000000,0.000000}%
\pgfsetfillcolor{currentfill}%
\pgfsetlinewidth{0.602250pt}%
\definecolor{currentstroke}{rgb}{0.000000,0.000000,0.000000}%
\pgfsetstrokecolor{currentstroke}%
\pgfsetdash{}{0pt}%
\pgfsys@defobject{currentmarker}{\pgfqpoint{-0.027778in}{0.000000in}}{\pgfqpoint{0.000000in}{0.000000in}}{%
\pgfpathmoveto{\pgfqpoint{0.000000in}{0.000000in}}%
\pgfpathlineto{\pgfqpoint{-0.027778in}{0.000000in}}%
\pgfusepath{stroke,fill}%
}%
\begin{pgfscope}%
\pgfsys@transformshift{0.612942in}{2.860542in}%
\pgfsys@useobject{currentmarker}{}%
\end{pgfscope}%
\end{pgfscope}%
\begin{pgfscope}%
\pgfsetbuttcap%
\pgfsetroundjoin%
\definecolor{currentfill}{rgb}{0.000000,0.000000,0.000000}%
\pgfsetfillcolor{currentfill}%
\pgfsetlinewidth{0.602250pt}%
\definecolor{currentstroke}{rgb}{0.000000,0.000000,0.000000}%
\pgfsetstrokecolor{currentstroke}%
\pgfsetdash{}{0pt}%
\pgfsys@defobject{currentmarker}{\pgfqpoint{-0.027778in}{0.000000in}}{\pgfqpoint{0.000000in}{0.000000in}}{%
\pgfpathmoveto{\pgfqpoint{0.000000in}{0.000000in}}%
\pgfpathlineto{\pgfqpoint{-0.027778in}{0.000000in}}%
\pgfusepath{stroke,fill}%
}%
\begin{pgfscope}%
\pgfsys@transformshift{0.612942in}{2.908526in}%
\pgfsys@useobject{currentmarker}{}%
\end{pgfscope}%
\end{pgfscope}%
\begin{pgfscope}%
\pgftext[x=0.136799in,y=1.718750in,,bottom,rotate=90.000000]{\rmfamily\fontsize{10.000000}{12.000000}\selectfont $E(k)$}%
\end{pgfscope}%
\begin{pgfscope}%
\pgfpathrectangle{\pgfqpoint{0.612942in}{0.402778in}}{\pgfqpoint{4.338447in}{2.631944in}}%
\pgfusepath{clip}%
\pgfsetrectcap%
\pgfsetroundjoin%
\pgfsetlinewidth{1.505625pt}%
\definecolor{currentstroke}{rgb}{0.121569,0.466667,0.705882}%
\pgfsetstrokecolor{currentstroke}%
\pgfsetdash{}{0pt}%
\pgfpathmoveto{\pgfqpoint{0.810144in}{2.915088in}}%
\pgfpathlineto{\pgfqpoint{1.204548in}{1.908387in}}%
\pgfpathlineto{\pgfqpoint{1.598953in}{1.011476in}}%
\pgfpathlineto{\pgfqpoint{1.993357in}{0.559574in}}%
\pgfpathlineto{\pgfqpoint{2.387761in}{0.598053in}}%
\pgfpathlineto{\pgfqpoint{2.782165in}{0.522412in}}%
\pgfpathlineto{\pgfqpoint{3.176570in}{0.529359in}}%
\pgfpathlineto{\pgfqpoint{3.570974in}{0.524875in}}%
\pgfpathlineto{\pgfqpoint{3.965378in}{0.525447in}}%
\pgfpathlineto{\pgfqpoint{4.359782in}{0.524003in}}%
\pgfpathlineto{\pgfqpoint{4.754187in}{0.523666in}}%
\pgfusepath{stroke}%
\end{pgfscope}%
\begin{pgfscope}%
\pgfsetrectcap%
\pgfsetmiterjoin%
\pgfsetlinewidth{0.803000pt}%
\definecolor{currentstroke}{rgb}{0.000000,0.000000,0.000000}%
\pgfsetstrokecolor{currentstroke}%
\pgfsetdash{}{0pt}%
\pgfpathmoveto{\pgfqpoint{0.612942in}{0.402778in}}%
\pgfpathlineto{\pgfqpoint{0.612942in}{3.034722in}}%
\pgfusepath{stroke}%
\end{pgfscope}%
\begin{pgfscope}%
\pgfsetrectcap%
\pgfsetmiterjoin%
\pgfsetlinewidth{0.803000pt}%
\definecolor{currentstroke}{rgb}{0.000000,0.000000,0.000000}%
\pgfsetstrokecolor{currentstroke}%
\pgfsetdash{}{0pt}%
\pgfpathmoveto{\pgfqpoint{4.951389in}{0.402778in}}%
\pgfpathlineto{\pgfqpoint{4.951389in}{3.034722in}}%
\pgfusepath{stroke}%
\end{pgfscope}%
\begin{pgfscope}%
\pgfsetrectcap%
\pgfsetmiterjoin%
\pgfsetlinewidth{0.803000pt}%
\definecolor{currentstroke}{rgb}{0.000000,0.000000,0.000000}%
\pgfsetstrokecolor{currentstroke}%
\pgfsetdash{}{0pt}%
\pgfpathmoveto{\pgfqpoint{0.612942in}{0.402778in}}%
\pgfpathlineto{\pgfqpoint{4.951389in}{0.402778in}}%
\pgfusepath{stroke}%
\end{pgfscope}%
\begin{pgfscope}%
\pgfsetrectcap%
\pgfsetmiterjoin%
\pgfsetlinewidth{0.803000pt}%
\definecolor{currentstroke}{rgb}{0.000000,0.000000,0.000000}%
\pgfsetstrokecolor{currentstroke}%
\pgfsetdash{}{0pt}%
\pgfpathmoveto{\pgfqpoint{0.612942in}{3.034722in}}%
\pgfpathlineto{\pgfqpoint{4.951389in}{3.034722in}}%
\pgfusepath{stroke}%
\end{pgfscope}%
\end{pgfpicture}%
\makeatother%
\endgroup%

        %% Creator: Matplotlib, PGF backend
%%
%% To include the figure in your LaTeX document, write
%%   \input{<filename>.pgf}
%%
%% Make sure the required packages are loaded in your preamble
%%   \usepackage{pgf}
%%
%% Figures using additional raster images can only be included by \input if
%% they are in the same directory as the main LaTeX file. For loading figures
%% from other directories you can use the `import` package
%%   \usepackage{import}
%% and then include the figures with
%%   \import{<path to file>}{<filename>.pgf}
%%
%% Matplotlib used the following preamble
%%   \usepackage{fontspec}
%%   \setmainfont{Times New Roman}
%%   \setsansfont{Lucida Grande}
%%   \setmonofont{Andale Mono}
%%
\begingroup%
\makeatletter%
\begin{pgfpicture}%
\pgfpathrectangle{\pgfpointorigin}{\pgfqpoint{5.000000in}{3.083333in}}%
\pgfusepath{use as bounding box, clip}%
\begin{pgfscope}%
\pgfsetbuttcap%
\pgfsetmiterjoin%
\definecolor{currentfill}{rgb}{1.000000,1.000000,1.000000}%
\pgfsetfillcolor{currentfill}%
\pgfsetlinewidth{0.000000pt}%
\definecolor{currentstroke}{rgb}{1.000000,1.000000,1.000000}%
\pgfsetstrokecolor{currentstroke}%
\pgfsetdash{}{0pt}%
\pgfpathmoveto{\pgfqpoint{0.000000in}{0.000000in}}%
\pgfpathlineto{\pgfqpoint{5.000000in}{0.000000in}}%
\pgfpathlineto{\pgfqpoint{5.000000in}{3.083333in}}%
\pgfpathlineto{\pgfqpoint{0.000000in}{3.083333in}}%
\pgfpathclose%
\pgfusepath{fill}%
\end{pgfscope}%
\begin{pgfscope}%
\pgfsetbuttcap%
\pgfsetmiterjoin%
\definecolor{currentfill}{rgb}{1.000000,1.000000,1.000000}%
\pgfsetfillcolor{currentfill}%
\pgfsetlinewidth{0.000000pt}%
\definecolor{currentstroke}{rgb}{0.000000,0.000000,0.000000}%
\pgfsetstrokecolor{currentstroke}%
\pgfsetstrokeopacity{0.000000}%
\pgfsetdash{}{0pt}%
\pgfpathmoveto{\pgfqpoint{0.612942in}{0.402778in}}%
\pgfpathlineto{\pgfqpoint{4.951389in}{0.402778in}}%
\pgfpathlineto{\pgfqpoint{4.951389in}{3.034722in}}%
\pgfpathlineto{\pgfqpoint{0.612942in}{3.034722in}}%
\pgfpathclose%
\pgfusepath{fill}%
\end{pgfscope}%
\begin{pgfscope}%
\pgfsetbuttcap%
\pgfsetroundjoin%
\definecolor{currentfill}{rgb}{0.000000,0.000000,0.000000}%
\pgfsetfillcolor{currentfill}%
\pgfsetlinewidth{0.803000pt}%
\definecolor{currentstroke}{rgb}{0.000000,0.000000,0.000000}%
\pgfsetstrokecolor{currentstroke}%
\pgfsetdash{}{0pt}%
\pgfsys@defobject{currentmarker}{\pgfqpoint{0.000000in}{-0.048611in}}{\pgfqpoint{0.000000in}{0.000000in}}{%
\pgfpathmoveto{\pgfqpoint{0.000000in}{0.000000in}}%
\pgfpathlineto{\pgfqpoint{0.000000in}{-0.048611in}}%
\pgfusepath{stroke,fill}%
}%
\begin{pgfscope}%
\pgfsys@transformshift{0.810144in}{0.402778in}%
\pgfsys@useobject{currentmarker}{}%
\end{pgfscope}%
\end{pgfscope}%
\begin{pgfscope}%
\pgftext[x=0.810144in,y=0.305556in,,top]{\rmfamily\fontsize{10.000000}{12.000000}\selectfont \(\displaystyle 1\)}%
\end{pgfscope}%
\begin{pgfscope}%
\pgfsetbuttcap%
\pgfsetroundjoin%
\definecolor{currentfill}{rgb}{0.000000,0.000000,0.000000}%
\pgfsetfillcolor{currentfill}%
\pgfsetlinewidth{0.803000pt}%
\definecolor{currentstroke}{rgb}{0.000000,0.000000,0.000000}%
\pgfsetstrokecolor{currentstroke}%
\pgfsetdash{}{0pt}%
\pgfsys@defobject{currentmarker}{\pgfqpoint{0.000000in}{-0.048611in}}{\pgfqpoint{0.000000in}{0.000000in}}{%
\pgfpathmoveto{\pgfqpoint{0.000000in}{0.000000in}}%
\pgfpathlineto{\pgfqpoint{0.000000in}{-0.048611in}}%
\pgfusepath{stroke,fill}%
}%
\begin{pgfscope}%
\pgfsys@transformshift{1.303149in}{0.402778in}%
\pgfsys@useobject{currentmarker}{}%
\end{pgfscope}%
\end{pgfscope}%
\begin{pgfscope}%
\pgftext[x=1.303149in,y=0.305556in,,top]{\rmfamily\fontsize{10.000000}{12.000000}\selectfont \(\displaystyle 2\)}%
\end{pgfscope}%
\begin{pgfscope}%
\pgfsetbuttcap%
\pgfsetroundjoin%
\definecolor{currentfill}{rgb}{0.000000,0.000000,0.000000}%
\pgfsetfillcolor{currentfill}%
\pgfsetlinewidth{0.803000pt}%
\definecolor{currentstroke}{rgb}{0.000000,0.000000,0.000000}%
\pgfsetstrokecolor{currentstroke}%
\pgfsetdash{}{0pt}%
\pgfsys@defobject{currentmarker}{\pgfqpoint{0.000000in}{-0.048611in}}{\pgfqpoint{0.000000in}{0.000000in}}{%
\pgfpathmoveto{\pgfqpoint{0.000000in}{0.000000in}}%
\pgfpathlineto{\pgfqpoint{0.000000in}{-0.048611in}}%
\pgfusepath{stroke,fill}%
}%
\begin{pgfscope}%
\pgfsys@transformshift{1.796155in}{0.402778in}%
\pgfsys@useobject{currentmarker}{}%
\end{pgfscope}%
\end{pgfscope}%
\begin{pgfscope}%
\pgftext[x=1.796155in,y=0.305556in,,top]{\rmfamily\fontsize{10.000000}{12.000000}\selectfont \(\displaystyle 3\)}%
\end{pgfscope}%
\begin{pgfscope}%
\pgfsetbuttcap%
\pgfsetroundjoin%
\definecolor{currentfill}{rgb}{0.000000,0.000000,0.000000}%
\pgfsetfillcolor{currentfill}%
\pgfsetlinewidth{0.803000pt}%
\definecolor{currentstroke}{rgb}{0.000000,0.000000,0.000000}%
\pgfsetstrokecolor{currentstroke}%
\pgfsetdash{}{0pt}%
\pgfsys@defobject{currentmarker}{\pgfqpoint{0.000000in}{-0.048611in}}{\pgfqpoint{0.000000in}{0.000000in}}{%
\pgfpathmoveto{\pgfqpoint{0.000000in}{0.000000in}}%
\pgfpathlineto{\pgfqpoint{0.000000in}{-0.048611in}}%
\pgfusepath{stroke,fill}%
}%
\begin{pgfscope}%
\pgfsys@transformshift{2.289160in}{0.402778in}%
\pgfsys@useobject{currentmarker}{}%
\end{pgfscope}%
\end{pgfscope}%
\begin{pgfscope}%
\pgftext[x=2.289160in,y=0.305556in,,top]{\rmfamily\fontsize{10.000000}{12.000000}\selectfont \(\displaystyle 4\)}%
\end{pgfscope}%
\begin{pgfscope}%
\pgfsetbuttcap%
\pgfsetroundjoin%
\definecolor{currentfill}{rgb}{0.000000,0.000000,0.000000}%
\pgfsetfillcolor{currentfill}%
\pgfsetlinewidth{0.803000pt}%
\definecolor{currentstroke}{rgb}{0.000000,0.000000,0.000000}%
\pgfsetstrokecolor{currentstroke}%
\pgfsetdash{}{0pt}%
\pgfsys@defobject{currentmarker}{\pgfqpoint{0.000000in}{-0.048611in}}{\pgfqpoint{0.000000in}{0.000000in}}{%
\pgfpathmoveto{\pgfqpoint{0.000000in}{0.000000in}}%
\pgfpathlineto{\pgfqpoint{0.000000in}{-0.048611in}}%
\pgfusepath{stroke,fill}%
}%
\begin{pgfscope}%
\pgfsys@transformshift{2.782165in}{0.402778in}%
\pgfsys@useobject{currentmarker}{}%
\end{pgfscope}%
\end{pgfscope}%
\begin{pgfscope}%
\pgftext[x=2.782165in,y=0.305556in,,top]{\rmfamily\fontsize{10.000000}{12.000000}\selectfont \(\displaystyle 5\)}%
\end{pgfscope}%
\begin{pgfscope}%
\pgfsetbuttcap%
\pgfsetroundjoin%
\definecolor{currentfill}{rgb}{0.000000,0.000000,0.000000}%
\pgfsetfillcolor{currentfill}%
\pgfsetlinewidth{0.803000pt}%
\definecolor{currentstroke}{rgb}{0.000000,0.000000,0.000000}%
\pgfsetstrokecolor{currentstroke}%
\pgfsetdash{}{0pt}%
\pgfsys@defobject{currentmarker}{\pgfqpoint{0.000000in}{-0.048611in}}{\pgfqpoint{0.000000in}{0.000000in}}{%
\pgfpathmoveto{\pgfqpoint{0.000000in}{0.000000in}}%
\pgfpathlineto{\pgfqpoint{0.000000in}{-0.048611in}}%
\pgfusepath{stroke,fill}%
}%
\begin{pgfscope}%
\pgfsys@transformshift{3.275171in}{0.402778in}%
\pgfsys@useobject{currentmarker}{}%
\end{pgfscope}%
\end{pgfscope}%
\begin{pgfscope}%
\pgftext[x=3.275171in,y=0.305556in,,top]{\rmfamily\fontsize{10.000000}{12.000000}\selectfont \(\displaystyle 6\)}%
\end{pgfscope}%
\begin{pgfscope}%
\pgfsetbuttcap%
\pgfsetroundjoin%
\definecolor{currentfill}{rgb}{0.000000,0.000000,0.000000}%
\pgfsetfillcolor{currentfill}%
\pgfsetlinewidth{0.803000pt}%
\definecolor{currentstroke}{rgb}{0.000000,0.000000,0.000000}%
\pgfsetstrokecolor{currentstroke}%
\pgfsetdash{}{0pt}%
\pgfsys@defobject{currentmarker}{\pgfqpoint{0.000000in}{-0.048611in}}{\pgfqpoint{0.000000in}{0.000000in}}{%
\pgfpathmoveto{\pgfqpoint{0.000000in}{0.000000in}}%
\pgfpathlineto{\pgfqpoint{0.000000in}{-0.048611in}}%
\pgfusepath{stroke,fill}%
}%
\begin{pgfscope}%
\pgfsys@transformshift{3.768176in}{0.402778in}%
\pgfsys@useobject{currentmarker}{}%
\end{pgfscope}%
\end{pgfscope}%
\begin{pgfscope}%
\pgftext[x=3.768176in,y=0.305556in,,top]{\rmfamily\fontsize{10.000000}{12.000000}\selectfont \(\displaystyle 7\)}%
\end{pgfscope}%
\begin{pgfscope}%
\pgfsetbuttcap%
\pgfsetroundjoin%
\definecolor{currentfill}{rgb}{0.000000,0.000000,0.000000}%
\pgfsetfillcolor{currentfill}%
\pgfsetlinewidth{0.803000pt}%
\definecolor{currentstroke}{rgb}{0.000000,0.000000,0.000000}%
\pgfsetstrokecolor{currentstroke}%
\pgfsetdash{}{0pt}%
\pgfsys@defobject{currentmarker}{\pgfqpoint{0.000000in}{-0.048611in}}{\pgfqpoint{0.000000in}{0.000000in}}{%
\pgfpathmoveto{\pgfqpoint{0.000000in}{0.000000in}}%
\pgfpathlineto{\pgfqpoint{0.000000in}{-0.048611in}}%
\pgfusepath{stroke,fill}%
}%
\begin{pgfscope}%
\pgfsys@transformshift{4.261181in}{0.402778in}%
\pgfsys@useobject{currentmarker}{}%
\end{pgfscope}%
\end{pgfscope}%
\begin{pgfscope}%
\pgftext[x=4.261181in,y=0.305556in,,top]{\rmfamily\fontsize{10.000000}{12.000000}\selectfont \(\displaystyle 8\)}%
\end{pgfscope}%
\begin{pgfscope}%
\pgfsetbuttcap%
\pgfsetroundjoin%
\definecolor{currentfill}{rgb}{0.000000,0.000000,0.000000}%
\pgfsetfillcolor{currentfill}%
\pgfsetlinewidth{0.803000pt}%
\definecolor{currentstroke}{rgb}{0.000000,0.000000,0.000000}%
\pgfsetstrokecolor{currentstroke}%
\pgfsetdash{}{0pt}%
\pgfsys@defobject{currentmarker}{\pgfqpoint{0.000000in}{-0.048611in}}{\pgfqpoint{0.000000in}{0.000000in}}{%
\pgfpathmoveto{\pgfqpoint{0.000000in}{0.000000in}}%
\pgfpathlineto{\pgfqpoint{0.000000in}{-0.048611in}}%
\pgfusepath{stroke,fill}%
}%
\begin{pgfscope}%
\pgfsys@transformshift{4.754187in}{0.402778in}%
\pgfsys@useobject{currentmarker}{}%
\end{pgfscope}%
\end{pgfscope}%
\begin{pgfscope}%
\pgftext[x=4.754187in,y=0.305556in,,top]{\rmfamily\fontsize{10.000000}{12.000000}\selectfont \(\displaystyle 9\)}%
\end{pgfscope}%
\begin{pgfscope}%
\pgftext[x=2.782165in,y=0.123861in,,top]{\rmfamily\fontsize{10.000000}{12.000000}\selectfont k}%
\end{pgfscope}%
\begin{pgfscope}%
\pgfsetbuttcap%
\pgfsetroundjoin%
\definecolor{currentfill}{rgb}{0.000000,0.000000,0.000000}%
\pgfsetfillcolor{currentfill}%
\pgfsetlinewidth{0.803000pt}%
\definecolor{currentstroke}{rgb}{0.000000,0.000000,0.000000}%
\pgfsetstrokecolor{currentstroke}%
\pgfsetdash{}{0pt}%
\pgfsys@defobject{currentmarker}{\pgfqpoint{-0.048611in}{0.000000in}}{\pgfqpoint{0.000000in}{0.000000in}}{%
\pgfpathmoveto{\pgfqpoint{0.000000in}{0.000000in}}%
\pgfpathlineto{\pgfqpoint{-0.048611in}{0.000000in}}%
\pgfusepath{stroke,fill}%
}%
\begin{pgfscope}%
\pgfsys@transformshift{0.612942in}{0.468135in}%
\pgfsys@useobject{currentmarker}{}%
\end{pgfscope}%
\end{pgfscope}%
\begin{pgfscope}%
\pgftext[x=0.172354in,y=0.419917in,left,base]{\rmfamily\fontsize{10.000000}{12.000000}\selectfont \(\displaystyle 10^{-10}\)}%
\end{pgfscope}%
\begin{pgfscope}%
\pgfsetbuttcap%
\pgfsetroundjoin%
\definecolor{currentfill}{rgb}{0.000000,0.000000,0.000000}%
\pgfsetfillcolor{currentfill}%
\pgfsetlinewidth{0.803000pt}%
\definecolor{currentstroke}{rgb}{0.000000,0.000000,0.000000}%
\pgfsetstrokecolor{currentstroke}%
\pgfsetdash{}{0pt}%
\pgfsys@defobject{currentmarker}{\pgfqpoint{-0.048611in}{0.000000in}}{\pgfqpoint{0.000000in}{0.000000in}}{%
\pgfpathmoveto{\pgfqpoint{0.000000in}{0.000000in}}%
\pgfpathlineto{\pgfqpoint{-0.048611in}{0.000000in}}%
\pgfusepath{stroke,fill}%
}%
\begin{pgfscope}%
\pgfsys@transformshift{0.612942in}{2.752236in}%
\pgfsys@useobject{currentmarker}{}%
\end{pgfscope}%
\end{pgfscope}%
\begin{pgfscope}%
\pgftext[x=0.227717in,y=2.704019in,left,base]{\rmfamily\fontsize{10.000000}{12.000000}\selectfont \(\displaystyle 10^{-9}\)}%
\end{pgfscope}%
\begin{pgfscope}%
\pgfsetbuttcap%
\pgfsetroundjoin%
\definecolor{currentfill}{rgb}{0.000000,0.000000,0.000000}%
\pgfsetfillcolor{currentfill}%
\pgfsetlinewidth{0.602250pt}%
\definecolor{currentstroke}{rgb}{0.000000,0.000000,0.000000}%
\pgfsetstrokecolor{currentstroke}%
\pgfsetdash{}{0pt}%
\pgfsys@defobject{currentmarker}{\pgfqpoint{-0.027778in}{0.000000in}}{\pgfqpoint{0.000000in}{0.000000in}}{%
\pgfpathmoveto{\pgfqpoint{0.000000in}{0.000000in}}%
\pgfpathlineto{\pgfqpoint{-0.027778in}{0.000000in}}%
\pgfusepath{stroke,fill}%
}%
\begin{pgfscope}%
\pgfsys@transformshift{0.612942in}{1.155718in}%
\pgfsys@useobject{currentmarker}{}%
\end{pgfscope}%
\end{pgfscope}%
\begin{pgfscope}%
\pgfsetbuttcap%
\pgfsetroundjoin%
\definecolor{currentfill}{rgb}{0.000000,0.000000,0.000000}%
\pgfsetfillcolor{currentfill}%
\pgfsetlinewidth{0.602250pt}%
\definecolor{currentstroke}{rgb}{0.000000,0.000000,0.000000}%
\pgfsetstrokecolor{currentstroke}%
\pgfsetdash{}{0pt}%
\pgfsys@defobject{currentmarker}{\pgfqpoint{-0.027778in}{0.000000in}}{\pgfqpoint{0.000000in}{0.000000in}}{%
\pgfpathmoveto{\pgfqpoint{0.000000in}{0.000000in}}%
\pgfpathlineto{\pgfqpoint{-0.027778in}{0.000000in}}%
\pgfusepath{stroke,fill}%
}%
\begin{pgfscope}%
\pgfsys@transformshift{0.612942in}{1.557928in}%
\pgfsys@useobject{currentmarker}{}%
\end{pgfscope}%
\end{pgfscope}%
\begin{pgfscope}%
\pgfsetbuttcap%
\pgfsetroundjoin%
\definecolor{currentfill}{rgb}{0.000000,0.000000,0.000000}%
\pgfsetfillcolor{currentfill}%
\pgfsetlinewidth{0.602250pt}%
\definecolor{currentstroke}{rgb}{0.000000,0.000000,0.000000}%
\pgfsetstrokecolor{currentstroke}%
\pgfsetdash{}{0pt}%
\pgfsys@defobject{currentmarker}{\pgfqpoint{-0.027778in}{0.000000in}}{\pgfqpoint{0.000000in}{0.000000in}}{%
\pgfpathmoveto{\pgfqpoint{0.000000in}{0.000000in}}%
\pgfpathlineto{\pgfqpoint{-0.027778in}{0.000000in}}%
\pgfusepath{stroke,fill}%
}%
\begin{pgfscope}%
\pgfsys@transformshift{0.612942in}{1.843301in}%
\pgfsys@useobject{currentmarker}{}%
\end{pgfscope}%
\end{pgfscope}%
\begin{pgfscope}%
\pgfsetbuttcap%
\pgfsetroundjoin%
\definecolor{currentfill}{rgb}{0.000000,0.000000,0.000000}%
\pgfsetfillcolor{currentfill}%
\pgfsetlinewidth{0.602250pt}%
\definecolor{currentstroke}{rgb}{0.000000,0.000000,0.000000}%
\pgfsetstrokecolor{currentstroke}%
\pgfsetdash{}{0pt}%
\pgfsys@defobject{currentmarker}{\pgfqpoint{-0.027778in}{0.000000in}}{\pgfqpoint{0.000000in}{0.000000in}}{%
\pgfpathmoveto{\pgfqpoint{0.000000in}{0.000000in}}%
\pgfpathlineto{\pgfqpoint{-0.027778in}{0.000000in}}%
\pgfusepath{stroke,fill}%
}%
\begin{pgfscope}%
\pgfsys@transformshift{0.612942in}{2.064653in}%
\pgfsys@useobject{currentmarker}{}%
\end{pgfscope}%
\end{pgfscope}%
\begin{pgfscope}%
\pgfsetbuttcap%
\pgfsetroundjoin%
\definecolor{currentfill}{rgb}{0.000000,0.000000,0.000000}%
\pgfsetfillcolor{currentfill}%
\pgfsetlinewidth{0.602250pt}%
\definecolor{currentstroke}{rgb}{0.000000,0.000000,0.000000}%
\pgfsetstrokecolor{currentstroke}%
\pgfsetdash{}{0pt}%
\pgfsys@defobject{currentmarker}{\pgfqpoint{-0.027778in}{0.000000in}}{\pgfqpoint{0.000000in}{0.000000in}}{%
\pgfpathmoveto{\pgfqpoint{0.000000in}{0.000000in}}%
\pgfpathlineto{\pgfqpoint{-0.027778in}{0.000000in}}%
\pgfusepath{stroke,fill}%
}%
\begin{pgfscope}%
\pgfsys@transformshift{0.612942in}{2.245511in}%
\pgfsys@useobject{currentmarker}{}%
\end{pgfscope}%
\end{pgfscope}%
\begin{pgfscope}%
\pgfsetbuttcap%
\pgfsetroundjoin%
\definecolor{currentfill}{rgb}{0.000000,0.000000,0.000000}%
\pgfsetfillcolor{currentfill}%
\pgfsetlinewidth{0.602250pt}%
\definecolor{currentstroke}{rgb}{0.000000,0.000000,0.000000}%
\pgfsetstrokecolor{currentstroke}%
\pgfsetdash{}{0pt}%
\pgfsys@defobject{currentmarker}{\pgfqpoint{-0.027778in}{0.000000in}}{\pgfqpoint{0.000000in}{0.000000in}}{%
\pgfpathmoveto{\pgfqpoint{0.000000in}{0.000000in}}%
\pgfpathlineto{\pgfqpoint{-0.027778in}{0.000000in}}%
\pgfusepath{stroke,fill}%
}%
\begin{pgfscope}%
\pgfsys@transformshift{0.612942in}{2.398425in}%
\pgfsys@useobject{currentmarker}{}%
\end{pgfscope}%
\end{pgfscope}%
\begin{pgfscope}%
\pgfsetbuttcap%
\pgfsetroundjoin%
\definecolor{currentfill}{rgb}{0.000000,0.000000,0.000000}%
\pgfsetfillcolor{currentfill}%
\pgfsetlinewidth{0.602250pt}%
\definecolor{currentstroke}{rgb}{0.000000,0.000000,0.000000}%
\pgfsetstrokecolor{currentstroke}%
\pgfsetdash{}{0pt}%
\pgfsys@defobject{currentmarker}{\pgfqpoint{-0.027778in}{0.000000in}}{\pgfqpoint{0.000000in}{0.000000in}}{%
\pgfpathmoveto{\pgfqpoint{0.000000in}{0.000000in}}%
\pgfpathlineto{\pgfqpoint{-0.027778in}{0.000000in}}%
\pgfusepath{stroke,fill}%
}%
\begin{pgfscope}%
\pgfsys@transformshift{0.612942in}{2.530884in}%
\pgfsys@useobject{currentmarker}{}%
\end{pgfscope}%
\end{pgfscope}%
\begin{pgfscope}%
\pgfsetbuttcap%
\pgfsetroundjoin%
\definecolor{currentfill}{rgb}{0.000000,0.000000,0.000000}%
\pgfsetfillcolor{currentfill}%
\pgfsetlinewidth{0.602250pt}%
\definecolor{currentstroke}{rgb}{0.000000,0.000000,0.000000}%
\pgfsetstrokecolor{currentstroke}%
\pgfsetdash{}{0pt}%
\pgfsys@defobject{currentmarker}{\pgfqpoint{-0.027778in}{0.000000in}}{\pgfqpoint{0.000000in}{0.000000in}}{%
\pgfpathmoveto{\pgfqpoint{0.000000in}{0.000000in}}%
\pgfpathlineto{\pgfqpoint{-0.027778in}{0.000000in}}%
\pgfusepath{stroke,fill}%
}%
\begin{pgfscope}%
\pgfsys@transformshift{0.612942in}{2.647722in}%
\pgfsys@useobject{currentmarker}{}%
\end{pgfscope}%
\end{pgfscope}%
\begin{pgfscope}%
\pgftext[x=0.116799in,y=1.718750in,,bottom,rotate=90.000000]{\rmfamily\fontsize{10.000000}{12.000000}\selectfont E}%
\end{pgfscope}%
\begin{pgfscope}%
\pgfpathrectangle{\pgfqpoint{0.612942in}{0.402778in}}{\pgfqpoint{4.338447in}{2.631944in}}%
\pgfusepath{clip}%
\pgfsetrectcap%
\pgfsetroundjoin%
\pgfsetlinewidth{1.505625pt}%
\definecolor{currentstroke}{rgb}{0.121569,0.466667,0.705882}%
\pgfsetstrokecolor{currentstroke}%
\pgfsetdash{}{0pt}%
\pgfpathmoveto{\pgfqpoint{0.810144in}{2.915088in}}%
\pgfpathlineto{\pgfqpoint{1.303149in}{0.615436in}}%
\pgfpathlineto{\pgfqpoint{1.796155in}{0.578455in}}%
\pgfpathlineto{\pgfqpoint{2.289160in}{0.569188in}}%
\pgfpathlineto{\pgfqpoint{2.782165in}{0.522412in}}%
\pgfpathlineto{\pgfqpoint{3.275171in}{0.534482in}}%
\pgfpathlineto{\pgfqpoint{3.768176in}{0.554987in}}%
\pgfpathlineto{\pgfqpoint{4.261181in}{0.554551in}}%
\pgfpathlineto{\pgfqpoint{4.754187in}{0.547197in}}%
\pgfusepath{stroke}%
\end{pgfscope}%
\begin{pgfscope}%
\pgfsetrectcap%
\pgfsetmiterjoin%
\pgfsetlinewidth{0.803000pt}%
\definecolor{currentstroke}{rgb}{0.000000,0.000000,0.000000}%
\pgfsetstrokecolor{currentstroke}%
\pgfsetdash{}{0pt}%
\pgfpathmoveto{\pgfqpoint{0.612942in}{0.402778in}}%
\pgfpathlineto{\pgfqpoint{0.612942in}{3.034722in}}%
\pgfusepath{stroke}%
\end{pgfscope}%
\begin{pgfscope}%
\pgfsetrectcap%
\pgfsetmiterjoin%
\pgfsetlinewidth{0.803000pt}%
\definecolor{currentstroke}{rgb}{0.000000,0.000000,0.000000}%
\pgfsetstrokecolor{currentstroke}%
\pgfsetdash{}{0pt}%
\pgfpathmoveto{\pgfqpoint{4.951389in}{0.402778in}}%
\pgfpathlineto{\pgfqpoint{4.951389in}{3.034722in}}%
\pgfusepath{stroke}%
\end{pgfscope}%
\begin{pgfscope}%
\pgfsetrectcap%
\pgfsetmiterjoin%
\pgfsetlinewidth{0.803000pt}%
\definecolor{currentstroke}{rgb}{0.000000,0.000000,0.000000}%
\pgfsetstrokecolor{currentstroke}%
\pgfsetdash{}{0pt}%
\pgfpathmoveto{\pgfqpoint{0.612942in}{0.402778in}}%
\pgfpathlineto{\pgfqpoint{4.951389in}{0.402778in}}%
\pgfusepath{stroke}%
\end{pgfscope}%
\begin{pgfscope}%
\pgfsetrectcap%
\pgfsetmiterjoin%
\pgfsetlinewidth{0.803000pt}%
\definecolor{currentstroke}{rgb}{0.000000,0.000000,0.000000}%
\pgfsetstrokecolor{currentstroke}%
\pgfsetdash{}{0pt}%
\pgfpathmoveto{\pgfqpoint{0.612942in}{3.034722in}}%
\pgfpathlineto{\pgfqpoint{4.951389in}{3.034722in}}%
\pgfusepath{stroke}%
\end{pgfscope}%
\end{pgfpicture}%
\makeatother%
\endgroup%

    \caption{Entwickung des globalen Fehlers über die Iteraionen.}
    \label{fig:error_global_sim}
\end{figure}

\subsubsection*{Kondensator}
Um das exakte Ergebnis für den belasteten Kondensator zu erhalten, konnte die Gleichung symbolisch gelöst werden. Die ersten drei Iterationen werden, wie im Python Experiment, strikt besser als die Vorherigen, wobei die späteren sich eher erratisch verhalten, wie am lokalen Fehler nachzuvollziehen ist. Jedoch zeigt der globale Fehler, dass sich nach der vierten Iteration keine nennenswerte Verbesserungen mehr einstellt. Bei den verwendeten zwölf Threads ergibt sich hier eine theoretische Beschleunigung von \(\frac{12}{4} = 3\), unter Vernachlässigung des groben Lösers und notwendiger Kommunikation zwischen den Threads.

\subsubsection*{Invertierer}
Für den Invertierer liegt keine geschlossene Lösung vor. Zum Vergleich wurde eine sequenziell gerechnete Simulation mit niedrigeren Toleranzen verwendet. Zusätzlich zum Ausgangssignal ist in Abbildung~\ref{fig:sol_sim} noch das Eingangssignal zur Orientierung eingezeichnet. Optisch lässt sich beim Invertierer bereits ab der dritten Iteration kein Unterschied mehr feststellen. Nachdem hier keine geschlossene Lösung vorliegt, ist auch der Fehler der finalen Iteration nicht glatt. Im globalen Fehler ist zu sehen, dass bereits die zweite Version nicht mehr nennenswert verbessert wird. Dies suggeriert weiteres Potenzial beim groben Löser. Im gezeigten Beispiel mit zwölf Threads ergibt sich eine eine theoretische Beschleunigung von \(\frac{12}{2} = 6\). Um zu zeigen, dass diese Ergebnisse durch eine größere Anzahl an Threads noch verbessert werden kann, wurde das gleiche Experiment mit 90 Threads wiederholt. Hiebei resultierte eine theoretische Beschleunigung von \(\frac{90}{7} \approx 13\).