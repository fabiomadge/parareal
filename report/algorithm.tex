Der Kerngedanke des Parareal Algorithmus ist die Kombination eines groben Lösers \(\mathcal{G}\) mit einem feinen \(\mathcal{F}\) und wurde erstmals in~\cite{Lions:2001} beschrieben. \(\mathcal{F}\) erfüllt die ursprünglich geforderten Anforderungen an die Genauigkeit, während \(\mathcal{G}\) diese nicht erfüllen muss, jedoch eine deutlich kürzere Rechenzeit erfordern soll. Erreicht werden kann dies durch die Verwendung von einer geringeren Anzahl an Zeitschritten, oder einem weniger komplexen Diskretisierungsverfahren. Die Parallelität entsteht durch Gebietszerlegung, also dem Lösen der Gleichung an mehreren Zeitpunkten zur gleichen Zeit. Um dies zu organisieren, wird die Simulationszeit \(\left(t_0, T\right] = \bigcup_{i=0}^{P-1} \left(t_i, t_{i+1}\right]\) in \(P\) Unterintervalle aufgeteilt. Die zur Lösung der Unterintervalle notwendige Rechenzeit sollte möglichst ähnlich sein. Gleichzeitig ist es geschickt, die Schnittstellen \(\left\{t_i \,\middle|\, 1 \leq i \leq P-1\right\}\) jeweils so zu wählen, dass der zu erwartende lokale Fehler möglichst gering ist. Da jedes Unterintervall parallel simuliert werden soll, werden Startwerte benötigt. Zu diesem Zweck wird \(\mathcal{G}\) auf \(\left(t_0, T\right]\) ausgeführt
\begin{displaymath}
    y_{j}^{0} = \mathcal{G}\!\left(y(t_0), t_0, t_{j}\right), \quad j \in \left\{i \,\middle|\, 0 \leq i < P\right\}.
\end{displaymath}
Der grobe Löser wird hier durch eine ternäre Funktion \(\mathcal{G}\!\left(y, t_{A}, t_{\Omega}\right) \approx y(t_\Omega)\) abstrahiert. Das erste Argument enthält den Startwert, das zweite den Startzeitpunkt und das letzte den Endzeitpunkt der Berechnung. Der Wert der Funktion ist eine Approximation für die Lösung der Differentialgleichung zum Zeitpunkt \(t_\Omega\). \(y_j^k\) bezeichnet den Startwert für das Unterintervall \(j\) in der Iteration \(k\). In jeder Iteration wird, ausgehend vom gemeinsamen approximierten Startwert, sowohl \(\mathcal{G}\), als auch \(\mathcal{F}\) ausgeführt. Erlangt wird der Startwert durch die Kombination der Ergebnisse der Löser im zeitlich direkt vorhergehenden Unterintervall
\begin{equation} \label{eq:1}
    y_{j+1}^{k+1} = \mathcal{G}\!\!\left(y_j^{k+1}, t_j, t_{j+1}\right) + \mathcal{F}\!\!\left(y_j^k, t_j, t_{j+1}\right) - \mathcal{G}\!\!\left(y_j^k, t_j, t_{j+1}\right).
\end{equation}
Offensichtlich würde \(y_{j+1}^{k+1} = \mathcal{F}\!\!\left(y_j^{k+1}, t_j, t_{j+1}\right)\) zu einem optimalen Startwert führen, jedoch ließen sich hier keine Gewinne durch Parallelisierung realisieren. Die Idee ist aber dieses Ergebnis möglichst genau vorherzusagen. Dazu wird die Auswirkung des neuen Startwerts auf das Ergebnis von \(\mathcal{F}\) durch die Auswirkung auf \(\mathcal{G}\) abgeschätzt. So kann der systematische Fehler von \(\mathcal{G}\) in eingeschänktem Maß isoliert werden. Diese Vorgehensweise kann anhand von Abbildung~\ref{fig:merge} nachvollzogen werden.
\begin{figure}[ht]
    \centering
        \begin{tikzpicture}[scale=2.2]
            \documentclass[]{minimal}
\usepackage{tikz}
\usetikzlibrary{positioning}
\usetikzlibrary{arrows}
\usetikzlibrary{patterns}

\begin{document}

\pagestyle{empty}

\begin{tikzpicture}[scale=2]


% coordinates
\draw[->] (0,0) -- (0,3) node[pos=1,below left] {y};

\draw[-] (0,0) -- (.25,0);
\draw[dotted] (.25,0) -- (.5,0);
\draw[-] (.5,0) -- (5.5,0);
\draw[dotted] (5.5,0) -- (5.75,0);
\draw[->] (5.75,0) -- (6,0) node[pos=1,below left] {x};

% threads
\draw[loosely dashed] (1,-.3) -- (1,3.3);
\draw[loosely dashed] (3,-.3) -- (3,3.3);
\draw[loosely dashed] (5,-.3) -- (5,3.3);
\node[below] at (2,0) {$t_{i}$};
\node[below] at (4,0) {$t_{i+1}$};

% exact
\draw[thick] (.5,-.3) parabola  bend +(5,2.6) (5.5,2.3);

\begin{scope}[very thick]

% pCoarse
\begin{scope}[loosely dotted, color=blue]
\draw (1,.8) to[out=45,in=215] (3,2.6);
\draw (3,2.15) to[out=39,in=192] (5,2.95);
\end{scope}

% pFine
\begin{scope}[densely dotted, color=blue]
\draw (1,.8) to[out=44,in=212] (3,2.35);
\draw (3,2.15) to[out=28,in=186] (5,2.75);
\end{scope}

% coarse
\begin{scope}[loosely dotted, color=orange]
\draw (1,.5) to[out=45,in=215] (3,2.2);
\draw (3,1.95) to[out=32,in=188] (5,2.6);
\end{scope}
\end{scope}

% merges
\begin{scope}[red, ultra thick]
\draw[->] (3,2.2) -- (3,1.95);
\draw[->, dashed] (3,2.6) -- (3,2.35);

\draw[->] (5,2.6) -- (5,2.4);
\draw[->, dashed] (5,2.95) -- (5,2.75);
\end{scope}

%legend
\draw[] (3.25,.25) rectangle (4.75, 1.65);

\begin{scope}[anchor=west]
    \node[] at (3.8,1.45) {fine};
    \node[] at (3.8,1.25) {coarse};
    \node[] at (3.8,.95) {$i_k$};
    \node[] at (3.8,.75) {$i_{k+1}$};
    \node[] at (3.8,.45) {correction};
\end{scope}

\draw[-, loosely dotted,very thick] (3.4,1.45) -- (3.7,1.45);
\draw[-, dotted,very thick] (3.4,1.25) -- (3.7,1.25);
\draw[pattern color=blue,pattern=north east lines] (3.4,.9) rectangle (3.7, 1);
\draw[pattern color=orange,pattern=north east lines] (3.4,.7) rectangle (3.7, .8);
\draw[->,red, ultra thick] (3.4,.45) -- (3.7,.45);

\end{tikzpicture}

\end{document}
        \end{tikzpicture}
    \caption{Entstehung der Korrektur}
    \label{fig:merge}
\end{figure}
Mit jeder Iteration \(i_k\) entspricht ein weiteres Unterintervall genau dem Ergebnis von \(\mathcal{F}\) auf dem Gesamtintervall, da sich der ursprüngliche Anfangswert propagiert. Somit müssen die ersten \(k\) Unterintervalle in \(i_k\) nicht mehr berechnet werden. Dort ist der Fixpunkt lokal bereits erreicht. Dadurch ergeben sich auch die Startwerte für eine neue Iteration
\begin{displaymath}
    y_{j+1}^{k+1} = \mathcal{F}\!\!\left(y_{j}^{k}, t_j, t_{j+1}\right) = \mathcal{F}\!\!\left(y_0, t_0, t_{j+1}\right), \quad j = k.
\end{displaymath}

Das resultierende Verfahren wird als \(\mathcal{P}\!\left(y, t_{A}, t_{\Omega},K,P\right) \approx y(t_\Omega)\) codiert. Über die bekannten Argumente hinaus, werden die Anzahl der Iterationen \(K\) und die Anzahl der Prozesse \(P\) spezifiziert. Wie durch den Buchstaben \(P\) und die Kriterien für die Wahl der Unterintervalle suggeriert, bietet es sich an die Unterintervalle jeweils in einem eigenen Prozess zu berechnen. Der dabei notwendige Ablauf und die Kommunikation zwischen den Prozessen lässt sich mit Hilfe von Abbildung~\ref{fig:sequence} vergegenwärtigen.
\begin{figure}[ht]
    \centering
        \documentclass[]{minimal}
\usepackage{tikz}
\usetikzlibrary{positioning}
\usetikzlibrary{arrows}

\begin{document}

\pagestyle{empty}

\begin{tikzpicture}[scale=2]
    \newcommand\Threads{4}
    \newcommand\Threadsmm{3}

    % Y-Axis
    \draw[->] (-.3,.3) -- (-.3,-1*\Threads-1.7) node[pos=1,below left, sloped] {Iteration};
    \foreach \x in {0,...,\Threads} {
        \draw (-.4, -1.1*\x) -- (-.2, -1.1*\x) node [pos=0, left] {\x};
    }

    % X-Axis
    \draw[->] (-.3,.3) -- (.75*\Threads+1.4,.3) node[pos=1,above left] {Thread};
    \foreach \x in {0,...,\Threads} {
        \draw(.75*\x, .2) rectangle (.5+0.75*\x, .4);
        \node at (0.75*\x+.25,.5) {\x};
        }


    % Computatations
    \foreach \y in {0,...,\Threads} {
        \foreach \x in {\y,...,\Threads} {
            \draw[fill=red] (.75*\x,-\y-.1-.1*\x) rectangle (.5+0.75*\x, -\y-.1*\x);
            \draw[fill=green] (.75*\x,-\y-1-.1*\x) rectangle (.5+0.75*\x, -\y-.1-.1*\x);
        }
    }

    % Communication
    % init
    \foreach \x in {0,...,\Threadsmm} {
        \draw[->] (.5+0.75*\x,-.1*\x-.1) -- (.75+0.75*\x,-.1*\x-.1);
    }
    % first of every iteration
    \foreach \x in {0,...,\Threadsmm} {
        \draw[->] (.5+0.75*\x,-1.1*\x-1) -- (.75+0.75*\x,-1.1*\x-1.1);
    }
    % merges
    \foreach \y in {1,...,\Threadsmm} {
        \foreach \x in {\y,...,\Threadsmm} {
        % previous coarse
        \draw[shorten >=0.08cm] (.5+0.75*\x,-\y-.1*\x+.9) -- (.71+0.75*\x,-\y-.1-.1*\x);
        % fine
        \draw[ shorten >=0.09cm] (.5+0.75*\x,-\y-.1*\x) -- (.71+0.75*\x,-\y-.1-.1*\x);
        % coarse
        \draw[-o, shorten >=0.0cm] (.5+0.75*\x,-\y-.1*\x-.1) -- (.75+0.75*\x,-\y-.1-.1*\x);
        }
    }



\end{tikzpicture}


\end{document}
    \caption{Auslastung und Kommunikation der Threads}
    \label{fig:sequence}
\end{figure}
Beide Achsen bilden eine Zeit ab. Auf der x-Achse finden sich die Prozesse, also die Unterintervalle; beginnend bei \(t_0\) und aufsteigend angeordnet. Im der Gesamtbetrachtung ergibt sich die Simulationszeit. Die y-Achse zeigt die Realzeit, also die Laufzeit des Programms. Innerhalb der Prozesse werden Paare roter(\(\mathcal{G}\)) und grüner(\(\mathcal{F}\)) Rechtecke wiederholt. Sie unterschieden sich in der Ausdehnung auf der y-Achse, nachdem sie sich in der Laufzeit unterscheiden sollen. Das Paar bildet zusammen eine Iteration. Die Anzahl der Iteration eines Prozesses steigt mit seinem Index, weil sich die exakte Lösungs mit jeder Iteration um ein Unterintervall fortsetzt und somit auch der Anfangswert konstant bleibt. Der Beginn der Berechnungen ist bei den Prozessen jeweils um die Rechenzeit für \(\mathcal{G}\) versetzt, nachdem das Ergebnis des vorherigen Unterintervalls Aufwirkungen auf den Startwert hat. Die erforderliche Kommunikation zwischen den Prozessen wird durch die Pfeile gezeigt. Während der reguläre Pfeil das weitergeben eines direkten Ergebnisses von einem der Löser bedeutet, werden bei dem anderen Pfeil vorher gemäß Gleichung~\ref{eq:1} drei Ergebnisse zusammengeführt. Nachdem ein Prozess eine Iteration beendet hat, blockiert er, bis er einen neuen Startwert vom vorherigen bekommt. Aus der Abbildung ist leicht ersichtlich, dass es nicht sinnvoll ist alle möglichen Iterationen auch durchzuführen. Dies gilt sowohl Laufzeit \(\mathrm{L}(f)\), als auch für die Prozessorzeit. Die Prozessorzeit ist, verglichen mit streng sequenzieller Ausführung bereits nach einer Iteration größer, weil nicht nur \(\mathcal{F}\), sondern auch \(\mathcal{G}\) auf dem Gesamtintervall evaluiert werden muss. Eine optimale Prozessorzeit ist jedoch nicht das Ziel des Algorithmus, sondern eine Optimierung der Laufzeit. Diese ist nach \(P\) Iterationen aber auch schlechter
\begin{align*}
    \mathrm{L}\!\left(\mathcal{F}\!\!\left(y_0, t_0, T\right)\right)
    &\approx P \cdot \mathrm{L}\!\left(\mathcal{F}\!\!\left(y_i, t_i, t_{i+1}\right)\right)\\
    \mathrm{L}\!\left(\mathcal{P}\!\!\left(y_0, t_0, T, K, P\right)\right)
    &\approx K \cdot \mathrm{L}\!\left(\mathcal{F}\!\!\left(y_i, t_i, t_{i+1}\right)\right) + (P-1) (K-1) \mathrm{L}\!\left(\mathcal{G}\!\!\left(y_i, t_i, t_{i+1}\right)\right).
\end{align*}
Dadurch wird klar, dass der Algorithmus nur dann sinnvoll eingesetzt werden kann, wenn \(K \ll P\). Gleichzeitig gilt aber auch, dass in diesem Fall das Ergebnis ein anderes ist.
\subsection*{Experimente}
Um das tatsächliche Verhalten des Algorithmus zu untersuchen, haben wir den Algorithmus in Python implementiert. Für die meisten Analysen wird eine Differentialgleichung deren Lösung durch die logistische Funktion beschrieben wird. Sie kann als Wachstumsmodell unter Berücksichtigung des Ressourcenverbrauchs verwendet werden
\begin{displaymath}
    y'(t) = k \cdot y(t) \cdot (L - y(t)).
\end{displaymath}
Diese Gleichung lässt jedoch eine Schar von Lösungen zu, welche jedoch symbolisch zu berechnen ist. Mit der Randbedingung \(y(0)= \frac{L}{2}\) ist nur noch eine Lösung möglich
\begin{displaymath}
    y(t) = \frac{L}{1+e^{-k L t}}.
\end{displaymath}
Verschiedene numerische Lösungen der Gleichung finden sich oben in Abbildung~\ref{fig:sequence}. Alle sind, unter der Nutzung von 12 Prozessen, mit Parareal gerechnet und unterscheiden sich in der Anzahl der Iterationen. Auffällig sind die Sprungsstellen an den Unterintervallgrenzen und die Tatsache, dass sich mit jeder Iteration das Ergebnis in allen Unterintervallen stärker an das exakte Ergebnis annähert. Der untere Graph zeigt das Gleiche, aber löst die Gleichung für den belasteten Kondensator aus Abbildung~\ref{fig:cap}.
\begin{figure}[ht]
    \centering
        %% Creator: Matplotlib, PGF backend
%%
%% To include the figure in your LaTeX document, write
%%   \input{<filename>.pgf}
%%
%% Make sure the required packages are loaded in your preamble
%%   \usepackage{pgf}
%%
%% Figures using additional raster images can only be included by \input if
%% they are in the same directory as the main LaTeX file. For loading figures
%% from other directories you can use the `import` package
%%   \usepackage{import}
%% and then include the figures with
%%   \import{<path to file>}{<filename>.pgf}
%%
%% Matplotlib used the following preamble
%%   \usepackage{fontspec}
%%   \setmainfont{Times New Roman}
%%   \setsansfont{Lucida Grande}
%%   \setmonofont{Andale Mono}
%%
\begingroup%
\makeatletter%
\begin{pgfpicture}%
\pgfpathrectangle{\pgfpointorigin}{\pgfqpoint{5.000000in}{3.000000in}}%
\pgfusepath{use as bounding box, clip}%
\begin{pgfscope}%
\pgfsetbuttcap%
\pgfsetmiterjoin%
\definecolor{currentfill}{rgb}{1.000000,1.000000,1.000000}%
\pgfsetfillcolor{currentfill}%
\pgfsetlinewidth{0.000000pt}%
\definecolor{currentstroke}{rgb}{1.000000,1.000000,1.000000}%
\pgfsetstrokecolor{currentstroke}%
\pgfsetdash{}{0pt}%
\pgfpathmoveto{\pgfqpoint{0.000000in}{0.000000in}}%
\pgfpathlineto{\pgfqpoint{5.000000in}{0.000000in}}%
\pgfpathlineto{\pgfqpoint{5.000000in}{3.000000in}}%
\pgfpathlineto{\pgfqpoint{0.000000in}{3.000000in}}%
\pgfpathclose%
\pgfusepath{fill}%
\end{pgfscope}%
\begin{pgfscope}%
\pgfsetbuttcap%
\pgfsetmiterjoin%
\definecolor{currentfill}{rgb}{1.000000,1.000000,1.000000}%
\pgfsetfillcolor{currentfill}%
\pgfsetlinewidth{0.000000pt}%
\definecolor{currentstroke}{rgb}{0.000000,0.000000,0.000000}%
\pgfsetstrokecolor{currentstroke}%
\pgfsetstrokeopacity{0.000000}%
\pgfsetdash{}{0pt}%
\pgfpathmoveto{\pgfqpoint{0.467954in}{0.402778in}}%
\pgfpathlineto{\pgfqpoint{4.951389in}{0.402778in}}%
\pgfpathlineto{\pgfqpoint{4.951389in}{2.951389in}}%
\pgfpathlineto{\pgfqpoint{0.467954in}{2.951389in}}%
\pgfpathclose%
\pgfusepath{fill}%
\end{pgfscope}%
\begin{pgfscope}%
\pgfsetbuttcap%
\pgfsetroundjoin%
\definecolor{currentfill}{rgb}{0.000000,0.000000,0.000000}%
\pgfsetfillcolor{currentfill}%
\pgfsetlinewidth{0.803000pt}%
\definecolor{currentstroke}{rgb}{0.000000,0.000000,0.000000}%
\pgfsetstrokecolor{currentstroke}%
\pgfsetdash{}{0pt}%
\pgfsys@defobject{currentmarker}{\pgfqpoint{0.000000in}{-0.048611in}}{\pgfqpoint{0.000000in}{0.000000in}}{%
\pgfpathmoveto{\pgfqpoint{0.000000in}{0.000000in}}%
\pgfpathlineto{\pgfqpoint{0.000000in}{-0.048611in}}%
\pgfusepath{stroke,fill}%
}%
\begin{pgfscope}%
\pgfsys@transformshift{0.671747in}{0.402778in}%
\pgfsys@useobject{currentmarker}{}%
\end{pgfscope}%
\end{pgfscope}%
\begin{pgfscope}%
\pgftext[x=0.671747in,y=0.305556in,,top]{\rmfamily\fontsize{10.000000}{12.000000}\selectfont \(\displaystyle -6\)}%
\end{pgfscope}%
\begin{pgfscope}%
\pgfsetbuttcap%
\pgfsetroundjoin%
\definecolor{currentfill}{rgb}{0.000000,0.000000,0.000000}%
\pgfsetfillcolor{currentfill}%
\pgfsetlinewidth{0.803000pt}%
\definecolor{currentstroke}{rgb}{0.000000,0.000000,0.000000}%
\pgfsetstrokecolor{currentstroke}%
\pgfsetdash{}{0pt}%
\pgfsys@defobject{currentmarker}{\pgfqpoint{0.000000in}{-0.048611in}}{\pgfqpoint{0.000000in}{0.000000in}}{%
\pgfpathmoveto{\pgfqpoint{0.000000in}{0.000000in}}%
\pgfpathlineto{\pgfqpoint{0.000000in}{-0.048611in}}%
\pgfusepath{stroke,fill}%
}%
\begin{pgfscope}%
\pgfsys@transformshift{1.351055in}{0.402778in}%
\pgfsys@useobject{currentmarker}{}%
\end{pgfscope}%
\end{pgfscope}%
\begin{pgfscope}%
\pgftext[x=1.351055in,y=0.305556in,,top]{\rmfamily\fontsize{10.000000}{12.000000}\selectfont \(\displaystyle -4\)}%
\end{pgfscope}%
\begin{pgfscope}%
\pgfsetbuttcap%
\pgfsetroundjoin%
\definecolor{currentfill}{rgb}{0.000000,0.000000,0.000000}%
\pgfsetfillcolor{currentfill}%
\pgfsetlinewidth{0.803000pt}%
\definecolor{currentstroke}{rgb}{0.000000,0.000000,0.000000}%
\pgfsetstrokecolor{currentstroke}%
\pgfsetdash{}{0pt}%
\pgfsys@defobject{currentmarker}{\pgfqpoint{0.000000in}{-0.048611in}}{\pgfqpoint{0.000000in}{0.000000in}}{%
\pgfpathmoveto{\pgfqpoint{0.000000in}{0.000000in}}%
\pgfpathlineto{\pgfqpoint{0.000000in}{-0.048611in}}%
\pgfusepath{stroke,fill}%
}%
\begin{pgfscope}%
\pgfsys@transformshift{2.030363in}{0.402778in}%
\pgfsys@useobject{currentmarker}{}%
\end{pgfscope}%
\end{pgfscope}%
\begin{pgfscope}%
\pgftext[x=2.030363in,y=0.305556in,,top]{\rmfamily\fontsize{10.000000}{12.000000}\selectfont \(\displaystyle -2\)}%
\end{pgfscope}%
\begin{pgfscope}%
\pgfsetbuttcap%
\pgfsetroundjoin%
\definecolor{currentfill}{rgb}{0.000000,0.000000,0.000000}%
\pgfsetfillcolor{currentfill}%
\pgfsetlinewidth{0.803000pt}%
\definecolor{currentstroke}{rgb}{0.000000,0.000000,0.000000}%
\pgfsetstrokecolor{currentstroke}%
\pgfsetdash{}{0pt}%
\pgfsys@defobject{currentmarker}{\pgfqpoint{0.000000in}{-0.048611in}}{\pgfqpoint{0.000000in}{0.000000in}}{%
\pgfpathmoveto{\pgfqpoint{0.000000in}{0.000000in}}%
\pgfpathlineto{\pgfqpoint{0.000000in}{-0.048611in}}%
\pgfusepath{stroke,fill}%
}%
\begin{pgfscope}%
\pgfsys@transformshift{2.709672in}{0.402778in}%
\pgfsys@useobject{currentmarker}{}%
\end{pgfscope}%
\end{pgfscope}%
\begin{pgfscope}%
\pgftext[x=2.709672in,y=0.305556in,,top]{\rmfamily\fontsize{10.000000}{12.000000}\selectfont \(\displaystyle 0\)}%
\end{pgfscope}%
\begin{pgfscope}%
\pgfsetbuttcap%
\pgfsetroundjoin%
\definecolor{currentfill}{rgb}{0.000000,0.000000,0.000000}%
\pgfsetfillcolor{currentfill}%
\pgfsetlinewidth{0.803000pt}%
\definecolor{currentstroke}{rgb}{0.000000,0.000000,0.000000}%
\pgfsetstrokecolor{currentstroke}%
\pgfsetdash{}{0pt}%
\pgfsys@defobject{currentmarker}{\pgfqpoint{0.000000in}{-0.048611in}}{\pgfqpoint{0.000000in}{0.000000in}}{%
\pgfpathmoveto{\pgfqpoint{0.000000in}{0.000000in}}%
\pgfpathlineto{\pgfqpoint{0.000000in}{-0.048611in}}%
\pgfusepath{stroke,fill}%
}%
\begin{pgfscope}%
\pgfsys@transformshift{3.388980in}{0.402778in}%
\pgfsys@useobject{currentmarker}{}%
\end{pgfscope}%
\end{pgfscope}%
\begin{pgfscope}%
\pgftext[x=3.388980in,y=0.305556in,,top]{\rmfamily\fontsize{10.000000}{12.000000}\selectfont \(\displaystyle 2\)}%
\end{pgfscope}%
\begin{pgfscope}%
\pgfsetbuttcap%
\pgfsetroundjoin%
\definecolor{currentfill}{rgb}{0.000000,0.000000,0.000000}%
\pgfsetfillcolor{currentfill}%
\pgfsetlinewidth{0.803000pt}%
\definecolor{currentstroke}{rgb}{0.000000,0.000000,0.000000}%
\pgfsetstrokecolor{currentstroke}%
\pgfsetdash{}{0pt}%
\pgfsys@defobject{currentmarker}{\pgfqpoint{0.000000in}{-0.048611in}}{\pgfqpoint{0.000000in}{0.000000in}}{%
\pgfpathmoveto{\pgfqpoint{0.000000in}{0.000000in}}%
\pgfpathlineto{\pgfqpoint{0.000000in}{-0.048611in}}%
\pgfusepath{stroke,fill}%
}%
\begin{pgfscope}%
\pgfsys@transformshift{4.068288in}{0.402778in}%
\pgfsys@useobject{currentmarker}{}%
\end{pgfscope}%
\end{pgfscope}%
\begin{pgfscope}%
\pgftext[x=4.068288in,y=0.305556in,,top]{\rmfamily\fontsize{10.000000}{12.000000}\selectfont \(\displaystyle 4\)}%
\end{pgfscope}%
\begin{pgfscope}%
\pgfsetbuttcap%
\pgfsetroundjoin%
\definecolor{currentfill}{rgb}{0.000000,0.000000,0.000000}%
\pgfsetfillcolor{currentfill}%
\pgfsetlinewidth{0.803000pt}%
\definecolor{currentstroke}{rgb}{0.000000,0.000000,0.000000}%
\pgfsetstrokecolor{currentstroke}%
\pgfsetdash{}{0pt}%
\pgfsys@defobject{currentmarker}{\pgfqpoint{0.000000in}{-0.048611in}}{\pgfqpoint{0.000000in}{0.000000in}}{%
\pgfpathmoveto{\pgfqpoint{0.000000in}{0.000000in}}%
\pgfpathlineto{\pgfqpoint{0.000000in}{-0.048611in}}%
\pgfusepath{stroke,fill}%
}%
\begin{pgfscope}%
\pgfsys@transformshift{4.747596in}{0.402778in}%
\pgfsys@useobject{currentmarker}{}%
\end{pgfscope}%
\end{pgfscope}%
\begin{pgfscope}%
\pgftext[x=4.747596in,y=0.305556in,,top]{\rmfamily\fontsize{10.000000}{12.000000}\selectfont \(\displaystyle 6\)}%
\end{pgfscope}%
\begin{pgfscope}%
\pgftext[x=2.709672in,y=0.123861in,,top]{\rmfamily\fontsize{10.000000}{12.000000}\selectfont t}%
\end{pgfscope}%
\begin{pgfscope}%
\pgfsetbuttcap%
\pgfsetroundjoin%
\definecolor{currentfill}{rgb}{0.000000,0.000000,0.000000}%
\pgfsetfillcolor{currentfill}%
\pgfsetlinewidth{0.803000pt}%
\definecolor{currentstroke}{rgb}{0.000000,0.000000,0.000000}%
\pgfsetstrokecolor{currentstroke}%
\pgfsetdash{}{0pt}%
\pgfsys@defobject{currentmarker}{\pgfqpoint{-0.048611in}{0.000000in}}{\pgfqpoint{0.000000in}{0.000000in}}{%
\pgfpathmoveto{\pgfqpoint{0.000000in}{0.000000in}}%
\pgfpathlineto{\pgfqpoint{-0.048611in}{0.000000in}}%
\pgfusepath{stroke,fill}%
}%
\begin{pgfscope}%
\pgfsys@transformshift{0.467954in}{0.512870in}%
\pgfsys@useobject{currentmarker}{}%
\end{pgfscope}%
\end{pgfscope}%
\begin{pgfscope}%
\pgftext[x=0.193262in,y=0.464653in,left,base]{\rmfamily\fontsize{10.000000}{12.000000}\selectfont \(\displaystyle 0.0\)}%
\end{pgfscope}%
\begin{pgfscope}%
\pgfsetbuttcap%
\pgfsetroundjoin%
\definecolor{currentfill}{rgb}{0.000000,0.000000,0.000000}%
\pgfsetfillcolor{currentfill}%
\pgfsetlinewidth{0.803000pt}%
\definecolor{currentstroke}{rgb}{0.000000,0.000000,0.000000}%
\pgfsetstrokecolor{currentstroke}%
\pgfsetdash{}{0pt}%
\pgfsys@defobject{currentmarker}{\pgfqpoint{-0.048611in}{0.000000in}}{\pgfqpoint{0.000000in}{0.000000in}}{%
\pgfpathmoveto{\pgfqpoint{0.000000in}{0.000000in}}%
\pgfpathlineto{\pgfqpoint{-0.048611in}{0.000000in}}%
\pgfusepath{stroke,fill}%
}%
\begin{pgfscope}%
\pgfsys@transformshift{0.467954in}{0.978232in}%
\pgfsys@useobject{currentmarker}{}%
\end{pgfscope}%
\end{pgfscope}%
\begin{pgfscope}%
\pgftext[x=0.193262in,y=0.930014in,left,base]{\rmfamily\fontsize{10.000000}{12.000000}\selectfont \(\displaystyle 0.2\)}%
\end{pgfscope}%
\begin{pgfscope}%
\pgfsetbuttcap%
\pgfsetroundjoin%
\definecolor{currentfill}{rgb}{0.000000,0.000000,0.000000}%
\pgfsetfillcolor{currentfill}%
\pgfsetlinewidth{0.803000pt}%
\definecolor{currentstroke}{rgb}{0.000000,0.000000,0.000000}%
\pgfsetstrokecolor{currentstroke}%
\pgfsetdash{}{0pt}%
\pgfsys@defobject{currentmarker}{\pgfqpoint{-0.048611in}{0.000000in}}{\pgfqpoint{0.000000in}{0.000000in}}{%
\pgfpathmoveto{\pgfqpoint{0.000000in}{0.000000in}}%
\pgfpathlineto{\pgfqpoint{-0.048611in}{0.000000in}}%
\pgfusepath{stroke,fill}%
}%
\begin{pgfscope}%
\pgfsys@transformshift{0.467954in}{1.443594in}%
\pgfsys@useobject{currentmarker}{}%
\end{pgfscope}%
\end{pgfscope}%
\begin{pgfscope}%
\pgftext[x=0.193262in,y=1.395376in,left,base]{\rmfamily\fontsize{10.000000}{12.000000}\selectfont \(\displaystyle 0.4\)}%
\end{pgfscope}%
\begin{pgfscope}%
\pgfsetbuttcap%
\pgfsetroundjoin%
\definecolor{currentfill}{rgb}{0.000000,0.000000,0.000000}%
\pgfsetfillcolor{currentfill}%
\pgfsetlinewidth{0.803000pt}%
\definecolor{currentstroke}{rgb}{0.000000,0.000000,0.000000}%
\pgfsetstrokecolor{currentstroke}%
\pgfsetdash{}{0pt}%
\pgfsys@defobject{currentmarker}{\pgfqpoint{-0.048611in}{0.000000in}}{\pgfqpoint{0.000000in}{0.000000in}}{%
\pgfpathmoveto{\pgfqpoint{0.000000in}{0.000000in}}%
\pgfpathlineto{\pgfqpoint{-0.048611in}{0.000000in}}%
\pgfusepath{stroke,fill}%
}%
\begin{pgfscope}%
\pgfsys@transformshift{0.467954in}{1.908956in}%
\pgfsys@useobject{currentmarker}{}%
\end{pgfscope}%
\end{pgfscope}%
\begin{pgfscope}%
\pgftext[x=0.193262in,y=1.860738in,left,base]{\rmfamily\fontsize{10.000000}{12.000000}\selectfont \(\displaystyle 0.6\)}%
\end{pgfscope}%
\begin{pgfscope}%
\pgfsetbuttcap%
\pgfsetroundjoin%
\definecolor{currentfill}{rgb}{0.000000,0.000000,0.000000}%
\pgfsetfillcolor{currentfill}%
\pgfsetlinewidth{0.803000pt}%
\definecolor{currentstroke}{rgb}{0.000000,0.000000,0.000000}%
\pgfsetstrokecolor{currentstroke}%
\pgfsetdash{}{0pt}%
\pgfsys@defobject{currentmarker}{\pgfqpoint{-0.048611in}{0.000000in}}{\pgfqpoint{0.000000in}{0.000000in}}{%
\pgfpathmoveto{\pgfqpoint{0.000000in}{0.000000in}}%
\pgfpathlineto{\pgfqpoint{-0.048611in}{0.000000in}}%
\pgfusepath{stroke,fill}%
}%
\begin{pgfscope}%
\pgfsys@transformshift{0.467954in}{2.374317in}%
\pgfsys@useobject{currentmarker}{}%
\end{pgfscope}%
\end{pgfscope}%
\begin{pgfscope}%
\pgftext[x=0.193262in,y=2.326100in,left,base]{\rmfamily\fontsize{10.000000}{12.000000}\selectfont \(\displaystyle 0.8\)}%
\end{pgfscope}%
\begin{pgfscope}%
\pgfsetbuttcap%
\pgfsetroundjoin%
\definecolor{currentfill}{rgb}{0.000000,0.000000,0.000000}%
\pgfsetfillcolor{currentfill}%
\pgfsetlinewidth{0.803000pt}%
\definecolor{currentstroke}{rgb}{0.000000,0.000000,0.000000}%
\pgfsetstrokecolor{currentstroke}%
\pgfsetdash{}{0pt}%
\pgfsys@defobject{currentmarker}{\pgfqpoint{-0.048611in}{0.000000in}}{\pgfqpoint{0.000000in}{0.000000in}}{%
\pgfpathmoveto{\pgfqpoint{0.000000in}{0.000000in}}%
\pgfpathlineto{\pgfqpoint{-0.048611in}{0.000000in}}%
\pgfusepath{stroke,fill}%
}%
\begin{pgfscope}%
\pgfsys@transformshift{0.467954in}{2.839679in}%
\pgfsys@useobject{currentmarker}{}%
\end{pgfscope}%
\end{pgfscope}%
\begin{pgfscope}%
\pgftext[x=0.193262in,y=2.791461in,left,base]{\rmfamily\fontsize{10.000000}{12.000000}\selectfont \(\displaystyle 1.0\)}%
\end{pgfscope}%
\begin{pgfscope}%
\pgftext[x=0.137707in,y=1.677083in,,bottom,rotate=90.000000]{\rmfamily\fontsize{10.000000}{12.000000}\selectfont y(t)}%
\end{pgfscope}%
\begin{pgfscope}%
\pgfpathrectangle{\pgfqpoint{0.467954in}{0.402778in}}{\pgfqpoint{4.483434in}{2.548611in}}%
\pgfusepath{clip}%
\pgfsetrectcap%
\pgfsetroundjoin%
\pgfsetlinewidth{1.505625pt}%
\definecolor{currentstroke}{rgb}{0.121569,0.466667,0.705882}%
\pgfsetstrokecolor{currentstroke}%
\pgfsetdash{}{0pt}%
\pgfpathmoveto{\pgfqpoint{0.671747in}{0.518624in}}%
\pgfpathlineto{\pgfqpoint{0.894583in}{0.523933in}}%
\pgfpathlineto{\pgfqpoint{1.011318in}{0.528440in}}%
\pgfpathlineto{\pgfqpoint{1.012873in}{0.532894in}}%
\pgfpathlineto{\pgfqpoint{1.139994in}{0.541870in}}%
\pgfpathlineto{\pgfqpoint{1.247463in}{0.552479in}}%
\pgfpathlineto{\pgfqpoint{1.340939in}{0.564748in}}%
\pgfpathlineto{\pgfqpoint{1.350952in}{0.566265in}}%
\pgfpathlineto{\pgfqpoint{1.352506in}{0.580630in}}%
\pgfpathlineto{\pgfqpoint{1.426578in}{0.596549in}}%
\pgfpathlineto{\pgfqpoint{1.494492in}{0.614263in}}%
\pgfpathlineto{\pgfqpoint{1.557555in}{0.633874in}}%
\pgfpathlineto{\pgfqpoint{1.616638in}{0.655455in}}%
\pgfpathlineto{\pgfqpoint{1.672549in}{0.679145in}}%
\pgfpathlineto{\pgfqpoint{1.690585in}{0.687533in}}%
\pgfpathlineto{\pgfqpoint{1.692140in}{0.726096in}}%
\pgfpathlineto{\pgfqpoint{1.742267in}{0.756455in}}%
\pgfpathlineto{\pgfqpoint{1.791150in}{0.789674in}}%
\pgfpathlineto{\pgfqpoint{1.839163in}{0.826046in}}%
\pgfpathlineto{\pgfqpoint{1.886678in}{0.865935in}}%
\pgfpathlineto{\pgfqpoint{1.934069in}{0.909781in}}%
\pgfpathlineto{\pgfqpoint{1.981646in}{0.958043in}}%
\pgfpathlineto{\pgfqpoint{2.029845in}{1.011376in}}%
\pgfpathlineto{\pgfqpoint{2.030218in}{1.011806in}}%
\pgfpathlineto{\pgfqpoint{2.031773in}{1.074884in}}%
\pgfpathlineto{\pgfqpoint{2.083020in}{1.141677in}}%
\pgfpathlineto{\pgfqpoint{2.136941in}{1.217118in}}%
\pgfpathlineto{\pgfqpoint{2.195028in}{1.303820in}}%
\pgfpathlineto{\pgfqpoint{2.259957in}{1.406451in}}%
\pgfpathlineto{\pgfqpoint{2.338880in}{1.537282in}}%
\pgfpathlineto{\pgfqpoint{2.369852in}{1.589818in}}%
\pgfpathlineto{\pgfqpoint{2.371407in}{1.624306in}}%
\pgfpathlineto{\pgfqpoint{2.545546in}{1.918895in}}%
\pgfpathlineto{\pgfqpoint{2.617440in}{2.033707in}}%
\pgfpathlineto{\pgfqpoint{2.679135in}{2.126549in}}%
\pgfpathlineto{\pgfqpoint{2.709485in}{2.169961in}}%
\pgfpathlineto{\pgfqpoint{2.709547in}{2.155912in}}%
\pgfpathlineto{\pgfqpoint{2.711040in}{2.158032in}}%
\pgfpathlineto{\pgfqpoint{2.765894in}{2.233191in}}%
\pgfpathlineto{\pgfqpoint{2.817949in}{2.299357in}}%
\pgfpathlineto{\pgfqpoint{2.868138in}{2.358221in}}%
\pgfpathlineto{\pgfqpoint{2.917146in}{2.410992in}}%
\pgfpathlineto{\pgfqpoint{2.965470in}{2.458530in}}%
\pgfpathlineto{\pgfqpoint{3.013544in}{2.501523in}}%
\pgfpathlineto{\pgfqpoint{3.049118in}{2.530684in}}%
\pgfpathlineto{\pgfqpoint{3.049181in}{2.508546in}}%
\pgfpathlineto{\pgfqpoint{3.050673in}{2.509792in}}%
\pgfpathlineto{\pgfqpoint{3.098935in}{2.548011in}}%
\pgfpathlineto{\pgfqpoint{3.147756in}{2.582749in}}%
\pgfpathlineto{\pgfqpoint{3.197510in}{2.614368in}}%
\pgfpathlineto{\pgfqpoint{3.248570in}{2.643160in}}%
\pgfpathlineto{\pgfqpoint{3.301371in}{2.669383in}}%
\pgfpathlineto{\pgfqpoint{3.356412in}{2.693256in}}%
\pgfpathlineto{\pgfqpoint{3.388752in}{2.705790in}}%
\pgfpathlineto{\pgfqpoint{3.388814in}{2.691892in}}%
\pgfpathlineto{\pgfqpoint{3.390307in}{2.692499in}}%
\pgfpathlineto{\pgfqpoint{3.448021in}{2.714259in}}%
\pgfpathlineto{\pgfqpoint{3.509094in}{2.733962in}}%
\pgfpathlineto{\pgfqpoint{3.574397in}{2.751754in}}%
\pgfpathlineto{\pgfqpoint{3.644923in}{2.767730in}}%
\pgfpathlineto{\pgfqpoint{3.721979in}{2.781972in}}%
\pgfpathlineto{\pgfqpoint{3.728447in}{2.783034in}}%
\pgfpathlineto{\pgfqpoint{3.728510in}{2.776244in}}%
\pgfpathlineto{\pgfqpoint{3.730002in}{2.776514in}}%
\pgfpathlineto{\pgfqpoint{3.811723in}{2.789732in}}%
\pgfpathlineto{\pgfqpoint{3.903022in}{2.801310in}}%
\pgfpathlineto{\pgfqpoint{4.006821in}{2.811290in}}%
\pgfpathlineto{\pgfqpoint{4.068143in}{2.815932in}}%
\pgfpathlineto{\pgfqpoint{4.068205in}{2.812914in}}%
\pgfpathlineto{\pgfqpoint{4.069698in}{2.813030in}}%
\pgfpathlineto{\pgfqpoint{4.194394in}{2.821153in}}%
\pgfpathlineto{\pgfqpoint{4.344900in}{2.827751in}}%
\pgfpathlineto{\pgfqpoint{4.427927in}{2.829108in}}%
\pgfpathlineto{\pgfqpoint{4.630861in}{2.833851in}}%
\pgfpathlineto{\pgfqpoint{4.747596in}{2.835543in}}%
\pgfpathlineto{\pgfqpoint{4.747596in}{2.835543in}}%
\pgfusepath{stroke}%
\end{pgfscope}%
\begin{pgfscope}%
\pgfpathrectangle{\pgfqpoint{0.467954in}{0.402778in}}{\pgfqpoint{4.483434in}{2.548611in}}%
\pgfusepath{clip}%
\pgfsetrectcap%
\pgfsetroundjoin%
\pgfsetlinewidth{1.505625pt}%
\definecolor{currentstroke}{rgb}{1.000000,0.498039,0.054902}%
\pgfsetstrokecolor{currentstroke}%
\pgfsetdash{}{0pt}%
\pgfpathmoveto{\pgfqpoint{0.671747in}{0.518624in}}%
\pgfpathlineto{\pgfqpoint{0.894583in}{0.523933in}}%
\pgfpathlineto{\pgfqpoint{1.060886in}{0.530867in}}%
\pgfpathlineto{\pgfqpoint{1.194289in}{0.539426in}}%
\pgfpathlineto{\pgfqpoint{1.306111in}{0.549616in}}%
\pgfpathlineto{\pgfqpoint{1.350952in}{0.554708in}}%
\pgfpathlineto{\pgfqpoint{1.351014in}{0.551882in}}%
\pgfpathlineto{\pgfqpoint{1.352506in}{0.552051in}}%
\pgfpathlineto{\pgfqpoint{1.446417in}{0.564254in}}%
\pgfpathlineto{\pgfqpoint{1.529755in}{0.578142in}}%
\pgfpathlineto{\pgfqpoint{1.605008in}{0.593768in}}%
\pgfpathlineto{\pgfqpoint{1.673855in}{0.611178in}}%
\pgfpathlineto{\pgfqpoint{1.690585in}{0.615921in}}%
\pgfpathlineto{\pgfqpoint{1.690647in}{0.603281in}}%
\pgfpathlineto{\pgfqpoint{1.692140in}{0.603664in}}%
\pgfpathlineto{\pgfqpoint{1.757939in}{0.622160in}}%
\pgfpathlineto{\pgfqpoint{1.819261in}{0.642580in}}%
\pgfpathlineto{\pgfqpoint{1.876976in}{0.665032in}}%
\pgfpathlineto{\pgfqpoint{1.931768in}{0.689645in}}%
\pgfpathlineto{\pgfqpoint{1.984258in}{0.716602in}}%
\pgfpathlineto{\pgfqpoint{2.030218in}{0.743200in}}%
\pgfpathlineto{\pgfqpoint{2.030280in}{0.727386in}}%
\pgfpathlineto{\pgfqpoint{2.031773in}{0.728243in}}%
\pgfpathlineto{\pgfqpoint{2.081838in}{0.758832in}}%
\pgfpathlineto{\pgfqpoint{2.130659in}{0.792288in}}%
\pgfpathlineto{\pgfqpoint{2.178610in}{0.828900in}}%
\pgfpathlineto{\pgfqpoint{2.226125in}{0.869087in}}%
\pgfpathlineto{\pgfqpoint{2.273515in}{0.913246in}}%
\pgfpathlineto{\pgfqpoint{2.321155in}{0.961897in}}%
\pgfpathlineto{\pgfqpoint{2.369416in}{1.015641in}}%
\pgfpathlineto{\pgfqpoint{2.369852in}{1.016146in}}%
\pgfpathlineto{\pgfqpoint{2.371407in}{1.047363in}}%
\pgfpathlineto{\pgfqpoint{2.421720in}{1.110774in}}%
\pgfpathlineto{\pgfqpoint{2.474273in}{1.182050in}}%
\pgfpathlineto{\pgfqpoint{2.530246in}{1.263272in}}%
\pgfpathlineto{\pgfqpoint{2.591630in}{1.357931in}}%
\pgfpathlineto{\pgfqpoint{2.662778in}{1.473549in}}%
\pgfpathlineto{\pgfqpoint{2.709485in}{1.551927in}}%
\pgfpathlineto{\pgfqpoint{2.711040in}{1.603719in}}%
\pgfpathlineto{\pgfqpoint{2.905329in}{1.932122in}}%
\pgfpathlineto{\pgfqpoint{2.975731in}{2.043916in}}%
\pgfpathlineto{\pgfqpoint{3.036742in}{2.135137in}}%
\pgfpathlineto{\pgfqpoint{3.049118in}{2.152906in}}%
\pgfpathlineto{\pgfqpoint{3.050735in}{2.164216in}}%
\pgfpathlineto{\pgfqpoint{3.105341in}{2.238616in}}%
\pgfpathlineto{\pgfqpoint{3.157209in}{2.304147in}}%
\pgfpathlineto{\pgfqpoint{3.207274in}{2.362490in}}%
\pgfpathlineto{\pgfqpoint{3.256220in}{2.414837in}}%
\pgfpathlineto{\pgfqpoint{3.304481in}{2.461976in}}%
\pgfpathlineto{\pgfqpoint{3.352556in}{2.504648in}}%
\pgfpathlineto{\pgfqpoint{3.388752in}{2.534066in}}%
\pgfpathlineto{\pgfqpoint{3.388814in}{2.525917in}}%
\pgfpathlineto{\pgfqpoint{3.390307in}{2.527108in}}%
\pgfpathlineto{\pgfqpoint{3.438755in}{2.563731in}}%
\pgfpathlineto{\pgfqpoint{3.487949in}{2.597059in}}%
\pgfpathlineto{\pgfqpoint{3.538201in}{2.627381in}}%
\pgfpathlineto{\pgfqpoint{3.589945in}{2.654999in}}%
\pgfpathlineto{\pgfqpoint{3.643679in}{2.680170in}}%
\pgfpathlineto{\pgfqpoint{3.699839in}{2.703051in}}%
\pgfpathlineto{\pgfqpoint{3.728447in}{2.713490in}}%
\pgfpathlineto{\pgfqpoint{3.728510in}{2.706993in}}%
\pgfpathlineto{\pgfqpoint{3.730002in}{2.707542in}}%
\pgfpathlineto{\pgfqpoint{3.789956in}{2.727896in}}%
\pgfpathlineto{\pgfqpoint{3.853828in}{2.746289in}}%
\pgfpathlineto{\pgfqpoint{3.922550in}{2.762831in}}%
\pgfpathlineto{\pgfqpoint{3.997368in}{2.777619in}}%
\pgfpathlineto{\pgfqpoint{4.068143in}{2.789038in}}%
\pgfpathlineto{\pgfqpoint{4.068205in}{2.785924in}}%
\pgfpathlineto{\pgfqpoint{4.069698in}{2.786154in}}%
\pgfpathlineto{\pgfqpoint{4.158011in}{2.798191in}}%
\pgfpathlineto{\pgfqpoint{4.257892in}{2.808620in}}%
\pgfpathlineto{\pgfqpoint{4.373260in}{2.817479in}}%
\pgfpathlineto{\pgfqpoint{4.438873in}{2.820180in}}%
\pgfpathlineto{\pgfqpoint{4.585398in}{2.826975in}}%
\pgfpathlineto{\pgfqpoint{4.747596in}{2.831782in}}%
\pgfpathlineto{\pgfqpoint{4.747596in}{2.831782in}}%
\pgfusepath{stroke}%
\end{pgfscope}%
\begin{pgfscope}%
\pgfpathrectangle{\pgfqpoint{0.467954in}{0.402778in}}{\pgfqpoint{4.483434in}{2.548611in}}%
\pgfusepath{clip}%
\pgfsetrectcap%
\pgfsetroundjoin%
\pgfsetlinewidth{1.505625pt}%
\definecolor{currentstroke}{rgb}{0.172549,0.627451,0.172549}%
\pgfsetstrokecolor{currentstroke}%
\pgfsetdash{}{0pt}%
\pgfpathmoveto{\pgfqpoint{0.671747in}{0.518624in}}%
\pgfpathlineto{\pgfqpoint{0.894583in}{0.523933in}}%
\pgfpathlineto{\pgfqpoint{1.060886in}{0.530867in}}%
\pgfpathlineto{\pgfqpoint{1.194289in}{0.539426in}}%
\pgfpathlineto{\pgfqpoint{1.306111in}{0.549616in}}%
\pgfpathlineto{\pgfqpoint{1.402758in}{0.561458in}}%
\pgfpathlineto{\pgfqpoint{1.488148in}{0.574975in}}%
\pgfpathlineto{\pgfqpoint{1.565018in}{0.590224in}}%
\pgfpathlineto{\pgfqpoint{1.635171in}{0.607251in}}%
\pgfpathlineto{\pgfqpoint{1.690585in}{0.623183in}}%
\pgfpathlineto{\pgfqpoint{1.692202in}{0.625135in}}%
\pgfpathlineto{\pgfqpoint{1.752902in}{0.645847in}}%
\pgfpathlineto{\pgfqpoint{1.810119in}{0.668613in}}%
\pgfpathlineto{\pgfqpoint{1.864537in}{0.693574in}}%
\pgfpathlineto{\pgfqpoint{1.916717in}{0.720897in}}%
\pgfpathlineto{\pgfqpoint{1.967155in}{0.750795in}}%
\pgfpathlineto{\pgfqpoint{2.016225in}{0.783475in}}%
\pgfpathlineto{\pgfqpoint{2.030218in}{0.793483in}}%
\pgfpathlineto{\pgfqpoint{2.031835in}{0.800068in}}%
\pgfpathlineto{\pgfqpoint{2.079661in}{0.837436in}}%
\pgfpathlineto{\pgfqpoint{2.127114in}{0.878455in}}%
\pgfpathlineto{\pgfqpoint{2.174505in}{0.923533in}}%
\pgfpathlineto{\pgfqpoint{2.222207in}{0.973203in}}%
\pgfpathlineto{\pgfqpoint{2.270717in}{1.028223in}}%
\pgfpathlineto{\pgfqpoint{2.320471in}{1.089380in}}%
\pgfpathlineto{\pgfqpoint{2.388012in}{1.178118in}}%
\pgfpathlineto{\pgfqpoint{2.443799in}{1.258799in}}%
\pgfpathlineto{\pgfqpoint{2.504809in}{1.352604in}}%
\pgfpathlineto{\pgfqpoint{2.575274in}{1.466830in}}%
\pgfpathlineto{\pgfqpoint{2.670179in}{1.627090in}}%
\pgfpathlineto{\pgfqpoint{2.709485in}{1.694375in}}%
\pgfpathlineto{\pgfqpoint{2.709547in}{1.682284in}}%
\pgfpathlineto{\pgfqpoint{2.711040in}{1.684841in}}%
\pgfpathlineto{\pgfqpoint{2.842142in}{1.906300in}}%
\pgfpathlineto{\pgfqpoint{2.915467in}{2.024001in}}%
\pgfpathlineto{\pgfqpoint{2.977846in}{2.118433in}}%
\pgfpathlineto{\pgfqpoint{3.034565in}{2.198804in}}%
\pgfpathlineto{\pgfqpoint{3.049118in}{2.218508in}}%
\pgfpathlineto{\pgfqpoint{3.049181in}{2.215073in}}%
\pgfpathlineto{\pgfqpoint{3.050673in}{2.217079in}}%
\pgfpathlineto{\pgfqpoint{3.103288in}{2.285131in}}%
\pgfpathlineto{\pgfqpoint{3.153851in}{2.345544in}}%
\pgfpathlineto{\pgfqpoint{3.203045in}{2.399569in}}%
\pgfpathlineto{\pgfqpoint{3.251493in}{2.448232in}}%
\pgfpathlineto{\pgfqpoint{3.299568in}{2.492179in}}%
\pgfpathlineto{\pgfqpoint{3.347705in}{2.532026in}}%
\pgfpathlineto{\pgfqpoint{3.404300in}{2.574546in}}%
\pgfpathlineto{\pgfqpoint{3.453805in}{2.606908in}}%
\pgfpathlineto{\pgfqpoint{3.504492in}{2.636358in}}%
\pgfpathlineto{\pgfqpoint{3.556796in}{2.663175in}}%
\pgfpathlineto{\pgfqpoint{3.611215in}{2.687597in}}%
\pgfpathlineto{\pgfqpoint{3.668245in}{2.709792in}}%
\pgfpathlineto{\pgfqpoint{3.749469in}{2.737480in}}%
\pgfpathlineto{\pgfqpoint{3.815704in}{2.754927in}}%
\pgfpathlineto{\pgfqpoint{3.887349in}{2.770566in}}%
\pgfpathlineto{\pgfqpoint{3.965774in}{2.784478in}}%
\pgfpathlineto{\pgfqpoint{4.052844in}{2.796730in}}%
\pgfpathlineto{\pgfqpoint{4.174741in}{2.810305in}}%
\pgfpathlineto{\pgfqpoint{4.293404in}{2.818889in}}%
\pgfpathlineto{\pgfqpoint{4.460454in}{2.827477in}}%
\pgfpathlineto{\pgfqpoint{4.648275in}{2.832644in}}%
\pgfpathlineto{\pgfqpoint{4.747596in}{2.834424in}}%
\pgfpathlineto{\pgfqpoint{4.747596in}{2.834424in}}%
\pgfusepath{stroke}%
\end{pgfscope}%
\begin{pgfscope}%
\pgfpathrectangle{\pgfqpoint{0.467954in}{0.402778in}}{\pgfqpoint{4.483434in}{2.548611in}}%
\pgfusepath{clip}%
\pgfsetrectcap%
\pgfsetroundjoin%
\pgfsetlinewidth{1.505625pt}%
\definecolor{currentstroke}{rgb}{0.839216,0.152941,0.156863}%
\pgfsetstrokecolor{currentstroke}%
\pgfsetdash{}{0pt}%
\pgfpathmoveto{\pgfqpoint{0.671747in}{0.518624in}}%
\pgfpathlineto{\pgfqpoint{0.894583in}{0.523933in}}%
\pgfpathlineto{\pgfqpoint{1.060886in}{0.530867in}}%
\pgfpathlineto{\pgfqpoint{1.194289in}{0.539426in}}%
\pgfpathlineto{\pgfqpoint{1.306111in}{0.549616in}}%
\pgfpathlineto{\pgfqpoint{1.402758in}{0.561458in}}%
\pgfpathlineto{\pgfqpoint{1.488148in}{0.574975in}}%
\pgfpathlineto{\pgfqpoint{1.565018in}{0.590224in}}%
\pgfpathlineto{\pgfqpoint{1.635171in}{0.607251in}}%
\pgfpathlineto{\pgfqpoint{1.699976in}{0.626125in}}%
\pgfpathlineto{\pgfqpoint{1.760489in}{0.646939in}}%
\pgfpathlineto{\pgfqpoint{1.817582in}{0.669824in}}%
\pgfpathlineto{\pgfqpoint{1.871876in}{0.694902in}}%
\pgfpathlineto{\pgfqpoint{1.923931in}{0.722337in}}%
\pgfpathlineto{\pgfqpoint{1.974307in}{0.752379in}}%
\pgfpathlineto{\pgfqpoint{2.023377in}{0.785247in}}%
\pgfpathlineto{\pgfqpoint{2.075743in}{0.824216in}}%
\pgfpathlineto{\pgfqpoint{2.123258in}{0.863912in}}%
\pgfpathlineto{\pgfqpoint{2.170649in}{0.907557in}}%
\pgfpathlineto{\pgfqpoint{2.218226in}{0.955610in}}%
\pgfpathlineto{\pgfqpoint{2.266426in}{1.008729in}}%
\pgfpathlineto{\pgfqpoint{2.315682in}{1.067662in}}%
\pgfpathlineto{\pgfqpoint{2.366680in}{1.133565in}}%
\pgfpathlineto{\pgfqpoint{2.414755in}{1.200561in}}%
\pgfpathlineto{\pgfqpoint{2.471785in}{1.284590in}}%
\pgfpathlineto{\pgfqpoint{2.534973in}{1.383349in}}%
\pgfpathlineto{\pgfqpoint{2.609853in}{1.506378in}}%
\pgfpathlineto{\pgfqpoint{2.709485in}{1.675788in}}%
\pgfpathlineto{\pgfqpoint{2.711164in}{1.680286in}}%
\pgfpathlineto{\pgfqpoint{2.844816in}{1.906120in}}%
\pgfpathlineto{\pgfqpoint{2.918203in}{2.023927in}}%
\pgfpathlineto{\pgfqpoint{2.980582in}{2.118363in}}%
\pgfpathlineto{\pgfqpoint{3.037302in}{2.198739in}}%
\pgfpathlineto{\pgfqpoint{3.092280in}{2.271291in}}%
\pgfpathlineto{\pgfqpoint{3.143216in}{2.333216in}}%
\pgfpathlineto{\pgfqpoint{3.192659in}{2.388523in}}%
\pgfpathlineto{\pgfqpoint{3.241231in}{2.438269in}}%
\pgfpathlineto{\pgfqpoint{3.289368in}{2.483184in}}%
\pgfpathlineto{\pgfqpoint{3.337443in}{2.523848in}}%
\pgfpathlineto{\pgfqpoint{3.385829in}{2.560755in}}%
\pgfpathlineto{\pgfqpoint{3.434899in}{2.594319in}}%
\pgfpathlineto{\pgfqpoint{3.485088in}{2.624912in}}%
\pgfpathlineto{\pgfqpoint{3.536708in}{2.652760in}}%
\pgfpathlineto{\pgfqpoint{3.590256in}{2.678131in}}%
\pgfpathlineto{\pgfqpoint{3.646167in}{2.701191in}}%
\pgfpathlineto{\pgfqpoint{3.705063in}{2.722124in}}%
\pgfpathlineto{\pgfqpoint{3.767193in}{2.740917in}}%
\pgfpathlineto{\pgfqpoint{3.834361in}{2.758016in}}%
\pgfpathlineto{\pgfqpoint{3.907189in}{2.773326in}}%
\pgfpathlineto{\pgfqpoint{3.987106in}{2.786922in}}%
\pgfpathlineto{\pgfqpoint{4.073803in}{2.798522in}}%
\pgfpathlineto{\pgfqpoint{4.174119in}{2.808908in}}%
\pgfpathlineto{\pgfqpoint{4.290046in}{2.817722in}}%
\pgfpathlineto{\pgfqpoint{4.423822in}{2.824742in}}%
\pgfpathlineto{\pgfqpoint{4.592364in}{2.830562in}}%
\pgfpathlineto{\pgfqpoint{4.747596in}{2.833898in}}%
\pgfpathlineto{\pgfqpoint{4.747596in}{2.833898in}}%
\pgfusepath{stroke}%
\end{pgfscope}%
\begin{pgfscope}%
\pgfsetrectcap%
\pgfsetmiterjoin%
\pgfsetlinewidth{0.803000pt}%
\definecolor{currentstroke}{rgb}{0.000000,0.000000,0.000000}%
\pgfsetstrokecolor{currentstroke}%
\pgfsetdash{}{0pt}%
\pgfpathmoveto{\pgfqpoint{0.467954in}{0.402778in}}%
\pgfpathlineto{\pgfqpoint{0.467954in}{2.951389in}}%
\pgfusepath{stroke}%
\end{pgfscope}%
\begin{pgfscope}%
\pgfsetrectcap%
\pgfsetmiterjoin%
\pgfsetlinewidth{0.803000pt}%
\definecolor{currentstroke}{rgb}{0.000000,0.000000,0.000000}%
\pgfsetstrokecolor{currentstroke}%
\pgfsetdash{}{0pt}%
\pgfpathmoveto{\pgfqpoint{4.951389in}{0.402778in}}%
\pgfpathlineto{\pgfqpoint{4.951389in}{2.951389in}}%
\pgfusepath{stroke}%
\end{pgfscope}%
\begin{pgfscope}%
\pgfsetrectcap%
\pgfsetmiterjoin%
\pgfsetlinewidth{0.803000pt}%
\definecolor{currentstroke}{rgb}{0.000000,0.000000,0.000000}%
\pgfsetstrokecolor{currentstroke}%
\pgfsetdash{}{0pt}%
\pgfpathmoveto{\pgfqpoint{0.467954in}{0.402778in}}%
\pgfpathlineto{\pgfqpoint{4.951389in}{0.402778in}}%
\pgfusepath{stroke}%
\end{pgfscope}%
\begin{pgfscope}%
\pgfsetrectcap%
\pgfsetmiterjoin%
\pgfsetlinewidth{0.803000pt}%
\definecolor{currentstroke}{rgb}{0.000000,0.000000,0.000000}%
\pgfsetstrokecolor{currentstroke}%
\pgfsetdash{}{0pt}%
\pgfpathmoveto{\pgfqpoint{0.467954in}{2.951389in}}%
\pgfpathlineto{\pgfqpoint{4.951389in}{2.951389in}}%
\pgfusepath{stroke}%
\end{pgfscope}%
\begin{pgfscope}%
\pgfsetbuttcap%
\pgfsetmiterjoin%
\definecolor{currentfill}{rgb}{1.000000,1.000000,1.000000}%
\pgfsetfillcolor{currentfill}%
\pgfsetfillopacity{0.800000}%
\pgfsetlinewidth{1.003750pt}%
\definecolor{currentstroke}{rgb}{0.800000,0.800000,0.800000}%
\pgfsetstrokecolor{currentstroke}%
\pgfsetstrokeopacity{0.800000}%
\pgfsetdash{}{0pt}%
\pgfpathmoveto{\pgfqpoint{0.565177in}{2.054797in}}%
\pgfpathlineto{\pgfqpoint{1.079066in}{2.054797in}}%
\pgfpathquadraticcurveto{\pgfqpoint{1.106843in}{2.054797in}}{\pgfqpoint{1.106843in}{2.082574in}}%
\pgfpathlineto{\pgfqpoint{1.106843in}{2.854167in}}%
\pgfpathquadraticcurveto{\pgfqpoint{1.106843in}{2.881944in}}{\pgfqpoint{1.079066in}{2.881944in}}%
\pgfpathlineto{\pgfqpoint{0.565177in}{2.881944in}}%
\pgfpathquadraticcurveto{\pgfqpoint{0.537399in}{2.881944in}}{\pgfqpoint{0.537399in}{2.854167in}}%
\pgfpathlineto{\pgfqpoint{0.537399in}{2.082574in}}%
\pgfpathquadraticcurveto{\pgfqpoint{0.537399in}{2.054797in}}{\pgfqpoint{0.565177in}{2.054797in}}%
\pgfpathclose%
\pgfusepath{stroke,fill}%
\end{pgfscope}%
\begin{pgfscope}%
\pgfsetrectcap%
\pgfsetroundjoin%
\pgfsetlinewidth{1.505625pt}%
\definecolor{currentstroke}{rgb}{0.121569,0.466667,0.705882}%
\pgfsetstrokecolor{currentstroke}%
\pgfsetdash{}{0pt}%
\pgfpathmoveto{\pgfqpoint{0.592954in}{2.777778in}}%
\pgfpathlineto{\pgfqpoint{0.870732in}{2.777778in}}%
\pgfusepath{stroke}%
\end{pgfscope}%
\begin{pgfscope}%
\pgftext[x=0.981843in,y=2.729167in,left,base]{\rmfamily\fontsize{10.000000}{12.000000}\selectfont 1}%
\end{pgfscope}%
\begin{pgfscope}%
\pgfsetrectcap%
\pgfsetroundjoin%
\pgfsetlinewidth{1.505625pt}%
\definecolor{currentstroke}{rgb}{1.000000,0.498039,0.054902}%
\pgfsetstrokecolor{currentstroke}%
\pgfsetdash{}{0pt}%
\pgfpathmoveto{\pgfqpoint{0.592954in}{2.581407in}}%
\pgfpathlineto{\pgfqpoint{0.870732in}{2.581407in}}%
\pgfusepath{stroke}%
\end{pgfscope}%
\begin{pgfscope}%
\pgftext[x=0.981843in,y=2.532796in,left,base]{\rmfamily\fontsize{10.000000}{12.000000}\selectfont 2}%
\end{pgfscope}%
\begin{pgfscope}%
\pgfsetrectcap%
\pgfsetroundjoin%
\pgfsetlinewidth{1.505625pt}%
\definecolor{currentstroke}{rgb}{0.172549,0.627451,0.172549}%
\pgfsetstrokecolor{currentstroke}%
\pgfsetdash{}{0pt}%
\pgfpathmoveto{\pgfqpoint{0.592954in}{2.385037in}}%
\pgfpathlineto{\pgfqpoint{0.870732in}{2.385037in}}%
\pgfusepath{stroke}%
\end{pgfscope}%
\begin{pgfscope}%
\pgftext[x=0.981843in,y=2.336426in,left,base]{\rmfamily\fontsize{10.000000}{12.000000}\selectfont 3}%
\end{pgfscope}%
\begin{pgfscope}%
\pgfsetrectcap%
\pgfsetroundjoin%
\pgfsetlinewidth{1.505625pt}%
\definecolor{currentstroke}{rgb}{0.839216,0.152941,0.156863}%
\pgfsetstrokecolor{currentstroke}%
\pgfsetdash{}{0pt}%
\pgfpathmoveto{\pgfqpoint{0.592954in}{2.188667in}}%
\pgfpathlineto{\pgfqpoint{0.870732in}{2.188667in}}%
\pgfusepath{stroke}%
\end{pgfscope}%
\begin{pgfscope}%
\pgftext[x=0.981843in,y=2.140056in,left,base]{\rmfamily\fontsize{10.000000}{12.000000}\selectfont 4}%
\end{pgfscope}%
\end{pgfpicture}%
\makeatother%
\endgroup%

        %% Creator: Matplotlib, PGF backend
%%
%% To include the figure in your LaTeX document, write
%%   \input{<filename>.pgf}
%%
%% Make sure the required packages are loaded in your preamble
%%   \usepackage{pgf}
%%
%% Figures using additional raster images can only be included by \input if
%% they are in the same directory as the main LaTeX file. For loading figures
%% from other directories you can use the `import` package
%%   \usepackage{import}
%% and then include the figures with
%%   \import{<path to file>}{<filename>.pgf}
%%
%% Matplotlib used the following preamble
%%   \usepackage{fontspec}
%%   \setmainfont{Times New Roman}
%%   \setsansfont{Lucida Grande}
%%   \setmonofont{Andale Mono}
%%
\begingroup%
\makeatletter%
\begin{pgfpicture}%
\pgfpathrectangle{\pgfpointorigin}{\pgfqpoint{5.000000in}{3.000000in}}%
\pgfusepath{use as bounding box, clip}%
\begin{pgfscope}%
\pgfsetbuttcap%
\pgfsetmiterjoin%
\definecolor{currentfill}{rgb}{1.000000,1.000000,1.000000}%
\pgfsetfillcolor{currentfill}%
\pgfsetlinewidth{0.000000pt}%
\definecolor{currentstroke}{rgb}{1.000000,1.000000,1.000000}%
\pgfsetstrokecolor{currentstroke}%
\pgfsetdash{}{0pt}%
\pgfpathmoveto{\pgfqpoint{0.000000in}{0.000000in}}%
\pgfpathlineto{\pgfqpoint{5.000000in}{0.000000in}}%
\pgfpathlineto{\pgfqpoint{5.000000in}{3.000000in}}%
\pgfpathlineto{\pgfqpoint{0.000000in}{3.000000in}}%
\pgfpathclose%
\pgfusepath{fill}%
\end{pgfscope}%
\begin{pgfscope}%
\pgfsetbuttcap%
\pgfsetmiterjoin%
\definecolor{currentfill}{rgb}{1.000000,1.000000,1.000000}%
\pgfsetfillcolor{currentfill}%
\pgfsetlinewidth{0.000000pt}%
\definecolor{currentstroke}{rgb}{0.000000,0.000000,0.000000}%
\pgfsetstrokecolor{currentstroke}%
\pgfsetstrokeopacity{0.000000}%
\pgfsetdash{}{0pt}%
\pgfpathmoveto{\pgfqpoint{0.467954in}{0.402778in}}%
\pgfpathlineto{\pgfqpoint{4.951389in}{0.402778in}}%
\pgfpathlineto{\pgfqpoint{4.951389in}{2.951389in}}%
\pgfpathlineto{\pgfqpoint{0.467954in}{2.951389in}}%
\pgfpathclose%
\pgfusepath{fill}%
\end{pgfscope}%
\begin{pgfscope}%
\pgfsetbuttcap%
\pgfsetroundjoin%
\definecolor{currentfill}{rgb}{0.000000,0.000000,0.000000}%
\pgfsetfillcolor{currentfill}%
\pgfsetlinewidth{0.803000pt}%
\definecolor{currentstroke}{rgb}{0.000000,0.000000,0.000000}%
\pgfsetstrokecolor{currentstroke}%
\pgfsetdash{}{0pt}%
\pgfsys@defobject{currentmarker}{\pgfqpoint{0.000000in}{-0.048611in}}{\pgfqpoint{0.000000in}{0.000000in}}{%
\pgfpathmoveto{\pgfqpoint{0.000000in}{0.000000in}}%
\pgfpathlineto{\pgfqpoint{0.000000in}{-0.048611in}}%
\pgfusepath{stroke,fill}%
}%
\begin{pgfscope}%
\pgfsys@transformshift{0.671747in}{0.402778in}%
\pgfsys@useobject{currentmarker}{}%
\end{pgfscope}%
\end{pgfscope}%
\begin{pgfscope}%
\pgftext[x=0.671747in,y=0.305556in,,top]{\rmfamily\fontsize{10.000000}{12.000000}\selectfont \(\displaystyle 0\)}%
\end{pgfscope}%
\begin{pgfscope}%
\pgfsetbuttcap%
\pgfsetroundjoin%
\definecolor{currentfill}{rgb}{0.000000,0.000000,0.000000}%
\pgfsetfillcolor{currentfill}%
\pgfsetlinewidth{0.803000pt}%
\definecolor{currentstroke}{rgb}{0.000000,0.000000,0.000000}%
\pgfsetstrokecolor{currentstroke}%
\pgfsetdash{}{0pt}%
\pgfsys@defobject{currentmarker}{\pgfqpoint{0.000000in}{-0.048611in}}{\pgfqpoint{0.000000in}{0.000000in}}{%
\pgfpathmoveto{\pgfqpoint{0.000000in}{0.000000in}}%
\pgfpathlineto{\pgfqpoint{0.000000in}{-0.048611in}}%
\pgfusepath{stroke,fill}%
}%
\begin{pgfscope}%
\pgfsys@transformshift{1.254011in}{0.402778in}%
\pgfsys@useobject{currentmarker}{}%
\end{pgfscope}%
\end{pgfscope}%
\begin{pgfscope}%
\pgftext[x=1.254011in,y=0.305556in,,top]{\rmfamily\fontsize{10.000000}{12.000000}\selectfont \(\displaystyle 1\)}%
\end{pgfscope}%
\begin{pgfscope}%
\pgfsetbuttcap%
\pgfsetroundjoin%
\definecolor{currentfill}{rgb}{0.000000,0.000000,0.000000}%
\pgfsetfillcolor{currentfill}%
\pgfsetlinewidth{0.803000pt}%
\definecolor{currentstroke}{rgb}{0.000000,0.000000,0.000000}%
\pgfsetstrokecolor{currentstroke}%
\pgfsetdash{}{0pt}%
\pgfsys@defobject{currentmarker}{\pgfqpoint{0.000000in}{-0.048611in}}{\pgfqpoint{0.000000in}{0.000000in}}{%
\pgfpathmoveto{\pgfqpoint{0.000000in}{0.000000in}}%
\pgfpathlineto{\pgfqpoint{0.000000in}{-0.048611in}}%
\pgfusepath{stroke,fill}%
}%
\begin{pgfscope}%
\pgfsys@transformshift{1.836275in}{0.402778in}%
\pgfsys@useobject{currentmarker}{}%
\end{pgfscope}%
\end{pgfscope}%
\begin{pgfscope}%
\pgftext[x=1.836275in,y=0.305556in,,top]{\rmfamily\fontsize{10.000000}{12.000000}\selectfont \(\displaystyle 2\)}%
\end{pgfscope}%
\begin{pgfscope}%
\pgfsetbuttcap%
\pgfsetroundjoin%
\definecolor{currentfill}{rgb}{0.000000,0.000000,0.000000}%
\pgfsetfillcolor{currentfill}%
\pgfsetlinewidth{0.803000pt}%
\definecolor{currentstroke}{rgb}{0.000000,0.000000,0.000000}%
\pgfsetstrokecolor{currentstroke}%
\pgfsetdash{}{0pt}%
\pgfsys@defobject{currentmarker}{\pgfqpoint{0.000000in}{-0.048611in}}{\pgfqpoint{0.000000in}{0.000000in}}{%
\pgfpathmoveto{\pgfqpoint{0.000000in}{0.000000in}}%
\pgfpathlineto{\pgfqpoint{0.000000in}{-0.048611in}}%
\pgfusepath{stroke,fill}%
}%
\begin{pgfscope}%
\pgfsys@transformshift{2.418540in}{0.402778in}%
\pgfsys@useobject{currentmarker}{}%
\end{pgfscope}%
\end{pgfscope}%
\begin{pgfscope}%
\pgftext[x=2.418540in,y=0.305556in,,top]{\rmfamily\fontsize{10.000000}{12.000000}\selectfont \(\displaystyle 3\)}%
\end{pgfscope}%
\begin{pgfscope}%
\pgfsetbuttcap%
\pgfsetroundjoin%
\definecolor{currentfill}{rgb}{0.000000,0.000000,0.000000}%
\pgfsetfillcolor{currentfill}%
\pgfsetlinewidth{0.803000pt}%
\definecolor{currentstroke}{rgb}{0.000000,0.000000,0.000000}%
\pgfsetstrokecolor{currentstroke}%
\pgfsetdash{}{0pt}%
\pgfsys@defobject{currentmarker}{\pgfqpoint{0.000000in}{-0.048611in}}{\pgfqpoint{0.000000in}{0.000000in}}{%
\pgfpathmoveto{\pgfqpoint{0.000000in}{0.000000in}}%
\pgfpathlineto{\pgfqpoint{0.000000in}{-0.048611in}}%
\pgfusepath{stroke,fill}%
}%
\begin{pgfscope}%
\pgfsys@transformshift{3.000804in}{0.402778in}%
\pgfsys@useobject{currentmarker}{}%
\end{pgfscope}%
\end{pgfscope}%
\begin{pgfscope}%
\pgftext[x=3.000804in,y=0.305556in,,top]{\rmfamily\fontsize{10.000000}{12.000000}\selectfont \(\displaystyle 4\)}%
\end{pgfscope}%
\begin{pgfscope}%
\pgfsetbuttcap%
\pgfsetroundjoin%
\definecolor{currentfill}{rgb}{0.000000,0.000000,0.000000}%
\pgfsetfillcolor{currentfill}%
\pgfsetlinewidth{0.803000pt}%
\definecolor{currentstroke}{rgb}{0.000000,0.000000,0.000000}%
\pgfsetstrokecolor{currentstroke}%
\pgfsetdash{}{0pt}%
\pgfsys@defobject{currentmarker}{\pgfqpoint{0.000000in}{-0.048611in}}{\pgfqpoint{0.000000in}{0.000000in}}{%
\pgfpathmoveto{\pgfqpoint{0.000000in}{0.000000in}}%
\pgfpathlineto{\pgfqpoint{0.000000in}{-0.048611in}}%
\pgfusepath{stroke,fill}%
}%
\begin{pgfscope}%
\pgfsys@transformshift{3.583068in}{0.402778in}%
\pgfsys@useobject{currentmarker}{}%
\end{pgfscope}%
\end{pgfscope}%
\begin{pgfscope}%
\pgftext[x=3.583068in,y=0.305556in,,top]{\rmfamily\fontsize{10.000000}{12.000000}\selectfont \(\displaystyle 5\)}%
\end{pgfscope}%
\begin{pgfscope}%
\pgfsetbuttcap%
\pgfsetroundjoin%
\definecolor{currentfill}{rgb}{0.000000,0.000000,0.000000}%
\pgfsetfillcolor{currentfill}%
\pgfsetlinewidth{0.803000pt}%
\definecolor{currentstroke}{rgb}{0.000000,0.000000,0.000000}%
\pgfsetstrokecolor{currentstroke}%
\pgfsetdash{}{0pt}%
\pgfsys@defobject{currentmarker}{\pgfqpoint{0.000000in}{-0.048611in}}{\pgfqpoint{0.000000in}{0.000000in}}{%
\pgfpathmoveto{\pgfqpoint{0.000000in}{0.000000in}}%
\pgfpathlineto{\pgfqpoint{0.000000in}{-0.048611in}}%
\pgfusepath{stroke,fill}%
}%
\begin{pgfscope}%
\pgfsys@transformshift{4.165332in}{0.402778in}%
\pgfsys@useobject{currentmarker}{}%
\end{pgfscope}%
\end{pgfscope}%
\begin{pgfscope}%
\pgftext[x=4.165332in,y=0.305556in,,top]{\rmfamily\fontsize{10.000000}{12.000000}\selectfont \(\displaystyle 6\)}%
\end{pgfscope}%
\begin{pgfscope}%
\pgfsetbuttcap%
\pgfsetroundjoin%
\definecolor{currentfill}{rgb}{0.000000,0.000000,0.000000}%
\pgfsetfillcolor{currentfill}%
\pgfsetlinewidth{0.803000pt}%
\definecolor{currentstroke}{rgb}{0.000000,0.000000,0.000000}%
\pgfsetstrokecolor{currentstroke}%
\pgfsetdash{}{0pt}%
\pgfsys@defobject{currentmarker}{\pgfqpoint{0.000000in}{-0.048611in}}{\pgfqpoint{0.000000in}{0.000000in}}{%
\pgfpathmoveto{\pgfqpoint{0.000000in}{0.000000in}}%
\pgfpathlineto{\pgfqpoint{0.000000in}{-0.048611in}}%
\pgfusepath{stroke,fill}%
}%
\begin{pgfscope}%
\pgfsys@transformshift{4.747596in}{0.402778in}%
\pgfsys@useobject{currentmarker}{}%
\end{pgfscope}%
\end{pgfscope}%
\begin{pgfscope}%
\pgftext[x=4.747596in,y=0.305556in,,top]{\rmfamily\fontsize{10.000000}{12.000000}\selectfont \(\displaystyle 7\)}%
\end{pgfscope}%
\begin{pgfscope}%
\pgftext[x=2.709672in,y=0.123861in,,top]{\rmfamily\fontsize{10.000000}{12.000000}\selectfont t}%
\end{pgfscope}%
\begin{pgfscope}%
\pgftext[x=4.951389in,y=0.137750in,right,top]{\rmfamily\fontsize{10.000000}{12.000000}\selectfont \(\displaystyle \times10^{-7}\)}%
\end{pgfscope}%
\begin{pgfscope}%
\pgfsetbuttcap%
\pgfsetroundjoin%
\definecolor{currentfill}{rgb}{0.000000,0.000000,0.000000}%
\pgfsetfillcolor{currentfill}%
\pgfsetlinewidth{0.803000pt}%
\definecolor{currentstroke}{rgb}{0.000000,0.000000,0.000000}%
\pgfsetstrokecolor{currentstroke}%
\pgfsetdash{}{0pt}%
\pgfsys@defobject{currentmarker}{\pgfqpoint{-0.048611in}{0.000000in}}{\pgfqpoint{0.000000in}{0.000000in}}{%
\pgfpathmoveto{\pgfqpoint{0.000000in}{0.000000in}}%
\pgfpathlineto{\pgfqpoint{-0.048611in}{0.000000in}}%
\pgfusepath{stroke,fill}%
}%
\begin{pgfscope}%
\pgfsys@transformshift{0.467954in}{0.518223in}%
\pgfsys@useobject{currentmarker}{}%
\end{pgfscope}%
\end{pgfscope}%
\begin{pgfscope}%
\pgftext[x=0.193262in,y=0.470005in,left,base]{\rmfamily\fontsize{10.000000}{12.000000}\selectfont \(\displaystyle 0.0\)}%
\end{pgfscope}%
\begin{pgfscope}%
\pgfsetbuttcap%
\pgfsetroundjoin%
\definecolor{currentfill}{rgb}{0.000000,0.000000,0.000000}%
\pgfsetfillcolor{currentfill}%
\pgfsetlinewidth{0.803000pt}%
\definecolor{currentstroke}{rgb}{0.000000,0.000000,0.000000}%
\pgfsetstrokecolor{currentstroke}%
\pgfsetdash{}{0pt}%
\pgfsys@defobject{currentmarker}{\pgfqpoint{-0.048611in}{0.000000in}}{\pgfqpoint{0.000000in}{0.000000in}}{%
\pgfpathmoveto{\pgfqpoint{0.000000in}{0.000000in}}%
\pgfpathlineto{\pgfqpoint{-0.048611in}{0.000000in}}%
\pgfusepath{stroke,fill}%
}%
\begin{pgfscope}%
\pgfsys@transformshift{0.467954in}{0.981687in}%
\pgfsys@useobject{currentmarker}{}%
\end{pgfscope}%
\end{pgfscope}%
\begin{pgfscope}%
\pgftext[x=0.193262in,y=0.933469in,left,base]{\rmfamily\fontsize{10.000000}{12.000000}\selectfont \(\displaystyle 0.2\)}%
\end{pgfscope}%
\begin{pgfscope}%
\pgfsetbuttcap%
\pgfsetroundjoin%
\definecolor{currentfill}{rgb}{0.000000,0.000000,0.000000}%
\pgfsetfillcolor{currentfill}%
\pgfsetlinewidth{0.803000pt}%
\definecolor{currentstroke}{rgb}{0.000000,0.000000,0.000000}%
\pgfsetstrokecolor{currentstroke}%
\pgfsetdash{}{0pt}%
\pgfsys@defobject{currentmarker}{\pgfqpoint{-0.048611in}{0.000000in}}{\pgfqpoint{0.000000in}{0.000000in}}{%
\pgfpathmoveto{\pgfqpoint{0.000000in}{0.000000in}}%
\pgfpathlineto{\pgfqpoint{-0.048611in}{0.000000in}}%
\pgfusepath{stroke,fill}%
}%
\begin{pgfscope}%
\pgfsys@transformshift{0.467954in}{1.445151in}%
\pgfsys@useobject{currentmarker}{}%
\end{pgfscope}%
\end{pgfscope}%
\begin{pgfscope}%
\pgftext[x=0.193262in,y=1.396933in,left,base]{\rmfamily\fontsize{10.000000}{12.000000}\selectfont \(\displaystyle 0.4\)}%
\end{pgfscope}%
\begin{pgfscope}%
\pgfsetbuttcap%
\pgfsetroundjoin%
\definecolor{currentfill}{rgb}{0.000000,0.000000,0.000000}%
\pgfsetfillcolor{currentfill}%
\pgfsetlinewidth{0.803000pt}%
\definecolor{currentstroke}{rgb}{0.000000,0.000000,0.000000}%
\pgfsetstrokecolor{currentstroke}%
\pgfsetdash{}{0pt}%
\pgfsys@defobject{currentmarker}{\pgfqpoint{-0.048611in}{0.000000in}}{\pgfqpoint{0.000000in}{0.000000in}}{%
\pgfpathmoveto{\pgfqpoint{0.000000in}{0.000000in}}%
\pgfpathlineto{\pgfqpoint{-0.048611in}{0.000000in}}%
\pgfusepath{stroke,fill}%
}%
\begin{pgfscope}%
\pgfsys@transformshift{0.467954in}{1.908615in}%
\pgfsys@useobject{currentmarker}{}%
\end{pgfscope}%
\end{pgfscope}%
\begin{pgfscope}%
\pgftext[x=0.193262in,y=1.860397in,left,base]{\rmfamily\fontsize{10.000000}{12.000000}\selectfont \(\displaystyle 0.6\)}%
\end{pgfscope}%
\begin{pgfscope}%
\pgfsetbuttcap%
\pgfsetroundjoin%
\definecolor{currentfill}{rgb}{0.000000,0.000000,0.000000}%
\pgfsetfillcolor{currentfill}%
\pgfsetlinewidth{0.803000pt}%
\definecolor{currentstroke}{rgb}{0.000000,0.000000,0.000000}%
\pgfsetstrokecolor{currentstroke}%
\pgfsetdash{}{0pt}%
\pgfsys@defobject{currentmarker}{\pgfqpoint{-0.048611in}{0.000000in}}{\pgfqpoint{0.000000in}{0.000000in}}{%
\pgfpathmoveto{\pgfqpoint{0.000000in}{0.000000in}}%
\pgfpathlineto{\pgfqpoint{-0.048611in}{0.000000in}}%
\pgfusepath{stroke,fill}%
}%
\begin{pgfscope}%
\pgfsys@transformshift{0.467954in}{2.372079in}%
\pgfsys@useobject{currentmarker}{}%
\end{pgfscope}%
\end{pgfscope}%
\begin{pgfscope}%
\pgftext[x=0.193262in,y=2.323861in,left,base]{\rmfamily\fontsize{10.000000}{12.000000}\selectfont \(\displaystyle 0.8\)}%
\end{pgfscope}%
\begin{pgfscope}%
\pgfsetbuttcap%
\pgfsetroundjoin%
\definecolor{currentfill}{rgb}{0.000000,0.000000,0.000000}%
\pgfsetfillcolor{currentfill}%
\pgfsetlinewidth{0.803000pt}%
\definecolor{currentstroke}{rgb}{0.000000,0.000000,0.000000}%
\pgfsetstrokecolor{currentstroke}%
\pgfsetdash{}{0pt}%
\pgfsys@defobject{currentmarker}{\pgfqpoint{-0.048611in}{0.000000in}}{\pgfqpoint{0.000000in}{0.000000in}}{%
\pgfpathmoveto{\pgfqpoint{0.000000in}{0.000000in}}%
\pgfpathlineto{\pgfqpoint{-0.048611in}{0.000000in}}%
\pgfusepath{stroke,fill}%
}%
\begin{pgfscope}%
\pgfsys@transformshift{0.467954in}{2.835543in}%
\pgfsys@useobject{currentmarker}{}%
\end{pgfscope}%
\end{pgfscope}%
\begin{pgfscope}%
\pgftext[x=0.193262in,y=2.787325in,left,base]{\rmfamily\fontsize{10.000000}{12.000000}\selectfont \(\displaystyle 1.0\)}%
\end{pgfscope}%
\begin{pgfscope}%
\pgftext[x=0.137707in,y=1.677083in,,bottom,rotate=90.000000]{\rmfamily\fontsize{10.000000}{12.000000}\selectfont y(t)}%
\end{pgfscope}%
\begin{pgfscope}%
\pgfpathrectangle{\pgfqpoint{0.467954in}{0.402778in}}{\pgfqpoint{4.483434in}{2.548611in}}%
\pgfusepath{clip}%
\pgfsetrectcap%
\pgfsetroundjoin%
\pgfsetlinewidth{1.505625pt}%
\definecolor{currentstroke}{rgb}{0.121569,0.466667,0.705882}%
\pgfsetstrokecolor{currentstroke}%
\pgfsetdash{}{0pt}%
\pgfpathmoveto{\pgfqpoint{0.671747in}{2.835543in}}%
\pgfpathlineto{\pgfqpoint{0.711550in}{2.681405in}}%
\pgfpathlineto{\pgfqpoint{0.751353in}{2.537520in}}%
\pgfpathlineto{\pgfqpoint{0.799117in}{2.377434in}}%
\pgfpathlineto{\pgfqpoint{0.846881in}{2.230039in}}%
\pgfpathlineto{\pgfqpoint{0.894645in}{2.094330in}}%
\pgfpathlineto{\pgfqpoint{0.942409in}{1.969379in}}%
\pgfpathlineto{\pgfqpoint{0.990173in}{1.854334in}}%
\pgfpathlineto{\pgfqpoint{1.037937in}{1.748410in}}%
\pgfpathlineto{\pgfqpoint{1.085700in}{1.650883in}}%
\pgfpathlineto{\pgfqpoint{1.109582in}{1.605059in}}%
\pgfpathlineto{\pgfqpoint{1.117543in}{1.400937in}}%
\pgfpathlineto{\pgfqpoint{1.165307in}{1.330957in}}%
\pgfpathlineto{\pgfqpoint{1.213071in}{1.266525in}}%
\pgfpathlineto{\pgfqpoint{1.260835in}{1.207201in}}%
\pgfpathlineto{\pgfqpoint{1.308598in}{1.152581in}}%
\pgfpathlineto{\pgfqpoint{1.356362in}{1.102290in}}%
\pgfpathlineto{\pgfqpoint{1.404126in}{1.055986in}}%
\pgfpathlineto{\pgfqpoint{1.459851in}{1.006584in}}%
\pgfpathlineto{\pgfqpoint{1.515575in}{0.961720in}}%
\pgfpathlineto{\pgfqpoint{1.563339in}{0.926561in}}%
\pgfpathlineto{\pgfqpoint{1.571300in}{0.847060in}}%
\pgfpathlineto{\pgfqpoint{1.634985in}{0.812768in}}%
\pgfpathlineto{\pgfqpoint{1.698670in}{0.782052in}}%
\pgfpathlineto{\pgfqpoint{1.762355in}{0.754539in}}%
\pgfpathlineto{\pgfqpoint{1.834001in}{0.727002in}}%
\pgfpathlineto{\pgfqpoint{1.905647in}{0.702673in}}%
\pgfpathlineto{\pgfqpoint{1.985253in}{0.678952in}}%
\pgfpathlineto{\pgfqpoint{2.017096in}{0.670340in}}%
\pgfpathlineto{\pgfqpoint{2.025056in}{0.640724in}}%
\pgfpathlineto{\pgfqpoint{2.120584in}{0.622071in}}%
\pgfpathlineto{\pgfqpoint{2.224072in}{0.605054in}}%
\pgfpathlineto{\pgfqpoint{2.335521in}{0.589833in}}%
\pgfpathlineto{\pgfqpoint{2.462892in}{0.575676in}}%
\pgfpathlineto{\pgfqpoint{2.470852in}{0.574891in}}%
\pgfpathlineto{\pgfqpoint{2.478813in}{0.563858in}}%
\pgfpathlineto{\pgfqpoint{2.638026in}{0.552875in}}%
\pgfpathlineto{\pgfqpoint{2.821121in}{0.543470in}}%
\pgfpathlineto{\pgfqpoint{2.924609in}{0.539333in}}%
\pgfpathlineto{\pgfqpoint{2.932570in}{0.535223in}}%
\pgfpathlineto{\pgfqpoint{3.195271in}{0.529016in}}%
\pgfpathlineto{\pgfqpoint{3.633106in}{0.522356in}}%
\pgfpathlineto{\pgfqpoint{4.269958in}{0.519345in}}%
\pgfpathlineto{\pgfqpoint{4.747596in}{0.518624in}}%
\pgfpathlineto{\pgfqpoint{4.747596in}{0.518624in}}%
\pgfusepath{stroke}%
\end{pgfscope}%
\begin{pgfscope}%
\pgfpathrectangle{\pgfqpoint{0.467954in}{0.402778in}}{\pgfqpoint{4.483434in}{2.548611in}}%
\pgfusepath{clip}%
\pgfsetrectcap%
\pgfsetroundjoin%
\pgfsetlinewidth{1.505625pt}%
\definecolor{currentstroke}{rgb}{1.000000,0.498039,0.054902}%
\pgfsetstrokecolor{currentstroke}%
\pgfsetdash{}{0pt}%
\pgfpathmoveto{\pgfqpoint{0.671747in}{2.835543in}}%
\pgfpathlineto{\pgfqpoint{0.711550in}{2.681405in}}%
\pgfpathlineto{\pgfqpoint{0.751353in}{2.537520in}}%
\pgfpathlineto{\pgfqpoint{0.799117in}{2.377434in}}%
\pgfpathlineto{\pgfqpoint{0.846881in}{2.230039in}}%
\pgfpathlineto{\pgfqpoint{0.894645in}{2.094330in}}%
\pgfpathlineto{\pgfqpoint{0.942409in}{1.969379in}}%
\pgfpathlineto{\pgfqpoint{0.990173in}{1.854334in}}%
\pgfpathlineto{\pgfqpoint{1.037937in}{1.748410in}}%
\pgfpathlineto{\pgfqpoint{1.085700in}{1.650883in}}%
\pgfpathlineto{\pgfqpoint{1.133464in}{1.561088in}}%
\pgfpathlineto{\pgfqpoint{1.181228in}{1.478412in}}%
\pgfpathlineto{\pgfqpoint{1.228992in}{1.402290in}}%
\pgfpathlineto{\pgfqpoint{1.276756in}{1.332203in}}%
\pgfpathlineto{\pgfqpoint{1.324520in}{1.267672in}}%
\pgfpathlineto{\pgfqpoint{1.372284in}{1.208257in}}%
\pgfpathlineto{\pgfqpoint{1.420047in}{1.153553in}}%
\pgfpathlineto{\pgfqpoint{1.467811in}{1.103185in}}%
\pgfpathlineto{\pgfqpoint{1.515575in}{1.056810in}}%
\pgfpathlineto{\pgfqpoint{1.563339in}{1.014112in}}%
\pgfpathlineto{\pgfqpoint{1.571300in}{0.991484in}}%
\pgfpathlineto{\pgfqpoint{1.627024in}{0.948007in}}%
\pgfpathlineto{\pgfqpoint{1.682749in}{0.908524in}}%
\pgfpathlineto{\pgfqpoint{1.738473in}{0.872669in}}%
\pgfpathlineto{\pgfqpoint{1.802158in}{0.835707in}}%
\pgfpathlineto{\pgfqpoint{1.865843in}{0.802599in}}%
\pgfpathlineto{\pgfqpoint{1.929529in}{0.772943in}}%
\pgfpathlineto{\pgfqpoint{2.001174in}{0.743261in}}%
\pgfpathlineto{\pgfqpoint{2.017096in}{0.737150in}}%
\pgfpathlineto{\pgfqpoint{2.025056in}{0.722063in}}%
\pgfpathlineto{\pgfqpoint{2.096702in}{0.698310in}}%
\pgfpathlineto{\pgfqpoint{2.176309in}{0.675149in}}%
\pgfpathlineto{\pgfqpoint{2.263876in}{0.653098in}}%
\pgfpathlineto{\pgfqpoint{2.351443in}{0.634145in}}%
\pgfpathlineto{\pgfqpoint{2.446970in}{0.616494in}}%
\pgfpathlineto{\pgfqpoint{2.470852in}{0.612518in}}%
\pgfpathlineto{\pgfqpoint{2.478813in}{0.604417in}}%
\pgfpathlineto{\pgfqpoint{2.590262in}{0.589308in}}%
\pgfpathlineto{\pgfqpoint{2.717632in}{0.575255in}}%
\pgfpathlineto{\pgfqpoint{2.860924in}{0.562738in}}%
\pgfpathlineto{\pgfqpoint{2.924609in}{0.558096in}}%
\pgfpathlineto{\pgfqpoint{2.932570in}{0.554154in}}%
\pgfpathlineto{\pgfqpoint{3.107704in}{0.544765in}}%
\pgfpathlineto{\pgfqpoint{3.314681in}{0.536780in}}%
\pgfpathlineto{\pgfqpoint{3.649028in}{0.527625in}}%
\pgfpathlineto{\pgfqpoint{4.078902in}{0.522224in}}%
\pgfpathlineto{\pgfqpoint{4.707793in}{0.519420in}}%
\pgfpathlineto{\pgfqpoint{4.747596in}{0.519341in}}%
\pgfpathlineto{\pgfqpoint{4.747596in}{0.519341in}}%
\pgfusepath{stroke}%
\end{pgfscope}%
\begin{pgfscope}%
\pgfpathrectangle{\pgfqpoint{0.467954in}{0.402778in}}{\pgfqpoint{4.483434in}{2.548611in}}%
\pgfusepath{clip}%
\pgfsetrectcap%
\pgfsetroundjoin%
\pgfsetlinewidth{1.505625pt}%
\definecolor{currentstroke}{rgb}{0.172549,0.627451,0.172549}%
\pgfsetstrokecolor{currentstroke}%
\pgfsetdash{}{0pt}%
\pgfpathmoveto{\pgfqpoint{0.671747in}{2.835543in}}%
\pgfpathlineto{\pgfqpoint{0.711550in}{2.681405in}}%
\pgfpathlineto{\pgfqpoint{0.751353in}{2.537520in}}%
\pgfpathlineto{\pgfqpoint{0.799117in}{2.377434in}}%
\pgfpathlineto{\pgfqpoint{0.846881in}{2.230039in}}%
\pgfpathlineto{\pgfqpoint{0.894645in}{2.094330in}}%
\pgfpathlineto{\pgfqpoint{0.942409in}{1.969379in}}%
\pgfpathlineto{\pgfqpoint{0.990173in}{1.854334in}}%
\pgfpathlineto{\pgfqpoint{1.037937in}{1.748410in}}%
\pgfpathlineto{\pgfqpoint{1.085700in}{1.650883in}}%
\pgfpathlineto{\pgfqpoint{1.133464in}{1.561088in}}%
\pgfpathlineto{\pgfqpoint{1.181228in}{1.478412in}}%
\pgfpathlineto{\pgfqpoint{1.228992in}{1.402290in}}%
\pgfpathlineto{\pgfqpoint{1.276756in}{1.332203in}}%
\pgfpathlineto{\pgfqpoint{1.324520in}{1.267672in}}%
\pgfpathlineto{\pgfqpoint{1.372284in}{1.208257in}}%
\pgfpathlineto{\pgfqpoint{1.420047in}{1.153553in}}%
\pgfpathlineto{\pgfqpoint{1.467811in}{1.103185in}}%
\pgfpathlineto{\pgfqpoint{1.515575in}{1.056810in}}%
\pgfpathlineto{\pgfqpoint{1.571300in}{1.007332in}}%
\pgfpathlineto{\pgfqpoint{1.627024in}{0.962400in}}%
\pgfpathlineto{\pgfqpoint{1.682749in}{0.921595in}}%
\pgfpathlineto{\pgfqpoint{1.738473in}{0.884539in}}%
\pgfpathlineto{\pgfqpoint{1.794198in}{0.850887in}}%
\pgfpathlineto{\pgfqpoint{1.857883in}{0.816196in}}%
\pgfpathlineto{\pgfqpoint{1.921568in}{0.785123in}}%
\pgfpathlineto{\pgfqpoint{1.985253in}{0.757290in}}%
\pgfpathlineto{\pgfqpoint{2.080781in}{0.719682in}}%
\pgfpathlineto{\pgfqpoint{2.152427in}{0.696206in}}%
\pgfpathlineto{\pgfqpoint{2.232033in}{0.673316in}}%
\pgfpathlineto{\pgfqpoint{2.319600in}{0.651522in}}%
\pgfpathlineto{\pgfqpoint{2.415128in}{0.631225in}}%
\pgfpathlineto{\pgfqpoint{2.550459in}{0.606314in}}%
\pgfpathlineto{\pgfqpoint{2.661908in}{0.590873in}}%
\pgfpathlineto{\pgfqpoint{2.789278in}{0.576510in}}%
\pgfpathlineto{\pgfqpoint{2.972373in}{0.559635in}}%
\pgfpathlineto{\pgfqpoint{3.139546in}{0.549238in}}%
\pgfpathlineto{\pgfqpoint{3.330602in}{0.540512in}}%
\pgfpathlineto{\pgfqpoint{3.601264in}{0.531694in}}%
\pgfpathlineto{\pgfqpoint{3.935611in}{0.525443in}}%
\pgfpathlineto{\pgfqpoint{4.389367in}{0.521342in}}%
\pgfpathlineto{\pgfqpoint{4.747596in}{0.519902in}}%
\pgfpathlineto{\pgfqpoint{4.747596in}{0.519902in}}%
\pgfusepath{stroke}%
\end{pgfscope}%
\begin{pgfscope}%
\pgfsetrectcap%
\pgfsetmiterjoin%
\pgfsetlinewidth{0.803000pt}%
\definecolor{currentstroke}{rgb}{0.000000,0.000000,0.000000}%
\pgfsetstrokecolor{currentstroke}%
\pgfsetdash{}{0pt}%
\pgfpathmoveto{\pgfqpoint{0.467954in}{0.402778in}}%
\pgfpathlineto{\pgfqpoint{0.467954in}{2.951389in}}%
\pgfusepath{stroke}%
\end{pgfscope}%
\begin{pgfscope}%
\pgfsetrectcap%
\pgfsetmiterjoin%
\pgfsetlinewidth{0.803000pt}%
\definecolor{currentstroke}{rgb}{0.000000,0.000000,0.000000}%
\pgfsetstrokecolor{currentstroke}%
\pgfsetdash{}{0pt}%
\pgfpathmoveto{\pgfqpoint{4.951389in}{0.402778in}}%
\pgfpathlineto{\pgfqpoint{4.951389in}{2.951389in}}%
\pgfusepath{stroke}%
\end{pgfscope}%
\begin{pgfscope}%
\pgfsetrectcap%
\pgfsetmiterjoin%
\pgfsetlinewidth{0.803000pt}%
\definecolor{currentstroke}{rgb}{0.000000,0.000000,0.000000}%
\pgfsetstrokecolor{currentstroke}%
\pgfsetdash{}{0pt}%
\pgfpathmoveto{\pgfqpoint{0.467954in}{0.402778in}}%
\pgfpathlineto{\pgfqpoint{4.951389in}{0.402778in}}%
\pgfusepath{stroke}%
\end{pgfscope}%
\begin{pgfscope}%
\pgfsetrectcap%
\pgfsetmiterjoin%
\pgfsetlinewidth{0.803000pt}%
\definecolor{currentstroke}{rgb}{0.000000,0.000000,0.000000}%
\pgfsetstrokecolor{currentstroke}%
\pgfsetdash{}{0pt}%
\pgfpathmoveto{\pgfqpoint{0.467954in}{2.951389in}}%
\pgfpathlineto{\pgfqpoint{4.951389in}{2.951389in}}%
\pgfusepath{stroke}%
\end{pgfscope}%
\begin{pgfscope}%
\pgfsetbuttcap%
\pgfsetmiterjoin%
\definecolor{currentfill}{rgb}{1.000000,1.000000,1.000000}%
\pgfsetfillcolor{currentfill}%
\pgfsetfillopacity{0.800000}%
\pgfsetlinewidth{1.003750pt}%
\definecolor{currentstroke}{rgb}{0.800000,0.800000,0.800000}%
\pgfsetstrokecolor{currentstroke}%
\pgfsetstrokeopacity{0.800000}%
\pgfsetdash{}{0pt}%
\pgfpathmoveto{\pgfqpoint{4.340278in}{2.251167in}}%
\pgfpathlineto{\pgfqpoint{4.854167in}{2.251167in}}%
\pgfpathquadraticcurveto{\pgfqpoint{4.881944in}{2.251167in}}{\pgfqpoint{4.881944in}{2.278945in}}%
\pgfpathlineto{\pgfqpoint{4.881944in}{2.854167in}}%
\pgfpathquadraticcurveto{\pgfqpoint{4.881944in}{2.881944in}}{\pgfqpoint{4.854167in}{2.881944in}}%
\pgfpathlineto{\pgfqpoint{4.340278in}{2.881944in}}%
\pgfpathquadraticcurveto{\pgfqpoint{4.312500in}{2.881944in}}{\pgfqpoint{4.312500in}{2.854167in}}%
\pgfpathlineto{\pgfqpoint{4.312500in}{2.278945in}}%
\pgfpathquadraticcurveto{\pgfqpoint{4.312500in}{2.251167in}}{\pgfqpoint{4.340278in}{2.251167in}}%
\pgfpathclose%
\pgfusepath{stroke,fill}%
\end{pgfscope}%
\begin{pgfscope}%
\pgfsetrectcap%
\pgfsetroundjoin%
\pgfsetlinewidth{1.505625pt}%
\definecolor{currentstroke}{rgb}{0.121569,0.466667,0.705882}%
\pgfsetstrokecolor{currentstroke}%
\pgfsetdash{}{0pt}%
\pgfpathmoveto{\pgfqpoint{4.368056in}{2.777778in}}%
\pgfpathlineto{\pgfqpoint{4.645833in}{2.777778in}}%
\pgfusepath{stroke}%
\end{pgfscope}%
\begin{pgfscope}%
\pgftext[x=4.756944in,y=2.729167in,left,base]{\rmfamily\fontsize{10.000000}{12.000000}\selectfont 1}%
\end{pgfscope}%
\begin{pgfscope}%
\pgfsetrectcap%
\pgfsetroundjoin%
\pgfsetlinewidth{1.505625pt}%
\definecolor{currentstroke}{rgb}{1.000000,0.498039,0.054902}%
\pgfsetstrokecolor{currentstroke}%
\pgfsetdash{}{0pt}%
\pgfpathmoveto{\pgfqpoint{4.368056in}{2.581407in}}%
\pgfpathlineto{\pgfqpoint{4.645833in}{2.581407in}}%
\pgfusepath{stroke}%
\end{pgfscope}%
\begin{pgfscope}%
\pgftext[x=4.756944in,y=2.532796in,left,base]{\rmfamily\fontsize{10.000000}{12.000000}\selectfont 2}%
\end{pgfscope}%
\begin{pgfscope}%
\pgfsetrectcap%
\pgfsetroundjoin%
\pgfsetlinewidth{1.505625pt}%
\definecolor{currentstroke}{rgb}{0.172549,0.627451,0.172549}%
\pgfsetstrokecolor{currentstroke}%
\pgfsetdash{}{0pt}%
\pgfpathmoveto{\pgfqpoint{4.368056in}{2.385037in}}%
\pgfpathlineto{\pgfqpoint{4.645833in}{2.385037in}}%
\pgfusepath{stroke}%
\end{pgfscope}%
\begin{pgfscope}%
\pgftext[x=4.756944in,y=2.336426in,left,base]{\rmfamily\fontsize{10.000000}{12.000000}\selectfont 3}%
\end{pgfscope}%
\end{pgfpicture}%
\makeatother%
\endgroup%

    \caption{Lösung der logistischen Gleichung mit \(L = k = 1\) (oben) und den belasteten Kondensator (unten) nach aufeinanderfolgenden Iterationen.}
    \label{fig:iters_log}
\end{figure}
Der folgendende Teil beschränkt sich auf die Betrachtung der logistischen Gleichung. Um die Entwicklung des Fehlers besser beobachten zu können, kann Abbildung~\ref{fig:iter_error_local} herangezogen werden. Durch die logarithmische Skalierung des lokalen Fehlers werden nun auch verbesserungen in der vierten und fünften Iteration sichtbar. Die Sprungsstellen sind auch hier nach den früheren Iterationen gut sichtbar, verschwinden aber mit der sechsten Iteration, die im Graphen bereits nicht mehr von der letzten Iteration zu unterscheiden ist. Der wird in Bezug auf die exakte Lösung berechnet und wird der Vollständigkeit halber sowohl absolut als auch relativ gezeigt.
\begin{figure}[ht]
    \centering
        %% Creator: Matplotlib, PGF backend
%%
%% To include the figure in your LaTeX document, write
%%   \input{<filename>.pgf}
%%
%% Make sure the required packages are loaded in your preamble
%%   \usepackage{pgf}
%%
%% Figures using additional raster images can only be included by \input if
%% they are in the same directory as the main LaTeX file. For loading figures
%% from other directories you can use the `import` package
%%   \usepackage{import}
%% and then include the figures with
%%   \import{<path to file>}{<filename>.pgf}
%%
%% Matplotlib used the following preamble
%%   \usepackage{fontspec}
%%   \setmainfont{Times New Roman}
%%   \setsansfont{Lucida Grande}
%%   \setmonofont{Andale Mono}
%%
\begingroup%
\makeatletter%
\begin{pgfpicture}%
\pgfpathrectangle{\pgfpointorigin}{\pgfqpoint{5.000000in}{3.097222in}}%
\pgfusepath{use as bounding box, clip}%
\begin{pgfscope}%
\pgfsetbuttcap%
\pgfsetmiterjoin%
\definecolor{currentfill}{rgb}{1.000000,1.000000,1.000000}%
\pgfsetfillcolor{currentfill}%
\pgfsetlinewidth{0.000000pt}%
\definecolor{currentstroke}{rgb}{1.000000,1.000000,1.000000}%
\pgfsetstrokecolor{currentstroke}%
\pgfsetdash{}{0pt}%
\pgfpathmoveto{\pgfqpoint{0.000000in}{0.000000in}}%
\pgfpathlineto{\pgfqpoint{5.000000in}{0.000000in}}%
\pgfpathlineto{\pgfqpoint{5.000000in}{3.097222in}}%
\pgfpathlineto{\pgfqpoint{0.000000in}{3.097222in}}%
\pgfpathclose%
\pgfusepath{fill}%
\end{pgfscope}%
\begin{pgfscope}%
\pgfsetbuttcap%
\pgfsetmiterjoin%
\definecolor{currentfill}{rgb}{1.000000,1.000000,1.000000}%
\pgfsetfillcolor{currentfill}%
\pgfsetlinewidth{0.000000pt}%
\definecolor{currentstroke}{rgb}{0.000000,0.000000,0.000000}%
\pgfsetstrokecolor{currentstroke}%
\pgfsetstrokeopacity{0.000000}%
\pgfsetdash{}{0pt}%
\pgfpathmoveto{\pgfqpoint{0.571156in}{0.402778in}}%
\pgfpathlineto{\pgfqpoint{4.951389in}{0.402778in}}%
\pgfpathlineto{\pgfqpoint{4.951389in}{3.048611in}}%
\pgfpathlineto{\pgfqpoint{0.571156in}{3.048611in}}%
\pgfpathclose%
\pgfusepath{fill}%
\end{pgfscope}%
\begin{pgfscope}%
\pgfsetbuttcap%
\pgfsetroundjoin%
\definecolor{currentfill}{rgb}{0.000000,0.000000,0.000000}%
\pgfsetfillcolor{currentfill}%
\pgfsetlinewidth{0.803000pt}%
\definecolor{currentstroke}{rgb}{0.000000,0.000000,0.000000}%
\pgfsetstrokecolor{currentstroke}%
\pgfsetdash{}{0pt}%
\pgfsys@defobject{currentmarker}{\pgfqpoint{0.000000in}{-0.048611in}}{\pgfqpoint{0.000000in}{0.000000in}}{%
\pgfpathmoveto{\pgfqpoint{0.000000in}{0.000000in}}%
\pgfpathlineto{\pgfqpoint{0.000000in}{-0.048611in}}%
\pgfusepath{stroke,fill}%
}%
\begin{pgfscope}%
\pgfsys@transformshift{0.770257in}{0.402778in}%
\pgfsys@useobject{currentmarker}{}%
\end{pgfscope}%
\end{pgfscope}%
\begin{pgfscope}%
\pgftext[x=0.770257in,y=0.305556in,,top]{\rmfamily\fontsize{10.000000}{12.000000}\selectfont \(\displaystyle -6\)}%
\end{pgfscope}%
\begin{pgfscope}%
\pgfsetbuttcap%
\pgfsetroundjoin%
\definecolor{currentfill}{rgb}{0.000000,0.000000,0.000000}%
\pgfsetfillcolor{currentfill}%
\pgfsetlinewidth{0.803000pt}%
\definecolor{currentstroke}{rgb}{0.000000,0.000000,0.000000}%
\pgfsetstrokecolor{currentstroke}%
\pgfsetdash{}{0pt}%
\pgfsys@defobject{currentmarker}{\pgfqpoint{0.000000in}{-0.048611in}}{\pgfqpoint{0.000000in}{0.000000in}}{%
\pgfpathmoveto{\pgfqpoint{0.000000in}{0.000000in}}%
\pgfpathlineto{\pgfqpoint{0.000000in}{-0.048611in}}%
\pgfusepath{stroke,fill}%
}%
\begin{pgfscope}%
\pgfsys@transformshift{1.433929in}{0.402778in}%
\pgfsys@useobject{currentmarker}{}%
\end{pgfscope}%
\end{pgfscope}%
\begin{pgfscope}%
\pgftext[x=1.433929in,y=0.305556in,,top]{\rmfamily\fontsize{10.000000}{12.000000}\selectfont \(\displaystyle -4\)}%
\end{pgfscope}%
\begin{pgfscope}%
\pgfsetbuttcap%
\pgfsetroundjoin%
\definecolor{currentfill}{rgb}{0.000000,0.000000,0.000000}%
\pgfsetfillcolor{currentfill}%
\pgfsetlinewidth{0.803000pt}%
\definecolor{currentstroke}{rgb}{0.000000,0.000000,0.000000}%
\pgfsetstrokecolor{currentstroke}%
\pgfsetdash{}{0pt}%
\pgfsys@defobject{currentmarker}{\pgfqpoint{0.000000in}{-0.048611in}}{\pgfqpoint{0.000000in}{0.000000in}}{%
\pgfpathmoveto{\pgfqpoint{0.000000in}{0.000000in}}%
\pgfpathlineto{\pgfqpoint{0.000000in}{-0.048611in}}%
\pgfusepath{stroke,fill}%
}%
\begin{pgfscope}%
\pgfsys@transformshift{2.097601in}{0.402778in}%
\pgfsys@useobject{currentmarker}{}%
\end{pgfscope}%
\end{pgfscope}%
\begin{pgfscope}%
\pgftext[x=2.097601in,y=0.305556in,,top]{\rmfamily\fontsize{10.000000}{12.000000}\selectfont \(\displaystyle -2\)}%
\end{pgfscope}%
\begin{pgfscope}%
\pgfsetbuttcap%
\pgfsetroundjoin%
\definecolor{currentfill}{rgb}{0.000000,0.000000,0.000000}%
\pgfsetfillcolor{currentfill}%
\pgfsetlinewidth{0.803000pt}%
\definecolor{currentstroke}{rgb}{0.000000,0.000000,0.000000}%
\pgfsetstrokecolor{currentstroke}%
\pgfsetdash{}{0pt}%
\pgfsys@defobject{currentmarker}{\pgfqpoint{0.000000in}{-0.048611in}}{\pgfqpoint{0.000000in}{0.000000in}}{%
\pgfpathmoveto{\pgfqpoint{0.000000in}{0.000000in}}%
\pgfpathlineto{\pgfqpoint{0.000000in}{-0.048611in}}%
\pgfusepath{stroke,fill}%
}%
\begin{pgfscope}%
\pgfsys@transformshift{2.761272in}{0.402778in}%
\pgfsys@useobject{currentmarker}{}%
\end{pgfscope}%
\end{pgfscope}%
\begin{pgfscope}%
\pgftext[x=2.761272in,y=0.305556in,,top]{\rmfamily\fontsize{10.000000}{12.000000}\selectfont \(\displaystyle 0\)}%
\end{pgfscope}%
\begin{pgfscope}%
\pgfsetbuttcap%
\pgfsetroundjoin%
\definecolor{currentfill}{rgb}{0.000000,0.000000,0.000000}%
\pgfsetfillcolor{currentfill}%
\pgfsetlinewidth{0.803000pt}%
\definecolor{currentstroke}{rgb}{0.000000,0.000000,0.000000}%
\pgfsetstrokecolor{currentstroke}%
\pgfsetdash{}{0pt}%
\pgfsys@defobject{currentmarker}{\pgfqpoint{0.000000in}{-0.048611in}}{\pgfqpoint{0.000000in}{0.000000in}}{%
\pgfpathmoveto{\pgfqpoint{0.000000in}{0.000000in}}%
\pgfpathlineto{\pgfqpoint{0.000000in}{-0.048611in}}%
\pgfusepath{stroke,fill}%
}%
\begin{pgfscope}%
\pgfsys@transformshift{3.424944in}{0.402778in}%
\pgfsys@useobject{currentmarker}{}%
\end{pgfscope}%
\end{pgfscope}%
\begin{pgfscope}%
\pgftext[x=3.424944in,y=0.305556in,,top]{\rmfamily\fontsize{10.000000}{12.000000}\selectfont \(\displaystyle 2\)}%
\end{pgfscope}%
\begin{pgfscope}%
\pgfsetbuttcap%
\pgfsetroundjoin%
\definecolor{currentfill}{rgb}{0.000000,0.000000,0.000000}%
\pgfsetfillcolor{currentfill}%
\pgfsetlinewidth{0.803000pt}%
\definecolor{currentstroke}{rgb}{0.000000,0.000000,0.000000}%
\pgfsetstrokecolor{currentstroke}%
\pgfsetdash{}{0pt}%
\pgfsys@defobject{currentmarker}{\pgfqpoint{0.000000in}{-0.048611in}}{\pgfqpoint{0.000000in}{0.000000in}}{%
\pgfpathmoveto{\pgfqpoint{0.000000in}{0.000000in}}%
\pgfpathlineto{\pgfqpoint{0.000000in}{-0.048611in}}%
\pgfusepath{stroke,fill}%
}%
\begin{pgfscope}%
\pgfsys@transformshift{4.088616in}{0.402778in}%
\pgfsys@useobject{currentmarker}{}%
\end{pgfscope}%
\end{pgfscope}%
\begin{pgfscope}%
\pgftext[x=4.088616in,y=0.305556in,,top]{\rmfamily\fontsize{10.000000}{12.000000}\selectfont \(\displaystyle 4\)}%
\end{pgfscope}%
\begin{pgfscope}%
\pgfsetbuttcap%
\pgfsetroundjoin%
\definecolor{currentfill}{rgb}{0.000000,0.000000,0.000000}%
\pgfsetfillcolor{currentfill}%
\pgfsetlinewidth{0.803000pt}%
\definecolor{currentstroke}{rgb}{0.000000,0.000000,0.000000}%
\pgfsetstrokecolor{currentstroke}%
\pgfsetdash{}{0pt}%
\pgfsys@defobject{currentmarker}{\pgfqpoint{0.000000in}{-0.048611in}}{\pgfqpoint{0.000000in}{0.000000in}}{%
\pgfpathmoveto{\pgfqpoint{0.000000in}{0.000000in}}%
\pgfpathlineto{\pgfqpoint{0.000000in}{-0.048611in}}%
\pgfusepath{stroke,fill}%
}%
\begin{pgfscope}%
\pgfsys@transformshift{4.752287in}{0.402778in}%
\pgfsys@useobject{currentmarker}{}%
\end{pgfscope}%
\end{pgfscope}%
\begin{pgfscope}%
\pgftext[x=4.752287in,y=0.305556in,,top]{\rmfamily\fontsize{10.000000}{12.000000}\selectfont \(\displaystyle 6\)}%
\end{pgfscope}%
\begin{pgfscope}%
\pgftext[x=2.761272in,y=0.123861in,,top]{\rmfamily\fontsize{10.000000}{12.000000}\selectfont t}%
\end{pgfscope}%
\begin{pgfscope}%
\pgfsetbuttcap%
\pgfsetroundjoin%
\definecolor{currentfill}{rgb}{0.000000,0.000000,0.000000}%
\pgfsetfillcolor{currentfill}%
\pgfsetlinewidth{0.803000pt}%
\definecolor{currentstroke}{rgb}{0.000000,0.000000,0.000000}%
\pgfsetstrokecolor{currentstroke}%
\pgfsetdash{}{0pt}%
\pgfsys@defobject{currentmarker}{\pgfqpoint{-0.048611in}{0.000000in}}{\pgfqpoint{0.000000in}{0.000000in}}{%
\pgfpathmoveto{\pgfqpoint{0.000000in}{0.000000in}}%
\pgfpathlineto{\pgfqpoint{-0.048611in}{0.000000in}}%
\pgfusepath{stroke,fill}%
}%
\begin{pgfscope}%
\pgfsys@transformshift{0.571156in}{0.432601in}%
\pgfsys@useobject{currentmarker}{}%
\end{pgfscope}%
\end{pgfscope}%
\begin{pgfscope}%
\pgftext[x=0.185931in,y=0.384383in,left,base]{\rmfamily\fontsize{10.000000}{12.000000}\selectfont \(\displaystyle 10^{-7}\)}%
\end{pgfscope}%
\begin{pgfscope}%
\pgfsetbuttcap%
\pgfsetroundjoin%
\definecolor{currentfill}{rgb}{0.000000,0.000000,0.000000}%
\pgfsetfillcolor{currentfill}%
\pgfsetlinewidth{0.803000pt}%
\definecolor{currentstroke}{rgb}{0.000000,0.000000,0.000000}%
\pgfsetstrokecolor{currentstroke}%
\pgfsetdash{}{0pt}%
\pgfsys@defobject{currentmarker}{\pgfqpoint{-0.048611in}{0.000000in}}{\pgfqpoint{0.000000in}{0.000000in}}{%
\pgfpathmoveto{\pgfqpoint{0.000000in}{0.000000in}}%
\pgfpathlineto{\pgfqpoint{-0.048611in}{0.000000in}}%
\pgfusepath{stroke,fill}%
}%
\begin{pgfscope}%
\pgfsys@transformshift{0.571156in}{0.817268in}%
\pgfsys@useobject{currentmarker}{}%
\end{pgfscope}%
\end{pgfscope}%
\begin{pgfscope}%
\pgftext[x=0.185931in,y=0.769050in,left,base]{\rmfamily\fontsize{10.000000}{12.000000}\selectfont \(\displaystyle 10^{-6}\)}%
\end{pgfscope}%
\begin{pgfscope}%
\pgfsetbuttcap%
\pgfsetroundjoin%
\definecolor{currentfill}{rgb}{0.000000,0.000000,0.000000}%
\pgfsetfillcolor{currentfill}%
\pgfsetlinewidth{0.803000pt}%
\definecolor{currentstroke}{rgb}{0.000000,0.000000,0.000000}%
\pgfsetstrokecolor{currentstroke}%
\pgfsetdash{}{0pt}%
\pgfsys@defobject{currentmarker}{\pgfqpoint{-0.048611in}{0.000000in}}{\pgfqpoint{0.000000in}{0.000000in}}{%
\pgfpathmoveto{\pgfqpoint{0.000000in}{0.000000in}}%
\pgfpathlineto{\pgfqpoint{-0.048611in}{0.000000in}}%
\pgfusepath{stroke,fill}%
}%
\begin{pgfscope}%
\pgfsys@transformshift{0.571156in}{1.201935in}%
\pgfsys@useobject{currentmarker}{}%
\end{pgfscope}%
\end{pgfscope}%
\begin{pgfscope}%
\pgftext[x=0.185931in,y=1.153717in,left,base]{\rmfamily\fontsize{10.000000}{12.000000}\selectfont \(\displaystyle 10^{-5}\)}%
\end{pgfscope}%
\begin{pgfscope}%
\pgfsetbuttcap%
\pgfsetroundjoin%
\definecolor{currentfill}{rgb}{0.000000,0.000000,0.000000}%
\pgfsetfillcolor{currentfill}%
\pgfsetlinewidth{0.803000pt}%
\definecolor{currentstroke}{rgb}{0.000000,0.000000,0.000000}%
\pgfsetstrokecolor{currentstroke}%
\pgfsetdash{}{0pt}%
\pgfsys@defobject{currentmarker}{\pgfqpoint{-0.048611in}{0.000000in}}{\pgfqpoint{0.000000in}{0.000000in}}{%
\pgfpathmoveto{\pgfqpoint{0.000000in}{0.000000in}}%
\pgfpathlineto{\pgfqpoint{-0.048611in}{0.000000in}}%
\pgfusepath{stroke,fill}%
}%
\begin{pgfscope}%
\pgfsys@transformshift{0.571156in}{1.586602in}%
\pgfsys@useobject{currentmarker}{}%
\end{pgfscope}%
\end{pgfscope}%
\begin{pgfscope}%
\pgftext[x=0.185931in,y=1.538384in,left,base]{\rmfamily\fontsize{10.000000}{12.000000}\selectfont \(\displaystyle 10^{-4}\)}%
\end{pgfscope}%
\begin{pgfscope}%
\pgfsetbuttcap%
\pgfsetroundjoin%
\definecolor{currentfill}{rgb}{0.000000,0.000000,0.000000}%
\pgfsetfillcolor{currentfill}%
\pgfsetlinewidth{0.803000pt}%
\definecolor{currentstroke}{rgb}{0.000000,0.000000,0.000000}%
\pgfsetstrokecolor{currentstroke}%
\pgfsetdash{}{0pt}%
\pgfsys@defobject{currentmarker}{\pgfqpoint{-0.048611in}{0.000000in}}{\pgfqpoint{0.000000in}{0.000000in}}{%
\pgfpathmoveto{\pgfqpoint{0.000000in}{0.000000in}}%
\pgfpathlineto{\pgfqpoint{-0.048611in}{0.000000in}}%
\pgfusepath{stroke,fill}%
}%
\begin{pgfscope}%
\pgfsys@transformshift{0.571156in}{1.971269in}%
\pgfsys@useobject{currentmarker}{}%
\end{pgfscope}%
\end{pgfscope}%
\begin{pgfscope}%
\pgftext[x=0.185931in,y=1.923051in,left,base]{\rmfamily\fontsize{10.000000}{12.000000}\selectfont \(\displaystyle 10^{-3}\)}%
\end{pgfscope}%
\begin{pgfscope}%
\pgfsetbuttcap%
\pgfsetroundjoin%
\definecolor{currentfill}{rgb}{0.000000,0.000000,0.000000}%
\pgfsetfillcolor{currentfill}%
\pgfsetlinewidth{0.803000pt}%
\definecolor{currentstroke}{rgb}{0.000000,0.000000,0.000000}%
\pgfsetstrokecolor{currentstroke}%
\pgfsetdash{}{0pt}%
\pgfsys@defobject{currentmarker}{\pgfqpoint{-0.048611in}{0.000000in}}{\pgfqpoint{0.000000in}{0.000000in}}{%
\pgfpathmoveto{\pgfqpoint{0.000000in}{0.000000in}}%
\pgfpathlineto{\pgfqpoint{-0.048611in}{0.000000in}}%
\pgfusepath{stroke,fill}%
}%
\begin{pgfscope}%
\pgfsys@transformshift{0.571156in}{2.355936in}%
\pgfsys@useobject{currentmarker}{}%
\end{pgfscope}%
\end{pgfscope}%
\begin{pgfscope}%
\pgftext[x=0.185931in,y=2.307718in,left,base]{\rmfamily\fontsize{10.000000}{12.000000}\selectfont \(\displaystyle 10^{-2}\)}%
\end{pgfscope}%
\begin{pgfscope}%
\pgfsetbuttcap%
\pgfsetroundjoin%
\definecolor{currentfill}{rgb}{0.000000,0.000000,0.000000}%
\pgfsetfillcolor{currentfill}%
\pgfsetlinewidth{0.803000pt}%
\definecolor{currentstroke}{rgb}{0.000000,0.000000,0.000000}%
\pgfsetstrokecolor{currentstroke}%
\pgfsetdash{}{0pt}%
\pgfsys@defobject{currentmarker}{\pgfqpoint{-0.048611in}{0.000000in}}{\pgfqpoint{0.000000in}{0.000000in}}{%
\pgfpathmoveto{\pgfqpoint{0.000000in}{0.000000in}}%
\pgfpathlineto{\pgfqpoint{-0.048611in}{0.000000in}}%
\pgfusepath{stroke,fill}%
}%
\begin{pgfscope}%
\pgfsys@transformshift{0.571156in}{2.740603in}%
\pgfsys@useobject{currentmarker}{}%
\end{pgfscope}%
\end{pgfscope}%
\begin{pgfscope}%
\pgftext[x=0.185931in,y=2.692385in,left,base]{\rmfamily\fontsize{10.000000}{12.000000}\selectfont \(\displaystyle 10^{-1}\)}%
\end{pgfscope}%
\begin{pgfscope}%
\pgfsetbuttcap%
\pgfsetroundjoin%
\definecolor{currentfill}{rgb}{0.000000,0.000000,0.000000}%
\pgfsetfillcolor{currentfill}%
\pgfsetlinewidth{0.602250pt}%
\definecolor{currentstroke}{rgb}{0.000000,0.000000,0.000000}%
\pgfsetstrokecolor{currentstroke}%
\pgfsetdash{}{0pt}%
\pgfsys@defobject{currentmarker}{\pgfqpoint{-0.027778in}{0.000000in}}{\pgfqpoint{0.000000in}{0.000000in}}{%
\pgfpathmoveto{\pgfqpoint{0.000000in}{0.000000in}}%
\pgfpathlineto{\pgfqpoint{-0.027778in}{0.000000in}}%
\pgfusepath{stroke,fill}%
}%
\begin{pgfscope}%
\pgfsys@transformshift{0.571156in}{0.415000in}%
\pgfsys@useobject{currentmarker}{}%
\end{pgfscope}%
\end{pgfscope}%
\begin{pgfscope}%
\pgfsetbuttcap%
\pgfsetroundjoin%
\definecolor{currentfill}{rgb}{0.000000,0.000000,0.000000}%
\pgfsetfillcolor{currentfill}%
\pgfsetlinewidth{0.602250pt}%
\definecolor{currentstroke}{rgb}{0.000000,0.000000,0.000000}%
\pgfsetstrokecolor{currentstroke}%
\pgfsetdash{}{0pt}%
\pgfsys@defobject{currentmarker}{\pgfqpoint{-0.027778in}{0.000000in}}{\pgfqpoint{0.000000in}{0.000000in}}{%
\pgfpathmoveto{\pgfqpoint{0.000000in}{0.000000in}}%
\pgfpathlineto{\pgfqpoint{-0.027778in}{0.000000in}}%
\pgfusepath{stroke,fill}%
}%
\begin{pgfscope}%
\pgfsys@transformshift{0.571156in}{0.548398in}%
\pgfsys@useobject{currentmarker}{}%
\end{pgfscope}%
\end{pgfscope}%
\begin{pgfscope}%
\pgfsetbuttcap%
\pgfsetroundjoin%
\definecolor{currentfill}{rgb}{0.000000,0.000000,0.000000}%
\pgfsetfillcolor{currentfill}%
\pgfsetlinewidth{0.602250pt}%
\definecolor{currentstroke}{rgb}{0.000000,0.000000,0.000000}%
\pgfsetstrokecolor{currentstroke}%
\pgfsetdash{}{0pt}%
\pgfsys@defobject{currentmarker}{\pgfqpoint{-0.027778in}{0.000000in}}{\pgfqpoint{0.000000in}{0.000000in}}{%
\pgfpathmoveto{\pgfqpoint{0.000000in}{0.000000in}}%
\pgfpathlineto{\pgfqpoint{-0.027778in}{0.000000in}}%
\pgfusepath{stroke,fill}%
}%
\begin{pgfscope}%
\pgfsys@transformshift{0.571156in}{0.616134in}%
\pgfsys@useobject{currentmarker}{}%
\end{pgfscope}%
\end{pgfscope}%
\begin{pgfscope}%
\pgfsetbuttcap%
\pgfsetroundjoin%
\definecolor{currentfill}{rgb}{0.000000,0.000000,0.000000}%
\pgfsetfillcolor{currentfill}%
\pgfsetlinewidth{0.602250pt}%
\definecolor{currentstroke}{rgb}{0.000000,0.000000,0.000000}%
\pgfsetstrokecolor{currentstroke}%
\pgfsetdash{}{0pt}%
\pgfsys@defobject{currentmarker}{\pgfqpoint{-0.027778in}{0.000000in}}{\pgfqpoint{0.000000in}{0.000000in}}{%
\pgfpathmoveto{\pgfqpoint{0.000000in}{0.000000in}}%
\pgfpathlineto{\pgfqpoint{-0.027778in}{0.000000in}}%
\pgfusepath{stroke,fill}%
}%
\begin{pgfscope}%
\pgfsys@transformshift{0.571156in}{0.664194in}%
\pgfsys@useobject{currentmarker}{}%
\end{pgfscope}%
\end{pgfscope}%
\begin{pgfscope}%
\pgfsetbuttcap%
\pgfsetroundjoin%
\definecolor{currentfill}{rgb}{0.000000,0.000000,0.000000}%
\pgfsetfillcolor{currentfill}%
\pgfsetlinewidth{0.602250pt}%
\definecolor{currentstroke}{rgb}{0.000000,0.000000,0.000000}%
\pgfsetstrokecolor{currentstroke}%
\pgfsetdash{}{0pt}%
\pgfsys@defobject{currentmarker}{\pgfqpoint{-0.027778in}{0.000000in}}{\pgfqpoint{0.000000in}{0.000000in}}{%
\pgfpathmoveto{\pgfqpoint{0.000000in}{0.000000in}}%
\pgfpathlineto{\pgfqpoint{-0.027778in}{0.000000in}}%
\pgfusepath{stroke,fill}%
}%
\begin{pgfscope}%
\pgfsys@transformshift{0.571156in}{0.701472in}%
\pgfsys@useobject{currentmarker}{}%
\end{pgfscope}%
\end{pgfscope}%
\begin{pgfscope}%
\pgfsetbuttcap%
\pgfsetroundjoin%
\definecolor{currentfill}{rgb}{0.000000,0.000000,0.000000}%
\pgfsetfillcolor{currentfill}%
\pgfsetlinewidth{0.602250pt}%
\definecolor{currentstroke}{rgb}{0.000000,0.000000,0.000000}%
\pgfsetstrokecolor{currentstroke}%
\pgfsetdash{}{0pt}%
\pgfsys@defobject{currentmarker}{\pgfqpoint{-0.027778in}{0.000000in}}{\pgfqpoint{0.000000in}{0.000000in}}{%
\pgfpathmoveto{\pgfqpoint{0.000000in}{0.000000in}}%
\pgfpathlineto{\pgfqpoint{-0.027778in}{0.000000in}}%
\pgfusepath{stroke,fill}%
}%
\begin{pgfscope}%
\pgfsys@transformshift{0.571156in}{0.731930in}%
\pgfsys@useobject{currentmarker}{}%
\end{pgfscope}%
\end{pgfscope}%
\begin{pgfscope}%
\pgfsetbuttcap%
\pgfsetroundjoin%
\definecolor{currentfill}{rgb}{0.000000,0.000000,0.000000}%
\pgfsetfillcolor{currentfill}%
\pgfsetlinewidth{0.602250pt}%
\definecolor{currentstroke}{rgb}{0.000000,0.000000,0.000000}%
\pgfsetstrokecolor{currentstroke}%
\pgfsetdash{}{0pt}%
\pgfsys@defobject{currentmarker}{\pgfqpoint{-0.027778in}{0.000000in}}{\pgfqpoint{0.000000in}{0.000000in}}{%
\pgfpathmoveto{\pgfqpoint{0.000000in}{0.000000in}}%
\pgfpathlineto{\pgfqpoint{-0.027778in}{0.000000in}}%
\pgfusepath{stroke,fill}%
}%
\begin{pgfscope}%
\pgfsys@transformshift{0.571156in}{0.757683in}%
\pgfsys@useobject{currentmarker}{}%
\end{pgfscope}%
\end{pgfscope}%
\begin{pgfscope}%
\pgfsetbuttcap%
\pgfsetroundjoin%
\definecolor{currentfill}{rgb}{0.000000,0.000000,0.000000}%
\pgfsetfillcolor{currentfill}%
\pgfsetlinewidth{0.602250pt}%
\definecolor{currentstroke}{rgb}{0.000000,0.000000,0.000000}%
\pgfsetstrokecolor{currentstroke}%
\pgfsetdash{}{0pt}%
\pgfsys@defobject{currentmarker}{\pgfqpoint{-0.027778in}{0.000000in}}{\pgfqpoint{0.000000in}{0.000000in}}{%
\pgfpathmoveto{\pgfqpoint{0.000000in}{0.000000in}}%
\pgfpathlineto{\pgfqpoint{-0.027778in}{0.000000in}}%
\pgfusepath{stroke,fill}%
}%
\begin{pgfscope}%
\pgfsys@transformshift{0.571156in}{0.779990in}%
\pgfsys@useobject{currentmarker}{}%
\end{pgfscope}%
\end{pgfscope}%
\begin{pgfscope}%
\pgfsetbuttcap%
\pgfsetroundjoin%
\definecolor{currentfill}{rgb}{0.000000,0.000000,0.000000}%
\pgfsetfillcolor{currentfill}%
\pgfsetlinewidth{0.602250pt}%
\definecolor{currentstroke}{rgb}{0.000000,0.000000,0.000000}%
\pgfsetstrokecolor{currentstroke}%
\pgfsetdash{}{0pt}%
\pgfsys@defobject{currentmarker}{\pgfqpoint{-0.027778in}{0.000000in}}{\pgfqpoint{0.000000in}{0.000000in}}{%
\pgfpathmoveto{\pgfqpoint{0.000000in}{0.000000in}}%
\pgfpathlineto{\pgfqpoint{-0.027778in}{0.000000in}}%
\pgfusepath{stroke,fill}%
}%
\begin{pgfscope}%
\pgfsys@transformshift{0.571156in}{0.799667in}%
\pgfsys@useobject{currentmarker}{}%
\end{pgfscope}%
\end{pgfscope}%
\begin{pgfscope}%
\pgfsetbuttcap%
\pgfsetroundjoin%
\definecolor{currentfill}{rgb}{0.000000,0.000000,0.000000}%
\pgfsetfillcolor{currentfill}%
\pgfsetlinewidth{0.602250pt}%
\definecolor{currentstroke}{rgb}{0.000000,0.000000,0.000000}%
\pgfsetstrokecolor{currentstroke}%
\pgfsetdash{}{0pt}%
\pgfsys@defobject{currentmarker}{\pgfqpoint{-0.027778in}{0.000000in}}{\pgfqpoint{0.000000in}{0.000000in}}{%
\pgfpathmoveto{\pgfqpoint{0.000000in}{0.000000in}}%
\pgfpathlineto{\pgfqpoint{-0.027778in}{0.000000in}}%
\pgfusepath{stroke,fill}%
}%
\begin{pgfscope}%
\pgfsys@transformshift{0.571156in}{0.933065in}%
\pgfsys@useobject{currentmarker}{}%
\end{pgfscope}%
\end{pgfscope}%
\begin{pgfscope}%
\pgfsetbuttcap%
\pgfsetroundjoin%
\definecolor{currentfill}{rgb}{0.000000,0.000000,0.000000}%
\pgfsetfillcolor{currentfill}%
\pgfsetlinewidth{0.602250pt}%
\definecolor{currentstroke}{rgb}{0.000000,0.000000,0.000000}%
\pgfsetstrokecolor{currentstroke}%
\pgfsetdash{}{0pt}%
\pgfsys@defobject{currentmarker}{\pgfqpoint{-0.027778in}{0.000000in}}{\pgfqpoint{0.000000in}{0.000000in}}{%
\pgfpathmoveto{\pgfqpoint{0.000000in}{0.000000in}}%
\pgfpathlineto{\pgfqpoint{-0.027778in}{0.000000in}}%
\pgfusepath{stroke,fill}%
}%
\begin{pgfscope}%
\pgfsys@transformshift{0.571156in}{1.000801in}%
\pgfsys@useobject{currentmarker}{}%
\end{pgfscope}%
\end{pgfscope}%
\begin{pgfscope}%
\pgfsetbuttcap%
\pgfsetroundjoin%
\definecolor{currentfill}{rgb}{0.000000,0.000000,0.000000}%
\pgfsetfillcolor{currentfill}%
\pgfsetlinewidth{0.602250pt}%
\definecolor{currentstroke}{rgb}{0.000000,0.000000,0.000000}%
\pgfsetstrokecolor{currentstroke}%
\pgfsetdash{}{0pt}%
\pgfsys@defobject{currentmarker}{\pgfqpoint{-0.027778in}{0.000000in}}{\pgfqpoint{0.000000in}{0.000000in}}{%
\pgfpathmoveto{\pgfqpoint{0.000000in}{0.000000in}}%
\pgfpathlineto{\pgfqpoint{-0.027778in}{0.000000in}}%
\pgfusepath{stroke,fill}%
}%
\begin{pgfscope}%
\pgfsys@transformshift{0.571156in}{1.048861in}%
\pgfsys@useobject{currentmarker}{}%
\end{pgfscope}%
\end{pgfscope}%
\begin{pgfscope}%
\pgfsetbuttcap%
\pgfsetroundjoin%
\definecolor{currentfill}{rgb}{0.000000,0.000000,0.000000}%
\pgfsetfillcolor{currentfill}%
\pgfsetlinewidth{0.602250pt}%
\definecolor{currentstroke}{rgb}{0.000000,0.000000,0.000000}%
\pgfsetstrokecolor{currentstroke}%
\pgfsetdash{}{0pt}%
\pgfsys@defobject{currentmarker}{\pgfqpoint{-0.027778in}{0.000000in}}{\pgfqpoint{0.000000in}{0.000000in}}{%
\pgfpathmoveto{\pgfqpoint{0.000000in}{0.000000in}}%
\pgfpathlineto{\pgfqpoint{-0.027778in}{0.000000in}}%
\pgfusepath{stroke,fill}%
}%
\begin{pgfscope}%
\pgfsys@transformshift{0.571156in}{1.086139in}%
\pgfsys@useobject{currentmarker}{}%
\end{pgfscope}%
\end{pgfscope}%
\begin{pgfscope}%
\pgfsetbuttcap%
\pgfsetroundjoin%
\definecolor{currentfill}{rgb}{0.000000,0.000000,0.000000}%
\pgfsetfillcolor{currentfill}%
\pgfsetlinewidth{0.602250pt}%
\definecolor{currentstroke}{rgb}{0.000000,0.000000,0.000000}%
\pgfsetstrokecolor{currentstroke}%
\pgfsetdash{}{0pt}%
\pgfsys@defobject{currentmarker}{\pgfqpoint{-0.027778in}{0.000000in}}{\pgfqpoint{0.000000in}{0.000000in}}{%
\pgfpathmoveto{\pgfqpoint{0.000000in}{0.000000in}}%
\pgfpathlineto{\pgfqpoint{-0.027778in}{0.000000in}}%
\pgfusepath{stroke,fill}%
}%
\begin{pgfscope}%
\pgfsys@transformshift{0.571156in}{1.116597in}%
\pgfsys@useobject{currentmarker}{}%
\end{pgfscope}%
\end{pgfscope}%
\begin{pgfscope}%
\pgfsetbuttcap%
\pgfsetroundjoin%
\definecolor{currentfill}{rgb}{0.000000,0.000000,0.000000}%
\pgfsetfillcolor{currentfill}%
\pgfsetlinewidth{0.602250pt}%
\definecolor{currentstroke}{rgb}{0.000000,0.000000,0.000000}%
\pgfsetstrokecolor{currentstroke}%
\pgfsetdash{}{0pt}%
\pgfsys@defobject{currentmarker}{\pgfqpoint{-0.027778in}{0.000000in}}{\pgfqpoint{0.000000in}{0.000000in}}{%
\pgfpathmoveto{\pgfqpoint{0.000000in}{0.000000in}}%
\pgfpathlineto{\pgfqpoint{-0.027778in}{0.000000in}}%
\pgfusepath{stroke,fill}%
}%
\begin{pgfscope}%
\pgfsys@transformshift{0.571156in}{1.142350in}%
\pgfsys@useobject{currentmarker}{}%
\end{pgfscope}%
\end{pgfscope}%
\begin{pgfscope}%
\pgfsetbuttcap%
\pgfsetroundjoin%
\definecolor{currentfill}{rgb}{0.000000,0.000000,0.000000}%
\pgfsetfillcolor{currentfill}%
\pgfsetlinewidth{0.602250pt}%
\definecolor{currentstroke}{rgb}{0.000000,0.000000,0.000000}%
\pgfsetstrokecolor{currentstroke}%
\pgfsetdash{}{0pt}%
\pgfsys@defobject{currentmarker}{\pgfqpoint{-0.027778in}{0.000000in}}{\pgfqpoint{0.000000in}{0.000000in}}{%
\pgfpathmoveto{\pgfqpoint{0.000000in}{0.000000in}}%
\pgfpathlineto{\pgfqpoint{-0.027778in}{0.000000in}}%
\pgfusepath{stroke,fill}%
}%
\begin{pgfscope}%
\pgfsys@transformshift{0.571156in}{1.164657in}%
\pgfsys@useobject{currentmarker}{}%
\end{pgfscope}%
\end{pgfscope}%
\begin{pgfscope}%
\pgfsetbuttcap%
\pgfsetroundjoin%
\definecolor{currentfill}{rgb}{0.000000,0.000000,0.000000}%
\pgfsetfillcolor{currentfill}%
\pgfsetlinewidth{0.602250pt}%
\definecolor{currentstroke}{rgb}{0.000000,0.000000,0.000000}%
\pgfsetstrokecolor{currentstroke}%
\pgfsetdash{}{0pt}%
\pgfsys@defobject{currentmarker}{\pgfqpoint{-0.027778in}{0.000000in}}{\pgfqpoint{0.000000in}{0.000000in}}{%
\pgfpathmoveto{\pgfqpoint{0.000000in}{0.000000in}}%
\pgfpathlineto{\pgfqpoint{-0.027778in}{0.000000in}}%
\pgfusepath{stroke,fill}%
}%
\begin{pgfscope}%
\pgfsys@transformshift{0.571156in}{1.184334in}%
\pgfsys@useobject{currentmarker}{}%
\end{pgfscope}%
\end{pgfscope}%
\begin{pgfscope}%
\pgfsetbuttcap%
\pgfsetroundjoin%
\definecolor{currentfill}{rgb}{0.000000,0.000000,0.000000}%
\pgfsetfillcolor{currentfill}%
\pgfsetlinewidth{0.602250pt}%
\definecolor{currentstroke}{rgb}{0.000000,0.000000,0.000000}%
\pgfsetstrokecolor{currentstroke}%
\pgfsetdash{}{0pt}%
\pgfsys@defobject{currentmarker}{\pgfqpoint{-0.027778in}{0.000000in}}{\pgfqpoint{0.000000in}{0.000000in}}{%
\pgfpathmoveto{\pgfqpoint{0.000000in}{0.000000in}}%
\pgfpathlineto{\pgfqpoint{-0.027778in}{0.000000in}}%
\pgfusepath{stroke,fill}%
}%
\begin{pgfscope}%
\pgfsys@transformshift{0.571156in}{1.317731in}%
\pgfsys@useobject{currentmarker}{}%
\end{pgfscope}%
\end{pgfscope}%
\begin{pgfscope}%
\pgfsetbuttcap%
\pgfsetroundjoin%
\definecolor{currentfill}{rgb}{0.000000,0.000000,0.000000}%
\pgfsetfillcolor{currentfill}%
\pgfsetlinewidth{0.602250pt}%
\definecolor{currentstroke}{rgb}{0.000000,0.000000,0.000000}%
\pgfsetstrokecolor{currentstroke}%
\pgfsetdash{}{0pt}%
\pgfsys@defobject{currentmarker}{\pgfqpoint{-0.027778in}{0.000000in}}{\pgfqpoint{0.000000in}{0.000000in}}{%
\pgfpathmoveto{\pgfqpoint{0.000000in}{0.000000in}}%
\pgfpathlineto{\pgfqpoint{-0.027778in}{0.000000in}}%
\pgfusepath{stroke,fill}%
}%
\begin{pgfscope}%
\pgfsys@transformshift{0.571156in}{1.385468in}%
\pgfsys@useobject{currentmarker}{}%
\end{pgfscope}%
\end{pgfscope}%
\begin{pgfscope}%
\pgfsetbuttcap%
\pgfsetroundjoin%
\definecolor{currentfill}{rgb}{0.000000,0.000000,0.000000}%
\pgfsetfillcolor{currentfill}%
\pgfsetlinewidth{0.602250pt}%
\definecolor{currentstroke}{rgb}{0.000000,0.000000,0.000000}%
\pgfsetstrokecolor{currentstroke}%
\pgfsetdash{}{0pt}%
\pgfsys@defobject{currentmarker}{\pgfqpoint{-0.027778in}{0.000000in}}{\pgfqpoint{0.000000in}{0.000000in}}{%
\pgfpathmoveto{\pgfqpoint{0.000000in}{0.000000in}}%
\pgfpathlineto{\pgfqpoint{-0.027778in}{0.000000in}}%
\pgfusepath{stroke,fill}%
}%
\begin{pgfscope}%
\pgfsys@transformshift{0.571156in}{1.433528in}%
\pgfsys@useobject{currentmarker}{}%
\end{pgfscope}%
\end{pgfscope}%
\begin{pgfscope}%
\pgfsetbuttcap%
\pgfsetroundjoin%
\definecolor{currentfill}{rgb}{0.000000,0.000000,0.000000}%
\pgfsetfillcolor{currentfill}%
\pgfsetlinewidth{0.602250pt}%
\definecolor{currentstroke}{rgb}{0.000000,0.000000,0.000000}%
\pgfsetstrokecolor{currentstroke}%
\pgfsetdash{}{0pt}%
\pgfsys@defobject{currentmarker}{\pgfqpoint{-0.027778in}{0.000000in}}{\pgfqpoint{0.000000in}{0.000000in}}{%
\pgfpathmoveto{\pgfqpoint{0.000000in}{0.000000in}}%
\pgfpathlineto{\pgfqpoint{-0.027778in}{0.000000in}}%
\pgfusepath{stroke,fill}%
}%
\begin{pgfscope}%
\pgfsys@transformshift{0.571156in}{1.470806in}%
\pgfsys@useobject{currentmarker}{}%
\end{pgfscope}%
\end{pgfscope}%
\begin{pgfscope}%
\pgfsetbuttcap%
\pgfsetroundjoin%
\definecolor{currentfill}{rgb}{0.000000,0.000000,0.000000}%
\pgfsetfillcolor{currentfill}%
\pgfsetlinewidth{0.602250pt}%
\definecolor{currentstroke}{rgb}{0.000000,0.000000,0.000000}%
\pgfsetstrokecolor{currentstroke}%
\pgfsetdash{}{0pt}%
\pgfsys@defobject{currentmarker}{\pgfqpoint{-0.027778in}{0.000000in}}{\pgfqpoint{0.000000in}{0.000000in}}{%
\pgfpathmoveto{\pgfqpoint{0.000000in}{0.000000in}}%
\pgfpathlineto{\pgfqpoint{-0.027778in}{0.000000in}}%
\pgfusepath{stroke,fill}%
}%
\begin{pgfscope}%
\pgfsys@transformshift{0.571156in}{1.501264in}%
\pgfsys@useobject{currentmarker}{}%
\end{pgfscope}%
\end{pgfscope}%
\begin{pgfscope}%
\pgfsetbuttcap%
\pgfsetroundjoin%
\definecolor{currentfill}{rgb}{0.000000,0.000000,0.000000}%
\pgfsetfillcolor{currentfill}%
\pgfsetlinewidth{0.602250pt}%
\definecolor{currentstroke}{rgb}{0.000000,0.000000,0.000000}%
\pgfsetstrokecolor{currentstroke}%
\pgfsetdash{}{0pt}%
\pgfsys@defobject{currentmarker}{\pgfqpoint{-0.027778in}{0.000000in}}{\pgfqpoint{0.000000in}{0.000000in}}{%
\pgfpathmoveto{\pgfqpoint{0.000000in}{0.000000in}}%
\pgfpathlineto{\pgfqpoint{-0.027778in}{0.000000in}}%
\pgfusepath{stroke,fill}%
}%
\begin{pgfscope}%
\pgfsys@transformshift{0.571156in}{1.527016in}%
\pgfsys@useobject{currentmarker}{}%
\end{pgfscope}%
\end{pgfscope}%
\begin{pgfscope}%
\pgfsetbuttcap%
\pgfsetroundjoin%
\definecolor{currentfill}{rgb}{0.000000,0.000000,0.000000}%
\pgfsetfillcolor{currentfill}%
\pgfsetlinewidth{0.602250pt}%
\definecolor{currentstroke}{rgb}{0.000000,0.000000,0.000000}%
\pgfsetstrokecolor{currentstroke}%
\pgfsetdash{}{0pt}%
\pgfsys@defobject{currentmarker}{\pgfqpoint{-0.027778in}{0.000000in}}{\pgfqpoint{0.000000in}{0.000000in}}{%
\pgfpathmoveto{\pgfqpoint{0.000000in}{0.000000in}}%
\pgfpathlineto{\pgfqpoint{-0.027778in}{0.000000in}}%
\pgfusepath{stroke,fill}%
}%
\begin{pgfscope}%
\pgfsys@transformshift{0.571156in}{1.549324in}%
\pgfsys@useobject{currentmarker}{}%
\end{pgfscope}%
\end{pgfscope}%
\begin{pgfscope}%
\pgfsetbuttcap%
\pgfsetroundjoin%
\definecolor{currentfill}{rgb}{0.000000,0.000000,0.000000}%
\pgfsetfillcolor{currentfill}%
\pgfsetlinewidth{0.602250pt}%
\definecolor{currentstroke}{rgb}{0.000000,0.000000,0.000000}%
\pgfsetstrokecolor{currentstroke}%
\pgfsetdash{}{0pt}%
\pgfsys@defobject{currentmarker}{\pgfqpoint{-0.027778in}{0.000000in}}{\pgfqpoint{0.000000in}{0.000000in}}{%
\pgfpathmoveto{\pgfqpoint{0.000000in}{0.000000in}}%
\pgfpathlineto{\pgfqpoint{-0.027778in}{0.000000in}}%
\pgfusepath{stroke,fill}%
}%
\begin{pgfscope}%
\pgfsys@transformshift{0.571156in}{1.569001in}%
\pgfsys@useobject{currentmarker}{}%
\end{pgfscope}%
\end{pgfscope}%
\begin{pgfscope}%
\pgfsetbuttcap%
\pgfsetroundjoin%
\definecolor{currentfill}{rgb}{0.000000,0.000000,0.000000}%
\pgfsetfillcolor{currentfill}%
\pgfsetlinewidth{0.602250pt}%
\definecolor{currentstroke}{rgb}{0.000000,0.000000,0.000000}%
\pgfsetstrokecolor{currentstroke}%
\pgfsetdash{}{0pt}%
\pgfsys@defobject{currentmarker}{\pgfqpoint{-0.027778in}{0.000000in}}{\pgfqpoint{0.000000in}{0.000000in}}{%
\pgfpathmoveto{\pgfqpoint{0.000000in}{0.000000in}}%
\pgfpathlineto{\pgfqpoint{-0.027778in}{0.000000in}}%
\pgfusepath{stroke,fill}%
}%
\begin{pgfscope}%
\pgfsys@transformshift{0.571156in}{1.702398in}%
\pgfsys@useobject{currentmarker}{}%
\end{pgfscope}%
\end{pgfscope}%
\begin{pgfscope}%
\pgfsetbuttcap%
\pgfsetroundjoin%
\definecolor{currentfill}{rgb}{0.000000,0.000000,0.000000}%
\pgfsetfillcolor{currentfill}%
\pgfsetlinewidth{0.602250pt}%
\definecolor{currentstroke}{rgb}{0.000000,0.000000,0.000000}%
\pgfsetstrokecolor{currentstroke}%
\pgfsetdash{}{0pt}%
\pgfsys@defobject{currentmarker}{\pgfqpoint{-0.027778in}{0.000000in}}{\pgfqpoint{0.000000in}{0.000000in}}{%
\pgfpathmoveto{\pgfqpoint{0.000000in}{0.000000in}}%
\pgfpathlineto{\pgfqpoint{-0.027778in}{0.000000in}}%
\pgfusepath{stroke,fill}%
}%
\begin{pgfscope}%
\pgfsys@transformshift{0.571156in}{1.770135in}%
\pgfsys@useobject{currentmarker}{}%
\end{pgfscope}%
\end{pgfscope}%
\begin{pgfscope}%
\pgfsetbuttcap%
\pgfsetroundjoin%
\definecolor{currentfill}{rgb}{0.000000,0.000000,0.000000}%
\pgfsetfillcolor{currentfill}%
\pgfsetlinewidth{0.602250pt}%
\definecolor{currentstroke}{rgb}{0.000000,0.000000,0.000000}%
\pgfsetstrokecolor{currentstroke}%
\pgfsetdash{}{0pt}%
\pgfsys@defobject{currentmarker}{\pgfqpoint{-0.027778in}{0.000000in}}{\pgfqpoint{0.000000in}{0.000000in}}{%
\pgfpathmoveto{\pgfqpoint{0.000000in}{0.000000in}}%
\pgfpathlineto{\pgfqpoint{-0.027778in}{0.000000in}}%
\pgfusepath{stroke,fill}%
}%
\begin{pgfscope}%
\pgfsys@transformshift{0.571156in}{1.818195in}%
\pgfsys@useobject{currentmarker}{}%
\end{pgfscope}%
\end{pgfscope}%
\begin{pgfscope}%
\pgfsetbuttcap%
\pgfsetroundjoin%
\definecolor{currentfill}{rgb}{0.000000,0.000000,0.000000}%
\pgfsetfillcolor{currentfill}%
\pgfsetlinewidth{0.602250pt}%
\definecolor{currentstroke}{rgb}{0.000000,0.000000,0.000000}%
\pgfsetstrokecolor{currentstroke}%
\pgfsetdash{}{0pt}%
\pgfsys@defobject{currentmarker}{\pgfqpoint{-0.027778in}{0.000000in}}{\pgfqpoint{0.000000in}{0.000000in}}{%
\pgfpathmoveto{\pgfqpoint{0.000000in}{0.000000in}}%
\pgfpathlineto{\pgfqpoint{-0.027778in}{0.000000in}}%
\pgfusepath{stroke,fill}%
}%
\begin{pgfscope}%
\pgfsys@transformshift{0.571156in}{1.855473in}%
\pgfsys@useobject{currentmarker}{}%
\end{pgfscope}%
\end{pgfscope}%
\begin{pgfscope}%
\pgfsetbuttcap%
\pgfsetroundjoin%
\definecolor{currentfill}{rgb}{0.000000,0.000000,0.000000}%
\pgfsetfillcolor{currentfill}%
\pgfsetlinewidth{0.602250pt}%
\definecolor{currentstroke}{rgb}{0.000000,0.000000,0.000000}%
\pgfsetstrokecolor{currentstroke}%
\pgfsetdash{}{0pt}%
\pgfsys@defobject{currentmarker}{\pgfqpoint{-0.027778in}{0.000000in}}{\pgfqpoint{0.000000in}{0.000000in}}{%
\pgfpathmoveto{\pgfqpoint{0.000000in}{0.000000in}}%
\pgfpathlineto{\pgfqpoint{-0.027778in}{0.000000in}}%
\pgfusepath{stroke,fill}%
}%
\begin{pgfscope}%
\pgfsys@transformshift{0.571156in}{1.885931in}%
\pgfsys@useobject{currentmarker}{}%
\end{pgfscope}%
\end{pgfscope}%
\begin{pgfscope}%
\pgfsetbuttcap%
\pgfsetroundjoin%
\definecolor{currentfill}{rgb}{0.000000,0.000000,0.000000}%
\pgfsetfillcolor{currentfill}%
\pgfsetlinewidth{0.602250pt}%
\definecolor{currentstroke}{rgb}{0.000000,0.000000,0.000000}%
\pgfsetstrokecolor{currentstroke}%
\pgfsetdash{}{0pt}%
\pgfsys@defobject{currentmarker}{\pgfqpoint{-0.027778in}{0.000000in}}{\pgfqpoint{0.000000in}{0.000000in}}{%
\pgfpathmoveto{\pgfqpoint{0.000000in}{0.000000in}}%
\pgfpathlineto{\pgfqpoint{-0.027778in}{0.000000in}}%
\pgfusepath{stroke,fill}%
}%
\begin{pgfscope}%
\pgfsys@transformshift{0.571156in}{1.911683in}%
\pgfsys@useobject{currentmarker}{}%
\end{pgfscope}%
\end{pgfscope}%
\begin{pgfscope}%
\pgfsetbuttcap%
\pgfsetroundjoin%
\definecolor{currentfill}{rgb}{0.000000,0.000000,0.000000}%
\pgfsetfillcolor{currentfill}%
\pgfsetlinewidth{0.602250pt}%
\definecolor{currentstroke}{rgb}{0.000000,0.000000,0.000000}%
\pgfsetstrokecolor{currentstroke}%
\pgfsetdash{}{0pt}%
\pgfsys@defobject{currentmarker}{\pgfqpoint{-0.027778in}{0.000000in}}{\pgfqpoint{0.000000in}{0.000000in}}{%
\pgfpathmoveto{\pgfqpoint{0.000000in}{0.000000in}}%
\pgfpathlineto{\pgfqpoint{-0.027778in}{0.000000in}}%
\pgfusepath{stroke,fill}%
}%
\begin{pgfscope}%
\pgfsys@transformshift{0.571156in}{1.933991in}%
\pgfsys@useobject{currentmarker}{}%
\end{pgfscope}%
\end{pgfscope}%
\begin{pgfscope}%
\pgfsetbuttcap%
\pgfsetroundjoin%
\definecolor{currentfill}{rgb}{0.000000,0.000000,0.000000}%
\pgfsetfillcolor{currentfill}%
\pgfsetlinewidth{0.602250pt}%
\definecolor{currentstroke}{rgb}{0.000000,0.000000,0.000000}%
\pgfsetstrokecolor{currentstroke}%
\pgfsetdash{}{0pt}%
\pgfsys@defobject{currentmarker}{\pgfqpoint{-0.027778in}{0.000000in}}{\pgfqpoint{0.000000in}{0.000000in}}{%
\pgfpathmoveto{\pgfqpoint{0.000000in}{0.000000in}}%
\pgfpathlineto{\pgfqpoint{-0.027778in}{0.000000in}}%
\pgfusepath{stroke,fill}%
}%
\begin{pgfscope}%
\pgfsys@transformshift{0.571156in}{1.953668in}%
\pgfsys@useobject{currentmarker}{}%
\end{pgfscope}%
\end{pgfscope}%
\begin{pgfscope}%
\pgfsetbuttcap%
\pgfsetroundjoin%
\definecolor{currentfill}{rgb}{0.000000,0.000000,0.000000}%
\pgfsetfillcolor{currentfill}%
\pgfsetlinewidth{0.602250pt}%
\definecolor{currentstroke}{rgb}{0.000000,0.000000,0.000000}%
\pgfsetstrokecolor{currentstroke}%
\pgfsetdash{}{0pt}%
\pgfsys@defobject{currentmarker}{\pgfqpoint{-0.027778in}{0.000000in}}{\pgfqpoint{0.000000in}{0.000000in}}{%
\pgfpathmoveto{\pgfqpoint{0.000000in}{0.000000in}}%
\pgfpathlineto{\pgfqpoint{-0.027778in}{0.000000in}}%
\pgfusepath{stroke,fill}%
}%
\begin{pgfscope}%
\pgfsys@transformshift{0.571156in}{2.087065in}%
\pgfsys@useobject{currentmarker}{}%
\end{pgfscope}%
\end{pgfscope}%
\begin{pgfscope}%
\pgfsetbuttcap%
\pgfsetroundjoin%
\definecolor{currentfill}{rgb}{0.000000,0.000000,0.000000}%
\pgfsetfillcolor{currentfill}%
\pgfsetlinewidth{0.602250pt}%
\definecolor{currentstroke}{rgb}{0.000000,0.000000,0.000000}%
\pgfsetstrokecolor{currentstroke}%
\pgfsetdash{}{0pt}%
\pgfsys@defobject{currentmarker}{\pgfqpoint{-0.027778in}{0.000000in}}{\pgfqpoint{0.000000in}{0.000000in}}{%
\pgfpathmoveto{\pgfqpoint{0.000000in}{0.000000in}}%
\pgfpathlineto{\pgfqpoint{-0.027778in}{0.000000in}}%
\pgfusepath{stroke,fill}%
}%
\begin{pgfscope}%
\pgfsys@transformshift{0.571156in}{2.154802in}%
\pgfsys@useobject{currentmarker}{}%
\end{pgfscope}%
\end{pgfscope}%
\begin{pgfscope}%
\pgfsetbuttcap%
\pgfsetroundjoin%
\definecolor{currentfill}{rgb}{0.000000,0.000000,0.000000}%
\pgfsetfillcolor{currentfill}%
\pgfsetlinewidth{0.602250pt}%
\definecolor{currentstroke}{rgb}{0.000000,0.000000,0.000000}%
\pgfsetstrokecolor{currentstroke}%
\pgfsetdash{}{0pt}%
\pgfsys@defobject{currentmarker}{\pgfqpoint{-0.027778in}{0.000000in}}{\pgfqpoint{0.000000in}{0.000000in}}{%
\pgfpathmoveto{\pgfqpoint{0.000000in}{0.000000in}}%
\pgfpathlineto{\pgfqpoint{-0.027778in}{0.000000in}}%
\pgfusepath{stroke,fill}%
}%
\begin{pgfscope}%
\pgfsys@transformshift{0.571156in}{2.202862in}%
\pgfsys@useobject{currentmarker}{}%
\end{pgfscope}%
\end{pgfscope}%
\begin{pgfscope}%
\pgfsetbuttcap%
\pgfsetroundjoin%
\definecolor{currentfill}{rgb}{0.000000,0.000000,0.000000}%
\pgfsetfillcolor{currentfill}%
\pgfsetlinewidth{0.602250pt}%
\definecolor{currentstroke}{rgb}{0.000000,0.000000,0.000000}%
\pgfsetstrokecolor{currentstroke}%
\pgfsetdash{}{0pt}%
\pgfsys@defobject{currentmarker}{\pgfqpoint{-0.027778in}{0.000000in}}{\pgfqpoint{0.000000in}{0.000000in}}{%
\pgfpathmoveto{\pgfqpoint{0.000000in}{0.000000in}}%
\pgfpathlineto{\pgfqpoint{-0.027778in}{0.000000in}}%
\pgfusepath{stroke,fill}%
}%
\begin{pgfscope}%
\pgfsys@transformshift{0.571156in}{2.240140in}%
\pgfsys@useobject{currentmarker}{}%
\end{pgfscope}%
\end{pgfscope}%
\begin{pgfscope}%
\pgfsetbuttcap%
\pgfsetroundjoin%
\definecolor{currentfill}{rgb}{0.000000,0.000000,0.000000}%
\pgfsetfillcolor{currentfill}%
\pgfsetlinewidth{0.602250pt}%
\definecolor{currentstroke}{rgb}{0.000000,0.000000,0.000000}%
\pgfsetstrokecolor{currentstroke}%
\pgfsetdash{}{0pt}%
\pgfsys@defobject{currentmarker}{\pgfqpoint{-0.027778in}{0.000000in}}{\pgfqpoint{0.000000in}{0.000000in}}{%
\pgfpathmoveto{\pgfqpoint{0.000000in}{0.000000in}}%
\pgfpathlineto{\pgfqpoint{-0.027778in}{0.000000in}}%
\pgfusepath{stroke,fill}%
}%
\begin{pgfscope}%
\pgfsys@transformshift{0.571156in}{2.270598in}%
\pgfsys@useobject{currentmarker}{}%
\end{pgfscope}%
\end{pgfscope}%
\begin{pgfscope}%
\pgfsetbuttcap%
\pgfsetroundjoin%
\definecolor{currentfill}{rgb}{0.000000,0.000000,0.000000}%
\pgfsetfillcolor{currentfill}%
\pgfsetlinewidth{0.602250pt}%
\definecolor{currentstroke}{rgb}{0.000000,0.000000,0.000000}%
\pgfsetstrokecolor{currentstroke}%
\pgfsetdash{}{0pt}%
\pgfsys@defobject{currentmarker}{\pgfqpoint{-0.027778in}{0.000000in}}{\pgfqpoint{0.000000in}{0.000000in}}{%
\pgfpathmoveto{\pgfqpoint{0.000000in}{0.000000in}}%
\pgfpathlineto{\pgfqpoint{-0.027778in}{0.000000in}}%
\pgfusepath{stroke,fill}%
}%
\begin{pgfscope}%
\pgfsys@transformshift{0.571156in}{2.296350in}%
\pgfsys@useobject{currentmarker}{}%
\end{pgfscope}%
\end{pgfscope}%
\begin{pgfscope}%
\pgfsetbuttcap%
\pgfsetroundjoin%
\definecolor{currentfill}{rgb}{0.000000,0.000000,0.000000}%
\pgfsetfillcolor{currentfill}%
\pgfsetlinewidth{0.602250pt}%
\definecolor{currentstroke}{rgb}{0.000000,0.000000,0.000000}%
\pgfsetstrokecolor{currentstroke}%
\pgfsetdash{}{0pt}%
\pgfsys@defobject{currentmarker}{\pgfqpoint{-0.027778in}{0.000000in}}{\pgfqpoint{0.000000in}{0.000000in}}{%
\pgfpathmoveto{\pgfqpoint{0.000000in}{0.000000in}}%
\pgfpathlineto{\pgfqpoint{-0.027778in}{0.000000in}}%
\pgfusepath{stroke,fill}%
}%
\begin{pgfscope}%
\pgfsys@transformshift{0.571156in}{2.318658in}%
\pgfsys@useobject{currentmarker}{}%
\end{pgfscope}%
\end{pgfscope}%
\begin{pgfscope}%
\pgfsetbuttcap%
\pgfsetroundjoin%
\definecolor{currentfill}{rgb}{0.000000,0.000000,0.000000}%
\pgfsetfillcolor{currentfill}%
\pgfsetlinewidth{0.602250pt}%
\definecolor{currentstroke}{rgb}{0.000000,0.000000,0.000000}%
\pgfsetstrokecolor{currentstroke}%
\pgfsetdash{}{0pt}%
\pgfsys@defobject{currentmarker}{\pgfqpoint{-0.027778in}{0.000000in}}{\pgfqpoint{0.000000in}{0.000000in}}{%
\pgfpathmoveto{\pgfqpoint{0.000000in}{0.000000in}}%
\pgfpathlineto{\pgfqpoint{-0.027778in}{0.000000in}}%
\pgfusepath{stroke,fill}%
}%
\begin{pgfscope}%
\pgfsys@transformshift{0.571156in}{2.338335in}%
\pgfsys@useobject{currentmarker}{}%
\end{pgfscope}%
\end{pgfscope}%
\begin{pgfscope}%
\pgfsetbuttcap%
\pgfsetroundjoin%
\definecolor{currentfill}{rgb}{0.000000,0.000000,0.000000}%
\pgfsetfillcolor{currentfill}%
\pgfsetlinewidth{0.602250pt}%
\definecolor{currentstroke}{rgb}{0.000000,0.000000,0.000000}%
\pgfsetstrokecolor{currentstroke}%
\pgfsetdash{}{0pt}%
\pgfsys@defobject{currentmarker}{\pgfqpoint{-0.027778in}{0.000000in}}{\pgfqpoint{0.000000in}{0.000000in}}{%
\pgfpathmoveto{\pgfqpoint{0.000000in}{0.000000in}}%
\pgfpathlineto{\pgfqpoint{-0.027778in}{0.000000in}}%
\pgfusepath{stroke,fill}%
}%
\begin{pgfscope}%
\pgfsys@transformshift{0.571156in}{2.471732in}%
\pgfsys@useobject{currentmarker}{}%
\end{pgfscope}%
\end{pgfscope}%
\begin{pgfscope}%
\pgfsetbuttcap%
\pgfsetroundjoin%
\definecolor{currentfill}{rgb}{0.000000,0.000000,0.000000}%
\pgfsetfillcolor{currentfill}%
\pgfsetlinewidth{0.602250pt}%
\definecolor{currentstroke}{rgb}{0.000000,0.000000,0.000000}%
\pgfsetstrokecolor{currentstroke}%
\pgfsetdash{}{0pt}%
\pgfsys@defobject{currentmarker}{\pgfqpoint{-0.027778in}{0.000000in}}{\pgfqpoint{0.000000in}{0.000000in}}{%
\pgfpathmoveto{\pgfqpoint{0.000000in}{0.000000in}}%
\pgfpathlineto{\pgfqpoint{-0.027778in}{0.000000in}}%
\pgfusepath{stroke,fill}%
}%
\begin{pgfscope}%
\pgfsys@transformshift{0.571156in}{2.539469in}%
\pgfsys@useobject{currentmarker}{}%
\end{pgfscope}%
\end{pgfscope}%
\begin{pgfscope}%
\pgfsetbuttcap%
\pgfsetroundjoin%
\definecolor{currentfill}{rgb}{0.000000,0.000000,0.000000}%
\pgfsetfillcolor{currentfill}%
\pgfsetlinewidth{0.602250pt}%
\definecolor{currentstroke}{rgb}{0.000000,0.000000,0.000000}%
\pgfsetstrokecolor{currentstroke}%
\pgfsetdash{}{0pt}%
\pgfsys@defobject{currentmarker}{\pgfqpoint{-0.027778in}{0.000000in}}{\pgfqpoint{0.000000in}{0.000000in}}{%
\pgfpathmoveto{\pgfqpoint{0.000000in}{0.000000in}}%
\pgfpathlineto{\pgfqpoint{-0.027778in}{0.000000in}}%
\pgfusepath{stroke,fill}%
}%
\begin{pgfscope}%
\pgfsys@transformshift{0.571156in}{2.587529in}%
\pgfsys@useobject{currentmarker}{}%
\end{pgfscope}%
\end{pgfscope}%
\begin{pgfscope}%
\pgfsetbuttcap%
\pgfsetroundjoin%
\definecolor{currentfill}{rgb}{0.000000,0.000000,0.000000}%
\pgfsetfillcolor{currentfill}%
\pgfsetlinewidth{0.602250pt}%
\definecolor{currentstroke}{rgb}{0.000000,0.000000,0.000000}%
\pgfsetstrokecolor{currentstroke}%
\pgfsetdash{}{0pt}%
\pgfsys@defobject{currentmarker}{\pgfqpoint{-0.027778in}{0.000000in}}{\pgfqpoint{0.000000in}{0.000000in}}{%
\pgfpathmoveto{\pgfqpoint{0.000000in}{0.000000in}}%
\pgfpathlineto{\pgfqpoint{-0.027778in}{0.000000in}}%
\pgfusepath{stroke,fill}%
}%
\begin{pgfscope}%
\pgfsys@transformshift{0.571156in}{2.624807in}%
\pgfsys@useobject{currentmarker}{}%
\end{pgfscope}%
\end{pgfscope}%
\begin{pgfscope}%
\pgfsetbuttcap%
\pgfsetroundjoin%
\definecolor{currentfill}{rgb}{0.000000,0.000000,0.000000}%
\pgfsetfillcolor{currentfill}%
\pgfsetlinewidth{0.602250pt}%
\definecolor{currentstroke}{rgb}{0.000000,0.000000,0.000000}%
\pgfsetstrokecolor{currentstroke}%
\pgfsetdash{}{0pt}%
\pgfsys@defobject{currentmarker}{\pgfqpoint{-0.027778in}{0.000000in}}{\pgfqpoint{0.000000in}{0.000000in}}{%
\pgfpathmoveto{\pgfqpoint{0.000000in}{0.000000in}}%
\pgfpathlineto{\pgfqpoint{-0.027778in}{0.000000in}}%
\pgfusepath{stroke,fill}%
}%
\begin{pgfscope}%
\pgfsys@transformshift{0.571156in}{2.655265in}%
\pgfsys@useobject{currentmarker}{}%
\end{pgfscope}%
\end{pgfscope}%
\begin{pgfscope}%
\pgfsetbuttcap%
\pgfsetroundjoin%
\definecolor{currentfill}{rgb}{0.000000,0.000000,0.000000}%
\pgfsetfillcolor{currentfill}%
\pgfsetlinewidth{0.602250pt}%
\definecolor{currentstroke}{rgb}{0.000000,0.000000,0.000000}%
\pgfsetstrokecolor{currentstroke}%
\pgfsetdash{}{0pt}%
\pgfsys@defobject{currentmarker}{\pgfqpoint{-0.027778in}{0.000000in}}{\pgfqpoint{0.000000in}{0.000000in}}{%
\pgfpathmoveto{\pgfqpoint{0.000000in}{0.000000in}}%
\pgfpathlineto{\pgfqpoint{-0.027778in}{0.000000in}}%
\pgfusepath{stroke,fill}%
}%
\begin{pgfscope}%
\pgfsys@transformshift{0.571156in}{2.681017in}%
\pgfsys@useobject{currentmarker}{}%
\end{pgfscope}%
\end{pgfscope}%
\begin{pgfscope}%
\pgfsetbuttcap%
\pgfsetroundjoin%
\definecolor{currentfill}{rgb}{0.000000,0.000000,0.000000}%
\pgfsetfillcolor{currentfill}%
\pgfsetlinewidth{0.602250pt}%
\definecolor{currentstroke}{rgb}{0.000000,0.000000,0.000000}%
\pgfsetstrokecolor{currentstroke}%
\pgfsetdash{}{0pt}%
\pgfsys@defobject{currentmarker}{\pgfqpoint{-0.027778in}{0.000000in}}{\pgfqpoint{0.000000in}{0.000000in}}{%
\pgfpathmoveto{\pgfqpoint{0.000000in}{0.000000in}}%
\pgfpathlineto{\pgfqpoint{-0.027778in}{0.000000in}}%
\pgfusepath{stroke,fill}%
}%
\begin{pgfscope}%
\pgfsys@transformshift{0.571156in}{2.703325in}%
\pgfsys@useobject{currentmarker}{}%
\end{pgfscope}%
\end{pgfscope}%
\begin{pgfscope}%
\pgfsetbuttcap%
\pgfsetroundjoin%
\definecolor{currentfill}{rgb}{0.000000,0.000000,0.000000}%
\pgfsetfillcolor{currentfill}%
\pgfsetlinewidth{0.602250pt}%
\definecolor{currentstroke}{rgb}{0.000000,0.000000,0.000000}%
\pgfsetstrokecolor{currentstroke}%
\pgfsetdash{}{0pt}%
\pgfsys@defobject{currentmarker}{\pgfqpoint{-0.027778in}{0.000000in}}{\pgfqpoint{0.000000in}{0.000000in}}{%
\pgfpathmoveto{\pgfqpoint{0.000000in}{0.000000in}}%
\pgfpathlineto{\pgfqpoint{-0.027778in}{0.000000in}}%
\pgfusepath{stroke,fill}%
}%
\begin{pgfscope}%
\pgfsys@transformshift{0.571156in}{2.723002in}%
\pgfsys@useobject{currentmarker}{}%
\end{pgfscope}%
\end{pgfscope}%
\begin{pgfscope}%
\pgfsetbuttcap%
\pgfsetroundjoin%
\definecolor{currentfill}{rgb}{0.000000,0.000000,0.000000}%
\pgfsetfillcolor{currentfill}%
\pgfsetlinewidth{0.602250pt}%
\definecolor{currentstroke}{rgb}{0.000000,0.000000,0.000000}%
\pgfsetstrokecolor{currentstroke}%
\pgfsetdash{}{0pt}%
\pgfsys@defobject{currentmarker}{\pgfqpoint{-0.027778in}{0.000000in}}{\pgfqpoint{0.000000in}{0.000000in}}{%
\pgfpathmoveto{\pgfqpoint{0.000000in}{0.000000in}}%
\pgfpathlineto{\pgfqpoint{-0.027778in}{0.000000in}}%
\pgfusepath{stroke,fill}%
}%
\begin{pgfscope}%
\pgfsys@transformshift{0.571156in}{2.856399in}%
\pgfsys@useobject{currentmarker}{}%
\end{pgfscope}%
\end{pgfscope}%
\begin{pgfscope}%
\pgfsetbuttcap%
\pgfsetroundjoin%
\definecolor{currentfill}{rgb}{0.000000,0.000000,0.000000}%
\pgfsetfillcolor{currentfill}%
\pgfsetlinewidth{0.602250pt}%
\definecolor{currentstroke}{rgb}{0.000000,0.000000,0.000000}%
\pgfsetstrokecolor{currentstroke}%
\pgfsetdash{}{0pt}%
\pgfsys@defobject{currentmarker}{\pgfqpoint{-0.027778in}{0.000000in}}{\pgfqpoint{0.000000in}{0.000000in}}{%
\pgfpathmoveto{\pgfqpoint{0.000000in}{0.000000in}}%
\pgfpathlineto{\pgfqpoint{-0.027778in}{0.000000in}}%
\pgfusepath{stroke,fill}%
}%
\begin{pgfscope}%
\pgfsys@transformshift{0.571156in}{2.924136in}%
\pgfsys@useobject{currentmarker}{}%
\end{pgfscope}%
\end{pgfscope}%
\begin{pgfscope}%
\pgfsetbuttcap%
\pgfsetroundjoin%
\definecolor{currentfill}{rgb}{0.000000,0.000000,0.000000}%
\pgfsetfillcolor{currentfill}%
\pgfsetlinewidth{0.602250pt}%
\definecolor{currentstroke}{rgb}{0.000000,0.000000,0.000000}%
\pgfsetstrokecolor{currentstroke}%
\pgfsetdash{}{0pt}%
\pgfsys@defobject{currentmarker}{\pgfqpoint{-0.027778in}{0.000000in}}{\pgfqpoint{0.000000in}{0.000000in}}{%
\pgfpathmoveto{\pgfqpoint{0.000000in}{0.000000in}}%
\pgfpathlineto{\pgfqpoint{-0.027778in}{0.000000in}}%
\pgfusepath{stroke,fill}%
}%
\begin{pgfscope}%
\pgfsys@transformshift{0.571156in}{2.972196in}%
\pgfsys@useobject{currentmarker}{}%
\end{pgfscope}%
\end{pgfscope}%
\begin{pgfscope}%
\pgfsetbuttcap%
\pgfsetroundjoin%
\definecolor{currentfill}{rgb}{0.000000,0.000000,0.000000}%
\pgfsetfillcolor{currentfill}%
\pgfsetlinewidth{0.602250pt}%
\definecolor{currentstroke}{rgb}{0.000000,0.000000,0.000000}%
\pgfsetstrokecolor{currentstroke}%
\pgfsetdash{}{0pt}%
\pgfsys@defobject{currentmarker}{\pgfqpoint{-0.027778in}{0.000000in}}{\pgfqpoint{0.000000in}{0.000000in}}{%
\pgfpathmoveto{\pgfqpoint{0.000000in}{0.000000in}}%
\pgfpathlineto{\pgfqpoint{-0.027778in}{0.000000in}}%
\pgfusepath{stroke,fill}%
}%
\begin{pgfscope}%
\pgfsys@transformshift{0.571156in}{3.009474in}%
\pgfsys@useobject{currentmarker}{}%
\end{pgfscope}%
\end{pgfscope}%
\begin{pgfscope}%
\pgfsetbuttcap%
\pgfsetroundjoin%
\definecolor{currentfill}{rgb}{0.000000,0.000000,0.000000}%
\pgfsetfillcolor{currentfill}%
\pgfsetlinewidth{0.602250pt}%
\definecolor{currentstroke}{rgb}{0.000000,0.000000,0.000000}%
\pgfsetstrokecolor{currentstroke}%
\pgfsetdash{}{0pt}%
\pgfsys@defobject{currentmarker}{\pgfqpoint{-0.027778in}{0.000000in}}{\pgfqpoint{0.000000in}{0.000000in}}{%
\pgfpathmoveto{\pgfqpoint{0.000000in}{0.000000in}}%
\pgfpathlineto{\pgfqpoint{-0.027778in}{0.000000in}}%
\pgfusepath{stroke,fill}%
}%
\begin{pgfscope}%
\pgfsys@transformshift{0.571156in}{3.039932in}%
\pgfsys@useobject{currentmarker}{}%
\end{pgfscope}%
\end{pgfscope}%
\begin{pgfscope}%
\pgftext[x=0.130375in,y=1.725694in,,bottom,rotate=90.000000]{\rmfamily\fontsize{10.000000}{12.000000}\selectfont E(t)}%
\end{pgfscope}%
\begin{pgfscope}%
\pgfpathrectangle{\pgfqpoint{0.571156in}{0.402778in}}{\pgfqpoint{4.380233in}{2.645833in}}%
\pgfusepath{clip}%
\pgfsetrectcap%
\pgfsetroundjoin%
\pgfsetlinewidth{1.505625pt}%
\definecolor{currentstroke}{rgb}{0.121569,0.466667,0.705882}%
\pgfsetstrokecolor{currentstroke}%
\pgfsetdash{}{0pt}%
\pgfpathmoveto{\pgfqpoint{0.774145in}{0.388889in}}%
\pgfpathlineto{\pgfqpoint{0.774146in}{0.523043in}}%
\pgfpathlineto{\pgfqpoint{0.778035in}{0.573048in}}%
\pgfpathlineto{\pgfqpoint{0.781923in}{0.579029in}}%
\pgfpathlineto{\pgfqpoint{0.785812in}{0.578017in}}%
\pgfpathlineto{\pgfqpoint{0.789701in}{0.579978in}}%
\pgfpathlineto{\pgfqpoint{0.797478in}{0.587286in}}%
\pgfpathlineto{\pgfqpoint{0.816922in}{0.596948in}}%
\pgfpathlineto{\pgfqpoint{0.995802in}{0.686354in}}%
\pgfpathlineto{\pgfqpoint{1.205792in}{0.790642in}}%
\pgfpathlineto{\pgfqpoint{1.209680in}{2.279404in}}%
\pgfpathlineto{\pgfqpoint{1.373006in}{2.359043in}}%
\pgfpathlineto{\pgfqpoint{1.586885in}{2.460694in}}%
\pgfpathlineto{\pgfqpoint{1.645215in}{2.487656in}}%
\pgfpathlineto{\pgfqpoint{1.649104in}{2.632763in}}%
\pgfpathlineto{\pgfqpoint{1.765765in}{2.683184in}}%
\pgfpathlineto{\pgfqpoint{1.882426in}{2.730483in}}%
\pgfpathlineto{\pgfqpoint{1.983532in}{2.768225in}}%
\pgfpathlineto{\pgfqpoint{2.072972in}{2.798490in}}%
\pgfpathlineto{\pgfqpoint{2.088527in}{2.803411in}}%
\pgfpathlineto{\pgfqpoint{2.092416in}{2.857807in}}%
\pgfpathlineto{\pgfqpoint{2.170190in}{2.878334in}}%
\pgfpathlineto{\pgfqpoint{2.244075in}{2.894801in}}%
\pgfpathlineto{\pgfqpoint{2.317960in}{2.908057in}}%
\pgfpathlineto{\pgfqpoint{2.387957in}{2.917456in}}%
\pgfpathlineto{\pgfqpoint{2.457954in}{2.923634in}}%
\pgfpathlineto{\pgfqpoint{2.524062in}{2.926422in}}%
\pgfpathlineto{\pgfqpoint{2.531839in}{2.926553in}}%
\pgfpathlineto{\pgfqpoint{2.535728in}{2.928277in}}%
\pgfpathlineto{\pgfqpoint{2.601836in}{2.927433in}}%
\pgfpathlineto{\pgfqpoint{2.667943in}{2.923608in}}%
\pgfpathlineto{\pgfqpoint{2.737940in}{2.916364in}}%
\pgfpathlineto{\pgfqpoint{2.807937in}{2.905948in}}%
\pgfpathlineto{\pgfqpoint{2.881822in}{2.891683in}}%
\pgfpathlineto{\pgfqpoint{2.955707in}{2.874287in}}%
\pgfpathlineto{\pgfqpoint{2.975151in}{2.869221in}}%
\pgfpathlineto{\pgfqpoint{2.979040in}{2.856428in}}%
\pgfpathlineto{\pgfqpoint{3.060702in}{2.833486in}}%
\pgfpathlineto{\pgfqpoint{3.146254in}{2.806141in}}%
\pgfpathlineto{\pgfqpoint{3.239583in}{2.772979in}}%
\pgfpathlineto{\pgfqpoint{3.344578in}{2.732223in}}%
\pgfpathlineto{\pgfqpoint{3.418463in}{2.701745in}}%
\pgfpathlineto{\pgfqpoint{3.422352in}{2.682863in}}%
\pgfpathlineto{\pgfqpoint{3.558456in}{2.624011in}}%
\pgfpathlineto{\pgfqpoint{3.710115in}{2.554874in}}%
\pgfpathlineto{\pgfqpoint{3.861775in}{2.483189in}}%
\pgfpathlineto{\pgfqpoint{3.865663in}{2.457139in}}%
\pgfpathlineto{\pgfqpoint{4.157316in}{2.315313in}}%
\pgfpathlineto{\pgfqpoint{4.305087in}{2.242213in}}%
\pgfpathlineto{\pgfqpoint{4.308975in}{2.204629in}}%
\pgfpathlineto{\pgfqpoint{4.752287in}{1.983237in}}%
\pgfpathlineto{\pgfqpoint{4.752287in}{1.983237in}}%
\pgfusepath{stroke}%
\end{pgfscope}%
\begin{pgfscope}%
\pgfpathrectangle{\pgfqpoint{0.571156in}{0.402778in}}{\pgfqpoint{4.380233in}{2.645833in}}%
\pgfusepath{clip}%
\pgfsetrectcap%
\pgfsetroundjoin%
\pgfsetlinewidth{1.505625pt}%
\definecolor{currentstroke}{rgb}{1.000000,0.498039,0.054902}%
\pgfsetstrokecolor{currentstroke}%
\pgfsetdash{}{0pt}%
\pgfpathmoveto{\pgfqpoint{0.774145in}{0.388889in}}%
\pgfpathlineto{\pgfqpoint{0.774146in}{0.523043in}}%
\pgfpathlineto{\pgfqpoint{0.778035in}{0.573048in}}%
\pgfpathlineto{\pgfqpoint{0.781923in}{0.579029in}}%
\pgfpathlineto{\pgfqpoint{0.785812in}{0.578017in}}%
\pgfpathlineto{\pgfqpoint{0.789701in}{0.579978in}}%
\pgfpathlineto{\pgfqpoint{0.797478in}{0.587286in}}%
\pgfpathlineto{\pgfqpoint{0.816922in}{0.596948in}}%
\pgfpathlineto{\pgfqpoint{0.995802in}{0.686354in}}%
\pgfpathlineto{\pgfqpoint{1.209680in}{0.792563in}}%
\pgfpathlineto{\pgfqpoint{1.213569in}{0.885108in}}%
\pgfpathlineto{\pgfqpoint{1.217458in}{0.908848in}}%
\pgfpathlineto{\pgfqpoint{1.221347in}{0.912752in}}%
\pgfpathlineto{\pgfqpoint{1.229124in}{0.915137in}}%
\pgfpathlineto{\pgfqpoint{1.240790in}{0.922737in}}%
\pgfpathlineto{\pgfqpoint{1.310787in}{0.957004in}}%
\pgfpathlineto{\pgfqpoint{1.575218in}{1.085087in}}%
\pgfpathlineto{\pgfqpoint{1.645215in}{1.118193in}}%
\pgfpathlineto{\pgfqpoint{1.649104in}{2.352860in}}%
\pgfpathlineto{\pgfqpoint{1.897981in}{2.467495in}}%
\pgfpathlineto{\pgfqpoint{2.030196in}{2.524502in}}%
\pgfpathlineto{\pgfqpoint{2.088527in}{2.548408in}}%
\pgfpathlineto{\pgfqpoint{2.092416in}{2.610235in}}%
\pgfpathlineto{\pgfqpoint{2.212965in}{2.657345in}}%
\pgfpathlineto{\pgfqpoint{2.306294in}{2.690484in}}%
\pgfpathlineto{\pgfqpoint{2.391846in}{2.717702in}}%
\pgfpathlineto{\pgfqpoint{2.469620in}{2.739370in}}%
\pgfpathlineto{\pgfqpoint{2.531839in}{2.754321in}}%
\pgfpathlineto{\pgfqpoint{2.535728in}{2.643128in}}%
\pgfpathlineto{\pgfqpoint{2.613502in}{2.656099in}}%
\pgfpathlineto{\pgfqpoint{2.679610in}{2.663848in}}%
\pgfpathlineto{\pgfqpoint{2.745718in}{2.668413in}}%
\pgfpathlineto{\pgfqpoint{2.807937in}{2.669724in}}%
\pgfpathlineto{\pgfqpoint{2.870156in}{2.668115in}}%
\pgfpathlineto{\pgfqpoint{2.936264in}{2.663238in}}%
\pgfpathlineto{\pgfqpoint{2.975151in}{2.658880in}}%
\pgfpathlineto{\pgfqpoint{2.979040in}{2.574655in}}%
\pgfpathlineto{\pgfqpoint{3.041259in}{2.564589in}}%
\pgfpathlineto{\pgfqpoint{3.111255in}{2.550207in}}%
\pgfpathlineto{\pgfqpoint{3.185141in}{2.531828in}}%
\pgfpathlineto{\pgfqpoint{3.262915in}{2.509301in}}%
\pgfpathlineto{\pgfqpoint{3.348466in}{2.481235in}}%
\pgfpathlineto{\pgfqpoint{3.418463in}{2.456071in}}%
\pgfpathlineto{\pgfqpoint{3.422352in}{2.509819in}}%
\pgfpathlineto{\pgfqpoint{3.496237in}{2.481640in}}%
\pgfpathlineto{\pgfqpoint{3.609009in}{2.435533in}}%
\pgfpathlineto{\pgfqpoint{3.737336in}{2.379632in}}%
\pgfpathlineto{\pgfqpoint{3.861775in}{2.322878in}}%
\pgfpathlineto{\pgfqpoint{3.865663in}{2.401419in}}%
\pgfpathlineto{\pgfqpoint{3.978436in}{2.348683in}}%
\pgfpathlineto{\pgfqpoint{4.196203in}{2.243858in}}%
\pgfpathlineto{\pgfqpoint{4.305087in}{2.190505in}}%
\pgfpathlineto{\pgfqpoint{4.308975in}{2.255493in}}%
\pgfpathlineto{\pgfqpoint{4.480078in}{2.171024in}}%
\pgfpathlineto{\pgfqpoint{4.752287in}{2.035343in}}%
\pgfpathlineto{\pgfqpoint{4.752287in}{2.035343in}}%
\pgfusepath{stroke}%
\end{pgfscope}%
\begin{pgfscope}%
\pgfpathrectangle{\pgfqpoint{0.571156in}{0.402778in}}{\pgfqpoint{4.380233in}{2.645833in}}%
\pgfusepath{clip}%
\pgfsetrectcap%
\pgfsetroundjoin%
\pgfsetlinewidth{1.505625pt}%
\definecolor{currentstroke}{rgb}{0.172549,0.627451,0.172549}%
\pgfsetstrokecolor{currentstroke}%
\pgfsetdash{}{0pt}%
\pgfpathmoveto{\pgfqpoint{0.774145in}{0.388889in}}%
\pgfpathlineto{\pgfqpoint{0.774146in}{0.523043in}}%
\pgfpathlineto{\pgfqpoint{0.778035in}{0.573048in}}%
\pgfpathlineto{\pgfqpoint{0.781923in}{0.579029in}}%
\pgfpathlineto{\pgfqpoint{0.785812in}{0.578017in}}%
\pgfpathlineto{\pgfqpoint{0.789701in}{0.579978in}}%
\pgfpathlineto{\pgfqpoint{0.797478in}{0.587286in}}%
\pgfpathlineto{\pgfqpoint{0.816922in}{0.596948in}}%
\pgfpathlineto{\pgfqpoint{0.995802in}{0.686354in}}%
\pgfpathlineto{\pgfqpoint{1.209680in}{0.792563in}}%
\pgfpathlineto{\pgfqpoint{1.213569in}{0.885108in}}%
\pgfpathlineto{\pgfqpoint{1.217458in}{0.908848in}}%
\pgfpathlineto{\pgfqpoint{1.221347in}{0.912752in}}%
\pgfpathlineto{\pgfqpoint{1.229124in}{0.915137in}}%
\pgfpathlineto{\pgfqpoint{1.240790in}{0.922737in}}%
\pgfpathlineto{\pgfqpoint{1.310787in}{0.957004in}}%
\pgfpathlineto{\pgfqpoint{1.575218in}{1.085087in}}%
\pgfpathlineto{\pgfqpoint{1.649104in}{1.120019in}}%
\pgfpathlineto{\pgfqpoint{1.652992in}{1.171257in}}%
\pgfpathlineto{\pgfqpoint{1.656881in}{1.186749in}}%
\pgfpathlineto{\pgfqpoint{1.660770in}{1.189835in}}%
\pgfpathlineto{\pgfqpoint{1.668547in}{1.192535in}}%
\pgfpathlineto{\pgfqpoint{1.684102in}{1.200916in}}%
\pgfpathlineto{\pgfqpoint{1.975755in}{1.330516in}}%
\pgfpathlineto{\pgfqpoint{2.088527in}{1.375894in}}%
\pgfpathlineto{\pgfqpoint{2.092416in}{2.306671in}}%
\pgfpathlineto{\pgfqpoint{2.135191in}{2.322849in}}%
\pgfpathlineto{\pgfqpoint{2.228520in}{2.355422in}}%
\pgfpathlineto{\pgfqpoint{2.314072in}{2.381986in}}%
\pgfpathlineto{\pgfqpoint{2.391846in}{2.402929in}}%
\pgfpathlineto{\pgfqpoint{2.465731in}{2.419614in}}%
\pgfpathlineto{\pgfqpoint{2.531839in}{2.431616in}}%
\pgfpathlineto{\pgfqpoint{2.535728in}{2.277381in}}%
\pgfpathlineto{\pgfqpoint{2.570726in}{2.282792in}}%
\pgfpathlineto{\pgfqpoint{2.636834in}{2.290319in}}%
\pgfpathlineto{\pgfqpoint{2.702942in}{2.294641in}}%
\pgfpathlineto{\pgfqpoint{2.765161in}{2.295706in}}%
\pgfpathlineto{\pgfqpoint{2.827380in}{2.293840in}}%
\pgfpathlineto{\pgfqpoint{2.893488in}{2.288681in}}%
\pgfpathlineto{\pgfqpoint{2.959596in}{2.280349in}}%
\pgfpathlineto{\pgfqpoint{2.975151in}{2.277943in}}%
\pgfpathlineto{\pgfqpoint{2.979040in}{2.180028in}}%
\pgfpathlineto{\pgfqpoint{3.006260in}{2.175803in}}%
\pgfpathlineto{\pgfqpoint{3.076257in}{2.161886in}}%
\pgfpathlineto{\pgfqpoint{3.150142in}{2.143948in}}%
\pgfpathlineto{\pgfqpoint{3.227916in}{2.121830in}}%
\pgfpathlineto{\pgfqpoint{3.313468in}{2.094148in}}%
\pgfpathlineto{\pgfqpoint{3.406797in}{2.060555in}}%
\pgfpathlineto{\pgfqpoint{3.418463in}{2.056138in}}%
\pgfpathlineto{\pgfqpoint{3.422352in}{2.019022in}}%
\pgfpathlineto{\pgfqpoint{3.434018in}{2.015488in}}%
\pgfpathlineto{\pgfqpoint{3.562345in}{1.963668in}}%
\pgfpathlineto{\pgfqpoint{3.690672in}{1.908032in}}%
\pgfpathlineto{\pgfqpoint{3.842331in}{1.838820in}}%
\pgfpathlineto{\pgfqpoint{3.861775in}{1.829741in}}%
\pgfpathlineto{\pgfqpoint{3.865663in}{1.920848in}}%
\pgfpathlineto{\pgfqpoint{3.885107in}{1.911082in}}%
\pgfpathlineto{\pgfqpoint{4.130095in}{1.793424in}}%
\pgfpathlineto{\pgfqpoint{4.305087in}{1.707488in}}%
\pgfpathlineto{\pgfqpoint{4.308975in}{1.970108in}}%
\pgfpathlineto{\pgfqpoint{4.386749in}{1.931444in}}%
\pgfpathlineto{\pgfqpoint{4.752287in}{1.748985in}}%
\pgfpathlineto{\pgfqpoint{4.752287in}{1.748985in}}%
\pgfusepath{stroke}%
\end{pgfscope}%
\begin{pgfscope}%
\pgfpathrectangle{\pgfqpoint{0.571156in}{0.402778in}}{\pgfqpoint{4.380233in}{2.645833in}}%
\pgfusepath{clip}%
\pgfsetrectcap%
\pgfsetroundjoin%
\pgfsetlinewidth{1.505625pt}%
\definecolor{currentstroke}{rgb}{0.839216,0.152941,0.156863}%
\pgfsetstrokecolor{currentstroke}%
\pgfsetdash{}{0pt}%
\pgfpathmoveto{\pgfqpoint{0.774145in}{0.388889in}}%
\pgfpathlineto{\pgfqpoint{0.774146in}{0.523043in}}%
\pgfpathlineto{\pgfqpoint{0.778035in}{0.573048in}}%
\pgfpathlineto{\pgfqpoint{0.781923in}{0.579029in}}%
\pgfpathlineto{\pgfqpoint{0.785812in}{0.578017in}}%
\pgfpathlineto{\pgfqpoint{0.789701in}{0.579978in}}%
\pgfpathlineto{\pgfqpoint{0.797478in}{0.587286in}}%
\pgfpathlineto{\pgfqpoint{0.816922in}{0.596948in}}%
\pgfpathlineto{\pgfqpoint{0.995802in}{0.686354in}}%
\pgfpathlineto{\pgfqpoint{1.209680in}{0.792563in}}%
\pgfpathlineto{\pgfqpoint{1.213569in}{0.885108in}}%
\pgfpathlineto{\pgfqpoint{1.217458in}{0.908848in}}%
\pgfpathlineto{\pgfqpoint{1.221347in}{0.912752in}}%
\pgfpathlineto{\pgfqpoint{1.229124in}{0.915137in}}%
\pgfpathlineto{\pgfqpoint{1.240790in}{0.922737in}}%
\pgfpathlineto{\pgfqpoint{1.310787in}{0.957004in}}%
\pgfpathlineto{\pgfqpoint{1.575218in}{1.085087in}}%
\pgfpathlineto{\pgfqpoint{1.649104in}{1.120019in}}%
\pgfpathlineto{\pgfqpoint{1.652992in}{1.171257in}}%
\pgfpathlineto{\pgfqpoint{1.656881in}{1.186749in}}%
\pgfpathlineto{\pgfqpoint{1.660770in}{1.189835in}}%
\pgfpathlineto{\pgfqpoint{1.668547in}{1.192535in}}%
\pgfpathlineto{\pgfqpoint{1.684102in}{1.200916in}}%
\pgfpathlineto{\pgfqpoint{1.975755in}{1.330516in}}%
\pgfpathlineto{\pgfqpoint{2.088527in}{1.375894in}}%
\pgfpathlineto{\pgfqpoint{2.092416in}{1.377394in}}%
\pgfpathlineto{\pgfqpoint{2.096304in}{1.408143in}}%
\pgfpathlineto{\pgfqpoint{2.100193in}{1.418311in}}%
\pgfpathlineto{\pgfqpoint{2.107970in}{1.421447in}}%
\pgfpathlineto{\pgfqpoint{2.216854in}{1.461450in}}%
\pgfpathlineto{\pgfqpoint{2.302406in}{1.488931in}}%
\pgfpathlineto{\pgfqpoint{2.380180in}{1.510837in}}%
\pgfpathlineto{\pgfqpoint{2.454065in}{1.528547in}}%
\pgfpathlineto{\pgfqpoint{2.524062in}{1.542228in}}%
\pgfpathlineto{\pgfqpoint{2.531839in}{1.543550in}}%
\pgfpathlineto{\pgfqpoint{2.535728in}{2.080283in}}%
\pgfpathlineto{\pgfqpoint{2.551282in}{2.083317in}}%
\pgfpathlineto{\pgfqpoint{2.617390in}{2.091925in}}%
\pgfpathlineto{\pgfqpoint{2.683498in}{2.097357in}}%
\pgfpathlineto{\pgfqpoint{2.745718in}{2.099488in}}%
\pgfpathlineto{\pgfqpoint{2.807937in}{2.098687in}}%
\pgfpathlineto{\pgfqpoint{2.874045in}{2.094641in}}%
\pgfpathlineto{\pgfqpoint{2.940153in}{2.087381in}}%
\pgfpathlineto{\pgfqpoint{2.975151in}{2.082281in}}%
\pgfpathlineto{\pgfqpoint{2.979040in}{2.125314in}}%
\pgfpathlineto{\pgfqpoint{3.002372in}{2.120925in}}%
\pgfpathlineto{\pgfqpoint{3.072368in}{2.106980in}}%
\pgfpathlineto{\pgfqpoint{3.146254in}{2.089019in}}%
\pgfpathlineto{\pgfqpoint{3.224028in}{2.066880in}}%
\pgfpathlineto{\pgfqpoint{3.309579in}{2.039177in}}%
\pgfpathlineto{\pgfqpoint{3.402908in}{2.005565in}}%
\pgfpathlineto{\pgfqpoint{3.418463in}{1.999664in}}%
\pgfpathlineto{\pgfqpoint{3.422352in}{2.010087in}}%
\pgfpathlineto{\pgfqpoint{3.434018in}{2.004606in}}%
\pgfpathlineto{\pgfqpoint{3.523458in}{1.968838in}}%
\pgfpathlineto{\pgfqpoint{3.644008in}{1.917530in}}%
\pgfpathlineto{\pgfqpoint{3.787889in}{1.852824in}}%
\pgfpathlineto{\pgfqpoint{3.861775in}{1.818516in}}%
\pgfpathlineto{\pgfqpoint{3.865663in}{1.864170in}}%
\pgfpathlineto{\pgfqpoint{3.881218in}{1.855976in}}%
\pgfpathlineto{\pgfqpoint{4.091208in}{1.755421in}}%
\pgfpathlineto{\pgfqpoint{4.305087in}{1.650570in}}%
\pgfpathlineto{\pgfqpoint{4.308975in}{1.633953in}}%
\pgfpathlineto{\pgfqpoint{4.320642in}{1.627138in}}%
\pgfpathlineto{\pgfqpoint{4.542298in}{1.517004in}}%
\pgfpathlineto{\pgfqpoint{4.752287in}{1.412011in}}%
\pgfpathlineto{\pgfqpoint{4.752287in}{1.412011in}}%
\pgfusepath{stroke}%
\end{pgfscope}%
\begin{pgfscope}%
\pgfpathrectangle{\pgfqpoint{0.571156in}{0.402778in}}{\pgfqpoint{4.380233in}{2.645833in}}%
\pgfusepath{clip}%
\pgfsetrectcap%
\pgfsetroundjoin%
\pgfsetlinewidth{1.505625pt}%
\definecolor{currentstroke}{rgb}{0.580392,0.403922,0.741176}%
\pgfsetstrokecolor{currentstroke}%
\pgfsetdash{}{0pt}%
\pgfpathmoveto{\pgfqpoint{0.774145in}{0.388889in}}%
\pgfpathlineto{\pgfqpoint{0.774146in}{0.523043in}}%
\pgfpathlineto{\pgfqpoint{0.778035in}{0.573048in}}%
\pgfpathlineto{\pgfqpoint{0.781923in}{0.579029in}}%
\pgfpathlineto{\pgfqpoint{0.785812in}{0.578017in}}%
\pgfpathlineto{\pgfqpoint{0.789701in}{0.579978in}}%
\pgfpathlineto{\pgfqpoint{0.797478in}{0.587286in}}%
\pgfpathlineto{\pgfqpoint{0.816922in}{0.596948in}}%
\pgfpathlineto{\pgfqpoint{0.995802in}{0.686354in}}%
\pgfpathlineto{\pgfqpoint{1.209680in}{0.792563in}}%
\pgfpathlineto{\pgfqpoint{1.213569in}{0.885108in}}%
\pgfpathlineto{\pgfqpoint{1.217458in}{0.908848in}}%
\pgfpathlineto{\pgfqpoint{1.221347in}{0.912752in}}%
\pgfpathlineto{\pgfqpoint{1.229124in}{0.915137in}}%
\pgfpathlineto{\pgfqpoint{1.240790in}{0.922737in}}%
\pgfpathlineto{\pgfqpoint{1.310787in}{0.957004in}}%
\pgfpathlineto{\pgfqpoint{1.575218in}{1.085087in}}%
\pgfpathlineto{\pgfqpoint{1.649104in}{1.120019in}}%
\pgfpathlineto{\pgfqpoint{1.652992in}{1.171257in}}%
\pgfpathlineto{\pgfqpoint{1.656881in}{1.186749in}}%
\pgfpathlineto{\pgfqpoint{1.660770in}{1.189835in}}%
\pgfpathlineto{\pgfqpoint{1.668547in}{1.192535in}}%
\pgfpathlineto{\pgfqpoint{1.684102in}{1.200916in}}%
\pgfpathlineto{\pgfqpoint{1.975755in}{1.330516in}}%
\pgfpathlineto{\pgfqpoint{2.088527in}{1.375894in}}%
\pgfpathlineto{\pgfqpoint{2.092416in}{1.377394in}}%
\pgfpathlineto{\pgfqpoint{2.096304in}{1.408143in}}%
\pgfpathlineto{\pgfqpoint{2.100193in}{1.418311in}}%
\pgfpathlineto{\pgfqpoint{2.107970in}{1.421447in}}%
\pgfpathlineto{\pgfqpoint{2.216854in}{1.461450in}}%
\pgfpathlineto{\pgfqpoint{2.302406in}{1.488931in}}%
\pgfpathlineto{\pgfqpoint{2.380180in}{1.510837in}}%
\pgfpathlineto{\pgfqpoint{2.454065in}{1.528547in}}%
\pgfpathlineto{\pgfqpoint{2.524062in}{1.542228in}}%
\pgfpathlineto{\pgfqpoint{2.535728in}{1.544196in}}%
\pgfpathlineto{\pgfqpoint{2.539616in}{1.555213in}}%
\pgfpathlineto{\pgfqpoint{2.543505in}{1.559089in}}%
\pgfpathlineto{\pgfqpoint{2.566837in}{1.562969in}}%
\pgfpathlineto{\pgfqpoint{2.640723in}{1.571590in}}%
\pgfpathlineto{\pgfqpoint{2.706830in}{1.575946in}}%
\pgfpathlineto{\pgfqpoint{2.769050in}{1.577044in}}%
\pgfpathlineto{\pgfqpoint{2.831269in}{1.575209in}}%
\pgfpathlineto{\pgfqpoint{2.897377in}{1.570084in}}%
\pgfpathlineto{\pgfqpoint{2.963485in}{1.561784in}}%
\pgfpathlineto{\pgfqpoint{2.975151in}{1.560000in}}%
\pgfpathlineto{\pgfqpoint{2.979040in}{1.788841in}}%
\pgfpathlineto{\pgfqpoint{2.986817in}{1.784247in}}%
\pgfpathlineto{\pgfqpoint{3.095701in}{1.762329in}}%
\pgfpathlineto{\pgfqpoint{3.169586in}{1.743451in}}%
\pgfpathlineto{\pgfqpoint{3.247360in}{1.720455in}}%
\pgfpathlineto{\pgfqpoint{3.332911in}{1.691944in}}%
\pgfpathlineto{\pgfqpoint{3.418463in}{1.660575in}}%
\pgfpathlineto{\pgfqpoint{3.422352in}{1.619559in}}%
\pgfpathlineto{\pgfqpoint{3.426240in}{1.610367in}}%
\pgfpathlineto{\pgfqpoint{3.430129in}{1.606213in}}%
\pgfpathlineto{\pgfqpoint{3.465127in}{1.592268in}}%
\pgfpathlineto{\pgfqpoint{3.581788in}{1.544299in}}%
\pgfpathlineto{\pgfqpoint{3.714004in}{1.486386in}}%
\pgfpathlineto{\pgfqpoint{3.861775in}{1.418488in}}%
\pgfpathlineto{\pgfqpoint{3.865663in}{1.327521in}}%
\pgfpathlineto{\pgfqpoint{3.869552in}{1.341636in}}%
\pgfpathlineto{\pgfqpoint{3.873441in}{1.344809in}}%
\pgfpathlineto{\pgfqpoint{3.892884in}{1.336196in}}%
\pgfpathlineto{\pgfqpoint{4.001768in}{1.284368in}}%
\pgfpathlineto{\pgfqpoint{4.250645in}{1.163198in}}%
\pgfpathlineto{\pgfqpoint{4.305087in}{1.136352in}}%
\pgfpathlineto{\pgfqpoint{4.308975in}{1.425410in}}%
\pgfpathlineto{\pgfqpoint{4.312864in}{1.426191in}}%
\pgfpathlineto{\pgfqpoint{4.320642in}{1.423313in}}%
\pgfpathlineto{\pgfqpoint{4.417859in}{1.375156in}}%
\pgfpathlineto{\pgfqpoint{4.752287in}{1.208229in}}%
\pgfpathlineto{\pgfqpoint{4.752287in}{1.208229in}}%
\pgfusepath{stroke}%
\end{pgfscope}%
\begin{pgfscope}%
\pgfpathrectangle{\pgfqpoint{0.571156in}{0.402778in}}{\pgfqpoint{4.380233in}{2.645833in}}%
\pgfusepath{clip}%
\pgfsetrectcap%
\pgfsetroundjoin%
\pgfsetlinewidth{1.505625pt}%
\definecolor{currentstroke}{rgb}{0.549020,0.337255,0.294118}%
\pgfsetstrokecolor{currentstroke}%
\pgfsetdash{}{0pt}%
\pgfpathmoveto{\pgfqpoint{0.774145in}{0.388889in}}%
\pgfpathlineto{\pgfqpoint{0.774146in}{0.523043in}}%
\pgfpathlineto{\pgfqpoint{0.778035in}{0.573048in}}%
\pgfpathlineto{\pgfqpoint{0.781923in}{0.579029in}}%
\pgfpathlineto{\pgfqpoint{0.785812in}{0.578017in}}%
\pgfpathlineto{\pgfqpoint{0.789701in}{0.579978in}}%
\pgfpathlineto{\pgfqpoint{0.797478in}{0.587286in}}%
\pgfpathlineto{\pgfqpoint{0.816922in}{0.596948in}}%
\pgfpathlineto{\pgfqpoint{0.995802in}{0.686354in}}%
\pgfpathlineto{\pgfqpoint{1.209680in}{0.792563in}}%
\pgfpathlineto{\pgfqpoint{1.213569in}{0.885108in}}%
\pgfpathlineto{\pgfqpoint{1.217458in}{0.908848in}}%
\pgfpathlineto{\pgfqpoint{1.221347in}{0.912752in}}%
\pgfpathlineto{\pgfqpoint{1.229124in}{0.915137in}}%
\pgfpathlineto{\pgfqpoint{1.240790in}{0.922737in}}%
\pgfpathlineto{\pgfqpoint{1.310787in}{0.957004in}}%
\pgfpathlineto{\pgfqpoint{1.575218in}{1.085087in}}%
\pgfpathlineto{\pgfqpoint{1.649104in}{1.120019in}}%
\pgfpathlineto{\pgfqpoint{1.652992in}{1.171257in}}%
\pgfpathlineto{\pgfqpoint{1.656881in}{1.186749in}}%
\pgfpathlineto{\pgfqpoint{1.660770in}{1.189835in}}%
\pgfpathlineto{\pgfqpoint{1.668547in}{1.192535in}}%
\pgfpathlineto{\pgfqpoint{1.684102in}{1.200916in}}%
\pgfpathlineto{\pgfqpoint{1.975755in}{1.330516in}}%
\pgfpathlineto{\pgfqpoint{2.088527in}{1.375894in}}%
\pgfpathlineto{\pgfqpoint{2.092416in}{1.377394in}}%
\pgfpathlineto{\pgfqpoint{2.096304in}{1.408143in}}%
\pgfpathlineto{\pgfqpoint{2.100193in}{1.418311in}}%
\pgfpathlineto{\pgfqpoint{2.107970in}{1.421447in}}%
\pgfpathlineto{\pgfqpoint{2.216854in}{1.461450in}}%
\pgfpathlineto{\pgfqpoint{2.302406in}{1.488931in}}%
\pgfpathlineto{\pgfqpoint{2.380180in}{1.510837in}}%
\pgfpathlineto{\pgfqpoint{2.454065in}{1.528547in}}%
\pgfpathlineto{\pgfqpoint{2.524062in}{1.542228in}}%
\pgfpathlineto{\pgfqpoint{2.535728in}{1.544196in}}%
\pgfpathlineto{\pgfqpoint{2.539616in}{1.555213in}}%
\pgfpathlineto{\pgfqpoint{2.543505in}{1.559089in}}%
\pgfpathlineto{\pgfqpoint{2.566837in}{1.562969in}}%
\pgfpathlineto{\pgfqpoint{2.640723in}{1.571590in}}%
\pgfpathlineto{\pgfqpoint{2.706830in}{1.575946in}}%
\pgfpathlineto{\pgfqpoint{2.769050in}{1.577044in}}%
\pgfpathlineto{\pgfqpoint{2.831269in}{1.575209in}}%
\pgfpathlineto{\pgfqpoint{2.897377in}{1.570084in}}%
\pgfpathlineto{\pgfqpoint{2.963485in}{1.561784in}}%
\pgfpathlineto{\pgfqpoint{2.979040in}{1.559385in}}%
\pgfpathlineto{\pgfqpoint{2.982928in}{1.548760in}}%
\pgfpathlineto{\pgfqpoint{2.986817in}{1.544577in}}%
\pgfpathlineto{\pgfqpoint{3.010149in}{1.540105in}}%
\pgfpathlineto{\pgfqpoint{3.072368in}{1.527663in}}%
\pgfpathlineto{\pgfqpoint{3.146254in}{1.509780in}}%
\pgfpathlineto{\pgfqpoint{3.224028in}{1.487711in}}%
\pgfpathlineto{\pgfqpoint{3.309579in}{1.460077in}}%
\pgfpathlineto{\pgfqpoint{3.402908in}{1.426526in}}%
\pgfpathlineto{\pgfqpoint{3.418463in}{1.420634in}}%
\pgfpathlineto{\pgfqpoint{3.422352in}{1.401567in}}%
\pgfpathlineto{\pgfqpoint{3.426240in}{1.369715in}}%
\pgfpathlineto{\pgfqpoint{3.430129in}{1.356685in}}%
\pgfpathlineto{\pgfqpoint{3.434018in}{1.354132in}}%
\pgfpathlineto{\pgfqpoint{3.441795in}{1.352025in}}%
\pgfpathlineto{\pgfqpoint{3.457350in}{1.344764in}}%
\pgfpathlineto{\pgfqpoint{3.519569in}{1.319792in}}%
\pgfpathlineto{\pgfqpoint{3.640119in}{1.268636in}}%
\pgfpathlineto{\pgfqpoint{3.784001in}{1.204059in}}%
\pgfpathlineto{\pgfqpoint{3.861775in}{1.167980in}}%
\pgfpathlineto{\pgfqpoint{3.873441in}{0.772111in}}%
\pgfpathlineto{\pgfqpoint{3.877330in}{0.756880in}}%
\pgfpathlineto{\pgfqpoint{3.881218in}{0.767763in}}%
\pgfpathlineto{\pgfqpoint{3.885107in}{0.765236in}}%
\pgfpathlineto{\pgfqpoint{3.892884in}{0.745709in}}%
\pgfpathlineto{\pgfqpoint{3.896773in}{0.742488in}}%
\pgfpathlineto{\pgfqpoint{3.908439in}{0.739288in}}%
\pgfpathlineto{\pgfqpoint{3.939549in}{0.724126in}}%
\pgfpathlineto{\pgfqpoint{4.165093in}{0.615220in}}%
\pgfpathlineto{\pgfqpoint{4.305087in}{0.546404in}}%
\pgfpathlineto{\pgfqpoint{4.308975in}{0.897686in}}%
\pgfpathlineto{\pgfqpoint{4.312864in}{0.815273in}}%
\pgfpathlineto{\pgfqpoint{4.316753in}{0.774731in}}%
\pgfpathlineto{\pgfqpoint{4.320642in}{0.769112in}}%
\pgfpathlineto{\pgfqpoint{4.324530in}{0.770736in}}%
\pgfpathlineto{\pgfqpoint{4.328419in}{0.768602in}}%
\pgfpathlineto{\pgfqpoint{4.336196in}{0.760362in}}%
\pgfpathlineto{\pgfqpoint{4.351751in}{0.752913in}}%
\pgfpathlineto{\pgfqpoint{4.452857in}{0.702625in}}%
\pgfpathlineto{\pgfqpoint{4.752287in}{0.553095in}}%
\pgfpathlineto{\pgfqpoint{4.752287in}{0.553095in}}%
\pgfusepath{stroke}%
\end{pgfscope}%
\begin{pgfscope}%
\pgfpathrectangle{\pgfqpoint{0.571156in}{0.402778in}}{\pgfqpoint{4.380233in}{2.645833in}}%
\pgfusepath{clip}%
\pgfsetrectcap%
\pgfsetroundjoin%
\pgfsetlinewidth{1.505625pt}%
\definecolor{currentstroke}{rgb}{0.890196,0.466667,0.760784}%
\pgfsetstrokecolor{currentstroke}%
\pgfsetdash{}{0pt}%
\pgfpathmoveto{\pgfqpoint{0.774145in}{0.388889in}}%
\pgfpathlineto{\pgfqpoint{0.774146in}{0.523043in}}%
\pgfpathlineto{\pgfqpoint{0.778035in}{0.573048in}}%
\pgfpathlineto{\pgfqpoint{0.781923in}{0.579029in}}%
\pgfpathlineto{\pgfqpoint{0.785812in}{0.578017in}}%
\pgfpathlineto{\pgfqpoint{0.789701in}{0.579978in}}%
\pgfpathlineto{\pgfqpoint{0.797478in}{0.587286in}}%
\pgfpathlineto{\pgfqpoint{0.816922in}{0.596948in}}%
\pgfpathlineto{\pgfqpoint{0.995802in}{0.686354in}}%
\pgfpathlineto{\pgfqpoint{1.209680in}{0.792563in}}%
\pgfpathlineto{\pgfqpoint{1.213569in}{0.885108in}}%
\pgfpathlineto{\pgfqpoint{1.217458in}{0.908848in}}%
\pgfpathlineto{\pgfqpoint{1.221347in}{0.912752in}}%
\pgfpathlineto{\pgfqpoint{1.229124in}{0.915137in}}%
\pgfpathlineto{\pgfqpoint{1.240790in}{0.922737in}}%
\pgfpathlineto{\pgfqpoint{1.310787in}{0.957004in}}%
\pgfpathlineto{\pgfqpoint{1.575218in}{1.085087in}}%
\pgfpathlineto{\pgfqpoint{1.649104in}{1.120019in}}%
\pgfpathlineto{\pgfqpoint{1.652992in}{1.171257in}}%
\pgfpathlineto{\pgfqpoint{1.656881in}{1.186749in}}%
\pgfpathlineto{\pgfqpoint{1.660770in}{1.189835in}}%
\pgfpathlineto{\pgfqpoint{1.668547in}{1.192535in}}%
\pgfpathlineto{\pgfqpoint{1.684102in}{1.200916in}}%
\pgfpathlineto{\pgfqpoint{1.975755in}{1.330516in}}%
\pgfpathlineto{\pgfqpoint{2.088527in}{1.375894in}}%
\pgfpathlineto{\pgfqpoint{2.092416in}{1.377394in}}%
\pgfpathlineto{\pgfqpoint{2.096304in}{1.408143in}}%
\pgfpathlineto{\pgfqpoint{2.100193in}{1.418311in}}%
\pgfpathlineto{\pgfqpoint{2.107970in}{1.421447in}}%
\pgfpathlineto{\pgfqpoint{2.216854in}{1.461450in}}%
\pgfpathlineto{\pgfqpoint{2.302406in}{1.488931in}}%
\pgfpathlineto{\pgfqpoint{2.380180in}{1.510837in}}%
\pgfpathlineto{\pgfqpoint{2.454065in}{1.528547in}}%
\pgfpathlineto{\pgfqpoint{2.524062in}{1.542228in}}%
\pgfpathlineto{\pgfqpoint{2.535728in}{1.544196in}}%
\pgfpathlineto{\pgfqpoint{2.539616in}{1.555213in}}%
\pgfpathlineto{\pgfqpoint{2.543505in}{1.559089in}}%
\pgfpathlineto{\pgfqpoint{2.566837in}{1.562969in}}%
\pgfpathlineto{\pgfqpoint{2.640723in}{1.571590in}}%
\pgfpathlineto{\pgfqpoint{2.706830in}{1.575946in}}%
\pgfpathlineto{\pgfqpoint{2.769050in}{1.577044in}}%
\pgfpathlineto{\pgfqpoint{2.831269in}{1.575209in}}%
\pgfpathlineto{\pgfqpoint{2.897377in}{1.570084in}}%
\pgfpathlineto{\pgfqpoint{2.963485in}{1.561784in}}%
\pgfpathlineto{\pgfqpoint{2.979040in}{1.559385in}}%
\pgfpathlineto{\pgfqpoint{2.982928in}{1.548760in}}%
\pgfpathlineto{\pgfqpoint{2.986817in}{1.544577in}}%
\pgfpathlineto{\pgfqpoint{3.010149in}{1.540105in}}%
\pgfpathlineto{\pgfqpoint{3.072368in}{1.527663in}}%
\pgfpathlineto{\pgfqpoint{3.146254in}{1.509780in}}%
\pgfpathlineto{\pgfqpoint{3.224028in}{1.487711in}}%
\pgfpathlineto{\pgfqpoint{3.309579in}{1.460077in}}%
\pgfpathlineto{\pgfqpoint{3.402908in}{1.426526in}}%
\pgfpathlineto{\pgfqpoint{3.422352in}{1.419149in}}%
\pgfpathlineto{\pgfqpoint{3.426240in}{1.390590in}}%
\pgfpathlineto{\pgfqpoint{3.430129in}{1.378957in}}%
\pgfpathlineto{\pgfqpoint{3.437906in}{1.375863in}}%
\pgfpathlineto{\pgfqpoint{3.546790in}{1.330959in}}%
\pgfpathlineto{\pgfqpoint{3.671228in}{1.277388in}}%
\pgfpathlineto{\pgfqpoint{3.818999in}{1.210342in}}%
\pgfpathlineto{\pgfqpoint{3.861775in}{1.190412in}}%
\pgfpathlineto{\pgfqpoint{3.865663in}{1.191893in}}%
\pgfpathlineto{\pgfqpoint{3.869552in}{1.147431in}}%
\pgfpathlineto{\pgfqpoint{3.873441in}{1.128556in}}%
\pgfpathlineto{\pgfqpoint{3.877330in}{1.125198in}}%
\pgfpathlineto{\pgfqpoint{3.885107in}{1.122909in}}%
\pgfpathlineto{\pgfqpoint{3.896773in}{1.115476in}}%
\pgfpathlineto{\pgfqpoint{3.947326in}{1.091726in}}%
\pgfpathlineto{\pgfqpoint{4.176760in}{0.980801in}}%
\pgfpathlineto{\pgfqpoint{4.305087in}{0.917684in}}%
\pgfpathlineto{\pgfqpoint{4.308975in}{0.944071in}}%
\pgfpathlineto{\pgfqpoint{4.312864in}{0.885022in}}%
\pgfpathlineto{\pgfqpoint{4.316753in}{0.858689in}}%
\pgfpathlineto{\pgfqpoint{4.320642in}{0.854540in}}%
\pgfpathlineto{\pgfqpoint{4.324530in}{0.854754in}}%
\pgfpathlineto{\pgfqpoint{4.332308in}{0.849243in}}%
\pgfpathlineto{\pgfqpoint{4.340085in}{0.844071in}}%
\pgfpathlineto{\pgfqpoint{4.378972in}{0.825112in}}%
\pgfpathlineto{\pgfqpoint{4.752287in}{0.638889in}}%
\pgfpathlineto{\pgfqpoint{4.752287in}{0.638889in}}%
\pgfusepath{stroke}%
\end{pgfscope}%
\begin{pgfscope}%
\pgfpathrectangle{\pgfqpoint{0.571156in}{0.402778in}}{\pgfqpoint{4.380233in}{2.645833in}}%
\pgfusepath{clip}%
\pgfsetrectcap%
\pgfsetroundjoin%
\pgfsetlinewidth{1.505625pt}%
\definecolor{currentstroke}{rgb}{0.498039,0.498039,0.498039}%
\pgfsetstrokecolor{currentstroke}%
\pgfsetdash{}{0pt}%
\pgfpathmoveto{\pgfqpoint{0.774145in}{0.388889in}}%
\pgfpathlineto{\pgfqpoint{0.774146in}{0.523043in}}%
\pgfpathlineto{\pgfqpoint{0.778035in}{0.573048in}}%
\pgfpathlineto{\pgfqpoint{0.781923in}{0.579029in}}%
\pgfpathlineto{\pgfqpoint{0.785812in}{0.578017in}}%
\pgfpathlineto{\pgfqpoint{0.789701in}{0.579978in}}%
\pgfpathlineto{\pgfqpoint{0.797478in}{0.587286in}}%
\pgfpathlineto{\pgfqpoint{0.816922in}{0.596948in}}%
\pgfpathlineto{\pgfqpoint{0.995802in}{0.686354in}}%
\pgfpathlineto{\pgfqpoint{1.209680in}{0.792563in}}%
\pgfpathlineto{\pgfqpoint{1.213569in}{0.885108in}}%
\pgfpathlineto{\pgfqpoint{1.217458in}{0.908848in}}%
\pgfpathlineto{\pgfqpoint{1.221347in}{0.912752in}}%
\pgfpathlineto{\pgfqpoint{1.229124in}{0.915137in}}%
\pgfpathlineto{\pgfqpoint{1.240790in}{0.922737in}}%
\pgfpathlineto{\pgfqpoint{1.310787in}{0.957004in}}%
\pgfpathlineto{\pgfqpoint{1.575218in}{1.085087in}}%
\pgfpathlineto{\pgfqpoint{1.649104in}{1.120019in}}%
\pgfpathlineto{\pgfqpoint{1.652992in}{1.171257in}}%
\pgfpathlineto{\pgfqpoint{1.656881in}{1.186749in}}%
\pgfpathlineto{\pgfqpoint{1.660770in}{1.189835in}}%
\pgfpathlineto{\pgfqpoint{1.668547in}{1.192535in}}%
\pgfpathlineto{\pgfqpoint{1.684102in}{1.200916in}}%
\pgfpathlineto{\pgfqpoint{1.975755in}{1.330516in}}%
\pgfpathlineto{\pgfqpoint{2.088527in}{1.375894in}}%
\pgfpathlineto{\pgfqpoint{2.092416in}{1.377394in}}%
\pgfpathlineto{\pgfqpoint{2.096304in}{1.408143in}}%
\pgfpathlineto{\pgfqpoint{2.100193in}{1.418311in}}%
\pgfpathlineto{\pgfqpoint{2.107970in}{1.421447in}}%
\pgfpathlineto{\pgfqpoint{2.216854in}{1.461450in}}%
\pgfpathlineto{\pgfqpoint{2.302406in}{1.488931in}}%
\pgfpathlineto{\pgfqpoint{2.380180in}{1.510837in}}%
\pgfpathlineto{\pgfqpoint{2.454065in}{1.528547in}}%
\pgfpathlineto{\pgfqpoint{2.524062in}{1.542228in}}%
\pgfpathlineto{\pgfqpoint{2.535728in}{1.544196in}}%
\pgfpathlineto{\pgfqpoint{2.539616in}{1.555213in}}%
\pgfpathlineto{\pgfqpoint{2.543505in}{1.559089in}}%
\pgfpathlineto{\pgfqpoint{2.566837in}{1.562969in}}%
\pgfpathlineto{\pgfqpoint{2.640723in}{1.571590in}}%
\pgfpathlineto{\pgfqpoint{2.706830in}{1.575946in}}%
\pgfpathlineto{\pgfqpoint{2.769050in}{1.577044in}}%
\pgfpathlineto{\pgfqpoint{2.831269in}{1.575209in}}%
\pgfpathlineto{\pgfqpoint{2.897377in}{1.570084in}}%
\pgfpathlineto{\pgfqpoint{2.963485in}{1.561784in}}%
\pgfpathlineto{\pgfqpoint{2.979040in}{1.559385in}}%
\pgfpathlineto{\pgfqpoint{2.982928in}{1.548760in}}%
\pgfpathlineto{\pgfqpoint{2.986817in}{1.544577in}}%
\pgfpathlineto{\pgfqpoint{3.010149in}{1.540105in}}%
\pgfpathlineto{\pgfqpoint{3.072368in}{1.527663in}}%
\pgfpathlineto{\pgfqpoint{3.146254in}{1.509780in}}%
\pgfpathlineto{\pgfqpoint{3.224028in}{1.487711in}}%
\pgfpathlineto{\pgfqpoint{3.309579in}{1.460077in}}%
\pgfpathlineto{\pgfqpoint{3.402908in}{1.426526in}}%
\pgfpathlineto{\pgfqpoint{3.422352in}{1.419149in}}%
\pgfpathlineto{\pgfqpoint{3.426240in}{1.390590in}}%
\pgfpathlineto{\pgfqpoint{3.430129in}{1.378957in}}%
\pgfpathlineto{\pgfqpoint{3.437906in}{1.375863in}}%
\pgfpathlineto{\pgfqpoint{3.546790in}{1.330959in}}%
\pgfpathlineto{\pgfqpoint{3.671228in}{1.277388in}}%
\pgfpathlineto{\pgfqpoint{3.818999in}{1.210342in}}%
\pgfpathlineto{\pgfqpoint{3.865663in}{1.188591in}}%
\pgfpathlineto{\pgfqpoint{3.869552in}{1.143156in}}%
\pgfpathlineto{\pgfqpoint{3.873441in}{1.123815in}}%
\pgfpathlineto{\pgfqpoint{3.877330in}{1.120413in}}%
\pgfpathlineto{\pgfqpoint{3.881218in}{1.120077in}}%
\pgfpathlineto{\pgfqpoint{3.892884in}{1.112662in}}%
\pgfpathlineto{\pgfqpoint{3.935660in}{1.092478in}}%
\pgfpathlineto{\pgfqpoint{4.165093in}{0.981713in}}%
\pgfpathlineto{\pgfqpoint{4.308975in}{0.909957in}}%
\pgfpathlineto{\pgfqpoint{4.312864in}{0.834713in}}%
\pgfpathlineto{\pgfqpoint{4.316753in}{0.798878in}}%
\pgfpathlineto{\pgfqpoint{4.320642in}{0.793761in}}%
\pgfpathlineto{\pgfqpoint{4.324530in}{0.794904in}}%
\pgfpathlineto{\pgfqpoint{4.328419in}{0.792798in}}%
\pgfpathlineto{\pgfqpoint{4.336196in}{0.785157in}}%
\pgfpathlineto{\pgfqpoint{4.355640in}{0.775684in}}%
\pgfpathlineto{\pgfqpoint{4.538409in}{0.684794in}}%
\pgfpathlineto{\pgfqpoint{4.752287in}{0.577870in}}%
\pgfpathlineto{\pgfqpoint{4.752287in}{0.577870in}}%
\pgfusepath{stroke}%
\end{pgfscope}%
\begin{pgfscope}%
\pgfpathrectangle{\pgfqpoint{0.571156in}{0.402778in}}{\pgfqpoint{4.380233in}{2.645833in}}%
\pgfusepath{clip}%
\pgfsetrectcap%
\pgfsetroundjoin%
\pgfsetlinewidth{1.505625pt}%
\definecolor{currentstroke}{rgb}{0.737255,0.741176,0.133333}%
\pgfsetstrokecolor{currentstroke}%
\pgfsetdash{}{0pt}%
\pgfpathmoveto{\pgfqpoint{0.774145in}{0.388889in}}%
\pgfpathlineto{\pgfqpoint{0.774146in}{0.523043in}}%
\pgfpathlineto{\pgfqpoint{0.778035in}{0.573048in}}%
\pgfpathlineto{\pgfqpoint{0.781923in}{0.579029in}}%
\pgfpathlineto{\pgfqpoint{0.785812in}{0.578017in}}%
\pgfpathlineto{\pgfqpoint{0.789701in}{0.579978in}}%
\pgfpathlineto{\pgfqpoint{0.797478in}{0.587286in}}%
\pgfpathlineto{\pgfqpoint{0.816922in}{0.596948in}}%
\pgfpathlineto{\pgfqpoint{0.995802in}{0.686354in}}%
\pgfpathlineto{\pgfqpoint{1.209680in}{0.792563in}}%
\pgfpathlineto{\pgfqpoint{1.213569in}{0.885108in}}%
\pgfpathlineto{\pgfqpoint{1.217458in}{0.908848in}}%
\pgfpathlineto{\pgfqpoint{1.221347in}{0.912752in}}%
\pgfpathlineto{\pgfqpoint{1.229124in}{0.915137in}}%
\pgfpathlineto{\pgfqpoint{1.240790in}{0.922737in}}%
\pgfpathlineto{\pgfqpoint{1.310787in}{0.957004in}}%
\pgfpathlineto{\pgfqpoint{1.575218in}{1.085087in}}%
\pgfpathlineto{\pgfqpoint{1.649104in}{1.120019in}}%
\pgfpathlineto{\pgfqpoint{1.652992in}{1.171257in}}%
\pgfpathlineto{\pgfqpoint{1.656881in}{1.186749in}}%
\pgfpathlineto{\pgfqpoint{1.660770in}{1.189835in}}%
\pgfpathlineto{\pgfqpoint{1.668547in}{1.192535in}}%
\pgfpathlineto{\pgfqpoint{1.684102in}{1.200916in}}%
\pgfpathlineto{\pgfqpoint{1.975755in}{1.330516in}}%
\pgfpathlineto{\pgfqpoint{2.088527in}{1.375894in}}%
\pgfpathlineto{\pgfqpoint{2.092416in}{1.377394in}}%
\pgfpathlineto{\pgfqpoint{2.096304in}{1.408143in}}%
\pgfpathlineto{\pgfqpoint{2.100193in}{1.418311in}}%
\pgfpathlineto{\pgfqpoint{2.107970in}{1.421447in}}%
\pgfpathlineto{\pgfqpoint{2.216854in}{1.461450in}}%
\pgfpathlineto{\pgfqpoint{2.302406in}{1.488931in}}%
\pgfpathlineto{\pgfqpoint{2.380180in}{1.510837in}}%
\pgfpathlineto{\pgfqpoint{2.454065in}{1.528547in}}%
\pgfpathlineto{\pgfqpoint{2.524062in}{1.542228in}}%
\pgfpathlineto{\pgfqpoint{2.535728in}{1.544196in}}%
\pgfpathlineto{\pgfqpoint{2.539616in}{1.555213in}}%
\pgfpathlineto{\pgfqpoint{2.543505in}{1.559089in}}%
\pgfpathlineto{\pgfqpoint{2.566837in}{1.562969in}}%
\pgfpathlineto{\pgfqpoint{2.640723in}{1.571590in}}%
\pgfpathlineto{\pgfqpoint{2.706830in}{1.575946in}}%
\pgfpathlineto{\pgfqpoint{2.769050in}{1.577044in}}%
\pgfpathlineto{\pgfqpoint{2.831269in}{1.575209in}}%
\pgfpathlineto{\pgfqpoint{2.897377in}{1.570084in}}%
\pgfpathlineto{\pgfqpoint{2.963485in}{1.561784in}}%
\pgfpathlineto{\pgfqpoint{2.979040in}{1.559385in}}%
\pgfpathlineto{\pgfqpoint{2.982928in}{1.548760in}}%
\pgfpathlineto{\pgfqpoint{2.986817in}{1.544577in}}%
\pgfpathlineto{\pgfqpoint{3.010149in}{1.540105in}}%
\pgfpathlineto{\pgfqpoint{3.072368in}{1.527663in}}%
\pgfpathlineto{\pgfqpoint{3.146254in}{1.509780in}}%
\pgfpathlineto{\pgfqpoint{3.224028in}{1.487711in}}%
\pgfpathlineto{\pgfqpoint{3.309579in}{1.460077in}}%
\pgfpathlineto{\pgfqpoint{3.402908in}{1.426526in}}%
\pgfpathlineto{\pgfqpoint{3.422352in}{1.419149in}}%
\pgfpathlineto{\pgfqpoint{3.426240in}{1.390590in}}%
\pgfpathlineto{\pgfqpoint{3.430129in}{1.378957in}}%
\pgfpathlineto{\pgfqpoint{3.437906in}{1.375863in}}%
\pgfpathlineto{\pgfqpoint{3.546790in}{1.330959in}}%
\pgfpathlineto{\pgfqpoint{3.671228in}{1.277388in}}%
\pgfpathlineto{\pgfqpoint{3.818999in}{1.210342in}}%
\pgfpathlineto{\pgfqpoint{3.865663in}{1.188591in}}%
\pgfpathlineto{\pgfqpoint{3.869552in}{1.143156in}}%
\pgfpathlineto{\pgfqpoint{3.873441in}{1.123815in}}%
\pgfpathlineto{\pgfqpoint{3.877330in}{1.120413in}}%
\pgfpathlineto{\pgfqpoint{3.881218in}{1.120077in}}%
\pgfpathlineto{\pgfqpoint{3.892884in}{1.112662in}}%
\pgfpathlineto{\pgfqpoint{3.935660in}{1.092478in}}%
\pgfpathlineto{\pgfqpoint{4.165093in}{0.981713in}}%
\pgfpathlineto{\pgfqpoint{4.308975in}{0.910968in}}%
\pgfpathlineto{\pgfqpoint{4.312864in}{0.836278in}}%
\pgfpathlineto{\pgfqpoint{4.316753in}{0.800793in}}%
\pgfpathlineto{\pgfqpoint{4.320642in}{0.795713in}}%
\pgfpathlineto{\pgfqpoint{4.324530in}{0.796820in}}%
\pgfpathlineto{\pgfqpoint{4.328419in}{0.794717in}}%
\pgfpathlineto{\pgfqpoint{4.336196in}{0.787119in}}%
\pgfpathlineto{\pgfqpoint{4.355640in}{0.777644in}}%
\pgfpathlineto{\pgfqpoint{4.542298in}{0.684815in}}%
\pgfpathlineto{\pgfqpoint{4.752287in}{0.579831in}}%
\pgfpathlineto{\pgfqpoint{4.752287in}{0.579831in}}%
\pgfusepath{stroke}%
\end{pgfscope}%
\begin{pgfscope}%
\pgfpathrectangle{\pgfqpoint{0.571156in}{0.402778in}}{\pgfqpoint{4.380233in}{2.645833in}}%
\pgfusepath{clip}%
\pgfsetrectcap%
\pgfsetroundjoin%
\pgfsetlinewidth{1.505625pt}%
\definecolor{currentstroke}{rgb}{0.090196,0.745098,0.811765}%
\pgfsetstrokecolor{currentstroke}%
\pgfsetdash{}{0pt}%
\pgfpathmoveto{\pgfqpoint{0.774145in}{0.388889in}}%
\pgfpathlineto{\pgfqpoint{0.774146in}{0.523043in}}%
\pgfpathlineto{\pgfqpoint{0.778035in}{0.573048in}}%
\pgfpathlineto{\pgfqpoint{0.781923in}{0.579029in}}%
\pgfpathlineto{\pgfqpoint{0.785812in}{0.578017in}}%
\pgfpathlineto{\pgfqpoint{0.789701in}{0.579978in}}%
\pgfpathlineto{\pgfqpoint{0.797478in}{0.587286in}}%
\pgfpathlineto{\pgfqpoint{0.816922in}{0.596948in}}%
\pgfpathlineto{\pgfqpoint{0.995802in}{0.686354in}}%
\pgfpathlineto{\pgfqpoint{1.209680in}{0.792563in}}%
\pgfpathlineto{\pgfqpoint{1.213569in}{0.885108in}}%
\pgfpathlineto{\pgfqpoint{1.217458in}{0.908848in}}%
\pgfpathlineto{\pgfqpoint{1.221347in}{0.912752in}}%
\pgfpathlineto{\pgfqpoint{1.229124in}{0.915137in}}%
\pgfpathlineto{\pgfqpoint{1.240790in}{0.922737in}}%
\pgfpathlineto{\pgfqpoint{1.310787in}{0.957004in}}%
\pgfpathlineto{\pgfqpoint{1.575218in}{1.085087in}}%
\pgfpathlineto{\pgfqpoint{1.649104in}{1.120019in}}%
\pgfpathlineto{\pgfqpoint{1.652992in}{1.171257in}}%
\pgfpathlineto{\pgfqpoint{1.656881in}{1.186749in}}%
\pgfpathlineto{\pgfqpoint{1.660770in}{1.189835in}}%
\pgfpathlineto{\pgfqpoint{1.668547in}{1.192535in}}%
\pgfpathlineto{\pgfqpoint{1.684102in}{1.200916in}}%
\pgfpathlineto{\pgfqpoint{1.975755in}{1.330516in}}%
\pgfpathlineto{\pgfqpoint{2.088527in}{1.375894in}}%
\pgfpathlineto{\pgfqpoint{2.092416in}{1.377394in}}%
\pgfpathlineto{\pgfqpoint{2.096304in}{1.408143in}}%
\pgfpathlineto{\pgfqpoint{2.100193in}{1.418311in}}%
\pgfpathlineto{\pgfqpoint{2.107970in}{1.421447in}}%
\pgfpathlineto{\pgfqpoint{2.216854in}{1.461450in}}%
\pgfpathlineto{\pgfqpoint{2.302406in}{1.488931in}}%
\pgfpathlineto{\pgfqpoint{2.380180in}{1.510837in}}%
\pgfpathlineto{\pgfqpoint{2.454065in}{1.528547in}}%
\pgfpathlineto{\pgfqpoint{2.524062in}{1.542228in}}%
\pgfpathlineto{\pgfqpoint{2.535728in}{1.544196in}}%
\pgfpathlineto{\pgfqpoint{2.539616in}{1.555213in}}%
\pgfpathlineto{\pgfqpoint{2.543505in}{1.559089in}}%
\pgfpathlineto{\pgfqpoint{2.566837in}{1.562969in}}%
\pgfpathlineto{\pgfqpoint{2.640723in}{1.571590in}}%
\pgfpathlineto{\pgfqpoint{2.706830in}{1.575946in}}%
\pgfpathlineto{\pgfqpoint{2.769050in}{1.577044in}}%
\pgfpathlineto{\pgfqpoint{2.831269in}{1.575209in}}%
\pgfpathlineto{\pgfqpoint{2.897377in}{1.570084in}}%
\pgfpathlineto{\pgfqpoint{2.963485in}{1.561784in}}%
\pgfpathlineto{\pgfqpoint{2.979040in}{1.559385in}}%
\pgfpathlineto{\pgfqpoint{2.982928in}{1.548760in}}%
\pgfpathlineto{\pgfqpoint{2.986817in}{1.544577in}}%
\pgfpathlineto{\pgfqpoint{3.010149in}{1.540105in}}%
\pgfpathlineto{\pgfqpoint{3.072368in}{1.527663in}}%
\pgfpathlineto{\pgfqpoint{3.146254in}{1.509780in}}%
\pgfpathlineto{\pgfqpoint{3.224028in}{1.487711in}}%
\pgfpathlineto{\pgfqpoint{3.309579in}{1.460077in}}%
\pgfpathlineto{\pgfqpoint{3.402908in}{1.426526in}}%
\pgfpathlineto{\pgfqpoint{3.422352in}{1.419149in}}%
\pgfpathlineto{\pgfqpoint{3.426240in}{1.390590in}}%
\pgfpathlineto{\pgfqpoint{3.430129in}{1.378957in}}%
\pgfpathlineto{\pgfqpoint{3.437906in}{1.375863in}}%
\pgfpathlineto{\pgfqpoint{3.546790in}{1.330959in}}%
\pgfpathlineto{\pgfqpoint{3.671228in}{1.277388in}}%
\pgfpathlineto{\pgfqpoint{3.818999in}{1.210342in}}%
\pgfpathlineto{\pgfqpoint{3.865663in}{1.188591in}}%
\pgfpathlineto{\pgfqpoint{3.869552in}{1.143156in}}%
\pgfpathlineto{\pgfqpoint{3.873441in}{1.123815in}}%
\pgfpathlineto{\pgfqpoint{3.877330in}{1.120413in}}%
\pgfpathlineto{\pgfqpoint{3.881218in}{1.120077in}}%
\pgfpathlineto{\pgfqpoint{3.892884in}{1.112662in}}%
\pgfpathlineto{\pgfqpoint{3.935660in}{1.092478in}}%
\pgfpathlineto{\pgfqpoint{4.165093in}{0.981713in}}%
\pgfpathlineto{\pgfqpoint{4.308975in}{0.910968in}}%
\pgfpathlineto{\pgfqpoint{4.312864in}{0.836278in}}%
\pgfpathlineto{\pgfqpoint{4.316753in}{0.800793in}}%
\pgfpathlineto{\pgfqpoint{4.320642in}{0.795713in}}%
\pgfpathlineto{\pgfqpoint{4.324530in}{0.796820in}}%
\pgfpathlineto{\pgfqpoint{4.328419in}{0.794717in}}%
\pgfpathlineto{\pgfqpoint{4.336196in}{0.787119in}}%
\pgfpathlineto{\pgfqpoint{4.355640in}{0.777644in}}%
\pgfpathlineto{\pgfqpoint{4.542298in}{0.684815in}}%
\pgfpathlineto{\pgfqpoint{4.752287in}{0.579831in}}%
\pgfpathlineto{\pgfqpoint{4.752287in}{0.579831in}}%
\pgfusepath{stroke}%
\end{pgfscope}%
\begin{pgfscope}%
\pgfpathrectangle{\pgfqpoint{0.571156in}{0.402778in}}{\pgfqpoint{4.380233in}{2.645833in}}%
\pgfusepath{clip}%
\pgfsetrectcap%
\pgfsetroundjoin%
\pgfsetlinewidth{1.505625pt}%
\definecolor{currentstroke}{rgb}{0.121569,0.466667,0.705882}%
\pgfsetstrokecolor{currentstroke}%
\pgfsetdash{}{0pt}%
\pgfpathmoveto{\pgfqpoint{0.774145in}{0.388889in}}%
\pgfpathlineto{\pgfqpoint{0.774146in}{0.523043in}}%
\pgfpathlineto{\pgfqpoint{0.778035in}{0.573048in}}%
\pgfpathlineto{\pgfqpoint{0.781923in}{0.579029in}}%
\pgfpathlineto{\pgfqpoint{0.785812in}{0.578017in}}%
\pgfpathlineto{\pgfqpoint{0.789701in}{0.579978in}}%
\pgfpathlineto{\pgfqpoint{0.797478in}{0.587286in}}%
\pgfpathlineto{\pgfqpoint{0.816922in}{0.596948in}}%
\pgfpathlineto{\pgfqpoint{0.995802in}{0.686354in}}%
\pgfpathlineto{\pgfqpoint{1.209680in}{0.792563in}}%
\pgfpathlineto{\pgfqpoint{1.213569in}{0.885108in}}%
\pgfpathlineto{\pgfqpoint{1.217458in}{0.908848in}}%
\pgfpathlineto{\pgfqpoint{1.221347in}{0.912752in}}%
\pgfpathlineto{\pgfqpoint{1.229124in}{0.915137in}}%
\pgfpathlineto{\pgfqpoint{1.240790in}{0.922737in}}%
\pgfpathlineto{\pgfqpoint{1.310787in}{0.957004in}}%
\pgfpathlineto{\pgfqpoint{1.575218in}{1.085087in}}%
\pgfpathlineto{\pgfqpoint{1.649104in}{1.120019in}}%
\pgfpathlineto{\pgfqpoint{1.652992in}{1.171257in}}%
\pgfpathlineto{\pgfqpoint{1.656881in}{1.186749in}}%
\pgfpathlineto{\pgfqpoint{1.660770in}{1.189835in}}%
\pgfpathlineto{\pgfqpoint{1.668547in}{1.192535in}}%
\pgfpathlineto{\pgfqpoint{1.684102in}{1.200916in}}%
\pgfpathlineto{\pgfqpoint{1.975755in}{1.330516in}}%
\pgfpathlineto{\pgfqpoint{2.088527in}{1.375894in}}%
\pgfpathlineto{\pgfqpoint{2.092416in}{1.377394in}}%
\pgfpathlineto{\pgfqpoint{2.096304in}{1.408143in}}%
\pgfpathlineto{\pgfqpoint{2.100193in}{1.418311in}}%
\pgfpathlineto{\pgfqpoint{2.107970in}{1.421447in}}%
\pgfpathlineto{\pgfqpoint{2.216854in}{1.461450in}}%
\pgfpathlineto{\pgfqpoint{2.302406in}{1.488931in}}%
\pgfpathlineto{\pgfqpoint{2.380180in}{1.510837in}}%
\pgfpathlineto{\pgfqpoint{2.454065in}{1.528547in}}%
\pgfpathlineto{\pgfqpoint{2.524062in}{1.542228in}}%
\pgfpathlineto{\pgfqpoint{2.535728in}{1.544196in}}%
\pgfpathlineto{\pgfqpoint{2.539616in}{1.555213in}}%
\pgfpathlineto{\pgfqpoint{2.543505in}{1.559089in}}%
\pgfpathlineto{\pgfqpoint{2.566837in}{1.562969in}}%
\pgfpathlineto{\pgfqpoint{2.640723in}{1.571590in}}%
\pgfpathlineto{\pgfqpoint{2.706830in}{1.575946in}}%
\pgfpathlineto{\pgfqpoint{2.769050in}{1.577044in}}%
\pgfpathlineto{\pgfqpoint{2.831269in}{1.575209in}}%
\pgfpathlineto{\pgfqpoint{2.897377in}{1.570084in}}%
\pgfpathlineto{\pgfqpoint{2.963485in}{1.561784in}}%
\pgfpathlineto{\pgfqpoint{2.979040in}{1.559385in}}%
\pgfpathlineto{\pgfqpoint{2.982928in}{1.548760in}}%
\pgfpathlineto{\pgfqpoint{2.986817in}{1.544577in}}%
\pgfpathlineto{\pgfqpoint{3.010149in}{1.540105in}}%
\pgfpathlineto{\pgfqpoint{3.072368in}{1.527663in}}%
\pgfpathlineto{\pgfqpoint{3.146254in}{1.509780in}}%
\pgfpathlineto{\pgfqpoint{3.224028in}{1.487711in}}%
\pgfpathlineto{\pgfqpoint{3.309579in}{1.460077in}}%
\pgfpathlineto{\pgfqpoint{3.402908in}{1.426526in}}%
\pgfpathlineto{\pgfqpoint{3.422352in}{1.419149in}}%
\pgfpathlineto{\pgfqpoint{3.426240in}{1.390590in}}%
\pgfpathlineto{\pgfqpoint{3.430129in}{1.378957in}}%
\pgfpathlineto{\pgfqpoint{3.437906in}{1.375863in}}%
\pgfpathlineto{\pgfqpoint{3.546790in}{1.330959in}}%
\pgfpathlineto{\pgfqpoint{3.671228in}{1.277388in}}%
\pgfpathlineto{\pgfqpoint{3.818999in}{1.210342in}}%
\pgfpathlineto{\pgfqpoint{3.865663in}{1.188591in}}%
\pgfpathlineto{\pgfqpoint{3.869552in}{1.143156in}}%
\pgfpathlineto{\pgfqpoint{3.873441in}{1.123815in}}%
\pgfpathlineto{\pgfqpoint{3.877330in}{1.120413in}}%
\pgfpathlineto{\pgfqpoint{3.881218in}{1.120077in}}%
\pgfpathlineto{\pgfqpoint{3.892884in}{1.112662in}}%
\pgfpathlineto{\pgfqpoint{3.935660in}{1.092478in}}%
\pgfpathlineto{\pgfqpoint{4.165093in}{0.981713in}}%
\pgfpathlineto{\pgfqpoint{4.308975in}{0.910968in}}%
\pgfpathlineto{\pgfqpoint{4.312864in}{0.836278in}}%
\pgfpathlineto{\pgfqpoint{4.316753in}{0.800793in}}%
\pgfpathlineto{\pgfqpoint{4.320642in}{0.795713in}}%
\pgfpathlineto{\pgfqpoint{4.324530in}{0.796820in}}%
\pgfpathlineto{\pgfqpoint{4.328419in}{0.794717in}}%
\pgfpathlineto{\pgfqpoint{4.336196in}{0.787119in}}%
\pgfpathlineto{\pgfqpoint{4.355640in}{0.777644in}}%
\pgfpathlineto{\pgfqpoint{4.542298in}{0.684815in}}%
\pgfpathlineto{\pgfqpoint{4.752287in}{0.579831in}}%
\pgfpathlineto{\pgfqpoint{4.752287in}{0.579831in}}%
\pgfusepath{stroke}%
\end{pgfscope}%
\begin{pgfscope}%
\pgfpathrectangle{\pgfqpoint{0.571156in}{0.402778in}}{\pgfqpoint{4.380233in}{2.645833in}}%
\pgfusepath{clip}%
\pgfsetrectcap%
\pgfsetroundjoin%
\pgfsetlinewidth{1.505625pt}%
\definecolor{currentstroke}{rgb}{1.000000,0.498039,0.054902}%
\pgfsetstrokecolor{currentstroke}%
\pgfsetdash{}{0pt}%
\pgfpathmoveto{\pgfqpoint{0.774145in}{0.388889in}}%
\pgfpathlineto{\pgfqpoint{0.774146in}{0.523043in}}%
\pgfpathlineto{\pgfqpoint{0.778035in}{0.573048in}}%
\pgfpathlineto{\pgfqpoint{0.781923in}{0.579029in}}%
\pgfpathlineto{\pgfqpoint{0.785812in}{0.578017in}}%
\pgfpathlineto{\pgfqpoint{0.789701in}{0.579978in}}%
\pgfpathlineto{\pgfqpoint{0.797478in}{0.587286in}}%
\pgfpathlineto{\pgfqpoint{0.816922in}{0.596948in}}%
\pgfpathlineto{\pgfqpoint{0.995802in}{0.686354in}}%
\pgfpathlineto{\pgfqpoint{1.209680in}{0.792563in}}%
\pgfpathlineto{\pgfqpoint{1.213569in}{0.885108in}}%
\pgfpathlineto{\pgfqpoint{1.217458in}{0.908848in}}%
\pgfpathlineto{\pgfqpoint{1.221347in}{0.912752in}}%
\pgfpathlineto{\pgfqpoint{1.229124in}{0.915137in}}%
\pgfpathlineto{\pgfqpoint{1.240790in}{0.922737in}}%
\pgfpathlineto{\pgfqpoint{1.310787in}{0.957004in}}%
\pgfpathlineto{\pgfqpoint{1.575218in}{1.085087in}}%
\pgfpathlineto{\pgfqpoint{1.649104in}{1.120019in}}%
\pgfpathlineto{\pgfqpoint{1.652992in}{1.171257in}}%
\pgfpathlineto{\pgfqpoint{1.656881in}{1.186749in}}%
\pgfpathlineto{\pgfqpoint{1.660770in}{1.189835in}}%
\pgfpathlineto{\pgfqpoint{1.668547in}{1.192535in}}%
\pgfpathlineto{\pgfqpoint{1.684102in}{1.200916in}}%
\pgfpathlineto{\pgfqpoint{1.975755in}{1.330516in}}%
\pgfpathlineto{\pgfqpoint{2.088527in}{1.375894in}}%
\pgfpathlineto{\pgfqpoint{2.092416in}{1.377394in}}%
\pgfpathlineto{\pgfqpoint{2.096304in}{1.408143in}}%
\pgfpathlineto{\pgfqpoint{2.100193in}{1.418311in}}%
\pgfpathlineto{\pgfqpoint{2.107970in}{1.421447in}}%
\pgfpathlineto{\pgfqpoint{2.216854in}{1.461450in}}%
\pgfpathlineto{\pgfqpoint{2.302406in}{1.488931in}}%
\pgfpathlineto{\pgfqpoint{2.380180in}{1.510837in}}%
\pgfpathlineto{\pgfqpoint{2.454065in}{1.528547in}}%
\pgfpathlineto{\pgfqpoint{2.524062in}{1.542228in}}%
\pgfpathlineto{\pgfqpoint{2.535728in}{1.544196in}}%
\pgfpathlineto{\pgfqpoint{2.539616in}{1.555213in}}%
\pgfpathlineto{\pgfqpoint{2.543505in}{1.559089in}}%
\pgfpathlineto{\pgfqpoint{2.566837in}{1.562969in}}%
\pgfpathlineto{\pgfqpoint{2.640723in}{1.571590in}}%
\pgfpathlineto{\pgfqpoint{2.706830in}{1.575946in}}%
\pgfpathlineto{\pgfqpoint{2.769050in}{1.577044in}}%
\pgfpathlineto{\pgfqpoint{2.831269in}{1.575209in}}%
\pgfpathlineto{\pgfqpoint{2.897377in}{1.570084in}}%
\pgfpathlineto{\pgfqpoint{2.963485in}{1.561784in}}%
\pgfpathlineto{\pgfqpoint{2.979040in}{1.559385in}}%
\pgfpathlineto{\pgfqpoint{2.982928in}{1.548760in}}%
\pgfpathlineto{\pgfqpoint{2.986817in}{1.544577in}}%
\pgfpathlineto{\pgfqpoint{3.010149in}{1.540105in}}%
\pgfpathlineto{\pgfqpoint{3.072368in}{1.527663in}}%
\pgfpathlineto{\pgfqpoint{3.146254in}{1.509780in}}%
\pgfpathlineto{\pgfqpoint{3.224028in}{1.487711in}}%
\pgfpathlineto{\pgfqpoint{3.309579in}{1.460077in}}%
\pgfpathlineto{\pgfqpoint{3.402908in}{1.426526in}}%
\pgfpathlineto{\pgfqpoint{3.422352in}{1.419149in}}%
\pgfpathlineto{\pgfqpoint{3.426240in}{1.390590in}}%
\pgfpathlineto{\pgfqpoint{3.430129in}{1.378957in}}%
\pgfpathlineto{\pgfqpoint{3.437906in}{1.375863in}}%
\pgfpathlineto{\pgfqpoint{3.546790in}{1.330959in}}%
\pgfpathlineto{\pgfqpoint{3.671228in}{1.277388in}}%
\pgfpathlineto{\pgfqpoint{3.818999in}{1.210342in}}%
\pgfpathlineto{\pgfqpoint{3.865663in}{1.188591in}}%
\pgfpathlineto{\pgfqpoint{3.869552in}{1.143156in}}%
\pgfpathlineto{\pgfqpoint{3.873441in}{1.123815in}}%
\pgfpathlineto{\pgfqpoint{3.877330in}{1.120413in}}%
\pgfpathlineto{\pgfqpoint{3.881218in}{1.120077in}}%
\pgfpathlineto{\pgfqpoint{3.892884in}{1.112662in}}%
\pgfpathlineto{\pgfqpoint{3.935660in}{1.092478in}}%
\pgfpathlineto{\pgfqpoint{4.165093in}{0.981713in}}%
\pgfpathlineto{\pgfqpoint{4.308975in}{0.910968in}}%
\pgfpathlineto{\pgfqpoint{4.312864in}{0.836278in}}%
\pgfpathlineto{\pgfqpoint{4.316753in}{0.800793in}}%
\pgfpathlineto{\pgfqpoint{4.320642in}{0.795713in}}%
\pgfpathlineto{\pgfqpoint{4.324530in}{0.796820in}}%
\pgfpathlineto{\pgfqpoint{4.328419in}{0.794717in}}%
\pgfpathlineto{\pgfqpoint{4.336196in}{0.787119in}}%
\pgfpathlineto{\pgfqpoint{4.355640in}{0.777644in}}%
\pgfpathlineto{\pgfqpoint{4.542298in}{0.684815in}}%
\pgfpathlineto{\pgfqpoint{4.752287in}{0.579831in}}%
\pgfpathlineto{\pgfqpoint{4.752287in}{0.579831in}}%
\pgfusepath{stroke}%
\end{pgfscope}%
\begin{pgfscope}%
\pgfsetrectcap%
\pgfsetmiterjoin%
\pgfsetlinewidth{0.803000pt}%
\definecolor{currentstroke}{rgb}{0.000000,0.000000,0.000000}%
\pgfsetstrokecolor{currentstroke}%
\pgfsetdash{}{0pt}%
\pgfpathmoveto{\pgfqpoint{0.571156in}{0.402778in}}%
\pgfpathlineto{\pgfqpoint{0.571156in}{3.048611in}}%
\pgfusepath{stroke}%
\end{pgfscope}%
\begin{pgfscope}%
\pgfsetrectcap%
\pgfsetmiterjoin%
\pgfsetlinewidth{0.803000pt}%
\definecolor{currentstroke}{rgb}{0.000000,0.000000,0.000000}%
\pgfsetstrokecolor{currentstroke}%
\pgfsetdash{}{0pt}%
\pgfpathmoveto{\pgfqpoint{4.951389in}{0.402778in}}%
\pgfpathlineto{\pgfqpoint{4.951389in}{3.048611in}}%
\pgfusepath{stroke}%
\end{pgfscope}%
\begin{pgfscope}%
\pgfsetrectcap%
\pgfsetmiterjoin%
\pgfsetlinewidth{0.803000pt}%
\definecolor{currentstroke}{rgb}{0.000000,0.000000,0.000000}%
\pgfsetstrokecolor{currentstroke}%
\pgfsetdash{}{0pt}%
\pgfpathmoveto{\pgfqpoint{0.571156in}{0.402778in}}%
\pgfpathlineto{\pgfqpoint{4.951389in}{0.402778in}}%
\pgfusepath{stroke}%
\end{pgfscope}%
\begin{pgfscope}%
\pgfsetrectcap%
\pgfsetmiterjoin%
\pgfsetlinewidth{0.803000pt}%
\definecolor{currentstroke}{rgb}{0.000000,0.000000,0.000000}%
\pgfsetstrokecolor{currentstroke}%
\pgfsetdash{}{0pt}%
\pgfpathmoveto{\pgfqpoint{0.571156in}{3.048611in}}%
\pgfpathlineto{\pgfqpoint{4.951389in}{3.048611in}}%
\pgfusepath{stroke}%
\end{pgfscope}%
\begin{pgfscope}%
\pgfsetbuttcap%
\pgfsetmiterjoin%
\definecolor{currentfill}{rgb}{1.000000,1.000000,1.000000}%
\pgfsetfillcolor{currentfill}%
\pgfsetfillopacity{0.800000}%
\pgfsetlinewidth{1.003750pt}%
\definecolor{currentstroke}{rgb}{0.800000,0.800000,0.800000}%
\pgfsetstrokecolor{currentstroke}%
\pgfsetstrokeopacity{0.800000}%
\pgfsetdash{}{0pt}%
\pgfpathmoveto{\pgfqpoint{0.668378in}{0.581057in}}%
\pgfpathlineto{\pgfqpoint{1.251711in}{0.581057in}}%
\pgfpathquadraticcurveto{\pgfqpoint{1.279489in}{0.581057in}}{\pgfqpoint{1.279489in}{0.608834in}}%
\pgfpathlineto{\pgfqpoint{1.279489in}{2.951389in}}%
\pgfpathquadraticcurveto{\pgfqpoint{1.279489in}{2.979167in}}{\pgfqpoint{1.251711in}{2.979167in}}%
\pgfpathlineto{\pgfqpoint{0.668378in}{2.979167in}}%
\pgfpathquadraticcurveto{\pgfqpoint{0.640600in}{2.979167in}}{\pgfqpoint{0.640600in}{2.951389in}}%
\pgfpathlineto{\pgfqpoint{0.640600in}{0.608834in}}%
\pgfpathquadraticcurveto{\pgfqpoint{0.640600in}{0.581057in}}{\pgfqpoint{0.668378in}{0.581057in}}%
\pgfpathclose%
\pgfusepath{stroke,fill}%
\end{pgfscope}%
\begin{pgfscope}%
\pgfsetrectcap%
\pgfsetroundjoin%
\pgfsetlinewidth{1.505625pt}%
\definecolor{currentstroke}{rgb}{0.121569,0.466667,0.705882}%
\pgfsetstrokecolor{currentstroke}%
\pgfsetdash{}{0pt}%
\pgfpathmoveto{\pgfqpoint{0.696156in}{2.875000in}}%
\pgfpathlineto{\pgfqpoint{0.973934in}{2.875000in}}%
\pgfusepath{stroke}%
\end{pgfscope}%
\begin{pgfscope}%
\pgftext[x=1.085045in,y=2.826389in,left,base]{\rmfamily\fontsize{10.000000}{12.000000}\selectfont 1}%
\end{pgfscope}%
\begin{pgfscope}%
\pgfsetrectcap%
\pgfsetroundjoin%
\pgfsetlinewidth{1.505625pt}%
\definecolor{currentstroke}{rgb}{1.000000,0.498039,0.054902}%
\pgfsetstrokecolor{currentstroke}%
\pgfsetdash{}{0pt}%
\pgfpathmoveto{\pgfqpoint{0.696156in}{2.678630in}}%
\pgfpathlineto{\pgfqpoint{0.973934in}{2.678630in}}%
\pgfusepath{stroke}%
\end{pgfscope}%
\begin{pgfscope}%
\pgftext[x=1.085045in,y=2.630019in,left,base]{\rmfamily\fontsize{10.000000}{12.000000}\selectfont 2}%
\end{pgfscope}%
\begin{pgfscope}%
\pgfsetrectcap%
\pgfsetroundjoin%
\pgfsetlinewidth{1.505625pt}%
\definecolor{currentstroke}{rgb}{0.172549,0.627451,0.172549}%
\pgfsetstrokecolor{currentstroke}%
\pgfsetdash{}{0pt}%
\pgfpathmoveto{\pgfqpoint{0.696156in}{2.482259in}}%
\pgfpathlineto{\pgfqpoint{0.973934in}{2.482259in}}%
\pgfusepath{stroke}%
\end{pgfscope}%
\begin{pgfscope}%
\pgftext[x=1.085045in,y=2.433648in,left,base]{\rmfamily\fontsize{10.000000}{12.000000}\selectfont 3}%
\end{pgfscope}%
\begin{pgfscope}%
\pgfsetrectcap%
\pgfsetroundjoin%
\pgfsetlinewidth{1.505625pt}%
\definecolor{currentstroke}{rgb}{0.839216,0.152941,0.156863}%
\pgfsetstrokecolor{currentstroke}%
\pgfsetdash{}{0pt}%
\pgfpathmoveto{\pgfqpoint{0.696156in}{2.285889in}}%
\pgfpathlineto{\pgfqpoint{0.973934in}{2.285889in}}%
\pgfusepath{stroke}%
\end{pgfscope}%
\begin{pgfscope}%
\pgftext[x=1.085045in,y=2.237278in,left,base]{\rmfamily\fontsize{10.000000}{12.000000}\selectfont 4}%
\end{pgfscope}%
\begin{pgfscope}%
\pgfsetrectcap%
\pgfsetroundjoin%
\pgfsetlinewidth{1.505625pt}%
\definecolor{currentstroke}{rgb}{0.580392,0.403922,0.741176}%
\pgfsetstrokecolor{currentstroke}%
\pgfsetdash{}{0pt}%
\pgfpathmoveto{\pgfqpoint{0.696156in}{2.089519in}}%
\pgfpathlineto{\pgfqpoint{0.973934in}{2.089519in}}%
\pgfusepath{stroke}%
\end{pgfscope}%
\begin{pgfscope}%
\pgftext[x=1.085045in,y=2.040908in,left,base]{\rmfamily\fontsize{10.000000}{12.000000}\selectfont 5}%
\end{pgfscope}%
\begin{pgfscope}%
\pgfsetrectcap%
\pgfsetroundjoin%
\pgfsetlinewidth{1.505625pt}%
\definecolor{currentstroke}{rgb}{0.549020,0.337255,0.294118}%
\pgfsetstrokecolor{currentstroke}%
\pgfsetdash{}{0pt}%
\pgfpathmoveto{\pgfqpoint{0.696156in}{1.893149in}}%
\pgfpathlineto{\pgfqpoint{0.973934in}{1.893149in}}%
\pgfusepath{stroke}%
\end{pgfscope}%
\begin{pgfscope}%
\pgftext[x=1.085045in,y=1.844537in,left,base]{\rmfamily\fontsize{10.000000}{12.000000}\selectfont 6}%
\end{pgfscope}%
\begin{pgfscope}%
\pgfsetrectcap%
\pgfsetroundjoin%
\pgfsetlinewidth{1.505625pt}%
\definecolor{currentstroke}{rgb}{0.890196,0.466667,0.760784}%
\pgfsetstrokecolor{currentstroke}%
\pgfsetdash{}{0pt}%
\pgfpathmoveto{\pgfqpoint{0.696156in}{1.696778in}}%
\pgfpathlineto{\pgfqpoint{0.973934in}{1.696778in}}%
\pgfusepath{stroke}%
\end{pgfscope}%
\begin{pgfscope}%
\pgftext[x=1.085045in,y=1.648167in,left,base]{\rmfamily\fontsize{10.000000}{12.000000}\selectfont 7}%
\end{pgfscope}%
\begin{pgfscope}%
\pgfsetrectcap%
\pgfsetroundjoin%
\pgfsetlinewidth{1.505625pt}%
\definecolor{currentstroke}{rgb}{0.498039,0.498039,0.498039}%
\pgfsetstrokecolor{currentstroke}%
\pgfsetdash{}{0pt}%
\pgfpathmoveto{\pgfqpoint{0.696156in}{1.500408in}}%
\pgfpathlineto{\pgfqpoint{0.973934in}{1.500408in}}%
\pgfusepath{stroke}%
\end{pgfscope}%
\begin{pgfscope}%
\pgftext[x=1.085045in,y=1.451797in,left,base]{\rmfamily\fontsize{10.000000}{12.000000}\selectfont 8}%
\end{pgfscope}%
\begin{pgfscope}%
\pgfsetrectcap%
\pgfsetroundjoin%
\pgfsetlinewidth{1.505625pt}%
\definecolor{currentstroke}{rgb}{0.737255,0.741176,0.133333}%
\pgfsetstrokecolor{currentstroke}%
\pgfsetdash{}{0pt}%
\pgfpathmoveto{\pgfqpoint{0.696156in}{1.304038in}}%
\pgfpathlineto{\pgfqpoint{0.973934in}{1.304038in}}%
\pgfusepath{stroke}%
\end{pgfscope}%
\begin{pgfscope}%
\pgftext[x=1.085045in,y=1.255427in,left,base]{\rmfamily\fontsize{10.000000}{12.000000}\selectfont 9}%
\end{pgfscope}%
\begin{pgfscope}%
\pgfsetrectcap%
\pgfsetroundjoin%
\pgfsetlinewidth{1.505625pt}%
\definecolor{currentstroke}{rgb}{0.090196,0.745098,0.811765}%
\pgfsetstrokecolor{currentstroke}%
\pgfsetdash{}{0pt}%
\pgfpathmoveto{\pgfqpoint{0.696156in}{1.107668in}}%
\pgfpathlineto{\pgfqpoint{0.973934in}{1.107668in}}%
\pgfusepath{stroke}%
\end{pgfscope}%
\begin{pgfscope}%
\pgftext[x=1.085045in,y=1.059056in,left,base]{\rmfamily\fontsize{10.000000}{12.000000}\selectfont 10}%
\end{pgfscope}%
\begin{pgfscope}%
\pgfsetrectcap%
\pgfsetroundjoin%
\pgfsetlinewidth{1.505625pt}%
\definecolor{currentstroke}{rgb}{0.121569,0.466667,0.705882}%
\pgfsetstrokecolor{currentstroke}%
\pgfsetdash{}{0pt}%
\pgfpathmoveto{\pgfqpoint{0.696156in}{0.911297in}}%
\pgfpathlineto{\pgfqpoint{0.973934in}{0.911297in}}%
\pgfusepath{stroke}%
\end{pgfscope}%
\begin{pgfscope}%
\pgftext[x=1.085045in,y=0.862686in,left,base]{\rmfamily\fontsize{10.000000}{12.000000}\selectfont 11}%
\end{pgfscope}%
\begin{pgfscope}%
\pgfsetrectcap%
\pgfsetroundjoin%
\pgfsetlinewidth{1.505625pt}%
\definecolor{currentstroke}{rgb}{1.000000,0.498039,0.054902}%
\pgfsetstrokecolor{currentstroke}%
\pgfsetdash{}{0pt}%
\pgfpathmoveto{\pgfqpoint{0.696156in}{0.714927in}}%
\pgfpathlineto{\pgfqpoint{0.973934in}{0.714927in}}%
\pgfusepath{stroke}%
\end{pgfscope}%
\begin{pgfscope}%
\pgftext[x=1.085045in,y=0.666316in,left,base]{\rmfamily\fontsize{10.000000}{12.000000}\selectfont 12}%
\end{pgfscope}%
\end{pgfpicture}%
\makeatother%
\endgroup%

        %% Creator: Matplotlib, PGF backend
%%
%% To include the figure in your LaTeX document, write
%%   \input{<filename>.pgf}
%%
%% Make sure the required packages are loaded in your preamble
%%   \usepackage{pgf}
%%
%% Figures using additional raster images can only be included by \input if
%% they are in the same directory as the main LaTeX file. For loading figures
%% from other directories you can use the `import` package
%%   \usepackage{import}
%% and then include the figures with
%%   \import{<path to file>}{<filename>.pgf}
%%
%% Matplotlib used the following preamble
%%   \usepackage{fontspec}
%%   \setmainfont{Times New Roman}
%%   \setsansfont{Lucida Grande}
%%   \setmonofont{Andale Mono}
%%
\begingroup%
\makeatletter%
\begin{pgfpicture}%
\pgfpathrectangle{\pgfpointorigin}{\pgfqpoint{5.000000in}{3.097222in}}%
\pgfusepath{use as bounding box, clip}%
\begin{pgfscope}%
\pgfsetbuttcap%
\pgfsetmiterjoin%
\definecolor{currentfill}{rgb}{1.000000,1.000000,1.000000}%
\pgfsetfillcolor{currentfill}%
\pgfsetlinewidth{0.000000pt}%
\definecolor{currentstroke}{rgb}{1.000000,1.000000,1.000000}%
\pgfsetstrokecolor{currentstroke}%
\pgfsetdash{}{0pt}%
\pgfpathmoveto{\pgfqpoint{0.000000in}{0.000000in}}%
\pgfpathlineto{\pgfqpoint{5.000000in}{0.000000in}}%
\pgfpathlineto{\pgfqpoint{5.000000in}{3.097222in}}%
\pgfpathlineto{\pgfqpoint{0.000000in}{3.097222in}}%
\pgfpathclose%
\pgfusepath{fill}%
\end{pgfscope}%
\begin{pgfscope}%
\pgfsetbuttcap%
\pgfsetmiterjoin%
\definecolor{currentfill}{rgb}{1.000000,1.000000,1.000000}%
\pgfsetfillcolor{currentfill}%
\pgfsetlinewidth{0.000000pt}%
\definecolor{currentstroke}{rgb}{0.000000,0.000000,0.000000}%
\pgfsetstrokecolor{currentstroke}%
\pgfsetstrokeopacity{0.000000}%
\pgfsetdash{}{0pt}%
\pgfpathmoveto{\pgfqpoint{0.626312in}{0.402778in}}%
\pgfpathlineto{\pgfqpoint{4.951389in}{0.402778in}}%
\pgfpathlineto{\pgfqpoint{4.951389in}{3.048611in}}%
\pgfpathlineto{\pgfqpoint{0.626312in}{3.048611in}}%
\pgfpathclose%
\pgfusepath{fill}%
\end{pgfscope}%
\begin{pgfscope}%
\pgfsetbuttcap%
\pgfsetroundjoin%
\definecolor{currentfill}{rgb}{0.000000,0.000000,0.000000}%
\pgfsetfillcolor{currentfill}%
\pgfsetlinewidth{0.803000pt}%
\definecolor{currentstroke}{rgb}{0.000000,0.000000,0.000000}%
\pgfsetstrokecolor{currentstroke}%
\pgfsetdash{}{0pt}%
\pgfsys@defobject{currentmarker}{\pgfqpoint{0.000000in}{-0.048611in}}{\pgfqpoint{0.000000in}{0.000000in}}{%
\pgfpathmoveto{\pgfqpoint{0.000000in}{0.000000in}}%
\pgfpathlineto{\pgfqpoint{0.000000in}{-0.048611in}}%
\pgfusepath{stroke,fill}%
}%
\begin{pgfscope}%
\pgfsys@transformshift{0.822906in}{0.402778in}%
\pgfsys@useobject{currentmarker}{}%
\end{pgfscope}%
\end{pgfscope}%
\begin{pgfscope}%
\pgftext[x=0.822906in,y=0.305556in,,top]{\rmfamily\fontsize{10.000000}{12.000000}\selectfont \(\displaystyle -6\)}%
\end{pgfscope}%
\begin{pgfscope}%
\pgfsetbuttcap%
\pgfsetroundjoin%
\definecolor{currentfill}{rgb}{0.000000,0.000000,0.000000}%
\pgfsetfillcolor{currentfill}%
\pgfsetlinewidth{0.803000pt}%
\definecolor{currentstroke}{rgb}{0.000000,0.000000,0.000000}%
\pgfsetstrokecolor{currentstroke}%
\pgfsetdash{}{0pt}%
\pgfsys@defobject{currentmarker}{\pgfqpoint{0.000000in}{-0.048611in}}{\pgfqpoint{0.000000in}{0.000000in}}{%
\pgfpathmoveto{\pgfqpoint{0.000000in}{0.000000in}}%
\pgfpathlineto{\pgfqpoint{0.000000in}{-0.048611in}}%
\pgfusepath{stroke,fill}%
}%
\begin{pgfscope}%
\pgfsys@transformshift{1.478221in}{0.402778in}%
\pgfsys@useobject{currentmarker}{}%
\end{pgfscope}%
\end{pgfscope}%
\begin{pgfscope}%
\pgftext[x=1.478221in,y=0.305556in,,top]{\rmfamily\fontsize{10.000000}{12.000000}\selectfont \(\displaystyle -4\)}%
\end{pgfscope}%
\begin{pgfscope}%
\pgfsetbuttcap%
\pgfsetroundjoin%
\definecolor{currentfill}{rgb}{0.000000,0.000000,0.000000}%
\pgfsetfillcolor{currentfill}%
\pgfsetlinewidth{0.803000pt}%
\definecolor{currentstroke}{rgb}{0.000000,0.000000,0.000000}%
\pgfsetstrokecolor{currentstroke}%
\pgfsetdash{}{0pt}%
\pgfsys@defobject{currentmarker}{\pgfqpoint{0.000000in}{-0.048611in}}{\pgfqpoint{0.000000in}{0.000000in}}{%
\pgfpathmoveto{\pgfqpoint{0.000000in}{0.000000in}}%
\pgfpathlineto{\pgfqpoint{0.000000in}{-0.048611in}}%
\pgfusepath{stroke,fill}%
}%
\begin{pgfscope}%
\pgfsys@transformshift{2.133536in}{0.402778in}%
\pgfsys@useobject{currentmarker}{}%
\end{pgfscope}%
\end{pgfscope}%
\begin{pgfscope}%
\pgftext[x=2.133536in,y=0.305556in,,top]{\rmfamily\fontsize{10.000000}{12.000000}\selectfont \(\displaystyle -2\)}%
\end{pgfscope}%
\begin{pgfscope}%
\pgfsetbuttcap%
\pgfsetroundjoin%
\definecolor{currentfill}{rgb}{0.000000,0.000000,0.000000}%
\pgfsetfillcolor{currentfill}%
\pgfsetlinewidth{0.803000pt}%
\definecolor{currentstroke}{rgb}{0.000000,0.000000,0.000000}%
\pgfsetstrokecolor{currentstroke}%
\pgfsetdash{}{0pt}%
\pgfsys@defobject{currentmarker}{\pgfqpoint{0.000000in}{-0.048611in}}{\pgfqpoint{0.000000in}{0.000000in}}{%
\pgfpathmoveto{\pgfqpoint{0.000000in}{0.000000in}}%
\pgfpathlineto{\pgfqpoint{0.000000in}{-0.048611in}}%
\pgfusepath{stroke,fill}%
}%
\begin{pgfscope}%
\pgfsys@transformshift{2.788850in}{0.402778in}%
\pgfsys@useobject{currentmarker}{}%
\end{pgfscope}%
\end{pgfscope}%
\begin{pgfscope}%
\pgftext[x=2.788850in,y=0.305556in,,top]{\rmfamily\fontsize{10.000000}{12.000000}\selectfont \(\displaystyle 0\)}%
\end{pgfscope}%
\begin{pgfscope}%
\pgfsetbuttcap%
\pgfsetroundjoin%
\definecolor{currentfill}{rgb}{0.000000,0.000000,0.000000}%
\pgfsetfillcolor{currentfill}%
\pgfsetlinewidth{0.803000pt}%
\definecolor{currentstroke}{rgb}{0.000000,0.000000,0.000000}%
\pgfsetstrokecolor{currentstroke}%
\pgfsetdash{}{0pt}%
\pgfsys@defobject{currentmarker}{\pgfqpoint{0.000000in}{-0.048611in}}{\pgfqpoint{0.000000in}{0.000000in}}{%
\pgfpathmoveto{\pgfqpoint{0.000000in}{0.000000in}}%
\pgfpathlineto{\pgfqpoint{0.000000in}{-0.048611in}}%
\pgfusepath{stroke,fill}%
}%
\begin{pgfscope}%
\pgfsys@transformshift{3.444165in}{0.402778in}%
\pgfsys@useobject{currentmarker}{}%
\end{pgfscope}%
\end{pgfscope}%
\begin{pgfscope}%
\pgftext[x=3.444165in,y=0.305556in,,top]{\rmfamily\fontsize{10.000000}{12.000000}\selectfont \(\displaystyle 2\)}%
\end{pgfscope}%
\begin{pgfscope}%
\pgfsetbuttcap%
\pgfsetroundjoin%
\definecolor{currentfill}{rgb}{0.000000,0.000000,0.000000}%
\pgfsetfillcolor{currentfill}%
\pgfsetlinewidth{0.803000pt}%
\definecolor{currentstroke}{rgb}{0.000000,0.000000,0.000000}%
\pgfsetstrokecolor{currentstroke}%
\pgfsetdash{}{0pt}%
\pgfsys@defobject{currentmarker}{\pgfqpoint{0.000000in}{-0.048611in}}{\pgfqpoint{0.000000in}{0.000000in}}{%
\pgfpathmoveto{\pgfqpoint{0.000000in}{0.000000in}}%
\pgfpathlineto{\pgfqpoint{0.000000in}{-0.048611in}}%
\pgfusepath{stroke,fill}%
}%
\begin{pgfscope}%
\pgfsys@transformshift{4.099480in}{0.402778in}%
\pgfsys@useobject{currentmarker}{}%
\end{pgfscope}%
\end{pgfscope}%
\begin{pgfscope}%
\pgftext[x=4.099480in,y=0.305556in,,top]{\rmfamily\fontsize{10.000000}{12.000000}\selectfont \(\displaystyle 4\)}%
\end{pgfscope}%
\begin{pgfscope}%
\pgfsetbuttcap%
\pgfsetroundjoin%
\definecolor{currentfill}{rgb}{0.000000,0.000000,0.000000}%
\pgfsetfillcolor{currentfill}%
\pgfsetlinewidth{0.803000pt}%
\definecolor{currentstroke}{rgb}{0.000000,0.000000,0.000000}%
\pgfsetstrokecolor{currentstroke}%
\pgfsetdash{}{0pt}%
\pgfsys@defobject{currentmarker}{\pgfqpoint{0.000000in}{-0.048611in}}{\pgfqpoint{0.000000in}{0.000000in}}{%
\pgfpathmoveto{\pgfqpoint{0.000000in}{0.000000in}}%
\pgfpathlineto{\pgfqpoint{0.000000in}{-0.048611in}}%
\pgfusepath{stroke,fill}%
}%
\begin{pgfscope}%
\pgfsys@transformshift{4.754794in}{0.402778in}%
\pgfsys@useobject{currentmarker}{}%
\end{pgfscope}%
\end{pgfscope}%
\begin{pgfscope}%
\pgftext[x=4.754794in,y=0.305556in,,top]{\rmfamily\fontsize{10.000000}{12.000000}\selectfont \(\displaystyle 6\)}%
\end{pgfscope}%
\begin{pgfscope}%
\pgftext[x=2.788850in,y=0.123861in,,top]{\rmfamily\fontsize{10.000000}{12.000000}\selectfont $t$}%
\end{pgfscope}%
\begin{pgfscope}%
\pgfsetbuttcap%
\pgfsetroundjoin%
\definecolor{currentfill}{rgb}{0.000000,0.000000,0.000000}%
\pgfsetfillcolor{currentfill}%
\pgfsetlinewidth{0.803000pt}%
\definecolor{currentstroke}{rgb}{0.000000,0.000000,0.000000}%
\pgfsetstrokecolor{currentstroke}%
\pgfsetdash{}{0pt}%
\pgfsys@defobject{currentmarker}{\pgfqpoint{-0.048611in}{0.000000in}}{\pgfqpoint{0.000000in}{0.000000in}}{%
\pgfpathmoveto{\pgfqpoint{0.000000in}{0.000000in}}%
\pgfpathlineto{\pgfqpoint{-0.048611in}{0.000000in}}%
\pgfusepath{stroke,fill}%
}%
\begin{pgfscope}%
\pgfsys@transformshift{0.626312in}{0.804392in}%
\pgfsys@useobject{currentmarker}{}%
\end{pgfscope}%
\end{pgfscope}%
\begin{pgfscope}%
\pgftext[x=0.185724in,y=0.756175in,left,base]{\rmfamily\fontsize{10.000000}{12.000000}\selectfont \(\displaystyle 10^{-11}\)}%
\end{pgfscope}%
\begin{pgfscope}%
\pgfsetbuttcap%
\pgfsetroundjoin%
\definecolor{currentfill}{rgb}{0.000000,0.000000,0.000000}%
\pgfsetfillcolor{currentfill}%
\pgfsetlinewidth{0.803000pt}%
\definecolor{currentstroke}{rgb}{0.000000,0.000000,0.000000}%
\pgfsetstrokecolor{currentstroke}%
\pgfsetdash{}{0pt}%
\pgfsys@defobject{currentmarker}{\pgfqpoint{-0.048611in}{0.000000in}}{\pgfqpoint{0.000000in}{0.000000in}}{%
\pgfpathmoveto{\pgfqpoint{0.000000in}{0.000000in}}%
\pgfpathlineto{\pgfqpoint{-0.048611in}{0.000000in}}%
\pgfusepath{stroke,fill}%
}%
\begin{pgfscope}%
\pgfsys@transformshift{0.626312in}{1.237742in}%
\pgfsys@useobject{currentmarker}{}%
\end{pgfscope}%
\end{pgfscope}%
\begin{pgfscope}%
\pgftext[x=0.241087in,y=1.189524in,left,base]{\rmfamily\fontsize{10.000000}{12.000000}\selectfont \(\displaystyle 10^{-9}\)}%
\end{pgfscope}%
\begin{pgfscope}%
\pgfsetbuttcap%
\pgfsetroundjoin%
\definecolor{currentfill}{rgb}{0.000000,0.000000,0.000000}%
\pgfsetfillcolor{currentfill}%
\pgfsetlinewidth{0.803000pt}%
\definecolor{currentstroke}{rgb}{0.000000,0.000000,0.000000}%
\pgfsetstrokecolor{currentstroke}%
\pgfsetdash{}{0pt}%
\pgfsys@defobject{currentmarker}{\pgfqpoint{-0.048611in}{0.000000in}}{\pgfqpoint{0.000000in}{0.000000in}}{%
\pgfpathmoveto{\pgfqpoint{0.000000in}{0.000000in}}%
\pgfpathlineto{\pgfqpoint{-0.048611in}{0.000000in}}%
\pgfusepath{stroke,fill}%
}%
\begin{pgfscope}%
\pgfsys@transformshift{0.626312in}{1.671091in}%
\pgfsys@useobject{currentmarker}{}%
\end{pgfscope}%
\end{pgfscope}%
\begin{pgfscope}%
\pgftext[x=0.241087in,y=1.622874in,left,base]{\rmfamily\fontsize{10.000000}{12.000000}\selectfont \(\displaystyle 10^{-7}\)}%
\end{pgfscope}%
\begin{pgfscope}%
\pgfsetbuttcap%
\pgfsetroundjoin%
\definecolor{currentfill}{rgb}{0.000000,0.000000,0.000000}%
\pgfsetfillcolor{currentfill}%
\pgfsetlinewidth{0.803000pt}%
\definecolor{currentstroke}{rgb}{0.000000,0.000000,0.000000}%
\pgfsetstrokecolor{currentstroke}%
\pgfsetdash{}{0pt}%
\pgfsys@defobject{currentmarker}{\pgfqpoint{-0.048611in}{0.000000in}}{\pgfqpoint{0.000000in}{0.000000in}}{%
\pgfpathmoveto{\pgfqpoint{0.000000in}{0.000000in}}%
\pgfpathlineto{\pgfqpoint{-0.048611in}{0.000000in}}%
\pgfusepath{stroke,fill}%
}%
\begin{pgfscope}%
\pgfsys@transformshift{0.626312in}{2.104441in}%
\pgfsys@useobject{currentmarker}{}%
\end{pgfscope}%
\end{pgfscope}%
\begin{pgfscope}%
\pgftext[x=0.241087in,y=2.056223in,left,base]{\rmfamily\fontsize{10.000000}{12.000000}\selectfont \(\displaystyle 10^{-5}\)}%
\end{pgfscope}%
\begin{pgfscope}%
\pgfsetbuttcap%
\pgfsetroundjoin%
\definecolor{currentfill}{rgb}{0.000000,0.000000,0.000000}%
\pgfsetfillcolor{currentfill}%
\pgfsetlinewidth{0.803000pt}%
\definecolor{currentstroke}{rgb}{0.000000,0.000000,0.000000}%
\pgfsetstrokecolor{currentstroke}%
\pgfsetdash{}{0pt}%
\pgfsys@defobject{currentmarker}{\pgfqpoint{-0.048611in}{0.000000in}}{\pgfqpoint{0.000000in}{0.000000in}}{%
\pgfpathmoveto{\pgfqpoint{0.000000in}{0.000000in}}%
\pgfpathlineto{\pgfqpoint{-0.048611in}{0.000000in}}%
\pgfusepath{stroke,fill}%
}%
\begin{pgfscope}%
\pgfsys@transformshift{0.626312in}{2.537790in}%
\pgfsys@useobject{currentmarker}{}%
\end{pgfscope}%
\end{pgfscope}%
\begin{pgfscope}%
\pgftext[x=0.241087in,y=2.489572in,left,base]{\rmfamily\fontsize{10.000000}{12.000000}\selectfont \(\displaystyle 10^{-3}\)}%
\end{pgfscope}%
\begin{pgfscope}%
\pgfsetbuttcap%
\pgfsetroundjoin%
\definecolor{currentfill}{rgb}{0.000000,0.000000,0.000000}%
\pgfsetfillcolor{currentfill}%
\pgfsetlinewidth{0.803000pt}%
\definecolor{currentstroke}{rgb}{0.000000,0.000000,0.000000}%
\pgfsetstrokecolor{currentstroke}%
\pgfsetdash{}{0pt}%
\pgfsys@defobject{currentmarker}{\pgfqpoint{-0.048611in}{0.000000in}}{\pgfqpoint{0.000000in}{0.000000in}}{%
\pgfpathmoveto{\pgfqpoint{0.000000in}{0.000000in}}%
\pgfpathlineto{\pgfqpoint{-0.048611in}{0.000000in}}%
\pgfusepath{stroke,fill}%
}%
\begin{pgfscope}%
\pgfsys@transformshift{0.626312in}{2.971140in}%
\pgfsys@useobject{currentmarker}{}%
\end{pgfscope}%
\end{pgfscope}%
\begin{pgfscope}%
\pgftext[x=0.241087in,y=2.922922in,left,base]{\rmfamily\fontsize{10.000000}{12.000000}\selectfont \(\displaystyle 10^{-1}\)}%
\end{pgfscope}%
\begin{pgfscope}%
\pgftext[x=0.130169in,y=1.725694in,,bottom,rotate=90.000000]{\rmfamily\fontsize{10.000000}{12.000000}\selectfont $E(t)$}%
\end{pgfscope}%
\begin{pgfscope}%
\pgfpathrectangle{\pgfqpoint{0.626312in}{0.402778in}}{\pgfqpoint{4.325077in}{2.645833in}}%
\pgfusepath{clip}%
\pgfsetrectcap%
\pgfsetroundjoin%
\pgfsetlinewidth{1.505625pt}%
\definecolor{currentstroke}{rgb}{0.121569,0.466667,0.705882}%
\pgfsetstrokecolor{currentstroke}%
\pgfsetdash{}{0pt}%
\pgfpathmoveto{\pgfqpoint{0.822966in}{0.388889in}}%
\pgfpathlineto{\pgfqpoint{0.824466in}{0.829598in}}%
\pgfpathlineto{\pgfqpoint{0.827826in}{0.937672in}}%
\pgfpathlineto{\pgfqpoint{0.832806in}{1.003458in}}%
\pgfpathlineto{\pgfqpoint{0.839465in}{1.051848in}}%
\pgfpathlineto{\pgfqpoint{0.847745in}{1.089978in}}%
\pgfpathlineto{\pgfqpoint{0.857764in}{1.121836in}}%
\pgfpathlineto{\pgfqpoint{0.869583in}{1.149274in}}%
\pgfpathlineto{\pgfqpoint{0.883442in}{1.173694in}}%
\pgfpathlineto{\pgfqpoint{0.899581in}{1.195879in}}%
\pgfpathlineto{\pgfqpoint{0.918360in}{1.216428in}}%
\pgfpathlineto{\pgfqpoint{0.940258in}{1.235784in}}%
\pgfpathlineto{\pgfqpoint{0.965936in}{1.254305in}}%
\pgfpathlineto{\pgfqpoint{0.996054in}{1.272162in}}%
\pgfpathlineto{\pgfqpoint{1.031572in}{1.289562in}}%
\pgfpathlineto{\pgfqpoint{1.073449in}{1.306567in}}%
\pgfpathlineto{\pgfqpoint{1.123005in}{1.323280in}}%
\pgfpathlineto{\pgfqpoint{1.150484in}{1.331359in}}%
\pgfpathlineto{\pgfqpoint{1.150544in}{2.824925in}}%
\pgfpathlineto{\pgfqpoint{1.152043in}{2.824922in}}%
\pgfpathlineto{\pgfqpoint{1.478121in}{2.823827in}}%
\pgfpathlineto{\pgfqpoint{1.478181in}{2.887702in}}%
\pgfpathlineto{\pgfqpoint{1.479681in}{2.887694in}}%
\pgfpathlineto{\pgfqpoint{1.805758in}{2.884723in}}%
\pgfpathlineto{\pgfqpoint{1.805818in}{2.918794in}}%
\pgfpathlineto{\pgfqpoint{1.807318in}{2.918772in}}%
\pgfpathlineto{\pgfqpoint{2.049401in}{2.913728in}}%
\pgfpathlineto{\pgfqpoint{2.133396in}{2.911020in}}%
\pgfpathlineto{\pgfqpoint{2.133456in}{2.928346in}}%
\pgfpathlineto{\pgfqpoint{2.134956in}{2.928291in}}%
\pgfpathlineto{\pgfqpoint{2.302284in}{2.920632in}}%
\pgfpathlineto{\pgfqpoint{2.445794in}{2.911030in}}%
\pgfpathlineto{\pgfqpoint{2.461033in}{2.909812in}}%
\pgfpathlineto{\pgfqpoint{2.461093in}{2.911470in}}%
\pgfpathlineto{\pgfqpoint{2.462593in}{2.911348in}}%
\pgfpathlineto{\pgfqpoint{2.591044in}{2.899320in}}%
\pgfpathlineto{\pgfqpoint{2.712656in}{2.884846in}}%
\pgfpathlineto{\pgfqpoint{2.788670in}{2.874177in}}%
\pgfpathlineto{\pgfqpoint{2.790230in}{2.862150in}}%
\pgfpathlineto{\pgfqpoint{2.910222in}{2.842780in}}%
\pgfpathlineto{\pgfqpoint{3.034593in}{2.819544in}}%
\pgfpathlineto{\pgfqpoint{3.116308in}{2.802685in}}%
\pgfpathlineto{\pgfqpoint{3.117868in}{2.790406in}}%
\pgfpathlineto{\pgfqpoint{3.260298in}{2.758453in}}%
\pgfpathlineto{\pgfqpoint{3.424386in}{2.718380in}}%
\pgfpathlineto{\pgfqpoint{3.443945in}{2.713418in}}%
\pgfpathlineto{\pgfqpoint{3.444005in}{2.717776in}}%
\pgfpathlineto{\pgfqpoint{3.445505in}{2.717394in}}%
\pgfpathlineto{\pgfqpoint{3.652071in}{2.663171in}}%
\pgfpathlineto{\pgfqpoint{3.771642in}{2.630673in}}%
\pgfpathlineto{\pgfqpoint{3.771702in}{2.648546in}}%
\pgfpathlineto{\pgfqpoint{3.773202in}{2.648134in}}%
\pgfpathlineto{\pgfqpoint{4.099340in}{2.557127in}}%
\pgfpathlineto{\pgfqpoint{4.099400in}{2.577629in}}%
\pgfpathlineto{\pgfqpoint{4.100900in}{2.577206in}}%
\pgfpathlineto{\pgfqpoint{4.427037in}{2.484520in}}%
\pgfpathlineto{\pgfqpoint{4.427097in}{2.502966in}}%
\pgfpathlineto{\pgfqpoint{4.428597in}{2.502538in}}%
\pgfpathlineto{\pgfqpoint{4.754794in}{2.409212in}}%
\pgfpathlineto{\pgfqpoint{4.754794in}{2.409212in}}%
\pgfusepath{stroke}%
\end{pgfscope}%
\begin{pgfscope}%
\pgfpathrectangle{\pgfqpoint{0.626312in}{0.402778in}}{\pgfqpoint{4.325077in}{2.645833in}}%
\pgfusepath{clip}%
\pgfsetrectcap%
\pgfsetroundjoin%
\pgfsetlinewidth{1.505625pt}%
\definecolor{currentstroke}{rgb}{1.000000,0.498039,0.054902}%
\pgfsetstrokecolor{currentstroke}%
\pgfsetdash{}{0pt}%
\pgfpathmoveto{\pgfqpoint{0.822966in}{0.388889in}}%
\pgfpathlineto{\pgfqpoint{0.824466in}{0.829598in}}%
\pgfpathlineto{\pgfqpoint{0.827826in}{0.937672in}}%
\pgfpathlineto{\pgfqpoint{0.832806in}{1.003458in}}%
\pgfpathlineto{\pgfqpoint{0.839465in}{1.051848in}}%
\pgfpathlineto{\pgfqpoint{0.847745in}{1.089978in}}%
\pgfpathlineto{\pgfqpoint{0.857764in}{1.121836in}}%
\pgfpathlineto{\pgfqpoint{0.869583in}{1.149274in}}%
\pgfpathlineto{\pgfqpoint{0.883442in}{1.173694in}}%
\pgfpathlineto{\pgfqpoint{0.899581in}{1.195879in}}%
\pgfpathlineto{\pgfqpoint{0.918360in}{1.216428in}}%
\pgfpathlineto{\pgfqpoint{0.940258in}{1.235784in}}%
\pgfpathlineto{\pgfqpoint{0.965936in}{1.254305in}}%
\pgfpathlineto{\pgfqpoint{0.996054in}{1.272162in}}%
\pgfpathlineto{\pgfqpoint{1.031572in}{1.289562in}}%
\pgfpathlineto{\pgfqpoint{1.073449in}{1.306567in}}%
\pgfpathlineto{\pgfqpoint{1.123005in}{1.323280in}}%
\pgfpathlineto{\pgfqpoint{1.181561in}{1.339687in}}%
\pgfpathlineto{\pgfqpoint{1.250737in}{1.355777in}}%
\pgfpathlineto{\pgfqpoint{1.332031in}{1.371431in}}%
\pgfpathlineto{\pgfqpoint{1.426585in}{1.386419in}}%
\pgfpathlineto{\pgfqpoint{1.478121in}{1.393447in}}%
\pgfpathlineto{\pgfqpoint{1.478181in}{2.445689in}}%
\pgfpathlineto{\pgfqpoint{1.479681in}{2.445682in}}%
\pgfpathlineto{\pgfqpoint{1.805758in}{2.442827in}}%
\pgfpathlineto{\pgfqpoint{1.805818in}{2.522867in}}%
\pgfpathlineto{\pgfqpoint{1.807318in}{2.522847in}}%
\pgfpathlineto{\pgfqpoint{2.055101in}{2.517917in}}%
\pgfpathlineto{\pgfqpoint{2.133396in}{2.515505in}}%
\pgfpathlineto{\pgfqpoint{2.133456in}{2.520964in}}%
\pgfpathlineto{\pgfqpoint{2.134956in}{2.520913in}}%
\pgfpathlineto{\pgfqpoint{2.306304in}{2.513485in}}%
\pgfpathlineto{\pgfqpoint{2.452094in}{2.504136in}}%
\pgfpathlineto{\pgfqpoint{2.461033in}{2.503453in}}%
\pgfpathlineto{\pgfqpoint{2.462593in}{2.342532in}}%
\pgfpathlineto{\pgfqpoint{2.592784in}{2.330899in}}%
\pgfpathlineto{\pgfqpoint{2.715235in}{2.316875in}}%
\pgfpathlineto{\pgfqpoint{2.788670in}{2.306923in}}%
\pgfpathlineto{\pgfqpoint{2.788730in}{2.344314in}}%
\pgfpathlineto{\pgfqpoint{2.790230in}{2.344098in}}%
\pgfpathlineto{\pgfqpoint{2.909862in}{2.325322in}}%
\pgfpathlineto{\pgfqpoint{3.033153in}{2.302818in}}%
\pgfpathlineto{\pgfqpoint{3.116308in}{2.285979in}}%
\pgfpathlineto{\pgfqpoint{3.116368in}{2.326852in}}%
\pgfpathlineto{\pgfqpoint{3.117868in}{2.326537in}}%
\pgfpathlineto{\pgfqpoint{3.258138in}{2.295480in}}%
\pgfpathlineto{\pgfqpoint{3.418627in}{2.256691in}}%
\pgfpathlineto{\pgfqpoint{3.443945in}{2.250317in}}%
\pgfpathlineto{\pgfqpoint{3.444005in}{2.330525in}}%
\pgfpathlineto{\pgfqpoint{3.445505in}{2.330145in}}%
\pgfpathlineto{\pgfqpoint{3.646791in}{2.277617in}}%
\pgfpathlineto{\pgfqpoint{3.771642in}{2.243811in}}%
\pgfpathlineto{\pgfqpoint{3.773202in}{2.220262in}}%
\pgfpathlineto{\pgfqpoint{4.099340in}{2.129444in}}%
\pgfpathlineto{\pgfqpoint{4.100900in}{2.012766in}}%
\pgfpathlineto{\pgfqpoint{4.427037in}{1.920171in}}%
\pgfpathlineto{\pgfqpoint{4.427097in}{2.120040in}}%
\pgfpathlineto{\pgfqpoint{4.428597in}{2.119612in}}%
\pgfpathlineto{\pgfqpoint{4.754794in}{2.026326in}}%
\pgfpathlineto{\pgfqpoint{4.754794in}{2.026326in}}%
\pgfusepath{stroke}%
\end{pgfscope}%
\begin{pgfscope}%
\pgfpathrectangle{\pgfqpoint{0.626312in}{0.402778in}}{\pgfqpoint{4.325077in}{2.645833in}}%
\pgfusepath{clip}%
\pgfsetrectcap%
\pgfsetroundjoin%
\pgfsetlinewidth{1.505625pt}%
\definecolor{currentstroke}{rgb}{0.172549,0.627451,0.172549}%
\pgfsetstrokecolor{currentstroke}%
\pgfsetdash{}{0pt}%
\pgfpathmoveto{\pgfqpoint{0.822966in}{0.388889in}}%
\pgfpathlineto{\pgfqpoint{0.824466in}{0.829598in}}%
\pgfpathlineto{\pgfqpoint{0.827826in}{0.937672in}}%
\pgfpathlineto{\pgfqpoint{0.832806in}{1.003458in}}%
\pgfpathlineto{\pgfqpoint{0.839465in}{1.051848in}}%
\pgfpathlineto{\pgfqpoint{0.847745in}{1.089978in}}%
\pgfpathlineto{\pgfqpoint{0.857764in}{1.121836in}}%
\pgfpathlineto{\pgfqpoint{0.869583in}{1.149274in}}%
\pgfpathlineto{\pgfqpoint{0.883442in}{1.173694in}}%
\pgfpathlineto{\pgfqpoint{0.899581in}{1.195879in}}%
\pgfpathlineto{\pgfqpoint{0.918360in}{1.216428in}}%
\pgfpathlineto{\pgfqpoint{0.940258in}{1.235784in}}%
\pgfpathlineto{\pgfqpoint{0.965936in}{1.254305in}}%
\pgfpathlineto{\pgfqpoint{0.996054in}{1.272162in}}%
\pgfpathlineto{\pgfqpoint{1.031572in}{1.289562in}}%
\pgfpathlineto{\pgfqpoint{1.073449in}{1.306567in}}%
\pgfpathlineto{\pgfqpoint{1.123005in}{1.323280in}}%
\pgfpathlineto{\pgfqpoint{1.181561in}{1.339687in}}%
\pgfpathlineto{\pgfqpoint{1.250737in}{1.355777in}}%
\pgfpathlineto{\pgfqpoint{1.332031in}{1.371431in}}%
\pgfpathlineto{\pgfqpoint{1.426585in}{1.386419in}}%
\pgfpathlineto{\pgfqpoint{1.534757in}{1.400388in}}%
\pgfpathlineto{\pgfqpoint{1.655409in}{1.412816in}}%
\pgfpathlineto{\pgfqpoint{1.784160in}{1.422958in}}%
\pgfpathlineto{\pgfqpoint{1.805758in}{1.424362in}}%
\pgfpathlineto{\pgfqpoint{1.805818in}{2.029162in}}%
\pgfpathlineto{\pgfqpoint{1.807318in}{2.029142in}}%
\pgfpathlineto{\pgfqpoint{2.054681in}{2.024246in}}%
\pgfpathlineto{\pgfqpoint{2.133396in}{2.021826in}}%
\pgfpathlineto{\pgfqpoint{2.133456in}{2.042324in}}%
\pgfpathlineto{\pgfqpoint{2.134956in}{2.042273in}}%
\pgfpathlineto{\pgfqpoint{2.306064in}{2.034857in}}%
\pgfpathlineto{\pgfqpoint{2.451734in}{2.025515in}}%
\pgfpathlineto{\pgfqpoint{2.461033in}{2.024804in}}%
\pgfpathlineto{\pgfqpoint{2.462593in}{1.993239in}}%
\pgfpathlineto{\pgfqpoint{2.592844in}{1.981604in}}%
\pgfpathlineto{\pgfqpoint{2.715355in}{1.967580in}}%
\pgfpathlineto{\pgfqpoint{2.788670in}{1.957650in}}%
\pgfpathlineto{\pgfqpoint{2.788730in}{1.979439in}}%
\pgfpathlineto{\pgfqpoint{2.790230in}{1.979223in}}%
\pgfpathlineto{\pgfqpoint{2.909862in}{1.960458in}}%
\pgfpathlineto{\pgfqpoint{3.033093in}{1.937975in}}%
\pgfpathlineto{\pgfqpoint{3.116308in}{1.921130in}}%
\pgfpathlineto{\pgfqpoint{3.116368in}{1.959404in}}%
\pgfpathlineto{\pgfqpoint{3.117868in}{1.959089in}}%
\pgfpathlineto{\pgfqpoint{3.258078in}{1.928043in}}%
\pgfpathlineto{\pgfqpoint{3.418447in}{1.889280in}}%
\pgfpathlineto{\pgfqpoint{3.443945in}{1.882860in}}%
\pgfpathlineto{\pgfqpoint{3.445505in}{1.819523in}}%
\pgfpathlineto{\pgfqpoint{3.646491in}{1.767051in}}%
\pgfpathlineto{\pgfqpoint{3.771642in}{1.733146in}}%
\pgfpathlineto{\pgfqpoint{3.771702in}{1.783672in}}%
\pgfpathlineto{\pgfqpoint{3.773202in}{1.783261in}}%
\pgfpathlineto{\pgfqpoint{4.099340in}{1.692411in}}%
\pgfpathlineto{\pgfqpoint{4.099400in}{1.773658in}}%
\pgfpathlineto{\pgfqpoint{4.100900in}{1.773235in}}%
\pgfpathlineto{\pgfqpoint{4.427037in}{1.680624in}}%
\pgfpathlineto{\pgfqpoint{4.428597in}{1.653033in}}%
\pgfpathlineto{\pgfqpoint{4.754794in}{1.559728in}}%
\pgfpathlineto{\pgfqpoint{4.754794in}{1.559728in}}%
\pgfusepath{stroke}%
\end{pgfscope}%
\begin{pgfscope}%
\pgfpathrectangle{\pgfqpoint{0.626312in}{0.402778in}}{\pgfqpoint{4.325077in}{2.645833in}}%
\pgfusepath{clip}%
\pgfsetrectcap%
\pgfsetroundjoin%
\pgfsetlinewidth{1.505625pt}%
\definecolor{currentstroke}{rgb}{0.839216,0.152941,0.156863}%
\pgfsetstrokecolor{currentstroke}%
\pgfsetdash{}{0pt}%
\pgfpathmoveto{\pgfqpoint{0.822966in}{0.388889in}}%
\pgfpathlineto{\pgfqpoint{0.824466in}{0.829598in}}%
\pgfpathlineto{\pgfqpoint{0.827826in}{0.937672in}}%
\pgfpathlineto{\pgfqpoint{0.832806in}{1.003458in}}%
\pgfpathlineto{\pgfqpoint{0.839465in}{1.051848in}}%
\pgfpathlineto{\pgfqpoint{0.847745in}{1.089978in}}%
\pgfpathlineto{\pgfqpoint{0.857764in}{1.121836in}}%
\pgfpathlineto{\pgfqpoint{0.869583in}{1.149274in}}%
\pgfpathlineto{\pgfqpoint{0.883442in}{1.173694in}}%
\pgfpathlineto{\pgfqpoint{0.899581in}{1.195879in}}%
\pgfpathlineto{\pgfqpoint{0.918360in}{1.216428in}}%
\pgfpathlineto{\pgfqpoint{0.940258in}{1.235784in}}%
\pgfpathlineto{\pgfqpoint{0.965936in}{1.254305in}}%
\pgfpathlineto{\pgfqpoint{0.996054in}{1.272162in}}%
\pgfpathlineto{\pgfqpoint{1.031572in}{1.289562in}}%
\pgfpathlineto{\pgfqpoint{1.073449in}{1.306567in}}%
\pgfpathlineto{\pgfqpoint{1.123005in}{1.323280in}}%
\pgfpathlineto{\pgfqpoint{1.181561in}{1.339687in}}%
\pgfpathlineto{\pgfqpoint{1.250737in}{1.355777in}}%
\pgfpathlineto{\pgfqpoint{1.332031in}{1.371431in}}%
\pgfpathlineto{\pgfqpoint{1.426585in}{1.386419in}}%
\pgfpathlineto{\pgfqpoint{1.534757in}{1.400388in}}%
\pgfpathlineto{\pgfqpoint{1.655409in}{1.412816in}}%
\pgfpathlineto{\pgfqpoint{1.784160in}{1.422958in}}%
\pgfpathlineto{\pgfqpoint{1.914711in}{1.430153in}}%
\pgfpathlineto{\pgfqpoint{2.041062in}{1.434056in}}%
\pgfpathlineto{\pgfqpoint{2.133396in}{1.434821in}}%
\pgfpathlineto{\pgfqpoint{2.133456in}{1.336454in}}%
\pgfpathlineto{\pgfqpoint{2.134956in}{1.336539in}}%
\pgfpathlineto{\pgfqpoint{2.210910in}{1.339299in}}%
\pgfpathlineto{\pgfqpoint{2.287825in}{1.339012in}}%
\pgfpathlineto{\pgfqpoint{2.364440in}{1.335636in}}%
\pgfpathlineto{\pgfqpoint{2.439614in}{1.329239in}}%
\pgfpathlineto{\pgfqpoint{2.461033in}{1.326841in}}%
\pgfpathlineto{\pgfqpoint{2.461093in}{1.601573in}}%
\pgfpathlineto{\pgfqpoint{2.462593in}{1.601454in}}%
\pgfpathlineto{\pgfqpoint{2.586544in}{1.590026in}}%
\pgfpathlineto{\pgfqpoint{2.705636in}{1.575952in}}%
\pgfpathlineto{\pgfqpoint{2.788670in}{1.564277in}}%
\pgfpathlineto{\pgfqpoint{2.790230in}{1.510580in}}%
\pgfpathlineto{\pgfqpoint{2.911482in}{1.490233in}}%
\pgfpathlineto{\pgfqpoint{3.042573in}{1.465045in}}%
\pgfpathlineto{\pgfqpoint{3.116308in}{1.449651in}}%
\pgfpathlineto{\pgfqpoint{3.116368in}{1.537266in}}%
\pgfpathlineto{\pgfqpoint{3.117868in}{1.536948in}}%
\pgfpathlineto{\pgfqpoint{3.265757in}{1.504052in}}%
\pgfpathlineto{\pgfqpoint{3.439325in}{1.462174in}}%
\pgfpathlineto{\pgfqpoint{3.443945in}{1.461022in}}%
\pgfpathlineto{\pgfqpoint{3.444005in}{1.521256in}}%
\pgfpathlineto{\pgfqpoint{3.445505in}{1.520879in}}%
\pgfpathlineto{\pgfqpoint{3.655790in}{1.466437in}}%
\pgfpathlineto{\pgfqpoint{3.771642in}{1.435409in}}%
\pgfpathlineto{\pgfqpoint{3.773202in}{1.427386in}}%
\pgfpathlineto{\pgfqpoint{4.099340in}{1.337913in}}%
\pgfpathlineto{\pgfqpoint{4.100900in}{1.253691in}}%
\pgfpathlineto{\pgfqpoint{4.427037in}{1.164714in}}%
\pgfpathlineto{\pgfqpoint{4.427097in}{1.279855in}}%
\pgfpathlineto{\pgfqpoint{4.428597in}{1.279422in}}%
\pgfpathlineto{\pgfqpoint{4.754794in}{1.185039in}}%
\pgfpathlineto{\pgfqpoint{4.754794in}{1.185039in}}%
\pgfusepath{stroke}%
\end{pgfscope}%
\begin{pgfscope}%
\pgfpathrectangle{\pgfqpoint{0.626312in}{0.402778in}}{\pgfqpoint{4.325077in}{2.645833in}}%
\pgfusepath{clip}%
\pgfsetrectcap%
\pgfsetroundjoin%
\pgfsetlinewidth{1.505625pt}%
\definecolor{currentstroke}{rgb}{0.580392,0.403922,0.741176}%
\pgfsetstrokecolor{currentstroke}%
\pgfsetdash{}{0pt}%
\pgfpathmoveto{\pgfqpoint{0.822966in}{0.388889in}}%
\pgfpathlineto{\pgfqpoint{0.824466in}{0.829598in}}%
\pgfpathlineto{\pgfqpoint{0.827826in}{0.937672in}}%
\pgfpathlineto{\pgfqpoint{0.832806in}{1.003458in}}%
\pgfpathlineto{\pgfqpoint{0.839465in}{1.051848in}}%
\pgfpathlineto{\pgfqpoint{0.847745in}{1.089978in}}%
\pgfpathlineto{\pgfqpoint{0.857764in}{1.121836in}}%
\pgfpathlineto{\pgfqpoint{0.869583in}{1.149274in}}%
\pgfpathlineto{\pgfqpoint{0.883442in}{1.173694in}}%
\pgfpathlineto{\pgfqpoint{0.899581in}{1.195879in}}%
\pgfpathlineto{\pgfqpoint{0.918360in}{1.216428in}}%
\pgfpathlineto{\pgfqpoint{0.940258in}{1.235784in}}%
\pgfpathlineto{\pgfqpoint{0.965936in}{1.254305in}}%
\pgfpathlineto{\pgfqpoint{0.996054in}{1.272162in}}%
\pgfpathlineto{\pgfqpoint{1.031572in}{1.289562in}}%
\pgfpathlineto{\pgfqpoint{1.073449in}{1.306567in}}%
\pgfpathlineto{\pgfqpoint{1.123005in}{1.323280in}}%
\pgfpathlineto{\pgfqpoint{1.181561in}{1.339687in}}%
\pgfpathlineto{\pgfqpoint{1.250737in}{1.355777in}}%
\pgfpathlineto{\pgfqpoint{1.332031in}{1.371431in}}%
\pgfpathlineto{\pgfqpoint{1.426585in}{1.386419in}}%
\pgfpathlineto{\pgfqpoint{1.534757in}{1.400388in}}%
\pgfpathlineto{\pgfqpoint{1.655409in}{1.412816in}}%
\pgfpathlineto{\pgfqpoint{1.784160in}{1.422958in}}%
\pgfpathlineto{\pgfqpoint{1.914711in}{1.430153in}}%
\pgfpathlineto{\pgfqpoint{2.041062in}{1.434056in}}%
\pgfpathlineto{\pgfqpoint{2.160214in}{1.434684in}}%
\pgfpathlineto{\pgfqpoint{2.271746in}{1.432217in}}%
\pgfpathlineto{\pgfqpoint{2.376679in}{1.426841in}}%
\pgfpathlineto{\pgfqpoint{2.461033in}{1.420154in}}%
\pgfpathlineto{\pgfqpoint{2.462653in}{1.419378in}}%
\pgfpathlineto{\pgfqpoint{2.561226in}{1.408660in}}%
\pgfpathlineto{\pgfqpoint{2.660039in}{1.394795in}}%
\pgfpathlineto{\pgfqpoint{2.763592in}{1.377099in}}%
\pgfpathlineto{\pgfqpoint{2.788670in}{1.372373in}}%
\pgfpathlineto{\pgfqpoint{2.788730in}{1.376382in}}%
\pgfpathlineto{\pgfqpoint{2.790230in}{1.376097in}}%
\pgfpathlineto{\pgfqpoint{2.912202in}{1.351350in}}%
\pgfpathlineto{\pgfqpoint{3.083130in}{1.313211in}}%
\pgfpathlineto{\pgfqpoint{3.116308in}{1.305624in}}%
\pgfpathlineto{\pgfqpoint{3.117868in}{1.301730in}}%
\pgfpathlineto{\pgfqpoint{3.404768in}{1.237508in}}%
\pgfpathlineto{\pgfqpoint{3.443945in}{1.229081in}}%
\pgfpathlineto{\pgfqpoint{3.445505in}{1.218014in}}%
\pgfpathlineto{\pgfqpoint{3.770862in}{1.151482in}}%
\pgfpathlineto{\pgfqpoint{3.771642in}{1.151323in}}%
\pgfpathlineto{\pgfqpoint{3.771702in}{1.156906in}}%
\pgfpathlineto{\pgfqpoint{3.773202in}{1.156594in}}%
\pgfpathlineto{\pgfqpoint{4.099340in}{1.087246in}}%
\pgfpathlineto{\pgfqpoint{4.099400in}{1.102636in}}%
\pgfpathlineto{\pgfqpoint{4.100900in}{1.102294in}}%
\pgfpathlineto{\pgfqpoint{4.396499in}{1.033747in}}%
\pgfpathlineto{\pgfqpoint{4.427037in}{1.026533in}}%
\pgfpathlineto{\pgfqpoint{4.427097in}{1.032083in}}%
\pgfpathlineto{\pgfqpoint{4.428597in}{1.031724in}}%
\pgfpathlineto{\pgfqpoint{4.754794in}{0.952511in}}%
\pgfpathlineto{\pgfqpoint{4.754794in}{0.952511in}}%
\pgfusepath{stroke}%
\end{pgfscope}%
\begin{pgfscope}%
\pgfpathrectangle{\pgfqpoint{0.626312in}{0.402778in}}{\pgfqpoint{4.325077in}{2.645833in}}%
\pgfusepath{clip}%
\pgfsetrectcap%
\pgfsetroundjoin%
\pgfsetlinewidth{1.505625pt}%
\definecolor{currentstroke}{rgb}{0.549020,0.337255,0.294118}%
\pgfsetstrokecolor{currentstroke}%
\pgfsetdash{}{0pt}%
\pgfpathmoveto{\pgfqpoint{0.822966in}{0.388889in}}%
\pgfpathlineto{\pgfqpoint{0.824466in}{0.829598in}}%
\pgfpathlineto{\pgfqpoint{0.827826in}{0.937672in}}%
\pgfpathlineto{\pgfqpoint{0.832806in}{1.003458in}}%
\pgfpathlineto{\pgfqpoint{0.839465in}{1.051848in}}%
\pgfpathlineto{\pgfqpoint{0.847745in}{1.089978in}}%
\pgfpathlineto{\pgfqpoint{0.857764in}{1.121836in}}%
\pgfpathlineto{\pgfqpoint{0.869583in}{1.149274in}}%
\pgfpathlineto{\pgfqpoint{0.883442in}{1.173694in}}%
\pgfpathlineto{\pgfqpoint{0.899581in}{1.195879in}}%
\pgfpathlineto{\pgfqpoint{0.918360in}{1.216428in}}%
\pgfpathlineto{\pgfqpoint{0.940258in}{1.235784in}}%
\pgfpathlineto{\pgfqpoint{0.965936in}{1.254305in}}%
\pgfpathlineto{\pgfqpoint{0.996054in}{1.272162in}}%
\pgfpathlineto{\pgfqpoint{1.031572in}{1.289562in}}%
\pgfpathlineto{\pgfqpoint{1.073449in}{1.306567in}}%
\pgfpathlineto{\pgfqpoint{1.123005in}{1.323280in}}%
\pgfpathlineto{\pgfqpoint{1.181561in}{1.339687in}}%
\pgfpathlineto{\pgfqpoint{1.250737in}{1.355777in}}%
\pgfpathlineto{\pgfqpoint{1.332031in}{1.371431in}}%
\pgfpathlineto{\pgfqpoint{1.426585in}{1.386419in}}%
\pgfpathlineto{\pgfqpoint{1.534757in}{1.400388in}}%
\pgfpathlineto{\pgfqpoint{1.655409in}{1.412816in}}%
\pgfpathlineto{\pgfqpoint{1.784160in}{1.422958in}}%
\pgfpathlineto{\pgfqpoint{1.914711in}{1.430153in}}%
\pgfpathlineto{\pgfqpoint{2.041062in}{1.434056in}}%
\pgfpathlineto{\pgfqpoint{2.160214in}{1.434684in}}%
\pgfpathlineto{\pgfqpoint{2.271746in}{1.432217in}}%
\pgfpathlineto{\pgfqpoint{2.376679in}{1.426841in}}%
\pgfpathlineto{\pgfqpoint{2.477292in}{1.418611in}}%
\pgfpathlineto{\pgfqpoint{2.575865in}{1.407446in}}%
\pgfpathlineto{\pgfqpoint{2.675398in}{1.393036in}}%
\pgfpathlineto{\pgfqpoint{2.780151in}{1.374700in}}%
\pgfpathlineto{\pgfqpoint{2.899063in}{1.350644in}}%
\pgfpathlineto{\pgfqpoint{3.065131in}{1.313571in}}%
\pgfpathlineto{\pgfqpoint{3.412867in}{1.235467in}}%
\pgfpathlineto{\pgfqpoint{3.730965in}{1.168418in}}%
\pgfpathlineto{\pgfqpoint{4.088541in}{1.091963in}}%
\pgfpathlineto{\pgfqpoint{4.409458in}{1.019775in}}%
\pgfpathlineto{\pgfqpoint{4.743515in}{0.941484in}}%
\pgfpathlineto{\pgfqpoint{4.754794in}{0.938784in}}%
\pgfpathlineto{\pgfqpoint{4.754794in}{0.938784in}}%
\pgfusepath{stroke}%
\end{pgfscope}%
\begin{pgfscope}%
\pgfsetrectcap%
\pgfsetmiterjoin%
\pgfsetlinewidth{0.803000pt}%
\definecolor{currentstroke}{rgb}{0.000000,0.000000,0.000000}%
\pgfsetstrokecolor{currentstroke}%
\pgfsetdash{}{0pt}%
\pgfpathmoveto{\pgfqpoint{0.626312in}{0.402778in}}%
\pgfpathlineto{\pgfqpoint{0.626312in}{3.048611in}}%
\pgfusepath{stroke}%
\end{pgfscope}%
\begin{pgfscope}%
\pgfsetrectcap%
\pgfsetmiterjoin%
\pgfsetlinewidth{0.803000pt}%
\definecolor{currentstroke}{rgb}{0.000000,0.000000,0.000000}%
\pgfsetstrokecolor{currentstroke}%
\pgfsetdash{}{0pt}%
\pgfpathmoveto{\pgfqpoint{4.951389in}{0.402778in}}%
\pgfpathlineto{\pgfqpoint{4.951389in}{3.048611in}}%
\pgfusepath{stroke}%
\end{pgfscope}%
\begin{pgfscope}%
\pgfsetrectcap%
\pgfsetmiterjoin%
\pgfsetlinewidth{0.803000pt}%
\definecolor{currentstroke}{rgb}{0.000000,0.000000,0.000000}%
\pgfsetstrokecolor{currentstroke}%
\pgfsetdash{}{0pt}%
\pgfpathmoveto{\pgfqpoint{0.626312in}{0.402778in}}%
\pgfpathlineto{\pgfqpoint{4.951389in}{0.402778in}}%
\pgfusepath{stroke}%
\end{pgfscope}%
\begin{pgfscope}%
\pgfsetrectcap%
\pgfsetmiterjoin%
\pgfsetlinewidth{0.803000pt}%
\definecolor{currentstroke}{rgb}{0.000000,0.000000,0.000000}%
\pgfsetstrokecolor{currentstroke}%
\pgfsetdash{}{0pt}%
\pgfpathmoveto{\pgfqpoint{0.626312in}{3.048611in}}%
\pgfpathlineto{\pgfqpoint{4.951389in}{3.048611in}}%
\pgfusepath{stroke}%
\end{pgfscope}%
\begin{pgfscope}%
\pgfsetbuttcap%
\pgfsetmiterjoin%
\definecolor{currentfill}{rgb}{1.000000,1.000000,1.000000}%
\pgfsetfillcolor{currentfill}%
\pgfsetfillopacity{0.800000}%
\pgfsetlinewidth{1.003750pt}%
\definecolor{currentstroke}{rgb}{0.800000,0.800000,0.800000}%
\pgfsetstrokecolor{currentstroke}%
\pgfsetstrokeopacity{0.800000}%
\pgfsetdash{}{0pt}%
\pgfpathmoveto{\pgfqpoint{0.723534in}{1.759278in}}%
\pgfpathlineto{\pgfqpoint{1.237423in}{1.759278in}}%
\pgfpathquadraticcurveto{\pgfqpoint{1.265201in}{1.759278in}}{\pgfqpoint{1.265201in}{1.787056in}}%
\pgfpathlineto{\pgfqpoint{1.265201in}{2.951389in}}%
\pgfpathquadraticcurveto{\pgfqpoint{1.265201in}{2.979167in}}{\pgfqpoint{1.237423in}{2.979167in}}%
\pgfpathlineto{\pgfqpoint{0.723534in}{2.979167in}}%
\pgfpathquadraticcurveto{\pgfqpoint{0.695756in}{2.979167in}}{\pgfqpoint{0.695756in}{2.951389in}}%
\pgfpathlineto{\pgfqpoint{0.695756in}{1.787056in}}%
\pgfpathquadraticcurveto{\pgfqpoint{0.695756in}{1.759278in}}{\pgfqpoint{0.723534in}{1.759278in}}%
\pgfpathclose%
\pgfusepath{stroke,fill}%
\end{pgfscope}%
\begin{pgfscope}%
\pgfsetrectcap%
\pgfsetroundjoin%
\pgfsetlinewidth{1.505625pt}%
\definecolor{currentstroke}{rgb}{0.121569,0.466667,0.705882}%
\pgfsetstrokecolor{currentstroke}%
\pgfsetdash{}{0pt}%
\pgfpathmoveto{\pgfqpoint{0.751312in}{2.875000in}}%
\pgfpathlineto{\pgfqpoint{1.029090in}{2.875000in}}%
\pgfusepath{stroke}%
\end{pgfscope}%
\begin{pgfscope}%
\pgftext[x=1.140201in,y=2.826389in,left,base]{\rmfamily\fontsize{10.000000}{12.000000}\selectfont 1}%
\end{pgfscope}%
\begin{pgfscope}%
\pgfsetrectcap%
\pgfsetroundjoin%
\pgfsetlinewidth{1.505625pt}%
\definecolor{currentstroke}{rgb}{1.000000,0.498039,0.054902}%
\pgfsetstrokecolor{currentstroke}%
\pgfsetdash{}{0pt}%
\pgfpathmoveto{\pgfqpoint{0.751312in}{2.678630in}}%
\pgfpathlineto{\pgfqpoint{1.029090in}{2.678630in}}%
\pgfusepath{stroke}%
\end{pgfscope}%
\begin{pgfscope}%
\pgftext[x=1.140201in,y=2.630019in,left,base]{\rmfamily\fontsize{10.000000}{12.000000}\selectfont 2}%
\end{pgfscope}%
\begin{pgfscope}%
\pgfsetrectcap%
\pgfsetroundjoin%
\pgfsetlinewidth{1.505625pt}%
\definecolor{currentstroke}{rgb}{0.172549,0.627451,0.172549}%
\pgfsetstrokecolor{currentstroke}%
\pgfsetdash{}{0pt}%
\pgfpathmoveto{\pgfqpoint{0.751312in}{2.482259in}}%
\pgfpathlineto{\pgfqpoint{1.029090in}{2.482259in}}%
\pgfusepath{stroke}%
\end{pgfscope}%
\begin{pgfscope}%
\pgftext[x=1.140201in,y=2.433648in,left,base]{\rmfamily\fontsize{10.000000}{12.000000}\selectfont 3}%
\end{pgfscope}%
\begin{pgfscope}%
\pgfsetrectcap%
\pgfsetroundjoin%
\pgfsetlinewidth{1.505625pt}%
\definecolor{currentstroke}{rgb}{0.839216,0.152941,0.156863}%
\pgfsetstrokecolor{currentstroke}%
\pgfsetdash{}{0pt}%
\pgfpathmoveto{\pgfqpoint{0.751312in}{2.285889in}}%
\pgfpathlineto{\pgfqpoint{1.029090in}{2.285889in}}%
\pgfusepath{stroke}%
\end{pgfscope}%
\begin{pgfscope}%
\pgftext[x=1.140201in,y=2.237278in,left,base]{\rmfamily\fontsize{10.000000}{12.000000}\selectfont 4}%
\end{pgfscope}%
\begin{pgfscope}%
\pgfsetrectcap%
\pgfsetroundjoin%
\pgfsetlinewidth{1.505625pt}%
\definecolor{currentstroke}{rgb}{0.580392,0.403922,0.741176}%
\pgfsetstrokecolor{currentstroke}%
\pgfsetdash{}{0pt}%
\pgfpathmoveto{\pgfqpoint{0.751312in}{2.089519in}}%
\pgfpathlineto{\pgfqpoint{1.029090in}{2.089519in}}%
\pgfusepath{stroke}%
\end{pgfscope}%
\begin{pgfscope}%
\pgftext[x=1.140201in,y=2.040908in,left,base]{\rmfamily\fontsize{10.000000}{12.000000}\selectfont 5}%
\end{pgfscope}%
\begin{pgfscope}%
\pgfsetrectcap%
\pgfsetroundjoin%
\pgfsetlinewidth{1.505625pt}%
\definecolor{currentstroke}{rgb}{0.549020,0.337255,0.294118}%
\pgfsetstrokecolor{currentstroke}%
\pgfsetdash{}{0pt}%
\pgfpathmoveto{\pgfqpoint{0.751312in}{1.893149in}}%
\pgfpathlineto{\pgfqpoint{1.029090in}{1.893149in}}%
\pgfusepath{stroke}%
\end{pgfscope}%
\begin{pgfscope}%
\pgftext[x=1.140201in,y=1.844537in,left,base]{\rmfamily\fontsize{10.000000}{12.000000}\selectfont 6}%
\end{pgfscope}%
\end{pgfpicture}%
\makeatother%
\endgroup%

    \caption{Lokaler Fehler, absolut (oben) und relativ (unten), der Parareal Lösungen der logistischen Gleichung.}
    \label{fig:iter_error_local}
\end{figure}
Typischerweise liegt die exakte Lösung jedoch nicht vor, wodurch der Vergleich mit einer numerischen Lösung relevant ist. Für Abbildung~\ref{fig:iter_error_local_num} wurde das sequenzielle Ergebnis eines Lösers mit der Genauigkeit von \(\mathcal{F}\) als Vergleich bemüht. Dieser ist natürlich identisch mit der zwölften Iteration. Ab der Zehnten Iteration ist der Fehler nicht mehr darstellbar.\\
\begin{figure}[ht]
    \centering
        %% Creator: Matplotlib, PGF backend
%%
%% To include the figure in your LaTeX document, write
%%   \input{<filename>.pgf}
%%
%% Make sure the required packages are loaded in your preamble
%%   \usepackage{pgf}
%%
%% Figures using additional raster images can only be included by \input if
%% they are in the same directory as the main LaTeX file. For loading figures
%% from other directories you can use the `import` package
%%   \usepackage{import}
%% and then include the figures with
%%   \import{<path to file>}{<filename>.pgf}
%%
%% Matplotlib used the following preamble
%%   \usepackage{fontspec}
%%   \setmainfont{Times New Roman}
%%   \setsansfont{Lucida Grande}
%%   \setmonofont{Andale Mono}
%%
\begingroup%
\makeatletter%
\begin{pgfpicture}%
\pgfpathrectangle{\pgfpointorigin}{\pgfqpoint{5.000000in}{3.097222in}}%
\pgfusepath{use as bounding box, clip}%
\begin{pgfscope}%
\pgfsetbuttcap%
\pgfsetmiterjoin%
\definecolor{currentfill}{rgb}{1.000000,1.000000,1.000000}%
\pgfsetfillcolor{currentfill}%
\pgfsetlinewidth{0.000000pt}%
\definecolor{currentstroke}{rgb}{1.000000,1.000000,1.000000}%
\pgfsetstrokecolor{currentstroke}%
\pgfsetdash{}{0pt}%
\pgfpathmoveto{\pgfqpoint{0.000000in}{0.000000in}}%
\pgfpathlineto{\pgfqpoint{5.000000in}{0.000000in}}%
\pgfpathlineto{\pgfqpoint{5.000000in}{3.097222in}}%
\pgfpathlineto{\pgfqpoint{0.000000in}{3.097222in}}%
\pgfpathclose%
\pgfusepath{fill}%
\end{pgfscope}%
\begin{pgfscope}%
\pgfsetbuttcap%
\pgfsetmiterjoin%
\definecolor{currentfill}{rgb}{1.000000,1.000000,1.000000}%
\pgfsetfillcolor{currentfill}%
\pgfsetlinewidth{0.000000pt}%
\definecolor{currentstroke}{rgb}{0.000000,0.000000,0.000000}%
\pgfsetstrokecolor{currentstroke}%
\pgfsetstrokeopacity{0.000000}%
\pgfsetdash{}{0pt}%
\pgfpathmoveto{\pgfqpoint{0.626312in}{0.402778in}}%
\pgfpathlineto{\pgfqpoint{4.951389in}{0.402778in}}%
\pgfpathlineto{\pgfqpoint{4.951389in}{3.048611in}}%
\pgfpathlineto{\pgfqpoint{0.626312in}{3.048611in}}%
\pgfpathclose%
\pgfusepath{fill}%
\end{pgfscope}%
\begin{pgfscope}%
\pgfsetbuttcap%
\pgfsetroundjoin%
\definecolor{currentfill}{rgb}{0.000000,0.000000,0.000000}%
\pgfsetfillcolor{currentfill}%
\pgfsetlinewidth{0.803000pt}%
\definecolor{currentstroke}{rgb}{0.000000,0.000000,0.000000}%
\pgfsetstrokecolor{currentstroke}%
\pgfsetdash{}{0pt}%
\pgfsys@defobject{currentmarker}{\pgfqpoint{0.000000in}{-0.048611in}}{\pgfqpoint{0.000000in}{0.000000in}}{%
\pgfpathmoveto{\pgfqpoint{0.000000in}{0.000000in}}%
\pgfpathlineto{\pgfqpoint{0.000000in}{-0.048611in}}%
\pgfusepath{stroke,fill}%
}%
\begin{pgfscope}%
\pgfsys@transformshift{0.822906in}{0.402778in}%
\pgfsys@useobject{currentmarker}{}%
\end{pgfscope}%
\end{pgfscope}%
\begin{pgfscope}%
\pgftext[x=0.822906in,y=0.305556in,,top]{\rmfamily\fontsize{10.000000}{12.000000}\selectfont \(\displaystyle -6\)}%
\end{pgfscope}%
\begin{pgfscope}%
\pgfsetbuttcap%
\pgfsetroundjoin%
\definecolor{currentfill}{rgb}{0.000000,0.000000,0.000000}%
\pgfsetfillcolor{currentfill}%
\pgfsetlinewidth{0.803000pt}%
\definecolor{currentstroke}{rgb}{0.000000,0.000000,0.000000}%
\pgfsetstrokecolor{currentstroke}%
\pgfsetdash{}{0pt}%
\pgfsys@defobject{currentmarker}{\pgfqpoint{0.000000in}{-0.048611in}}{\pgfqpoint{0.000000in}{0.000000in}}{%
\pgfpathmoveto{\pgfqpoint{0.000000in}{0.000000in}}%
\pgfpathlineto{\pgfqpoint{0.000000in}{-0.048611in}}%
\pgfusepath{stroke,fill}%
}%
\begin{pgfscope}%
\pgfsys@transformshift{1.478221in}{0.402778in}%
\pgfsys@useobject{currentmarker}{}%
\end{pgfscope}%
\end{pgfscope}%
\begin{pgfscope}%
\pgftext[x=1.478221in,y=0.305556in,,top]{\rmfamily\fontsize{10.000000}{12.000000}\selectfont \(\displaystyle -4\)}%
\end{pgfscope}%
\begin{pgfscope}%
\pgfsetbuttcap%
\pgfsetroundjoin%
\definecolor{currentfill}{rgb}{0.000000,0.000000,0.000000}%
\pgfsetfillcolor{currentfill}%
\pgfsetlinewidth{0.803000pt}%
\definecolor{currentstroke}{rgb}{0.000000,0.000000,0.000000}%
\pgfsetstrokecolor{currentstroke}%
\pgfsetdash{}{0pt}%
\pgfsys@defobject{currentmarker}{\pgfqpoint{0.000000in}{-0.048611in}}{\pgfqpoint{0.000000in}{0.000000in}}{%
\pgfpathmoveto{\pgfqpoint{0.000000in}{0.000000in}}%
\pgfpathlineto{\pgfqpoint{0.000000in}{-0.048611in}}%
\pgfusepath{stroke,fill}%
}%
\begin{pgfscope}%
\pgfsys@transformshift{2.133536in}{0.402778in}%
\pgfsys@useobject{currentmarker}{}%
\end{pgfscope}%
\end{pgfscope}%
\begin{pgfscope}%
\pgftext[x=2.133536in,y=0.305556in,,top]{\rmfamily\fontsize{10.000000}{12.000000}\selectfont \(\displaystyle -2\)}%
\end{pgfscope}%
\begin{pgfscope}%
\pgfsetbuttcap%
\pgfsetroundjoin%
\definecolor{currentfill}{rgb}{0.000000,0.000000,0.000000}%
\pgfsetfillcolor{currentfill}%
\pgfsetlinewidth{0.803000pt}%
\definecolor{currentstroke}{rgb}{0.000000,0.000000,0.000000}%
\pgfsetstrokecolor{currentstroke}%
\pgfsetdash{}{0pt}%
\pgfsys@defobject{currentmarker}{\pgfqpoint{0.000000in}{-0.048611in}}{\pgfqpoint{0.000000in}{0.000000in}}{%
\pgfpathmoveto{\pgfqpoint{0.000000in}{0.000000in}}%
\pgfpathlineto{\pgfqpoint{0.000000in}{-0.048611in}}%
\pgfusepath{stroke,fill}%
}%
\begin{pgfscope}%
\pgfsys@transformshift{2.788850in}{0.402778in}%
\pgfsys@useobject{currentmarker}{}%
\end{pgfscope}%
\end{pgfscope}%
\begin{pgfscope}%
\pgftext[x=2.788850in,y=0.305556in,,top]{\rmfamily\fontsize{10.000000}{12.000000}\selectfont \(\displaystyle 0\)}%
\end{pgfscope}%
\begin{pgfscope}%
\pgfsetbuttcap%
\pgfsetroundjoin%
\definecolor{currentfill}{rgb}{0.000000,0.000000,0.000000}%
\pgfsetfillcolor{currentfill}%
\pgfsetlinewidth{0.803000pt}%
\definecolor{currentstroke}{rgb}{0.000000,0.000000,0.000000}%
\pgfsetstrokecolor{currentstroke}%
\pgfsetdash{}{0pt}%
\pgfsys@defobject{currentmarker}{\pgfqpoint{0.000000in}{-0.048611in}}{\pgfqpoint{0.000000in}{0.000000in}}{%
\pgfpathmoveto{\pgfqpoint{0.000000in}{0.000000in}}%
\pgfpathlineto{\pgfqpoint{0.000000in}{-0.048611in}}%
\pgfusepath{stroke,fill}%
}%
\begin{pgfscope}%
\pgfsys@transformshift{3.444165in}{0.402778in}%
\pgfsys@useobject{currentmarker}{}%
\end{pgfscope}%
\end{pgfscope}%
\begin{pgfscope}%
\pgftext[x=3.444165in,y=0.305556in,,top]{\rmfamily\fontsize{10.000000}{12.000000}\selectfont \(\displaystyle 2\)}%
\end{pgfscope}%
\begin{pgfscope}%
\pgfsetbuttcap%
\pgfsetroundjoin%
\definecolor{currentfill}{rgb}{0.000000,0.000000,0.000000}%
\pgfsetfillcolor{currentfill}%
\pgfsetlinewidth{0.803000pt}%
\definecolor{currentstroke}{rgb}{0.000000,0.000000,0.000000}%
\pgfsetstrokecolor{currentstroke}%
\pgfsetdash{}{0pt}%
\pgfsys@defobject{currentmarker}{\pgfqpoint{0.000000in}{-0.048611in}}{\pgfqpoint{0.000000in}{0.000000in}}{%
\pgfpathmoveto{\pgfqpoint{0.000000in}{0.000000in}}%
\pgfpathlineto{\pgfqpoint{0.000000in}{-0.048611in}}%
\pgfusepath{stroke,fill}%
}%
\begin{pgfscope}%
\pgfsys@transformshift{4.099480in}{0.402778in}%
\pgfsys@useobject{currentmarker}{}%
\end{pgfscope}%
\end{pgfscope}%
\begin{pgfscope}%
\pgftext[x=4.099480in,y=0.305556in,,top]{\rmfamily\fontsize{10.000000}{12.000000}\selectfont \(\displaystyle 4\)}%
\end{pgfscope}%
\begin{pgfscope}%
\pgfsetbuttcap%
\pgfsetroundjoin%
\definecolor{currentfill}{rgb}{0.000000,0.000000,0.000000}%
\pgfsetfillcolor{currentfill}%
\pgfsetlinewidth{0.803000pt}%
\definecolor{currentstroke}{rgb}{0.000000,0.000000,0.000000}%
\pgfsetstrokecolor{currentstroke}%
\pgfsetdash{}{0pt}%
\pgfsys@defobject{currentmarker}{\pgfqpoint{0.000000in}{-0.048611in}}{\pgfqpoint{0.000000in}{0.000000in}}{%
\pgfpathmoveto{\pgfqpoint{0.000000in}{0.000000in}}%
\pgfpathlineto{\pgfqpoint{0.000000in}{-0.048611in}}%
\pgfusepath{stroke,fill}%
}%
\begin{pgfscope}%
\pgfsys@transformshift{4.754794in}{0.402778in}%
\pgfsys@useobject{currentmarker}{}%
\end{pgfscope}%
\end{pgfscope}%
\begin{pgfscope}%
\pgftext[x=4.754794in,y=0.305556in,,top]{\rmfamily\fontsize{10.000000}{12.000000}\selectfont \(\displaystyle 6\)}%
\end{pgfscope}%
\begin{pgfscope}%
\pgftext[x=2.788850in,y=0.123861in,,top]{\rmfamily\fontsize{10.000000}{12.000000}\selectfont t}%
\end{pgfscope}%
\begin{pgfscope}%
\pgfsetbuttcap%
\pgfsetroundjoin%
\definecolor{currentfill}{rgb}{0.000000,0.000000,0.000000}%
\pgfsetfillcolor{currentfill}%
\pgfsetlinewidth{0.803000pt}%
\definecolor{currentstroke}{rgb}{0.000000,0.000000,0.000000}%
\pgfsetstrokecolor{currentstroke}%
\pgfsetdash{}{0pt}%
\pgfsys@defobject{currentmarker}{\pgfqpoint{-0.048611in}{0.000000in}}{\pgfqpoint{0.000000in}{0.000000in}}{%
\pgfpathmoveto{\pgfqpoint{0.000000in}{0.000000in}}%
\pgfpathlineto{\pgfqpoint{-0.048611in}{0.000000in}}%
\pgfusepath{stroke,fill}%
}%
\begin{pgfscope}%
\pgfsys@transformshift{0.626312in}{0.677432in}%
\pgfsys@useobject{currentmarker}{}%
\end{pgfscope}%
\end{pgfscope}%
\begin{pgfscope}%
\pgftext[x=0.185724in,y=0.629214in,left,base]{\rmfamily\fontsize{10.000000}{12.000000}\selectfont \(\displaystyle 10^{-15}\)}%
\end{pgfscope}%
\begin{pgfscope}%
\pgfsetbuttcap%
\pgfsetroundjoin%
\definecolor{currentfill}{rgb}{0.000000,0.000000,0.000000}%
\pgfsetfillcolor{currentfill}%
\pgfsetlinewidth{0.803000pt}%
\definecolor{currentstroke}{rgb}{0.000000,0.000000,0.000000}%
\pgfsetstrokecolor{currentstroke}%
\pgfsetdash{}{0pt}%
\pgfsys@defobject{currentmarker}{\pgfqpoint{-0.048611in}{0.000000in}}{\pgfqpoint{0.000000in}{0.000000in}}{%
\pgfpathmoveto{\pgfqpoint{0.000000in}{0.000000in}}%
\pgfpathlineto{\pgfqpoint{-0.048611in}{0.000000in}}%
\pgfusepath{stroke,fill}%
}%
\begin{pgfscope}%
\pgfsys@transformshift{0.626312in}{1.003592in}%
\pgfsys@useobject{currentmarker}{}%
\end{pgfscope}%
\end{pgfscope}%
\begin{pgfscope}%
\pgftext[x=0.185724in,y=0.955374in,left,base]{\rmfamily\fontsize{10.000000}{12.000000}\selectfont \(\displaystyle 10^{-13}\)}%
\end{pgfscope}%
\begin{pgfscope}%
\pgfsetbuttcap%
\pgfsetroundjoin%
\definecolor{currentfill}{rgb}{0.000000,0.000000,0.000000}%
\pgfsetfillcolor{currentfill}%
\pgfsetlinewidth{0.803000pt}%
\definecolor{currentstroke}{rgb}{0.000000,0.000000,0.000000}%
\pgfsetstrokecolor{currentstroke}%
\pgfsetdash{}{0pt}%
\pgfsys@defobject{currentmarker}{\pgfqpoint{-0.048611in}{0.000000in}}{\pgfqpoint{0.000000in}{0.000000in}}{%
\pgfpathmoveto{\pgfqpoint{0.000000in}{0.000000in}}%
\pgfpathlineto{\pgfqpoint{-0.048611in}{0.000000in}}%
\pgfusepath{stroke,fill}%
}%
\begin{pgfscope}%
\pgfsys@transformshift{0.626312in}{1.329752in}%
\pgfsys@useobject{currentmarker}{}%
\end{pgfscope}%
\end{pgfscope}%
\begin{pgfscope}%
\pgftext[x=0.185724in,y=1.281535in,left,base]{\rmfamily\fontsize{10.000000}{12.000000}\selectfont \(\displaystyle 10^{-11}\)}%
\end{pgfscope}%
\begin{pgfscope}%
\pgfsetbuttcap%
\pgfsetroundjoin%
\definecolor{currentfill}{rgb}{0.000000,0.000000,0.000000}%
\pgfsetfillcolor{currentfill}%
\pgfsetlinewidth{0.803000pt}%
\definecolor{currentstroke}{rgb}{0.000000,0.000000,0.000000}%
\pgfsetstrokecolor{currentstroke}%
\pgfsetdash{}{0pt}%
\pgfsys@defobject{currentmarker}{\pgfqpoint{-0.048611in}{0.000000in}}{\pgfqpoint{0.000000in}{0.000000in}}{%
\pgfpathmoveto{\pgfqpoint{0.000000in}{0.000000in}}%
\pgfpathlineto{\pgfqpoint{-0.048611in}{0.000000in}}%
\pgfusepath{stroke,fill}%
}%
\begin{pgfscope}%
\pgfsys@transformshift{0.626312in}{1.655913in}%
\pgfsys@useobject{currentmarker}{}%
\end{pgfscope}%
\end{pgfscope}%
\begin{pgfscope}%
\pgftext[x=0.241087in,y=1.607695in,left,base]{\rmfamily\fontsize{10.000000}{12.000000}\selectfont \(\displaystyle 10^{-9}\)}%
\end{pgfscope}%
\begin{pgfscope}%
\pgfsetbuttcap%
\pgfsetroundjoin%
\definecolor{currentfill}{rgb}{0.000000,0.000000,0.000000}%
\pgfsetfillcolor{currentfill}%
\pgfsetlinewidth{0.803000pt}%
\definecolor{currentstroke}{rgb}{0.000000,0.000000,0.000000}%
\pgfsetstrokecolor{currentstroke}%
\pgfsetdash{}{0pt}%
\pgfsys@defobject{currentmarker}{\pgfqpoint{-0.048611in}{0.000000in}}{\pgfqpoint{0.000000in}{0.000000in}}{%
\pgfpathmoveto{\pgfqpoint{0.000000in}{0.000000in}}%
\pgfpathlineto{\pgfqpoint{-0.048611in}{0.000000in}}%
\pgfusepath{stroke,fill}%
}%
\begin{pgfscope}%
\pgfsys@transformshift{0.626312in}{1.982073in}%
\pgfsys@useobject{currentmarker}{}%
\end{pgfscope}%
\end{pgfscope}%
\begin{pgfscope}%
\pgftext[x=0.241087in,y=1.933856in,left,base]{\rmfamily\fontsize{10.000000}{12.000000}\selectfont \(\displaystyle 10^{-7}\)}%
\end{pgfscope}%
\begin{pgfscope}%
\pgfsetbuttcap%
\pgfsetroundjoin%
\definecolor{currentfill}{rgb}{0.000000,0.000000,0.000000}%
\pgfsetfillcolor{currentfill}%
\pgfsetlinewidth{0.803000pt}%
\definecolor{currentstroke}{rgb}{0.000000,0.000000,0.000000}%
\pgfsetstrokecolor{currentstroke}%
\pgfsetdash{}{0pt}%
\pgfsys@defobject{currentmarker}{\pgfqpoint{-0.048611in}{0.000000in}}{\pgfqpoint{0.000000in}{0.000000in}}{%
\pgfpathmoveto{\pgfqpoint{0.000000in}{0.000000in}}%
\pgfpathlineto{\pgfqpoint{-0.048611in}{0.000000in}}%
\pgfusepath{stroke,fill}%
}%
\begin{pgfscope}%
\pgfsys@transformshift{0.626312in}{2.308234in}%
\pgfsys@useobject{currentmarker}{}%
\end{pgfscope}%
\end{pgfscope}%
\begin{pgfscope}%
\pgftext[x=0.241087in,y=2.260016in,left,base]{\rmfamily\fontsize{10.000000}{12.000000}\selectfont \(\displaystyle 10^{-5}\)}%
\end{pgfscope}%
\begin{pgfscope}%
\pgfsetbuttcap%
\pgfsetroundjoin%
\definecolor{currentfill}{rgb}{0.000000,0.000000,0.000000}%
\pgfsetfillcolor{currentfill}%
\pgfsetlinewidth{0.803000pt}%
\definecolor{currentstroke}{rgb}{0.000000,0.000000,0.000000}%
\pgfsetstrokecolor{currentstroke}%
\pgfsetdash{}{0pt}%
\pgfsys@defobject{currentmarker}{\pgfqpoint{-0.048611in}{0.000000in}}{\pgfqpoint{0.000000in}{0.000000in}}{%
\pgfpathmoveto{\pgfqpoint{0.000000in}{0.000000in}}%
\pgfpathlineto{\pgfqpoint{-0.048611in}{0.000000in}}%
\pgfusepath{stroke,fill}%
}%
\begin{pgfscope}%
\pgfsys@transformshift{0.626312in}{2.634394in}%
\pgfsys@useobject{currentmarker}{}%
\end{pgfscope}%
\end{pgfscope}%
\begin{pgfscope}%
\pgftext[x=0.241087in,y=2.586176in,left,base]{\rmfamily\fontsize{10.000000}{12.000000}\selectfont \(\displaystyle 10^{-3}\)}%
\end{pgfscope}%
\begin{pgfscope}%
\pgfsetbuttcap%
\pgfsetroundjoin%
\definecolor{currentfill}{rgb}{0.000000,0.000000,0.000000}%
\pgfsetfillcolor{currentfill}%
\pgfsetlinewidth{0.803000pt}%
\definecolor{currentstroke}{rgb}{0.000000,0.000000,0.000000}%
\pgfsetstrokecolor{currentstroke}%
\pgfsetdash{}{0pt}%
\pgfsys@defobject{currentmarker}{\pgfqpoint{-0.048611in}{0.000000in}}{\pgfqpoint{0.000000in}{0.000000in}}{%
\pgfpathmoveto{\pgfqpoint{0.000000in}{0.000000in}}%
\pgfpathlineto{\pgfqpoint{-0.048611in}{0.000000in}}%
\pgfusepath{stroke,fill}%
}%
\begin{pgfscope}%
\pgfsys@transformshift{0.626312in}{2.960555in}%
\pgfsys@useobject{currentmarker}{}%
\end{pgfscope}%
\end{pgfscope}%
\begin{pgfscope}%
\pgftext[x=0.241087in,y=2.912337in,left,base]{\rmfamily\fontsize{10.000000}{12.000000}\selectfont \(\displaystyle 10^{-1}\)}%
\end{pgfscope}%
\begin{pgfscope}%
\pgftext[x=0.130169in,y=1.725694in,,bottom,rotate=90.000000]{\rmfamily\fontsize{10.000000}{12.000000}\selectfont E(t)}%
\end{pgfscope}%
\begin{pgfscope}%
\pgfpathrectangle{\pgfqpoint{0.626312in}{0.402778in}}{\pgfqpoint{4.325077in}{2.645833in}}%
\pgfusepath{clip}%
\pgfsetrectcap%
\pgfsetroundjoin%
\pgfsetlinewidth{1.505625pt}%
\definecolor{currentstroke}{rgb}{0.121569,0.466667,0.705882}%
\pgfsetstrokecolor{currentstroke}%
\pgfsetdash{}{0pt}%
\pgfpathmoveto{\pgfqpoint{1.150543in}{0.388889in}}%
\pgfpathlineto{\pgfqpoint{1.150544in}{2.850506in}}%
\pgfpathlineto{\pgfqpoint{1.152043in}{2.850504in}}%
\pgfpathlineto{\pgfqpoint{1.478121in}{2.849680in}}%
\pgfpathlineto{\pgfqpoint{1.478181in}{2.897755in}}%
\pgfpathlineto{\pgfqpoint{1.479681in}{2.897749in}}%
\pgfpathlineto{\pgfqpoint{1.805758in}{2.895513in}}%
\pgfpathlineto{\pgfqpoint{1.805818in}{2.921156in}}%
\pgfpathlineto{\pgfqpoint{1.807318in}{2.921140in}}%
\pgfpathlineto{\pgfqpoint{2.082039in}{2.916607in}}%
\pgfpathlineto{\pgfqpoint{2.133396in}{2.915306in}}%
\pgfpathlineto{\pgfqpoint{2.133456in}{2.928346in}}%
\pgfpathlineto{\pgfqpoint{2.134956in}{2.928305in}}%
\pgfpathlineto{\pgfqpoint{2.326102in}{2.921504in}}%
\pgfpathlineto{\pgfqpoint{2.540108in}{2.910376in}}%
\pgfpathlineto{\pgfqpoint{2.682298in}{2.898548in}}%
\pgfpathlineto{\pgfqpoint{2.788670in}{2.887575in}}%
\pgfpathlineto{\pgfqpoint{2.790230in}{2.878524in}}%
\pgfpathlineto{\pgfqpoint{2.928341in}{2.861543in}}%
\pgfpathlineto{\pgfqpoint{3.072871in}{2.840624in}}%
\pgfpathlineto{\pgfqpoint{3.116308in}{2.833767in}}%
\pgfpathlineto{\pgfqpoint{3.117868in}{2.824525in}}%
\pgfpathlineto{\pgfqpoint{3.282316in}{2.796565in}}%
\pgfpathlineto{\pgfqpoint{3.443945in}{2.766581in}}%
\pgfpathlineto{\pgfqpoint{3.444005in}{2.769861in}}%
\pgfpathlineto{\pgfqpoint{3.445505in}{2.769573in}}%
\pgfpathlineto{\pgfqpoint{3.685008in}{2.722073in}}%
\pgfpathlineto{\pgfqpoint{3.771642in}{2.704303in}}%
\pgfpathlineto{\pgfqpoint{3.771702in}{2.717755in}}%
\pgfpathlineto{\pgfqpoint{3.773202in}{2.717445in}}%
\pgfpathlineto{\pgfqpoint{4.099340in}{2.648948in}}%
\pgfpathlineto{\pgfqpoint{4.099400in}{2.664379in}}%
\pgfpathlineto{\pgfqpoint{4.100900in}{2.664060in}}%
\pgfpathlineto{\pgfqpoint{4.427037in}{2.594301in}}%
\pgfpathlineto{\pgfqpoint{4.427097in}{2.608184in}}%
\pgfpathlineto{\pgfqpoint{4.428597in}{2.607861in}}%
\pgfpathlineto{\pgfqpoint{4.754794in}{2.537620in}}%
\pgfpathlineto{\pgfqpoint{4.754794in}{2.537620in}}%
\pgfusepath{stroke}%
\end{pgfscope}%
\begin{pgfscope}%
\pgfpathrectangle{\pgfqpoint{0.626312in}{0.402778in}}{\pgfqpoint{4.325077in}{2.645833in}}%
\pgfusepath{clip}%
\pgfsetrectcap%
\pgfsetroundjoin%
\pgfsetlinewidth{1.505625pt}%
\definecolor{currentstroke}{rgb}{1.000000,0.498039,0.054902}%
\pgfsetstrokecolor{currentstroke}%
\pgfsetdash{}{0pt}%
\pgfpathmoveto{\pgfqpoint{1.478180in}{0.388889in}}%
\pgfpathlineto{\pgfqpoint{1.478181in}{2.565076in}}%
\pgfpathlineto{\pgfqpoint{1.479681in}{2.565070in}}%
\pgfpathlineto{\pgfqpoint{1.805758in}{2.562922in}}%
\pgfpathlineto{\pgfqpoint{1.805818in}{2.623163in}}%
\pgfpathlineto{\pgfqpoint{1.807318in}{2.623148in}}%
\pgfpathlineto{\pgfqpoint{2.088339in}{2.618714in}}%
\pgfpathlineto{\pgfqpoint{2.133396in}{2.617622in}}%
\pgfpathlineto{\pgfqpoint{2.133456in}{2.621731in}}%
\pgfpathlineto{\pgfqpoint{2.134956in}{2.621692in}}%
\pgfpathlineto{\pgfqpoint{2.330602in}{2.615092in}}%
\pgfpathlineto{\pgfqpoint{2.461033in}{2.608551in}}%
\pgfpathlineto{\pgfqpoint{2.462593in}{2.487437in}}%
\pgfpathlineto{\pgfqpoint{2.611983in}{2.477181in}}%
\pgfpathlineto{\pgfqpoint{2.751893in}{2.464498in}}%
\pgfpathlineto{\pgfqpoint{2.788670in}{2.460636in}}%
\pgfpathlineto{\pgfqpoint{2.788730in}{2.488772in}}%
\pgfpathlineto{\pgfqpoint{2.790230in}{2.488610in}}%
\pgfpathlineto{\pgfqpoint{2.927981in}{2.472136in}}%
\pgfpathlineto{\pgfqpoint{3.071131in}{2.451870in}}%
\pgfpathlineto{\pgfqpoint{3.116308in}{2.444867in}}%
\pgfpathlineto{\pgfqpoint{3.116368in}{2.475633in}}%
\pgfpathlineto{\pgfqpoint{3.117868in}{2.475396in}}%
\pgfpathlineto{\pgfqpoint{3.279736in}{2.448228in}}%
\pgfpathlineto{\pgfqpoint{3.443945in}{2.418029in}}%
\pgfpathlineto{\pgfqpoint{3.444005in}{2.478397in}}%
\pgfpathlineto{\pgfqpoint{3.445505in}{2.478111in}}%
\pgfpathlineto{\pgfqpoint{3.678829in}{2.432101in}}%
\pgfpathlineto{\pgfqpoint{3.771642in}{2.413132in}}%
\pgfpathlineto{\pgfqpoint{3.773202in}{2.395407in}}%
\pgfpathlineto{\pgfqpoint{4.099340in}{2.327053in}}%
\pgfpathlineto{\pgfqpoint{4.100900in}{2.239231in}}%
\pgfpathlineto{\pgfqpoint{4.427037in}{2.169539in}}%
\pgfpathlineto{\pgfqpoint{4.427097in}{2.319974in}}%
\pgfpathlineto{\pgfqpoint{4.428597in}{2.319652in}}%
\pgfpathlineto{\pgfqpoint{4.754794in}{2.249440in}}%
\pgfpathlineto{\pgfqpoint{4.754794in}{2.249440in}}%
\pgfusepath{stroke}%
\end{pgfscope}%
\begin{pgfscope}%
\pgfpathrectangle{\pgfqpoint{0.626312in}{0.402778in}}{\pgfqpoint{4.325077in}{2.645833in}}%
\pgfusepath{clip}%
\pgfsetrectcap%
\pgfsetroundjoin%
\pgfsetlinewidth{1.505625pt}%
\definecolor{currentstroke}{rgb}{0.172549,0.627451,0.172549}%
\pgfsetstrokecolor{currentstroke}%
\pgfsetdash{}{0pt}%
\pgfpathmoveto{\pgfqpoint{1.805818in}{0.388889in}}%
\pgfpathlineto{\pgfqpoint{1.805818in}{2.251461in}}%
\pgfpathlineto{\pgfqpoint{1.807318in}{2.251445in}}%
\pgfpathlineto{\pgfqpoint{2.088219in}{2.247011in}}%
\pgfpathlineto{\pgfqpoint{2.133396in}{2.245915in}}%
\pgfpathlineto{\pgfqpoint{2.133456in}{2.261370in}}%
\pgfpathlineto{\pgfqpoint{2.134956in}{2.261331in}}%
\pgfpathlineto{\pgfqpoint{2.330542in}{2.254729in}}%
\pgfpathlineto{\pgfqpoint{2.461033in}{2.248180in}}%
\pgfpathlineto{\pgfqpoint{2.462593in}{2.224698in}}%
\pgfpathlineto{\pgfqpoint{2.611983in}{2.214440in}}%
\pgfpathlineto{\pgfqpoint{2.751893in}{2.201755in}}%
\pgfpathlineto{\pgfqpoint{2.788670in}{2.197893in}}%
\pgfpathlineto{\pgfqpoint{2.788730in}{2.214264in}}%
\pgfpathlineto{\pgfqpoint{2.790230in}{2.214101in}}%
\pgfpathlineto{\pgfqpoint{2.927981in}{2.197629in}}%
\pgfpathlineto{\pgfqpoint{3.071131in}{2.177365in}}%
\pgfpathlineto{\pgfqpoint{3.116308in}{2.170363in}}%
\pgfpathlineto{\pgfqpoint{3.116368in}{2.199137in}}%
\pgfpathlineto{\pgfqpoint{3.117868in}{2.198900in}}%
\pgfpathlineto{\pgfqpoint{3.279736in}{2.171730in}}%
\pgfpathlineto{\pgfqpoint{3.443945in}{2.141529in}}%
\pgfpathlineto{\pgfqpoint{3.445505in}{2.093923in}}%
\pgfpathlineto{\pgfqpoint{3.678889in}{2.047897in}}%
\pgfpathlineto{\pgfqpoint{3.771642in}{2.028939in}}%
\pgfpathlineto{\pgfqpoint{3.771702in}{2.066900in}}%
\pgfpathlineto{\pgfqpoint{3.773202in}{2.066592in}}%
\pgfpathlineto{\pgfqpoint{4.099340in}{1.998236in}}%
\pgfpathlineto{\pgfqpoint{4.099400in}{2.059319in}}%
\pgfpathlineto{\pgfqpoint{4.100900in}{2.059001in}}%
\pgfpathlineto{\pgfqpoint{4.427037in}{1.989309in}}%
\pgfpathlineto{\pgfqpoint{4.428597in}{1.968563in}}%
\pgfpathlineto{\pgfqpoint{4.754794in}{1.898352in}}%
\pgfpathlineto{\pgfqpoint{4.754794in}{1.898352in}}%
\pgfusepath{stroke}%
\end{pgfscope}%
\begin{pgfscope}%
\pgfpathrectangle{\pgfqpoint{0.626312in}{0.402778in}}{\pgfqpoint{4.325077in}{2.645833in}}%
\pgfusepath{clip}%
\pgfsetrectcap%
\pgfsetroundjoin%
\pgfsetlinewidth{1.505625pt}%
\definecolor{currentstroke}{rgb}{0.839216,0.152941,0.156863}%
\pgfsetstrokecolor{currentstroke}%
\pgfsetdash{}{0pt}%
\pgfpathmoveto{\pgfqpoint{2.133455in}{0.388889in}}%
\pgfpathlineto{\pgfqpoint{2.133456in}{1.773563in}}%
\pgfpathlineto{\pgfqpoint{2.134956in}{1.773524in}}%
\pgfpathlineto{\pgfqpoint{2.330602in}{1.766919in}}%
\pgfpathlineto{\pgfqpoint{2.461033in}{1.760373in}}%
\pgfpathlineto{\pgfqpoint{2.461093in}{1.918619in}}%
\pgfpathlineto{\pgfqpoint{2.462593in}{1.918532in}}%
\pgfpathlineto{\pgfqpoint{2.611983in}{1.908274in}}%
\pgfpathlineto{\pgfqpoint{2.751833in}{1.895596in}}%
\pgfpathlineto{\pgfqpoint{2.788670in}{1.891727in}}%
\pgfpathlineto{\pgfqpoint{2.790230in}{1.842641in}}%
\pgfpathlineto{\pgfqpoint{2.927861in}{1.826185in}}%
\pgfpathlineto{\pgfqpoint{3.071011in}{1.805923in}}%
\pgfpathlineto{\pgfqpoint{3.116308in}{1.798902in}}%
\pgfpathlineto{\pgfqpoint{3.116368in}{1.875296in}}%
\pgfpathlineto{\pgfqpoint{3.117868in}{1.875059in}}%
\pgfpathlineto{\pgfqpoint{3.279796in}{1.847879in}}%
\pgfpathlineto{\pgfqpoint{3.443945in}{1.817688in}}%
\pgfpathlineto{\pgfqpoint{3.444005in}{1.866060in}}%
\pgfpathlineto{\pgfqpoint{3.445505in}{1.865775in}}%
\pgfpathlineto{\pgfqpoint{3.678889in}{1.819749in}}%
\pgfpathlineto{\pgfqpoint{3.771642in}{1.800791in}}%
\pgfpathlineto{\pgfqpoint{3.773202in}{1.794410in}}%
\pgfpathlineto{\pgfqpoint{4.099340in}{1.726055in}}%
\pgfpathlineto{\pgfqpoint{4.100900in}{1.654353in}}%
\pgfpathlineto{\pgfqpoint{4.427037in}{1.584660in}}%
\pgfpathlineto{\pgfqpoint{4.427097in}{1.691765in}}%
\pgfpathlineto{\pgfqpoint{4.428597in}{1.691443in}}%
\pgfpathlineto{\pgfqpoint{4.754794in}{1.621232in}}%
\pgfpathlineto{\pgfqpoint{4.754794in}{1.621232in}}%
\pgfusepath{stroke}%
\end{pgfscope}%
\begin{pgfscope}%
\pgfpathrectangle{\pgfqpoint{0.626312in}{0.402778in}}{\pgfqpoint{4.325077in}{2.645833in}}%
\pgfusepath{clip}%
\pgfsetrectcap%
\pgfsetroundjoin%
\pgfsetlinewidth{1.505625pt}%
\definecolor{currentstroke}{rgb}{0.580392,0.403922,0.741176}%
\pgfsetstrokecolor{currentstroke}%
\pgfsetdash{}{0pt}%
\pgfpathmoveto{\pgfqpoint{2.461093in}{0.388889in}}%
\pgfpathlineto{\pgfqpoint{2.461093in}{1.437936in}}%
\pgfpathlineto{\pgfqpoint{2.462593in}{1.437849in}}%
\pgfpathlineto{\pgfqpoint{2.606643in}{1.428017in}}%
\pgfpathlineto{\pgfqpoint{2.748113in}{1.415293in}}%
\pgfpathlineto{\pgfqpoint{2.788670in}{1.411044in}}%
\pgfpathlineto{\pgfqpoint{2.788730in}{1.522010in}}%
\pgfpathlineto{\pgfqpoint{2.790230in}{1.521848in}}%
\pgfpathlineto{\pgfqpoint{2.925221in}{1.505738in}}%
\pgfpathlineto{\pgfqpoint{3.074791in}{1.484556in}}%
\pgfpathlineto{\pgfqpoint{3.116308in}{1.478110in}}%
\pgfpathlineto{\pgfqpoint{3.117868in}{1.285304in}}%
\pgfpathlineto{\pgfqpoint{3.285736in}{1.257122in}}%
\pgfpathlineto{\pgfqpoint{3.327313in}{1.249670in}}%
\pgfpathlineto{\pgfqpoint{3.443945in}{1.227960in}}%
\pgfpathlineto{\pgfqpoint{3.444005in}{1.489935in}}%
\pgfpathlineto{\pgfqpoint{3.445505in}{1.489650in}}%
\pgfpathlineto{\pgfqpoint{3.680329in}{1.443326in}}%
\pgfpathlineto{\pgfqpoint{3.771642in}{1.424663in}}%
\pgfpathlineto{\pgfqpoint{3.773202in}{1.351925in}}%
\pgfpathlineto{\pgfqpoint{3.991347in}{1.306431in}}%
\pgfpathlineto{\pgfqpoint{4.099340in}{1.283571in}}%
\pgfpathlineto{\pgfqpoint{4.099400in}{1.409234in}}%
\pgfpathlineto{\pgfqpoint{4.100900in}{1.408916in}}%
\pgfpathlineto{\pgfqpoint{4.427037in}{1.339214in}}%
\pgfpathlineto{\pgfqpoint{4.427097in}{1.370915in}}%
\pgfpathlineto{\pgfqpoint{4.428597in}{1.370592in}}%
\pgfpathlineto{\pgfqpoint{4.754794in}{1.300341in}}%
\pgfpathlineto{\pgfqpoint{4.754794in}{1.300341in}}%
\pgfusepath{stroke}%
\end{pgfscope}%
\begin{pgfscope}%
\pgfpathrectangle{\pgfqpoint{0.626312in}{0.402778in}}{\pgfqpoint{4.325077in}{2.645833in}}%
\pgfusepath{clip}%
\pgfsetrectcap%
\pgfsetroundjoin%
\pgfsetlinewidth{1.505625pt}%
\definecolor{currentstroke}{rgb}{0.549020,0.337255,0.294118}%
\pgfsetstrokecolor{currentstroke}%
\pgfsetdash{}{0pt}%
\pgfpathmoveto{\pgfqpoint{2.788730in}{0.388889in}}%
\pgfpathlineto{\pgfqpoint{2.788730in}{1.054657in}}%
\pgfpathlineto{\pgfqpoint{2.790230in}{1.054496in}}%
\pgfpathlineto{\pgfqpoint{2.903382in}{1.040772in}}%
\pgfpathlineto{\pgfqpoint{2.905422in}{1.040511in}}%
\pgfpathlineto{\pgfqpoint{2.943340in}{1.035642in}}%
\pgfpathlineto{\pgfqpoint{2.945500in}{1.035219in}}%
\pgfpathlineto{\pgfqpoint{2.974118in}{1.031350in}}%
\pgfpathlineto{\pgfqpoint{2.976157in}{1.031023in}}%
\pgfpathlineto{\pgfqpoint{2.998896in}{1.027769in}}%
\pgfpathlineto{\pgfqpoint{3.000876in}{1.027536in}}%
\pgfpathlineto{\pgfqpoint{3.027874in}{1.024030in}}%
\pgfpathlineto{\pgfqpoint{3.029914in}{1.023712in}}%
\pgfpathlineto{\pgfqpoint{3.058832in}{1.020153in}}%
\pgfpathlineto{\pgfqpoint{3.060752in}{1.019937in}}%
\pgfpathlineto{\pgfqpoint{3.081870in}{1.017041in}}%
\pgfpathlineto{\pgfqpoint{3.083850in}{1.016734in}}%
\pgfpathlineto{\pgfqpoint{3.098429in}{1.014352in}}%
\pgfpathlineto{\pgfqpoint{3.100409in}{1.014045in}}%
\pgfpathlineto{\pgfqpoint{3.113608in}{1.011740in}}%
\pgfpathlineto{\pgfqpoint{3.115768in}{1.011132in}}%
\pgfpathlineto{\pgfqpoint{3.116308in}{1.011100in}}%
\pgfpathlineto{\pgfqpoint{3.116368in}{1.162141in}}%
\pgfpathlineto{\pgfqpoint{3.117868in}{1.161951in}}%
\pgfpathlineto{\pgfqpoint{3.133227in}{1.159520in}}%
\pgfpathlineto{\pgfqpoint{3.152185in}{1.156336in}}%
\pgfpathlineto{\pgfqpoint{3.167484in}{1.153976in}}%
\pgfpathlineto{\pgfqpoint{3.185063in}{1.151026in}}%
\pgfpathlineto{\pgfqpoint{3.196942in}{1.148899in}}%
\pgfpathlineto{\pgfqpoint{3.213681in}{1.146047in}}%
\pgfpathlineto{\pgfqpoint{3.225500in}{1.143962in}}%
\pgfpathlineto{\pgfqpoint{3.241399in}{1.141202in}}%
\pgfpathlineto{\pgfqpoint{3.252858in}{1.139172in}}%
\pgfpathlineto{\pgfqpoint{3.268577in}{1.136327in}}%
\pgfpathlineto{\pgfqpoint{3.280816in}{1.134290in}}%
\pgfpathlineto{\pgfqpoint{3.294975in}{1.131755in}}%
\pgfpathlineto{\pgfqpoint{3.305355in}{1.129818in}}%
\pgfpathlineto{\pgfqpoint{3.319214in}{1.127223in}}%
\pgfpathlineto{\pgfqpoint{3.331033in}{1.125261in}}%
\pgfpathlineto{\pgfqpoint{3.342732in}{1.123195in}}%
\pgfpathlineto{\pgfqpoint{3.353171in}{1.121300in}}%
\pgfpathlineto{\pgfqpoint{3.364991in}{1.119065in}}%
\pgfpathlineto{\pgfqpoint{3.374890in}{1.117222in}}%
\pgfpathlineto{\pgfqpoint{3.384669in}{1.115526in}}%
\pgfpathlineto{\pgfqpoint{3.395468in}{1.113664in}}%
\pgfpathlineto{\pgfqpoint{3.405068in}{1.111928in}}%
\pgfpathlineto{\pgfqpoint{3.414487in}{1.110134in}}%
\pgfpathlineto{\pgfqpoint{3.423007in}{1.108686in}}%
\pgfpathlineto{\pgfqpoint{3.433866in}{1.106847in}}%
\pgfpathlineto{\pgfqpoint{3.441905in}{1.105480in}}%
\pgfpathlineto{\pgfqpoint{3.443945in}{1.105129in}}%
\pgfpathlineto{\pgfqpoint{3.444005in}{1.109021in}}%
\pgfpathlineto{\pgfqpoint{3.445505in}{1.108700in}}%
\pgfpathlineto{\pgfqpoint{3.455344in}{1.106751in}}%
\pgfpathlineto{\pgfqpoint{3.464104in}{1.105006in}}%
\pgfpathlineto{\pgfqpoint{3.473703in}{1.103016in}}%
\pgfpathlineto{\pgfqpoint{3.481922in}{1.101287in}}%
\pgfpathlineto{\pgfqpoint{3.490082in}{1.099733in}}%
\pgfpathlineto{\pgfqpoint{3.498901in}{1.098004in}}%
\pgfpathlineto{\pgfqpoint{3.507481in}{1.096155in}}%
\pgfpathlineto{\pgfqpoint{3.515880in}{1.094445in}}%
\pgfpathlineto{\pgfqpoint{3.523800in}{1.092739in}}%
\pgfpathlineto{\pgfqpoint{3.532619in}{1.091035in}}%
\pgfpathlineto{\pgfqpoint{3.538979in}{1.089916in}}%
\pgfpathlineto{\pgfqpoint{3.546538in}{1.088241in}}%
\pgfpathlineto{\pgfqpoint{3.554097in}{1.086482in}}%
\pgfpathlineto{\pgfqpoint{3.562497in}{1.084782in}}%
\pgfpathlineto{\pgfqpoint{3.568736in}{1.083509in}}%
\pgfpathlineto{\pgfqpoint{3.577076in}{1.081841in}}%
\pgfpathlineto{\pgfqpoint{3.582895in}{1.080679in}}%
\pgfpathlineto{\pgfqpoint{3.591475in}{1.078985in}}%
\pgfpathlineto{\pgfqpoint{3.596935in}{1.077913in}}%
\pgfpathlineto{\pgfqpoint{3.604254in}{1.076250in}}%
\pgfpathlineto{\pgfqpoint{3.609654in}{1.075147in}}%
\pgfpathlineto{\pgfqpoint{3.619193in}{1.073459in}}%
\pgfpathlineto{\pgfqpoint{3.624233in}{1.072456in}}%
\pgfpathlineto{\pgfqpoint{3.632392in}{1.070807in}}%
\pgfpathlineto{\pgfqpoint{3.638212in}{1.069296in}}%
\pgfpathlineto{\pgfqpoint{3.646071in}{1.067658in}}%
\pgfpathlineto{\pgfqpoint{3.651531in}{1.066234in}}%
\pgfpathlineto{\pgfqpoint{3.658550in}{1.064621in}}%
\pgfpathlineto{\pgfqpoint{3.662810in}{1.063810in}}%
\pgfpathlineto{\pgfqpoint{3.669769in}{1.062188in}}%
\pgfpathlineto{\pgfqpoint{3.674029in}{1.061317in}}%
\pgfpathlineto{\pgfqpoint{3.681409in}{1.059678in}}%
\pgfpathlineto{\pgfqpoint{3.685368in}{1.058940in}}%
\pgfpathlineto{\pgfqpoint{3.693588in}{1.057322in}}%
\pgfpathlineto{\pgfqpoint{3.697428in}{1.056604in}}%
\pgfpathlineto{\pgfqpoint{3.707747in}{1.054957in}}%
\pgfpathlineto{\pgfqpoint{3.711467in}{1.054262in}}%
\pgfpathlineto{\pgfqpoint{3.718666in}{1.052651in}}%
\pgfpathlineto{\pgfqpoint{3.722566in}{1.051767in}}%
\pgfpathlineto{\pgfqpoint{3.730185in}{1.050145in}}%
\pgfpathlineto{\pgfqpoint{3.734085in}{1.049190in}}%
\pgfpathlineto{\pgfqpoint{3.744044in}{1.047535in}}%
\pgfpathlineto{\pgfqpoint{3.747464in}{1.046870in}}%
\pgfpathlineto{\pgfqpoint{3.755564in}{1.045282in}}%
\pgfpathlineto{\pgfqpoint{3.758863in}{1.044647in}}%
\pgfpathlineto{\pgfqpoint{3.766783in}{1.043066in}}%
\pgfpathlineto{\pgfqpoint{3.769903in}{1.042511in}}%
\pgfpathlineto{\pgfqpoint{3.771642in}{1.042350in}}%
\pgfpathlineto{\pgfqpoint{3.771702in}{1.079985in}}%
\pgfpathlineto{\pgfqpoint{3.773202in}{1.079773in}}%
\pgfpathlineto{\pgfqpoint{3.782622in}{1.078141in}}%
\pgfpathlineto{\pgfqpoint{3.788981in}{1.076743in}}%
\pgfpathlineto{\pgfqpoint{3.796841in}{1.075126in}}%
\pgfpathlineto{\pgfqpoint{3.802660in}{1.073893in}}%
\pgfpathlineto{\pgfqpoint{3.811060in}{1.072275in}}%
\pgfpathlineto{\pgfqpoint{3.817179in}{1.070773in}}%
\pgfpathlineto{\pgfqpoint{3.825759in}{1.069155in}}%
\pgfpathlineto{\pgfqpoint{3.831278in}{1.067862in}}%
\pgfpathlineto{\pgfqpoint{3.837218in}{1.066275in}}%
\pgfpathlineto{\pgfqpoint{3.842258in}{1.065143in}}%
\pgfpathlineto{\pgfqpoint{3.850897in}{1.063547in}}%
\pgfpathlineto{\pgfqpoint{3.855817in}{1.062412in}}%
\pgfpathlineto{\pgfqpoint{3.865716in}{1.060823in}}%
\pgfpathlineto{\pgfqpoint{3.870096in}{1.059911in}}%
\pgfpathlineto{\pgfqpoint{3.878675in}{1.058320in}}%
\pgfpathlineto{\pgfqpoint{3.882875in}{1.057457in}}%
\pgfpathlineto{\pgfqpoint{3.890914in}{1.055859in}}%
\pgfpathlineto{\pgfqpoint{3.894934in}{1.055048in}}%
\pgfpathlineto{\pgfqpoint{3.902133in}{1.053487in}}%
\pgfpathlineto{\pgfqpoint{3.905853in}{1.052815in}}%
\pgfpathlineto{\pgfqpoint{3.914673in}{1.051239in}}%
\pgfpathlineto{\pgfqpoint{3.918692in}{1.050295in}}%
\pgfpathlineto{\pgfqpoint{3.924932in}{1.048727in}}%
\pgfpathlineto{\pgfqpoint{3.928712in}{1.047883in}}%
\pgfpathlineto{\pgfqpoint{3.935671in}{1.046304in}}%
\pgfpathlineto{\pgfqpoint{3.939391in}{1.045433in}}%
\pgfpathlineto{\pgfqpoint{3.946170in}{1.043850in}}%
\pgfpathlineto{\pgfqpoint{3.949470in}{1.043231in}}%
\pgfpathlineto{\pgfqpoint{3.957150in}{1.041642in}}%
\pgfpathlineto{\pgfqpoint{3.960749in}{1.040718in}}%
\pgfpathlineto{\pgfqpoint{3.966749in}{1.039183in}}%
\pgfpathlineto{\pgfqpoint{3.969989in}{1.038477in}}%
\pgfpathlineto{\pgfqpoint{3.975868in}{1.036948in}}%
\pgfpathlineto{\pgfqpoint{3.978628in}{1.036629in}}%
\pgfpathlineto{\pgfqpoint{3.985468in}{1.035056in}}%
\pgfpathlineto{\pgfqpoint{3.989187in}{1.033836in}}%
\pgfpathlineto{\pgfqpoint{3.997467in}{1.032249in}}%
\pgfpathlineto{\pgfqpoint{4.000407in}{1.031639in}}%
\pgfpathlineto{\pgfqpoint{4.007306in}{1.030069in}}%
\pgfpathlineto{\pgfqpoint{4.010786in}{1.028934in}}%
\pgfpathlineto{\pgfqpoint{4.016905in}{1.027367in}}%
\pgfpathlineto{\pgfqpoint{4.019665in}{1.026833in}}%
\pgfpathlineto{\pgfqpoint{4.027345in}{1.025272in}}%
\pgfpathlineto{\pgfqpoint{4.030345in}{1.024424in}}%
\pgfpathlineto{\pgfqpoint{4.037064in}{1.022878in}}%
\pgfpathlineto{\pgfqpoint{4.039884in}{1.022189in}}%
\pgfpathlineto{\pgfqpoint{4.047623in}{1.020654in}}%
\pgfpathlineto{\pgfqpoint{4.050383in}{1.019944in}}%
\pgfpathlineto{\pgfqpoint{4.059683in}{1.018421in}}%
\pgfpathlineto{\pgfqpoint{4.062382in}{1.017755in}}%
\pgfpathlineto{\pgfqpoint{4.069822in}{1.016196in}}%
\pgfpathlineto{\pgfqpoint{4.072402in}{1.015578in}}%
\pgfpathlineto{\pgfqpoint{4.079601in}{1.014042in}}%
\pgfpathlineto{\pgfqpoint{4.082481in}{1.013056in}}%
\pgfpathlineto{\pgfqpoint{4.091060in}{1.011533in}}%
\pgfpathlineto{\pgfqpoint{4.093520in}{1.010948in}}%
\pgfpathlineto{\pgfqpoint{4.099340in}{1.010272in}}%
\pgfpathlineto{\pgfqpoint{4.099400in}{1.045127in}}%
\pgfpathlineto{\pgfqpoint{4.100900in}{1.044898in}}%
\pgfpathlineto{\pgfqpoint{4.104739in}{1.043939in}}%
\pgfpathlineto{\pgfqpoint{4.112839in}{1.042398in}}%
\pgfpathlineto{\pgfqpoint{4.116559in}{1.041453in}}%
\pgfpathlineto{\pgfqpoint{4.125318in}{1.039905in}}%
\pgfpathlineto{\pgfqpoint{4.128978in}{1.038927in}}%
\pgfpathlineto{\pgfqpoint{4.137377in}{1.037377in}}%
\pgfpathlineto{\pgfqpoint{4.140617in}{1.036668in}}%
\pgfpathlineto{\pgfqpoint{4.148716in}{1.035122in}}%
\pgfpathlineto{\pgfqpoint{4.151896in}{1.034391in}}%
\pgfpathlineto{\pgfqpoint{4.160296in}{1.032848in}}%
\pgfpathlineto{\pgfqpoint{4.163655in}{1.031934in}}%
\pgfpathlineto{\pgfqpoint{4.172355in}{1.030392in}}%
\pgfpathlineto{\pgfqpoint{4.175115in}{1.029945in}}%
\pgfpathlineto{\pgfqpoint{4.184114in}{1.028417in}}%
\pgfpathlineto{\pgfqpoint{4.187114in}{1.027674in}}%
\pgfpathlineto{\pgfqpoint{4.195573in}{1.026158in}}%
\pgfpathlineto{\pgfqpoint{4.198513in}{1.025451in}}%
\pgfpathlineto{\pgfqpoint{4.204752in}{1.023894in}}%
\pgfpathlineto{\pgfqpoint{4.207452in}{1.023346in}}%
\pgfpathlineto{\pgfqpoint{4.215912in}{1.021800in}}%
\pgfpathlineto{\pgfqpoint{4.218492in}{1.021298in}}%
\pgfpathlineto{\pgfqpoint{4.224611in}{1.019777in}}%
\pgfpathlineto{\pgfqpoint{4.227311in}{1.019134in}}%
\pgfpathlineto{\pgfqpoint{4.235110in}{1.017628in}}%
\pgfpathlineto{\pgfqpoint{4.237750in}{1.017031in}}%
\pgfpathlineto{\pgfqpoint{4.246390in}{1.015479in}}%
\pgfpathlineto{\pgfqpoint{4.248729in}{1.015136in}}%
\pgfpathlineto{\pgfqpoint{4.256409in}{1.013614in}}%
\pgfpathlineto{\pgfqpoint{4.258809in}{1.013193in}}%
\pgfpathlineto{\pgfqpoint{4.265948in}{1.011703in}}%
\pgfpathlineto{\pgfqpoint{4.268468in}{1.011056in}}%
\pgfpathlineto{\pgfqpoint{4.273568in}{1.009526in}}%
\pgfpathlineto{\pgfqpoint{4.276148in}{1.008785in}}%
\pgfpathlineto{\pgfqpoint{4.282927in}{1.007277in}}%
\pgfpathlineto{\pgfqpoint{4.285207in}{1.006818in}}%
\pgfpathlineto{\pgfqpoint{4.297146in}{1.005257in}}%
\pgfpathlineto{\pgfqpoint{4.299366in}{1.004864in}}%
\pgfpathlineto{\pgfqpoint{4.306265in}{1.003351in}}%
\pgfpathlineto{\pgfqpoint{4.309025in}{1.002221in}}%
\pgfpathlineto{\pgfqpoint{4.318205in}{1.000730in}}%
\pgfpathlineto{\pgfqpoint{4.320664in}{0.999894in}}%
\pgfpathlineto{\pgfqpoint{4.326364in}{0.998362in}}%
\pgfpathlineto{\pgfqpoint{4.328464in}{0.998015in}}%
\pgfpathlineto{\pgfqpoint{4.335843in}{0.996526in}}%
\pgfpathlineto{\pgfqpoint{4.338183in}{0.995817in}}%
\pgfpathlineto{\pgfqpoint{4.344903in}{0.994283in}}%
\pgfpathlineto{\pgfqpoint{4.347303in}{0.993368in}}%
\pgfpathlineto{\pgfqpoint{4.356842in}{0.991869in}}%
\pgfpathlineto{\pgfqpoint{4.358822in}{0.991585in}}%
\pgfpathlineto{\pgfqpoint{4.370041in}{0.990046in}}%
\pgfpathlineto{\pgfqpoint{4.372501in}{0.988978in}}%
\pgfpathlineto{\pgfqpoint{4.382640in}{0.987484in}}%
\pgfpathlineto{\pgfqpoint{4.384980in}{0.986580in}}%
\pgfpathlineto{\pgfqpoint{4.394639in}{0.985036in}}%
\pgfpathlineto{\pgfqpoint{4.396979in}{0.983996in}}%
\pgfpathlineto{\pgfqpoint{4.404839in}{0.982506in}}%
\pgfpathlineto{\pgfqpoint{4.406939in}{0.981860in}}%
\pgfpathlineto{\pgfqpoint{4.416538in}{0.980322in}}%
\pgfpathlineto{\pgfqpoint{4.418458in}{0.979988in}}%
\pgfpathlineto{\pgfqpoint{4.425357in}{0.978527in}}%
\pgfpathlineto{\pgfqpoint{4.427037in}{0.978299in}}%
\pgfpathlineto{\pgfqpoint{4.427097in}{1.027842in}}%
\pgfpathlineto{\pgfqpoint{4.428597in}{1.027332in}}%
\pgfpathlineto{\pgfqpoint{4.434297in}{1.025836in}}%
\pgfpathlineto{\pgfqpoint{4.437776in}{1.024607in}}%
\pgfpathlineto{\pgfqpoint{4.445096in}{1.023110in}}%
\pgfpathlineto{\pgfqpoint{4.448096in}{1.022260in}}%
\pgfpathlineto{\pgfqpoint{4.456075in}{1.020713in}}%
\pgfpathlineto{\pgfqpoint{4.459015in}{1.019834in}}%
\pgfpathlineto{\pgfqpoint{4.467294in}{1.018297in}}%
\pgfpathlineto{\pgfqpoint{4.470054in}{1.017583in}}%
\pgfpathlineto{\pgfqpoint{4.476774in}{1.016066in}}%
\pgfpathlineto{\pgfqpoint{4.479414in}{1.015396in}}%
\pgfpathlineto{\pgfqpoint{4.487693in}{1.013898in}}%
\pgfpathlineto{\pgfqpoint{4.490093in}{1.013484in}}%
\pgfpathlineto{\pgfqpoint{4.497412in}{1.011946in}}%
\pgfpathlineto{\pgfqpoint{4.499632in}{1.011662in}}%
\pgfpathlineto{\pgfqpoint{4.506052in}{1.010157in}}%
\pgfpathlineto{\pgfqpoint{4.508452in}{1.009648in}}%
\pgfpathlineto{\pgfqpoint{4.515411in}{1.008100in}}%
\pgfpathlineto{\pgfqpoint{4.517931in}{1.007427in}}%
\pgfpathlineto{\pgfqpoint{4.525130in}{1.005905in}}%
\pgfpathlineto{\pgfqpoint{4.527470in}{1.005366in}}%
\pgfpathlineto{\pgfqpoint{4.534850in}{1.003878in}}%
\pgfpathlineto{\pgfqpoint{4.537310in}{1.003164in}}%
\pgfpathlineto{\pgfqpoint{4.544329in}{1.001630in}}%
\pgfpathlineto{\pgfqpoint{4.546609in}{1.001139in}}%
\pgfpathlineto{\pgfqpoint{4.553808in}{0.999644in}}%
\pgfpathlineto{\pgfqpoint{4.555788in}{0.999475in}}%
\pgfpathlineto{\pgfqpoint{4.564788in}{0.997943in}}%
\pgfpathlineto{\pgfqpoint{4.567007in}{0.997426in}}%
\pgfpathlineto{\pgfqpoint{4.577267in}{0.995936in}}%
\pgfpathlineto{\pgfqpoint{4.579547in}{0.995315in}}%
\pgfpathlineto{\pgfqpoint{4.585666in}{0.993785in}}%
\pgfpathlineto{\pgfqpoint{4.587646in}{0.993510in}}%
\pgfpathlineto{\pgfqpoint{4.594426in}{0.992033in}}%
\pgfpathlineto{\pgfqpoint{4.596525in}{0.991564in}}%
\pgfpathlineto{\pgfqpoint{4.604325in}{0.990045in}}%
\pgfpathlineto{\pgfqpoint{4.606485in}{0.989467in}}%
\pgfpathlineto{\pgfqpoint{4.611704in}{0.988002in}}%
\pgfpathlineto{\pgfqpoint{4.613684in}{0.987705in}}%
\pgfpathlineto{\pgfqpoint{4.621004in}{0.986202in}}%
\pgfpathlineto{\pgfqpoint{4.623104in}{0.985592in}}%
\pgfpathlineto{\pgfqpoint{4.632163in}{0.984042in}}%
\pgfpathlineto{\pgfqpoint{4.634143in}{0.983623in}}%
\pgfpathlineto{\pgfqpoint{4.639602in}{0.982139in}}%
\pgfpathlineto{\pgfqpoint{4.641642in}{0.981601in}}%
\pgfpathlineto{\pgfqpoint{4.646202in}{0.980075in}}%
\pgfpathlineto{\pgfqpoint{4.648362in}{0.979299in}}%
\pgfpathlineto{\pgfqpoint{4.656881in}{0.977833in}}%
\pgfpathlineto{\pgfqpoint{4.658981in}{0.977147in}}%
\pgfpathlineto{\pgfqpoint{4.669540in}{0.975634in}}%
\pgfpathlineto{\pgfqpoint{4.671520in}{0.975044in}}%
\pgfpathlineto{\pgfqpoint{4.677460in}{0.973610in}}%
\pgfpathlineto{\pgfqpoint{4.679320in}{0.973246in}}%
\pgfpathlineto{\pgfqpoint{4.685439in}{0.971774in}}%
\pgfpathlineto{\pgfqpoint{4.687299in}{0.971401in}}%
\pgfpathlineto{\pgfqpoint{4.695819in}{0.969888in}}%
\pgfpathlineto{\pgfqpoint{4.697738in}{0.969378in}}%
\pgfpathlineto{\pgfqpoint{4.702298in}{0.967954in}}%
\pgfpathlineto{\pgfqpoint{4.704158in}{0.967561in}}%
\pgfpathlineto{\pgfqpoint{4.713097in}{0.966098in}}%
\pgfpathlineto{\pgfqpoint{4.714897in}{0.965828in}}%
\pgfpathlineto{\pgfqpoint{4.720777in}{0.964331in}}%
\pgfpathlineto{\pgfqpoint{4.722757in}{0.963640in}}%
\pgfpathlineto{\pgfqpoint{4.730376in}{0.962094in}}%
\pgfpathlineto{\pgfqpoint{4.732176in}{0.961667in}}%
\pgfpathlineto{\pgfqpoint{4.737276in}{0.960225in}}%
\pgfpathlineto{\pgfqpoint{4.739256in}{0.959346in}}%
\pgfpathlineto{\pgfqpoint{4.743995in}{0.957855in}}%
\pgfpathlineto{\pgfqpoint{4.745795in}{0.957401in}}%
\pgfpathlineto{\pgfqpoint{4.752695in}{0.955868in}}%
\pgfpathlineto{\pgfqpoint{4.754435in}{0.955557in}}%
\pgfpathlineto{\pgfqpoint{4.754794in}{0.955557in}}%
\pgfpathlineto{\pgfqpoint{4.754794in}{0.955557in}}%
\pgfusepath{stroke}%
\end{pgfscope}%
\begin{pgfscope}%
\pgfpathrectangle{\pgfqpoint{0.626312in}{0.402778in}}{\pgfqpoint{4.325077in}{2.645833in}}%
\pgfusepath{clip}%
\pgfsetrectcap%
\pgfsetroundjoin%
\pgfsetlinewidth{1.505625pt}%
\definecolor{currentstroke}{rgb}{0.890196,0.466667,0.760784}%
\pgfsetstrokecolor{currentstroke}%
\pgfsetdash{}{0pt}%
\pgfpathmoveto{\pgfqpoint{3.116368in}{0.388889in}}%
\pgfpathlineto{\pgfqpoint{3.116368in}{0.642136in}}%
\pgfpathlineto{\pgfqpoint{3.117868in}{0.642049in}}%
\pgfpathlineto{\pgfqpoint{3.153685in}{0.640055in}}%
\pgfpathlineto{\pgfqpoint{3.155245in}{0.619596in}}%
\pgfpathlineto{\pgfqpoint{3.321494in}{0.612289in}}%
\pgfpathlineto{\pgfqpoint{3.443945in}{0.608561in}}%
\pgfpathlineto{\pgfqpoint{3.444005in}{0.668569in}}%
\pgfpathlineto{\pgfqpoint{3.445505in}{0.668531in}}%
\pgfpathlineto{\pgfqpoint{3.603354in}{0.665240in}}%
\pgfpathlineto{\pgfqpoint{3.604914in}{0.654296in}}%
\pgfpathlineto{\pgfqpoint{3.604974in}{0.654295in}}%
\pgfpathlineto{\pgfqpoint{3.606534in}{0.641356in}}%
\pgfpathlineto{\pgfqpoint{3.693348in}{0.640089in}}%
\pgfpathlineto{\pgfqpoint{3.694908in}{0.624265in}}%
\pgfpathlineto{\pgfqpoint{3.704387in}{0.624146in}}%
\pgfpathlineto{\pgfqpoint{3.705947in}{0.603751in}}%
\pgfpathlineto{\pgfqpoint{3.771642in}{0.603009in}}%
\pgfpathlineto{\pgfqpoint{3.771702in}{0.763688in}}%
\pgfpathlineto{\pgfqpoint{3.773202in}{0.763672in}}%
\pgfpathlineto{\pgfqpoint{3.805660in}{0.763356in}}%
\pgfpathlineto{\pgfqpoint{3.807220in}{0.760856in}}%
\pgfpathlineto{\pgfqpoint{3.809440in}{0.760836in}}%
\pgfpathlineto{\pgfqpoint{3.811000in}{0.758246in}}%
\pgfpathlineto{\pgfqpoint{3.824139in}{0.758128in}}%
\pgfpathlineto{\pgfqpoint{3.825699in}{0.755441in}}%
\pgfpathlineto{\pgfqpoint{3.879695in}{0.755004in}}%
\pgfpathlineto{\pgfqpoint{3.881315in}{0.749323in}}%
\pgfpathlineto{\pgfqpoint{3.882695in}{0.749312in}}%
\pgfpathlineto{\pgfqpoint{3.884255in}{0.746287in}}%
\pgfpathlineto{\pgfqpoint{3.912873in}{0.746084in}}%
\pgfpathlineto{\pgfqpoint{3.914433in}{0.742925in}}%
\pgfpathlineto{\pgfqpoint{3.928952in}{0.742829in}}%
\pgfpathlineto{\pgfqpoint{3.930511in}{0.739525in}}%
\pgfpathlineto{\pgfqpoint{3.933451in}{0.739506in}}%
\pgfpathlineto{\pgfqpoint{3.935011in}{0.736040in}}%
\pgfpathlineto{\pgfqpoint{3.962369in}{0.735873in}}%
\pgfpathlineto{\pgfqpoint{3.963929in}{0.732232in}}%
\pgfpathlineto{\pgfqpoint{3.992967in}{0.732069in}}%
\pgfpathlineto{\pgfqpoint{3.994527in}{0.728232in}}%
\pgfpathlineto{\pgfqpoint{3.998367in}{0.728212in}}%
\pgfpathlineto{\pgfqpoint{3.999927in}{0.724155in}}%
\pgfpathlineto{\pgfqpoint{4.017325in}{0.724066in}}%
\pgfpathlineto{\pgfqpoint{4.018885in}{0.719765in}}%
\pgfpathlineto{\pgfqpoint{4.044324in}{0.719644in}}%
\pgfpathlineto{\pgfqpoint{4.045883in}{0.715066in}}%
\pgfpathlineto{\pgfqpoint{4.059923in}{0.715003in}}%
\pgfpathlineto{\pgfqpoint{4.061482in}{0.710110in}}%
\pgfpathlineto{\pgfqpoint{4.098860in}{0.709956in}}%
\pgfpathlineto{\pgfqpoint{4.100420in}{0.704701in}}%
\pgfpathlineto{\pgfqpoint{4.104199in}{0.704686in}}%
\pgfpathlineto{\pgfqpoint{4.105759in}{0.699011in}}%
\pgfpathlineto{\pgfqpoint{4.152496in}{0.698845in}}%
\pgfpathlineto{\pgfqpoint{4.154056in}{0.692677in}}%
\pgfpathlineto{\pgfqpoint{4.196173in}{0.692547in}}%
\pgfpathlineto{\pgfqpoint{4.197733in}{0.685792in}}%
\pgfpathlineto{\pgfqpoint{4.216452in}{0.685739in}}%
\pgfpathlineto{\pgfqpoint{4.218012in}{0.678273in}}%
\pgfpathlineto{\pgfqpoint{4.236490in}{0.678224in}}%
\pgfpathlineto{\pgfqpoint{4.238050in}{0.669878in}}%
\pgfpathlineto{\pgfqpoint{4.267328in}{0.669806in}}%
\pgfpathlineto{\pgfqpoint{4.268888in}{0.660345in}}%
\pgfpathlineto{\pgfqpoint{4.306925in}{0.660261in}}%
\pgfpathlineto{\pgfqpoint{4.308485in}{0.649340in}}%
\pgfpathlineto{\pgfqpoint{4.359062in}{0.649243in}}%
\pgfpathlineto{\pgfqpoint{4.360622in}{0.636327in}}%
\pgfpathlineto{\pgfqpoint{4.427037in}{0.636221in}}%
\pgfpathlineto{\pgfqpoint{4.427097in}{0.771987in}}%
\pgfpathlineto{\pgfqpoint{4.428597in}{0.769870in}}%
\pgfpathlineto{\pgfqpoint{4.442816in}{0.769850in}}%
\pgfpathlineto{\pgfqpoint{4.444376in}{0.767668in}}%
\pgfpathlineto{\pgfqpoint{4.468374in}{0.767637in}}%
\pgfpathlineto{\pgfqpoint{4.469934in}{0.765386in}}%
\pgfpathlineto{\pgfqpoint{4.489013in}{0.765363in}}%
\pgfpathlineto{\pgfqpoint{4.490573in}{0.763038in}}%
\pgfpathlineto{\pgfqpoint{4.508392in}{0.763018in}}%
\pgfpathlineto{\pgfqpoint{4.509951in}{0.760615in}}%
\pgfpathlineto{\pgfqpoint{4.519371in}{0.760604in}}%
\pgfpathlineto{\pgfqpoint{4.520931in}{0.758117in}}%
\pgfpathlineto{\pgfqpoint{4.522431in}{0.758116in}}%
\pgfpathlineto{\pgfqpoint{4.523990in}{0.755538in}}%
\pgfpathlineto{\pgfqpoint{4.534610in}{0.755527in}}%
\pgfpathlineto{\pgfqpoint{4.536170in}{0.752852in}}%
\pgfpathlineto{\pgfqpoint{4.540609in}{0.752848in}}%
\pgfpathlineto{\pgfqpoint{4.542169in}{0.750068in}}%
\pgfpathlineto{\pgfqpoint{4.544929in}{0.750066in}}%
\pgfpathlineto{\pgfqpoint{4.546489in}{0.747173in}}%
\pgfpathlineto{\pgfqpoint{4.546909in}{0.747172in}}%
\pgfpathlineto{\pgfqpoint{4.548469in}{0.744157in}}%
\pgfpathlineto{\pgfqpoint{4.551829in}{0.744153in}}%
\pgfpathlineto{\pgfqpoint{4.553448in}{0.737709in}}%
\pgfpathlineto{\pgfqpoint{4.559568in}{0.737703in}}%
\pgfpathlineto{\pgfqpoint{4.561128in}{0.734246in}}%
\pgfpathlineto{\pgfqpoint{4.563288in}{0.734243in}}%
\pgfpathlineto{\pgfqpoint{4.564908in}{0.726780in}}%
\pgfpathlineto{\pgfqpoint{4.578467in}{0.726767in}}%
\pgfpathlineto{\pgfqpoint{4.580027in}{0.722717in}}%
\pgfpathlineto{\pgfqpoint{4.644162in}{0.722664in}}%
\pgfpathlineto{\pgfqpoint{4.645722in}{0.718370in}}%
\pgfpathlineto{\pgfqpoint{4.658921in}{0.718360in}}%
\pgfpathlineto{\pgfqpoint{4.660481in}{0.713788in}}%
\pgfpathlineto{\pgfqpoint{4.683939in}{0.713772in}}%
\pgfpathlineto{\pgfqpoint{4.685499in}{0.708884in}}%
\pgfpathlineto{\pgfqpoint{4.685679in}{0.708884in}}%
\pgfpathlineto{\pgfqpoint{4.687239in}{0.703634in}}%
\pgfpathlineto{\pgfqpoint{4.754794in}{0.703594in}}%
\pgfpathlineto{\pgfqpoint{4.754794in}{0.703594in}}%
\pgfusepath{stroke}%
\end{pgfscope}%
\begin{pgfscope}%
\pgfpathrectangle{\pgfqpoint{0.626312in}{0.402778in}}{\pgfqpoint{4.325077in}{2.645833in}}%
\pgfusepath{clip}%
\pgfsetrectcap%
\pgfsetroundjoin%
\pgfsetlinewidth{1.505625pt}%
\definecolor{currentstroke}{rgb}{0.498039,0.498039,0.498039}%
\pgfsetstrokecolor{currentstroke}%
\pgfsetdash{}{0pt}%
\pgfpathmoveto{\pgfqpoint{3.444005in}{0.388889in}}%
\pgfpathlineto{\pgfqpoint{3.444005in}{0.530751in}}%
\pgfpathlineto{\pgfqpoint{3.445505in}{0.530712in}}%
\pgfpathlineto{\pgfqpoint{3.676669in}{0.526321in}}%
\pgfpathlineto{\pgfqpoint{3.771642in}{0.525200in}}%
\pgfpathlineto{\pgfqpoint{3.771642in}{0.388889in}}%
\pgfpathlineto{\pgfqpoint{3.771642in}{0.388889in}}%
\pgfusepath{stroke}%
\end{pgfscope}%
\begin{pgfscope}%
\pgfpathrectangle{\pgfqpoint{0.626312in}{0.402778in}}{\pgfqpoint{4.325077in}{2.645833in}}%
\pgfusepath{clip}%
\pgfsetrectcap%
\pgfsetroundjoin%
\pgfsetlinewidth{1.505625pt}%
\definecolor{currentstroke}{rgb}{0.737255,0.741176,0.133333}%
\pgfsetstrokecolor{currentstroke}%
\pgfsetdash{}{0pt}%
\pgfpathmoveto{\pgfqpoint{3.771702in}{0.388889in}}%
\pgfpathlineto{\pgfqpoint{3.771702in}{0.525199in}}%
\pgfpathlineto{\pgfqpoint{3.773202in}{0.525184in}}%
\pgfpathlineto{\pgfqpoint{4.099340in}{0.523043in}}%
\pgfpathlineto{\pgfqpoint{4.099340in}{0.388889in}}%
\pgfpathlineto{\pgfqpoint{4.099340in}{0.388889in}}%
\pgfusepath{stroke}%
\end{pgfscope}%
\begin{pgfscope}%
\pgfpathrectangle{\pgfqpoint{0.626312in}{0.402778in}}{\pgfqpoint{4.325077in}{2.645833in}}%
\pgfusepath{clip}%
\pgfsetrectcap%
\pgfsetroundjoin%
\pgfsetlinewidth{1.505625pt}%
\definecolor{currentstroke}{rgb}{0.090196,0.745098,0.811765}%
\pgfsetstrokecolor{currentstroke}%
\pgfsetdash{}{0pt}%
\pgfpathmoveto{\pgfqpoint{0.000000in}{0.000000in}}%
\pgfusepath{stroke}%
\end{pgfscope}%
\begin{pgfscope}%
\pgfpathrectangle{\pgfqpoint{0.626312in}{0.402778in}}{\pgfqpoint{4.325077in}{2.645833in}}%
\pgfusepath{clip}%
\pgfsetrectcap%
\pgfsetroundjoin%
\pgfsetlinewidth{1.505625pt}%
\definecolor{currentstroke}{rgb}{0.121569,0.466667,0.705882}%
\pgfsetstrokecolor{currentstroke}%
\pgfsetdash{}{0pt}%
\pgfpathmoveto{\pgfqpoint{0.000000in}{0.000000in}}%
\pgfusepath{stroke}%
\end{pgfscope}%
\begin{pgfscope}%
\pgfpathrectangle{\pgfqpoint{0.626312in}{0.402778in}}{\pgfqpoint{4.325077in}{2.645833in}}%
\pgfusepath{clip}%
\pgfsetrectcap%
\pgfsetroundjoin%
\pgfsetlinewidth{1.505625pt}%
\definecolor{currentstroke}{rgb}{1.000000,0.498039,0.054902}%
\pgfsetstrokecolor{currentstroke}%
\pgfsetdash{}{0pt}%
\pgfpathmoveto{\pgfqpoint{0.000000in}{0.000000in}}%
\pgfusepath{stroke}%
\end{pgfscope}%
\begin{pgfscope}%
\pgfsetrectcap%
\pgfsetmiterjoin%
\pgfsetlinewidth{0.803000pt}%
\definecolor{currentstroke}{rgb}{0.000000,0.000000,0.000000}%
\pgfsetstrokecolor{currentstroke}%
\pgfsetdash{}{0pt}%
\pgfpathmoveto{\pgfqpoint{0.626312in}{0.402778in}}%
\pgfpathlineto{\pgfqpoint{0.626312in}{3.048611in}}%
\pgfusepath{stroke}%
\end{pgfscope}%
\begin{pgfscope}%
\pgfsetrectcap%
\pgfsetmiterjoin%
\pgfsetlinewidth{0.803000pt}%
\definecolor{currentstroke}{rgb}{0.000000,0.000000,0.000000}%
\pgfsetstrokecolor{currentstroke}%
\pgfsetdash{}{0pt}%
\pgfpathmoveto{\pgfqpoint{4.951389in}{0.402778in}}%
\pgfpathlineto{\pgfqpoint{4.951389in}{3.048611in}}%
\pgfusepath{stroke}%
\end{pgfscope}%
\begin{pgfscope}%
\pgfsetrectcap%
\pgfsetmiterjoin%
\pgfsetlinewidth{0.803000pt}%
\definecolor{currentstroke}{rgb}{0.000000,0.000000,0.000000}%
\pgfsetstrokecolor{currentstroke}%
\pgfsetdash{}{0pt}%
\pgfpathmoveto{\pgfqpoint{0.626312in}{0.402778in}}%
\pgfpathlineto{\pgfqpoint{4.951389in}{0.402778in}}%
\pgfusepath{stroke}%
\end{pgfscope}%
\begin{pgfscope}%
\pgfsetrectcap%
\pgfsetmiterjoin%
\pgfsetlinewidth{0.803000pt}%
\definecolor{currentstroke}{rgb}{0.000000,0.000000,0.000000}%
\pgfsetstrokecolor{currentstroke}%
\pgfsetdash{}{0pt}%
\pgfpathmoveto{\pgfqpoint{0.626312in}{3.048611in}}%
\pgfpathlineto{\pgfqpoint{4.951389in}{3.048611in}}%
\pgfusepath{stroke}%
\end{pgfscope}%
\begin{pgfscope}%
\pgfsetbuttcap%
\pgfsetmiterjoin%
\definecolor{currentfill}{rgb}{1.000000,1.000000,1.000000}%
\pgfsetfillcolor{currentfill}%
\pgfsetfillopacity{0.800000}%
\pgfsetlinewidth{1.003750pt}%
\definecolor{currentstroke}{rgb}{0.800000,0.800000,0.800000}%
\pgfsetstrokecolor{currentstroke}%
\pgfsetstrokeopacity{0.800000}%
\pgfsetdash{}{0pt}%
\pgfpathmoveto{\pgfqpoint{0.723534in}{0.581057in}}%
\pgfpathlineto{\pgfqpoint{1.306867in}{0.581057in}}%
\pgfpathquadraticcurveto{\pgfqpoint{1.334645in}{0.581057in}}{\pgfqpoint{1.334645in}{0.608834in}}%
\pgfpathlineto{\pgfqpoint{1.334645in}{2.951389in}}%
\pgfpathquadraticcurveto{\pgfqpoint{1.334645in}{2.979167in}}{\pgfqpoint{1.306867in}{2.979167in}}%
\pgfpathlineto{\pgfqpoint{0.723534in}{2.979167in}}%
\pgfpathquadraticcurveto{\pgfqpoint{0.695756in}{2.979167in}}{\pgfqpoint{0.695756in}{2.951389in}}%
\pgfpathlineto{\pgfqpoint{0.695756in}{0.608834in}}%
\pgfpathquadraticcurveto{\pgfqpoint{0.695756in}{0.581057in}}{\pgfqpoint{0.723534in}{0.581057in}}%
\pgfpathclose%
\pgfusepath{stroke,fill}%
\end{pgfscope}%
\begin{pgfscope}%
\pgfsetrectcap%
\pgfsetroundjoin%
\pgfsetlinewidth{1.505625pt}%
\definecolor{currentstroke}{rgb}{0.121569,0.466667,0.705882}%
\pgfsetstrokecolor{currentstroke}%
\pgfsetdash{}{0pt}%
\pgfpathmoveto{\pgfqpoint{0.751312in}{2.875000in}}%
\pgfpathlineto{\pgfqpoint{1.029090in}{2.875000in}}%
\pgfusepath{stroke}%
\end{pgfscope}%
\begin{pgfscope}%
\pgftext[x=1.140201in,y=2.826389in,left,base]{\rmfamily\fontsize{10.000000}{12.000000}\selectfont 1}%
\end{pgfscope}%
\begin{pgfscope}%
\pgfsetrectcap%
\pgfsetroundjoin%
\pgfsetlinewidth{1.505625pt}%
\definecolor{currentstroke}{rgb}{1.000000,0.498039,0.054902}%
\pgfsetstrokecolor{currentstroke}%
\pgfsetdash{}{0pt}%
\pgfpathmoveto{\pgfqpoint{0.751312in}{2.678630in}}%
\pgfpathlineto{\pgfqpoint{1.029090in}{2.678630in}}%
\pgfusepath{stroke}%
\end{pgfscope}%
\begin{pgfscope}%
\pgftext[x=1.140201in,y=2.630019in,left,base]{\rmfamily\fontsize{10.000000}{12.000000}\selectfont 2}%
\end{pgfscope}%
\begin{pgfscope}%
\pgfsetrectcap%
\pgfsetroundjoin%
\pgfsetlinewidth{1.505625pt}%
\definecolor{currentstroke}{rgb}{0.172549,0.627451,0.172549}%
\pgfsetstrokecolor{currentstroke}%
\pgfsetdash{}{0pt}%
\pgfpathmoveto{\pgfqpoint{0.751312in}{2.482259in}}%
\pgfpathlineto{\pgfqpoint{1.029090in}{2.482259in}}%
\pgfusepath{stroke}%
\end{pgfscope}%
\begin{pgfscope}%
\pgftext[x=1.140201in,y=2.433648in,left,base]{\rmfamily\fontsize{10.000000}{12.000000}\selectfont 3}%
\end{pgfscope}%
\begin{pgfscope}%
\pgfsetrectcap%
\pgfsetroundjoin%
\pgfsetlinewidth{1.505625pt}%
\definecolor{currentstroke}{rgb}{0.839216,0.152941,0.156863}%
\pgfsetstrokecolor{currentstroke}%
\pgfsetdash{}{0pt}%
\pgfpathmoveto{\pgfqpoint{0.751312in}{2.285889in}}%
\pgfpathlineto{\pgfqpoint{1.029090in}{2.285889in}}%
\pgfusepath{stroke}%
\end{pgfscope}%
\begin{pgfscope}%
\pgftext[x=1.140201in,y=2.237278in,left,base]{\rmfamily\fontsize{10.000000}{12.000000}\selectfont 4}%
\end{pgfscope}%
\begin{pgfscope}%
\pgfsetrectcap%
\pgfsetroundjoin%
\pgfsetlinewidth{1.505625pt}%
\definecolor{currentstroke}{rgb}{0.580392,0.403922,0.741176}%
\pgfsetstrokecolor{currentstroke}%
\pgfsetdash{}{0pt}%
\pgfpathmoveto{\pgfqpoint{0.751312in}{2.089519in}}%
\pgfpathlineto{\pgfqpoint{1.029090in}{2.089519in}}%
\pgfusepath{stroke}%
\end{pgfscope}%
\begin{pgfscope}%
\pgftext[x=1.140201in,y=2.040908in,left,base]{\rmfamily\fontsize{10.000000}{12.000000}\selectfont 5}%
\end{pgfscope}%
\begin{pgfscope}%
\pgfsetrectcap%
\pgfsetroundjoin%
\pgfsetlinewidth{1.505625pt}%
\definecolor{currentstroke}{rgb}{0.549020,0.337255,0.294118}%
\pgfsetstrokecolor{currentstroke}%
\pgfsetdash{}{0pt}%
\pgfpathmoveto{\pgfqpoint{0.751312in}{1.893149in}}%
\pgfpathlineto{\pgfqpoint{1.029090in}{1.893149in}}%
\pgfusepath{stroke}%
\end{pgfscope}%
\begin{pgfscope}%
\pgftext[x=1.140201in,y=1.844537in,left,base]{\rmfamily\fontsize{10.000000}{12.000000}\selectfont 6}%
\end{pgfscope}%
\begin{pgfscope}%
\pgfsetrectcap%
\pgfsetroundjoin%
\pgfsetlinewidth{1.505625pt}%
\definecolor{currentstroke}{rgb}{0.890196,0.466667,0.760784}%
\pgfsetstrokecolor{currentstroke}%
\pgfsetdash{}{0pt}%
\pgfpathmoveto{\pgfqpoint{0.751312in}{1.696778in}}%
\pgfpathlineto{\pgfqpoint{1.029090in}{1.696778in}}%
\pgfusepath{stroke}%
\end{pgfscope}%
\begin{pgfscope}%
\pgftext[x=1.140201in,y=1.648167in,left,base]{\rmfamily\fontsize{10.000000}{12.000000}\selectfont 7}%
\end{pgfscope}%
\begin{pgfscope}%
\pgfsetrectcap%
\pgfsetroundjoin%
\pgfsetlinewidth{1.505625pt}%
\definecolor{currentstroke}{rgb}{0.498039,0.498039,0.498039}%
\pgfsetstrokecolor{currentstroke}%
\pgfsetdash{}{0pt}%
\pgfpathmoveto{\pgfqpoint{0.751312in}{1.500408in}}%
\pgfpathlineto{\pgfqpoint{1.029090in}{1.500408in}}%
\pgfusepath{stroke}%
\end{pgfscope}%
\begin{pgfscope}%
\pgftext[x=1.140201in,y=1.451797in,left,base]{\rmfamily\fontsize{10.000000}{12.000000}\selectfont 8}%
\end{pgfscope}%
\begin{pgfscope}%
\pgfsetrectcap%
\pgfsetroundjoin%
\pgfsetlinewidth{1.505625pt}%
\definecolor{currentstroke}{rgb}{0.737255,0.741176,0.133333}%
\pgfsetstrokecolor{currentstroke}%
\pgfsetdash{}{0pt}%
\pgfpathmoveto{\pgfqpoint{0.751312in}{1.304038in}}%
\pgfpathlineto{\pgfqpoint{1.029090in}{1.304038in}}%
\pgfusepath{stroke}%
\end{pgfscope}%
\begin{pgfscope}%
\pgftext[x=1.140201in,y=1.255427in,left,base]{\rmfamily\fontsize{10.000000}{12.000000}\selectfont 9}%
\end{pgfscope}%
\begin{pgfscope}%
\pgfsetrectcap%
\pgfsetroundjoin%
\pgfsetlinewidth{1.505625pt}%
\definecolor{currentstroke}{rgb}{0.090196,0.745098,0.811765}%
\pgfsetstrokecolor{currentstroke}%
\pgfsetdash{}{0pt}%
\pgfpathmoveto{\pgfqpoint{0.751312in}{1.107668in}}%
\pgfpathlineto{\pgfqpoint{1.029090in}{1.107668in}}%
\pgfusepath{stroke}%
\end{pgfscope}%
\begin{pgfscope}%
\pgftext[x=1.140201in,y=1.059056in,left,base]{\rmfamily\fontsize{10.000000}{12.000000}\selectfont 10}%
\end{pgfscope}%
\begin{pgfscope}%
\pgfsetrectcap%
\pgfsetroundjoin%
\pgfsetlinewidth{1.505625pt}%
\definecolor{currentstroke}{rgb}{0.121569,0.466667,0.705882}%
\pgfsetstrokecolor{currentstroke}%
\pgfsetdash{}{0pt}%
\pgfpathmoveto{\pgfqpoint{0.751312in}{0.911297in}}%
\pgfpathlineto{\pgfqpoint{1.029090in}{0.911297in}}%
\pgfusepath{stroke}%
\end{pgfscope}%
\begin{pgfscope}%
\pgftext[x=1.140201in,y=0.862686in,left,base]{\rmfamily\fontsize{10.000000}{12.000000}\selectfont 11}%
\end{pgfscope}%
\begin{pgfscope}%
\pgfsetrectcap%
\pgfsetroundjoin%
\pgfsetlinewidth{1.505625pt}%
\definecolor{currentstroke}{rgb}{1.000000,0.498039,0.054902}%
\pgfsetstrokecolor{currentstroke}%
\pgfsetdash{}{0pt}%
\pgfpathmoveto{\pgfqpoint{0.751312in}{0.714927in}}%
\pgfpathlineto{\pgfqpoint{1.029090in}{0.714927in}}%
\pgfusepath{stroke}%
\end{pgfscope}%
\begin{pgfscope}%
\pgftext[x=1.140201in,y=0.666316in,left,base]{\rmfamily\fontsize{10.000000}{12.000000}\selectfont 12}%
\end{pgfscope}%
\end{pgfpicture}%
\makeatother%
\endgroup%

    \caption{Lokaler relativer Fehler der Parareal Lösungen der logistischen Gleichung mit der numerischen Lösung als Vergleichswert.}
    \label{fig:iter_error_local_num}
\end{figure}

Bisher hat die Genauigkeit der zugrunde liegenden Diskretisierungsverfahren in den Experimenten keine Rolle gespielt. Untersucht wird nun wie sich eine größere Präzision von \(\mathcal{F}\), bei unverändertem \(\mathcal{G}\), auf dem Gesamtfehler auswirkt. Zur Berechnung wird die 2-Norm verwendet. Verschiedene Iterationszahlen werden in Abbildung~\ref{fig:iter_error} berücksichtigt. Die einzelnen Funktionen haben jeweils zwei Phasen. Eine der starken Verbesserung, wo die Erhöhung der Genauigkeit mit einer starken Verbesserung des Fehlers einhergeht. In der zweiten Phase wird dieser Zusammenhang abgeschwächt. Mit jeder zusätzlichen Iteration verschiebt sich der Übergang in Richtung der größeren Genauigkeit.
\begin{figure}[ht]
    \centering
        %% Creator: Matplotlib, PGF backend
%%
%% To include the figure in your LaTeX document, write
%%   \input{<filename>.pgf}
%%
%% Make sure the required packages are loaded in your preamble
%%   \usepackage{pgf}
%%
%% Figures using additional raster images can only be included by \input if
%% they are in the same directory as the main LaTeX file. For loading figures
%% from other directories you can use the `import` package
%%   \usepackage{import}
%% and then include the figures with
%%   \import{<path to file>}{<filename>.pgf}
%%
%% Matplotlib used the following preamble
%%   \usepackage{fontspec}
%%   \setmainfont{DejaVu Serif}
%%   \setsansfont{DejaVu Sans}
%%   \setmonofont{DejaVu Sans Mono}
%%
\begingroup%
\makeatletter%
\begin{pgfpicture}%
\pgfpathrectangle{\pgfpointorigin}{\pgfqpoint{5.000000in}{3.000000in}}%
\pgfusepath{use as bounding box, clip}%
\begin{pgfscope}%
\pgfsetbuttcap%
\pgfsetmiterjoin%
\definecolor{currentfill}{rgb}{1.000000,1.000000,1.000000}%
\pgfsetfillcolor{currentfill}%
\pgfsetlinewidth{0.000000pt}%
\definecolor{currentstroke}{rgb}{1.000000,1.000000,1.000000}%
\pgfsetstrokecolor{currentstroke}%
\pgfsetdash{}{0pt}%
\pgfpathmoveto{\pgfqpoint{0.000000in}{0.000000in}}%
\pgfpathlineto{\pgfqpoint{5.000000in}{0.000000in}}%
\pgfpathlineto{\pgfqpoint{5.000000in}{3.000000in}}%
\pgfpathlineto{\pgfqpoint{0.000000in}{3.000000in}}%
\pgfpathclose%
\pgfusepath{fill}%
\end{pgfscope}%
\begin{pgfscope}%
\pgfsetbuttcap%
\pgfsetmiterjoin%
\definecolor{currentfill}{rgb}{1.000000,1.000000,1.000000}%
\pgfsetfillcolor{currentfill}%
\pgfsetlinewidth{0.000000pt}%
\definecolor{currentstroke}{rgb}{0.000000,0.000000,0.000000}%
\pgfsetstrokecolor{currentstroke}%
\pgfsetstrokeopacity{0.000000}%
\pgfsetdash{}{0pt}%
\pgfpathmoveto{\pgfqpoint{0.626312in}{0.398769in}}%
\pgfpathlineto{\pgfqpoint{4.951389in}{0.398769in}}%
\pgfpathlineto{\pgfqpoint{4.951389in}{2.951389in}}%
\pgfpathlineto{\pgfqpoint{0.626312in}{2.951389in}}%
\pgfpathclose%
\pgfusepath{fill}%
\end{pgfscope}%
\begin{pgfscope}%
\pgfsetbuttcap%
\pgfsetroundjoin%
\definecolor{currentfill}{rgb}{0.000000,0.000000,0.000000}%
\pgfsetfillcolor{currentfill}%
\pgfsetlinewidth{0.803000pt}%
\definecolor{currentstroke}{rgb}{0.000000,0.000000,0.000000}%
\pgfsetstrokecolor{currentstroke}%
\pgfsetdash{}{0pt}%
\pgfsys@defobject{currentmarker}{\pgfqpoint{0.000000in}{-0.048611in}}{\pgfqpoint{0.000000in}{0.000000in}}{%
\pgfpathmoveto{\pgfqpoint{0.000000in}{0.000000in}}%
\pgfpathlineto{\pgfqpoint{0.000000in}{-0.048611in}}%
\pgfusepath{stroke,fill}%
}%
\begin{pgfscope}%
\pgfsys@transformshift{0.822906in}{0.398769in}%
\pgfsys@useobject{currentmarker}{}%
\end{pgfscope}%
\end{pgfscope}%
\begin{pgfscope}%
\pgftext[x=0.822906in,y=0.301547in,,top]{\rmfamily\fontsize{10.000000}{12.000000}\selectfont \(\displaystyle 10^{0}\)}%
\end{pgfscope}%
\begin{pgfscope}%
\pgfsetbuttcap%
\pgfsetroundjoin%
\definecolor{currentfill}{rgb}{0.000000,0.000000,0.000000}%
\pgfsetfillcolor{currentfill}%
\pgfsetlinewidth{0.803000pt}%
\definecolor{currentstroke}{rgb}{0.000000,0.000000,0.000000}%
\pgfsetstrokecolor{currentstroke}%
\pgfsetdash{}{0pt}%
\pgfsys@defobject{currentmarker}{\pgfqpoint{0.000000in}{-0.048611in}}{\pgfqpoint{0.000000in}{0.000000in}}{%
\pgfpathmoveto{\pgfqpoint{0.000000in}{0.000000in}}%
\pgfpathlineto{\pgfqpoint{0.000000in}{-0.048611in}}%
\pgfusepath{stroke,fill}%
}%
\begin{pgfscope}%
\pgfsys@transformshift{1.390795in}{0.398769in}%
\pgfsys@useobject{currentmarker}{}%
\end{pgfscope}%
\end{pgfscope}%
\begin{pgfscope}%
\pgftext[x=1.390795in,y=0.301547in,,top]{\rmfamily\fontsize{10.000000}{12.000000}\selectfont \(\displaystyle 10^{1}\)}%
\end{pgfscope}%
\begin{pgfscope}%
\pgfsetbuttcap%
\pgfsetroundjoin%
\definecolor{currentfill}{rgb}{0.000000,0.000000,0.000000}%
\pgfsetfillcolor{currentfill}%
\pgfsetlinewidth{0.803000pt}%
\definecolor{currentstroke}{rgb}{0.000000,0.000000,0.000000}%
\pgfsetstrokecolor{currentstroke}%
\pgfsetdash{}{0pt}%
\pgfsys@defobject{currentmarker}{\pgfqpoint{0.000000in}{-0.048611in}}{\pgfqpoint{0.000000in}{0.000000in}}{%
\pgfpathmoveto{\pgfqpoint{0.000000in}{0.000000in}}%
\pgfpathlineto{\pgfqpoint{0.000000in}{-0.048611in}}%
\pgfusepath{stroke,fill}%
}%
\begin{pgfscope}%
\pgfsys@transformshift{1.958684in}{0.398769in}%
\pgfsys@useobject{currentmarker}{}%
\end{pgfscope}%
\end{pgfscope}%
\begin{pgfscope}%
\pgftext[x=1.958684in,y=0.301547in,,top]{\rmfamily\fontsize{10.000000}{12.000000}\selectfont \(\displaystyle 10^{2}\)}%
\end{pgfscope}%
\begin{pgfscope}%
\pgfsetbuttcap%
\pgfsetroundjoin%
\definecolor{currentfill}{rgb}{0.000000,0.000000,0.000000}%
\pgfsetfillcolor{currentfill}%
\pgfsetlinewidth{0.803000pt}%
\definecolor{currentstroke}{rgb}{0.000000,0.000000,0.000000}%
\pgfsetstrokecolor{currentstroke}%
\pgfsetdash{}{0pt}%
\pgfsys@defobject{currentmarker}{\pgfqpoint{0.000000in}{-0.048611in}}{\pgfqpoint{0.000000in}{0.000000in}}{%
\pgfpathmoveto{\pgfqpoint{0.000000in}{0.000000in}}%
\pgfpathlineto{\pgfqpoint{0.000000in}{-0.048611in}}%
\pgfusepath{stroke,fill}%
}%
\begin{pgfscope}%
\pgfsys@transformshift{2.526574in}{0.398769in}%
\pgfsys@useobject{currentmarker}{}%
\end{pgfscope}%
\end{pgfscope}%
\begin{pgfscope}%
\pgftext[x=2.526574in,y=0.301547in,,top]{\rmfamily\fontsize{10.000000}{12.000000}\selectfont \(\displaystyle 10^{3}\)}%
\end{pgfscope}%
\begin{pgfscope}%
\pgfsetbuttcap%
\pgfsetroundjoin%
\definecolor{currentfill}{rgb}{0.000000,0.000000,0.000000}%
\pgfsetfillcolor{currentfill}%
\pgfsetlinewidth{0.803000pt}%
\definecolor{currentstroke}{rgb}{0.000000,0.000000,0.000000}%
\pgfsetstrokecolor{currentstroke}%
\pgfsetdash{}{0pt}%
\pgfsys@defobject{currentmarker}{\pgfqpoint{0.000000in}{-0.048611in}}{\pgfqpoint{0.000000in}{0.000000in}}{%
\pgfpathmoveto{\pgfqpoint{0.000000in}{0.000000in}}%
\pgfpathlineto{\pgfqpoint{0.000000in}{-0.048611in}}%
\pgfusepath{stroke,fill}%
}%
\begin{pgfscope}%
\pgfsys@transformshift{3.094463in}{0.398769in}%
\pgfsys@useobject{currentmarker}{}%
\end{pgfscope}%
\end{pgfscope}%
\begin{pgfscope}%
\pgftext[x=3.094463in,y=0.301547in,,top]{\rmfamily\fontsize{10.000000}{12.000000}\selectfont \(\displaystyle 10^{4}\)}%
\end{pgfscope}%
\begin{pgfscope}%
\pgfsetbuttcap%
\pgfsetroundjoin%
\definecolor{currentfill}{rgb}{0.000000,0.000000,0.000000}%
\pgfsetfillcolor{currentfill}%
\pgfsetlinewidth{0.803000pt}%
\definecolor{currentstroke}{rgb}{0.000000,0.000000,0.000000}%
\pgfsetstrokecolor{currentstroke}%
\pgfsetdash{}{0pt}%
\pgfsys@defobject{currentmarker}{\pgfqpoint{0.000000in}{-0.048611in}}{\pgfqpoint{0.000000in}{0.000000in}}{%
\pgfpathmoveto{\pgfqpoint{0.000000in}{0.000000in}}%
\pgfpathlineto{\pgfqpoint{0.000000in}{-0.048611in}}%
\pgfusepath{stroke,fill}%
}%
\begin{pgfscope}%
\pgfsys@transformshift{3.662352in}{0.398769in}%
\pgfsys@useobject{currentmarker}{}%
\end{pgfscope}%
\end{pgfscope}%
\begin{pgfscope}%
\pgftext[x=3.662352in,y=0.301547in,,top]{\rmfamily\fontsize{10.000000}{12.000000}\selectfont \(\displaystyle 10^{5}\)}%
\end{pgfscope}%
\begin{pgfscope}%
\pgfsetbuttcap%
\pgfsetroundjoin%
\definecolor{currentfill}{rgb}{0.000000,0.000000,0.000000}%
\pgfsetfillcolor{currentfill}%
\pgfsetlinewidth{0.803000pt}%
\definecolor{currentstroke}{rgb}{0.000000,0.000000,0.000000}%
\pgfsetstrokecolor{currentstroke}%
\pgfsetdash{}{0pt}%
\pgfsys@defobject{currentmarker}{\pgfqpoint{0.000000in}{-0.048611in}}{\pgfqpoint{0.000000in}{0.000000in}}{%
\pgfpathmoveto{\pgfqpoint{0.000000in}{0.000000in}}%
\pgfpathlineto{\pgfqpoint{0.000000in}{-0.048611in}}%
\pgfusepath{stroke,fill}%
}%
\begin{pgfscope}%
\pgfsys@transformshift{4.230241in}{0.398769in}%
\pgfsys@useobject{currentmarker}{}%
\end{pgfscope}%
\end{pgfscope}%
\begin{pgfscope}%
\pgftext[x=4.230241in,y=0.301547in,,top]{\rmfamily\fontsize{10.000000}{12.000000}\selectfont \(\displaystyle 10^{6}\)}%
\end{pgfscope}%
\begin{pgfscope}%
\pgfsetbuttcap%
\pgfsetroundjoin%
\definecolor{currentfill}{rgb}{0.000000,0.000000,0.000000}%
\pgfsetfillcolor{currentfill}%
\pgfsetlinewidth{0.803000pt}%
\definecolor{currentstroke}{rgb}{0.000000,0.000000,0.000000}%
\pgfsetstrokecolor{currentstroke}%
\pgfsetdash{}{0pt}%
\pgfsys@defobject{currentmarker}{\pgfqpoint{0.000000in}{-0.048611in}}{\pgfqpoint{0.000000in}{0.000000in}}{%
\pgfpathmoveto{\pgfqpoint{0.000000in}{0.000000in}}%
\pgfpathlineto{\pgfqpoint{0.000000in}{-0.048611in}}%
\pgfusepath{stroke,fill}%
}%
\begin{pgfscope}%
\pgfsys@transformshift{4.798130in}{0.398769in}%
\pgfsys@useobject{currentmarker}{}%
\end{pgfscope}%
\end{pgfscope}%
\begin{pgfscope}%
\pgftext[x=4.798130in,y=0.301547in,,top]{\rmfamily\fontsize{10.000000}{12.000000}\selectfont \(\displaystyle 10^{7}\)}%
\end{pgfscope}%
\begin{pgfscope}%
\pgfsetbuttcap%
\pgfsetroundjoin%
\definecolor{currentfill}{rgb}{0.000000,0.000000,0.000000}%
\pgfsetfillcolor{currentfill}%
\pgfsetlinewidth{0.602250pt}%
\definecolor{currentstroke}{rgb}{0.000000,0.000000,0.000000}%
\pgfsetstrokecolor{currentstroke}%
\pgfsetdash{}{0pt}%
\pgfsys@defobject{currentmarker}{\pgfqpoint{0.000000in}{-0.027778in}}{\pgfqpoint{0.000000in}{0.000000in}}{%
\pgfpathmoveto{\pgfqpoint{0.000000in}{0.000000in}}%
\pgfpathlineto{\pgfqpoint{0.000000in}{-0.027778in}}%
\pgfusepath{stroke,fill}%
}%
\begin{pgfscope}%
\pgfsys@transformshift{0.651955in}{0.398769in}%
\pgfsys@useobject{currentmarker}{}%
\end{pgfscope}%
\end{pgfscope}%
\begin{pgfscope}%
\pgfsetbuttcap%
\pgfsetroundjoin%
\definecolor{currentfill}{rgb}{0.000000,0.000000,0.000000}%
\pgfsetfillcolor{currentfill}%
\pgfsetlinewidth{0.602250pt}%
\definecolor{currentstroke}{rgb}{0.000000,0.000000,0.000000}%
\pgfsetstrokecolor{currentstroke}%
\pgfsetdash{}{0pt}%
\pgfsys@defobject{currentmarker}{\pgfqpoint{0.000000in}{-0.027778in}}{\pgfqpoint{0.000000in}{0.000000in}}{%
\pgfpathmoveto{\pgfqpoint{0.000000in}{0.000000in}}%
\pgfpathlineto{\pgfqpoint{0.000000in}{-0.027778in}}%
\pgfusepath{stroke,fill}%
}%
\begin{pgfscope}%
\pgfsys@transformshift{0.696921in}{0.398769in}%
\pgfsys@useobject{currentmarker}{}%
\end{pgfscope}%
\end{pgfscope}%
\begin{pgfscope}%
\pgfsetbuttcap%
\pgfsetroundjoin%
\definecolor{currentfill}{rgb}{0.000000,0.000000,0.000000}%
\pgfsetfillcolor{currentfill}%
\pgfsetlinewidth{0.602250pt}%
\definecolor{currentstroke}{rgb}{0.000000,0.000000,0.000000}%
\pgfsetstrokecolor{currentstroke}%
\pgfsetdash{}{0pt}%
\pgfsys@defobject{currentmarker}{\pgfqpoint{0.000000in}{-0.027778in}}{\pgfqpoint{0.000000in}{0.000000in}}{%
\pgfpathmoveto{\pgfqpoint{0.000000in}{0.000000in}}%
\pgfpathlineto{\pgfqpoint{0.000000in}{-0.027778in}}%
\pgfusepath{stroke,fill}%
}%
\begin{pgfscope}%
\pgfsys@transformshift{0.734939in}{0.398769in}%
\pgfsys@useobject{currentmarker}{}%
\end{pgfscope}%
\end{pgfscope}%
\begin{pgfscope}%
\pgfsetbuttcap%
\pgfsetroundjoin%
\definecolor{currentfill}{rgb}{0.000000,0.000000,0.000000}%
\pgfsetfillcolor{currentfill}%
\pgfsetlinewidth{0.602250pt}%
\definecolor{currentstroke}{rgb}{0.000000,0.000000,0.000000}%
\pgfsetstrokecolor{currentstroke}%
\pgfsetdash{}{0pt}%
\pgfsys@defobject{currentmarker}{\pgfqpoint{0.000000in}{-0.027778in}}{\pgfqpoint{0.000000in}{0.000000in}}{%
\pgfpathmoveto{\pgfqpoint{0.000000in}{0.000000in}}%
\pgfpathlineto{\pgfqpoint{0.000000in}{-0.027778in}}%
\pgfusepath{stroke,fill}%
}%
\begin{pgfscope}%
\pgfsys@transformshift{0.767872in}{0.398769in}%
\pgfsys@useobject{currentmarker}{}%
\end{pgfscope}%
\end{pgfscope}%
\begin{pgfscope}%
\pgfsetbuttcap%
\pgfsetroundjoin%
\definecolor{currentfill}{rgb}{0.000000,0.000000,0.000000}%
\pgfsetfillcolor{currentfill}%
\pgfsetlinewidth{0.602250pt}%
\definecolor{currentstroke}{rgb}{0.000000,0.000000,0.000000}%
\pgfsetstrokecolor{currentstroke}%
\pgfsetdash{}{0pt}%
\pgfsys@defobject{currentmarker}{\pgfqpoint{0.000000in}{-0.027778in}}{\pgfqpoint{0.000000in}{0.000000in}}{%
\pgfpathmoveto{\pgfqpoint{0.000000in}{0.000000in}}%
\pgfpathlineto{\pgfqpoint{0.000000in}{-0.027778in}}%
\pgfusepath{stroke,fill}%
}%
\begin{pgfscope}%
\pgfsys@transformshift{0.796921in}{0.398769in}%
\pgfsys@useobject{currentmarker}{}%
\end{pgfscope}%
\end{pgfscope}%
\begin{pgfscope}%
\pgfsetbuttcap%
\pgfsetroundjoin%
\definecolor{currentfill}{rgb}{0.000000,0.000000,0.000000}%
\pgfsetfillcolor{currentfill}%
\pgfsetlinewidth{0.602250pt}%
\definecolor{currentstroke}{rgb}{0.000000,0.000000,0.000000}%
\pgfsetstrokecolor{currentstroke}%
\pgfsetdash{}{0pt}%
\pgfsys@defobject{currentmarker}{\pgfqpoint{0.000000in}{-0.027778in}}{\pgfqpoint{0.000000in}{0.000000in}}{%
\pgfpathmoveto{\pgfqpoint{0.000000in}{0.000000in}}%
\pgfpathlineto{\pgfqpoint{0.000000in}{-0.027778in}}%
\pgfusepath{stroke,fill}%
}%
\begin{pgfscope}%
\pgfsys@transformshift{0.993858in}{0.398769in}%
\pgfsys@useobject{currentmarker}{}%
\end{pgfscope}%
\end{pgfscope}%
\begin{pgfscope}%
\pgfsetbuttcap%
\pgfsetroundjoin%
\definecolor{currentfill}{rgb}{0.000000,0.000000,0.000000}%
\pgfsetfillcolor{currentfill}%
\pgfsetlinewidth{0.602250pt}%
\definecolor{currentstroke}{rgb}{0.000000,0.000000,0.000000}%
\pgfsetstrokecolor{currentstroke}%
\pgfsetdash{}{0pt}%
\pgfsys@defobject{currentmarker}{\pgfqpoint{0.000000in}{-0.027778in}}{\pgfqpoint{0.000000in}{0.000000in}}{%
\pgfpathmoveto{\pgfqpoint{0.000000in}{0.000000in}}%
\pgfpathlineto{\pgfqpoint{0.000000in}{-0.027778in}}%
\pgfusepath{stroke,fill}%
}%
\begin{pgfscope}%
\pgfsys@transformshift{1.093858in}{0.398769in}%
\pgfsys@useobject{currentmarker}{}%
\end{pgfscope}%
\end{pgfscope}%
\begin{pgfscope}%
\pgfsetbuttcap%
\pgfsetroundjoin%
\definecolor{currentfill}{rgb}{0.000000,0.000000,0.000000}%
\pgfsetfillcolor{currentfill}%
\pgfsetlinewidth{0.602250pt}%
\definecolor{currentstroke}{rgb}{0.000000,0.000000,0.000000}%
\pgfsetstrokecolor{currentstroke}%
\pgfsetdash{}{0pt}%
\pgfsys@defobject{currentmarker}{\pgfqpoint{0.000000in}{-0.027778in}}{\pgfqpoint{0.000000in}{0.000000in}}{%
\pgfpathmoveto{\pgfqpoint{0.000000in}{0.000000in}}%
\pgfpathlineto{\pgfqpoint{0.000000in}{-0.027778in}}%
\pgfusepath{stroke,fill}%
}%
\begin{pgfscope}%
\pgfsys@transformshift{1.164810in}{0.398769in}%
\pgfsys@useobject{currentmarker}{}%
\end{pgfscope}%
\end{pgfscope}%
\begin{pgfscope}%
\pgfsetbuttcap%
\pgfsetroundjoin%
\definecolor{currentfill}{rgb}{0.000000,0.000000,0.000000}%
\pgfsetfillcolor{currentfill}%
\pgfsetlinewidth{0.602250pt}%
\definecolor{currentstroke}{rgb}{0.000000,0.000000,0.000000}%
\pgfsetstrokecolor{currentstroke}%
\pgfsetdash{}{0pt}%
\pgfsys@defobject{currentmarker}{\pgfqpoint{0.000000in}{-0.027778in}}{\pgfqpoint{0.000000in}{0.000000in}}{%
\pgfpathmoveto{\pgfqpoint{0.000000in}{0.000000in}}%
\pgfpathlineto{\pgfqpoint{0.000000in}{-0.027778in}}%
\pgfusepath{stroke,fill}%
}%
\begin{pgfscope}%
\pgfsys@transformshift{1.219844in}{0.398769in}%
\pgfsys@useobject{currentmarker}{}%
\end{pgfscope}%
\end{pgfscope}%
\begin{pgfscope}%
\pgfsetbuttcap%
\pgfsetroundjoin%
\definecolor{currentfill}{rgb}{0.000000,0.000000,0.000000}%
\pgfsetfillcolor{currentfill}%
\pgfsetlinewidth{0.602250pt}%
\definecolor{currentstroke}{rgb}{0.000000,0.000000,0.000000}%
\pgfsetstrokecolor{currentstroke}%
\pgfsetdash{}{0pt}%
\pgfsys@defobject{currentmarker}{\pgfqpoint{0.000000in}{-0.027778in}}{\pgfqpoint{0.000000in}{0.000000in}}{%
\pgfpathmoveto{\pgfqpoint{0.000000in}{0.000000in}}%
\pgfpathlineto{\pgfqpoint{0.000000in}{-0.027778in}}%
\pgfusepath{stroke,fill}%
}%
\begin{pgfscope}%
\pgfsys@transformshift{1.264810in}{0.398769in}%
\pgfsys@useobject{currentmarker}{}%
\end{pgfscope}%
\end{pgfscope}%
\begin{pgfscope}%
\pgfsetbuttcap%
\pgfsetroundjoin%
\definecolor{currentfill}{rgb}{0.000000,0.000000,0.000000}%
\pgfsetfillcolor{currentfill}%
\pgfsetlinewidth{0.602250pt}%
\definecolor{currentstroke}{rgb}{0.000000,0.000000,0.000000}%
\pgfsetstrokecolor{currentstroke}%
\pgfsetdash{}{0pt}%
\pgfsys@defobject{currentmarker}{\pgfqpoint{0.000000in}{-0.027778in}}{\pgfqpoint{0.000000in}{0.000000in}}{%
\pgfpathmoveto{\pgfqpoint{0.000000in}{0.000000in}}%
\pgfpathlineto{\pgfqpoint{0.000000in}{-0.027778in}}%
\pgfusepath{stroke,fill}%
}%
\begin{pgfscope}%
\pgfsys@transformshift{1.302828in}{0.398769in}%
\pgfsys@useobject{currentmarker}{}%
\end{pgfscope}%
\end{pgfscope}%
\begin{pgfscope}%
\pgfsetbuttcap%
\pgfsetroundjoin%
\definecolor{currentfill}{rgb}{0.000000,0.000000,0.000000}%
\pgfsetfillcolor{currentfill}%
\pgfsetlinewidth{0.602250pt}%
\definecolor{currentstroke}{rgb}{0.000000,0.000000,0.000000}%
\pgfsetstrokecolor{currentstroke}%
\pgfsetdash{}{0pt}%
\pgfsys@defobject{currentmarker}{\pgfqpoint{0.000000in}{-0.027778in}}{\pgfqpoint{0.000000in}{0.000000in}}{%
\pgfpathmoveto{\pgfqpoint{0.000000in}{0.000000in}}%
\pgfpathlineto{\pgfqpoint{0.000000in}{-0.027778in}}%
\pgfusepath{stroke,fill}%
}%
\begin{pgfscope}%
\pgfsys@transformshift{1.335761in}{0.398769in}%
\pgfsys@useobject{currentmarker}{}%
\end{pgfscope}%
\end{pgfscope}%
\begin{pgfscope}%
\pgfsetbuttcap%
\pgfsetroundjoin%
\definecolor{currentfill}{rgb}{0.000000,0.000000,0.000000}%
\pgfsetfillcolor{currentfill}%
\pgfsetlinewidth{0.602250pt}%
\definecolor{currentstroke}{rgb}{0.000000,0.000000,0.000000}%
\pgfsetstrokecolor{currentstroke}%
\pgfsetdash{}{0pt}%
\pgfsys@defobject{currentmarker}{\pgfqpoint{0.000000in}{-0.027778in}}{\pgfqpoint{0.000000in}{0.000000in}}{%
\pgfpathmoveto{\pgfqpoint{0.000000in}{0.000000in}}%
\pgfpathlineto{\pgfqpoint{0.000000in}{-0.027778in}}%
\pgfusepath{stroke,fill}%
}%
\begin{pgfscope}%
\pgfsys@transformshift{1.364810in}{0.398769in}%
\pgfsys@useobject{currentmarker}{}%
\end{pgfscope}%
\end{pgfscope}%
\begin{pgfscope}%
\pgfsetbuttcap%
\pgfsetroundjoin%
\definecolor{currentfill}{rgb}{0.000000,0.000000,0.000000}%
\pgfsetfillcolor{currentfill}%
\pgfsetlinewidth{0.602250pt}%
\definecolor{currentstroke}{rgb}{0.000000,0.000000,0.000000}%
\pgfsetstrokecolor{currentstroke}%
\pgfsetdash{}{0pt}%
\pgfsys@defobject{currentmarker}{\pgfqpoint{0.000000in}{-0.027778in}}{\pgfqpoint{0.000000in}{0.000000in}}{%
\pgfpathmoveto{\pgfqpoint{0.000000in}{0.000000in}}%
\pgfpathlineto{\pgfqpoint{0.000000in}{-0.027778in}}%
\pgfusepath{stroke,fill}%
}%
\begin{pgfscope}%
\pgfsys@transformshift{1.561747in}{0.398769in}%
\pgfsys@useobject{currentmarker}{}%
\end{pgfscope}%
\end{pgfscope}%
\begin{pgfscope}%
\pgfsetbuttcap%
\pgfsetroundjoin%
\definecolor{currentfill}{rgb}{0.000000,0.000000,0.000000}%
\pgfsetfillcolor{currentfill}%
\pgfsetlinewidth{0.602250pt}%
\definecolor{currentstroke}{rgb}{0.000000,0.000000,0.000000}%
\pgfsetstrokecolor{currentstroke}%
\pgfsetdash{}{0pt}%
\pgfsys@defobject{currentmarker}{\pgfqpoint{0.000000in}{-0.027778in}}{\pgfqpoint{0.000000in}{0.000000in}}{%
\pgfpathmoveto{\pgfqpoint{0.000000in}{0.000000in}}%
\pgfpathlineto{\pgfqpoint{0.000000in}{-0.027778in}}%
\pgfusepath{stroke,fill}%
}%
\begin{pgfscope}%
\pgfsys@transformshift{1.661747in}{0.398769in}%
\pgfsys@useobject{currentmarker}{}%
\end{pgfscope}%
\end{pgfscope}%
\begin{pgfscope}%
\pgfsetbuttcap%
\pgfsetroundjoin%
\definecolor{currentfill}{rgb}{0.000000,0.000000,0.000000}%
\pgfsetfillcolor{currentfill}%
\pgfsetlinewidth{0.602250pt}%
\definecolor{currentstroke}{rgb}{0.000000,0.000000,0.000000}%
\pgfsetstrokecolor{currentstroke}%
\pgfsetdash{}{0pt}%
\pgfsys@defobject{currentmarker}{\pgfqpoint{0.000000in}{-0.027778in}}{\pgfqpoint{0.000000in}{0.000000in}}{%
\pgfpathmoveto{\pgfqpoint{0.000000in}{0.000000in}}%
\pgfpathlineto{\pgfqpoint{0.000000in}{-0.027778in}}%
\pgfusepath{stroke,fill}%
}%
\begin{pgfscope}%
\pgfsys@transformshift{1.732699in}{0.398769in}%
\pgfsys@useobject{currentmarker}{}%
\end{pgfscope}%
\end{pgfscope}%
\begin{pgfscope}%
\pgfsetbuttcap%
\pgfsetroundjoin%
\definecolor{currentfill}{rgb}{0.000000,0.000000,0.000000}%
\pgfsetfillcolor{currentfill}%
\pgfsetlinewidth{0.602250pt}%
\definecolor{currentstroke}{rgb}{0.000000,0.000000,0.000000}%
\pgfsetstrokecolor{currentstroke}%
\pgfsetdash{}{0pt}%
\pgfsys@defobject{currentmarker}{\pgfqpoint{0.000000in}{-0.027778in}}{\pgfqpoint{0.000000in}{0.000000in}}{%
\pgfpathmoveto{\pgfqpoint{0.000000in}{0.000000in}}%
\pgfpathlineto{\pgfqpoint{0.000000in}{-0.027778in}}%
\pgfusepath{stroke,fill}%
}%
\begin{pgfscope}%
\pgfsys@transformshift{1.787733in}{0.398769in}%
\pgfsys@useobject{currentmarker}{}%
\end{pgfscope}%
\end{pgfscope}%
\begin{pgfscope}%
\pgfsetbuttcap%
\pgfsetroundjoin%
\definecolor{currentfill}{rgb}{0.000000,0.000000,0.000000}%
\pgfsetfillcolor{currentfill}%
\pgfsetlinewidth{0.602250pt}%
\definecolor{currentstroke}{rgb}{0.000000,0.000000,0.000000}%
\pgfsetstrokecolor{currentstroke}%
\pgfsetdash{}{0pt}%
\pgfsys@defobject{currentmarker}{\pgfqpoint{0.000000in}{-0.027778in}}{\pgfqpoint{0.000000in}{0.000000in}}{%
\pgfpathmoveto{\pgfqpoint{0.000000in}{0.000000in}}%
\pgfpathlineto{\pgfqpoint{0.000000in}{-0.027778in}}%
\pgfusepath{stroke,fill}%
}%
\begin{pgfscope}%
\pgfsys@transformshift{1.832699in}{0.398769in}%
\pgfsys@useobject{currentmarker}{}%
\end{pgfscope}%
\end{pgfscope}%
\begin{pgfscope}%
\pgfsetbuttcap%
\pgfsetroundjoin%
\definecolor{currentfill}{rgb}{0.000000,0.000000,0.000000}%
\pgfsetfillcolor{currentfill}%
\pgfsetlinewidth{0.602250pt}%
\definecolor{currentstroke}{rgb}{0.000000,0.000000,0.000000}%
\pgfsetstrokecolor{currentstroke}%
\pgfsetdash{}{0pt}%
\pgfsys@defobject{currentmarker}{\pgfqpoint{0.000000in}{-0.027778in}}{\pgfqpoint{0.000000in}{0.000000in}}{%
\pgfpathmoveto{\pgfqpoint{0.000000in}{0.000000in}}%
\pgfpathlineto{\pgfqpoint{0.000000in}{-0.027778in}}%
\pgfusepath{stroke,fill}%
}%
\begin{pgfscope}%
\pgfsys@transformshift{1.870717in}{0.398769in}%
\pgfsys@useobject{currentmarker}{}%
\end{pgfscope}%
\end{pgfscope}%
\begin{pgfscope}%
\pgfsetbuttcap%
\pgfsetroundjoin%
\definecolor{currentfill}{rgb}{0.000000,0.000000,0.000000}%
\pgfsetfillcolor{currentfill}%
\pgfsetlinewidth{0.602250pt}%
\definecolor{currentstroke}{rgb}{0.000000,0.000000,0.000000}%
\pgfsetstrokecolor{currentstroke}%
\pgfsetdash{}{0pt}%
\pgfsys@defobject{currentmarker}{\pgfqpoint{0.000000in}{-0.027778in}}{\pgfqpoint{0.000000in}{0.000000in}}{%
\pgfpathmoveto{\pgfqpoint{0.000000in}{0.000000in}}%
\pgfpathlineto{\pgfqpoint{0.000000in}{-0.027778in}}%
\pgfusepath{stroke,fill}%
}%
\begin{pgfscope}%
\pgfsys@transformshift{1.903650in}{0.398769in}%
\pgfsys@useobject{currentmarker}{}%
\end{pgfscope}%
\end{pgfscope}%
\begin{pgfscope}%
\pgfsetbuttcap%
\pgfsetroundjoin%
\definecolor{currentfill}{rgb}{0.000000,0.000000,0.000000}%
\pgfsetfillcolor{currentfill}%
\pgfsetlinewidth{0.602250pt}%
\definecolor{currentstroke}{rgb}{0.000000,0.000000,0.000000}%
\pgfsetstrokecolor{currentstroke}%
\pgfsetdash{}{0pt}%
\pgfsys@defobject{currentmarker}{\pgfqpoint{0.000000in}{-0.027778in}}{\pgfqpoint{0.000000in}{0.000000in}}{%
\pgfpathmoveto{\pgfqpoint{0.000000in}{0.000000in}}%
\pgfpathlineto{\pgfqpoint{0.000000in}{-0.027778in}}%
\pgfusepath{stroke,fill}%
}%
\begin{pgfscope}%
\pgfsys@transformshift{1.932699in}{0.398769in}%
\pgfsys@useobject{currentmarker}{}%
\end{pgfscope}%
\end{pgfscope}%
\begin{pgfscope}%
\pgfsetbuttcap%
\pgfsetroundjoin%
\definecolor{currentfill}{rgb}{0.000000,0.000000,0.000000}%
\pgfsetfillcolor{currentfill}%
\pgfsetlinewidth{0.602250pt}%
\definecolor{currentstroke}{rgb}{0.000000,0.000000,0.000000}%
\pgfsetstrokecolor{currentstroke}%
\pgfsetdash{}{0pt}%
\pgfsys@defobject{currentmarker}{\pgfqpoint{0.000000in}{-0.027778in}}{\pgfqpoint{0.000000in}{0.000000in}}{%
\pgfpathmoveto{\pgfqpoint{0.000000in}{0.000000in}}%
\pgfpathlineto{\pgfqpoint{0.000000in}{-0.027778in}}%
\pgfusepath{stroke,fill}%
}%
\begin{pgfscope}%
\pgfsys@transformshift{2.129636in}{0.398769in}%
\pgfsys@useobject{currentmarker}{}%
\end{pgfscope}%
\end{pgfscope}%
\begin{pgfscope}%
\pgfsetbuttcap%
\pgfsetroundjoin%
\definecolor{currentfill}{rgb}{0.000000,0.000000,0.000000}%
\pgfsetfillcolor{currentfill}%
\pgfsetlinewidth{0.602250pt}%
\definecolor{currentstroke}{rgb}{0.000000,0.000000,0.000000}%
\pgfsetstrokecolor{currentstroke}%
\pgfsetdash{}{0pt}%
\pgfsys@defobject{currentmarker}{\pgfqpoint{0.000000in}{-0.027778in}}{\pgfqpoint{0.000000in}{0.000000in}}{%
\pgfpathmoveto{\pgfqpoint{0.000000in}{0.000000in}}%
\pgfpathlineto{\pgfqpoint{0.000000in}{-0.027778in}}%
\pgfusepath{stroke,fill}%
}%
\begin{pgfscope}%
\pgfsys@transformshift{2.229636in}{0.398769in}%
\pgfsys@useobject{currentmarker}{}%
\end{pgfscope}%
\end{pgfscope}%
\begin{pgfscope}%
\pgfsetbuttcap%
\pgfsetroundjoin%
\definecolor{currentfill}{rgb}{0.000000,0.000000,0.000000}%
\pgfsetfillcolor{currentfill}%
\pgfsetlinewidth{0.602250pt}%
\definecolor{currentstroke}{rgb}{0.000000,0.000000,0.000000}%
\pgfsetstrokecolor{currentstroke}%
\pgfsetdash{}{0pt}%
\pgfsys@defobject{currentmarker}{\pgfqpoint{0.000000in}{-0.027778in}}{\pgfqpoint{0.000000in}{0.000000in}}{%
\pgfpathmoveto{\pgfqpoint{0.000000in}{0.000000in}}%
\pgfpathlineto{\pgfqpoint{0.000000in}{-0.027778in}}%
\pgfusepath{stroke,fill}%
}%
\begin{pgfscope}%
\pgfsys@transformshift{2.300588in}{0.398769in}%
\pgfsys@useobject{currentmarker}{}%
\end{pgfscope}%
\end{pgfscope}%
\begin{pgfscope}%
\pgfsetbuttcap%
\pgfsetroundjoin%
\definecolor{currentfill}{rgb}{0.000000,0.000000,0.000000}%
\pgfsetfillcolor{currentfill}%
\pgfsetlinewidth{0.602250pt}%
\definecolor{currentstroke}{rgb}{0.000000,0.000000,0.000000}%
\pgfsetstrokecolor{currentstroke}%
\pgfsetdash{}{0pt}%
\pgfsys@defobject{currentmarker}{\pgfqpoint{0.000000in}{-0.027778in}}{\pgfqpoint{0.000000in}{0.000000in}}{%
\pgfpathmoveto{\pgfqpoint{0.000000in}{0.000000in}}%
\pgfpathlineto{\pgfqpoint{0.000000in}{-0.027778in}}%
\pgfusepath{stroke,fill}%
}%
\begin{pgfscope}%
\pgfsys@transformshift{2.355622in}{0.398769in}%
\pgfsys@useobject{currentmarker}{}%
\end{pgfscope}%
\end{pgfscope}%
\begin{pgfscope}%
\pgfsetbuttcap%
\pgfsetroundjoin%
\definecolor{currentfill}{rgb}{0.000000,0.000000,0.000000}%
\pgfsetfillcolor{currentfill}%
\pgfsetlinewidth{0.602250pt}%
\definecolor{currentstroke}{rgb}{0.000000,0.000000,0.000000}%
\pgfsetstrokecolor{currentstroke}%
\pgfsetdash{}{0pt}%
\pgfsys@defobject{currentmarker}{\pgfqpoint{0.000000in}{-0.027778in}}{\pgfqpoint{0.000000in}{0.000000in}}{%
\pgfpathmoveto{\pgfqpoint{0.000000in}{0.000000in}}%
\pgfpathlineto{\pgfqpoint{0.000000in}{-0.027778in}}%
\pgfusepath{stroke,fill}%
}%
\begin{pgfscope}%
\pgfsys@transformshift{2.400588in}{0.398769in}%
\pgfsys@useobject{currentmarker}{}%
\end{pgfscope}%
\end{pgfscope}%
\begin{pgfscope}%
\pgfsetbuttcap%
\pgfsetroundjoin%
\definecolor{currentfill}{rgb}{0.000000,0.000000,0.000000}%
\pgfsetfillcolor{currentfill}%
\pgfsetlinewidth{0.602250pt}%
\definecolor{currentstroke}{rgb}{0.000000,0.000000,0.000000}%
\pgfsetstrokecolor{currentstroke}%
\pgfsetdash{}{0pt}%
\pgfsys@defobject{currentmarker}{\pgfqpoint{0.000000in}{-0.027778in}}{\pgfqpoint{0.000000in}{0.000000in}}{%
\pgfpathmoveto{\pgfqpoint{0.000000in}{0.000000in}}%
\pgfpathlineto{\pgfqpoint{0.000000in}{-0.027778in}}%
\pgfusepath{stroke,fill}%
}%
\begin{pgfscope}%
\pgfsys@transformshift{2.438606in}{0.398769in}%
\pgfsys@useobject{currentmarker}{}%
\end{pgfscope}%
\end{pgfscope}%
\begin{pgfscope}%
\pgfsetbuttcap%
\pgfsetroundjoin%
\definecolor{currentfill}{rgb}{0.000000,0.000000,0.000000}%
\pgfsetfillcolor{currentfill}%
\pgfsetlinewidth{0.602250pt}%
\definecolor{currentstroke}{rgb}{0.000000,0.000000,0.000000}%
\pgfsetstrokecolor{currentstroke}%
\pgfsetdash{}{0pt}%
\pgfsys@defobject{currentmarker}{\pgfqpoint{0.000000in}{-0.027778in}}{\pgfqpoint{0.000000in}{0.000000in}}{%
\pgfpathmoveto{\pgfqpoint{0.000000in}{0.000000in}}%
\pgfpathlineto{\pgfqpoint{0.000000in}{-0.027778in}}%
\pgfusepath{stroke,fill}%
}%
\begin{pgfscope}%
\pgfsys@transformshift{2.471539in}{0.398769in}%
\pgfsys@useobject{currentmarker}{}%
\end{pgfscope}%
\end{pgfscope}%
\begin{pgfscope}%
\pgfsetbuttcap%
\pgfsetroundjoin%
\definecolor{currentfill}{rgb}{0.000000,0.000000,0.000000}%
\pgfsetfillcolor{currentfill}%
\pgfsetlinewidth{0.602250pt}%
\definecolor{currentstroke}{rgb}{0.000000,0.000000,0.000000}%
\pgfsetstrokecolor{currentstroke}%
\pgfsetdash{}{0pt}%
\pgfsys@defobject{currentmarker}{\pgfqpoint{0.000000in}{-0.027778in}}{\pgfqpoint{0.000000in}{0.000000in}}{%
\pgfpathmoveto{\pgfqpoint{0.000000in}{0.000000in}}%
\pgfpathlineto{\pgfqpoint{0.000000in}{-0.027778in}}%
\pgfusepath{stroke,fill}%
}%
\begin{pgfscope}%
\pgfsys@transformshift{2.500588in}{0.398769in}%
\pgfsys@useobject{currentmarker}{}%
\end{pgfscope}%
\end{pgfscope}%
\begin{pgfscope}%
\pgfsetbuttcap%
\pgfsetroundjoin%
\definecolor{currentfill}{rgb}{0.000000,0.000000,0.000000}%
\pgfsetfillcolor{currentfill}%
\pgfsetlinewidth{0.602250pt}%
\definecolor{currentstroke}{rgb}{0.000000,0.000000,0.000000}%
\pgfsetstrokecolor{currentstroke}%
\pgfsetdash{}{0pt}%
\pgfsys@defobject{currentmarker}{\pgfqpoint{0.000000in}{-0.027778in}}{\pgfqpoint{0.000000in}{0.000000in}}{%
\pgfpathmoveto{\pgfqpoint{0.000000in}{0.000000in}}%
\pgfpathlineto{\pgfqpoint{0.000000in}{-0.027778in}}%
\pgfusepath{stroke,fill}%
}%
\begin{pgfscope}%
\pgfsys@transformshift{2.697525in}{0.398769in}%
\pgfsys@useobject{currentmarker}{}%
\end{pgfscope}%
\end{pgfscope}%
\begin{pgfscope}%
\pgfsetbuttcap%
\pgfsetroundjoin%
\definecolor{currentfill}{rgb}{0.000000,0.000000,0.000000}%
\pgfsetfillcolor{currentfill}%
\pgfsetlinewidth{0.602250pt}%
\definecolor{currentstroke}{rgb}{0.000000,0.000000,0.000000}%
\pgfsetstrokecolor{currentstroke}%
\pgfsetdash{}{0pt}%
\pgfsys@defobject{currentmarker}{\pgfqpoint{0.000000in}{-0.027778in}}{\pgfqpoint{0.000000in}{0.000000in}}{%
\pgfpathmoveto{\pgfqpoint{0.000000in}{0.000000in}}%
\pgfpathlineto{\pgfqpoint{0.000000in}{-0.027778in}}%
\pgfusepath{stroke,fill}%
}%
\begin{pgfscope}%
\pgfsys@transformshift{2.797526in}{0.398769in}%
\pgfsys@useobject{currentmarker}{}%
\end{pgfscope}%
\end{pgfscope}%
\begin{pgfscope}%
\pgfsetbuttcap%
\pgfsetroundjoin%
\definecolor{currentfill}{rgb}{0.000000,0.000000,0.000000}%
\pgfsetfillcolor{currentfill}%
\pgfsetlinewidth{0.602250pt}%
\definecolor{currentstroke}{rgb}{0.000000,0.000000,0.000000}%
\pgfsetstrokecolor{currentstroke}%
\pgfsetdash{}{0pt}%
\pgfsys@defobject{currentmarker}{\pgfqpoint{0.000000in}{-0.027778in}}{\pgfqpoint{0.000000in}{0.000000in}}{%
\pgfpathmoveto{\pgfqpoint{0.000000in}{0.000000in}}%
\pgfpathlineto{\pgfqpoint{0.000000in}{-0.027778in}}%
\pgfusepath{stroke,fill}%
}%
\begin{pgfscope}%
\pgfsys@transformshift{2.868477in}{0.398769in}%
\pgfsys@useobject{currentmarker}{}%
\end{pgfscope}%
\end{pgfscope}%
\begin{pgfscope}%
\pgfsetbuttcap%
\pgfsetroundjoin%
\definecolor{currentfill}{rgb}{0.000000,0.000000,0.000000}%
\pgfsetfillcolor{currentfill}%
\pgfsetlinewidth{0.602250pt}%
\definecolor{currentstroke}{rgb}{0.000000,0.000000,0.000000}%
\pgfsetstrokecolor{currentstroke}%
\pgfsetdash{}{0pt}%
\pgfsys@defobject{currentmarker}{\pgfqpoint{0.000000in}{-0.027778in}}{\pgfqpoint{0.000000in}{0.000000in}}{%
\pgfpathmoveto{\pgfqpoint{0.000000in}{0.000000in}}%
\pgfpathlineto{\pgfqpoint{0.000000in}{-0.027778in}}%
\pgfusepath{stroke,fill}%
}%
\begin{pgfscope}%
\pgfsys@transformshift{2.923511in}{0.398769in}%
\pgfsys@useobject{currentmarker}{}%
\end{pgfscope}%
\end{pgfscope}%
\begin{pgfscope}%
\pgfsetbuttcap%
\pgfsetroundjoin%
\definecolor{currentfill}{rgb}{0.000000,0.000000,0.000000}%
\pgfsetfillcolor{currentfill}%
\pgfsetlinewidth{0.602250pt}%
\definecolor{currentstroke}{rgb}{0.000000,0.000000,0.000000}%
\pgfsetstrokecolor{currentstroke}%
\pgfsetdash{}{0pt}%
\pgfsys@defobject{currentmarker}{\pgfqpoint{0.000000in}{-0.027778in}}{\pgfqpoint{0.000000in}{0.000000in}}{%
\pgfpathmoveto{\pgfqpoint{0.000000in}{0.000000in}}%
\pgfpathlineto{\pgfqpoint{0.000000in}{-0.027778in}}%
\pgfusepath{stroke,fill}%
}%
\begin{pgfscope}%
\pgfsys@transformshift{2.968477in}{0.398769in}%
\pgfsys@useobject{currentmarker}{}%
\end{pgfscope}%
\end{pgfscope}%
\begin{pgfscope}%
\pgfsetbuttcap%
\pgfsetroundjoin%
\definecolor{currentfill}{rgb}{0.000000,0.000000,0.000000}%
\pgfsetfillcolor{currentfill}%
\pgfsetlinewidth{0.602250pt}%
\definecolor{currentstroke}{rgb}{0.000000,0.000000,0.000000}%
\pgfsetstrokecolor{currentstroke}%
\pgfsetdash{}{0pt}%
\pgfsys@defobject{currentmarker}{\pgfqpoint{0.000000in}{-0.027778in}}{\pgfqpoint{0.000000in}{0.000000in}}{%
\pgfpathmoveto{\pgfqpoint{0.000000in}{0.000000in}}%
\pgfpathlineto{\pgfqpoint{0.000000in}{-0.027778in}}%
\pgfusepath{stroke,fill}%
}%
\begin{pgfscope}%
\pgfsys@transformshift{3.006496in}{0.398769in}%
\pgfsys@useobject{currentmarker}{}%
\end{pgfscope}%
\end{pgfscope}%
\begin{pgfscope}%
\pgfsetbuttcap%
\pgfsetroundjoin%
\definecolor{currentfill}{rgb}{0.000000,0.000000,0.000000}%
\pgfsetfillcolor{currentfill}%
\pgfsetlinewidth{0.602250pt}%
\definecolor{currentstroke}{rgb}{0.000000,0.000000,0.000000}%
\pgfsetstrokecolor{currentstroke}%
\pgfsetdash{}{0pt}%
\pgfsys@defobject{currentmarker}{\pgfqpoint{0.000000in}{-0.027778in}}{\pgfqpoint{0.000000in}{0.000000in}}{%
\pgfpathmoveto{\pgfqpoint{0.000000in}{0.000000in}}%
\pgfpathlineto{\pgfqpoint{0.000000in}{-0.027778in}}%
\pgfusepath{stroke,fill}%
}%
\begin{pgfscope}%
\pgfsys@transformshift{3.039429in}{0.398769in}%
\pgfsys@useobject{currentmarker}{}%
\end{pgfscope}%
\end{pgfscope}%
\begin{pgfscope}%
\pgfsetbuttcap%
\pgfsetroundjoin%
\definecolor{currentfill}{rgb}{0.000000,0.000000,0.000000}%
\pgfsetfillcolor{currentfill}%
\pgfsetlinewidth{0.602250pt}%
\definecolor{currentstroke}{rgb}{0.000000,0.000000,0.000000}%
\pgfsetstrokecolor{currentstroke}%
\pgfsetdash{}{0pt}%
\pgfsys@defobject{currentmarker}{\pgfqpoint{0.000000in}{-0.027778in}}{\pgfqpoint{0.000000in}{0.000000in}}{%
\pgfpathmoveto{\pgfqpoint{0.000000in}{0.000000in}}%
\pgfpathlineto{\pgfqpoint{0.000000in}{-0.027778in}}%
\pgfusepath{stroke,fill}%
}%
\begin{pgfscope}%
\pgfsys@transformshift{3.068478in}{0.398769in}%
\pgfsys@useobject{currentmarker}{}%
\end{pgfscope}%
\end{pgfscope}%
\begin{pgfscope}%
\pgfsetbuttcap%
\pgfsetroundjoin%
\definecolor{currentfill}{rgb}{0.000000,0.000000,0.000000}%
\pgfsetfillcolor{currentfill}%
\pgfsetlinewidth{0.602250pt}%
\definecolor{currentstroke}{rgb}{0.000000,0.000000,0.000000}%
\pgfsetstrokecolor{currentstroke}%
\pgfsetdash{}{0pt}%
\pgfsys@defobject{currentmarker}{\pgfqpoint{0.000000in}{-0.027778in}}{\pgfqpoint{0.000000in}{0.000000in}}{%
\pgfpathmoveto{\pgfqpoint{0.000000in}{0.000000in}}%
\pgfpathlineto{\pgfqpoint{0.000000in}{-0.027778in}}%
\pgfusepath{stroke,fill}%
}%
\begin{pgfscope}%
\pgfsys@transformshift{3.265414in}{0.398769in}%
\pgfsys@useobject{currentmarker}{}%
\end{pgfscope}%
\end{pgfscope}%
\begin{pgfscope}%
\pgfsetbuttcap%
\pgfsetroundjoin%
\definecolor{currentfill}{rgb}{0.000000,0.000000,0.000000}%
\pgfsetfillcolor{currentfill}%
\pgfsetlinewidth{0.602250pt}%
\definecolor{currentstroke}{rgb}{0.000000,0.000000,0.000000}%
\pgfsetstrokecolor{currentstroke}%
\pgfsetdash{}{0pt}%
\pgfsys@defobject{currentmarker}{\pgfqpoint{0.000000in}{-0.027778in}}{\pgfqpoint{0.000000in}{0.000000in}}{%
\pgfpathmoveto{\pgfqpoint{0.000000in}{0.000000in}}%
\pgfpathlineto{\pgfqpoint{0.000000in}{-0.027778in}}%
\pgfusepath{stroke,fill}%
}%
\begin{pgfscope}%
\pgfsys@transformshift{3.365415in}{0.398769in}%
\pgfsys@useobject{currentmarker}{}%
\end{pgfscope}%
\end{pgfscope}%
\begin{pgfscope}%
\pgfsetbuttcap%
\pgfsetroundjoin%
\definecolor{currentfill}{rgb}{0.000000,0.000000,0.000000}%
\pgfsetfillcolor{currentfill}%
\pgfsetlinewidth{0.602250pt}%
\definecolor{currentstroke}{rgb}{0.000000,0.000000,0.000000}%
\pgfsetstrokecolor{currentstroke}%
\pgfsetdash{}{0pt}%
\pgfsys@defobject{currentmarker}{\pgfqpoint{0.000000in}{-0.027778in}}{\pgfqpoint{0.000000in}{0.000000in}}{%
\pgfpathmoveto{\pgfqpoint{0.000000in}{0.000000in}}%
\pgfpathlineto{\pgfqpoint{0.000000in}{-0.027778in}}%
\pgfusepath{stroke,fill}%
}%
\begin{pgfscope}%
\pgfsys@transformshift{3.436366in}{0.398769in}%
\pgfsys@useobject{currentmarker}{}%
\end{pgfscope}%
\end{pgfscope}%
\begin{pgfscope}%
\pgfsetbuttcap%
\pgfsetroundjoin%
\definecolor{currentfill}{rgb}{0.000000,0.000000,0.000000}%
\pgfsetfillcolor{currentfill}%
\pgfsetlinewidth{0.602250pt}%
\definecolor{currentstroke}{rgb}{0.000000,0.000000,0.000000}%
\pgfsetstrokecolor{currentstroke}%
\pgfsetdash{}{0pt}%
\pgfsys@defobject{currentmarker}{\pgfqpoint{0.000000in}{-0.027778in}}{\pgfqpoint{0.000000in}{0.000000in}}{%
\pgfpathmoveto{\pgfqpoint{0.000000in}{0.000000in}}%
\pgfpathlineto{\pgfqpoint{0.000000in}{-0.027778in}}%
\pgfusepath{stroke,fill}%
}%
\begin{pgfscope}%
\pgfsys@transformshift{3.491400in}{0.398769in}%
\pgfsys@useobject{currentmarker}{}%
\end{pgfscope}%
\end{pgfscope}%
\begin{pgfscope}%
\pgfsetbuttcap%
\pgfsetroundjoin%
\definecolor{currentfill}{rgb}{0.000000,0.000000,0.000000}%
\pgfsetfillcolor{currentfill}%
\pgfsetlinewidth{0.602250pt}%
\definecolor{currentstroke}{rgb}{0.000000,0.000000,0.000000}%
\pgfsetstrokecolor{currentstroke}%
\pgfsetdash{}{0pt}%
\pgfsys@defobject{currentmarker}{\pgfqpoint{0.000000in}{-0.027778in}}{\pgfqpoint{0.000000in}{0.000000in}}{%
\pgfpathmoveto{\pgfqpoint{0.000000in}{0.000000in}}%
\pgfpathlineto{\pgfqpoint{0.000000in}{-0.027778in}}%
\pgfusepath{stroke,fill}%
}%
\begin{pgfscope}%
\pgfsys@transformshift{3.536366in}{0.398769in}%
\pgfsys@useobject{currentmarker}{}%
\end{pgfscope}%
\end{pgfscope}%
\begin{pgfscope}%
\pgfsetbuttcap%
\pgfsetroundjoin%
\definecolor{currentfill}{rgb}{0.000000,0.000000,0.000000}%
\pgfsetfillcolor{currentfill}%
\pgfsetlinewidth{0.602250pt}%
\definecolor{currentstroke}{rgb}{0.000000,0.000000,0.000000}%
\pgfsetstrokecolor{currentstroke}%
\pgfsetdash{}{0pt}%
\pgfsys@defobject{currentmarker}{\pgfqpoint{0.000000in}{-0.027778in}}{\pgfqpoint{0.000000in}{0.000000in}}{%
\pgfpathmoveto{\pgfqpoint{0.000000in}{0.000000in}}%
\pgfpathlineto{\pgfqpoint{0.000000in}{-0.027778in}}%
\pgfusepath{stroke,fill}%
}%
\begin{pgfscope}%
\pgfsys@transformshift{3.574385in}{0.398769in}%
\pgfsys@useobject{currentmarker}{}%
\end{pgfscope}%
\end{pgfscope}%
\begin{pgfscope}%
\pgfsetbuttcap%
\pgfsetroundjoin%
\definecolor{currentfill}{rgb}{0.000000,0.000000,0.000000}%
\pgfsetfillcolor{currentfill}%
\pgfsetlinewidth{0.602250pt}%
\definecolor{currentstroke}{rgb}{0.000000,0.000000,0.000000}%
\pgfsetstrokecolor{currentstroke}%
\pgfsetdash{}{0pt}%
\pgfsys@defobject{currentmarker}{\pgfqpoint{0.000000in}{-0.027778in}}{\pgfqpoint{0.000000in}{0.000000in}}{%
\pgfpathmoveto{\pgfqpoint{0.000000in}{0.000000in}}%
\pgfpathlineto{\pgfqpoint{0.000000in}{-0.027778in}}%
\pgfusepath{stroke,fill}%
}%
\begin{pgfscope}%
\pgfsys@transformshift{3.607318in}{0.398769in}%
\pgfsys@useobject{currentmarker}{}%
\end{pgfscope}%
\end{pgfscope}%
\begin{pgfscope}%
\pgfsetbuttcap%
\pgfsetroundjoin%
\definecolor{currentfill}{rgb}{0.000000,0.000000,0.000000}%
\pgfsetfillcolor{currentfill}%
\pgfsetlinewidth{0.602250pt}%
\definecolor{currentstroke}{rgb}{0.000000,0.000000,0.000000}%
\pgfsetstrokecolor{currentstroke}%
\pgfsetdash{}{0pt}%
\pgfsys@defobject{currentmarker}{\pgfqpoint{0.000000in}{-0.027778in}}{\pgfqpoint{0.000000in}{0.000000in}}{%
\pgfpathmoveto{\pgfqpoint{0.000000in}{0.000000in}}%
\pgfpathlineto{\pgfqpoint{0.000000in}{-0.027778in}}%
\pgfusepath{stroke,fill}%
}%
\begin{pgfscope}%
\pgfsys@transformshift{3.636367in}{0.398769in}%
\pgfsys@useobject{currentmarker}{}%
\end{pgfscope}%
\end{pgfscope}%
\begin{pgfscope}%
\pgfsetbuttcap%
\pgfsetroundjoin%
\definecolor{currentfill}{rgb}{0.000000,0.000000,0.000000}%
\pgfsetfillcolor{currentfill}%
\pgfsetlinewidth{0.602250pt}%
\definecolor{currentstroke}{rgb}{0.000000,0.000000,0.000000}%
\pgfsetstrokecolor{currentstroke}%
\pgfsetdash{}{0pt}%
\pgfsys@defobject{currentmarker}{\pgfqpoint{0.000000in}{-0.027778in}}{\pgfqpoint{0.000000in}{0.000000in}}{%
\pgfpathmoveto{\pgfqpoint{0.000000in}{0.000000in}}%
\pgfpathlineto{\pgfqpoint{0.000000in}{-0.027778in}}%
\pgfusepath{stroke,fill}%
}%
\begin{pgfscope}%
\pgfsys@transformshift{3.833304in}{0.398769in}%
\pgfsys@useobject{currentmarker}{}%
\end{pgfscope}%
\end{pgfscope}%
\begin{pgfscope}%
\pgfsetbuttcap%
\pgfsetroundjoin%
\definecolor{currentfill}{rgb}{0.000000,0.000000,0.000000}%
\pgfsetfillcolor{currentfill}%
\pgfsetlinewidth{0.602250pt}%
\definecolor{currentstroke}{rgb}{0.000000,0.000000,0.000000}%
\pgfsetstrokecolor{currentstroke}%
\pgfsetdash{}{0pt}%
\pgfsys@defobject{currentmarker}{\pgfqpoint{0.000000in}{-0.027778in}}{\pgfqpoint{0.000000in}{0.000000in}}{%
\pgfpathmoveto{\pgfqpoint{0.000000in}{0.000000in}}%
\pgfpathlineto{\pgfqpoint{0.000000in}{-0.027778in}}%
\pgfusepath{stroke,fill}%
}%
\begin{pgfscope}%
\pgfsys@transformshift{3.933304in}{0.398769in}%
\pgfsys@useobject{currentmarker}{}%
\end{pgfscope}%
\end{pgfscope}%
\begin{pgfscope}%
\pgfsetbuttcap%
\pgfsetroundjoin%
\definecolor{currentfill}{rgb}{0.000000,0.000000,0.000000}%
\pgfsetfillcolor{currentfill}%
\pgfsetlinewidth{0.602250pt}%
\definecolor{currentstroke}{rgb}{0.000000,0.000000,0.000000}%
\pgfsetstrokecolor{currentstroke}%
\pgfsetdash{}{0pt}%
\pgfsys@defobject{currentmarker}{\pgfqpoint{0.000000in}{-0.027778in}}{\pgfqpoint{0.000000in}{0.000000in}}{%
\pgfpathmoveto{\pgfqpoint{0.000000in}{0.000000in}}%
\pgfpathlineto{\pgfqpoint{0.000000in}{-0.027778in}}%
\pgfusepath{stroke,fill}%
}%
\begin{pgfscope}%
\pgfsys@transformshift{4.004255in}{0.398769in}%
\pgfsys@useobject{currentmarker}{}%
\end{pgfscope}%
\end{pgfscope}%
\begin{pgfscope}%
\pgfsetbuttcap%
\pgfsetroundjoin%
\definecolor{currentfill}{rgb}{0.000000,0.000000,0.000000}%
\pgfsetfillcolor{currentfill}%
\pgfsetlinewidth{0.602250pt}%
\definecolor{currentstroke}{rgb}{0.000000,0.000000,0.000000}%
\pgfsetstrokecolor{currentstroke}%
\pgfsetdash{}{0pt}%
\pgfsys@defobject{currentmarker}{\pgfqpoint{0.000000in}{-0.027778in}}{\pgfqpoint{0.000000in}{0.000000in}}{%
\pgfpathmoveto{\pgfqpoint{0.000000in}{0.000000in}}%
\pgfpathlineto{\pgfqpoint{0.000000in}{-0.027778in}}%
\pgfusepath{stroke,fill}%
}%
\begin{pgfscope}%
\pgfsys@transformshift{4.059289in}{0.398769in}%
\pgfsys@useobject{currentmarker}{}%
\end{pgfscope}%
\end{pgfscope}%
\begin{pgfscope}%
\pgfsetbuttcap%
\pgfsetroundjoin%
\definecolor{currentfill}{rgb}{0.000000,0.000000,0.000000}%
\pgfsetfillcolor{currentfill}%
\pgfsetlinewidth{0.602250pt}%
\definecolor{currentstroke}{rgb}{0.000000,0.000000,0.000000}%
\pgfsetstrokecolor{currentstroke}%
\pgfsetdash{}{0pt}%
\pgfsys@defobject{currentmarker}{\pgfqpoint{0.000000in}{-0.027778in}}{\pgfqpoint{0.000000in}{0.000000in}}{%
\pgfpathmoveto{\pgfqpoint{0.000000in}{0.000000in}}%
\pgfpathlineto{\pgfqpoint{0.000000in}{-0.027778in}}%
\pgfusepath{stroke,fill}%
}%
\begin{pgfscope}%
\pgfsys@transformshift{4.104256in}{0.398769in}%
\pgfsys@useobject{currentmarker}{}%
\end{pgfscope}%
\end{pgfscope}%
\begin{pgfscope}%
\pgfsetbuttcap%
\pgfsetroundjoin%
\definecolor{currentfill}{rgb}{0.000000,0.000000,0.000000}%
\pgfsetfillcolor{currentfill}%
\pgfsetlinewidth{0.602250pt}%
\definecolor{currentstroke}{rgb}{0.000000,0.000000,0.000000}%
\pgfsetstrokecolor{currentstroke}%
\pgfsetdash{}{0pt}%
\pgfsys@defobject{currentmarker}{\pgfqpoint{0.000000in}{-0.027778in}}{\pgfqpoint{0.000000in}{0.000000in}}{%
\pgfpathmoveto{\pgfqpoint{0.000000in}{0.000000in}}%
\pgfpathlineto{\pgfqpoint{0.000000in}{-0.027778in}}%
\pgfusepath{stroke,fill}%
}%
\begin{pgfscope}%
\pgfsys@transformshift{4.142274in}{0.398769in}%
\pgfsys@useobject{currentmarker}{}%
\end{pgfscope}%
\end{pgfscope}%
\begin{pgfscope}%
\pgfsetbuttcap%
\pgfsetroundjoin%
\definecolor{currentfill}{rgb}{0.000000,0.000000,0.000000}%
\pgfsetfillcolor{currentfill}%
\pgfsetlinewidth{0.602250pt}%
\definecolor{currentstroke}{rgb}{0.000000,0.000000,0.000000}%
\pgfsetstrokecolor{currentstroke}%
\pgfsetdash{}{0pt}%
\pgfsys@defobject{currentmarker}{\pgfqpoint{0.000000in}{-0.027778in}}{\pgfqpoint{0.000000in}{0.000000in}}{%
\pgfpathmoveto{\pgfqpoint{0.000000in}{0.000000in}}%
\pgfpathlineto{\pgfqpoint{0.000000in}{-0.027778in}}%
\pgfusepath{stroke,fill}%
}%
\begin{pgfscope}%
\pgfsys@transformshift{4.175207in}{0.398769in}%
\pgfsys@useobject{currentmarker}{}%
\end{pgfscope}%
\end{pgfscope}%
\begin{pgfscope}%
\pgfsetbuttcap%
\pgfsetroundjoin%
\definecolor{currentfill}{rgb}{0.000000,0.000000,0.000000}%
\pgfsetfillcolor{currentfill}%
\pgfsetlinewidth{0.602250pt}%
\definecolor{currentstroke}{rgb}{0.000000,0.000000,0.000000}%
\pgfsetstrokecolor{currentstroke}%
\pgfsetdash{}{0pt}%
\pgfsys@defobject{currentmarker}{\pgfqpoint{0.000000in}{-0.027778in}}{\pgfqpoint{0.000000in}{0.000000in}}{%
\pgfpathmoveto{\pgfqpoint{0.000000in}{0.000000in}}%
\pgfpathlineto{\pgfqpoint{0.000000in}{-0.027778in}}%
\pgfusepath{stroke,fill}%
}%
\begin{pgfscope}%
\pgfsys@transformshift{4.204256in}{0.398769in}%
\pgfsys@useobject{currentmarker}{}%
\end{pgfscope}%
\end{pgfscope}%
\begin{pgfscope}%
\pgfsetbuttcap%
\pgfsetroundjoin%
\definecolor{currentfill}{rgb}{0.000000,0.000000,0.000000}%
\pgfsetfillcolor{currentfill}%
\pgfsetlinewidth{0.602250pt}%
\definecolor{currentstroke}{rgb}{0.000000,0.000000,0.000000}%
\pgfsetstrokecolor{currentstroke}%
\pgfsetdash{}{0pt}%
\pgfsys@defobject{currentmarker}{\pgfqpoint{0.000000in}{-0.027778in}}{\pgfqpoint{0.000000in}{0.000000in}}{%
\pgfpathmoveto{\pgfqpoint{0.000000in}{0.000000in}}%
\pgfpathlineto{\pgfqpoint{0.000000in}{-0.027778in}}%
\pgfusepath{stroke,fill}%
}%
\begin{pgfscope}%
\pgfsys@transformshift{4.401193in}{0.398769in}%
\pgfsys@useobject{currentmarker}{}%
\end{pgfscope}%
\end{pgfscope}%
\begin{pgfscope}%
\pgfsetbuttcap%
\pgfsetroundjoin%
\definecolor{currentfill}{rgb}{0.000000,0.000000,0.000000}%
\pgfsetfillcolor{currentfill}%
\pgfsetlinewidth{0.602250pt}%
\definecolor{currentstroke}{rgb}{0.000000,0.000000,0.000000}%
\pgfsetstrokecolor{currentstroke}%
\pgfsetdash{}{0pt}%
\pgfsys@defobject{currentmarker}{\pgfqpoint{0.000000in}{-0.027778in}}{\pgfqpoint{0.000000in}{0.000000in}}{%
\pgfpathmoveto{\pgfqpoint{0.000000in}{0.000000in}}%
\pgfpathlineto{\pgfqpoint{0.000000in}{-0.027778in}}%
\pgfusepath{stroke,fill}%
}%
\begin{pgfscope}%
\pgfsys@transformshift{4.501193in}{0.398769in}%
\pgfsys@useobject{currentmarker}{}%
\end{pgfscope}%
\end{pgfscope}%
\begin{pgfscope}%
\pgfsetbuttcap%
\pgfsetroundjoin%
\definecolor{currentfill}{rgb}{0.000000,0.000000,0.000000}%
\pgfsetfillcolor{currentfill}%
\pgfsetlinewidth{0.602250pt}%
\definecolor{currentstroke}{rgb}{0.000000,0.000000,0.000000}%
\pgfsetstrokecolor{currentstroke}%
\pgfsetdash{}{0pt}%
\pgfsys@defobject{currentmarker}{\pgfqpoint{0.000000in}{-0.027778in}}{\pgfqpoint{0.000000in}{0.000000in}}{%
\pgfpathmoveto{\pgfqpoint{0.000000in}{0.000000in}}%
\pgfpathlineto{\pgfqpoint{0.000000in}{-0.027778in}}%
\pgfusepath{stroke,fill}%
}%
\begin{pgfscope}%
\pgfsys@transformshift{4.572144in}{0.398769in}%
\pgfsys@useobject{currentmarker}{}%
\end{pgfscope}%
\end{pgfscope}%
\begin{pgfscope}%
\pgfsetbuttcap%
\pgfsetroundjoin%
\definecolor{currentfill}{rgb}{0.000000,0.000000,0.000000}%
\pgfsetfillcolor{currentfill}%
\pgfsetlinewidth{0.602250pt}%
\definecolor{currentstroke}{rgb}{0.000000,0.000000,0.000000}%
\pgfsetstrokecolor{currentstroke}%
\pgfsetdash{}{0pt}%
\pgfsys@defobject{currentmarker}{\pgfqpoint{0.000000in}{-0.027778in}}{\pgfqpoint{0.000000in}{0.000000in}}{%
\pgfpathmoveto{\pgfqpoint{0.000000in}{0.000000in}}%
\pgfpathlineto{\pgfqpoint{0.000000in}{-0.027778in}}%
\pgfusepath{stroke,fill}%
}%
\begin{pgfscope}%
\pgfsys@transformshift{4.627178in}{0.398769in}%
\pgfsys@useobject{currentmarker}{}%
\end{pgfscope}%
\end{pgfscope}%
\begin{pgfscope}%
\pgfsetbuttcap%
\pgfsetroundjoin%
\definecolor{currentfill}{rgb}{0.000000,0.000000,0.000000}%
\pgfsetfillcolor{currentfill}%
\pgfsetlinewidth{0.602250pt}%
\definecolor{currentstroke}{rgb}{0.000000,0.000000,0.000000}%
\pgfsetstrokecolor{currentstroke}%
\pgfsetdash{}{0pt}%
\pgfsys@defobject{currentmarker}{\pgfqpoint{0.000000in}{-0.027778in}}{\pgfqpoint{0.000000in}{0.000000in}}{%
\pgfpathmoveto{\pgfqpoint{0.000000in}{0.000000in}}%
\pgfpathlineto{\pgfqpoint{0.000000in}{-0.027778in}}%
\pgfusepath{stroke,fill}%
}%
\begin{pgfscope}%
\pgfsys@transformshift{4.672145in}{0.398769in}%
\pgfsys@useobject{currentmarker}{}%
\end{pgfscope}%
\end{pgfscope}%
\begin{pgfscope}%
\pgfsetbuttcap%
\pgfsetroundjoin%
\definecolor{currentfill}{rgb}{0.000000,0.000000,0.000000}%
\pgfsetfillcolor{currentfill}%
\pgfsetlinewidth{0.602250pt}%
\definecolor{currentstroke}{rgb}{0.000000,0.000000,0.000000}%
\pgfsetstrokecolor{currentstroke}%
\pgfsetdash{}{0pt}%
\pgfsys@defobject{currentmarker}{\pgfqpoint{0.000000in}{-0.027778in}}{\pgfqpoint{0.000000in}{0.000000in}}{%
\pgfpathmoveto{\pgfqpoint{0.000000in}{0.000000in}}%
\pgfpathlineto{\pgfqpoint{0.000000in}{-0.027778in}}%
\pgfusepath{stroke,fill}%
}%
\begin{pgfscope}%
\pgfsys@transformshift{4.710163in}{0.398769in}%
\pgfsys@useobject{currentmarker}{}%
\end{pgfscope}%
\end{pgfscope}%
\begin{pgfscope}%
\pgfsetbuttcap%
\pgfsetroundjoin%
\definecolor{currentfill}{rgb}{0.000000,0.000000,0.000000}%
\pgfsetfillcolor{currentfill}%
\pgfsetlinewidth{0.602250pt}%
\definecolor{currentstroke}{rgb}{0.000000,0.000000,0.000000}%
\pgfsetstrokecolor{currentstroke}%
\pgfsetdash{}{0pt}%
\pgfsys@defobject{currentmarker}{\pgfqpoint{0.000000in}{-0.027778in}}{\pgfqpoint{0.000000in}{0.000000in}}{%
\pgfpathmoveto{\pgfqpoint{0.000000in}{0.000000in}}%
\pgfpathlineto{\pgfqpoint{0.000000in}{-0.027778in}}%
\pgfusepath{stroke,fill}%
}%
\begin{pgfscope}%
\pgfsys@transformshift{4.743096in}{0.398769in}%
\pgfsys@useobject{currentmarker}{}%
\end{pgfscope}%
\end{pgfscope}%
\begin{pgfscope}%
\pgfsetbuttcap%
\pgfsetroundjoin%
\definecolor{currentfill}{rgb}{0.000000,0.000000,0.000000}%
\pgfsetfillcolor{currentfill}%
\pgfsetlinewidth{0.602250pt}%
\definecolor{currentstroke}{rgb}{0.000000,0.000000,0.000000}%
\pgfsetstrokecolor{currentstroke}%
\pgfsetdash{}{0pt}%
\pgfsys@defobject{currentmarker}{\pgfqpoint{0.000000in}{-0.027778in}}{\pgfqpoint{0.000000in}{0.000000in}}{%
\pgfpathmoveto{\pgfqpoint{0.000000in}{0.000000in}}%
\pgfpathlineto{\pgfqpoint{0.000000in}{-0.027778in}}%
\pgfusepath{stroke,fill}%
}%
\begin{pgfscope}%
\pgfsys@transformshift{4.772145in}{0.398769in}%
\pgfsys@useobject{currentmarker}{}%
\end{pgfscope}%
\end{pgfscope}%
\begin{pgfscope}%
\pgftext[x=2.788850in,y=0.111578in,,top]{\rmfamily\fontsize{10.000000}{12.000000}\selectfont h}%
\end{pgfscope}%
\begin{pgfscope}%
\pgfsetbuttcap%
\pgfsetroundjoin%
\definecolor{currentfill}{rgb}{0.000000,0.000000,0.000000}%
\pgfsetfillcolor{currentfill}%
\pgfsetlinewidth{0.803000pt}%
\definecolor{currentstroke}{rgb}{0.000000,0.000000,0.000000}%
\pgfsetstrokecolor{currentstroke}%
\pgfsetdash{}{0pt}%
\pgfsys@defobject{currentmarker}{\pgfqpoint{-0.048611in}{0.000000in}}{\pgfqpoint{0.000000in}{0.000000in}}{%
\pgfpathmoveto{\pgfqpoint{0.000000in}{0.000000in}}%
\pgfpathlineto{\pgfqpoint{-0.048611in}{0.000000in}}%
\pgfusepath{stroke,fill}%
}%
\begin{pgfscope}%
\pgfsys@transformshift{0.626312in}{0.587038in}%
\pgfsys@useobject{currentmarker}{}%
\end{pgfscope}%
\end{pgfscope}%
\begin{pgfscope}%
\pgftext[x=0.185724in,y=0.534276in,left,base]{\rmfamily\fontsize{10.000000}{12.000000}\selectfont \(\displaystyle 10^{-15}\)}%
\end{pgfscope}%
\begin{pgfscope}%
\pgfsetbuttcap%
\pgfsetroundjoin%
\definecolor{currentfill}{rgb}{0.000000,0.000000,0.000000}%
\pgfsetfillcolor{currentfill}%
\pgfsetlinewidth{0.803000pt}%
\definecolor{currentstroke}{rgb}{0.000000,0.000000,0.000000}%
\pgfsetstrokecolor{currentstroke}%
\pgfsetdash{}{0pt}%
\pgfsys@defobject{currentmarker}{\pgfqpoint{-0.048611in}{0.000000in}}{\pgfqpoint{0.000000in}{0.000000in}}{%
\pgfpathmoveto{\pgfqpoint{0.000000in}{0.000000in}}%
\pgfpathlineto{\pgfqpoint{-0.048611in}{0.000000in}}%
\pgfusepath{stroke,fill}%
}%
\begin{pgfscope}%
\pgfsys@transformshift{0.626312in}{0.892974in}%
\pgfsys@useobject{currentmarker}{}%
\end{pgfscope}%
\end{pgfscope}%
\begin{pgfscope}%
\pgftext[x=0.185724in,y=0.840212in,left,base]{\rmfamily\fontsize{10.000000}{12.000000}\selectfont \(\displaystyle 10^{-13}\)}%
\end{pgfscope}%
\begin{pgfscope}%
\pgfsetbuttcap%
\pgfsetroundjoin%
\definecolor{currentfill}{rgb}{0.000000,0.000000,0.000000}%
\pgfsetfillcolor{currentfill}%
\pgfsetlinewidth{0.803000pt}%
\definecolor{currentstroke}{rgb}{0.000000,0.000000,0.000000}%
\pgfsetstrokecolor{currentstroke}%
\pgfsetdash{}{0pt}%
\pgfsys@defobject{currentmarker}{\pgfqpoint{-0.048611in}{0.000000in}}{\pgfqpoint{0.000000in}{0.000000in}}{%
\pgfpathmoveto{\pgfqpoint{0.000000in}{0.000000in}}%
\pgfpathlineto{\pgfqpoint{-0.048611in}{0.000000in}}%
\pgfusepath{stroke,fill}%
}%
\begin{pgfscope}%
\pgfsys@transformshift{0.626312in}{1.198910in}%
\pgfsys@useobject{currentmarker}{}%
\end{pgfscope}%
\end{pgfscope}%
\begin{pgfscope}%
\pgftext[x=0.185724in,y=1.146148in,left,base]{\rmfamily\fontsize{10.000000}{12.000000}\selectfont \(\displaystyle 10^{-11}\)}%
\end{pgfscope}%
\begin{pgfscope}%
\pgfsetbuttcap%
\pgfsetroundjoin%
\definecolor{currentfill}{rgb}{0.000000,0.000000,0.000000}%
\pgfsetfillcolor{currentfill}%
\pgfsetlinewidth{0.803000pt}%
\definecolor{currentstroke}{rgb}{0.000000,0.000000,0.000000}%
\pgfsetstrokecolor{currentstroke}%
\pgfsetdash{}{0pt}%
\pgfsys@defobject{currentmarker}{\pgfqpoint{-0.048611in}{0.000000in}}{\pgfqpoint{0.000000in}{0.000000in}}{%
\pgfpathmoveto{\pgfqpoint{0.000000in}{0.000000in}}%
\pgfpathlineto{\pgfqpoint{-0.048611in}{0.000000in}}%
\pgfusepath{stroke,fill}%
}%
\begin{pgfscope}%
\pgfsys@transformshift{0.626312in}{1.504846in}%
\pgfsys@useobject{currentmarker}{}%
\end{pgfscope}%
\end{pgfscope}%
\begin{pgfscope}%
\pgftext[x=0.241087in,y=1.452084in,left,base]{\rmfamily\fontsize{10.000000}{12.000000}\selectfont \(\displaystyle 10^{-9}\)}%
\end{pgfscope}%
\begin{pgfscope}%
\pgfsetbuttcap%
\pgfsetroundjoin%
\definecolor{currentfill}{rgb}{0.000000,0.000000,0.000000}%
\pgfsetfillcolor{currentfill}%
\pgfsetlinewidth{0.803000pt}%
\definecolor{currentstroke}{rgb}{0.000000,0.000000,0.000000}%
\pgfsetstrokecolor{currentstroke}%
\pgfsetdash{}{0pt}%
\pgfsys@defobject{currentmarker}{\pgfqpoint{-0.048611in}{0.000000in}}{\pgfqpoint{0.000000in}{0.000000in}}{%
\pgfpathmoveto{\pgfqpoint{0.000000in}{0.000000in}}%
\pgfpathlineto{\pgfqpoint{-0.048611in}{0.000000in}}%
\pgfusepath{stroke,fill}%
}%
\begin{pgfscope}%
\pgfsys@transformshift{0.626312in}{1.810782in}%
\pgfsys@useobject{currentmarker}{}%
\end{pgfscope}%
\end{pgfscope}%
\begin{pgfscope}%
\pgftext[x=0.241087in,y=1.758021in,left,base]{\rmfamily\fontsize{10.000000}{12.000000}\selectfont \(\displaystyle 10^{-7}\)}%
\end{pgfscope}%
\begin{pgfscope}%
\pgfsetbuttcap%
\pgfsetroundjoin%
\definecolor{currentfill}{rgb}{0.000000,0.000000,0.000000}%
\pgfsetfillcolor{currentfill}%
\pgfsetlinewidth{0.803000pt}%
\definecolor{currentstroke}{rgb}{0.000000,0.000000,0.000000}%
\pgfsetstrokecolor{currentstroke}%
\pgfsetdash{}{0pt}%
\pgfsys@defobject{currentmarker}{\pgfqpoint{-0.048611in}{0.000000in}}{\pgfqpoint{0.000000in}{0.000000in}}{%
\pgfpathmoveto{\pgfqpoint{0.000000in}{0.000000in}}%
\pgfpathlineto{\pgfqpoint{-0.048611in}{0.000000in}}%
\pgfusepath{stroke,fill}%
}%
\begin{pgfscope}%
\pgfsys@transformshift{0.626312in}{2.116718in}%
\pgfsys@useobject{currentmarker}{}%
\end{pgfscope}%
\end{pgfscope}%
\begin{pgfscope}%
\pgftext[x=0.241087in,y=2.063957in,left,base]{\rmfamily\fontsize{10.000000}{12.000000}\selectfont \(\displaystyle 10^{-5}\)}%
\end{pgfscope}%
\begin{pgfscope}%
\pgfsetbuttcap%
\pgfsetroundjoin%
\definecolor{currentfill}{rgb}{0.000000,0.000000,0.000000}%
\pgfsetfillcolor{currentfill}%
\pgfsetlinewidth{0.803000pt}%
\definecolor{currentstroke}{rgb}{0.000000,0.000000,0.000000}%
\pgfsetstrokecolor{currentstroke}%
\pgfsetdash{}{0pt}%
\pgfsys@defobject{currentmarker}{\pgfqpoint{-0.048611in}{0.000000in}}{\pgfqpoint{0.000000in}{0.000000in}}{%
\pgfpathmoveto{\pgfqpoint{0.000000in}{0.000000in}}%
\pgfpathlineto{\pgfqpoint{-0.048611in}{0.000000in}}%
\pgfusepath{stroke,fill}%
}%
\begin{pgfscope}%
\pgfsys@transformshift{0.626312in}{2.422654in}%
\pgfsys@useobject{currentmarker}{}%
\end{pgfscope}%
\end{pgfscope}%
\begin{pgfscope}%
\pgftext[x=0.241087in,y=2.369893in,left,base]{\rmfamily\fontsize{10.000000}{12.000000}\selectfont \(\displaystyle 10^{-3}\)}%
\end{pgfscope}%
\begin{pgfscope}%
\pgfsetbuttcap%
\pgfsetroundjoin%
\definecolor{currentfill}{rgb}{0.000000,0.000000,0.000000}%
\pgfsetfillcolor{currentfill}%
\pgfsetlinewidth{0.803000pt}%
\definecolor{currentstroke}{rgb}{0.000000,0.000000,0.000000}%
\pgfsetstrokecolor{currentstroke}%
\pgfsetdash{}{0pt}%
\pgfsys@defobject{currentmarker}{\pgfqpoint{-0.048611in}{0.000000in}}{\pgfqpoint{0.000000in}{0.000000in}}{%
\pgfpathmoveto{\pgfqpoint{0.000000in}{0.000000in}}%
\pgfpathlineto{\pgfqpoint{-0.048611in}{0.000000in}}%
\pgfusepath{stroke,fill}%
}%
\begin{pgfscope}%
\pgfsys@transformshift{0.626312in}{2.728590in}%
\pgfsys@useobject{currentmarker}{}%
\end{pgfscope}%
\end{pgfscope}%
\begin{pgfscope}%
\pgftext[x=0.241087in,y=2.675829in,left,base]{\rmfamily\fontsize{10.000000}{12.000000}\selectfont \(\displaystyle 10^{-1}\)}%
\end{pgfscope}%
\begin{pgfscope}%
\pgftext[x=0.130169in,y=1.675079in,,bottom,rotate=90.000000]{\rmfamily\fontsize{10.000000}{12.000000}\selectfont E(h)}%
\end{pgfscope}%
\begin{pgfscope}%
\pgfpathrectangle{\pgfqpoint{0.626312in}{0.398769in}}{\pgfqpoint{4.325077in}{2.552620in}} %
\pgfusepath{clip}%
\pgfsetrectcap%
\pgfsetroundjoin%
\pgfsetlinewidth{1.505625pt}%
\definecolor{currentstroke}{rgb}{0.121569,0.466667,0.705882}%
\pgfsetstrokecolor{currentstroke}%
\pgfsetdash{}{0pt}%
\pgfpathmoveto{\pgfqpoint{0.822906in}{2.835361in}}%
\pgfpathlineto{\pgfqpoint{0.993858in}{2.815815in}}%
\pgfpathlineto{\pgfqpoint{1.164810in}{2.690444in}}%
\pgfpathlineto{\pgfqpoint{1.335761in}{2.689011in}}%
\pgfpathlineto{\pgfqpoint{1.506713in}{2.649317in}}%
\pgfpathlineto{\pgfqpoint{1.677665in}{2.621089in}}%
\pgfpathlineto{\pgfqpoint{1.848616in}{2.611698in}}%
\pgfpathlineto{\pgfqpoint{2.019568in}{2.589069in}}%
\pgfpathlineto{\pgfqpoint{2.190520in}{2.566246in}}%
\pgfpathlineto{\pgfqpoint{2.361471in}{2.544142in}}%
\pgfpathlineto{\pgfqpoint{2.532423in}{2.521210in}}%
\pgfpathlineto{\pgfqpoint{2.703375in}{2.498477in}}%
\pgfpathlineto{\pgfqpoint{2.874326in}{2.475652in}}%
\pgfpathlineto{\pgfqpoint{3.045278in}{2.452634in}}%
\pgfpathlineto{\pgfqpoint{3.216230in}{2.429614in}}%
\pgfpathlineto{\pgfqpoint{3.387181in}{2.406604in}}%
\pgfpathlineto{\pgfqpoint{3.558133in}{2.383582in}}%
\pgfpathlineto{\pgfqpoint{3.729085in}{2.360562in}}%
\pgfpathlineto{\pgfqpoint{3.900036in}{2.337541in}}%
\pgfpathlineto{\pgfqpoint{4.070988in}{2.314517in}}%
\pgfpathlineto{\pgfqpoint{4.241939in}{2.291493in}}%
\pgfpathlineto{\pgfqpoint{4.412891in}{2.268470in}}%
\pgfpathlineto{\pgfqpoint{4.583843in}{2.245446in}}%
\pgfpathlineto{\pgfqpoint{4.754794in}{2.222422in}}%
\pgfusepath{stroke}%
\end{pgfscope}%
\begin{pgfscope}%
\pgfpathrectangle{\pgfqpoint{0.626312in}{0.398769in}}{\pgfqpoint{4.325077in}{2.552620in}} %
\pgfusepath{clip}%
\pgfsetrectcap%
\pgfsetroundjoin%
\pgfsetlinewidth{1.505625pt}%
\definecolor{currentstroke}{rgb}{1.000000,0.498039,0.054902}%
\pgfsetstrokecolor{currentstroke}%
\pgfsetdash{}{0pt}%
\pgfpathmoveto{\pgfqpoint{0.822906in}{2.835361in}}%
\pgfpathlineto{\pgfqpoint{0.993858in}{2.815862in}}%
\pgfpathlineto{\pgfqpoint{1.164810in}{2.800535in}}%
\pgfpathlineto{\pgfqpoint{1.335761in}{2.671115in}}%
\pgfpathlineto{\pgfqpoint{1.506713in}{2.674710in}}%
\pgfpathlineto{\pgfqpoint{1.677665in}{2.568583in}}%
\pgfpathlineto{\pgfqpoint{1.848616in}{2.468998in}}%
\pgfpathlineto{\pgfqpoint{2.019568in}{2.488253in}}%
\pgfpathlineto{\pgfqpoint{2.190520in}{2.473791in}}%
\pgfpathlineto{\pgfqpoint{2.361471in}{2.452527in}}%
\pgfpathlineto{\pgfqpoint{2.532423in}{2.430261in}}%
\pgfpathlineto{\pgfqpoint{2.703375in}{2.408734in}}%
\pgfpathlineto{\pgfqpoint{2.874326in}{2.385975in}}%
\pgfpathlineto{\pgfqpoint{3.045278in}{2.362927in}}%
\pgfpathlineto{\pgfqpoint{3.216230in}{2.339916in}}%
\pgfpathlineto{\pgfqpoint{3.387181in}{2.316890in}}%
\pgfpathlineto{\pgfqpoint{3.558133in}{2.293872in}}%
\pgfpathlineto{\pgfqpoint{3.729085in}{2.270871in}}%
\pgfpathlineto{\pgfqpoint{3.900036in}{2.247848in}}%
\pgfpathlineto{\pgfqpoint{4.070988in}{2.224824in}}%
\pgfpathlineto{\pgfqpoint{4.241939in}{2.201801in}}%
\pgfpathlineto{\pgfqpoint{4.412891in}{2.178776in}}%
\pgfpathlineto{\pgfqpoint{4.583843in}{2.155753in}}%
\pgfpathlineto{\pgfqpoint{4.754794in}{2.132729in}}%
\pgfusepath{stroke}%
\end{pgfscope}%
\begin{pgfscope}%
\pgfpathrectangle{\pgfqpoint{0.626312in}{0.398769in}}{\pgfqpoint{4.325077in}{2.552620in}} %
\pgfusepath{clip}%
\pgfsetrectcap%
\pgfsetroundjoin%
\pgfsetlinewidth{1.505625pt}%
\definecolor{currentstroke}{rgb}{0.172549,0.627451,0.172549}%
\pgfsetstrokecolor{currentstroke}%
\pgfsetdash{}{0pt}%
\pgfpathmoveto{\pgfqpoint{0.822906in}{2.835361in}}%
\pgfpathlineto{\pgfqpoint{0.993858in}{2.815862in}}%
\pgfpathlineto{\pgfqpoint{1.164810in}{2.799789in}}%
\pgfpathlineto{\pgfqpoint{1.335761in}{2.671115in}}%
\pgfpathlineto{\pgfqpoint{1.506713in}{2.677944in}}%
\pgfpathlineto{\pgfqpoint{1.677665in}{2.571593in}}%
\pgfpathlineto{\pgfqpoint{1.848616in}{2.454897in}}%
\pgfpathlineto{\pgfqpoint{2.019568in}{2.382490in}}%
\pgfpathlineto{\pgfqpoint{2.190520in}{2.344595in}}%
\pgfpathlineto{\pgfqpoint{2.361471in}{2.318101in}}%
\pgfpathlineto{\pgfqpoint{2.532423in}{2.294606in}}%
\pgfpathlineto{\pgfqpoint{2.703375in}{2.272610in}}%
\pgfpathlineto{\pgfqpoint{2.874326in}{2.249759in}}%
\pgfpathlineto{\pgfqpoint{3.045278in}{2.226752in}}%
\pgfpathlineto{\pgfqpoint{3.216230in}{2.203733in}}%
\pgfpathlineto{\pgfqpoint{3.387181in}{2.180720in}}%
\pgfpathlineto{\pgfqpoint{3.558133in}{2.157701in}}%
\pgfpathlineto{\pgfqpoint{3.729085in}{2.134696in}}%
\pgfpathlineto{\pgfqpoint{3.900036in}{2.111676in}}%
\pgfpathlineto{\pgfqpoint{4.070988in}{2.088653in}}%
\pgfpathlineto{\pgfqpoint{4.241939in}{2.065629in}}%
\pgfpathlineto{\pgfqpoint{4.412891in}{2.042605in}}%
\pgfpathlineto{\pgfqpoint{4.583843in}{2.019581in}}%
\pgfpathlineto{\pgfqpoint{4.754794in}{1.996557in}}%
\pgfusepath{stroke}%
\end{pgfscope}%
\begin{pgfscope}%
\pgfpathrectangle{\pgfqpoint{0.626312in}{0.398769in}}{\pgfqpoint{4.325077in}{2.552620in}} %
\pgfusepath{clip}%
\pgfsetrectcap%
\pgfsetroundjoin%
\pgfsetlinewidth{1.505625pt}%
\definecolor{currentstroke}{rgb}{0.839216,0.152941,0.156863}%
\pgfsetstrokecolor{currentstroke}%
\pgfsetdash{}{0pt}%
\pgfpathmoveto{\pgfqpoint{0.822906in}{2.835361in}}%
\pgfpathlineto{\pgfqpoint{0.993858in}{2.815862in}}%
\pgfpathlineto{\pgfqpoint{1.164810in}{2.800134in}}%
\pgfpathlineto{\pgfqpoint{1.335761in}{2.671115in}}%
\pgfpathlineto{\pgfqpoint{1.506713in}{2.677944in}}%
\pgfpathlineto{\pgfqpoint{1.677665in}{2.571787in}}%
\pgfpathlineto{\pgfqpoint{1.848616in}{2.449418in}}%
\pgfpathlineto{\pgfqpoint{2.019568in}{2.344805in}}%
\pgfpathlineto{\pgfqpoint{2.190520in}{2.265327in}}%
\pgfpathlineto{\pgfqpoint{2.361471in}{2.220972in}}%
\pgfpathlineto{\pgfqpoint{2.532423in}{2.191170in}}%
\pgfpathlineto{\pgfqpoint{2.703375in}{2.166763in}}%
\pgfpathlineto{\pgfqpoint{2.874326in}{2.144064in}}%
\pgfpathlineto{\pgfqpoint{3.045278in}{2.120924in}}%
\pgfpathlineto{\pgfqpoint{3.216230in}{2.097870in}}%
\pgfpathlineto{\pgfqpoint{3.387181in}{2.074904in}}%
\pgfpathlineto{\pgfqpoint{3.558133in}{2.051880in}}%
\pgfpathlineto{\pgfqpoint{3.729085in}{2.028862in}}%
\pgfpathlineto{\pgfqpoint{3.900036in}{2.005850in}}%
\pgfpathlineto{\pgfqpoint{4.070988in}{1.982826in}}%
\pgfpathlineto{\pgfqpoint{4.241939in}{1.959802in}}%
\pgfpathlineto{\pgfqpoint{4.412891in}{1.936779in}}%
\pgfpathlineto{\pgfqpoint{4.583843in}{1.913755in}}%
\pgfpathlineto{\pgfqpoint{4.754794in}{1.890731in}}%
\pgfusepath{stroke}%
\end{pgfscope}%
\begin{pgfscope}%
\pgfpathrectangle{\pgfqpoint{0.626312in}{0.398769in}}{\pgfqpoint{4.325077in}{2.552620in}} %
\pgfusepath{clip}%
\pgfsetrectcap%
\pgfsetroundjoin%
\pgfsetlinewidth{1.505625pt}%
\definecolor{currentstroke}{rgb}{0.580392,0.403922,0.741176}%
\pgfsetstrokecolor{currentstroke}%
\pgfsetdash{}{0pt}%
\pgfpathmoveto{\pgfqpoint{0.822906in}{2.835361in}}%
\pgfpathlineto{\pgfqpoint{0.993858in}{2.815862in}}%
\pgfpathlineto{\pgfqpoint{1.164810in}{2.800134in}}%
\pgfpathlineto{\pgfqpoint{1.335761in}{2.671115in}}%
\pgfpathlineto{\pgfqpoint{1.506713in}{2.677944in}}%
\pgfpathlineto{\pgfqpoint{1.677665in}{2.571757in}}%
\pgfpathlineto{\pgfqpoint{1.848616in}{2.446755in}}%
\pgfpathlineto{\pgfqpoint{2.019568in}{2.330418in}}%
\pgfpathlineto{\pgfqpoint{2.190520in}{2.216668in}}%
\pgfpathlineto{\pgfqpoint{2.361471in}{2.111426in}}%
\pgfpathlineto{\pgfqpoint{2.532423in}{2.035524in}}%
\pgfpathlineto{\pgfqpoint{2.703375in}{1.994719in}}%
\pgfpathlineto{\pgfqpoint{2.874326in}{1.968189in}}%
\pgfpathlineto{\pgfqpoint{3.045278in}{1.944187in}}%
\pgfpathlineto{\pgfqpoint{3.216230in}{1.920921in}}%
\pgfpathlineto{\pgfqpoint{3.387181in}{1.897911in}}%
\pgfpathlineto{\pgfqpoint{3.558133in}{1.874873in}}%
\pgfpathlineto{\pgfqpoint{3.729085in}{1.851843in}}%
\pgfpathlineto{\pgfqpoint{3.900036in}{1.828828in}}%
\pgfpathlineto{\pgfqpoint{4.070988in}{1.805803in}}%
\pgfpathlineto{\pgfqpoint{4.241939in}{1.782779in}}%
\pgfpathlineto{\pgfqpoint{4.412891in}{1.759756in}}%
\pgfpathlineto{\pgfqpoint{4.583843in}{1.736732in}}%
\pgfpathlineto{\pgfqpoint{4.754794in}{1.713708in}}%
\pgfusepath{stroke}%
\end{pgfscope}%
\begin{pgfscope}%
\pgfpathrectangle{\pgfqpoint{0.626312in}{0.398769in}}{\pgfqpoint{4.325077in}{2.552620in}} %
\pgfusepath{clip}%
\pgfsetrectcap%
\pgfsetroundjoin%
\pgfsetlinewidth{1.505625pt}%
\definecolor{currentstroke}{rgb}{0.549020,0.337255,0.294118}%
\pgfsetstrokecolor{currentstroke}%
\pgfsetdash{}{0pt}%
\pgfpathmoveto{\pgfqpoint{0.822906in}{2.835361in}}%
\pgfpathlineto{\pgfqpoint{0.993858in}{2.815862in}}%
\pgfpathlineto{\pgfqpoint{1.164810in}{2.800134in}}%
\pgfpathlineto{\pgfqpoint{1.335761in}{2.671115in}}%
\pgfpathlineto{\pgfqpoint{1.506713in}{2.677944in}}%
\pgfpathlineto{\pgfqpoint{1.677665in}{2.571757in}}%
\pgfpathlineto{\pgfqpoint{1.848616in}{2.446642in}}%
\pgfpathlineto{\pgfqpoint{2.019568in}{2.329729in}}%
\pgfpathlineto{\pgfqpoint{2.190520in}{2.213596in}}%
\pgfpathlineto{\pgfqpoint{2.361471in}{2.097146in}}%
\pgfpathlineto{\pgfqpoint{2.532423in}{1.981608in}}%
\pgfpathlineto{\pgfqpoint{2.703375in}{1.866133in}}%
\pgfpathlineto{\pgfqpoint{2.874326in}{1.756426in}}%
\pgfpathlineto{\pgfqpoint{3.045278in}{1.690393in}}%
\pgfpathlineto{\pgfqpoint{3.216230in}{1.663350in}}%
\pgfpathlineto{\pgfqpoint{3.387181in}{1.640712in}}%
\pgfpathlineto{\pgfqpoint{3.558133in}{1.617834in}}%
\pgfpathlineto{\pgfqpoint{3.729085in}{1.594841in}}%
\pgfpathlineto{\pgfqpoint{3.900036in}{1.571847in}}%
\pgfpathlineto{\pgfqpoint{4.070988in}{1.548824in}}%
\pgfpathlineto{\pgfqpoint{4.241939in}{1.525802in}}%
\pgfpathlineto{\pgfqpoint{4.412891in}{1.502781in}}%
\pgfpathlineto{\pgfqpoint{4.583843in}{1.479757in}}%
\pgfpathlineto{\pgfqpoint{4.754794in}{1.456733in}}%
\pgfusepath{stroke}%
\end{pgfscope}%
\begin{pgfscope}%
\pgfpathrectangle{\pgfqpoint{0.626312in}{0.398769in}}{\pgfqpoint{4.325077in}{2.552620in}} %
\pgfusepath{clip}%
\pgfsetrectcap%
\pgfsetroundjoin%
\pgfsetlinewidth{1.505625pt}%
\definecolor{currentstroke}{rgb}{0.890196,0.466667,0.760784}%
\pgfsetstrokecolor{currentstroke}%
\pgfsetdash{}{0pt}%
\pgfpathmoveto{\pgfqpoint{0.822906in}{2.835361in}}%
\pgfpathlineto{\pgfqpoint{0.993858in}{2.815862in}}%
\pgfpathlineto{\pgfqpoint{1.164810in}{2.800134in}}%
\pgfpathlineto{\pgfqpoint{1.335761in}{2.671115in}}%
\pgfpathlineto{\pgfqpoint{1.506713in}{2.677944in}}%
\pgfpathlineto{\pgfqpoint{1.677665in}{2.571757in}}%
\pgfpathlineto{\pgfqpoint{1.848616in}{2.446643in}}%
\pgfpathlineto{\pgfqpoint{2.019568in}{2.329732in}}%
\pgfpathlineto{\pgfqpoint{2.190520in}{2.213607in}}%
\pgfpathlineto{\pgfqpoint{2.361471in}{2.097205in}}%
\pgfpathlineto{\pgfqpoint{2.532423in}{1.981815in}}%
\pgfpathlineto{\pgfqpoint{2.703375in}{1.866529in}}%
\pgfpathlineto{\pgfqpoint{2.874326in}{1.751271in}}%
\pgfpathlineto{\pgfqpoint{3.045278in}{1.636766in}}%
\pgfpathlineto{\pgfqpoint{3.216230in}{1.529569in}}%
\pgfpathlineto{\pgfqpoint{3.387181in}{1.462861in}}%
\pgfpathlineto{\pgfqpoint{3.558133in}{1.433331in}}%
\pgfpathlineto{\pgfqpoint{3.729085in}{1.409773in}}%
\pgfpathlineto{\pgfqpoint{3.900036in}{1.386717in}}%
\pgfpathlineto{\pgfqpoint{4.070988in}{1.363684in}}%
\pgfpathlineto{\pgfqpoint{4.241939in}{1.340659in}}%
\pgfpathlineto{\pgfqpoint{4.412891in}{1.317637in}}%
\pgfpathlineto{\pgfqpoint{4.583843in}{1.294613in}}%
\pgfpathlineto{\pgfqpoint{4.754794in}{1.271589in}}%
\pgfusepath{stroke}%
\end{pgfscope}%
\begin{pgfscope}%
\pgfpathrectangle{\pgfqpoint{0.626312in}{0.398769in}}{\pgfqpoint{4.325077in}{2.552620in}} %
\pgfusepath{clip}%
\pgfsetrectcap%
\pgfsetroundjoin%
\pgfsetlinewidth{1.505625pt}%
\definecolor{currentstroke}{rgb}{0.498039,0.498039,0.498039}%
\pgfsetstrokecolor{currentstroke}%
\pgfsetdash{}{0pt}%
\pgfpathmoveto{\pgfqpoint{0.822906in}{2.835361in}}%
\pgfpathlineto{\pgfqpoint{0.993858in}{2.815862in}}%
\pgfpathlineto{\pgfqpoint{1.164810in}{2.800134in}}%
\pgfpathlineto{\pgfqpoint{1.335761in}{2.671115in}}%
\pgfpathlineto{\pgfqpoint{1.506713in}{2.677944in}}%
\pgfpathlineto{\pgfqpoint{1.677665in}{2.571757in}}%
\pgfpathlineto{\pgfqpoint{1.848616in}{2.446643in}}%
\pgfpathlineto{\pgfqpoint{2.019568in}{2.329732in}}%
\pgfpathlineto{\pgfqpoint{2.190520in}{2.213607in}}%
\pgfpathlineto{\pgfqpoint{2.361471in}{2.097204in}}%
\pgfpathlineto{\pgfqpoint{2.532423in}{1.981813in}}%
\pgfpathlineto{\pgfqpoint{2.703375in}{1.866518in}}%
\pgfpathlineto{\pgfqpoint{2.874326in}{1.751197in}}%
\pgfpathlineto{\pgfqpoint{3.045278in}{1.636044in}}%
\pgfpathlineto{\pgfqpoint{3.216230in}{1.520911in}}%
\pgfpathlineto{\pgfqpoint{3.387181in}{1.405850in}}%
\pgfpathlineto{\pgfqpoint{3.558133in}{1.292083in}}%
\pgfpathlineto{\pgfqpoint{3.729085in}{1.193708in}}%
\pgfpathlineto{\pgfqpoint{3.900036in}{1.144979in}}%
\pgfpathlineto{\pgfqpoint{4.070988in}{1.119478in}}%
\pgfpathlineto{\pgfqpoint{4.241939in}{1.096307in}}%
\pgfpathlineto{\pgfqpoint{4.412891in}{1.073283in}}%
\pgfpathlineto{\pgfqpoint{4.583843in}{1.050257in}}%
\pgfpathlineto{\pgfqpoint{4.754794in}{1.027231in}}%
\pgfusepath{stroke}%
\end{pgfscope}%
\begin{pgfscope}%
\pgfpathrectangle{\pgfqpoint{0.626312in}{0.398769in}}{\pgfqpoint{4.325077in}{2.552620in}} %
\pgfusepath{clip}%
\pgfsetrectcap%
\pgfsetroundjoin%
\pgfsetlinewidth{1.505625pt}%
\definecolor{currentstroke}{rgb}{0.737255,0.741176,0.133333}%
\pgfsetstrokecolor{currentstroke}%
\pgfsetdash{}{0pt}%
\pgfpathmoveto{\pgfqpoint{0.822906in}{2.835361in}}%
\pgfpathlineto{\pgfqpoint{0.993858in}{2.815862in}}%
\pgfpathlineto{\pgfqpoint{1.164810in}{2.800134in}}%
\pgfpathlineto{\pgfqpoint{1.335761in}{2.671115in}}%
\pgfpathlineto{\pgfqpoint{1.506713in}{2.677944in}}%
\pgfpathlineto{\pgfqpoint{1.677665in}{2.571757in}}%
\pgfpathlineto{\pgfqpoint{1.848616in}{2.446643in}}%
\pgfpathlineto{\pgfqpoint{2.019568in}{2.329732in}}%
\pgfpathlineto{\pgfqpoint{2.190520in}{2.213607in}}%
\pgfpathlineto{\pgfqpoint{2.361471in}{2.097204in}}%
\pgfpathlineto{\pgfqpoint{2.532423in}{1.981813in}}%
\pgfpathlineto{\pgfqpoint{2.703375in}{1.866518in}}%
\pgfpathlineto{\pgfqpoint{2.874326in}{1.751198in}}%
\pgfpathlineto{\pgfqpoint{3.045278in}{1.636045in}}%
\pgfpathlineto{\pgfqpoint{3.216230in}{1.520909in}}%
\pgfpathlineto{\pgfqpoint{3.387181in}{1.405769in}}%
\pgfpathlineto{\pgfqpoint{3.558133in}{1.290645in}}%
\pgfpathlineto{\pgfqpoint{3.729085in}{1.175522in}}%
\pgfpathlineto{\pgfqpoint{3.900036in}{1.060400in}}%
\pgfpathlineto{\pgfqpoint{4.070988in}{0.945288in}}%
\pgfpathlineto{\pgfqpoint{4.241939in}{0.830296in}}%
\pgfpathlineto{\pgfqpoint{4.412891in}{0.715251in}}%
\pgfpathlineto{\pgfqpoint{4.583843in}{0.603557in}}%
\pgfpathlineto{\pgfqpoint{4.754794in}{0.514797in}}%
\pgfusepath{stroke}%
\end{pgfscope}%
\begin{pgfscope}%
\pgfsetrectcap%
\pgfsetmiterjoin%
\pgfsetlinewidth{0.803000pt}%
\definecolor{currentstroke}{rgb}{0.000000,0.000000,0.000000}%
\pgfsetstrokecolor{currentstroke}%
\pgfsetdash{}{0pt}%
\pgfpathmoveto{\pgfqpoint{0.626312in}{0.398769in}}%
\pgfpathlineto{\pgfqpoint{0.626312in}{2.951389in}}%
\pgfusepath{stroke}%
\end{pgfscope}%
\begin{pgfscope}%
\pgfsetrectcap%
\pgfsetmiterjoin%
\pgfsetlinewidth{0.803000pt}%
\definecolor{currentstroke}{rgb}{0.000000,0.000000,0.000000}%
\pgfsetstrokecolor{currentstroke}%
\pgfsetdash{}{0pt}%
\pgfpathmoveto{\pgfqpoint{4.951389in}{0.398769in}}%
\pgfpathlineto{\pgfqpoint{4.951389in}{2.951389in}}%
\pgfusepath{stroke}%
\end{pgfscope}%
\begin{pgfscope}%
\pgfsetrectcap%
\pgfsetmiterjoin%
\pgfsetlinewidth{0.803000pt}%
\definecolor{currentstroke}{rgb}{0.000000,0.000000,0.000000}%
\pgfsetstrokecolor{currentstroke}%
\pgfsetdash{}{0pt}%
\pgfpathmoveto{\pgfqpoint{0.626312in}{0.398769in}}%
\pgfpathlineto{\pgfqpoint{4.951389in}{0.398769in}}%
\pgfusepath{stroke}%
\end{pgfscope}%
\begin{pgfscope}%
\pgfsetrectcap%
\pgfsetmiterjoin%
\pgfsetlinewidth{0.803000pt}%
\definecolor{currentstroke}{rgb}{0.000000,0.000000,0.000000}%
\pgfsetstrokecolor{currentstroke}%
\pgfsetdash{}{0pt}%
\pgfpathmoveto{\pgfqpoint{0.626312in}{2.951389in}}%
\pgfpathlineto{\pgfqpoint{4.951389in}{2.951389in}}%
\pgfusepath{stroke}%
\end{pgfscope}%
\begin{pgfscope}%
\pgfsetbuttcap%
\pgfsetmiterjoin%
\definecolor{currentfill}{rgb}{1.000000,1.000000,1.000000}%
\pgfsetfillcolor{currentfill}%
\pgfsetfillopacity{0.800000}%
\pgfsetlinewidth{1.003750pt}%
\definecolor{currentstroke}{rgb}{0.800000,0.800000,0.800000}%
\pgfsetstrokecolor{currentstroke}%
\pgfsetstrokeopacity{0.800000}%
\pgfsetdash{}{0pt}%
\pgfpathmoveto{\pgfqpoint{0.723534in}{0.468213in}}%
\pgfpathlineto{\pgfqpoint{1.256344in}{0.468213in}}%
\pgfpathquadraticcurveto{\pgfqpoint{1.284122in}{0.468213in}}{\pgfqpoint{1.284122in}{0.495991in}}%
\pgfpathlineto{\pgfqpoint{1.284122in}{2.316817in}}%
\pgfpathquadraticcurveto{\pgfqpoint{1.284122in}{2.344595in}}{\pgfqpoint{1.256344in}{2.344595in}}%
\pgfpathlineto{\pgfqpoint{0.723534in}{2.344595in}}%
\pgfpathquadraticcurveto{\pgfqpoint{0.695756in}{2.344595in}}{\pgfqpoint{0.695756in}{2.316817in}}%
\pgfpathlineto{\pgfqpoint{0.695756in}{0.495991in}}%
\pgfpathquadraticcurveto{\pgfqpoint{0.695756in}{0.468213in}}{\pgfqpoint{0.723534in}{0.468213in}}%
\pgfpathclose%
\pgfusepath{stroke,fill}%
\end{pgfscope}%
\begin{pgfscope}%
\pgfsetrectcap%
\pgfsetroundjoin%
\pgfsetlinewidth{1.505625pt}%
\definecolor{currentstroke}{rgb}{0.121569,0.466667,0.705882}%
\pgfsetstrokecolor{currentstroke}%
\pgfsetdash{}{0pt}%
\pgfpathmoveto{\pgfqpoint{0.751312in}{2.232127in}}%
\pgfpathlineto{\pgfqpoint{1.029090in}{2.232127in}}%
\pgfusepath{stroke}%
\end{pgfscope}%
\begin{pgfscope}%
\pgftext[x=1.140201in,y=2.183516in,left,base]{\rmfamily\fontsize{10.000000}{12.000000}\selectfont 1}%
\end{pgfscope}%
\begin{pgfscope}%
\pgfsetrectcap%
\pgfsetroundjoin%
\pgfsetlinewidth{1.505625pt}%
\definecolor{currentstroke}{rgb}{1.000000,0.498039,0.054902}%
\pgfsetstrokecolor{currentstroke}%
\pgfsetdash{}{0pt}%
\pgfpathmoveto{\pgfqpoint{0.751312in}{2.028270in}}%
\pgfpathlineto{\pgfqpoint{1.029090in}{2.028270in}}%
\pgfusepath{stroke}%
\end{pgfscope}%
\begin{pgfscope}%
\pgftext[x=1.140201in,y=1.979659in,left,base]{\rmfamily\fontsize{10.000000}{12.000000}\selectfont 2}%
\end{pgfscope}%
\begin{pgfscope}%
\pgfsetrectcap%
\pgfsetroundjoin%
\pgfsetlinewidth{1.505625pt}%
\definecolor{currentstroke}{rgb}{0.172549,0.627451,0.172549}%
\pgfsetstrokecolor{currentstroke}%
\pgfsetdash{}{0pt}%
\pgfpathmoveto{\pgfqpoint{0.751312in}{1.824413in}}%
\pgfpathlineto{\pgfqpoint{1.029090in}{1.824413in}}%
\pgfusepath{stroke}%
\end{pgfscope}%
\begin{pgfscope}%
\pgftext[x=1.140201in,y=1.775802in,left,base]{\rmfamily\fontsize{10.000000}{12.000000}\selectfont 3}%
\end{pgfscope}%
\begin{pgfscope}%
\pgfsetrectcap%
\pgfsetroundjoin%
\pgfsetlinewidth{1.505625pt}%
\definecolor{currentstroke}{rgb}{0.839216,0.152941,0.156863}%
\pgfsetstrokecolor{currentstroke}%
\pgfsetdash{}{0pt}%
\pgfpathmoveto{\pgfqpoint{0.751312in}{1.620556in}}%
\pgfpathlineto{\pgfqpoint{1.029090in}{1.620556in}}%
\pgfusepath{stroke}%
\end{pgfscope}%
\begin{pgfscope}%
\pgftext[x=1.140201in,y=1.571945in,left,base]{\rmfamily\fontsize{10.000000}{12.000000}\selectfont 4}%
\end{pgfscope}%
\begin{pgfscope}%
\pgfsetrectcap%
\pgfsetroundjoin%
\pgfsetlinewidth{1.505625pt}%
\definecolor{currentstroke}{rgb}{0.580392,0.403922,0.741176}%
\pgfsetstrokecolor{currentstroke}%
\pgfsetdash{}{0pt}%
\pgfpathmoveto{\pgfqpoint{0.751312in}{1.416699in}}%
\pgfpathlineto{\pgfqpoint{1.029090in}{1.416699in}}%
\pgfusepath{stroke}%
\end{pgfscope}%
\begin{pgfscope}%
\pgftext[x=1.140201in,y=1.368087in,left,base]{\rmfamily\fontsize{10.000000}{12.000000}\selectfont 5}%
\end{pgfscope}%
\begin{pgfscope}%
\pgfsetrectcap%
\pgfsetroundjoin%
\pgfsetlinewidth{1.505625pt}%
\definecolor{currentstroke}{rgb}{0.549020,0.337255,0.294118}%
\pgfsetstrokecolor{currentstroke}%
\pgfsetdash{}{0pt}%
\pgfpathmoveto{\pgfqpoint{0.751312in}{1.212841in}}%
\pgfpathlineto{\pgfqpoint{1.029090in}{1.212841in}}%
\pgfusepath{stroke}%
\end{pgfscope}%
\begin{pgfscope}%
\pgftext[x=1.140201in,y=1.164230in,left,base]{\rmfamily\fontsize{10.000000}{12.000000}\selectfont 6}%
\end{pgfscope}%
\begin{pgfscope}%
\pgfsetrectcap%
\pgfsetroundjoin%
\pgfsetlinewidth{1.505625pt}%
\definecolor{currentstroke}{rgb}{0.890196,0.466667,0.760784}%
\pgfsetstrokecolor{currentstroke}%
\pgfsetdash{}{0pt}%
\pgfpathmoveto{\pgfqpoint{0.751312in}{1.008984in}}%
\pgfpathlineto{\pgfqpoint{1.029090in}{1.008984in}}%
\pgfusepath{stroke}%
\end{pgfscope}%
\begin{pgfscope}%
\pgftext[x=1.140201in,y=0.960373in,left,base]{\rmfamily\fontsize{10.000000}{12.000000}\selectfont 7}%
\end{pgfscope}%
\begin{pgfscope}%
\pgfsetrectcap%
\pgfsetroundjoin%
\pgfsetlinewidth{1.505625pt}%
\definecolor{currentstroke}{rgb}{0.498039,0.498039,0.498039}%
\pgfsetstrokecolor{currentstroke}%
\pgfsetdash{}{0pt}%
\pgfpathmoveto{\pgfqpoint{0.751312in}{0.805127in}}%
\pgfpathlineto{\pgfqpoint{1.029090in}{0.805127in}}%
\pgfusepath{stroke}%
\end{pgfscope}%
\begin{pgfscope}%
\pgftext[x=1.140201in,y=0.756516in,left,base]{\rmfamily\fontsize{10.000000}{12.000000}\selectfont 8}%
\end{pgfscope}%
\begin{pgfscope}%
\pgfsetrectcap%
\pgfsetroundjoin%
\pgfsetlinewidth{1.505625pt}%
\definecolor{currentstroke}{rgb}{0.737255,0.741176,0.133333}%
\pgfsetstrokecolor{currentstroke}%
\pgfsetdash{}{0pt}%
\pgfpathmoveto{\pgfqpoint{0.751312in}{0.601270in}}%
\pgfpathlineto{\pgfqpoint{1.029090in}{0.601270in}}%
\pgfusepath{stroke}%
\end{pgfscope}%
\begin{pgfscope}%
\pgftext[x=1.140201in,y=0.552659in,left,base]{\rmfamily\fontsize{10.000000}{12.000000}\selectfont 9}%
\end{pgfscope}%
\end{pgfpicture}%
\makeatother%
\endgroup%

    \caption{Globaler Fehler bei steigender Präzision von \(\mathcal{F}\).}
    \label{fig:iter_error}
\end{figure}

\subsubsection*{Automatischer Abbruch}
Der erwartbare Wunsch des Nutzers eines Lösers ist es, die Parallelisierung durch eine boolesche Option zu aktivieren. Dies ist jedoch nicht ohne weiteres möglich, nachdem vor der Ausführung des Algorithmus einige Entscheidungen getroffen werden müssen. Dazu zählen die Präzision von \(\mathcal{G}\), die Anzahl der Prozesse und die Anzahl der Iterationen. Vor allem eine feste Anzahl von Iterationen vorzugeben, scheint unpraktikabel. Dies macht ein dynamisches Abbruchkriterium notwendig, welches während der Ausführung den richtigen Zeitpunkt füt den Abbruch findet. Der Vergleich der Abbildungen~\ref{fig:iters_log}, \ref{fig:iter_error_local} und \ref{fig:iter_error_local_num} suggeriert ein erstes Problem. Alle drei lassen verschiedene Iterationsanzahlen als sinnvoll erscheinen. Während der Vergleich mit der exakten Lösung impraktikabel ist, stellt es Offensichtlich das Optimum dar. So liegen die Ergebnisse ab der siebten Iteration zwar näher an der numerischen Lösung, durch den ihr innewohnenden Fehler ist das Ergebnis dadurch effektiv aber nicht besser. Unglücklicherweise liegen jedoch beide Lösungen zur Laufzeit nicht vor.\\

Verfügbar sind jedoch die Größe der Korrekturen beim Zusammenführen der Ergebnisse(Abbildung~\ref{fig:error_corr}), und die Größe der Sprungsstellen am Ende einer Iteration(Abbildung~\ref{fig:error_disc}). Zu sehen sind für einige Genauigkeiten jeweils wie sich die Beobachtete Größe zum globalen Fehler verhält. Die Korrekturen entwickeln sich ungeachtet des Erreichen des Fehlerplateaus konstant weiter und sind folglich kein guter Indikator. Ganz anderes verhält es sich jedoch mit den Sprungsstellen sie scheinen sich ähnlich schnell wie die Fehler zu stabilisieren, dadurch erscheinen sie ein gutes Abbruchkriterium darzustellen.

\begin{figure}[ht]
    \centering
        %% Creator: Matplotlib, PGF backend
%%
%% To include the figure in your LaTeX document, write
%%   \input{<filename>.pgf}
%%
%% Make sure the required packages are loaded in your preamble
%%   \usepackage{pgf}
%%
%% Figures using additional raster images can only be included by \input if
%% they are in the same directory as the main LaTeX file. For loading figures
%% from other directories you can use the `import` package
%%   \usepackage{import}
%% and then include the figures with
%%   \import{<path to file>}{<filename>.pgf}
%%
%% Matplotlib used the following preamble
%%   \usepackage{fontspec}
%%   \setmainfont{Times New Roman}
%%   \setsansfont{Lucida Grande}
%%   \setmonofont{Andale Mono}
%%
\begingroup%
\makeatletter%
\begin{pgfpicture}%
\pgfpathrectangle{\pgfpointorigin}{\pgfqpoint{5.000000in}{3.083333in}}%
\pgfusepath{use as bounding box, clip}%
\begin{pgfscope}%
\pgfsetbuttcap%
\pgfsetmiterjoin%
\definecolor{currentfill}{rgb}{1.000000,1.000000,1.000000}%
\pgfsetfillcolor{currentfill}%
\pgfsetlinewidth{0.000000pt}%
\definecolor{currentstroke}{rgb}{1.000000,1.000000,1.000000}%
\pgfsetstrokecolor{currentstroke}%
\pgfsetdash{}{0pt}%
\pgfpathmoveto{\pgfqpoint{0.000000in}{0.000000in}}%
\pgfpathlineto{\pgfqpoint{5.000000in}{0.000000in}}%
\pgfpathlineto{\pgfqpoint{5.000000in}{3.083333in}}%
\pgfpathlineto{\pgfqpoint{0.000000in}{3.083333in}}%
\pgfpathclose%
\pgfusepath{fill}%
\end{pgfscope}%
\begin{pgfscope}%
\pgfsetbuttcap%
\pgfsetmiterjoin%
\definecolor{currentfill}{rgb}{1.000000,1.000000,1.000000}%
\pgfsetfillcolor{currentfill}%
\pgfsetlinewidth{0.000000pt}%
\definecolor{currentstroke}{rgb}{0.000000,0.000000,0.000000}%
\pgfsetstrokecolor{currentstroke}%
\pgfsetstrokeopacity{0.000000}%
\pgfsetdash{}{0pt}%
\pgfpathmoveto{\pgfqpoint{0.612942in}{0.402778in}}%
\pgfpathlineto{\pgfqpoint{4.951389in}{0.402778in}}%
\pgfpathlineto{\pgfqpoint{4.951389in}{3.034722in}}%
\pgfpathlineto{\pgfqpoint{0.612942in}{3.034722in}}%
\pgfpathclose%
\pgfusepath{fill}%
\end{pgfscope}%
\begin{pgfscope}%
\pgfpathrectangle{\pgfqpoint{0.612942in}{0.402778in}}{\pgfqpoint{4.338447in}{2.631944in}}%
\pgfusepath{clip}%
\pgfsetbuttcap%
\pgfsetroundjoin%
\definecolor{currentfill}{rgb}{0.121569,0.466667,0.705882}%
\pgfsetfillcolor{currentfill}%
\pgfsetfillopacity{0.500000}%
\pgfsetlinewidth{1.003750pt}%
\definecolor{currentstroke}{rgb}{0.121569,0.466667,0.705882}%
\pgfsetstrokecolor{currentstroke}%
\pgfsetstrokeopacity{0.500000}%
\pgfsetdash{}{0pt}%
\pgfpathmoveto{\pgfqpoint{4.748600in}{2.827273in}}%
\pgfpathcurveto{\pgfqpoint{4.759650in}{2.827273in}}{\pgfqpoint{4.770249in}{2.831663in}}{\pgfqpoint{4.778062in}{2.839477in}}%
\pgfpathcurveto{\pgfqpoint{4.785876in}{2.847290in}}{\pgfqpoint{4.790266in}{2.857889in}}{\pgfqpoint{4.790266in}{2.868939in}}%
\pgfpathcurveto{\pgfqpoint{4.790266in}{2.879989in}}{\pgfqpoint{4.785876in}{2.890589in}}{\pgfqpoint{4.778062in}{2.898402in}}%
\pgfpathcurveto{\pgfqpoint{4.770249in}{2.906216in}}{\pgfqpoint{4.759650in}{2.910606in}}{\pgfqpoint{4.748600in}{2.910606in}}%
\pgfpathcurveto{\pgfqpoint{4.737550in}{2.910606in}}{\pgfqpoint{4.726951in}{2.906216in}}{\pgfqpoint{4.719137in}{2.898402in}}%
\pgfpathcurveto{\pgfqpoint{4.711323in}{2.890589in}}{\pgfqpoint{4.706933in}{2.879989in}}{\pgfqpoint{4.706933in}{2.868939in}}%
\pgfpathcurveto{\pgfqpoint{4.706933in}{2.857889in}}{\pgfqpoint{4.711323in}{2.847290in}}{\pgfqpoint{4.719137in}{2.839477in}}%
\pgfpathcurveto{\pgfqpoint{4.726951in}{2.831663in}}{\pgfqpoint{4.737550in}{2.827273in}}{\pgfqpoint{4.748600in}{2.827273in}}%
\pgfpathclose%
\pgfusepath{stroke,fill}%
\end{pgfscope}%
\begin{pgfscope}%
\pgfpathrectangle{\pgfqpoint{0.612942in}{0.402778in}}{\pgfqpoint{4.338447in}{2.631944in}}%
\pgfusepath{clip}%
\pgfsetbuttcap%
\pgfsetroundjoin%
\definecolor{currentfill}{rgb}{0.121569,0.466667,0.705882}%
\pgfsetfillcolor{currentfill}%
\pgfsetfillopacity{0.500000}%
\pgfsetlinewidth{1.003750pt}%
\definecolor{currentstroke}{rgb}{0.121569,0.466667,0.705882}%
\pgfsetstrokecolor{currentstroke}%
\pgfsetstrokeopacity{0.500000}%
\pgfsetdash{}{0pt}%
\pgfpathmoveto{\pgfqpoint{3.803729in}{2.592726in}}%
\pgfpathcurveto{\pgfqpoint{3.814779in}{2.592726in}}{\pgfqpoint{3.825378in}{2.597116in}}{\pgfqpoint{3.833191in}{2.604930in}}%
\pgfpathcurveto{\pgfqpoint{3.841005in}{2.612744in}}{\pgfqpoint{3.845395in}{2.623343in}}{\pgfqpoint{3.845395in}{2.634393in}}%
\pgfpathcurveto{\pgfqpoint{3.845395in}{2.645443in}}{\pgfqpoint{3.841005in}{2.656042in}}{\pgfqpoint{3.833191in}{2.663855in}}%
\pgfpathcurveto{\pgfqpoint{3.825378in}{2.671669in}}{\pgfqpoint{3.814779in}{2.676059in}}{\pgfqpoint{3.803729in}{2.676059in}}%
\pgfpathcurveto{\pgfqpoint{3.792678in}{2.676059in}}{\pgfqpoint{3.782079in}{2.671669in}}{\pgfqpoint{3.774266in}{2.663855in}}%
\pgfpathcurveto{\pgfqpoint{3.766452in}{2.656042in}}{\pgfqpoint{3.762062in}{2.645443in}}{\pgfqpoint{3.762062in}{2.634393in}}%
\pgfpathcurveto{\pgfqpoint{3.762062in}{2.623343in}}{\pgfqpoint{3.766452in}{2.612744in}}{\pgfqpoint{3.774266in}{2.604930in}}%
\pgfpathcurveto{\pgfqpoint{3.782079in}{2.597116in}}{\pgfqpoint{3.792678in}{2.592726in}}{\pgfqpoint{3.803729in}{2.592726in}}%
\pgfpathclose%
\pgfusepath{stroke,fill}%
\end{pgfscope}%
\begin{pgfscope}%
\pgfpathrectangle{\pgfqpoint{0.612942in}{0.402778in}}{\pgfqpoint{4.338447in}{2.631944in}}%
\pgfusepath{clip}%
\pgfsetbuttcap%
\pgfsetroundjoin%
\definecolor{currentfill}{rgb}{0.121569,0.466667,0.705882}%
\pgfsetfillcolor{currentfill}%
\pgfsetfillopacity{0.500000}%
\pgfsetlinewidth{1.003750pt}%
\definecolor{currentstroke}{rgb}{0.121569,0.466667,0.705882}%
\pgfsetstrokecolor{currentstroke}%
\pgfsetstrokeopacity{0.500000}%
\pgfsetdash{}{0pt}%
\pgfpathmoveto{\pgfqpoint{2.999350in}{2.344076in}}%
\pgfpathcurveto{\pgfqpoint{3.010400in}{2.344076in}}{\pgfqpoint{3.020999in}{2.348467in}}{\pgfqpoint{3.028813in}{2.356280in}}%
\pgfpathcurveto{\pgfqpoint{3.036626in}{2.364094in}}{\pgfqpoint{3.041017in}{2.374693in}}{\pgfqpoint{3.041017in}{2.385743in}}%
\pgfpathcurveto{\pgfqpoint{3.041017in}{2.396793in}}{\pgfqpoint{3.036626in}{2.407392in}}{\pgfqpoint{3.028813in}{2.415206in}}%
\pgfpathcurveto{\pgfqpoint{3.020999in}{2.423020in}}{\pgfqpoint{3.010400in}{2.427410in}}{\pgfqpoint{2.999350in}{2.427410in}}%
\pgfpathcurveto{\pgfqpoint{2.988300in}{2.427410in}}{\pgfqpoint{2.977701in}{2.423020in}}{\pgfqpoint{2.969887in}{2.415206in}}%
\pgfpathcurveto{\pgfqpoint{2.962074in}{2.407392in}}{\pgfqpoint{2.957683in}{2.396793in}}{\pgfqpoint{2.957683in}{2.385743in}}%
\pgfpathcurveto{\pgfqpoint{2.957683in}{2.374693in}}{\pgfqpoint{2.962074in}{2.364094in}}{\pgfqpoint{2.969887in}{2.356280in}}%
\pgfpathcurveto{\pgfqpoint{2.977701in}{2.348467in}}{\pgfqpoint{2.988300in}{2.344076in}}{\pgfqpoint{2.999350in}{2.344076in}}%
\pgfpathclose%
\pgfusepath{stroke,fill}%
\end{pgfscope}%
\begin{pgfscope}%
\pgfpathrectangle{\pgfqpoint{0.612942in}{0.402778in}}{\pgfqpoint{4.338447in}{2.631944in}}%
\pgfusepath{clip}%
\pgfsetbuttcap%
\pgfsetroundjoin%
\definecolor{currentfill}{rgb}{0.121569,0.466667,0.705882}%
\pgfsetfillcolor{currentfill}%
\pgfsetfillopacity{0.500000}%
\pgfsetlinewidth{1.003750pt}%
\definecolor{currentstroke}{rgb}{0.121569,0.466667,0.705882}%
\pgfsetstrokecolor{currentstroke}%
\pgfsetstrokeopacity{0.500000}%
\pgfsetdash{}{0pt}%
\pgfpathmoveto{\pgfqpoint{2.319473in}{2.108142in}}%
\pgfpathcurveto{\pgfqpoint{2.330523in}{2.108142in}}{\pgfqpoint{2.341122in}{2.112532in}}{\pgfqpoint{2.348935in}{2.120346in}}%
\pgfpathcurveto{\pgfqpoint{2.356749in}{2.128159in}}{\pgfqpoint{2.361139in}{2.138758in}}{\pgfqpoint{2.361139in}{2.149809in}}%
\pgfpathcurveto{\pgfqpoint{2.361139in}{2.160859in}}{\pgfqpoint{2.356749in}{2.171458in}}{\pgfqpoint{2.348935in}{2.179271in}}%
\pgfpathcurveto{\pgfqpoint{2.341122in}{2.187085in}}{\pgfqpoint{2.330523in}{2.191475in}}{\pgfqpoint{2.319473in}{2.191475in}}%
\pgfpathcurveto{\pgfqpoint{2.308422in}{2.191475in}}{\pgfqpoint{2.297823in}{2.187085in}}{\pgfqpoint{2.290010in}{2.179271in}}%
\pgfpathcurveto{\pgfqpoint{2.282196in}{2.171458in}}{\pgfqpoint{2.277806in}{2.160859in}}{\pgfqpoint{2.277806in}{2.149809in}}%
\pgfpathcurveto{\pgfqpoint{2.277806in}{2.138758in}}{\pgfqpoint{2.282196in}{2.128159in}}{\pgfqpoint{2.290010in}{2.120346in}}%
\pgfpathcurveto{\pgfqpoint{2.297823in}{2.112532in}}{\pgfqpoint{2.308422in}{2.108142in}}{\pgfqpoint{2.319473in}{2.108142in}}%
\pgfpathclose%
\pgfusepath{stroke,fill}%
\end{pgfscope}%
\begin{pgfscope}%
\pgfpathrectangle{\pgfqpoint{0.612942in}{0.402778in}}{\pgfqpoint{4.338447in}{2.631944in}}%
\pgfusepath{clip}%
\pgfsetbuttcap%
\pgfsetroundjoin%
\definecolor{currentfill}{rgb}{0.121569,0.466667,0.705882}%
\pgfsetfillcolor{currentfill}%
\pgfsetfillopacity{0.500000}%
\pgfsetlinewidth{1.003750pt}%
\definecolor{currentstroke}{rgb}{0.121569,0.466667,0.705882}%
\pgfsetstrokecolor{currentstroke}%
\pgfsetstrokeopacity{0.500000}%
\pgfsetdash{}{0pt}%
\pgfpathmoveto{\pgfqpoint{2.271149in}{1.854184in}}%
\pgfpathcurveto{\pgfqpoint{2.282200in}{1.854184in}}{\pgfqpoint{2.292799in}{1.858575in}}{\pgfqpoint{2.300612in}{1.866388in}}%
\pgfpathcurveto{\pgfqpoint{2.308426in}{1.874202in}}{\pgfqpoint{2.312816in}{1.884801in}}{\pgfqpoint{2.312816in}{1.895851in}}%
\pgfpathcurveto{\pgfqpoint{2.312816in}{1.906901in}}{\pgfqpoint{2.308426in}{1.917500in}}{\pgfqpoint{2.300612in}{1.925314in}}%
\pgfpathcurveto{\pgfqpoint{2.292799in}{1.933128in}}{\pgfqpoint{2.282200in}{1.937518in}}{\pgfqpoint{2.271149in}{1.937518in}}%
\pgfpathcurveto{\pgfqpoint{2.260099in}{1.937518in}}{\pgfqpoint{2.249500in}{1.933128in}}{\pgfqpoint{2.241687in}{1.925314in}}%
\pgfpathcurveto{\pgfqpoint{2.233873in}{1.917500in}}{\pgfqpoint{2.229483in}{1.906901in}}{\pgfqpoint{2.229483in}{1.895851in}}%
\pgfpathcurveto{\pgfqpoint{2.229483in}{1.884801in}}{\pgfqpoint{2.233873in}{1.874202in}}{\pgfqpoint{2.241687in}{1.866388in}}%
\pgfpathcurveto{\pgfqpoint{2.249500in}{1.858575in}}{\pgfqpoint{2.260099in}{1.854184in}}{\pgfqpoint{2.271149in}{1.854184in}}%
\pgfpathclose%
\pgfusepath{stroke,fill}%
\end{pgfscope}%
\begin{pgfscope}%
\pgfpathrectangle{\pgfqpoint{0.612942in}{0.402778in}}{\pgfqpoint{4.338447in}{2.631944in}}%
\pgfusepath{clip}%
\pgfsetbuttcap%
\pgfsetroundjoin%
\definecolor{currentfill}{rgb}{0.121569,0.466667,0.705882}%
\pgfsetfillcolor{currentfill}%
\pgfsetfillopacity{0.500000}%
\pgfsetlinewidth{1.003750pt}%
\definecolor{currentstroke}{rgb}{0.121569,0.466667,0.705882}%
\pgfsetstrokecolor{currentstroke}%
\pgfsetstrokeopacity{0.500000}%
\pgfsetdash{}{0pt}%
\pgfpathmoveto{\pgfqpoint{2.270770in}{1.593318in}}%
\pgfpathcurveto{\pgfqpoint{2.281820in}{1.593318in}}{\pgfqpoint{2.292419in}{1.597708in}}{\pgfqpoint{2.300232in}{1.605522in}}%
\pgfpathcurveto{\pgfqpoint{2.308046in}{1.613336in}}{\pgfqpoint{2.312436in}{1.623935in}}{\pgfqpoint{2.312436in}{1.634985in}}%
\pgfpathcurveto{\pgfqpoint{2.312436in}{1.646035in}}{\pgfqpoint{2.308046in}{1.656634in}}{\pgfqpoint{2.300232in}{1.664447in}}%
\pgfpathcurveto{\pgfqpoint{2.292419in}{1.672261in}}{\pgfqpoint{2.281820in}{1.676651in}}{\pgfqpoint{2.270770in}{1.676651in}}%
\pgfpathcurveto{\pgfqpoint{2.259719in}{1.676651in}}{\pgfqpoint{2.249120in}{1.672261in}}{\pgfqpoint{2.241307in}{1.664447in}}%
\pgfpathcurveto{\pgfqpoint{2.233493in}{1.656634in}}{\pgfqpoint{2.229103in}{1.646035in}}{\pgfqpoint{2.229103in}{1.634985in}}%
\pgfpathcurveto{\pgfqpoint{2.229103in}{1.623935in}}{\pgfqpoint{2.233493in}{1.613336in}}{\pgfqpoint{2.241307in}{1.605522in}}%
\pgfpathcurveto{\pgfqpoint{2.249120in}{1.597708in}}{\pgfqpoint{2.259719in}{1.593318in}}{\pgfqpoint{2.270770in}{1.593318in}}%
\pgfpathclose%
\pgfusepath{stroke,fill}%
\end{pgfscope}%
\begin{pgfscope}%
\pgfpathrectangle{\pgfqpoint{0.612942in}{0.402778in}}{\pgfqpoint{4.338447in}{2.631944in}}%
\pgfusepath{clip}%
\pgfsetbuttcap%
\pgfsetroundjoin%
\definecolor{currentfill}{rgb}{0.121569,0.466667,0.705882}%
\pgfsetfillcolor{currentfill}%
\pgfsetfillopacity{0.500000}%
\pgfsetlinewidth{1.003750pt}%
\definecolor{currentstroke}{rgb}{0.121569,0.466667,0.705882}%
\pgfsetstrokecolor{currentstroke}%
\pgfsetstrokeopacity{0.500000}%
\pgfsetdash{}{0pt}%
\pgfpathmoveto{\pgfqpoint{2.270772in}{1.287837in}}%
\pgfpathcurveto{\pgfqpoint{2.281822in}{1.287837in}}{\pgfqpoint{2.292421in}{1.292228in}}{\pgfqpoint{2.300235in}{1.300041in}}%
\pgfpathcurveto{\pgfqpoint{2.308048in}{1.307855in}}{\pgfqpoint{2.312438in}{1.318454in}}{\pgfqpoint{2.312438in}{1.329504in}}%
\pgfpathcurveto{\pgfqpoint{2.312438in}{1.340554in}}{\pgfqpoint{2.308048in}{1.351153in}}{\pgfqpoint{2.300235in}{1.358967in}}%
\pgfpathcurveto{\pgfqpoint{2.292421in}{1.366780in}}{\pgfqpoint{2.281822in}{1.371171in}}{\pgfqpoint{2.270772in}{1.371171in}}%
\pgfpathcurveto{\pgfqpoint{2.259722in}{1.371171in}}{\pgfqpoint{2.249123in}{1.366780in}}{\pgfqpoint{2.241309in}{1.358967in}}%
\pgfpathcurveto{\pgfqpoint{2.233495in}{1.351153in}}{\pgfqpoint{2.229105in}{1.340554in}}{\pgfqpoint{2.229105in}{1.329504in}}%
\pgfpathcurveto{\pgfqpoint{2.229105in}{1.318454in}}{\pgfqpoint{2.233495in}{1.307855in}}{\pgfqpoint{2.241309in}{1.300041in}}%
\pgfpathcurveto{\pgfqpoint{2.249123in}{1.292228in}}{\pgfqpoint{2.259722in}{1.287837in}}{\pgfqpoint{2.270772in}{1.287837in}}%
\pgfpathclose%
\pgfusepath{stroke,fill}%
\end{pgfscope}%
\begin{pgfscope}%
\pgfpathrectangle{\pgfqpoint{0.612942in}{0.402778in}}{\pgfqpoint{4.338447in}{2.631944in}}%
\pgfusepath{clip}%
\pgfsetbuttcap%
\pgfsetroundjoin%
\definecolor{currentfill}{rgb}{0.121569,0.466667,0.705882}%
\pgfsetfillcolor{currentfill}%
\pgfsetfillopacity{0.500000}%
\pgfsetlinewidth{1.003750pt}%
\definecolor{currentstroke}{rgb}{0.121569,0.466667,0.705882}%
\pgfsetstrokecolor{currentstroke}%
\pgfsetstrokeopacity{0.500000}%
\pgfsetdash{}{0pt}%
\pgfpathmoveto{\pgfqpoint{2.270772in}{0.526873in}}%
\pgfpathcurveto{\pgfqpoint{2.281822in}{0.526873in}}{\pgfqpoint{2.292421in}{0.531263in}}{\pgfqpoint{2.300235in}{0.539076in}}%
\pgfpathcurveto{\pgfqpoint{2.308048in}{0.546890in}}{\pgfqpoint{2.312438in}{0.557489in}}{\pgfqpoint{2.312438in}{0.568539in}}%
\pgfpathcurveto{\pgfqpoint{2.312438in}{0.579589in}}{\pgfqpoint{2.308048in}{0.590188in}}{\pgfqpoint{2.300235in}{0.598002in}}%
\pgfpathcurveto{\pgfqpoint{2.292421in}{0.605816in}}{\pgfqpoint{2.281822in}{0.610206in}}{\pgfqpoint{2.270772in}{0.610206in}}%
\pgfpathcurveto{\pgfqpoint{2.259722in}{0.610206in}}{\pgfqpoint{2.249123in}{0.605816in}}{\pgfqpoint{2.241309in}{0.598002in}}%
\pgfpathcurveto{\pgfqpoint{2.233495in}{0.590188in}}{\pgfqpoint{2.229105in}{0.579589in}}{\pgfqpoint{2.229105in}{0.568539in}}%
\pgfpathcurveto{\pgfqpoint{2.229105in}{0.557489in}}{\pgfqpoint{2.233495in}{0.546890in}}{\pgfqpoint{2.241309in}{0.539076in}}%
\pgfpathcurveto{\pgfqpoint{2.249123in}{0.531263in}}{\pgfqpoint{2.259722in}{0.526873in}}{\pgfqpoint{2.270772in}{0.526873in}}%
\pgfpathclose%
\pgfusepath{stroke,fill}%
\end{pgfscope}%
\begin{pgfscope}%
\pgfpathrectangle{\pgfqpoint{0.612942in}{0.402778in}}{\pgfqpoint{4.338447in}{2.631944in}}%
\pgfusepath{clip}%
\pgfsetbuttcap%
\pgfsetroundjoin%
\definecolor{currentfill}{rgb}{0.121569,0.466667,0.705882}%
\pgfsetfillcolor{currentfill}%
\pgfsetfillopacity{0.500000}%
\pgfsetlinewidth{1.003750pt}%
\definecolor{currentstroke}{rgb}{0.121569,0.466667,0.705882}%
\pgfsetstrokecolor{currentstroke}%
\pgfsetstrokeopacity{0.500000}%
\pgfsetdash{}{0pt}%
\pgfpathmoveto{\pgfqpoint{2.270772in}{0.526873in}}%
\pgfpathcurveto{\pgfqpoint{2.281822in}{0.526873in}}{\pgfqpoint{2.292421in}{0.531263in}}{\pgfqpoint{2.300235in}{0.539076in}}%
\pgfpathcurveto{\pgfqpoint{2.308048in}{0.546890in}}{\pgfqpoint{2.312438in}{0.557489in}}{\pgfqpoint{2.312438in}{0.568539in}}%
\pgfpathcurveto{\pgfqpoint{2.312438in}{0.579589in}}{\pgfqpoint{2.308048in}{0.590188in}}{\pgfqpoint{2.300235in}{0.598002in}}%
\pgfpathcurveto{\pgfqpoint{2.292421in}{0.605816in}}{\pgfqpoint{2.281822in}{0.610206in}}{\pgfqpoint{2.270772in}{0.610206in}}%
\pgfpathcurveto{\pgfqpoint{2.259722in}{0.610206in}}{\pgfqpoint{2.249123in}{0.605816in}}{\pgfqpoint{2.241309in}{0.598002in}}%
\pgfpathcurveto{\pgfqpoint{2.233495in}{0.590188in}}{\pgfqpoint{2.229105in}{0.579589in}}{\pgfqpoint{2.229105in}{0.568539in}}%
\pgfpathcurveto{\pgfqpoint{2.229105in}{0.557489in}}{\pgfqpoint{2.233495in}{0.546890in}}{\pgfqpoint{2.241309in}{0.539076in}}%
\pgfpathcurveto{\pgfqpoint{2.249123in}{0.531263in}}{\pgfqpoint{2.259722in}{0.526873in}}{\pgfqpoint{2.270772in}{0.526873in}}%
\pgfpathclose%
\pgfusepath{stroke,fill}%
\end{pgfscope}%
\begin{pgfscope}%
\pgfpathrectangle{\pgfqpoint{0.612942in}{0.402778in}}{\pgfqpoint{4.338447in}{2.631944in}}%
\pgfusepath{clip}%
\pgfsetbuttcap%
\pgfsetroundjoin%
\definecolor{currentfill}{rgb}{0.121569,0.466667,0.705882}%
\pgfsetfillcolor{currentfill}%
\pgfsetfillopacity{0.500000}%
\pgfsetlinewidth{1.003750pt}%
\definecolor{currentstroke}{rgb}{0.121569,0.466667,0.705882}%
\pgfsetstrokecolor{currentstroke}%
\pgfsetstrokeopacity{0.500000}%
\pgfsetdash{}{0pt}%
\pgfpathmoveto{\pgfqpoint{2.270772in}{0.526873in}}%
\pgfpathcurveto{\pgfqpoint{2.281822in}{0.526873in}}{\pgfqpoint{2.292421in}{0.531263in}}{\pgfqpoint{2.300235in}{0.539076in}}%
\pgfpathcurveto{\pgfqpoint{2.308048in}{0.546890in}}{\pgfqpoint{2.312438in}{0.557489in}}{\pgfqpoint{2.312438in}{0.568539in}}%
\pgfpathcurveto{\pgfqpoint{2.312438in}{0.579589in}}{\pgfqpoint{2.308048in}{0.590188in}}{\pgfqpoint{2.300235in}{0.598002in}}%
\pgfpathcurveto{\pgfqpoint{2.292421in}{0.605816in}}{\pgfqpoint{2.281822in}{0.610206in}}{\pgfqpoint{2.270772in}{0.610206in}}%
\pgfpathcurveto{\pgfqpoint{2.259722in}{0.610206in}}{\pgfqpoint{2.249123in}{0.605816in}}{\pgfqpoint{2.241309in}{0.598002in}}%
\pgfpathcurveto{\pgfqpoint{2.233495in}{0.590188in}}{\pgfqpoint{2.229105in}{0.579589in}}{\pgfqpoint{2.229105in}{0.568539in}}%
\pgfpathcurveto{\pgfqpoint{2.229105in}{0.557489in}}{\pgfqpoint{2.233495in}{0.546890in}}{\pgfqpoint{2.241309in}{0.539076in}}%
\pgfpathcurveto{\pgfqpoint{2.249123in}{0.531263in}}{\pgfqpoint{2.259722in}{0.526873in}}{\pgfqpoint{2.270772in}{0.526873in}}%
\pgfpathclose%
\pgfusepath{stroke,fill}%
\end{pgfscope}%
\begin{pgfscope}%
\pgfpathrectangle{\pgfqpoint{0.612942in}{0.402778in}}{\pgfqpoint{4.338447in}{2.631944in}}%
\pgfusepath{clip}%
\pgfsetbuttcap%
\pgfsetroundjoin%
\definecolor{currentfill}{rgb}{1.000000,0.498039,0.054902}%
\pgfsetfillcolor{currentfill}%
\pgfsetfillopacity{0.500000}%
\pgfsetlinewidth{1.003750pt}%
\definecolor{currentstroke}{rgb}{1.000000,0.498039,0.054902}%
\pgfsetstrokecolor{currentstroke}%
\pgfsetstrokeopacity{0.500000}%
\pgfsetdash{}{0pt}%
\pgfpathmoveto{\pgfqpoint{4.606463in}{2.827280in}}%
\pgfpathcurveto{\pgfqpoint{4.617513in}{2.827280in}}{\pgfqpoint{4.628112in}{2.831670in}}{\pgfqpoint{4.635926in}{2.839484in}}%
\pgfpathcurveto{\pgfqpoint{4.643739in}{2.847298in}}{\pgfqpoint{4.648130in}{2.857897in}}{\pgfqpoint{4.648130in}{2.868947in}}%
\pgfpathcurveto{\pgfqpoint{4.648130in}{2.879997in}}{\pgfqpoint{4.643739in}{2.890596in}}{\pgfqpoint{4.635926in}{2.898410in}}%
\pgfpathcurveto{\pgfqpoint{4.628112in}{2.906223in}}{\pgfqpoint{4.617513in}{2.910613in}}{\pgfqpoint{4.606463in}{2.910613in}}%
\pgfpathcurveto{\pgfqpoint{4.595413in}{2.910613in}}{\pgfqpoint{4.584814in}{2.906223in}}{\pgfqpoint{4.577000in}{2.898410in}}%
\pgfpathcurveto{\pgfqpoint{4.569187in}{2.890596in}}{\pgfqpoint{4.564796in}{2.879997in}}{\pgfqpoint{4.564796in}{2.868947in}}%
\pgfpathcurveto{\pgfqpoint{4.564796in}{2.857897in}}{\pgfqpoint{4.569187in}{2.847298in}}{\pgfqpoint{4.577000in}{2.839484in}}%
\pgfpathcurveto{\pgfqpoint{4.584814in}{2.831670in}}{\pgfqpoint{4.595413in}{2.827280in}}{\pgfqpoint{4.606463in}{2.827280in}}%
\pgfpathclose%
\pgfusepath{stroke,fill}%
\end{pgfscope}%
\begin{pgfscope}%
\pgfpathrectangle{\pgfqpoint{0.612942in}{0.402778in}}{\pgfqpoint{4.338447in}{2.631944in}}%
\pgfusepath{clip}%
\pgfsetbuttcap%
\pgfsetroundjoin%
\definecolor{currentfill}{rgb}{1.000000,0.498039,0.054902}%
\pgfsetfillcolor{currentfill}%
\pgfsetfillopacity{0.500000}%
\pgfsetlinewidth{1.003750pt}%
\definecolor{currentstroke}{rgb}{1.000000,0.498039,0.054902}%
\pgfsetstrokecolor{currentstroke}%
\pgfsetstrokeopacity{0.500000}%
\pgfsetdash{}{0pt}%
\pgfpathmoveto{\pgfqpoint{3.661664in}{2.592758in}}%
\pgfpathcurveto{\pgfqpoint{3.672714in}{2.592758in}}{\pgfqpoint{3.683313in}{2.597148in}}{\pgfqpoint{3.691127in}{2.604962in}}%
\pgfpathcurveto{\pgfqpoint{3.698940in}{2.612776in}}{\pgfqpoint{3.703331in}{2.623375in}}{\pgfqpoint{3.703331in}{2.634425in}}%
\pgfpathcurveto{\pgfqpoint{3.703331in}{2.645475in}}{\pgfqpoint{3.698940in}{2.656074in}}{\pgfqpoint{3.691127in}{2.663887in}}%
\pgfpathcurveto{\pgfqpoint{3.683313in}{2.671701in}}{\pgfqpoint{3.672714in}{2.676091in}}{\pgfqpoint{3.661664in}{2.676091in}}%
\pgfpathcurveto{\pgfqpoint{3.650614in}{2.676091in}}{\pgfqpoint{3.640015in}{2.671701in}}{\pgfqpoint{3.632201in}{2.663887in}}%
\pgfpathcurveto{\pgfqpoint{3.624387in}{2.656074in}}{\pgfqpoint{3.619997in}{2.645475in}}{\pgfqpoint{3.619997in}{2.634425in}}%
\pgfpathcurveto{\pgfqpoint{3.619997in}{2.623375in}}{\pgfqpoint{3.624387in}{2.612776in}}{\pgfqpoint{3.632201in}{2.604962in}}%
\pgfpathcurveto{\pgfqpoint{3.640015in}{2.597148in}}{\pgfqpoint{3.650614in}{2.592758in}}{\pgfqpoint{3.661664in}{2.592758in}}%
\pgfpathclose%
\pgfusepath{stroke,fill}%
\end{pgfscope}%
\begin{pgfscope}%
\pgfpathrectangle{\pgfqpoint{0.612942in}{0.402778in}}{\pgfqpoint{4.338447in}{2.631944in}}%
\pgfusepath{clip}%
\pgfsetbuttcap%
\pgfsetroundjoin%
\definecolor{currentfill}{rgb}{1.000000,0.498039,0.054902}%
\pgfsetfillcolor{currentfill}%
\pgfsetfillopacity{0.500000}%
\pgfsetlinewidth{1.003750pt}%
\definecolor{currentstroke}{rgb}{1.000000,0.498039,0.054902}%
\pgfsetstrokecolor{currentstroke}%
\pgfsetstrokeopacity{0.500000}%
\pgfsetdash{}{0pt}%
\pgfpathmoveto{\pgfqpoint{2.861334in}{2.344064in}}%
\pgfpathcurveto{\pgfqpoint{2.872385in}{2.344064in}}{\pgfqpoint{2.882984in}{2.348454in}}{\pgfqpoint{2.890797in}{2.356268in}}%
\pgfpathcurveto{\pgfqpoint{2.898611in}{2.364081in}}{\pgfqpoint{2.903001in}{2.374680in}}{\pgfqpoint{2.903001in}{2.385730in}}%
\pgfpathcurveto{\pgfqpoint{2.903001in}{2.396780in}}{\pgfqpoint{2.898611in}{2.407380in}}{\pgfqpoint{2.890797in}{2.415193in}}%
\pgfpathcurveto{\pgfqpoint{2.882984in}{2.423007in}}{\pgfqpoint{2.872385in}{2.427397in}}{\pgfqpoint{2.861334in}{2.427397in}}%
\pgfpathcurveto{\pgfqpoint{2.850284in}{2.427397in}}{\pgfqpoint{2.839685in}{2.423007in}}{\pgfqpoint{2.831872in}{2.415193in}}%
\pgfpathcurveto{\pgfqpoint{2.824058in}{2.407380in}}{\pgfqpoint{2.819668in}{2.396780in}}{\pgfqpoint{2.819668in}{2.385730in}}%
\pgfpathcurveto{\pgfqpoint{2.819668in}{2.374680in}}{\pgfqpoint{2.824058in}{2.364081in}}{\pgfqpoint{2.831872in}{2.356268in}}%
\pgfpathcurveto{\pgfqpoint{2.839685in}{2.348454in}}{\pgfqpoint{2.850284in}{2.344064in}}{\pgfqpoint{2.861334in}{2.344064in}}%
\pgfpathclose%
\pgfusepath{stroke,fill}%
\end{pgfscope}%
\begin{pgfscope}%
\pgfpathrectangle{\pgfqpoint{0.612942in}{0.402778in}}{\pgfqpoint{4.338447in}{2.631944in}}%
\pgfusepath{clip}%
\pgfsetbuttcap%
\pgfsetroundjoin%
\definecolor{currentfill}{rgb}{1.000000,0.498039,0.054902}%
\pgfsetfillcolor{currentfill}%
\pgfsetfillopacity{0.500000}%
\pgfsetlinewidth{1.003750pt}%
\definecolor{currentstroke}{rgb}{1.000000,0.498039,0.054902}%
\pgfsetstrokecolor{currentstroke}%
\pgfsetstrokeopacity{0.500000}%
\pgfsetdash{}{0pt}%
\pgfpathmoveto{\pgfqpoint{1.946331in}{2.108262in}}%
\pgfpathcurveto{\pgfqpoint{1.957381in}{2.108262in}}{\pgfqpoint{1.967980in}{2.112652in}}{\pgfqpoint{1.975794in}{2.120465in}}%
\pgfpathcurveto{\pgfqpoint{1.983607in}{2.128279in}}{\pgfqpoint{1.987998in}{2.138878in}}{\pgfqpoint{1.987998in}{2.149928in}}%
\pgfpathcurveto{\pgfqpoint{1.987998in}{2.160978in}}{\pgfqpoint{1.983607in}{2.171577in}}{\pgfqpoint{1.975794in}{2.179391in}}%
\pgfpathcurveto{\pgfqpoint{1.967980in}{2.187205in}}{\pgfqpoint{1.957381in}{2.191595in}}{\pgfqpoint{1.946331in}{2.191595in}}%
\pgfpathcurveto{\pgfqpoint{1.935281in}{2.191595in}}{\pgfqpoint{1.924682in}{2.187205in}}{\pgfqpoint{1.916868in}{2.179391in}}%
\pgfpathcurveto{\pgfqpoint{1.909054in}{2.171577in}}{\pgfqpoint{1.904664in}{2.160978in}}{\pgfqpoint{1.904664in}{2.149928in}}%
\pgfpathcurveto{\pgfqpoint{1.904664in}{2.138878in}}{\pgfqpoint{1.909054in}{2.128279in}}{\pgfqpoint{1.916868in}{2.120465in}}%
\pgfpathcurveto{\pgfqpoint{1.924682in}{2.112652in}}{\pgfqpoint{1.935281in}{2.108262in}}{\pgfqpoint{1.946331in}{2.108262in}}%
\pgfpathclose%
\pgfusepath{stroke,fill}%
\end{pgfscope}%
\begin{pgfscope}%
\pgfpathrectangle{\pgfqpoint{0.612942in}{0.402778in}}{\pgfqpoint{4.338447in}{2.631944in}}%
\pgfusepath{clip}%
\pgfsetbuttcap%
\pgfsetroundjoin%
\definecolor{currentfill}{rgb}{1.000000,0.498039,0.054902}%
\pgfsetfillcolor{currentfill}%
\pgfsetfillopacity{0.500000}%
\pgfsetlinewidth{1.003750pt}%
\definecolor{currentstroke}{rgb}{1.000000,0.498039,0.054902}%
\pgfsetstrokecolor{currentstroke}%
\pgfsetstrokeopacity{0.500000}%
\pgfsetdash{}{0pt}%
\pgfpathmoveto{\pgfqpoint{1.565928in}{1.854215in}}%
\pgfpathcurveto{\pgfqpoint{1.576978in}{1.854215in}}{\pgfqpoint{1.587577in}{1.858605in}}{\pgfqpoint{1.595391in}{1.866419in}}%
\pgfpathcurveto{\pgfqpoint{1.603204in}{1.874233in}}{\pgfqpoint{1.607595in}{1.884832in}}{\pgfqpoint{1.607595in}{1.895882in}}%
\pgfpathcurveto{\pgfqpoint{1.607595in}{1.906932in}}{\pgfqpoint{1.603204in}{1.917531in}}{\pgfqpoint{1.595391in}{1.925345in}}%
\pgfpathcurveto{\pgfqpoint{1.587577in}{1.933158in}}{\pgfqpoint{1.576978in}{1.937548in}}{\pgfqpoint{1.565928in}{1.937548in}}%
\pgfpathcurveto{\pgfqpoint{1.554878in}{1.937548in}}{\pgfqpoint{1.544279in}{1.933158in}}{\pgfqpoint{1.536465in}{1.925345in}}%
\pgfpathcurveto{\pgfqpoint{1.528652in}{1.917531in}}{\pgfqpoint{1.524261in}{1.906932in}}{\pgfqpoint{1.524261in}{1.895882in}}%
\pgfpathcurveto{\pgfqpoint{1.524261in}{1.884832in}}{\pgfqpoint{1.528652in}{1.874233in}}{\pgfqpoint{1.536465in}{1.866419in}}%
\pgfpathcurveto{\pgfqpoint{1.544279in}{1.858605in}}{\pgfqpoint{1.554878in}{1.854215in}}{\pgfqpoint{1.565928in}{1.854215in}}%
\pgfpathclose%
\pgfusepath{stroke,fill}%
\end{pgfscope}%
\begin{pgfscope}%
\pgfpathrectangle{\pgfqpoint{0.612942in}{0.402778in}}{\pgfqpoint{4.338447in}{2.631944in}}%
\pgfusepath{clip}%
\pgfsetbuttcap%
\pgfsetroundjoin%
\definecolor{currentfill}{rgb}{1.000000,0.498039,0.054902}%
\pgfsetfillcolor{currentfill}%
\pgfsetfillopacity{0.500000}%
\pgfsetlinewidth{1.003750pt}%
\definecolor{currentstroke}{rgb}{1.000000,0.498039,0.054902}%
\pgfsetstrokecolor{currentstroke}%
\pgfsetstrokeopacity{0.500000}%
\pgfsetdash{}{0pt}%
\pgfpathmoveto{\pgfqpoint{1.559784in}{1.593421in}}%
\pgfpathcurveto{\pgfqpoint{1.570834in}{1.593421in}}{\pgfqpoint{1.581433in}{1.597812in}}{\pgfqpoint{1.589246in}{1.605625in}}%
\pgfpathcurveto{\pgfqpoint{1.597060in}{1.613439in}}{\pgfqpoint{1.601450in}{1.624038in}}{\pgfqpoint{1.601450in}{1.635088in}}%
\pgfpathcurveto{\pgfqpoint{1.601450in}{1.646138in}}{\pgfqpoint{1.597060in}{1.656737in}}{\pgfqpoint{1.589246in}{1.664551in}}%
\pgfpathcurveto{\pgfqpoint{1.581433in}{1.672365in}}{\pgfqpoint{1.570834in}{1.676755in}}{\pgfqpoint{1.559784in}{1.676755in}}%
\pgfpathcurveto{\pgfqpoint{1.548733in}{1.676755in}}{\pgfqpoint{1.538134in}{1.672365in}}{\pgfqpoint{1.530321in}{1.664551in}}%
\pgfpathcurveto{\pgfqpoint{1.522507in}{1.656737in}}{\pgfqpoint{1.518117in}{1.646138in}}{\pgfqpoint{1.518117in}{1.635088in}}%
\pgfpathcurveto{\pgfqpoint{1.518117in}{1.624038in}}{\pgfqpoint{1.522507in}{1.613439in}}{\pgfqpoint{1.530321in}{1.605625in}}%
\pgfpathcurveto{\pgfqpoint{1.538134in}{1.597812in}}{\pgfqpoint{1.548733in}{1.593421in}}{\pgfqpoint{1.559784in}{1.593421in}}%
\pgfpathclose%
\pgfusepath{stroke,fill}%
\end{pgfscope}%
\begin{pgfscope}%
\pgfpathrectangle{\pgfqpoint{0.612942in}{0.402778in}}{\pgfqpoint{4.338447in}{2.631944in}}%
\pgfusepath{clip}%
\pgfsetbuttcap%
\pgfsetroundjoin%
\definecolor{currentfill}{rgb}{1.000000,0.498039,0.054902}%
\pgfsetfillcolor{currentfill}%
\pgfsetfillopacity{0.500000}%
\pgfsetlinewidth{1.003750pt}%
\definecolor{currentstroke}{rgb}{1.000000,0.498039,0.054902}%
\pgfsetstrokecolor{currentstroke}%
\pgfsetstrokeopacity{0.500000}%
\pgfsetdash{}{0pt}%
\pgfpathmoveto{\pgfqpoint{1.559820in}{1.288324in}}%
\pgfpathcurveto{\pgfqpoint{1.570870in}{1.288324in}}{\pgfqpoint{1.581469in}{1.292714in}}{\pgfqpoint{1.589283in}{1.300528in}}%
\pgfpathcurveto{\pgfqpoint{1.597096in}{1.308341in}}{\pgfqpoint{1.601487in}{1.318940in}}{\pgfqpoint{1.601487in}{1.329991in}}%
\pgfpathcurveto{\pgfqpoint{1.601487in}{1.341041in}}{\pgfqpoint{1.597096in}{1.351640in}}{\pgfqpoint{1.589283in}{1.359453in}}%
\pgfpathcurveto{\pgfqpoint{1.581469in}{1.367267in}}{\pgfqpoint{1.570870in}{1.371657in}}{\pgfqpoint{1.559820in}{1.371657in}}%
\pgfpathcurveto{\pgfqpoint{1.548770in}{1.371657in}}{\pgfqpoint{1.538171in}{1.367267in}}{\pgfqpoint{1.530357in}{1.359453in}}%
\pgfpathcurveto{\pgfqpoint{1.522544in}{1.351640in}}{\pgfqpoint{1.518153in}{1.341041in}}{\pgfqpoint{1.518153in}{1.329991in}}%
\pgfpathcurveto{\pgfqpoint{1.518153in}{1.318940in}}{\pgfqpoint{1.522544in}{1.308341in}}{\pgfqpoint{1.530357in}{1.300528in}}%
\pgfpathcurveto{\pgfqpoint{1.538171in}{1.292714in}}{\pgfqpoint{1.548770in}{1.288324in}}{\pgfqpoint{1.559820in}{1.288324in}}%
\pgfpathclose%
\pgfusepath{stroke,fill}%
\end{pgfscope}%
\begin{pgfscope}%
\pgfpathrectangle{\pgfqpoint{0.612942in}{0.402778in}}{\pgfqpoint{4.338447in}{2.631944in}}%
\pgfusepath{clip}%
\pgfsetbuttcap%
\pgfsetroundjoin%
\definecolor{currentfill}{rgb}{1.000000,0.498039,0.054902}%
\pgfsetfillcolor{currentfill}%
\pgfsetfillopacity{0.500000}%
\pgfsetlinewidth{1.003750pt}%
\definecolor{currentstroke}{rgb}{1.000000,0.498039,0.054902}%
\pgfsetstrokecolor{currentstroke}%
\pgfsetstrokeopacity{0.500000}%
\pgfsetdash{}{0pt}%
\pgfpathmoveto{\pgfqpoint{1.559820in}{1.032298in}}%
\pgfpathcurveto{\pgfqpoint{1.570870in}{1.032298in}}{\pgfqpoint{1.581469in}{1.036689in}}{\pgfqpoint{1.589283in}{1.044502in}}%
\pgfpathcurveto{\pgfqpoint{1.597096in}{1.052316in}}{\pgfqpoint{1.601487in}{1.062915in}}{\pgfqpoint{1.601487in}{1.073965in}}%
\pgfpathcurveto{\pgfqpoint{1.601487in}{1.085015in}}{\pgfqpoint{1.597096in}{1.095614in}}{\pgfqpoint{1.589283in}{1.103428in}}%
\pgfpathcurveto{\pgfqpoint{1.581469in}{1.111242in}}{\pgfqpoint{1.570870in}{1.115632in}}{\pgfqpoint{1.559820in}{1.115632in}}%
\pgfpathcurveto{\pgfqpoint{1.548770in}{1.115632in}}{\pgfqpoint{1.538171in}{1.111242in}}{\pgfqpoint{1.530357in}{1.103428in}}%
\pgfpathcurveto{\pgfqpoint{1.522544in}{1.095614in}}{\pgfqpoint{1.518153in}{1.085015in}}{\pgfqpoint{1.518153in}{1.073965in}}%
\pgfpathcurveto{\pgfqpoint{1.518153in}{1.062915in}}{\pgfqpoint{1.522544in}{1.052316in}}{\pgfqpoint{1.530357in}{1.044502in}}%
\pgfpathcurveto{\pgfqpoint{1.538171in}{1.036689in}}{\pgfqpoint{1.548770in}{1.032298in}}{\pgfqpoint{1.559820in}{1.032298in}}%
\pgfpathclose%
\pgfusepath{stroke,fill}%
\end{pgfscope}%
\begin{pgfscope}%
\pgfpathrectangle{\pgfqpoint{0.612942in}{0.402778in}}{\pgfqpoint{4.338447in}{2.631944in}}%
\pgfusepath{clip}%
\pgfsetbuttcap%
\pgfsetroundjoin%
\definecolor{currentfill}{rgb}{1.000000,0.498039,0.054902}%
\pgfsetfillcolor{currentfill}%
\pgfsetfillopacity{0.500000}%
\pgfsetlinewidth{1.003750pt}%
\definecolor{currentstroke}{rgb}{1.000000,0.498039,0.054902}%
\pgfsetstrokecolor{currentstroke}%
\pgfsetstrokeopacity{0.500000}%
\pgfsetdash{}{0pt}%
\pgfpathmoveto{\pgfqpoint{1.559820in}{0.526873in}}%
\pgfpathcurveto{\pgfqpoint{1.570870in}{0.526873in}}{\pgfqpoint{1.581469in}{0.531263in}}{\pgfqpoint{1.589283in}{0.539076in}}%
\pgfpathcurveto{\pgfqpoint{1.597096in}{0.546890in}}{\pgfqpoint{1.601487in}{0.557489in}}{\pgfqpoint{1.601487in}{0.568539in}}%
\pgfpathcurveto{\pgfqpoint{1.601487in}{0.579589in}}{\pgfqpoint{1.597096in}{0.590188in}}{\pgfqpoint{1.589283in}{0.598002in}}%
\pgfpathcurveto{\pgfqpoint{1.581469in}{0.605816in}}{\pgfqpoint{1.570870in}{0.610206in}}{\pgfqpoint{1.559820in}{0.610206in}}%
\pgfpathcurveto{\pgfqpoint{1.548770in}{0.610206in}}{\pgfqpoint{1.538171in}{0.605816in}}{\pgfqpoint{1.530357in}{0.598002in}}%
\pgfpathcurveto{\pgfqpoint{1.522544in}{0.590188in}}{\pgfqpoint{1.518153in}{0.579589in}}{\pgfqpoint{1.518153in}{0.568539in}}%
\pgfpathcurveto{\pgfqpoint{1.518153in}{0.557489in}}{\pgfqpoint{1.522544in}{0.546890in}}{\pgfqpoint{1.530357in}{0.539076in}}%
\pgfpathcurveto{\pgfqpoint{1.538171in}{0.531263in}}{\pgfqpoint{1.548770in}{0.526873in}}{\pgfqpoint{1.559820in}{0.526873in}}%
\pgfpathclose%
\pgfusepath{stroke,fill}%
\end{pgfscope}%
\begin{pgfscope}%
\pgfpathrectangle{\pgfqpoint{0.612942in}{0.402778in}}{\pgfqpoint{4.338447in}{2.631944in}}%
\pgfusepath{clip}%
\pgfsetbuttcap%
\pgfsetroundjoin%
\definecolor{currentfill}{rgb}{1.000000,0.498039,0.054902}%
\pgfsetfillcolor{currentfill}%
\pgfsetfillopacity{0.500000}%
\pgfsetlinewidth{1.003750pt}%
\definecolor{currentstroke}{rgb}{1.000000,0.498039,0.054902}%
\pgfsetstrokecolor{currentstroke}%
\pgfsetstrokeopacity{0.500000}%
\pgfsetdash{}{0pt}%
\pgfpathmoveto{\pgfqpoint{1.559820in}{0.526873in}}%
\pgfpathcurveto{\pgfqpoint{1.570870in}{0.526873in}}{\pgfqpoint{1.581469in}{0.531263in}}{\pgfqpoint{1.589283in}{0.539076in}}%
\pgfpathcurveto{\pgfqpoint{1.597096in}{0.546890in}}{\pgfqpoint{1.601487in}{0.557489in}}{\pgfqpoint{1.601487in}{0.568539in}}%
\pgfpathcurveto{\pgfqpoint{1.601487in}{0.579589in}}{\pgfqpoint{1.597096in}{0.590188in}}{\pgfqpoint{1.589283in}{0.598002in}}%
\pgfpathcurveto{\pgfqpoint{1.581469in}{0.605816in}}{\pgfqpoint{1.570870in}{0.610206in}}{\pgfqpoint{1.559820in}{0.610206in}}%
\pgfpathcurveto{\pgfqpoint{1.548770in}{0.610206in}}{\pgfqpoint{1.538171in}{0.605816in}}{\pgfqpoint{1.530357in}{0.598002in}}%
\pgfpathcurveto{\pgfqpoint{1.522544in}{0.590188in}}{\pgfqpoint{1.518153in}{0.579589in}}{\pgfqpoint{1.518153in}{0.568539in}}%
\pgfpathcurveto{\pgfqpoint{1.518153in}{0.557489in}}{\pgfqpoint{1.522544in}{0.546890in}}{\pgfqpoint{1.530357in}{0.539076in}}%
\pgfpathcurveto{\pgfqpoint{1.538171in}{0.531263in}}{\pgfqpoint{1.548770in}{0.526873in}}{\pgfqpoint{1.559820in}{0.526873in}}%
\pgfpathclose%
\pgfusepath{stroke,fill}%
\end{pgfscope}%
\begin{pgfscope}%
\pgfpathrectangle{\pgfqpoint{0.612942in}{0.402778in}}{\pgfqpoint{4.338447in}{2.631944in}}%
\pgfusepath{clip}%
\pgfsetbuttcap%
\pgfsetroundjoin%
\definecolor{currentfill}{rgb}{0.172549,0.627451,0.172549}%
\pgfsetfillcolor{currentfill}%
\pgfsetfillopacity{0.500000}%
\pgfsetlinewidth{1.003750pt}%
\definecolor{currentstroke}{rgb}{0.172549,0.627451,0.172549}%
\pgfsetstrokecolor{currentstroke}%
\pgfsetstrokeopacity{0.500000}%
\pgfsetdash{}{0pt}%
\pgfpathmoveto{\pgfqpoint{4.464337in}{2.827294in}}%
\pgfpathcurveto{\pgfqpoint{4.475387in}{2.827294in}}{\pgfqpoint{4.485987in}{2.831684in}}{\pgfqpoint{4.493800in}{2.839498in}}%
\pgfpathcurveto{\pgfqpoint{4.501614in}{2.847312in}}{\pgfqpoint{4.506004in}{2.857911in}}{\pgfqpoint{4.506004in}{2.868961in}}%
\pgfpathcurveto{\pgfqpoint{4.506004in}{2.880011in}}{\pgfqpoint{4.501614in}{2.890610in}}{\pgfqpoint{4.493800in}{2.898424in}}%
\pgfpathcurveto{\pgfqpoint{4.485987in}{2.906237in}}{\pgfqpoint{4.475387in}{2.910627in}}{\pgfqpoint{4.464337in}{2.910627in}}%
\pgfpathcurveto{\pgfqpoint{4.453287in}{2.910627in}}{\pgfqpoint{4.442688in}{2.906237in}}{\pgfqpoint{4.434875in}{2.898424in}}%
\pgfpathcurveto{\pgfqpoint{4.427061in}{2.890610in}}{\pgfqpoint{4.422671in}{2.880011in}}{\pgfqpoint{4.422671in}{2.868961in}}%
\pgfpathcurveto{\pgfqpoint{4.422671in}{2.857911in}}{\pgfqpoint{4.427061in}{2.847312in}}{\pgfqpoint{4.434875in}{2.839498in}}%
\pgfpathcurveto{\pgfqpoint{4.442688in}{2.831684in}}{\pgfqpoint{4.453287in}{2.827294in}}{\pgfqpoint{4.464337in}{2.827294in}}%
\pgfpathclose%
\pgfusepath{stroke,fill}%
\end{pgfscope}%
\begin{pgfscope}%
\pgfpathrectangle{\pgfqpoint{0.612942in}{0.402778in}}{\pgfqpoint{4.338447in}{2.631944in}}%
\pgfusepath{clip}%
\pgfsetbuttcap%
\pgfsetroundjoin%
\definecolor{currentfill}{rgb}{0.172549,0.627451,0.172549}%
\pgfsetfillcolor{currentfill}%
\pgfsetfillopacity{0.500000}%
\pgfsetlinewidth{1.003750pt}%
\definecolor{currentstroke}{rgb}{0.172549,0.627451,0.172549}%
\pgfsetstrokecolor{currentstroke}%
\pgfsetstrokeopacity{0.500000}%
\pgfsetdash{}{0pt}%
\pgfpathmoveto{\pgfqpoint{3.519551in}{2.592798in}}%
\pgfpathcurveto{\pgfqpoint{3.530601in}{2.592798in}}{\pgfqpoint{3.541200in}{2.597188in}}{\pgfqpoint{3.549013in}{2.605001in}}%
\pgfpathcurveto{\pgfqpoint{3.556827in}{2.612815in}}{\pgfqpoint{3.561217in}{2.623414in}}{\pgfqpoint{3.561217in}{2.634464in}}%
\pgfpathcurveto{\pgfqpoint{3.561217in}{2.645514in}}{\pgfqpoint{3.556827in}{2.656113in}}{\pgfqpoint{3.549013in}{2.663927in}}%
\pgfpathcurveto{\pgfqpoint{3.541200in}{2.671741in}}{\pgfqpoint{3.530601in}{2.676131in}}{\pgfqpoint{3.519551in}{2.676131in}}%
\pgfpathcurveto{\pgfqpoint{3.508500in}{2.676131in}}{\pgfqpoint{3.497901in}{2.671741in}}{\pgfqpoint{3.490088in}{2.663927in}}%
\pgfpathcurveto{\pgfqpoint{3.482274in}{2.656113in}}{\pgfqpoint{3.477884in}{2.645514in}}{\pgfqpoint{3.477884in}{2.634464in}}%
\pgfpathcurveto{\pgfqpoint{3.477884in}{2.623414in}}{\pgfqpoint{3.482274in}{2.612815in}}{\pgfqpoint{3.490088in}{2.605001in}}%
\pgfpathcurveto{\pgfqpoint{3.497901in}{2.597188in}}{\pgfqpoint{3.508500in}{2.592798in}}{\pgfqpoint{3.519551in}{2.592798in}}%
\pgfpathclose%
\pgfusepath{stroke,fill}%
\end{pgfscope}%
\begin{pgfscope}%
\pgfpathrectangle{\pgfqpoint{0.612942in}{0.402778in}}{\pgfqpoint{4.338447in}{2.631944in}}%
\pgfusepath{clip}%
\pgfsetbuttcap%
\pgfsetroundjoin%
\definecolor{currentfill}{rgb}{0.172549,0.627451,0.172549}%
\pgfsetfillcolor{currentfill}%
\pgfsetfillopacity{0.500000}%
\pgfsetlinewidth{1.003750pt}%
\definecolor{currentstroke}{rgb}{0.172549,0.627451,0.172549}%
\pgfsetstrokecolor{currentstroke}%
\pgfsetstrokeopacity{0.500000}%
\pgfsetdash{}{0pt}%
\pgfpathmoveto{\pgfqpoint{2.719480in}{2.344102in}}%
\pgfpathcurveto{\pgfqpoint{2.730530in}{2.344102in}}{\pgfqpoint{2.741129in}{2.348492in}}{\pgfqpoint{2.748943in}{2.356305in}}%
\pgfpathcurveto{\pgfqpoint{2.756756in}{2.364119in}}{\pgfqpoint{2.761147in}{2.374718in}}{\pgfqpoint{2.761147in}{2.385768in}}%
\pgfpathcurveto{\pgfqpoint{2.761147in}{2.396818in}}{\pgfqpoint{2.756756in}{2.407417in}}{\pgfqpoint{2.748943in}{2.415231in}}%
\pgfpathcurveto{\pgfqpoint{2.741129in}{2.423045in}}{\pgfqpoint{2.730530in}{2.427435in}}{\pgfqpoint{2.719480in}{2.427435in}}%
\pgfpathcurveto{\pgfqpoint{2.708430in}{2.427435in}}{\pgfqpoint{2.697831in}{2.423045in}}{\pgfqpoint{2.690017in}{2.415231in}}%
\pgfpathcurveto{\pgfqpoint{2.682204in}{2.407417in}}{\pgfqpoint{2.677813in}{2.396818in}}{\pgfqpoint{2.677813in}{2.385768in}}%
\pgfpathcurveto{\pgfqpoint{2.677813in}{2.374718in}}{\pgfqpoint{2.682204in}{2.364119in}}{\pgfqpoint{2.690017in}{2.356305in}}%
\pgfpathcurveto{\pgfqpoint{2.697831in}{2.348492in}}{\pgfqpoint{2.708430in}{2.344102in}}{\pgfqpoint{2.719480in}{2.344102in}}%
\pgfpathclose%
\pgfusepath{stroke,fill}%
\end{pgfscope}%
\begin{pgfscope}%
\pgfpathrectangle{\pgfqpoint{0.612942in}{0.402778in}}{\pgfqpoint{4.338447in}{2.631944in}}%
\pgfusepath{clip}%
\pgfsetbuttcap%
\pgfsetroundjoin%
\definecolor{currentfill}{rgb}{0.172549,0.627451,0.172549}%
\pgfsetfillcolor{currentfill}%
\pgfsetfillopacity{0.500000}%
\pgfsetlinewidth{1.003750pt}%
\definecolor{currentstroke}{rgb}{0.172549,0.627451,0.172549}%
\pgfsetstrokecolor{currentstroke}%
\pgfsetstrokeopacity{0.500000}%
\pgfsetdash{}{0pt}%
\pgfpathmoveto{\pgfqpoint{1.783010in}{2.108292in}}%
\pgfpathcurveto{\pgfqpoint{1.794060in}{2.108292in}}{\pgfqpoint{1.804659in}{2.112682in}}{\pgfqpoint{1.812473in}{2.120496in}}%
\pgfpathcurveto{\pgfqpoint{1.820287in}{2.128309in}}{\pgfqpoint{1.824677in}{2.138909in}}{\pgfqpoint{1.824677in}{2.149959in}}%
\pgfpathcurveto{\pgfqpoint{1.824677in}{2.161009in}}{\pgfqpoint{1.820287in}{2.171608in}}{\pgfqpoint{1.812473in}{2.179421in}}%
\pgfpathcurveto{\pgfqpoint{1.804659in}{2.187235in}}{\pgfqpoint{1.794060in}{2.191625in}}{\pgfqpoint{1.783010in}{2.191625in}}%
\pgfpathcurveto{\pgfqpoint{1.771960in}{2.191625in}}{\pgfqpoint{1.761361in}{2.187235in}}{\pgfqpoint{1.753547in}{2.179421in}}%
\pgfpathcurveto{\pgfqpoint{1.745734in}{2.171608in}}{\pgfqpoint{1.741343in}{2.161009in}}{\pgfqpoint{1.741343in}{2.149959in}}%
\pgfpathcurveto{\pgfqpoint{1.741343in}{2.138909in}}{\pgfqpoint{1.745734in}{2.128309in}}{\pgfqpoint{1.753547in}{2.120496in}}%
\pgfpathcurveto{\pgfqpoint{1.761361in}{2.112682in}}{\pgfqpoint{1.771960in}{2.108292in}}{\pgfqpoint{1.783010in}{2.108292in}}%
\pgfpathclose%
\pgfusepath{stroke,fill}%
\end{pgfscope}%
\begin{pgfscope}%
\pgfpathrectangle{\pgfqpoint{0.612942in}{0.402778in}}{\pgfqpoint{4.338447in}{2.631944in}}%
\pgfusepath{clip}%
\pgfsetbuttcap%
\pgfsetroundjoin%
\definecolor{currentfill}{rgb}{0.172549,0.627451,0.172549}%
\pgfsetfillcolor{currentfill}%
\pgfsetfillopacity{0.500000}%
\pgfsetlinewidth{1.003750pt}%
\definecolor{currentstroke}{rgb}{0.172549,0.627451,0.172549}%
\pgfsetstrokecolor{currentstroke}%
\pgfsetstrokeopacity{0.500000}%
\pgfsetdash{}{0pt}%
\pgfpathmoveto{\pgfqpoint{0.946005in}{1.854198in}}%
\pgfpathcurveto{\pgfqpoint{0.957055in}{1.854198in}}{\pgfqpoint{0.967654in}{1.858589in}}{\pgfqpoint{0.975468in}{1.866402in}}%
\pgfpathcurveto{\pgfqpoint{0.983281in}{1.874216in}}{\pgfqpoint{0.987672in}{1.884815in}}{\pgfqpoint{0.987672in}{1.895865in}}%
\pgfpathcurveto{\pgfqpoint{0.987672in}{1.906915in}}{\pgfqpoint{0.983281in}{1.917514in}}{\pgfqpoint{0.975468in}{1.925328in}}%
\pgfpathcurveto{\pgfqpoint{0.967654in}{1.933141in}}{\pgfqpoint{0.957055in}{1.937532in}}{\pgfqpoint{0.946005in}{1.937532in}}%
\pgfpathcurveto{\pgfqpoint{0.934955in}{1.937532in}}{\pgfqpoint{0.924356in}{1.933141in}}{\pgfqpoint{0.916542in}{1.925328in}}%
\pgfpathcurveto{\pgfqpoint{0.908728in}{1.917514in}}{\pgfqpoint{0.904338in}{1.906915in}}{\pgfqpoint{0.904338in}{1.895865in}}%
\pgfpathcurveto{\pgfqpoint{0.904338in}{1.884815in}}{\pgfqpoint{0.908728in}{1.874216in}}{\pgfqpoint{0.916542in}{1.866402in}}%
\pgfpathcurveto{\pgfqpoint{0.924356in}{1.858589in}}{\pgfqpoint{0.934955in}{1.854198in}}{\pgfqpoint{0.946005in}{1.854198in}}%
\pgfpathclose%
\pgfusepath{stroke,fill}%
\end{pgfscope}%
\begin{pgfscope}%
\pgfpathrectangle{\pgfqpoint{0.612942in}{0.402778in}}{\pgfqpoint{4.338447in}{2.631944in}}%
\pgfusepath{clip}%
\pgfsetbuttcap%
\pgfsetroundjoin%
\definecolor{currentfill}{rgb}{0.172549,0.627451,0.172549}%
\pgfsetfillcolor{currentfill}%
\pgfsetfillopacity{0.500000}%
\pgfsetlinewidth{1.003750pt}%
\definecolor{currentstroke}{rgb}{0.172549,0.627451,0.172549}%
\pgfsetstrokecolor{currentstroke}%
\pgfsetstrokeopacity{0.500000}%
\pgfsetdash{}{0pt}%
\pgfpathmoveto{\pgfqpoint{0.848275in}{1.593428in}}%
\pgfpathcurveto{\pgfqpoint{0.859326in}{1.593428in}}{\pgfqpoint{0.869925in}{1.597819in}}{\pgfqpoint{0.877738in}{1.605632in}}%
\pgfpathcurveto{\pgfqpoint{0.885552in}{1.613446in}}{\pgfqpoint{0.889942in}{1.624045in}}{\pgfqpoint{0.889942in}{1.635095in}}%
\pgfpathcurveto{\pgfqpoint{0.889942in}{1.646145in}}{\pgfqpoint{0.885552in}{1.656744in}}{\pgfqpoint{0.877738in}{1.664558in}}%
\pgfpathcurveto{\pgfqpoint{0.869925in}{1.672372in}}{\pgfqpoint{0.859326in}{1.676762in}}{\pgfqpoint{0.848275in}{1.676762in}}%
\pgfpathcurveto{\pgfqpoint{0.837225in}{1.676762in}}{\pgfqpoint{0.826626in}{1.672372in}}{\pgfqpoint{0.818813in}{1.664558in}}%
\pgfpathcurveto{\pgfqpoint{0.810999in}{1.656744in}}{\pgfqpoint{0.806609in}{1.646145in}}{\pgfqpoint{0.806609in}{1.635095in}}%
\pgfpathcurveto{\pgfqpoint{0.806609in}{1.624045in}}{\pgfqpoint{0.810999in}{1.613446in}}{\pgfqpoint{0.818813in}{1.605632in}}%
\pgfpathcurveto{\pgfqpoint{0.826626in}{1.597819in}}{\pgfqpoint{0.837225in}{1.593428in}}{\pgfqpoint{0.848275in}{1.593428in}}%
\pgfpathclose%
\pgfusepath{stroke,fill}%
\end{pgfscope}%
\begin{pgfscope}%
\pgfpathrectangle{\pgfqpoint{0.612942in}{0.402778in}}{\pgfqpoint{4.338447in}{2.631944in}}%
\pgfusepath{clip}%
\pgfsetbuttcap%
\pgfsetroundjoin%
\definecolor{currentfill}{rgb}{0.172549,0.627451,0.172549}%
\pgfsetfillcolor{currentfill}%
\pgfsetfillopacity{0.500000}%
\pgfsetlinewidth{1.003750pt}%
\definecolor{currentstroke}{rgb}{0.172549,0.627451,0.172549}%
\pgfsetstrokecolor{currentstroke}%
\pgfsetstrokeopacity{0.500000}%
\pgfsetdash{}{0pt}%
\pgfpathmoveto{\pgfqpoint{0.848853in}{1.292518in}}%
\pgfpathcurveto{\pgfqpoint{0.859904in}{1.292518in}}{\pgfqpoint{0.870503in}{1.296908in}}{\pgfqpoint{0.878316in}{1.304722in}}%
\pgfpathcurveto{\pgfqpoint{0.886130in}{1.312535in}}{\pgfqpoint{0.890520in}{1.323134in}}{\pgfqpoint{0.890520in}{1.334184in}}%
\pgfpathcurveto{\pgfqpoint{0.890520in}{1.345235in}}{\pgfqpoint{0.886130in}{1.355834in}}{\pgfqpoint{0.878316in}{1.363647in}}%
\pgfpathcurveto{\pgfqpoint{0.870503in}{1.371461in}}{\pgfqpoint{0.859904in}{1.375851in}}{\pgfqpoint{0.848853in}{1.375851in}}%
\pgfpathcurveto{\pgfqpoint{0.837803in}{1.375851in}}{\pgfqpoint{0.827204in}{1.371461in}}{\pgfqpoint{0.819391in}{1.363647in}}%
\pgfpathcurveto{\pgfqpoint{0.811577in}{1.355834in}}{\pgfqpoint{0.807187in}{1.345235in}}{\pgfqpoint{0.807187in}{1.334184in}}%
\pgfpathcurveto{\pgfqpoint{0.807187in}{1.323134in}}{\pgfqpoint{0.811577in}{1.312535in}}{\pgfqpoint{0.819391in}{1.304722in}}%
\pgfpathcurveto{\pgfqpoint{0.827204in}{1.296908in}}{\pgfqpoint{0.837803in}{1.292518in}}{\pgfqpoint{0.848853in}{1.292518in}}%
\pgfpathclose%
\pgfusepath{stroke,fill}%
\end{pgfscope}%
\begin{pgfscope}%
\pgfpathrectangle{\pgfqpoint{0.612942in}{0.402778in}}{\pgfqpoint{4.338447in}{2.631944in}}%
\pgfusepath{clip}%
\pgfsetbuttcap%
\pgfsetroundjoin%
\definecolor{currentfill}{rgb}{0.172549,0.627451,0.172549}%
\pgfsetfillcolor{currentfill}%
\pgfsetfillopacity{0.500000}%
\pgfsetlinewidth{1.003750pt}%
\definecolor{currentstroke}{rgb}{0.172549,0.627451,0.172549}%
\pgfsetstrokecolor{currentstroke}%
\pgfsetstrokeopacity{0.500000}%
\pgfsetdash{}{0pt}%
\pgfpathmoveto{\pgfqpoint{0.848853in}{1.107519in}}%
\pgfpathcurveto{\pgfqpoint{0.859903in}{1.107519in}}{\pgfqpoint{0.870502in}{1.111909in}}{\pgfqpoint{0.878316in}{1.119723in}}%
\pgfpathcurveto{\pgfqpoint{0.886129in}{1.127536in}}{\pgfqpoint{0.890519in}{1.138135in}}{\pgfqpoint{0.890519in}{1.149185in}}%
\pgfpathcurveto{\pgfqpoint{0.890519in}{1.160236in}}{\pgfqpoint{0.886129in}{1.170835in}}{\pgfqpoint{0.878316in}{1.178648in}}%
\pgfpathcurveto{\pgfqpoint{0.870502in}{1.186462in}}{\pgfqpoint{0.859903in}{1.190852in}}{\pgfqpoint{0.848853in}{1.190852in}}%
\pgfpathcurveto{\pgfqpoint{0.837803in}{1.190852in}}{\pgfqpoint{0.827204in}{1.186462in}}{\pgfqpoint{0.819390in}{1.178648in}}%
\pgfpathcurveto{\pgfqpoint{0.811576in}{1.170835in}}{\pgfqpoint{0.807186in}{1.160236in}}{\pgfqpoint{0.807186in}{1.149185in}}%
\pgfpathcurveto{\pgfqpoint{0.807186in}{1.138135in}}{\pgfqpoint{0.811576in}{1.127536in}}{\pgfqpoint{0.819390in}{1.119723in}}%
\pgfpathcurveto{\pgfqpoint{0.827204in}{1.111909in}}{\pgfqpoint{0.837803in}{1.107519in}}{\pgfqpoint{0.848853in}{1.107519in}}%
\pgfpathclose%
\pgfusepath{stroke,fill}%
\end{pgfscope}%
\begin{pgfscope}%
\pgfpathrectangle{\pgfqpoint{0.612942in}{0.402778in}}{\pgfqpoint{4.338447in}{2.631944in}}%
\pgfusepath{clip}%
\pgfsetbuttcap%
\pgfsetroundjoin%
\definecolor{currentfill}{rgb}{0.172549,0.627451,0.172549}%
\pgfsetfillcolor{currentfill}%
\pgfsetfillopacity{0.500000}%
\pgfsetlinewidth{1.003750pt}%
\definecolor{currentstroke}{rgb}{0.172549,0.627451,0.172549}%
\pgfsetstrokecolor{currentstroke}%
\pgfsetstrokeopacity{0.500000}%
\pgfsetdash{}{0pt}%
\pgfpathmoveto{\pgfqpoint{0.848853in}{0.526873in}}%
\pgfpathcurveto{\pgfqpoint{0.859903in}{0.526873in}}{\pgfqpoint{0.870502in}{0.531263in}}{\pgfqpoint{0.878316in}{0.539076in}}%
\pgfpathcurveto{\pgfqpoint{0.886129in}{0.546890in}}{\pgfqpoint{0.890519in}{0.557489in}}{\pgfqpoint{0.890519in}{0.568539in}}%
\pgfpathcurveto{\pgfqpoint{0.890519in}{0.579589in}}{\pgfqpoint{0.886129in}{0.590188in}}{\pgfqpoint{0.878316in}{0.598002in}}%
\pgfpathcurveto{\pgfqpoint{0.870502in}{0.605816in}}{\pgfqpoint{0.859903in}{0.610206in}}{\pgfqpoint{0.848853in}{0.610206in}}%
\pgfpathcurveto{\pgfqpoint{0.837803in}{0.610206in}}{\pgfqpoint{0.827204in}{0.605816in}}{\pgfqpoint{0.819390in}{0.598002in}}%
\pgfpathcurveto{\pgfqpoint{0.811576in}{0.590188in}}{\pgfqpoint{0.807186in}{0.579589in}}{\pgfqpoint{0.807186in}{0.568539in}}%
\pgfpathcurveto{\pgfqpoint{0.807186in}{0.557489in}}{\pgfqpoint{0.811576in}{0.546890in}}{\pgfqpoint{0.819390in}{0.539076in}}%
\pgfpathcurveto{\pgfqpoint{0.827204in}{0.531263in}}{\pgfqpoint{0.837803in}{0.526873in}}{\pgfqpoint{0.848853in}{0.526873in}}%
\pgfpathclose%
\pgfusepath{stroke,fill}%
\end{pgfscope}%
\begin{pgfscope}%
\pgfpathrectangle{\pgfqpoint{0.612942in}{0.402778in}}{\pgfqpoint{4.338447in}{2.631944in}}%
\pgfusepath{clip}%
\pgfsetbuttcap%
\pgfsetroundjoin%
\definecolor{currentfill}{rgb}{0.172549,0.627451,0.172549}%
\pgfsetfillcolor{currentfill}%
\pgfsetfillopacity{0.500000}%
\pgfsetlinewidth{1.003750pt}%
\definecolor{currentstroke}{rgb}{0.172549,0.627451,0.172549}%
\pgfsetstrokecolor{currentstroke}%
\pgfsetstrokeopacity{0.500000}%
\pgfsetdash{}{0pt}%
\pgfpathmoveto{\pgfqpoint{0.848853in}{0.526873in}}%
\pgfpathcurveto{\pgfqpoint{0.859903in}{0.526873in}}{\pgfqpoint{0.870502in}{0.531263in}}{\pgfqpoint{0.878316in}{0.539076in}}%
\pgfpathcurveto{\pgfqpoint{0.886129in}{0.546890in}}{\pgfqpoint{0.890519in}{0.557489in}}{\pgfqpoint{0.890519in}{0.568539in}}%
\pgfpathcurveto{\pgfqpoint{0.890519in}{0.579589in}}{\pgfqpoint{0.886129in}{0.590188in}}{\pgfqpoint{0.878316in}{0.598002in}}%
\pgfpathcurveto{\pgfqpoint{0.870502in}{0.605816in}}{\pgfqpoint{0.859903in}{0.610206in}}{\pgfqpoint{0.848853in}{0.610206in}}%
\pgfpathcurveto{\pgfqpoint{0.837803in}{0.610206in}}{\pgfqpoint{0.827204in}{0.605816in}}{\pgfqpoint{0.819390in}{0.598002in}}%
\pgfpathcurveto{\pgfqpoint{0.811576in}{0.590188in}}{\pgfqpoint{0.807186in}{0.579589in}}{\pgfqpoint{0.807186in}{0.568539in}}%
\pgfpathcurveto{\pgfqpoint{0.807186in}{0.557489in}}{\pgfqpoint{0.811576in}{0.546890in}}{\pgfqpoint{0.819390in}{0.539076in}}%
\pgfpathcurveto{\pgfqpoint{0.827204in}{0.531263in}}{\pgfqpoint{0.837803in}{0.526873in}}{\pgfqpoint{0.848853in}{0.526873in}}%
\pgfpathclose%
\pgfusepath{stroke,fill}%
\end{pgfscope}%
\begin{pgfscope}%
\pgfsetbuttcap%
\pgfsetroundjoin%
\definecolor{currentfill}{rgb}{0.000000,0.000000,0.000000}%
\pgfsetfillcolor{currentfill}%
\pgfsetlinewidth{0.803000pt}%
\definecolor{currentstroke}{rgb}{0.000000,0.000000,0.000000}%
\pgfsetstrokecolor{currentstroke}%
\pgfsetdash{}{0pt}%
\pgfsys@defobject{currentmarker}{\pgfqpoint{0.000000in}{-0.048611in}}{\pgfqpoint{0.000000in}{0.000000in}}{%
\pgfpathmoveto{\pgfqpoint{0.000000in}{0.000000in}}%
\pgfpathlineto{\pgfqpoint{0.000000in}{-0.048611in}}%
\pgfusepath{stroke,fill}%
}%
\begin{pgfscope}%
\pgfsys@transformshift{0.925320in}{0.402778in}%
\pgfsys@useobject{currentmarker}{}%
\end{pgfscope}%
\end{pgfscope}%
\begin{pgfscope}%
\pgftext[x=0.925320in,y=0.305556in,,top]{\rmfamily\fontsize{10.000000}{12.000000}\selectfont \(\displaystyle 10^{-12}\)}%
\end{pgfscope}%
\begin{pgfscope}%
\pgfsetbuttcap%
\pgfsetroundjoin%
\definecolor{currentfill}{rgb}{0.000000,0.000000,0.000000}%
\pgfsetfillcolor{currentfill}%
\pgfsetlinewidth{0.803000pt}%
\definecolor{currentstroke}{rgb}{0.000000,0.000000,0.000000}%
\pgfsetstrokecolor{currentstroke}%
\pgfsetdash{}{0pt}%
\pgfsys@defobject{currentmarker}{\pgfqpoint{0.000000in}{-0.048611in}}{\pgfqpoint{0.000000in}{0.000000in}}{%
\pgfpathmoveto{\pgfqpoint{0.000000in}{0.000000in}}%
\pgfpathlineto{\pgfqpoint{0.000000in}{-0.048611in}}%
\pgfusepath{stroke,fill}%
}%
\begin{pgfscope}%
\pgfsys@transformshift{1.397679in}{0.402778in}%
\pgfsys@useobject{currentmarker}{}%
\end{pgfscope}%
\end{pgfscope}%
\begin{pgfscope}%
\pgftext[x=1.397679in,y=0.305556in,,top]{\rmfamily\fontsize{10.000000}{12.000000}\selectfont \(\displaystyle 10^{-11}\)}%
\end{pgfscope}%
\begin{pgfscope}%
\pgfsetbuttcap%
\pgfsetroundjoin%
\definecolor{currentfill}{rgb}{0.000000,0.000000,0.000000}%
\pgfsetfillcolor{currentfill}%
\pgfsetlinewidth{0.803000pt}%
\definecolor{currentstroke}{rgb}{0.000000,0.000000,0.000000}%
\pgfsetstrokecolor{currentstroke}%
\pgfsetdash{}{0pt}%
\pgfsys@defobject{currentmarker}{\pgfqpoint{0.000000in}{-0.048611in}}{\pgfqpoint{0.000000in}{0.000000in}}{%
\pgfpathmoveto{\pgfqpoint{0.000000in}{0.000000in}}%
\pgfpathlineto{\pgfqpoint{0.000000in}{-0.048611in}}%
\pgfusepath{stroke,fill}%
}%
\begin{pgfscope}%
\pgfsys@transformshift{1.870038in}{0.402778in}%
\pgfsys@useobject{currentmarker}{}%
\end{pgfscope}%
\end{pgfscope}%
\begin{pgfscope}%
\pgftext[x=1.870038in,y=0.305556in,,top]{\rmfamily\fontsize{10.000000}{12.000000}\selectfont \(\displaystyle 10^{-10}\)}%
\end{pgfscope}%
\begin{pgfscope}%
\pgfsetbuttcap%
\pgfsetroundjoin%
\definecolor{currentfill}{rgb}{0.000000,0.000000,0.000000}%
\pgfsetfillcolor{currentfill}%
\pgfsetlinewidth{0.803000pt}%
\definecolor{currentstroke}{rgb}{0.000000,0.000000,0.000000}%
\pgfsetstrokecolor{currentstroke}%
\pgfsetdash{}{0pt}%
\pgfsys@defobject{currentmarker}{\pgfqpoint{0.000000in}{-0.048611in}}{\pgfqpoint{0.000000in}{0.000000in}}{%
\pgfpathmoveto{\pgfqpoint{0.000000in}{0.000000in}}%
\pgfpathlineto{\pgfqpoint{0.000000in}{-0.048611in}}%
\pgfusepath{stroke,fill}%
}%
\begin{pgfscope}%
\pgfsys@transformshift{2.342396in}{0.402778in}%
\pgfsys@useobject{currentmarker}{}%
\end{pgfscope}%
\end{pgfscope}%
\begin{pgfscope}%
\pgftext[x=2.342396in,y=0.305556in,,top]{\rmfamily\fontsize{10.000000}{12.000000}\selectfont \(\displaystyle 10^{-9}\)}%
\end{pgfscope}%
\begin{pgfscope}%
\pgfsetbuttcap%
\pgfsetroundjoin%
\definecolor{currentfill}{rgb}{0.000000,0.000000,0.000000}%
\pgfsetfillcolor{currentfill}%
\pgfsetlinewidth{0.803000pt}%
\definecolor{currentstroke}{rgb}{0.000000,0.000000,0.000000}%
\pgfsetstrokecolor{currentstroke}%
\pgfsetdash{}{0pt}%
\pgfsys@defobject{currentmarker}{\pgfqpoint{0.000000in}{-0.048611in}}{\pgfqpoint{0.000000in}{0.000000in}}{%
\pgfpathmoveto{\pgfqpoint{0.000000in}{0.000000in}}%
\pgfpathlineto{\pgfqpoint{0.000000in}{-0.048611in}}%
\pgfusepath{stroke,fill}%
}%
\begin{pgfscope}%
\pgfsys@transformshift{2.814755in}{0.402778in}%
\pgfsys@useobject{currentmarker}{}%
\end{pgfscope}%
\end{pgfscope}%
\begin{pgfscope}%
\pgftext[x=2.814755in,y=0.305556in,,top]{\rmfamily\fontsize{10.000000}{12.000000}\selectfont \(\displaystyle 10^{-8}\)}%
\end{pgfscope}%
\begin{pgfscope}%
\pgfsetbuttcap%
\pgfsetroundjoin%
\definecolor{currentfill}{rgb}{0.000000,0.000000,0.000000}%
\pgfsetfillcolor{currentfill}%
\pgfsetlinewidth{0.803000pt}%
\definecolor{currentstroke}{rgb}{0.000000,0.000000,0.000000}%
\pgfsetstrokecolor{currentstroke}%
\pgfsetdash{}{0pt}%
\pgfsys@defobject{currentmarker}{\pgfqpoint{0.000000in}{-0.048611in}}{\pgfqpoint{0.000000in}{0.000000in}}{%
\pgfpathmoveto{\pgfqpoint{0.000000in}{0.000000in}}%
\pgfpathlineto{\pgfqpoint{0.000000in}{-0.048611in}}%
\pgfusepath{stroke,fill}%
}%
\begin{pgfscope}%
\pgfsys@transformshift{3.287113in}{0.402778in}%
\pgfsys@useobject{currentmarker}{}%
\end{pgfscope}%
\end{pgfscope}%
\begin{pgfscope}%
\pgftext[x=3.287113in,y=0.305556in,,top]{\rmfamily\fontsize{10.000000}{12.000000}\selectfont \(\displaystyle 10^{-7}\)}%
\end{pgfscope}%
\begin{pgfscope}%
\pgfsetbuttcap%
\pgfsetroundjoin%
\definecolor{currentfill}{rgb}{0.000000,0.000000,0.000000}%
\pgfsetfillcolor{currentfill}%
\pgfsetlinewidth{0.803000pt}%
\definecolor{currentstroke}{rgb}{0.000000,0.000000,0.000000}%
\pgfsetstrokecolor{currentstroke}%
\pgfsetdash{}{0pt}%
\pgfsys@defobject{currentmarker}{\pgfqpoint{0.000000in}{-0.048611in}}{\pgfqpoint{0.000000in}{0.000000in}}{%
\pgfpathmoveto{\pgfqpoint{0.000000in}{0.000000in}}%
\pgfpathlineto{\pgfqpoint{0.000000in}{-0.048611in}}%
\pgfusepath{stroke,fill}%
}%
\begin{pgfscope}%
\pgfsys@transformshift{3.759472in}{0.402778in}%
\pgfsys@useobject{currentmarker}{}%
\end{pgfscope}%
\end{pgfscope}%
\begin{pgfscope}%
\pgftext[x=3.759472in,y=0.305556in,,top]{\rmfamily\fontsize{10.000000}{12.000000}\selectfont \(\displaystyle 10^{-6}\)}%
\end{pgfscope}%
\begin{pgfscope}%
\pgfsetbuttcap%
\pgfsetroundjoin%
\definecolor{currentfill}{rgb}{0.000000,0.000000,0.000000}%
\pgfsetfillcolor{currentfill}%
\pgfsetlinewidth{0.803000pt}%
\definecolor{currentstroke}{rgb}{0.000000,0.000000,0.000000}%
\pgfsetstrokecolor{currentstroke}%
\pgfsetdash{}{0pt}%
\pgfsys@defobject{currentmarker}{\pgfqpoint{0.000000in}{-0.048611in}}{\pgfqpoint{0.000000in}{0.000000in}}{%
\pgfpathmoveto{\pgfqpoint{0.000000in}{0.000000in}}%
\pgfpathlineto{\pgfqpoint{0.000000in}{-0.048611in}}%
\pgfusepath{stroke,fill}%
}%
\begin{pgfscope}%
\pgfsys@transformshift{4.231831in}{0.402778in}%
\pgfsys@useobject{currentmarker}{}%
\end{pgfscope}%
\end{pgfscope}%
\begin{pgfscope}%
\pgftext[x=4.231831in,y=0.305556in,,top]{\rmfamily\fontsize{10.000000}{12.000000}\selectfont \(\displaystyle 10^{-5}\)}%
\end{pgfscope}%
\begin{pgfscope}%
\pgfsetbuttcap%
\pgfsetroundjoin%
\definecolor{currentfill}{rgb}{0.000000,0.000000,0.000000}%
\pgfsetfillcolor{currentfill}%
\pgfsetlinewidth{0.803000pt}%
\definecolor{currentstroke}{rgb}{0.000000,0.000000,0.000000}%
\pgfsetstrokecolor{currentstroke}%
\pgfsetdash{}{0pt}%
\pgfsys@defobject{currentmarker}{\pgfqpoint{0.000000in}{-0.048611in}}{\pgfqpoint{0.000000in}{0.000000in}}{%
\pgfpathmoveto{\pgfqpoint{0.000000in}{0.000000in}}%
\pgfpathlineto{\pgfqpoint{0.000000in}{-0.048611in}}%
\pgfusepath{stroke,fill}%
}%
\begin{pgfscope}%
\pgfsys@transformshift{4.704189in}{0.402778in}%
\pgfsys@useobject{currentmarker}{}%
\end{pgfscope}%
\end{pgfscope}%
\begin{pgfscope}%
\pgftext[x=4.704189in,y=0.305556in,,top]{\rmfamily\fontsize{10.000000}{12.000000}\selectfont \(\displaystyle 10^{-4}\)}%
\end{pgfscope}%
\begin{pgfscope}%
\pgfsetbuttcap%
\pgfsetroundjoin%
\definecolor{currentfill}{rgb}{0.000000,0.000000,0.000000}%
\pgfsetfillcolor{currentfill}%
\pgfsetlinewidth{0.602250pt}%
\definecolor{currentstroke}{rgb}{0.000000,0.000000,0.000000}%
\pgfsetstrokecolor{currentstroke}%
\pgfsetdash{}{0pt}%
\pgfsys@defobject{currentmarker}{\pgfqpoint{0.000000in}{-0.027778in}}{\pgfqpoint{0.000000in}{0.000000in}}{%
\pgfpathmoveto{\pgfqpoint{0.000000in}{0.000000in}}%
\pgfpathlineto{\pgfqpoint{0.000000in}{-0.027778in}}%
\pgfusepath{stroke,fill}%
}%
\begin{pgfscope}%
\pgfsys@transformshift{0.678334in}{0.402778in}%
\pgfsys@useobject{currentmarker}{}%
\end{pgfscope}%
\end{pgfscope}%
\begin{pgfscope}%
\pgfsetbuttcap%
\pgfsetroundjoin%
\definecolor{currentfill}{rgb}{0.000000,0.000000,0.000000}%
\pgfsetfillcolor{currentfill}%
\pgfsetlinewidth{0.602250pt}%
\definecolor{currentstroke}{rgb}{0.000000,0.000000,0.000000}%
\pgfsetstrokecolor{currentstroke}%
\pgfsetdash{}{0pt}%
\pgfsys@defobject{currentmarker}{\pgfqpoint{0.000000in}{-0.027778in}}{\pgfqpoint{0.000000in}{0.000000in}}{%
\pgfpathmoveto{\pgfqpoint{0.000000in}{0.000000in}}%
\pgfpathlineto{\pgfqpoint{0.000000in}{-0.027778in}}%
\pgfusepath{stroke,fill}%
}%
\begin{pgfscope}%
\pgfsys@transformshift{0.737350in}{0.402778in}%
\pgfsys@useobject{currentmarker}{}%
\end{pgfscope}%
\end{pgfscope}%
\begin{pgfscope}%
\pgfsetbuttcap%
\pgfsetroundjoin%
\definecolor{currentfill}{rgb}{0.000000,0.000000,0.000000}%
\pgfsetfillcolor{currentfill}%
\pgfsetlinewidth{0.602250pt}%
\definecolor{currentstroke}{rgb}{0.000000,0.000000,0.000000}%
\pgfsetstrokecolor{currentstroke}%
\pgfsetdash{}{0pt}%
\pgfsys@defobject{currentmarker}{\pgfqpoint{0.000000in}{-0.027778in}}{\pgfqpoint{0.000000in}{0.000000in}}{%
\pgfpathmoveto{\pgfqpoint{0.000000in}{0.000000in}}%
\pgfpathlineto{\pgfqpoint{0.000000in}{-0.027778in}}%
\pgfusepath{stroke,fill}%
}%
\begin{pgfscope}%
\pgfsys@transformshift{0.783126in}{0.402778in}%
\pgfsys@useobject{currentmarker}{}%
\end{pgfscope}%
\end{pgfscope}%
\begin{pgfscope}%
\pgfsetbuttcap%
\pgfsetroundjoin%
\definecolor{currentfill}{rgb}{0.000000,0.000000,0.000000}%
\pgfsetfillcolor{currentfill}%
\pgfsetlinewidth{0.602250pt}%
\definecolor{currentstroke}{rgb}{0.000000,0.000000,0.000000}%
\pgfsetstrokecolor{currentstroke}%
\pgfsetdash{}{0pt}%
\pgfsys@defobject{currentmarker}{\pgfqpoint{0.000000in}{-0.027778in}}{\pgfqpoint{0.000000in}{0.000000in}}{%
\pgfpathmoveto{\pgfqpoint{0.000000in}{0.000000in}}%
\pgfpathlineto{\pgfqpoint{0.000000in}{-0.027778in}}%
\pgfusepath{stroke,fill}%
}%
\begin{pgfscope}%
\pgfsys@transformshift{0.820528in}{0.402778in}%
\pgfsys@useobject{currentmarker}{}%
\end{pgfscope}%
\end{pgfscope}%
\begin{pgfscope}%
\pgfsetbuttcap%
\pgfsetroundjoin%
\definecolor{currentfill}{rgb}{0.000000,0.000000,0.000000}%
\pgfsetfillcolor{currentfill}%
\pgfsetlinewidth{0.602250pt}%
\definecolor{currentstroke}{rgb}{0.000000,0.000000,0.000000}%
\pgfsetstrokecolor{currentstroke}%
\pgfsetdash{}{0pt}%
\pgfsys@defobject{currentmarker}{\pgfqpoint{0.000000in}{-0.027778in}}{\pgfqpoint{0.000000in}{0.000000in}}{%
\pgfpathmoveto{\pgfqpoint{0.000000in}{0.000000in}}%
\pgfpathlineto{\pgfqpoint{0.000000in}{-0.027778in}}%
\pgfusepath{stroke,fill}%
}%
\begin{pgfscope}%
\pgfsys@transformshift{0.852151in}{0.402778in}%
\pgfsys@useobject{currentmarker}{}%
\end{pgfscope}%
\end{pgfscope}%
\begin{pgfscope}%
\pgfsetbuttcap%
\pgfsetroundjoin%
\definecolor{currentfill}{rgb}{0.000000,0.000000,0.000000}%
\pgfsetfillcolor{currentfill}%
\pgfsetlinewidth{0.602250pt}%
\definecolor{currentstroke}{rgb}{0.000000,0.000000,0.000000}%
\pgfsetstrokecolor{currentstroke}%
\pgfsetdash{}{0pt}%
\pgfsys@defobject{currentmarker}{\pgfqpoint{0.000000in}{-0.027778in}}{\pgfqpoint{0.000000in}{0.000000in}}{%
\pgfpathmoveto{\pgfqpoint{0.000000in}{0.000000in}}%
\pgfpathlineto{\pgfqpoint{0.000000in}{-0.027778in}}%
\pgfusepath{stroke,fill}%
}%
\begin{pgfscope}%
\pgfsys@transformshift{0.879544in}{0.402778in}%
\pgfsys@useobject{currentmarker}{}%
\end{pgfscope}%
\end{pgfscope}%
\begin{pgfscope}%
\pgfsetbuttcap%
\pgfsetroundjoin%
\definecolor{currentfill}{rgb}{0.000000,0.000000,0.000000}%
\pgfsetfillcolor{currentfill}%
\pgfsetlinewidth{0.602250pt}%
\definecolor{currentstroke}{rgb}{0.000000,0.000000,0.000000}%
\pgfsetstrokecolor{currentstroke}%
\pgfsetdash{}{0pt}%
\pgfsys@defobject{currentmarker}{\pgfqpoint{0.000000in}{-0.027778in}}{\pgfqpoint{0.000000in}{0.000000in}}{%
\pgfpathmoveto{\pgfqpoint{0.000000in}{0.000000in}}%
\pgfpathlineto{\pgfqpoint{0.000000in}{-0.027778in}}%
\pgfusepath{stroke,fill}%
}%
\begin{pgfscope}%
\pgfsys@transformshift{0.903706in}{0.402778in}%
\pgfsys@useobject{currentmarker}{}%
\end{pgfscope}%
\end{pgfscope}%
\begin{pgfscope}%
\pgfsetbuttcap%
\pgfsetroundjoin%
\definecolor{currentfill}{rgb}{0.000000,0.000000,0.000000}%
\pgfsetfillcolor{currentfill}%
\pgfsetlinewidth{0.602250pt}%
\definecolor{currentstroke}{rgb}{0.000000,0.000000,0.000000}%
\pgfsetstrokecolor{currentstroke}%
\pgfsetdash{}{0pt}%
\pgfsys@defobject{currentmarker}{\pgfqpoint{0.000000in}{-0.027778in}}{\pgfqpoint{0.000000in}{0.000000in}}{%
\pgfpathmoveto{\pgfqpoint{0.000000in}{0.000000in}}%
\pgfpathlineto{\pgfqpoint{0.000000in}{-0.027778in}}%
\pgfusepath{stroke,fill}%
}%
\begin{pgfscope}%
\pgfsys@transformshift{1.067515in}{0.402778in}%
\pgfsys@useobject{currentmarker}{}%
\end{pgfscope}%
\end{pgfscope}%
\begin{pgfscope}%
\pgfsetbuttcap%
\pgfsetroundjoin%
\definecolor{currentfill}{rgb}{0.000000,0.000000,0.000000}%
\pgfsetfillcolor{currentfill}%
\pgfsetlinewidth{0.602250pt}%
\definecolor{currentstroke}{rgb}{0.000000,0.000000,0.000000}%
\pgfsetstrokecolor{currentstroke}%
\pgfsetdash{}{0pt}%
\pgfsys@defobject{currentmarker}{\pgfqpoint{0.000000in}{-0.027778in}}{\pgfqpoint{0.000000in}{0.000000in}}{%
\pgfpathmoveto{\pgfqpoint{0.000000in}{0.000000in}}%
\pgfpathlineto{\pgfqpoint{0.000000in}{-0.027778in}}%
\pgfusepath{stroke,fill}%
}%
\begin{pgfscope}%
\pgfsys@transformshift{1.150693in}{0.402778in}%
\pgfsys@useobject{currentmarker}{}%
\end{pgfscope}%
\end{pgfscope}%
\begin{pgfscope}%
\pgfsetbuttcap%
\pgfsetroundjoin%
\definecolor{currentfill}{rgb}{0.000000,0.000000,0.000000}%
\pgfsetfillcolor{currentfill}%
\pgfsetlinewidth{0.602250pt}%
\definecolor{currentstroke}{rgb}{0.000000,0.000000,0.000000}%
\pgfsetstrokecolor{currentstroke}%
\pgfsetdash{}{0pt}%
\pgfsys@defobject{currentmarker}{\pgfqpoint{0.000000in}{-0.027778in}}{\pgfqpoint{0.000000in}{0.000000in}}{%
\pgfpathmoveto{\pgfqpoint{0.000000in}{0.000000in}}%
\pgfpathlineto{\pgfqpoint{0.000000in}{-0.027778in}}%
\pgfusepath{stroke,fill}%
}%
\begin{pgfscope}%
\pgfsys@transformshift{1.209709in}{0.402778in}%
\pgfsys@useobject{currentmarker}{}%
\end{pgfscope}%
\end{pgfscope}%
\begin{pgfscope}%
\pgfsetbuttcap%
\pgfsetroundjoin%
\definecolor{currentfill}{rgb}{0.000000,0.000000,0.000000}%
\pgfsetfillcolor{currentfill}%
\pgfsetlinewidth{0.602250pt}%
\definecolor{currentstroke}{rgb}{0.000000,0.000000,0.000000}%
\pgfsetstrokecolor{currentstroke}%
\pgfsetdash{}{0pt}%
\pgfsys@defobject{currentmarker}{\pgfqpoint{0.000000in}{-0.027778in}}{\pgfqpoint{0.000000in}{0.000000in}}{%
\pgfpathmoveto{\pgfqpoint{0.000000in}{0.000000in}}%
\pgfpathlineto{\pgfqpoint{0.000000in}{-0.027778in}}%
\pgfusepath{stroke,fill}%
}%
\begin{pgfscope}%
\pgfsys@transformshift{1.255485in}{0.402778in}%
\pgfsys@useobject{currentmarker}{}%
\end{pgfscope}%
\end{pgfscope}%
\begin{pgfscope}%
\pgfsetbuttcap%
\pgfsetroundjoin%
\definecolor{currentfill}{rgb}{0.000000,0.000000,0.000000}%
\pgfsetfillcolor{currentfill}%
\pgfsetlinewidth{0.602250pt}%
\definecolor{currentstroke}{rgb}{0.000000,0.000000,0.000000}%
\pgfsetstrokecolor{currentstroke}%
\pgfsetdash{}{0pt}%
\pgfsys@defobject{currentmarker}{\pgfqpoint{0.000000in}{-0.027778in}}{\pgfqpoint{0.000000in}{0.000000in}}{%
\pgfpathmoveto{\pgfqpoint{0.000000in}{0.000000in}}%
\pgfpathlineto{\pgfqpoint{0.000000in}{-0.027778in}}%
\pgfusepath{stroke,fill}%
}%
\begin{pgfscope}%
\pgfsys@transformshift{1.292887in}{0.402778in}%
\pgfsys@useobject{currentmarker}{}%
\end{pgfscope}%
\end{pgfscope}%
\begin{pgfscope}%
\pgfsetbuttcap%
\pgfsetroundjoin%
\definecolor{currentfill}{rgb}{0.000000,0.000000,0.000000}%
\pgfsetfillcolor{currentfill}%
\pgfsetlinewidth{0.602250pt}%
\definecolor{currentstroke}{rgb}{0.000000,0.000000,0.000000}%
\pgfsetstrokecolor{currentstroke}%
\pgfsetdash{}{0pt}%
\pgfsys@defobject{currentmarker}{\pgfqpoint{0.000000in}{-0.027778in}}{\pgfqpoint{0.000000in}{0.000000in}}{%
\pgfpathmoveto{\pgfqpoint{0.000000in}{0.000000in}}%
\pgfpathlineto{\pgfqpoint{0.000000in}{-0.027778in}}%
\pgfusepath{stroke,fill}%
}%
\begin{pgfscope}%
\pgfsys@transformshift{1.324510in}{0.402778in}%
\pgfsys@useobject{currentmarker}{}%
\end{pgfscope}%
\end{pgfscope}%
\begin{pgfscope}%
\pgfsetbuttcap%
\pgfsetroundjoin%
\definecolor{currentfill}{rgb}{0.000000,0.000000,0.000000}%
\pgfsetfillcolor{currentfill}%
\pgfsetlinewidth{0.602250pt}%
\definecolor{currentstroke}{rgb}{0.000000,0.000000,0.000000}%
\pgfsetstrokecolor{currentstroke}%
\pgfsetdash{}{0pt}%
\pgfsys@defobject{currentmarker}{\pgfqpoint{0.000000in}{-0.027778in}}{\pgfqpoint{0.000000in}{0.000000in}}{%
\pgfpathmoveto{\pgfqpoint{0.000000in}{0.000000in}}%
\pgfpathlineto{\pgfqpoint{0.000000in}{-0.027778in}}%
\pgfusepath{stroke,fill}%
}%
\begin{pgfscope}%
\pgfsys@transformshift{1.351903in}{0.402778in}%
\pgfsys@useobject{currentmarker}{}%
\end{pgfscope}%
\end{pgfscope}%
\begin{pgfscope}%
\pgfsetbuttcap%
\pgfsetroundjoin%
\definecolor{currentfill}{rgb}{0.000000,0.000000,0.000000}%
\pgfsetfillcolor{currentfill}%
\pgfsetlinewidth{0.602250pt}%
\definecolor{currentstroke}{rgb}{0.000000,0.000000,0.000000}%
\pgfsetstrokecolor{currentstroke}%
\pgfsetdash{}{0pt}%
\pgfsys@defobject{currentmarker}{\pgfqpoint{0.000000in}{-0.027778in}}{\pgfqpoint{0.000000in}{0.000000in}}{%
\pgfpathmoveto{\pgfqpoint{0.000000in}{0.000000in}}%
\pgfpathlineto{\pgfqpoint{0.000000in}{-0.027778in}}%
\pgfusepath{stroke,fill}%
}%
\begin{pgfscope}%
\pgfsys@transformshift{1.376065in}{0.402778in}%
\pgfsys@useobject{currentmarker}{}%
\end{pgfscope}%
\end{pgfscope}%
\begin{pgfscope}%
\pgfsetbuttcap%
\pgfsetroundjoin%
\definecolor{currentfill}{rgb}{0.000000,0.000000,0.000000}%
\pgfsetfillcolor{currentfill}%
\pgfsetlinewidth{0.602250pt}%
\definecolor{currentstroke}{rgb}{0.000000,0.000000,0.000000}%
\pgfsetstrokecolor{currentstroke}%
\pgfsetdash{}{0pt}%
\pgfsys@defobject{currentmarker}{\pgfqpoint{0.000000in}{-0.027778in}}{\pgfqpoint{0.000000in}{0.000000in}}{%
\pgfpathmoveto{\pgfqpoint{0.000000in}{0.000000in}}%
\pgfpathlineto{\pgfqpoint{0.000000in}{-0.027778in}}%
\pgfusepath{stroke,fill}%
}%
\begin{pgfscope}%
\pgfsys@transformshift{1.539873in}{0.402778in}%
\pgfsys@useobject{currentmarker}{}%
\end{pgfscope}%
\end{pgfscope}%
\begin{pgfscope}%
\pgfsetbuttcap%
\pgfsetroundjoin%
\definecolor{currentfill}{rgb}{0.000000,0.000000,0.000000}%
\pgfsetfillcolor{currentfill}%
\pgfsetlinewidth{0.602250pt}%
\definecolor{currentstroke}{rgb}{0.000000,0.000000,0.000000}%
\pgfsetstrokecolor{currentstroke}%
\pgfsetdash{}{0pt}%
\pgfsys@defobject{currentmarker}{\pgfqpoint{0.000000in}{-0.027778in}}{\pgfqpoint{0.000000in}{0.000000in}}{%
\pgfpathmoveto{\pgfqpoint{0.000000in}{0.000000in}}%
\pgfpathlineto{\pgfqpoint{0.000000in}{-0.027778in}}%
\pgfusepath{stroke,fill}%
}%
\begin{pgfscope}%
\pgfsys@transformshift{1.623051in}{0.402778in}%
\pgfsys@useobject{currentmarker}{}%
\end{pgfscope}%
\end{pgfscope}%
\begin{pgfscope}%
\pgfsetbuttcap%
\pgfsetroundjoin%
\definecolor{currentfill}{rgb}{0.000000,0.000000,0.000000}%
\pgfsetfillcolor{currentfill}%
\pgfsetlinewidth{0.602250pt}%
\definecolor{currentstroke}{rgb}{0.000000,0.000000,0.000000}%
\pgfsetstrokecolor{currentstroke}%
\pgfsetdash{}{0pt}%
\pgfsys@defobject{currentmarker}{\pgfqpoint{0.000000in}{-0.027778in}}{\pgfqpoint{0.000000in}{0.000000in}}{%
\pgfpathmoveto{\pgfqpoint{0.000000in}{0.000000in}}%
\pgfpathlineto{\pgfqpoint{0.000000in}{-0.027778in}}%
\pgfusepath{stroke,fill}%
}%
\begin{pgfscope}%
\pgfsys@transformshift{1.682067in}{0.402778in}%
\pgfsys@useobject{currentmarker}{}%
\end{pgfscope}%
\end{pgfscope}%
\begin{pgfscope}%
\pgfsetbuttcap%
\pgfsetroundjoin%
\definecolor{currentfill}{rgb}{0.000000,0.000000,0.000000}%
\pgfsetfillcolor{currentfill}%
\pgfsetlinewidth{0.602250pt}%
\definecolor{currentstroke}{rgb}{0.000000,0.000000,0.000000}%
\pgfsetstrokecolor{currentstroke}%
\pgfsetdash{}{0pt}%
\pgfsys@defobject{currentmarker}{\pgfqpoint{0.000000in}{-0.027778in}}{\pgfqpoint{0.000000in}{0.000000in}}{%
\pgfpathmoveto{\pgfqpoint{0.000000in}{0.000000in}}%
\pgfpathlineto{\pgfqpoint{0.000000in}{-0.027778in}}%
\pgfusepath{stroke,fill}%
}%
\begin{pgfscope}%
\pgfsys@transformshift{1.727844in}{0.402778in}%
\pgfsys@useobject{currentmarker}{}%
\end{pgfscope}%
\end{pgfscope}%
\begin{pgfscope}%
\pgfsetbuttcap%
\pgfsetroundjoin%
\definecolor{currentfill}{rgb}{0.000000,0.000000,0.000000}%
\pgfsetfillcolor{currentfill}%
\pgfsetlinewidth{0.602250pt}%
\definecolor{currentstroke}{rgb}{0.000000,0.000000,0.000000}%
\pgfsetstrokecolor{currentstroke}%
\pgfsetdash{}{0pt}%
\pgfsys@defobject{currentmarker}{\pgfqpoint{0.000000in}{-0.027778in}}{\pgfqpoint{0.000000in}{0.000000in}}{%
\pgfpathmoveto{\pgfqpoint{0.000000in}{0.000000in}}%
\pgfpathlineto{\pgfqpoint{0.000000in}{-0.027778in}}%
\pgfusepath{stroke,fill}%
}%
\begin{pgfscope}%
\pgfsys@transformshift{1.765245in}{0.402778in}%
\pgfsys@useobject{currentmarker}{}%
\end{pgfscope}%
\end{pgfscope}%
\begin{pgfscope}%
\pgfsetbuttcap%
\pgfsetroundjoin%
\definecolor{currentfill}{rgb}{0.000000,0.000000,0.000000}%
\pgfsetfillcolor{currentfill}%
\pgfsetlinewidth{0.602250pt}%
\definecolor{currentstroke}{rgb}{0.000000,0.000000,0.000000}%
\pgfsetstrokecolor{currentstroke}%
\pgfsetdash{}{0pt}%
\pgfsys@defobject{currentmarker}{\pgfqpoint{0.000000in}{-0.027778in}}{\pgfqpoint{0.000000in}{0.000000in}}{%
\pgfpathmoveto{\pgfqpoint{0.000000in}{0.000000in}}%
\pgfpathlineto{\pgfqpoint{0.000000in}{-0.027778in}}%
\pgfusepath{stroke,fill}%
}%
\begin{pgfscope}%
\pgfsys@transformshift{1.796868in}{0.402778in}%
\pgfsys@useobject{currentmarker}{}%
\end{pgfscope}%
\end{pgfscope}%
\begin{pgfscope}%
\pgfsetbuttcap%
\pgfsetroundjoin%
\definecolor{currentfill}{rgb}{0.000000,0.000000,0.000000}%
\pgfsetfillcolor{currentfill}%
\pgfsetlinewidth{0.602250pt}%
\definecolor{currentstroke}{rgb}{0.000000,0.000000,0.000000}%
\pgfsetstrokecolor{currentstroke}%
\pgfsetdash{}{0pt}%
\pgfsys@defobject{currentmarker}{\pgfqpoint{0.000000in}{-0.027778in}}{\pgfqpoint{0.000000in}{0.000000in}}{%
\pgfpathmoveto{\pgfqpoint{0.000000in}{0.000000in}}%
\pgfpathlineto{\pgfqpoint{0.000000in}{-0.027778in}}%
\pgfusepath{stroke,fill}%
}%
\begin{pgfscope}%
\pgfsys@transformshift{1.824261in}{0.402778in}%
\pgfsys@useobject{currentmarker}{}%
\end{pgfscope}%
\end{pgfscope}%
\begin{pgfscope}%
\pgfsetbuttcap%
\pgfsetroundjoin%
\definecolor{currentfill}{rgb}{0.000000,0.000000,0.000000}%
\pgfsetfillcolor{currentfill}%
\pgfsetlinewidth{0.602250pt}%
\definecolor{currentstroke}{rgb}{0.000000,0.000000,0.000000}%
\pgfsetstrokecolor{currentstroke}%
\pgfsetdash{}{0pt}%
\pgfsys@defobject{currentmarker}{\pgfqpoint{0.000000in}{-0.027778in}}{\pgfqpoint{0.000000in}{0.000000in}}{%
\pgfpathmoveto{\pgfqpoint{0.000000in}{0.000000in}}%
\pgfpathlineto{\pgfqpoint{0.000000in}{-0.027778in}}%
\pgfusepath{stroke,fill}%
}%
\begin{pgfscope}%
\pgfsys@transformshift{1.848424in}{0.402778in}%
\pgfsys@useobject{currentmarker}{}%
\end{pgfscope}%
\end{pgfscope}%
\begin{pgfscope}%
\pgfsetbuttcap%
\pgfsetroundjoin%
\definecolor{currentfill}{rgb}{0.000000,0.000000,0.000000}%
\pgfsetfillcolor{currentfill}%
\pgfsetlinewidth{0.602250pt}%
\definecolor{currentstroke}{rgb}{0.000000,0.000000,0.000000}%
\pgfsetstrokecolor{currentstroke}%
\pgfsetdash{}{0pt}%
\pgfsys@defobject{currentmarker}{\pgfqpoint{0.000000in}{-0.027778in}}{\pgfqpoint{0.000000in}{0.000000in}}{%
\pgfpathmoveto{\pgfqpoint{0.000000in}{0.000000in}}%
\pgfpathlineto{\pgfqpoint{0.000000in}{-0.027778in}}%
\pgfusepath{stroke,fill}%
}%
\begin{pgfscope}%
\pgfsys@transformshift{2.012232in}{0.402778in}%
\pgfsys@useobject{currentmarker}{}%
\end{pgfscope}%
\end{pgfscope}%
\begin{pgfscope}%
\pgfsetbuttcap%
\pgfsetroundjoin%
\definecolor{currentfill}{rgb}{0.000000,0.000000,0.000000}%
\pgfsetfillcolor{currentfill}%
\pgfsetlinewidth{0.602250pt}%
\definecolor{currentstroke}{rgb}{0.000000,0.000000,0.000000}%
\pgfsetstrokecolor{currentstroke}%
\pgfsetdash{}{0pt}%
\pgfsys@defobject{currentmarker}{\pgfqpoint{0.000000in}{-0.027778in}}{\pgfqpoint{0.000000in}{0.000000in}}{%
\pgfpathmoveto{\pgfqpoint{0.000000in}{0.000000in}}%
\pgfpathlineto{\pgfqpoint{0.000000in}{-0.027778in}}%
\pgfusepath{stroke,fill}%
}%
\begin{pgfscope}%
\pgfsys@transformshift{2.095410in}{0.402778in}%
\pgfsys@useobject{currentmarker}{}%
\end{pgfscope}%
\end{pgfscope}%
\begin{pgfscope}%
\pgfsetbuttcap%
\pgfsetroundjoin%
\definecolor{currentfill}{rgb}{0.000000,0.000000,0.000000}%
\pgfsetfillcolor{currentfill}%
\pgfsetlinewidth{0.602250pt}%
\definecolor{currentstroke}{rgb}{0.000000,0.000000,0.000000}%
\pgfsetstrokecolor{currentstroke}%
\pgfsetdash{}{0pt}%
\pgfsys@defobject{currentmarker}{\pgfqpoint{0.000000in}{-0.027778in}}{\pgfqpoint{0.000000in}{0.000000in}}{%
\pgfpathmoveto{\pgfqpoint{0.000000in}{0.000000in}}%
\pgfpathlineto{\pgfqpoint{0.000000in}{-0.027778in}}%
\pgfusepath{stroke,fill}%
}%
\begin{pgfscope}%
\pgfsys@transformshift{2.154426in}{0.402778in}%
\pgfsys@useobject{currentmarker}{}%
\end{pgfscope}%
\end{pgfscope}%
\begin{pgfscope}%
\pgfsetbuttcap%
\pgfsetroundjoin%
\definecolor{currentfill}{rgb}{0.000000,0.000000,0.000000}%
\pgfsetfillcolor{currentfill}%
\pgfsetlinewidth{0.602250pt}%
\definecolor{currentstroke}{rgb}{0.000000,0.000000,0.000000}%
\pgfsetstrokecolor{currentstroke}%
\pgfsetdash{}{0pt}%
\pgfsys@defobject{currentmarker}{\pgfqpoint{0.000000in}{-0.027778in}}{\pgfqpoint{0.000000in}{0.000000in}}{%
\pgfpathmoveto{\pgfqpoint{0.000000in}{0.000000in}}%
\pgfpathlineto{\pgfqpoint{0.000000in}{-0.027778in}}%
\pgfusepath{stroke,fill}%
}%
\begin{pgfscope}%
\pgfsys@transformshift{2.200202in}{0.402778in}%
\pgfsys@useobject{currentmarker}{}%
\end{pgfscope}%
\end{pgfscope}%
\begin{pgfscope}%
\pgfsetbuttcap%
\pgfsetroundjoin%
\definecolor{currentfill}{rgb}{0.000000,0.000000,0.000000}%
\pgfsetfillcolor{currentfill}%
\pgfsetlinewidth{0.602250pt}%
\definecolor{currentstroke}{rgb}{0.000000,0.000000,0.000000}%
\pgfsetstrokecolor{currentstroke}%
\pgfsetdash{}{0pt}%
\pgfsys@defobject{currentmarker}{\pgfqpoint{0.000000in}{-0.027778in}}{\pgfqpoint{0.000000in}{0.000000in}}{%
\pgfpathmoveto{\pgfqpoint{0.000000in}{0.000000in}}%
\pgfpathlineto{\pgfqpoint{0.000000in}{-0.027778in}}%
\pgfusepath{stroke,fill}%
}%
\begin{pgfscope}%
\pgfsys@transformshift{2.237604in}{0.402778in}%
\pgfsys@useobject{currentmarker}{}%
\end{pgfscope}%
\end{pgfscope}%
\begin{pgfscope}%
\pgfsetbuttcap%
\pgfsetroundjoin%
\definecolor{currentfill}{rgb}{0.000000,0.000000,0.000000}%
\pgfsetfillcolor{currentfill}%
\pgfsetlinewidth{0.602250pt}%
\definecolor{currentstroke}{rgb}{0.000000,0.000000,0.000000}%
\pgfsetstrokecolor{currentstroke}%
\pgfsetdash{}{0pt}%
\pgfsys@defobject{currentmarker}{\pgfqpoint{0.000000in}{-0.027778in}}{\pgfqpoint{0.000000in}{0.000000in}}{%
\pgfpathmoveto{\pgfqpoint{0.000000in}{0.000000in}}%
\pgfpathlineto{\pgfqpoint{0.000000in}{-0.027778in}}%
\pgfusepath{stroke,fill}%
}%
\begin{pgfscope}%
\pgfsys@transformshift{2.269227in}{0.402778in}%
\pgfsys@useobject{currentmarker}{}%
\end{pgfscope}%
\end{pgfscope}%
\begin{pgfscope}%
\pgfsetbuttcap%
\pgfsetroundjoin%
\definecolor{currentfill}{rgb}{0.000000,0.000000,0.000000}%
\pgfsetfillcolor{currentfill}%
\pgfsetlinewidth{0.602250pt}%
\definecolor{currentstroke}{rgb}{0.000000,0.000000,0.000000}%
\pgfsetstrokecolor{currentstroke}%
\pgfsetdash{}{0pt}%
\pgfsys@defobject{currentmarker}{\pgfqpoint{0.000000in}{-0.027778in}}{\pgfqpoint{0.000000in}{0.000000in}}{%
\pgfpathmoveto{\pgfqpoint{0.000000in}{0.000000in}}%
\pgfpathlineto{\pgfqpoint{0.000000in}{-0.027778in}}%
\pgfusepath{stroke,fill}%
}%
\begin{pgfscope}%
\pgfsys@transformshift{2.296620in}{0.402778in}%
\pgfsys@useobject{currentmarker}{}%
\end{pgfscope}%
\end{pgfscope}%
\begin{pgfscope}%
\pgfsetbuttcap%
\pgfsetroundjoin%
\definecolor{currentfill}{rgb}{0.000000,0.000000,0.000000}%
\pgfsetfillcolor{currentfill}%
\pgfsetlinewidth{0.602250pt}%
\definecolor{currentstroke}{rgb}{0.000000,0.000000,0.000000}%
\pgfsetstrokecolor{currentstroke}%
\pgfsetdash{}{0pt}%
\pgfsys@defobject{currentmarker}{\pgfqpoint{0.000000in}{-0.027778in}}{\pgfqpoint{0.000000in}{0.000000in}}{%
\pgfpathmoveto{\pgfqpoint{0.000000in}{0.000000in}}%
\pgfpathlineto{\pgfqpoint{0.000000in}{-0.027778in}}%
\pgfusepath{stroke,fill}%
}%
\begin{pgfscope}%
\pgfsys@transformshift{2.320782in}{0.402778in}%
\pgfsys@useobject{currentmarker}{}%
\end{pgfscope}%
\end{pgfscope}%
\begin{pgfscope}%
\pgfsetbuttcap%
\pgfsetroundjoin%
\definecolor{currentfill}{rgb}{0.000000,0.000000,0.000000}%
\pgfsetfillcolor{currentfill}%
\pgfsetlinewidth{0.602250pt}%
\definecolor{currentstroke}{rgb}{0.000000,0.000000,0.000000}%
\pgfsetstrokecolor{currentstroke}%
\pgfsetdash{}{0pt}%
\pgfsys@defobject{currentmarker}{\pgfqpoint{0.000000in}{-0.027778in}}{\pgfqpoint{0.000000in}{0.000000in}}{%
\pgfpathmoveto{\pgfqpoint{0.000000in}{0.000000in}}%
\pgfpathlineto{\pgfqpoint{0.000000in}{-0.027778in}}%
\pgfusepath{stroke,fill}%
}%
\begin{pgfscope}%
\pgfsys@transformshift{2.484590in}{0.402778in}%
\pgfsys@useobject{currentmarker}{}%
\end{pgfscope}%
\end{pgfscope}%
\begin{pgfscope}%
\pgfsetbuttcap%
\pgfsetroundjoin%
\definecolor{currentfill}{rgb}{0.000000,0.000000,0.000000}%
\pgfsetfillcolor{currentfill}%
\pgfsetlinewidth{0.602250pt}%
\definecolor{currentstroke}{rgb}{0.000000,0.000000,0.000000}%
\pgfsetstrokecolor{currentstroke}%
\pgfsetdash{}{0pt}%
\pgfsys@defobject{currentmarker}{\pgfqpoint{0.000000in}{-0.027778in}}{\pgfqpoint{0.000000in}{0.000000in}}{%
\pgfpathmoveto{\pgfqpoint{0.000000in}{0.000000in}}%
\pgfpathlineto{\pgfqpoint{0.000000in}{-0.027778in}}%
\pgfusepath{stroke,fill}%
}%
\begin{pgfscope}%
\pgfsys@transformshift{2.567769in}{0.402778in}%
\pgfsys@useobject{currentmarker}{}%
\end{pgfscope}%
\end{pgfscope}%
\begin{pgfscope}%
\pgfsetbuttcap%
\pgfsetroundjoin%
\definecolor{currentfill}{rgb}{0.000000,0.000000,0.000000}%
\pgfsetfillcolor{currentfill}%
\pgfsetlinewidth{0.602250pt}%
\definecolor{currentstroke}{rgb}{0.000000,0.000000,0.000000}%
\pgfsetstrokecolor{currentstroke}%
\pgfsetdash{}{0pt}%
\pgfsys@defobject{currentmarker}{\pgfqpoint{0.000000in}{-0.027778in}}{\pgfqpoint{0.000000in}{0.000000in}}{%
\pgfpathmoveto{\pgfqpoint{0.000000in}{0.000000in}}%
\pgfpathlineto{\pgfqpoint{0.000000in}{-0.027778in}}%
\pgfusepath{stroke,fill}%
}%
\begin{pgfscope}%
\pgfsys@transformshift{2.626784in}{0.402778in}%
\pgfsys@useobject{currentmarker}{}%
\end{pgfscope}%
\end{pgfscope}%
\begin{pgfscope}%
\pgfsetbuttcap%
\pgfsetroundjoin%
\definecolor{currentfill}{rgb}{0.000000,0.000000,0.000000}%
\pgfsetfillcolor{currentfill}%
\pgfsetlinewidth{0.602250pt}%
\definecolor{currentstroke}{rgb}{0.000000,0.000000,0.000000}%
\pgfsetstrokecolor{currentstroke}%
\pgfsetdash{}{0pt}%
\pgfsys@defobject{currentmarker}{\pgfqpoint{0.000000in}{-0.027778in}}{\pgfqpoint{0.000000in}{0.000000in}}{%
\pgfpathmoveto{\pgfqpoint{0.000000in}{0.000000in}}%
\pgfpathlineto{\pgfqpoint{0.000000in}{-0.027778in}}%
\pgfusepath{stroke,fill}%
}%
\begin{pgfscope}%
\pgfsys@transformshift{2.672561in}{0.402778in}%
\pgfsys@useobject{currentmarker}{}%
\end{pgfscope}%
\end{pgfscope}%
\begin{pgfscope}%
\pgfsetbuttcap%
\pgfsetroundjoin%
\definecolor{currentfill}{rgb}{0.000000,0.000000,0.000000}%
\pgfsetfillcolor{currentfill}%
\pgfsetlinewidth{0.602250pt}%
\definecolor{currentstroke}{rgb}{0.000000,0.000000,0.000000}%
\pgfsetstrokecolor{currentstroke}%
\pgfsetdash{}{0pt}%
\pgfsys@defobject{currentmarker}{\pgfqpoint{0.000000in}{-0.027778in}}{\pgfqpoint{0.000000in}{0.000000in}}{%
\pgfpathmoveto{\pgfqpoint{0.000000in}{0.000000in}}%
\pgfpathlineto{\pgfqpoint{0.000000in}{-0.027778in}}%
\pgfusepath{stroke,fill}%
}%
\begin{pgfscope}%
\pgfsys@transformshift{2.709963in}{0.402778in}%
\pgfsys@useobject{currentmarker}{}%
\end{pgfscope}%
\end{pgfscope}%
\begin{pgfscope}%
\pgfsetbuttcap%
\pgfsetroundjoin%
\definecolor{currentfill}{rgb}{0.000000,0.000000,0.000000}%
\pgfsetfillcolor{currentfill}%
\pgfsetlinewidth{0.602250pt}%
\definecolor{currentstroke}{rgb}{0.000000,0.000000,0.000000}%
\pgfsetstrokecolor{currentstroke}%
\pgfsetdash{}{0pt}%
\pgfsys@defobject{currentmarker}{\pgfqpoint{0.000000in}{-0.027778in}}{\pgfqpoint{0.000000in}{0.000000in}}{%
\pgfpathmoveto{\pgfqpoint{0.000000in}{0.000000in}}%
\pgfpathlineto{\pgfqpoint{0.000000in}{-0.027778in}}%
\pgfusepath{stroke,fill}%
}%
\begin{pgfscope}%
\pgfsys@transformshift{2.741586in}{0.402778in}%
\pgfsys@useobject{currentmarker}{}%
\end{pgfscope}%
\end{pgfscope}%
\begin{pgfscope}%
\pgfsetbuttcap%
\pgfsetroundjoin%
\definecolor{currentfill}{rgb}{0.000000,0.000000,0.000000}%
\pgfsetfillcolor{currentfill}%
\pgfsetlinewidth{0.602250pt}%
\definecolor{currentstroke}{rgb}{0.000000,0.000000,0.000000}%
\pgfsetstrokecolor{currentstroke}%
\pgfsetdash{}{0pt}%
\pgfsys@defobject{currentmarker}{\pgfqpoint{0.000000in}{-0.027778in}}{\pgfqpoint{0.000000in}{0.000000in}}{%
\pgfpathmoveto{\pgfqpoint{0.000000in}{0.000000in}}%
\pgfpathlineto{\pgfqpoint{0.000000in}{-0.027778in}}%
\pgfusepath{stroke,fill}%
}%
\begin{pgfscope}%
\pgfsys@transformshift{2.768979in}{0.402778in}%
\pgfsys@useobject{currentmarker}{}%
\end{pgfscope}%
\end{pgfscope}%
\begin{pgfscope}%
\pgfsetbuttcap%
\pgfsetroundjoin%
\definecolor{currentfill}{rgb}{0.000000,0.000000,0.000000}%
\pgfsetfillcolor{currentfill}%
\pgfsetlinewidth{0.602250pt}%
\definecolor{currentstroke}{rgb}{0.000000,0.000000,0.000000}%
\pgfsetstrokecolor{currentstroke}%
\pgfsetdash{}{0pt}%
\pgfsys@defobject{currentmarker}{\pgfqpoint{0.000000in}{-0.027778in}}{\pgfqpoint{0.000000in}{0.000000in}}{%
\pgfpathmoveto{\pgfqpoint{0.000000in}{0.000000in}}%
\pgfpathlineto{\pgfqpoint{0.000000in}{-0.027778in}}%
\pgfusepath{stroke,fill}%
}%
\begin{pgfscope}%
\pgfsys@transformshift{2.793141in}{0.402778in}%
\pgfsys@useobject{currentmarker}{}%
\end{pgfscope}%
\end{pgfscope}%
\begin{pgfscope}%
\pgfsetbuttcap%
\pgfsetroundjoin%
\definecolor{currentfill}{rgb}{0.000000,0.000000,0.000000}%
\pgfsetfillcolor{currentfill}%
\pgfsetlinewidth{0.602250pt}%
\definecolor{currentstroke}{rgb}{0.000000,0.000000,0.000000}%
\pgfsetstrokecolor{currentstroke}%
\pgfsetdash{}{0pt}%
\pgfsys@defobject{currentmarker}{\pgfqpoint{0.000000in}{-0.027778in}}{\pgfqpoint{0.000000in}{0.000000in}}{%
\pgfpathmoveto{\pgfqpoint{0.000000in}{0.000000in}}%
\pgfpathlineto{\pgfqpoint{0.000000in}{-0.027778in}}%
\pgfusepath{stroke,fill}%
}%
\begin{pgfscope}%
\pgfsys@transformshift{2.956949in}{0.402778in}%
\pgfsys@useobject{currentmarker}{}%
\end{pgfscope}%
\end{pgfscope}%
\begin{pgfscope}%
\pgfsetbuttcap%
\pgfsetroundjoin%
\definecolor{currentfill}{rgb}{0.000000,0.000000,0.000000}%
\pgfsetfillcolor{currentfill}%
\pgfsetlinewidth{0.602250pt}%
\definecolor{currentstroke}{rgb}{0.000000,0.000000,0.000000}%
\pgfsetstrokecolor{currentstroke}%
\pgfsetdash{}{0pt}%
\pgfsys@defobject{currentmarker}{\pgfqpoint{0.000000in}{-0.027778in}}{\pgfqpoint{0.000000in}{0.000000in}}{%
\pgfpathmoveto{\pgfqpoint{0.000000in}{0.000000in}}%
\pgfpathlineto{\pgfqpoint{0.000000in}{-0.027778in}}%
\pgfusepath{stroke,fill}%
}%
\begin{pgfscope}%
\pgfsys@transformshift{3.040127in}{0.402778in}%
\pgfsys@useobject{currentmarker}{}%
\end{pgfscope}%
\end{pgfscope}%
\begin{pgfscope}%
\pgfsetbuttcap%
\pgfsetroundjoin%
\definecolor{currentfill}{rgb}{0.000000,0.000000,0.000000}%
\pgfsetfillcolor{currentfill}%
\pgfsetlinewidth{0.602250pt}%
\definecolor{currentstroke}{rgb}{0.000000,0.000000,0.000000}%
\pgfsetstrokecolor{currentstroke}%
\pgfsetdash{}{0pt}%
\pgfsys@defobject{currentmarker}{\pgfqpoint{0.000000in}{-0.027778in}}{\pgfqpoint{0.000000in}{0.000000in}}{%
\pgfpathmoveto{\pgfqpoint{0.000000in}{0.000000in}}%
\pgfpathlineto{\pgfqpoint{0.000000in}{-0.027778in}}%
\pgfusepath{stroke,fill}%
}%
\begin{pgfscope}%
\pgfsys@transformshift{3.099143in}{0.402778in}%
\pgfsys@useobject{currentmarker}{}%
\end{pgfscope}%
\end{pgfscope}%
\begin{pgfscope}%
\pgfsetbuttcap%
\pgfsetroundjoin%
\definecolor{currentfill}{rgb}{0.000000,0.000000,0.000000}%
\pgfsetfillcolor{currentfill}%
\pgfsetlinewidth{0.602250pt}%
\definecolor{currentstroke}{rgb}{0.000000,0.000000,0.000000}%
\pgfsetstrokecolor{currentstroke}%
\pgfsetdash{}{0pt}%
\pgfsys@defobject{currentmarker}{\pgfqpoint{0.000000in}{-0.027778in}}{\pgfqpoint{0.000000in}{0.000000in}}{%
\pgfpathmoveto{\pgfqpoint{0.000000in}{0.000000in}}%
\pgfpathlineto{\pgfqpoint{0.000000in}{-0.027778in}}%
\pgfusepath{stroke,fill}%
}%
\begin{pgfscope}%
\pgfsys@transformshift{3.144919in}{0.402778in}%
\pgfsys@useobject{currentmarker}{}%
\end{pgfscope}%
\end{pgfscope}%
\begin{pgfscope}%
\pgfsetbuttcap%
\pgfsetroundjoin%
\definecolor{currentfill}{rgb}{0.000000,0.000000,0.000000}%
\pgfsetfillcolor{currentfill}%
\pgfsetlinewidth{0.602250pt}%
\definecolor{currentstroke}{rgb}{0.000000,0.000000,0.000000}%
\pgfsetstrokecolor{currentstroke}%
\pgfsetdash{}{0pt}%
\pgfsys@defobject{currentmarker}{\pgfqpoint{0.000000in}{-0.027778in}}{\pgfqpoint{0.000000in}{0.000000in}}{%
\pgfpathmoveto{\pgfqpoint{0.000000in}{0.000000in}}%
\pgfpathlineto{\pgfqpoint{0.000000in}{-0.027778in}}%
\pgfusepath{stroke,fill}%
}%
\begin{pgfscope}%
\pgfsys@transformshift{3.182321in}{0.402778in}%
\pgfsys@useobject{currentmarker}{}%
\end{pgfscope}%
\end{pgfscope}%
\begin{pgfscope}%
\pgfsetbuttcap%
\pgfsetroundjoin%
\definecolor{currentfill}{rgb}{0.000000,0.000000,0.000000}%
\pgfsetfillcolor{currentfill}%
\pgfsetlinewidth{0.602250pt}%
\definecolor{currentstroke}{rgb}{0.000000,0.000000,0.000000}%
\pgfsetstrokecolor{currentstroke}%
\pgfsetdash{}{0pt}%
\pgfsys@defobject{currentmarker}{\pgfqpoint{0.000000in}{-0.027778in}}{\pgfqpoint{0.000000in}{0.000000in}}{%
\pgfpathmoveto{\pgfqpoint{0.000000in}{0.000000in}}%
\pgfpathlineto{\pgfqpoint{0.000000in}{-0.027778in}}%
\pgfusepath{stroke,fill}%
}%
\begin{pgfscope}%
\pgfsys@transformshift{3.213944in}{0.402778in}%
\pgfsys@useobject{currentmarker}{}%
\end{pgfscope}%
\end{pgfscope}%
\begin{pgfscope}%
\pgfsetbuttcap%
\pgfsetroundjoin%
\definecolor{currentfill}{rgb}{0.000000,0.000000,0.000000}%
\pgfsetfillcolor{currentfill}%
\pgfsetlinewidth{0.602250pt}%
\definecolor{currentstroke}{rgb}{0.000000,0.000000,0.000000}%
\pgfsetstrokecolor{currentstroke}%
\pgfsetdash{}{0pt}%
\pgfsys@defobject{currentmarker}{\pgfqpoint{0.000000in}{-0.027778in}}{\pgfqpoint{0.000000in}{0.000000in}}{%
\pgfpathmoveto{\pgfqpoint{0.000000in}{0.000000in}}%
\pgfpathlineto{\pgfqpoint{0.000000in}{-0.027778in}}%
\pgfusepath{stroke,fill}%
}%
\begin{pgfscope}%
\pgfsys@transformshift{3.241337in}{0.402778in}%
\pgfsys@useobject{currentmarker}{}%
\end{pgfscope}%
\end{pgfscope}%
\begin{pgfscope}%
\pgfsetbuttcap%
\pgfsetroundjoin%
\definecolor{currentfill}{rgb}{0.000000,0.000000,0.000000}%
\pgfsetfillcolor{currentfill}%
\pgfsetlinewidth{0.602250pt}%
\definecolor{currentstroke}{rgb}{0.000000,0.000000,0.000000}%
\pgfsetstrokecolor{currentstroke}%
\pgfsetdash{}{0pt}%
\pgfsys@defobject{currentmarker}{\pgfqpoint{0.000000in}{-0.027778in}}{\pgfqpoint{0.000000in}{0.000000in}}{%
\pgfpathmoveto{\pgfqpoint{0.000000in}{0.000000in}}%
\pgfpathlineto{\pgfqpoint{0.000000in}{-0.027778in}}%
\pgfusepath{stroke,fill}%
}%
\begin{pgfscope}%
\pgfsys@transformshift{3.265500in}{0.402778in}%
\pgfsys@useobject{currentmarker}{}%
\end{pgfscope}%
\end{pgfscope}%
\begin{pgfscope}%
\pgfsetbuttcap%
\pgfsetroundjoin%
\definecolor{currentfill}{rgb}{0.000000,0.000000,0.000000}%
\pgfsetfillcolor{currentfill}%
\pgfsetlinewidth{0.602250pt}%
\definecolor{currentstroke}{rgb}{0.000000,0.000000,0.000000}%
\pgfsetstrokecolor{currentstroke}%
\pgfsetdash{}{0pt}%
\pgfsys@defobject{currentmarker}{\pgfqpoint{0.000000in}{-0.027778in}}{\pgfqpoint{0.000000in}{0.000000in}}{%
\pgfpathmoveto{\pgfqpoint{0.000000in}{0.000000in}}%
\pgfpathlineto{\pgfqpoint{0.000000in}{-0.027778in}}%
\pgfusepath{stroke,fill}%
}%
\begin{pgfscope}%
\pgfsys@transformshift{3.429308in}{0.402778in}%
\pgfsys@useobject{currentmarker}{}%
\end{pgfscope}%
\end{pgfscope}%
\begin{pgfscope}%
\pgfsetbuttcap%
\pgfsetroundjoin%
\definecolor{currentfill}{rgb}{0.000000,0.000000,0.000000}%
\pgfsetfillcolor{currentfill}%
\pgfsetlinewidth{0.602250pt}%
\definecolor{currentstroke}{rgb}{0.000000,0.000000,0.000000}%
\pgfsetstrokecolor{currentstroke}%
\pgfsetdash{}{0pt}%
\pgfsys@defobject{currentmarker}{\pgfqpoint{0.000000in}{-0.027778in}}{\pgfqpoint{0.000000in}{0.000000in}}{%
\pgfpathmoveto{\pgfqpoint{0.000000in}{0.000000in}}%
\pgfpathlineto{\pgfqpoint{0.000000in}{-0.027778in}}%
\pgfusepath{stroke,fill}%
}%
\begin{pgfscope}%
\pgfsys@transformshift{3.512486in}{0.402778in}%
\pgfsys@useobject{currentmarker}{}%
\end{pgfscope}%
\end{pgfscope}%
\begin{pgfscope}%
\pgfsetbuttcap%
\pgfsetroundjoin%
\definecolor{currentfill}{rgb}{0.000000,0.000000,0.000000}%
\pgfsetfillcolor{currentfill}%
\pgfsetlinewidth{0.602250pt}%
\definecolor{currentstroke}{rgb}{0.000000,0.000000,0.000000}%
\pgfsetstrokecolor{currentstroke}%
\pgfsetdash{}{0pt}%
\pgfsys@defobject{currentmarker}{\pgfqpoint{0.000000in}{-0.027778in}}{\pgfqpoint{0.000000in}{0.000000in}}{%
\pgfpathmoveto{\pgfqpoint{0.000000in}{0.000000in}}%
\pgfpathlineto{\pgfqpoint{0.000000in}{-0.027778in}}%
\pgfusepath{stroke,fill}%
}%
\begin{pgfscope}%
\pgfsys@transformshift{3.571502in}{0.402778in}%
\pgfsys@useobject{currentmarker}{}%
\end{pgfscope}%
\end{pgfscope}%
\begin{pgfscope}%
\pgfsetbuttcap%
\pgfsetroundjoin%
\definecolor{currentfill}{rgb}{0.000000,0.000000,0.000000}%
\pgfsetfillcolor{currentfill}%
\pgfsetlinewidth{0.602250pt}%
\definecolor{currentstroke}{rgb}{0.000000,0.000000,0.000000}%
\pgfsetstrokecolor{currentstroke}%
\pgfsetdash{}{0pt}%
\pgfsys@defobject{currentmarker}{\pgfqpoint{0.000000in}{-0.027778in}}{\pgfqpoint{0.000000in}{0.000000in}}{%
\pgfpathmoveto{\pgfqpoint{0.000000in}{0.000000in}}%
\pgfpathlineto{\pgfqpoint{0.000000in}{-0.027778in}}%
\pgfusepath{stroke,fill}%
}%
\begin{pgfscope}%
\pgfsys@transformshift{3.617278in}{0.402778in}%
\pgfsys@useobject{currentmarker}{}%
\end{pgfscope}%
\end{pgfscope}%
\begin{pgfscope}%
\pgfsetbuttcap%
\pgfsetroundjoin%
\definecolor{currentfill}{rgb}{0.000000,0.000000,0.000000}%
\pgfsetfillcolor{currentfill}%
\pgfsetlinewidth{0.602250pt}%
\definecolor{currentstroke}{rgb}{0.000000,0.000000,0.000000}%
\pgfsetstrokecolor{currentstroke}%
\pgfsetdash{}{0pt}%
\pgfsys@defobject{currentmarker}{\pgfqpoint{0.000000in}{-0.027778in}}{\pgfqpoint{0.000000in}{0.000000in}}{%
\pgfpathmoveto{\pgfqpoint{0.000000in}{0.000000in}}%
\pgfpathlineto{\pgfqpoint{0.000000in}{-0.027778in}}%
\pgfusepath{stroke,fill}%
}%
\begin{pgfscope}%
\pgfsys@transformshift{3.654680in}{0.402778in}%
\pgfsys@useobject{currentmarker}{}%
\end{pgfscope}%
\end{pgfscope}%
\begin{pgfscope}%
\pgfsetbuttcap%
\pgfsetroundjoin%
\definecolor{currentfill}{rgb}{0.000000,0.000000,0.000000}%
\pgfsetfillcolor{currentfill}%
\pgfsetlinewidth{0.602250pt}%
\definecolor{currentstroke}{rgb}{0.000000,0.000000,0.000000}%
\pgfsetstrokecolor{currentstroke}%
\pgfsetdash{}{0pt}%
\pgfsys@defobject{currentmarker}{\pgfqpoint{0.000000in}{-0.027778in}}{\pgfqpoint{0.000000in}{0.000000in}}{%
\pgfpathmoveto{\pgfqpoint{0.000000in}{0.000000in}}%
\pgfpathlineto{\pgfqpoint{0.000000in}{-0.027778in}}%
\pgfusepath{stroke,fill}%
}%
\begin{pgfscope}%
\pgfsys@transformshift{3.686303in}{0.402778in}%
\pgfsys@useobject{currentmarker}{}%
\end{pgfscope}%
\end{pgfscope}%
\begin{pgfscope}%
\pgfsetbuttcap%
\pgfsetroundjoin%
\definecolor{currentfill}{rgb}{0.000000,0.000000,0.000000}%
\pgfsetfillcolor{currentfill}%
\pgfsetlinewidth{0.602250pt}%
\definecolor{currentstroke}{rgb}{0.000000,0.000000,0.000000}%
\pgfsetstrokecolor{currentstroke}%
\pgfsetdash{}{0pt}%
\pgfsys@defobject{currentmarker}{\pgfqpoint{0.000000in}{-0.027778in}}{\pgfqpoint{0.000000in}{0.000000in}}{%
\pgfpathmoveto{\pgfqpoint{0.000000in}{0.000000in}}%
\pgfpathlineto{\pgfqpoint{0.000000in}{-0.027778in}}%
\pgfusepath{stroke,fill}%
}%
\begin{pgfscope}%
\pgfsys@transformshift{3.713696in}{0.402778in}%
\pgfsys@useobject{currentmarker}{}%
\end{pgfscope}%
\end{pgfscope}%
\begin{pgfscope}%
\pgfsetbuttcap%
\pgfsetroundjoin%
\definecolor{currentfill}{rgb}{0.000000,0.000000,0.000000}%
\pgfsetfillcolor{currentfill}%
\pgfsetlinewidth{0.602250pt}%
\definecolor{currentstroke}{rgb}{0.000000,0.000000,0.000000}%
\pgfsetstrokecolor{currentstroke}%
\pgfsetdash{}{0pt}%
\pgfsys@defobject{currentmarker}{\pgfqpoint{0.000000in}{-0.027778in}}{\pgfqpoint{0.000000in}{0.000000in}}{%
\pgfpathmoveto{\pgfqpoint{0.000000in}{0.000000in}}%
\pgfpathlineto{\pgfqpoint{0.000000in}{-0.027778in}}%
\pgfusepath{stroke,fill}%
}%
\begin{pgfscope}%
\pgfsys@transformshift{3.737858in}{0.402778in}%
\pgfsys@useobject{currentmarker}{}%
\end{pgfscope}%
\end{pgfscope}%
\begin{pgfscope}%
\pgfsetbuttcap%
\pgfsetroundjoin%
\definecolor{currentfill}{rgb}{0.000000,0.000000,0.000000}%
\pgfsetfillcolor{currentfill}%
\pgfsetlinewidth{0.602250pt}%
\definecolor{currentstroke}{rgb}{0.000000,0.000000,0.000000}%
\pgfsetstrokecolor{currentstroke}%
\pgfsetdash{}{0pt}%
\pgfsys@defobject{currentmarker}{\pgfqpoint{0.000000in}{-0.027778in}}{\pgfqpoint{0.000000in}{0.000000in}}{%
\pgfpathmoveto{\pgfqpoint{0.000000in}{0.000000in}}%
\pgfpathlineto{\pgfqpoint{0.000000in}{-0.027778in}}%
\pgfusepath{stroke,fill}%
}%
\begin{pgfscope}%
\pgfsys@transformshift{3.901666in}{0.402778in}%
\pgfsys@useobject{currentmarker}{}%
\end{pgfscope}%
\end{pgfscope}%
\begin{pgfscope}%
\pgfsetbuttcap%
\pgfsetroundjoin%
\definecolor{currentfill}{rgb}{0.000000,0.000000,0.000000}%
\pgfsetfillcolor{currentfill}%
\pgfsetlinewidth{0.602250pt}%
\definecolor{currentstroke}{rgb}{0.000000,0.000000,0.000000}%
\pgfsetstrokecolor{currentstroke}%
\pgfsetdash{}{0pt}%
\pgfsys@defobject{currentmarker}{\pgfqpoint{0.000000in}{-0.027778in}}{\pgfqpoint{0.000000in}{0.000000in}}{%
\pgfpathmoveto{\pgfqpoint{0.000000in}{0.000000in}}%
\pgfpathlineto{\pgfqpoint{0.000000in}{-0.027778in}}%
\pgfusepath{stroke,fill}%
}%
\begin{pgfscope}%
\pgfsys@transformshift{3.984844in}{0.402778in}%
\pgfsys@useobject{currentmarker}{}%
\end{pgfscope}%
\end{pgfscope}%
\begin{pgfscope}%
\pgfsetbuttcap%
\pgfsetroundjoin%
\definecolor{currentfill}{rgb}{0.000000,0.000000,0.000000}%
\pgfsetfillcolor{currentfill}%
\pgfsetlinewidth{0.602250pt}%
\definecolor{currentstroke}{rgb}{0.000000,0.000000,0.000000}%
\pgfsetstrokecolor{currentstroke}%
\pgfsetdash{}{0pt}%
\pgfsys@defobject{currentmarker}{\pgfqpoint{0.000000in}{-0.027778in}}{\pgfqpoint{0.000000in}{0.000000in}}{%
\pgfpathmoveto{\pgfqpoint{0.000000in}{0.000000in}}%
\pgfpathlineto{\pgfqpoint{0.000000in}{-0.027778in}}%
\pgfusepath{stroke,fill}%
}%
\begin{pgfscope}%
\pgfsys@transformshift{4.043860in}{0.402778in}%
\pgfsys@useobject{currentmarker}{}%
\end{pgfscope}%
\end{pgfscope}%
\begin{pgfscope}%
\pgfsetbuttcap%
\pgfsetroundjoin%
\definecolor{currentfill}{rgb}{0.000000,0.000000,0.000000}%
\pgfsetfillcolor{currentfill}%
\pgfsetlinewidth{0.602250pt}%
\definecolor{currentstroke}{rgb}{0.000000,0.000000,0.000000}%
\pgfsetstrokecolor{currentstroke}%
\pgfsetdash{}{0pt}%
\pgfsys@defobject{currentmarker}{\pgfqpoint{0.000000in}{-0.027778in}}{\pgfqpoint{0.000000in}{0.000000in}}{%
\pgfpathmoveto{\pgfqpoint{0.000000in}{0.000000in}}%
\pgfpathlineto{\pgfqpoint{0.000000in}{-0.027778in}}%
\pgfusepath{stroke,fill}%
}%
\begin{pgfscope}%
\pgfsys@transformshift{4.089637in}{0.402778in}%
\pgfsys@useobject{currentmarker}{}%
\end{pgfscope}%
\end{pgfscope}%
\begin{pgfscope}%
\pgfsetbuttcap%
\pgfsetroundjoin%
\definecolor{currentfill}{rgb}{0.000000,0.000000,0.000000}%
\pgfsetfillcolor{currentfill}%
\pgfsetlinewidth{0.602250pt}%
\definecolor{currentstroke}{rgb}{0.000000,0.000000,0.000000}%
\pgfsetstrokecolor{currentstroke}%
\pgfsetdash{}{0pt}%
\pgfsys@defobject{currentmarker}{\pgfqpoint{0.000000in}{-0.027778in}}{\pgfqpoint{0.000000in}{0.000000in}}{%
\pgfpathmoveto{\pgfqpoint{0.000000in}{0.000000in}}%
\pgfpathlineto{\pgfqpoint{0.000000in}{-0.027778in}}%
\pgfusepath{stroke,fill}%
}%
\begin{pgfscope}%
\pgfsys@transformshift{4.127039in}{0.402778in}%
\pgfsys@useobject{currentmarker}{}%
\end{pgfscope}%
\end{pgfscope}%
\begin{pgfscope}%
\pgfsetbuttcap%
\pgfsetroundjoin%
\definecolor{currentfill}{rgb}{0.000000,0.000000,0.000000}%
\pgfsetfillcolor{currentfill}%
\pgfsetlinewidth{0.602250pt}%
\definecolor{currentstroke}{rgb}{0.000000,0.000000,0.000000}%
\pgfsetstrokecolor{currentstroke}%
\pgfsetdash{}{0pt}%
\pgfsys@defobject{currentmarker}{\pgfqpoint{0.000000in}{-0.027778in}}{\pgfqpoint{0.000000in}{0.000000in}}{%
\pgfpathmoveto{\pgfqpoint{0.000000in}{0.000000in}}%
\pgfpathlineto{\pgfqpoint{0.000000in}{-0.027778in}}%
\pgfusepath{stroke,fill}%
}%
\begin{pgfscope}%
\pgfsys@transformshift{4.158661in}{0.402778in}%
\pgfsys@useobject{currentmarker}{}%
\end{pgfscope}%
\end{pgfscope}%
\begin{pgfscope}%
\pgfsetbuttcap%
\pgfsetroundjoin%
\definecolor{currentfill}{rgb}{0.000000,0.000000,0.000000}%
\pgfsetfillcolor{currentfill}%
\pgfsetlinewidth{0.602250pt}%
\definecolor{currentstroke}{rgb}{0.000000,0.000000,0.000000}%
\pgfsetstrokecolor{currentstroke}%
\pgfsetdash{}{0pt}%
\pgfsys@defobject{currentmarker}{\pgfqpoint{0.000000in}{-0.027778in}}{\pgfqpoint{0.000000in}{0.000000in}}{%
\pgfpathmoveto{\pgfqpoint{0.000000in}{0.000000in}}%
\pgfpathlineto{\pgfqpoint{0.000000in}{-0.027778in}}%
\pgfusepath{stroke,fill}%
}%
\begin{pgfscope}%
\pgfsys@transformshift{4.186054in}{0.402778in}%
\pgfsys@useobject{currentmarker}{}%
\end{pgfscope}%
\end{pgfscope}%
\begin{pgfscope}%
\pgfsetbuttcap%
\pgfsetroundjoin%
\definecolor{currentfill}{rgb}{0.000000,0.000000,0.000000}%
\pgfsetfillcolor{currentfill}%
\pgfsetlinewidth{0.602250pt}%
\definecolor{currentstroke}{rgb}{0.000000,0.000000,0.000000}%
\pgfsetstrokecolor{currentstroke}%
\pgfsetdash{}{0pt}%
\pgfsys@defobject{currentmarker}{\pgfqpoint{0.000000in}{-0.027778in}}{\pgfqpoint{0.000000in}{0.000000in}}{%
\pgfpathmoveto{\pgfqpoint{0.000000in}{0.000000in}}%
\pgfpathlineto{\pgfqpoint{0.000000in}{-0.027778in}}%
\pgfusepath{stroke,fill}%
}%
\begin{pgfscope}%
\pgfsys@transformshift{4.210217in}{0.402778in}%
\pgfsys@useobject{currentmarker}{}%
\end{pgfscope}%
\end{pgfscope}%
\begin{pgfscope}%
\pgfsetbuttcap%
\pgfsetroundjoin%
\definecolor{currentfill}{rgb}{0.000000,0.000000,0.000000}%
\pgfsetfillcolor{currentfill}%
\pgfsetlinewidth{0.602250pt}%
\definecolor{currentstroke}{rgb}{0.000000,0.000000,0.000000}%
\pgfsetstrokecolor{currentstroke}%
\pgfsetdash{}{0pt}%
\pgfsys@defobject{currentmarker}{\pgfqpoint{0.000000in}{-0.027778in}}{\pgfqpoint{0.000000in}{0.000000in}}{%
\pgfpathmoveto{\pgfqpoint{0.000000in}{0.000000in}}%
\pgfpathlineto{\pgfqpoint{0.000000in}{-0.027778in}}%
\pgfusepath{stroke,fill}%
}%
\begin{pgfscope}%
\pgfsys@transformshift{4.374025in}{0.402778in}%
\pgfsys@useobject{currentmarker}{}%
\end{pgfscope}%
\end{pgfscope}%
\begin{pgfscope}%
\pgfsetbuttcap%
\pgfsetroundjoin%
\definecolor{currentfill}{rgb}{0.000000,0.000000,0.000000}%
\pgfsetfillcolor{currentfill}%
\pgfsetlinewidth{0.602250pt}%
\definecolor{currentstroke}{rgb}{0.000000,0.000000,0.000000}%
\pgfsetstrokecolor{currentstroke}%
\pgfsetdash{}{0pt}%
\pgfsys@defobject{currentmarker}{\pgfqpoint{0.000000in}{-0.027778in}}{\pgfqpoint{0.000000in}{0.000000in}}{%
\pgfpathmoveto{\pgfqpoint{0.000000in}{0.000000in}}%
\pgfpathlineto{\pgfqpoint{0.000000in}{-0.027778in}}%
\pgfusepath{stroke,fill}%
}%
\begin{pgfscope}%
\pgfsys@transformshift{4.457203in}{0.402778in}%
\pgfsys@useobject{currentmarker}{}%
\end{pgfscope}%
\end{pgfscope}%
\begin{pgfscope}%
\pgfsetbuttcap%
\pgfsetroundjoin%
\definecolor{currentfill}{rgb}{0.000000,0.000000,0.000000}%
\pgfsetfillcolor{currentfill}%
\pgfsetlinewidth{0.602250pt}%
\definecolor{currentstroke}{rgb}{0.000000,0.000000,0.000000}%
\pgfsetstrokecolor{currentstroke}%
\pgfsetdash{}{0pt}%
\pgfsys@defobject{currentmarker}{\pgfqpoint{0.000000in}{-0.027778in}}{\pgfqpoint{0.000000in}{0.000000in}}{%
\pgfpathmoveto{\pgfqpoint{0.000000in}{0.000000in}}%
\pgfpathlineto{\pgfqpoint{0.000000in}{-0.027778in}}%
\pgfusepath{stroke,fill}%
}%
\begin{pgfscope}%
\pgfsys@transformshift{4.516219in}{0.402778in}%
\pgfsys@useobject{currentmarker}{}%
\end{pgfscope}%
\end{pgfscope}%
\begin{pgfscope}%
\pgfsetbuttcap%
\pgfsetroundjoin%
\definecolor{currentfill}{rgb}{0.000000,0.000000,0.000000}%
\pgfsetfillcolor{currentfill}%
\pgfsetlinewidth{0.602250pt}%
\definecolor{currentstroke}{rgb}{0.000000,0.000000,0.000000}%
\pgfsetstrokecolor{currentstroke}%
\pgfsetdash{}{0pt}%
\pgfsys@defobject{currentmarker}{\pgfqpoint{0.000000in}{-0.027778in}}{\pgfqpoint{0.000000in}{0.000000in}}{%
\pgfpathmoveto{\pgfqpoint{0.000000in}{0.000000in}}%
\pgfpathlineto{\pgfqpoint{0.000000in}{-0.027778in}}%
\pgfusepath{stroke,fill}%
}%
\begin{pgfscope}%
\pgfsys@transformshift{4.561995in}{0.402778in}%
\pgfsys@useobject{currentmarker}{}%
\end{pgfscope}%
\end{pgfscope}%
\begin{pgfscope}%
\pgfsetbuttcap%
\pgfsetroundjoin%
\definecolor{currentfill}{rgb}{0.000000,0.000000,0.000000}%
\pgfsetfillcolor{currentfill}%
\pgfsetlinewidth{0.602250pt}%
\definecolor{currentstroke}{rgb}{0.000000,0.000000,0.000000}%
\pgfsetstrokecolor{currentstroke}%
\pgfsetdash{}{0pt}%
\pgfsys@defobject{currentmarker}{\pgfqpoint{0.000000in}{-0.027778in}}{\pgfqpoint{0.000000in}{0.000000in}}{%
\pgfpathmoveto{\pgfqpoint{0.000000in}{0.000000in}}%
\pgfpathlineto{\pgfqpoint{0.000000in}{-0.027778in}}%
\pgfusepath{stroke,fill}%
}%
\begin{pgfscope}%
\pgfsys@transformshift{4.599397in}{0.402778in}%
\pgfsys@useobject{currentmarker}{}%
\end{pgfscope}%
\end{pgfscope}%
\begin{pgfscope}%
\pgfsetbuttcap%
\pgfsetroundjoin%
\definecolor{currentfill}{rgb}{0.000000,0.000000,0.000000}%
\pgfsetfillcolor{currentfill}%
\pgfsetlinewidth{0.602250pt}%
\definecolor{currentstroke}{rgb}{0.000000,0.000000,0.000000}%
\pgfsetstrokecolor{currentstroke}%
\pgfsetdash{}{0pt}%
\pgfsys@defobject{currentmarker}{\pgfqpoint{0.000000in}{-0.027778in}}{\pgfqpoint{0.000000in}{0.000000in}}{%
\pgfpathmoveto{\pgfqpoint{0.000000in}{0.000000in}}%
\pgfpathlineto{\pgfqpoint{0.000000in}{-0.027778in}}%
\pgfusepath{stroke,fill}%
}%
\begin{pgfscope}%
\pgfsys@transformshift{4.631020in}{0.402778in}%
\pgfsys@useobject{currentmarker}{}%
\end{pgfscope}%
\end{pgfscope}%
\begin{pgfscope}%
\pgfsetbuttcap%
\pgfsetroundjoin%
\definecolor{currentfill}{rgb}{0.000000,0.000000,0.000000}%
\pgfsetfillcolor{currentfill}%
\pgfsetlinewidth{0.602250pt}%
\definecolor{currentstroke}{rgb}{0.000000,0.000000,0.000000}%
\pgfsetstrokecolor{currentstroke}%
\pgfsetdash{}{0pt}%
\pgfsys@defobject{currentmarker}{\pgfqpoint{0.000000in}{-0.027778in}}{\pgfqpoint{0.000000in}{0.000000in}}{%
\pgfpathmoveto{\pgfqpoint{0.000000in}{0.000000in}}%
\pgfpathlineto{\pgfqpoint{0.000000in}{-0.027778in}}%
\pgfusepath{stroke,fill}%
}%
\begin{pgfscope}%
\pgfsys@transformshift{4.658413in}{0.402778in}%
\pgfsys@useobject{currentmarker}{}%
\end{pgfscope}%
\end{pgfscope}%
\begin{pgfscope}%
\pgfsetbuttcap%
\pgfsetroundjoin%
\definecolor{currentfill}{rgb}{0.000000,0.000000,0.000000}%
\pgfsetfillcolor{currentfill}%
\pgfsetlinewidth{0.602250pt}%
\definecolor{currentstroke}{rgb}{0.000000,0.000000,0.000000}%
\pgfsetstrokecolor{currentstroke}%
\pgfsetdash{}{0pt}%
\pgfsys@defobject{currentmarker}{\pgfqpoint{0.000000in}{-0.027778in}}{\pgfqpoint{0.000000in}{0.000000in}}{%
\pgfpathmoveto{\pgfqpoint{0.000000in}{0.000000in}}%
\pgfpathlineto{\pgfqpoint{0.000000in}{-0.027778in}}%
\pgfusepath{stroke,fill}%
}%
\begin{pgfscope}%
\pgfsys@transformshift{4.682575in}{0.402778in}%
\pgfsys@useobject{currentmarker}{}%
\end{pgfscope}%
\end{pgfscope}%
\begin{pgfscope}%
\pgfsetbuttcap%
\pgfsetroundjoin%
\definecolor{currentfill}{rgb}{0.000000,0.000000,0.000000}%
\pgfsetfillcolor{currentfill}%
\pgfsetlinewidth{0.602250pt}%
\definecolor{currentstroke}{rgb}{0.000000,0.000000,0.000000}%
\pgfsetstrokecolor{currentstroke}%
\pgfsetdash{}{0pt}%
\pgfsys@defobject{currentmarker}{\pgfqpoint{0.000000in}{-0.027778in}}{\pgfqpoint{0.000000in}{0.000000in}}{%
\pgfpathmoveto{\pgfqpoint{0.000000in}{0.000000in}}%
\pgfpathlineto{\pgfqpoint{0.000000in}{-0.027778in}}%
\pgfusepath{stroke,fill}%
}%
\begin{pgfscope}%
\pgfsys@transformshift{4.846383in}{0.402778in}%
\pgfsys@useobject{currentmarker}{}%
\end{pgfscope}%
\end{pgfscope}%
\begin{pgfscope}%
\pgfsetbuttcap%
\pgfsetroundjoin%
\definecolor{currentfill}{rgb}{0.000000,0.000000,0.000000}%
\pgfsetfillcolor{currentfill}%
\pgfsetlinewidth{0.602250pt}%
\definecolor{currentstroke}{rgb}{0.000000,0.000000,0.000000}%
\pgfsetstrokecolor{currentstroke}%
\pgfsetdash{}{0pt}%
\pgfsys@defobject{currentmarker}{\pgfqpoint{0.000000in}{-0.027778in}}{\pgfqpoint{0.000000in}{0.000000in}}{%
\pgfpathmoveto{\pgfqpoint{0.000000in}{0.000000in}}%
\pgfpathlineto{\pgfqpoint{0.000000in}{-0.027778in}}%
\pgfusepath{stroke,fill}%
}%
\begin{pgfscope}%
\pgfsys@transformshift{4.929562in}{0.402778in}%
\pgfsys@useobject{currentmarker}{}%
\end{pgfscope}%
\end{pgfscope}%
\begin{pgfscope}%
\pgftext[x=2.782165in,y=0.123861in,,top]{\rmfamily\fontsize{10.000000}{12.000000}\selectfont E}%
\end{pgfscope}%
\begin{pgfscope}%
\pgfsetbuttcap%
\pgfsetroundjoin%
\definecolor{currentfill}{rgb}{0.000000,0.000000,0.000000}%
\pgfsetfillcolor{currentfill}%
\pgfsetlinewidth{0.803000pt}%
\definecolor{currentstroke}{rgb}{0.000000,0.000000,0.000000}%
\pgfsetstrokecolor{currentstroke}%
\pgfsetdash{}{0pt}%
\pgfsys@defobject{currentmarker}{\pgfqpoint{-0.048611in}{0.000000in}}{\pgfqpoint{0.000000in}{0.000000in}}{%
\pgfpathmoveto{\pgfqpoint{0.000000in}{0.000000in}}%
\pgfpathlineto{\pgfqpoint{-0.048611in}{0.000000in}}%
\pgfusepath{stroke,fill}%
}%
\begin{pgfscope}%
\pgfsys@transformshift{0.612942in}{0.693477in}%
\pgfsys@useobject{currentmarker}{}%
\end{pgfscope}%
\end{pgfscope}%
\begin{pgfscope}%
\pgftext[x=0.172354in,y=0.645260in,left,base]{\rmfamily\fontsize{10.000000}{12.000000}\selectfont \(\displaystyle 10^{-19}\)}%
\end{pgfscope}%
\begin{pgfscope}%
\pgfsetbuttcap%
\pgfsetroundjoin%
\definecolor{currentfill}{rgb}{0.000000,0.000000,0.000000}%
\pgfsetfillcolor{currentfill}%
\pgfsetlinewidth{0.803000pt}%
\definecolor{currentstroke}{rgb}{0.000000,0.000000,0.000000}%
\pgfsetstrokecolor{currentstroke}%
\pgfsetdash{}{0pt}%
\pgfsys@defobject{currentmarker}{\pgfqpoint{-0.048611in}{0.000000in}}{\pgfqpoint{0.000000in}{0.000000in}}{%
\pgfpathmoveto{\pgfqpoint{0.000000in}{0.000000in}}%
\pgfpathlineto{\pgfqpoint{-0.048611in}{0.000000in}}%
\pgfusepath{stroke,fill}%
}%
\begin{pgfscope}%
\pgfsys@transformshift{0.612942in}{1.068292in}%
\pgfsys@useobject{currentmarker}{}%
\end{pgfscope}%
\end{pgfscope}%
\begin{pgfscope}%
\pgftext[x=0.172354in,y=1.020074in,left,base]{\rmfamily\fontsize{10.000000}{12.000000}\selectfont \(\displaystyle 10^{-16}\)}%
\end{pgfscope}%
\begin{pgfscope}%
\pgfsetbuttcap%
\pgfsetroundjoin%
\definecolor{currentfill}{rgb}{0.000000,0.000000,0.000000}%
\pgfsetfillcolor{currentfill}%
\pgfsetlinewidth{0.803000pt}%
\definecolor{currentstroke}{rgb}{0.000000,0.000000,0.000000}%
\pgfsetstrokecolor{currentstroke}%
\pgfsetdash{}{0pt}%
\pgfsys@defobject{currentmarker}{\pgfqpoint{-0.048611in}{0.000000in}}{\pgfqpoint{0.000000in}{0.000000in}}{%
\pgfpathmoveto{\pgfqpoint{0.000000in}{0.000000in}}%
\pgfpathlineto{\pgfqpoint{-0.048611in}{0.000000in}}%
\pgfusepath{stroke,fill}%
}%
\begin{pgfscope}%
\pgfsys@transformshift{0.612942in}{1.443106in}%
\pgfsys@useobject{currentmarker}{}%
\end{pgfscope}%
\end{pgfscope}%
\begin{pgfscope}%
\pgftext[x=0.172354in,y=1.394888in,left,base]{\rmfamily\fontsize{10.000000}{12.000000}\selectfont \(\displaystyle 10^{-13}\)}%
\end{pgfscope}%
\begin{pgfscope}%
\pgfsetbuttcap%
\pgfsetroundjoin%
\definecolor{currentfill}{rgb}{0.000000,0.000000,0.000000}%
\pgfsetfillcolor{currentfill}%
\pgfsetlinewidth{0.803000pt}%
\definecolor{currentstroke}{rgb}{0.000000,0.000000,0.000000}%
\pgfsetstrokecolor{currentstroke}%
\pgfsetdash{}{0pt}%
\pgfsys@defobject{currentmarker}{\pgfqpoint{-0.048611in}{0.000000in}}{\pgfqpoint{0.000000in}{0.000000in}}{%
\pgfpathmoveto{\pgfqpoint{0.000000in}{0.000000in}}%
\pgfpathlineto{\pgfqpoint{-0.048611in}{0.000000in}}%
\pgfusepath{stroke,fill}%
}%
\begin{pgfscope}%
\pgfsys@transformshift{0.612942in}{1.817920in}%
\pgfsys@useobject{currentmarker}{}%
\end{pgfscope}%
\end{pgfscope}%
\begin{pgfscope}%
\pgftext[x=0.172354in,y=1.769703in,left,base]{\rmfamily\fontsize{10.000000}{12.000000}\selectfont \(\displaystyle 10^{-10}\)}%
\end{pgfscope}%
\begin{pgfscope}%
\pgfsetbuttcap%
\pgfsetroundjoin%
\definecolor{currentfill}{rgb}{0.000000,0.000000,0.000000}%
\pgfsetfillcolor{currentfill}%
\pgfsetlinewidth{0.803000pt}%
\definecolor{currentstroke}{rgb}{0.000000,0.000000,0.000000}%
\pgfsetstrokecolor{currentstroke}%
\pgfsetdash{}{0pt}%
\pgfsys@defobject{currentmarker}{\pgfqpoint{-0.048611in}{0.000000in}}{\pgfqpoint{0.000000in}{0.000000in}}{%
\pgfpathmoveto{\pgfqpoint{0.000000in}{0.000000in}}%
\pgfpathlineto{\pgfqpoint{-0.048611in}{0.000000in}}%
\pgfusepath{stroke,fill}%
}%
\begin{pgfscope}%
\pgfsys@transformshift{0.612942in}{2.192735in}%
\pgfsys@useobject{currentmarker}{}%
\end{pgfscope}%
\end{pgfscope}%
\begin{pgfscope}%
\pgftext[x=0.227717in,y=2.144517in,left,base]{\rmfamily\fontsize{10.000000}{12.000000}\selectfont \(\displaystyle 10^{-7}\)}%
\end{pgfscope}%
\begin{pgfscope}%
\pgfsetbuttcap%
\pgfsetroundjoin%
\definecolor{currentfill}{rgb}{0.000000,0.000000,0.000000}%
\pgfsetfillcolor{currentfill}%
\pgfsetlinewidth{0.803000pt}%
\definecolor{currentstroke}{rgb}{0.000000,0.000000,0.000000}%
\pgfsetstrokecolor{currentstroke}%
\pgfsetdash{}{0pt}%
\pgfsys@defobject{currentmarker}{\pgfqpoint{-0.048611in}{0.000000in}}{\pgfqpoint{0.000000in}{0.000000in}}{%
\pgfpathmoveto{\pgfqpoint{0.000000in}{0.000000in}}%
\pgfpathlineto{\pgfqpoint{-0.048611in}{0.000000in}}%
\pgfusepath{stroke,fill}%
}%
\begin{pgfscope}%
\pgfsys@transformshift{0.612942in}{2.567549in}%
\pgfsys@useobject{currentmarker}{}%
\end{pgfscope}%
\end{pgfscope}%
\begin{pgfscope}%
\pgftext[x=0.227717in,y=2.519331in,left,base]{\rmfamily\fontsize{10.000000}{12.000000}\selectfont \(\displaystyle 10^{-4}\)}%
\end{pgfscope}%
\begin{pgfscope}%
\pgfsetbuttcap%
\pgfsetroundjoin%
\definecolor{currentfill}{rgb}{0.000000,0.000000,0.000000}%
\pgfsetfillcolor{currentfill}%
\pgfsetlinewidth{0.803000pt}%
\definecolor{currentstroke}{rgb}{0.000000,0.000000,0.000000}%
\pgfsetstrokecolor{currentstroke}%
\pgfsetdash{}{0pt}%
\pgfsys@defobject{currentmarker}{\pgfqpoint{-0.048611in}{0.000000in}}{\pgfqpoint{0.000000in}{0.000000in}}{%
\pgfpathmoveto{\pgfqpoint{0.000000in}{0.000000in}}%
\pgfpathlineto{\pgfqpoint{-0.048611in}{0.000000in}}%
\pgfusepath{stroke,fill}%
}%
\begin{pgfscope}%
\pgfsys@transformshift{0.612942in}{2.942363in}%
\pgfsys@useobject{currentmarker}{}%
\end{pgfscope}%
\end{pgfscope}%
\begin{pgfscope}%
\pgftext[x=0.227717in,y=2.894145in,left,base]{\rmfamily\fontsize{10.000000}{12.000000}\selectfont \(\displaystyle 10^{-1}\)}%
\end{pgfscope}%
\begin{pgfscope}%
\pgftext[x=0.116799in,y=1.718750in,,bottom,rotate=90.000000]{\rmfamily\fontsize{10.000000}{12.000000}\selectfont Max Korrektur}%
\end{pgfscope}%
\begin{pgfscope}%
\pgfsetrectcap%
\pgfsetmiterjoin%
\pgfsetlinewidth{0.803000pt}%
\definecolor{currentstroke}{rgb}{0.000000,0.000000,0.000000}%
\pgfsetstrokecolor{currentstroke}%
\pgfsetdash{}{0pt}%
\pgfpathmoveto{\pgfqpoint{0.612942in}{0.402778in}}%
\pgfpathlineto{\pgfqpoint{0.612942in}{3.034722in}}%
\pgfusepath{stroke}%
\end{pgfscope}%
\begin{pgfscope}%
\pgfsetrectcap%
\pgfsetmiterjoin%
\pgfsetlinewidth{0.803000pt}%
\definecolor{currentstroke}{rgb}{0.000000,0.000000,0.000000}%
\pgfsetstrokecolor{currentstroke}%
\pgfsetdash{}{0pt}%
\pgfpathmoveto{\pgfqpoint{4.951389in}{0.402778in}}%
\pgfpathlineto{\pgfqpoint{4.951389in}{3.034722in}}%
\pgfusepath{stroke}%
\end{pgfscope}%
\begin{pgfscope}%
\pgfsetrectcap%
\pgfsetmiterjoin%
\pgfsetlinewidth{0.803000pt}%
\definecolor{currentstroke}{rgb}{0.000000,0.000000,0.000000}%
\pgfsetstrokecolor{currentstroke}%
\pgfsetdash{}{0pt}%
\pgfpathmoveto{\pgfqpoint{0.612942in}{0.402778in}}%
\pgfpathlineto{\pgfqpoint{4.951389in}{0.402778in}}%
\pgfusepath{stroke}%
\end{pgfscope}%
\begin{pgfscope}%
\pgfsetrectcap%
\pgfsetmiterjoin%
\pgfsetlinewidth{0.803000pt}%
\definecolor{currentstroke}{rgb}{0.000000,0.000000,0.000000}%
\pgfsetstrokecolor{currentstroke}%
\pgfsetdash{}{0pt}%
\pgfpathmoveto{\pgfqpoint{0.612942in}{3.034722in}}%
\pgfpathlineto{\pgfqpoint{4.951389in}{3.034722in}}%
\pgfusepath{stroke}%
\end{pgfscope}%
\begin{pgfscope}%
\pgfsetbuttcap%
\pgfsetmiterjoin%
\definecolor{currentfill}{rgb}{1.000000,1.000000,1.000000}%
\pgfsetfillcolor{currentfill}%
\pgfsetfillopacity{0.800000}%
\pgfsetlinewidth{1.003750pt}%
\definecolor{currentstroke}{rgb}{0.800000,0.800000,0.800000}%
\pgfsetstrokecolor{currentstroke}%
\pgfsetstrokeopacity{0.800000}%
\pgfsetdash{}{0pt}%
\pgfpathmoveto{\pgfqpoint{0.710164in}{2.334500in}}%
\pgfpathlineto{\pgfqpoint{1.571275in}{2.334500in}}%
\pgfpathquadraticcurveto{\pgfqpoint{1.599053in}{2.334500in}}{\pgfqpoint{1.599053in}{2.362278in}}%
\pgfpathlineto{\pgfqpoint{1.599053in}{2.937500in}}%
\pgfpathquadraticcurveto{\pgfqpoint{1.599053in}{2.965278in}}{\pgfqpoint{1.571275in}{2.965278in}}%
\pgfpathlineto{\pgfqpoint{0.710164in}{2.965278in}}%
\pgfpathquadraticcurveto{\pgfqpoint{0.682386in}{2.965278in}}{\pgfqpoint{0.682386in}{2.937500in}}%
\pgfpathlineto{\pgfqpoint{0.682386in}{2.362278in}}%
\pgfpathquadraticcurveto{\pgfqpoint{0.682386in}{2.334500in}}{\pgfqpoint{0.710164in}{2.334500in}}%
\pgfpathclose%
\pgfusepath{stroke,fill}%
\end{pgfscope}%
\begin{pgfscope}%
\pgfsetbuttcap%
\pgfsetroundjoin%
\definecolor{currentfill}{rgb}{0.121569,0.466667,0.705882}%
\pgfsetfillcolor{currentfill}%
\pgfsetfillopacity{0.500000}%
\pgfsetlinewidth{1.003750pt}%
\definecolor{currentstroke}{rgb}{0.121569,0.466667,0.705882}%
\pgfsetstrokecolor{currentstroke}%
\pgfsetstrokeopacity{0.500000}%
\pgfsetdash{}{0pt}%
\pgfpathmoveto{\pgfqpoint{0.876831in}{2.807292in}}%
\pgfpathcurveto{\pgfqpoint{0.887881in}{2.807292in}}{\pgfqpoint{0.898480in}{2.811682in}}{\pgfqpoint{0.906293in}{2.819496in}}%
\pgfpathcurveto{\pgfqpoint{0.914107in}{2.827309in}}{\pgfqpoint{0.918497in}{2.837908in}}{\pgfqpoint{0.918497in}{2.848958in}}%
\pgfpathcurveto{\pgfqpoint{0.918497in}{2.860008in}}{\pgfqpoint{0.914107in}{2.870607in}}{\pgfqpoint{0.906293in}{2.878421in}}%
\pgfpathcurveto{\pgfqpoint{0.898480in}{2.886235in}}{\pgfqpoint{0.887881in}{2.890625in}}{\pgfqpoint{0.876831in}{2.890625in}}%
\pgfpathcurveto{\pgfqpoint{0.865781in}{2.890625in}}{\pgfqpoint{0.855182in}{2.886235in}}{\pgfqpoint{0.847368in}{2.878421in}}%
\pgfpathcurveto{\pgfqpoint{0.839554in}{2.870607in}}{\pgfqpoint{0.835164in}{2.860008in}}{\pgfqpoint{0.835164in}{2.848958in}}%
\pgfpathcurveto{\pgfqpoint{0.835164in}{2.837908in}}{\pgfqpoint{0.839554in}{2.827309in}}{\pgfqpoint{0.847368in}{2.819496in}}%
\pgfpathcurveto{\pgfqpoint{0.855182in}{2.811682in}}{\pgfqpoint{0.865781in}{2.807292in}}{\pgfqpoint{0.876831in}{2.807292in}}%
\pgfpathclose%
\pgfusepath{stroke,fill}%
\end{pgfscope}%
\begin{pgfscope}%
\pgftext[x=1.126831in,y=2.812500in,left,base]{\rmfamily\fontsize{10.000000}{12.000000}\selectfont 16384}%
\end{pgfscope}%
\begin{pgfscope}%
\pgfsetbuttcap%
\pgfsetroundjoin%
\definecolor{currentfill}{rgb}{1.000000,0.498039,0.054902}%
\pgfsetfillcolor{currentfill}%
\pgfsetfillopacity{0.500000}%
\pgfsetlinewidth{1.003750pt}%
\definecolor{currentstroke}{rgb}{1.000000,0.498039,0.054902}%
\pgfsetstrokecolor{currentstroke}%
\pgfsetstrokeopacity{0.500000}%
\pgfsetdash{}{0pt}%
\pgfpathmoveto{\pgfqpoint{0.876831in}{2.610921in}}%
\pgfpathcurveto{\pgfqpoint{0.887881in}{2.610921in}}{\pgfqpoint{0.898480in}{2.615312in}}{\pgfqpoint{0.906293in}{2.623125in}}%
\pgfpathcurveto{\pgfqpoint{0.914107in}{2.630939in}}{\pgfqpoint{0.918497in}{2.641538in}}{\pgfqpoint{0.918497in}{2.652588in}}%
\pgfpathcurveto{\pgfqpoint{0.918497in}{2.663638in}}{\pgfqpoint{0.914107in}{2.674237in}}{\pgfqpoint{0.906293in}{2.682051in}}%
\pgfpathcurveto{\pgfqpoint{0.898480in}{2.689864in}}{\pgfqpoint{0.887881in}{2.694255in}}{\pgfqpoint{0.876831in}{2.694255in}}%
\pgfpathcurveto{\pgfqpoint{0.865781in}{2.694255in}}{\pgfqpoint{0.855182in}{2.689864in}}{\pgfqpoint{0.847368in}{2.682051in}}%
\pgfpathcurveto{\pgfqpoint{0.839554in}{2.674237in}}{\pgfqpoint{0.835164in}{2.663638in}}{\pgfqpoint{0.835164in}{2.652588in}}%
\pgfpathcurveto{\pgfqpoint{0.835164in}{2.641538in}}{\pgfqpoint{0.839554in}{2.630939in}}{\pgfqpoint{0.847368in}{2.623125in}}%
\pgfpathcurveto{\pgfqpoint{0.855182in}{2.615312in}}{\pgfqpoint{0.865781in}{2.610921in}}{\pgfqpoint{0.876831in}{2.610921in}}%
\pgfpathclose%
\pgfusepath{stroke,fill}%
\end{pgfscope}%
\begin{pgfscope}%
\pgftext[x=1.126831in,y=2.616130in,left,base]{\rmfamily\fontsize{10.000000}{12.000000}\selectfont 65536}%
\end{pgfscope}%
\begin{pgfscope}%
\pgfsetbuttcap%
\pgfsetroundjoin%
\definecolor{currentfill}{rgb}{0.172549,0.627451,0.172549}%
\pgfsetfillcolor{currentfill}%
\pgfsetfillopacity{0.500000}%
\pgfsetlinewidth{1.003750pt}%
\definecolor{currentstroke}{rgb}{0.172549,0.627451,0.172549}%
\pgfsetstrokecolor{currentstroke}%
\pgfsetstrokeopacity{0.500000}%
\pgfsetdash{}{0pt}%
\pgfpathmoveto{\pgfqpoint{0.876831in}{2.414551in}}%
\pgfpathcurveto{\pgfqpoint{0.887881in}{2.414551in}}{\pgfqpoint{0.898480in}{2.418941in}}{\pgfqpoint{0.906293in}{2.426755in}}%
\pgfpathcurveto{\pgfqpoint{0.914107in}{2.434569in}}{\pgfqpoint{0.918497in}{2.445168in}}{\pgfqpoint{0.918497in}{2.456218in}}%
\pgfpathcurveto{\pgfqpoint{0.918497in}{2.467268in}}{\pgfqpoint{0.914107in}{2.477867in}}{\pgfqpoint{0.906293in}{2.485681in}}%
\pgfpathcurveto{\pgfqpoint{0.898480in}{2.493494in}}{\pgfqpoint{0.887881in}{2.497884in}}{\pgfqpoint{0.876831in}{2.497884in}}%
\pgfpathcurveto{\pgfqpoint{0.865781in}{2.497884in}}{\pgfqpoint{0.855182in}{2.493494in}}{\pgfqpoint{0.847368in}{2.485681in}}%
\pgfpathcurveto{\pgfqpoint{0.839554in}{2.477867in}}{\pgfqpoint{0.835164in}{2.467268in}}{\pgfqpoint{0.835164in}{2.456218in}}%
\pgfpathcurveto{\pgfqpoint{0.835164in}{2.445168in}}{\pgfqpoint{0.839554in}{2.434569in}}{\pgfqpoint{0.847368in}{2.426755in}}%
\pgfpathcurveto{\pgfqpoint{0.855182in}{2.418941in}}{\pgfqpoint{0.865781in}{2.414551in}}{\pgfqpoint{0.876831in}{2.414551in}}%
\pgfpathclose%
\pgfusepath{stroke,fill}%
\end{pgfscope}%
\begin{pgfscope}%
\pgftext[x=1.126831in,y=2.419759in,left,base]{\rmfamily\fontsize{10.000000}{12.000000}\selectfont 262144}%
\end{pgfscope}%
\end{pgfpicture}%
\makeatother%
\endgroup%

    \caption{Feste Anzahl der Schritte mit verschieden Iterationszahlen; Vergleich von globalem Fehler und größter Korrektur}
    \label{fig:error_corr}
\end{figure}
\begin{figure}[ht]
    \centering
        %% Creator: Matplotlib, PGF backend
%%
%% To include the figure in your LaTeX document, write
%%   \input{<filename>.pgf}
%%
%% Make sure the required packages are loaded in your preamble
%%   \usepackage{pgf}
%%
%% Figures using additional raster images can only be included by \input if
%% they are in the same directory as the main LaTeX file. For loading figures
%% from other directories you can use the `import` package
%%   \usepackage{import}
%% and then include the figures with
%%   \import{<path to file>}{<filename>.pgf}
%%
%% Matplotlib used the following preamble
%%   \usepackage{fontspec}
%%   \setmainfont{Times New Roman}
%%   \setsansfont{Lucida Grande}
%%   \setmonofont{Andale Mono}
%%
\begingroup%
\makeatletter%
\begin{pgfpicture}%
\pgfpathrectangle{\pgfpointorigin}{\pgfqpoint{5.000000in}{3.083333in}}%
\pgfusepath{use as bounding box, clip}%
\begin{pgfscope}%
\pgfsetbuttcap%
\pgfsetmiterjoin%
\definecolor{currentfill}{rgb}{1.000000,1.000000,1.000000}%
\pgfsetfillcolor{currentfill}%
\pgfsetlinewidth{0.000000pt}%
\definecolor{currentstroke}{rgb}{1.000000,1.000000,1.000000}%
\pgfsetstrokecolor{currentstroke}%
\pgfsetdash{}{0pt}%
\pgfpathmoveto{\pgfqpoint{0.000000in}{0.000000in}}%
\pgfpathlineto{\pgfqpoint{5.000000in}{0.000000in}}%
\pgfpathlineto{\pgfqpoint{5.000000in}{3.083333in}}%
\pgfpathlineto{\pgfqpoint{0.000000in}{3.083333in}}%
\pgfpathclose%
\pgfusepath{fill}%
\end{pgfscope}%
\begin{pgfscope}%
\pgfsetbuttcap%
\pgfsetmiterjoin%
\definecolor{currentfill}{rgb}{1.000000,1.000000,1.000000}%
\pgfsetfillcolor{currentfill}%
\pgfsetlinewidth{0.000000pt}%
\definecolor{currentstroke}{rgb}{0.000000,0.000000,0.000000}%
\pgfsetstrokecolor{currentstroke}%
\pgfsetstrokeopacity{0.000000}%
\pgfsetdash{}{0pt}%
\pgfpathmoveto{\pgfqpoint{0.557786in}{0.402778in}}%
\pgfpathlineto{\pgfqpoint{4.951389in}{0.402778in}}%
\pgfpathlineto{\pgfqpoint{4.951389in}{3.034722in}}%
\pgfpathlineto{\pgfqpoint{0.557786in}{3.034722in}}%
\pgfpathclose%
\pgfusepath{fill}%
\end{pgfscope}%
\begin{pgfscope}%
\pgfpathrectangle{\pgfqpoint{0.557786in}{0.402778in}}{\pgfqpoint{4.393603in}{2.631944in}}%
\pgfusepath{clip}%
\pgfsetbuttcap%
\pgfsetroundjoin%
\definecolor{currentfill}{rgb}{0.121569,0.466667,0.705882}%
\pgfsetfillcolor{currentfill}%
\pgfsetfillopacity{0.500000}%
\pgfsetlinewidth{1.003750pt}%
\definecolor{currentstroke}{rgb}{0.121569,0.466667,0.705882}%
\pgfsetstrokecolor{currentstroke}%
\pgfsetstrokeopacity{0.500000}%
\pgfsetdash{}{0pt}%
\pgfpathmoveto{\pgfqpoint{4.746022in}{2.789753in}}%
\pgfpathcurveto{\pgfqpoint{4.757072in}{2.789753in}}{\pgfqpoint{4.767671in}{2.794144in}}{\pgfqpoint{4.775484in}{2.801957in}}%
\pgfpathcurveto{\pgfqpoint{4.783298in}{2.809771in}}{\pgfqpoint{4.787688in}{2.820370in}}{\pgfqpoint{4.787688in}{2.831420in}}%
\pgfpathcurveto{\pgfqpoint{4.787688in}{2.842470in}}{\pgfqpoint{4.783298in}{2.853069in}}{\pgfqpoint{4.775484in}{2.860883in}}%
\pgfpathcurveto{\pgfqpoint{4.767671in}{2.868697in}}{\pgfqpoint{4.757072in}{2.873087in}}{\pgfqpoint{4.746022in}{2.873087in}}%
\pgfpathcurveto{\pgfqpoint{4.734971in}{2.873087in}}{\pgfqpoint{4.724372in}{2.868697in}}{\pgfqpoint{4.716559in}{2.860883in}}%
\pgfpathcurveto{\pgfqpoint{4.708745in}{2.853069in}}{\pgfqpoint{4.704355in}{2.842470in}}{\pgfqpoint{4.704355in}{2.831420in}}%
\pgfpathcurveto{\pgfqpoint{4.704355in}{2.820370in}}{\pgfqpoint{4.708745in}{2.809771in}}{\pgfqpoint{4.716559in}{2.801957in}}%
\pgfpathcurveto{\pgfqpoint{4.724372in}{2.794144in}}{\pgfqpoint{4.734971in}{2.789753in}}{\pgfqpoint{4.746022in}{2.789753in}}%
\pgfpathclose%
\pgfusepath{stroke,fill}%
\end{pgfscope}%
\begin{pgfscope}%
\pgfpathrectangle{\pgfqpoint{0.557786in}{0.402778in}}{\pgfqpoint{4.393603in}{2.631944in}}%
\pgfusepath{clip}%
\pgfsetbuttcap%
\pgfsetroundjoin%
\definecolor{currentfill}{rgb}{0.121569,0.466667,0.705882}%
\pgfsetfillcolor{currentfill}%
\pgfsetfillopacity{0.500000}%
\pgfsetlinewidth{1.003750pt}%
\definecolor{currentstroke}{rgb}{0.121569,0.466667,0.705882}%
\pgfsetstrokecolor{currentstroke}%
\pgfsetstrokeopacity{0.500000}%
\pgfsetdash{}{0pt}%
\pgfpathmoveto{\pgfqpoint{3.789138in}{2.096919in}}%
\pgfpathcurveto{\pgfqpoint{3.800188in}{2.096919in}}{\pgfqpoint{3.810787in}{2.101310in}}{\pgfqpoint{3.818601in}{2.109123in}}%
\pgfpathcurveto{\pgfqpoint{3.826414in}{2.116937in}}{\pgfqpoint{3.830805in}{2.127536in}}{\pgfqpoint{3.830805in}{2.138586in}}%
\pgfpathcurveto{\pgfqpoint{3.830805in}{2.149636in}}{\pgfqpoint{3.826414in}{2.160235in}}{\pgfqpoint{3.818601in}{2.168049in}}%
\pgfpathcurveto{\pgfqpoint{3.810787in}{2.175862in}}{\pgfqpoint{3.800188in}{2.180253in}}{\pgfqpoint{3.789138in}{2.180253in}}%
\pgfpathcurveto{\pgfqpoint{3.778088in}{2.180253in}}{\pgfqpoint{3.767489in}{2.175862in}}{\pgfqpoint{3.759675in}{2.168049in}}%
\pgfpathcurveto{\pgfqpoint{3.751862in}{2.160235in}}{\pgfqpoint{3.747471in}{2.149636in}}{\pgfqpoint{3.747471in}{2.138586in}}%
\pgfpathcurveto{\pgfqpoint{3.747471in}{2.127536in}}{\pgfqpoint{3.751862in}{2.116937in}}{\pgfqpoint{3.759675in}{2.109123in}}%
\pgfpathcurveto{\pgfqpoint{3.767489in}{2.101310in}}{\pgfqpoint{3.778088in}{2.096919in}}{\pgfqpoint{3.789138in}{2.096919in}}%
\pgfpathclose%
\pgfusepath{stroke,fill}%
\end{pgfscope}%
\begin{pgfscope}%
\pgfpathrectangle{\pgfqpoint{0.557786in}{0.402778in}}{\pgfqpoint{4.393603in}{2.631944in}}%
\pgfusepath{clip}%
\pgfsetbuttcap%
\pgfsetroundjoin%
\definecolor{currentfill}{rgb}{0.121569,0.466667,0.705882}%
\pgfsetfillcolor{currentfill}%
\pgfsetfillopacity{0.500000}%
\pgfsetlinewidth{1.003750pt}%
\definecolor{currentstroke}{rgb}{0.121569,0.466667,0.705882}%
\pgfsetstrokecolor{currentstroke}%
\pgfsetstrokeopacity{0.500000}%
\pgfsetdash{}{0pt}%
\pgfpathmoveto{\pgfqpoint{2.974533in}{1.820424in}}%
\pgfpathcurveto{\pgfqpoint{2.985583in}{1.820424in}}{\pgfqpoint{2.996182in}{1.824814in}}{\pgfqpoint{3.003996in}{1.832628in}}%
\pgfpathcurveto{\pgfqpoint{3.011810in}{1.840441in}}{\pgfqpoint{3.016200in}{1.851040in}}{\pgfqpoint{3.016200in}{1.862090in}}%
\pgfpathcurveto{\pgfqpoint{3.016200in}{1.873140in}}{\pgfqpoint{3.011810in}{1.883739in}}{\pgfqpoint{3.003996in}{1.891553in}}%
\pgfpathcurveto{\pgfqpoint{2.996182in}{1.899367in}}{\pgfqpoint{2.985583in}{1.903757in}}{\pgfqpoint{2.974533in}{1.903757in}}%
\pgfpathcurveto{\pgfqpoint{2.963483in}{1.903757in}}{\pgfqpoint{2.952884in}{1.899367in}}{\pgfqpoint{2.945070in}{1.891553in}}%
\pgfpathcurveto{\pgfqpoint{2.937257in}{1.883739in}}{\pgfqpoint{2.932866in}{1.873140in}}{\pgfqpoint{2.932866in}{1.862090in}}%
\pgfpathcurveto{\pgfqpoint{2.932866in}{1.851040in}}{\pgfqpoint{2.937257in}{1.840441in}}{\pgfqpoint{2.945070in}{1.832628in}}%
\pgfpathcurveto{\pgfqpoint{2.952884in}{1.824814in}}{\pgfqpoint{2.963483in}{1.820424in}}{\pgfqpoint{2.974533in}{1.820424in}}%
\pgfpathclose%
\pgfusepath{stroke,fill}%
\end{pgfscope}%
\begin{pgfscope}%
\pgfpathrectangle{\pgfqpoint{0.557786in}{0.402778in}}{\pgfqpoint{4.393603in}{2.631944in}}%
\pgfusepath{clip}%
\pgfsetbuttcap%
\pgfsetroundjoin%
\definecolor{currentfill}{rgb}{0.121569,0.466667,0.705882}%
\pgfsetfillcolor{currentfill}%
\pgfsetfillopacity{0.500000}%
\pgfsetlinewidth{1.003750pt}%
\definecolor{currentstroke}{rgb}{0.121569,0.466667,0.705882}%
\pgfsetstrokecolor{currentstroke}%
\pgfsetstrokeopacity{0.500000}%
\pgfsetdash{}{0pt}%
\pgfpathmoveto{\pgfqpoint{2.286012in}{1.822148in}}%
\pgfpathcurveto{\pgfqpoint{2.297062in}{1.822148in}}{\pgfqpoint{2.307661in}{1.826538in}}{\pgfqpoint{2.315475in}{1.834352in}}%
\pgfpathcurveto{\pgfqpoint{2.323289in}{1.842165in}}{\pgfqpoint{2.327679in}{1.852764in}}{\pgfqpoint{2.327679in}{1.863814in}}%
\pgfpathcurveto{\pgfqpoint{2.327679in}{1.874865in}}{\pgfqpoint{2.323289in}{1.885464in}}{\pgfqpoint{2.315475in}{1.893277in}}%
\pgfpathcurveto{\pgfqpoint{2.307661in}{1.901091in}}{\pgfqpoint{2.297062in}{1.905481in}}{\pgfqpoint{2.286012in}{1.905481in}}%
\pgfpathcurveto{\pgfqpoint{2.274962in}{1.905481in}}{\pgfqpoint{2.264363in}{1.901091in}}{\pgfqpoint{2.256549in}{1.893277in}}%
\pgfpathcurveto{\pgfqpoint{2.248736in}{1.885464in}}{\pgfqpoint{2.244346in}{1.874865in}}{\pgfqpoint{2.244346in}{1.863814in}}%
\pgfpathcurveto{\pgfqpoint{2.244346in}{1.852764in}}{\pgfqpoint{2.248736in}{1.842165in}}{\pgfqpoint{2.256549in}{1.834352in}}%
\pgfpathcurveto{\pgfqpoint{2.264363in}{1.826538in}}{\pgfqpoint{2.274962in}{1.822148in}}{\pgfqpoint{2.286012in}{1.822148in}}%
\pgfpathclose%
\pgfusepath{stroke,fill}%
\end{pgfscope}%
\begin{pgfscope}%
\pgfpathrectangle{\pgfqpoint{0.557786in}{0.402778in}}{\pgfqpoint{4.393603in}{2.631944in}}%
\pgfusepath{clip}%
\pgfsetbuttcap%
\pgfsetroundjoin%
\definecolor{currentfill}{rgb}{0.121569,0.466667,0.705882}%
\pgfsetfillcolor{currentfill}%
\pgfsetfillopacity{0.500000}%
\pgfsetlinewidth{1.003750pt}%
\definecolor{currentstroke}{rgb}{0.121569,0.466667,0.705882}%
\pgfsetstrokecolor{currentstroke}%
\pgfsetstrokeopacity{0.500000}%
\pgfsetdash{}{0pt}%
\pgfpathmoveto{\pgfqpoint{2.237075in}{1.822204in}}%
\pgfpathcurveto{\pgfqpoint{2.248125in}{1.822204in}}{\pgfqpoint{2.258724in}{1.826594in}}{\pgfqpoint{2.266537in}{1.834407in}}%
\pgfpathcurveto{\pgfqpoint{2.274351in}{1.842221in}}{\pgfqpoint{2.278741in}{1.852820in}}{\pgfqpoint{2.278741in}{1.863870in}}%
\pgfpathcurveto{\pgfqpoint{2.278741in}{1.874920in}}{\pgfqpoint{2.274351in}{1.885519in}}{\pgfqpoint{2.266537in}{1.893333in}}%
\pgfpathcurveto{\pgfqpoint{2.258724in}{1.901147in}}{\pgfqpoint{2.248125in}{1.905537in}}{\pgfqpoint{2.237075in}{1.905537in}}%
\pgfpathcurveto{\pgfqpoint{2.226025in}{1.905537in}}{\pgfqpoint{2.215426in}{1.901147in}}{\pgfqpoint{2.207612in}{1.893333in}}%
\pgfpathcurveto{\pgfqpoint{2.199798in}{1.885519in}}{\pgfqpoint{2.195408in}{1.874920in}}{\pgfqpoint{2.195408in}{1.863870in}}%
\pgfpathcurveto{\pgfqpoint{2.195408in}{1.852820in}}{\pgfqpoint{2.199798in}{1.842221in}}{\pgfqpoint{2.207612in}{1.834407in}}%
\pgfpathcurveto{\pgfqpoint{2.215426in}{1.826594in}}{\pgfqpoint{2.226025in}{1.822204in}}{\pgfqpoint{2.237075in}{1.822204in}}%
\pgfpathclose%
\pgfusepath{stroke,fill}%
\end{pgfscope}%
\begin{pgfscope}%
\pgfpathrectangle{\pgfqpoint{0.557786in}{0.402778in}}{\pgfqpoint{4.393603in}{2.631944in}}%
\pgfusepath{clip}%
\pgfsetbuttcap%
\pgfsetroundjoin%
\definecolor{currentfill}{rgb}{0.121569,0.466667,0.705882}%
\pgfsetfillcolor{currentfill}%
\pgfsetfillopacity{0.500000}%
\pgfsetlinewidth{1.003750pt}%
\definecolor{currentstroke}{rgb}{0.121569,0.466667,0.705882}%
\pgfsetstrokecolor{currentstroke}%
\pgfsetstrokeopacity{0.500000}%
\pgfsetdash{}{0pt}%
\pgfpathmoveto{\pgfqpoint{2.236690in}{1.822203in}}%
\pgfpathcurveto{\pgfqpoint{2.247740in}{1.822203in}}{\pgfqpoint{2.258339in}{1.826593in}}{\pgfqpoint{2.266153in}{1.834407in}}%
\pgfpathcurveto{\pgfqpoint{2.273966in}{1.842220in}}{\pgfqpoint{2.278357in}{1.852819in}}{\pgfqpoint{2.278357in}{1.863869in}}%
\pgfpathcurveto{\pgfqpoint{2.278357in}{1.874920in}}{\pgfqpoint{2.273966in}{1.885519in}}{\pgfqpoint{2.266153in}{1.893332in}}%
\pgfpathcurveto{\pgfqpoint{2.258339in}{1.901146in}}{\pgfqpoint{2.247740in}{1.905536in}}{\pgfqpoint{2.236690in}{1.905536in}}%
\pgfpathcurveto{\pgfqpoint{2.225640in}{1.905536in}}{\pgfqpoint{2.215041in}{1.901146in}}{\pgfqpoint{2.207227in}{1.893332in}}%
\pgfpathcurveto{\pgfqpoint{2.199414in}{1.885519in}}{\pgfqpoint{2.195023in}{1.874920in}}{\pgfqpoint{2.195023in}{1.863869in}}%
\pgfpathcurveto{\pgfqpoint{2.195023in}{1.852819in}}{\pgfqpoint{2.199414in}{1.842220in}}{\pgfqpoint{2.207227in}{1.834407in}}%
\pgfpathcurveto{\pgfqpoint{2.215041in}{1.826593in}}{\pgfqpoint{2.225640in}{1.822203in}}{\pgfqpoint{2.236690in}{1.822203in}}%
\pgfpathclose%
\pgfusepath{stroke,fill}%
\end{pgfscope}%
\begin{pgfscope}%
\pgfpathrectangle{\pgfqpoint{0.557786in}{0.402778in}}{\pgfqpoint{4.393603in}{2.631944in}}%
\pgfusepath{clip}%
\pgfsetbuttcap%
\pgfsetroundjoin%
\definecolor{currentfill}{rgb}{0.121569,0.466667,0.705882}%
\pgfsetfillcolor{currentfill}%
\pgfsetfillopacity{0.500000}%
\pgfsetlinewidth{1.003750pt}%
\definecolor{currentstroke}{rgb}{0.121569,0.466667,0.705882}%
\pgfsetstrokecolor{currentstroke}%
\pgfsetstrokeopacity{0.500000}%
\pgfsetdash{}{0pt}%
\pgfpathmoveto{\pgfqpoint{2.236692in}{1.822203in}}%
\pgfpathcurveto{\pgfqpoint{2.247742in}{1.822203in}}{\pgfqpoint{2.258341in}{1.826593in}}{\pgfqpoint{2.266155in}{1.834407in}}%
\pgfpathcurveto{\pgfqpoint{2.273969in}{1.842220in}}{\pgfqpoint{2.278359in}{1.852819in}}{\pgfqpoint{2.278359in}{1.863869in}}%
\pgfpathcurveto{\pgfqpoint{2.278359in}{1.874920in}}{\pgfqpoint{2.273969in}{1.885519in}}{\pgfqpoint{2.266155in}{1.893332in}}%
\pgfpathcurveto{\pgfqpoint{2.258341in}{1.901146in}}{\pgfqpoint{2.247742in}{1.905536in}}{\pgfqpoint{2.236692in}{1.905536in}}%
\pgfpathcurveto{\pgfqpoint{2.225642in}{1.905536in}}{\pgfqpoint{2.215043in}{1.901146in}}{\pgfqpoint{2.207229in}{1.893332in}}%
\pgfpathcurveto{\pgfqpoint{2.199416in}{1.885519in}}{\pgfqpoint{2.195026in}{1.874920in}}{\pgfqpoint{2.195026in}{1.863869in}}%
\pgfpathcurveto{\pgfqpoint{2.195026in}{1.852819in}}{\pgfqpoint{2.199416in}{1.842220in}}{\pgfqpoint{2.207229in}{1.834407in}}%
\pgfpathcurveto{\pgfqpoint{2.215043in}{1.826593in}}{\pgfqpoint{2.225642in}{1.822203in}}{\pgfqpoint{2.236692in}{1.822203in}}%
\pgfpathclose%
\pgfusepath{stroke,fill}%
\end{pgfscope}%
\begin{pgfscope}%
\pgfpathrectangle{\pgfqpoint{0.557786in}{0.402778in}}{\pgfqpoint{4.393603in}{2.631944in}}%
\pgfusepath{clip}%
\pgfsetbuttcap%
\pgfsetroundjoin%
\definecolor{currentfill}{rgb}{0.121569,0.466667,0.705882}%
\pgfsetfillcolor{currentfill}%
\pgfsetfillopacity{0.500000}%
\pgfsetlinewidth{1.003750pt}%
\definecolor{currentstroke}{rgb}{0.121569,0.466667,0.705882}%
\pgfsetstrokecolor{currentstroke}%
\pgfsetstrokeopacity{0.500000}%
\pgfsetdash{}{0pt}%
\pgfpathmoveto{\pgfqpoint{2.236692in}{1.822203in}}%
\pgfpathcurveto{\pgfqpoint{2.247742in}{1.822203in}}{\pgfqpoint{2.258341in}{1.826593in}}{\pgfqpoint{2.266155in}{1.834407in}}%
\pgfpathcurveto{\pgfqpoint{2.273969in}{1.842220in}}{\pgfqpoint{2.278359in}{1.852819in}}{\pgfqpoint{2.278359in}{1.863869in}}%
\pgfpathcurveto{\pgfqpoint{2.278359in}{1.874920in}}{\pgfqpoint{2.273969in}{1.885519in}}{\pgfqpoint{2.266155in}{1.893332in}}%
\pgfpathcurveto{\pgfqpoint{2.258341in}{1.901146in}}{\pgfqpoint{2.247742in}{1.905536in}}{\pgfqpoint{2.236692in}{1.905536in}}%
\pgfpathcurveto{\pgfqpoint{2.225642in}{1.905536in}}{\pgfqpoint{2.215043in}{1.901146in}}{\pgfqpoint{2.207229in}{1.893332in}}%
\pgfpathcurveto{\pgfqpoint{2.199416in}{1.885519in}}{\pgfqpoint{2.195026in}{1.874920in}}{\pgfqpoint{2.195026in}{1.863869in}}%
\pgfpathcurveto{\pgfqpoint{2.195026in}{1.852819in}}{\pgfqpoint{2.199416in}{1.842220in}}{\pgfqpoint{2.207229in}{1.834407in}}%
\pgfpathcurveto{\pgfqpoint{2.215043in}{1.826593in}}{\pgfqpoint{2.225642in}{1.822203in}}{\pgfqpoint{2.236692in}{1.822203in}}%
\pgfpathclose%
\pgfusepath{stroke,fill}%
\end{pgfscope}%
\begin{pgfscope}%
\pgfpathrectangle{\pgfqpoint{0.557786in}{0.402778in}}{\pgfqpoint{4.393603in}{2.631944in}}%
\pgfusepath{clip}%
\pgfsetbuttcap%
\pgfsetroundjoin%
\definecolor{currentfill}{rgb}{0.121569,0.466667,0.705882}%
\pgfsetfillcolor{currentfill}%
\pgfsetfillopacity{0.500000}%
\pgfsetlinewidth{1.003750pt}%
\definecolor{currentstroke}{rgb}{0.121569,0.466667,0.705882}%
\pgfsetstrokecolor{currentstroke}%
\pgfsetstrokeopacity{0.500000}%
\pgfsetdash{}{0pt}%
\pgfpathmoveto{\pgfqpoint{2.236692in}{1.822203in}}%
\pgfpathcurveto{\pgfqpoint{2.247742in}{1.822203in}}{\pgfqpoint{2.258341in}{1.826593in}}{\pgfqpoint{2.266155in}{1.834407in}}%
\pgfpathcurveto{\pgfqpoint{2.273969in}{1.842220in}}{\pgfqpoint{2.278359in}{1.852819in}}{\pgfqpoint{2.278359in}{1.863869in}}%
\pgfpathcurveto{\pgfqpoint{2.278359in}{1.874920in}}{\pgfqpoint{2.273969in}{1.885519in}}{\pgfqpoint{2.266155in}{1.893332in}}%
\pgfpathcurveto{\pgfqpoint{2.258341in}{1.901146in}}{\pgfqpoint{2.247742in}{1.905536in}}{\pgfqpoint{2.236692in}{1.905536in}}%
\pgfpathcurveto{\pgfqpoint{2.225642in}{1.905536in}}{\pgfqpoint{2.215043in}{1.901146in}}{\pgfqpoint{2.207229in}{1.893332in}}%
\pgfpathcurveto{\pgfqpoint{2.199416in}{1.885519in}}{\pgfqpoint{2.195026in}{1.874920in}}{\pgfqpoint{2.195026in}{1.863869in}}%
\pgfpathcurveto{\pgfqpoint{2.195026in}{1.852819in}}{\pgfqpoint{2.199416in}{1.842220in}}{\pgfqpoint{2.207229in}{1.834407in}}%
\pgfpathcurveto{\pgfqpoint{2.215043in}{1.826593in}}{\pgfqpoint{2.225642in}{1.822203in}}{\pgfqpoint{2.236692in}{1.822203in}}%
\pgfpathclose%
\pgfusepath{stroke,fill}%
\end{pgfscope}%
\begin{pgfscope}%
\pgfpathrectangle{\pgfqpoint{0.557786in}{0.402778in}}{\pgfqpoint{4.393603in}{2.631944in}}%
\pgfusepath{clip}%
\pgfsetbuttcap%
\pgfsetroundjoin%
\definecolor{currentfill}{rgb}{0.121569,0.466667,0.705882}%
\pgfsetfillcolor{currentfill}%
\pgfsetfillopacity{0.500000}%
\pgfsetlinewidth{1.003750pt}%
\definecolor{currentstroke}{rgb}{0.121569,0.466667,0.705882}%
\pgfsetstrokecolor{currentstroke}%
\pgfsetstrokeopacity{0.500000}%
\pgfsetdash{}{0pt}%
\pgfpathmoveto{\pgfqpoint{2.236692in}{1.822203in}}%
\pgfpathcurveto{\pgfqpoint{2.247742in}{1.822203in}}{\pgfqpoint{2.258341in}{1.826593in}}{\pgfqpoint{2.266155in}{1.834407in}}%
\pgfpathcurveto{\pgfqpoint{2.273969in}{1.842220in}}{\pgfqpoint{2.278359in}{1.852819in}}{\pgfqpoint{2.278359in}{1.863869in}}%
\pgfpathcurveto{\pgfqpoint{2.278359in}{1.874920in}}{\pgfqpoint{2.273969in}{1.885519in}}{\pgfqpoint{2.266155in}{1.893332in}}%
\pgfpathcurveto{\pgfqpoint{2.258341in}{1.901146in}}{\pgfqpoint{2.247742in}{1.905536in}}{\pgfqpoint{2.236692in}{1.905536in}}%
\pgfpathcurveto{\pgfqpoint{2.225642in}{1.905536in}}{\pgfqpoint{2.215043in}{1.901146in}}{\pgfqpoint{2.207229in}{1.893332in}}%
\pgfpathcurveto{\pgfqpoint{2.199416in}{1.885519in}}{\pgfqpoint{2.195026in}{1.874920in}}{\pgfqpoint{2.195026in}{1.863869in}}%
\pgfpathcurveto{\pgfqpoint{2.195026in}{1.852819in}}{\pgfqpoint{2.199416in}{1.842220in}}{\pgfqpoint{2.207229in}{1.834407in}}%
\pgfpathcurveto{\pgfqpoint{2.215043in}{1.826593in}}{\pgfqpoint{2.225642in}{1.822203in}}{\pgfqpoint{2.236692in}{1.822203in}}%
\pgfpathclose%
\pgfusepath{stroke,fill}%
\end{pgfscope}%
\begin{pgfscope}%
\pgfpathrectangle{\pgfqpoint{0.557786in}{0.402778in}}{\pgfqpoint{4.393603in}{2.631944in}}%
\pgfusepath{clip}%
\pgfsetbuttcap%
\pgfsetroundjoin%
\definecolor{currentfill}{rgb}{1.000000,0.498039,0.054902}%
\pgfsetfillcolor{currentfill}%
\pgfsetfillopacity{0.500000}%
\pgfsetlinewidth{1.003750pt}%
\definecolor{currentstroke}{rgb}{1.000000,0.498039,0.054902}%
\pgfsetstrokecolor{currentstroke}%
\pgfsetstrokeopacity{0.500000}%
\pgfsetdash{}{0pt}%
\pgfpathmoveto{\pgfqpoint{4.602078in}{2.831083in}}%
\pgfpathcurveto{\pgfqpoint{4.613128in}{2.831083in}}{\pgfqpoint{4.623727in}{2.835473in}}{\pgfqpoint{4.631541in}{2.843287in}}%
\pgfpathcurveto{\pgfqpoint{4.639354in}{2.851101in}}{\pgfqpoint{4.643745in}{2.861700in}}{\pgfqpoint{4.643745in}{2.872750in}}%
\pgfpathcurveto{\pgfqpoint{4.643745in}{2.883800in}}{\pgfqpoint{4.639354in}{2.894399in}}{\pgfqpoint{4.631541in}{2.902213in}}%
\pgfpathcurveto{\pgfqpoint{4.623727in}{2.910026in}}{\pgfqpoint{4.613128in}{2.914416in}}{\pgfqpoint{4.602078in}{2.914416in}}%
\pgfpathcurveto{\pgfqpoint{4.591028in}{2.914416in}}{\pgfqpoint{4.580429in}{2.910026in}}{\pgfqpoint{4.572615in}{2.902213in}}%
\pgfpathcurveto{\pgfqpoint{4.564801in}{2.894399in}}{\pgfqpoint{4.560411in}{2.883800in}}{\pgfqpoint{4.560411in}{2.872750in}}%
\pgfpathcurveto{\pgfqpoint{4.560411in}{2.861700in}}{\pgfqpoint{4.564801in}{2.851101in}}{\pgfqpoint{4.572615in}{2.843287in}}%
\pgfpathcurveto{\pgfqpoint{4.580429in}{2.835473in}}{\pgfqpoint{4.591028in}{2.831083in}}{\pgfqpoint{4.602078in}{2.831083in}}%
\pgfpathclose%
\pgfusepath{stroke,fill}%
\end{pgfscope}%
\begin{pgfscope}%
\pgfpathrectangle{\pgfqpoint{0.557786in}{0.402778in}}{\pgfqpoint{4.393603in}{2.631944in}}%
\pgfusepath{clip}%
\pgfsetbuttcap%
\pgfsetroundjoin%
\definecolor{currentfill}{rgb}{1.000000,0.498039,0.054902}%
\pgfsetfillcolor{currentfill}%
\pgfsetfillopacity{0.500000}%
\pgfsetlinewidth{1.003750pt}%
\definecolor{currentstroke}{rgb}{1.000000,0.498039,0.054902}%
\pgfsetstrokecolor{currentstroke}%
\pgfsetstrokeopacity{0.500000}%
\pgfsetdash{}{0pt}%
\pgfpathmoveto{\pgfqpoint{3.645267in}{1.850218in}}%
\pgfpathcurveto{\pgfqpoint{3.656317in}{1.850218in}}{\pgfqpoint{3.666916in}{1.854608in}}{\pgfqpoint{3.674730in}{1.862422in}}%
\pgfpathcurveto{\pgfqpoint{3.682544in}{1.870235in}}{\pgfqpoint{3.686934in}{1.880834in}}{\pgfqpoint{3.686934in}{1.891884in}}%
\pgfpathcurveto{\pgfqpoint{3.686934in}{1.902935in}}{\pgfqpoint{3.682544in}{1.913534in}}{\pgfqpoint{3.674730in}{1.921347in}}%
\pgfpathcurveto{\pgfqpoint{3.666916in}{1.929161in}}{\pgfqpoint{3.656317in}{1.933551in}}{\pgfqpoint{3.645267in}{1.933551in}}%
\pgfpathcurveto{\pgfqpoint{3.634217in}{1.933551in}}{\pgfqpoint{3.623618in}{1.929161in}}{\pgfqpoint{3.615804in}{1.921347in}}%
\pgfpathcurveto{\pgfqpoint{3.607991in}{1.913534in}}{\pgfqpoint{3.603600in}{1.902935in}}{\pgfqpoint{3.603600in}{1.891884in}}%
\pgfpathcurveto{\pgfqpoint{3.603600in}{1.880834in}}{\pgfqpoint{3.607991in}{1.870235in}}{\pgfqpoint{3.615804in}{1.862422in}}%
\pgfpathcurveto{\pgfqpoint{3.623618in}{1.854608in}}{\pgfqpoint{3.634217in}{1.850218in}}{\pgfqpoint{3.645267in}{1.850218in}}%
\pgfpathclose%
\pgfusepath{stroke,fill}%
\end{pgfscope}%
\begin{pgfscope}%
\pgfpathrectangle{\pgfqpoint{0.557786in}{0.402778in}}{\pgfqpoint{4.393603in}{2.631944in}}%
\pgfusepath{clip}%
\pgfsetbuttcap%
\pgfsetroundjoin%
\definecolor{currentfill}{rgb}{1.000000,0.498039,0.054902}%
\pgfsetfillcolor{currentfill}%
\pgfsetfillopacity{0.500000}%
\pgfsetlinewidth{1.003750pt}%
\definecolor{currentstroke}{rgb}{1.000000,0.498039,0.054902}%
\pgfsetstrokecolor{currentstroke}%
\pgfsetstrokeopacity{0.500000}%
\pgfsetdash{}{0pt}%
\pgfpathmoveto{\pgfqpoint{2.834763in}{1.175280in}}%
\pgfpathcurveto{\pgfqpoint{2.845813in}{1.175280in}}{\pgfqpoint{2.856412in}{1.179671in}}{\pgfqpoint{2.864226in}{1.187484in}}%
\pgfpathcurveto{\pgfqpoint{2.872039in}{1.195298in}}{\pgfqpoint{2.876430in}{1.205897in}}{\pgfqpoint{2.876430in}{1.216947in}}%
\pgfpathcurveto{\pgfqpoint{2.876430in}{1.227997in}}{\pgfqpoint{2.872039in}{1.238596in}}{\pgfqpoint{2.864226in}{1.246410in}}%
\pgfpathcurveto{\pgfqpoint{2.856412in}{1.254223in}}{\pgfqpoint{2.845813in}{1.258614in}}{\pgfqpoint{2.834763in}{1.258614in}}%
\pgfpathcurveto{\pgfqpoint{2.823713in}{1.258614in}}{\pgfqpoint{2.813114in}{1.254223in}}{\pgfqpoint{2.805300in}{1.246410in}}%
\pgfpathcurveto{\pgfqpoint{2.797487in}{1.238596in}}{\pgfqpoint{2.793096in}{1.227997in}}{\pgfqpoint{2.793096in}{1.216947in}}%
\pgfpathcurveto{\pgfqpoint{2.793096in}{1.205897in}}{\pgfqpoint{2.797487in}{1.195298in}}{\pgfqpoint{2.805300in}{1.187484in}}%
\pgfpathcurveto{\pgfqpoint{2.813114in}{1.179671in}}{\pgfqpoint{2.823713in}{1.175280in}}{\pgfqpoint{2.834763in}{1.175280in}}%
\pgfpathclose%
\pgfusepath{stroke,fill}%
\end{pgfscope}%
\begin{pgfscope}%
\pgfpathrectangle{\pgfqpoint{0.557786in}{0.402778in}}{\pgfqpoint{4.393603in}{2.631944in}}%
\pgfusepath{clip}%
\pgfsetbuttcap%
\pgfsetroundjoin%
\definecolor{currentfill}{rgb}{1.000000,0.498039,0.054902}%
\pgfsetfillcolor{currentfill}%
\pgfsetfillopacity{0.500000}%
\pgfsetlinewidth{1.003750pt}%
\definecolor{currentstroke}{rgb}{1.000000,0.498039,0.054902}%
\pgfsetstrokecolor{currentstroke}%
\pgfsetstrokeopacity{0.500000}%
\pgfsetdash{}{0pt}%
\pgfpathmoveto{\pgfqpoint{1.908127in}{1.182174in}}%
\pgfpathcurveto{\pgfqpoint{1.919177in}{1.182174in}}{\pgfqpoint{1.929776in}{1.186564in}}{\pgfqpoint{1.937589in}{1.194378in}}%
\pgfpathcurveto{\pgfqpoint{1.945403in}{1.202191in}}{\pgfqpoint{1.949793in}{1.212790in}}{\pgfqpoint{1.949793in}{1.223841in}}%
\pgfpathcurveto{\pgfqpoint{1.949793in}{1.234891in}}{\pgfqpoint{1.945403in}{1.245490in}}{\pgfqpoint{1.937589in}{1.253303in}}%
\pgfpathcurveto{\pgfqpoint{1.929776in}{1.261117in}}{\pgfqpoint{1.919177in}{1.265507in}}{\pgfqpoint{1.908127in}{1.265507in}}%
\pgfpathcurveto{\pgfqpoint{1.897076in}{1.265507in}}{\pgfqpoint{1.886477in}{1.261117in}}{\pgfqpoint{1.878664in}{1.253303in}}%
\pgfpathcurveto{\pgfqpoint{1.870850in}{1.245490in}}{\pgfqpoint{1.866460in}{1.234891in}}{\pgfqpoint{1.866460in}{1.223841in}}%
\pgfpathcurveto{\pgfqpoint{1.866460in}{1.212790in}}{\pgfqpoint{1.870850in}{1.202191in}}{\pgfqpoint{1.878664in}{1.194378in}}%
\pgfpathcurveto{\pgfqpoint{1.886477in}{1.186564in}}{\pgfqpoint{1.897076in}{1.182174in}}{\pgfqpoint{1.908127in}{1.182174in}}%
\pgfpathclose%
\pgfusepath{stroke,fill}%
\end{pgfscope}%
\begin{pgfscope}%
\pgfpathrectangle{\pgfqpoint{0.557786in}{0.402778in}}{\pgfqpoint{4.393603in}{2.631944in}}%
\pgfusepath{clip}%
\pgfsetbuttcap%
\pgfsetroundjoin%
\definecolor{currentfill}{rgb}{1.000000,0.498039,0.054902}%
\pgfsetfillcolor{currentfill}%
\pgfsetfillopacity{0.500000}%
\pgfsetlinewidth{1.003750pt}%
\definecolor{currentstroke}{rgb}{1.000000,0.498039,0.054902}%
\pgfsetstrokecolor{currentstroke}%
\pgfsetstrokeopacity{0.500000}%
\pgfsetdash{}{0pt}%
\pgfpathmoveto{\pgfqpoint{1.522888in}{1.182398in}}%
\pgfpathcurveto{\pgfqpoint{1.533938in}{1.182398in}}{\pgfqpoint{1.544537in}{1.186788in}}{\pgfqpoint{1.552350in}{1.194602in}}%
\pgfpathcurveto{\pgfqpoint{1.560164in}{1.202415in}}{\pgfqpoint{1.564554in}{1.213014in}}{\pgfqpoint{1.564554in}{1.224064in}}%
\pgfpathcurveto{\pgfqpoint{1.564554in}{1.235115in}}{\pgfqpoint{1.560164in}{1.245714in}}{\pgfqpoint{1.552350in}{1.253527in}}%
\pgfpathcurveto{\pgfqpoint{1.544537in}{1.261341in}}{\pgfqpoint{1.533938in}{1.265731in}}{\pgfqpoint{1.522888in}{1.265731in}}%
\pgfpathcurveto{\pgfqpoint{1.511837in}{1.265731in}}{\pgfqpoint{1.501238in}{1.261341in}}{\pgfqpoint{1.493425in}{1.253527in}}%
\pgfpathcurveto{\pgfqpoint{1.485611in}{1.245714in}}{\pgfqpoint{1.481221in}{1.235115in}}{\pgfqpoint{1.481221in}{1.224064in}}%
\pgfpathcurveto{\pgfqpoint{1.481221in}{1.213014in}}{\pgfqpoint{1.485611in}{1.202415in}}{\pgfqpoint{1.493425in}{1.194602in}}%
\pgfpathcurveto{\pgfqpoint{1.501238in}{1.186788in}}{\pgfqpoint{1.511837in}{1.182398in}}{\pgfqpoint{1.522888in}{1.182398in}}%
\pgfpathclose%
\pgfusepath{stroke,fill}%
\end{pgfscope}%
\begin{pgfscope}%
\pgfpathrectangle{\pgfqpoint{0.557786in}{0.402778in}}{\pgfqpoint{4.393603in}{2.631944in}}%
\pgfusepath{clip}%
\pgfsetbuttcap%
\pgfsetroundjoin%
\definecolor{currentfill}{rgb}{1.000000,0.498039,0.054902}%
\pgfsetfillcolor{currentfill}%
\pgfsetfillopacity{0.500000}%
\pgfsetlinewidth{1.003750pt}%
\definecolor{currentstroke}{rgb}{1.000000,0.498039,0.054902}%
\pgfsetstrokecolor{currentstroke}%
\pgfsetstrokeopacity{0.500000}%
\pgfsetdash{}{0pt}%
\pgfpathmoveto{\pgfqpoint{1.516665in}{1.182395in}}%
\pgfpathcurveto{\pgfqpoint{1.527715in}{1.182395in}}{\pgfqpoint{1.538314in}{1.186785in}}{\pgfqpoint{1.546128in}{1.194599in}}%
\pgfpathcurveto{\pgfqpoint{1.553941in}{1.202412in}}{\pgfqpoint{1.558332in}{1.213011in}}{\pgfqpoint{1.558332in}{1.224062in}}%
\pgfpathcurveto{\pgfqpoint{1.558332in}{1.235112in}}{\pgfqpoint{1.553941in}{1.245711in}}{\pgfqpoint{1.546128in}{1.253524in}}%
\pgfpathcurveto{\pgfqpoint{1.538314in}{1.261338in}}{\pgfqpoint{1.527715in}{1.265728in}}{\pgfqpoint{1.516665in}{1.265728in}}%
\pgfpathcurveto{\pgfqpoint{1.505615in}{1.265728in}}{\pgfqpoint{1.495016in}{1.261338in}}{\pgfqpoint{1.487202in}{1.253524in}}%
\pgfpathcurveto{\pgfqpoint{1.479389in}{1.245711in}}{\pgfqpoint{1.474998in}{1.235112in}}{\pgfqpoint{1.474998in}{1.224062in}}%
\pgfpathcurveto{\pgfqpoint{1.474998in}{1.213011in}}{\pgfqpoint{1.479389in}{1.202412in}}{\pgfqpoint{1.487202in}{1.194599in}}%
\pgfpathcurveto{\pgfqpoint{1.495016in}{1.186785in}}{\pgfqpoint{1.505615in}{1.182395in}}{\pgfqpoint{1.516665in}{1.182395in}}%
\pgfpathclose%
\pgfusepath{stroke,fill}%
\end{pgfscope}%
\begin{pgfscope}%
\pgfpathrectangle{\pgfqpoint{0.557786in}{0.402778in}}{\pgfqpoint{4.393603in}{2.631944in}}%
\pgfusepath{clip}%
\pgfsetbuttcap%
\pgfsetroundjoin%
\definecolor{currentfill}{rgb}{1.000000,0.498039,0.054902}%
\pgfsetfillcolor{currentfill}%
\pgfsetfillopacity{0.500000}%
\pgfsetlinewidth{1.003750pt}%
\definecolor{currentstroke}{rgb}{1.000000,0.498039,0.054902}%
\pgfsetstrokecolor{currentstroke}%
\pgfsetstrokeopacity{0.500000}%
\pgfsetdash{}{0pt}%
\pgfpathmoveto{\pgfqpoint{1.516702in}{1.182395in}}%
\pgfpathcurveto{\pgfqpoint{1.527752in}{1.182395in}}{\pgfqpoint{1.538351in}{1.186785in}}{\pgfqpoint{1.546165in}{1.194599in}}%
\pgfpathcurveto{\pgfqpoint{1.553978in}{1.202412in}}{\pgfqpoint{1.558369in}{1.213011in}}{\pgfqpoint{1.558369in}{1.224062in}}%
\pgfpathcurveto{\pgfqpoint{1.558369in}{1.235112in}}{\pgfqpoint{1.553978in}{1.245711in}}{\pgfqpoint{1.546165in}{1.253524in}}%
\pgfpathcurveto{\pgfqpoint{1.538351in}{1.261338in}}{\pgfqpoint{1.527752in}{1.265728in}}{\pgfqpoint{1.516702in}{1.265728in}}%
\pgfpathcurveto{\pgfqpoint{1.505652in}{1.265728in}}{\pgfqpoint{1.495053in}{1.261338in}}{\pgfqpoint{1.487239in}{1.253524in}}%
\pgfpathcurveto{\pgfqpoint{1.479426in}{1.245711in}}{\pgfqpoint{1.475035in}{1.235112in}}{\pgfqpoint{1.475035in}{1.224062in}}%
\pgfpathcurveto{\pgfqpoint{1.475035in}{1.213011in}}{\pgfqpoint{1.479426in}{1.202412in}}{\pgfqpoint{1.487239in}{1.194599in}}%
\pgfpathcurveto{\pgfqpoint{1.495053in}{1.186785in}}{\pgfqpoint{1.505652in}{1.182395in}}{\pgfqpoint{1.516702in}{1.182395in}}%
\pgfpathclose%
\pgfusepath{stroke,fill}%
\end{pgfscope}%
\begin{pgfscope}%
\pgfpathrectangle{\pgfqpoint{0.557786in}{0.402778in}}{\pgfqpoint{4.393603in}{2.631944in}}%
\pgfusepath{clip}%
\pgfsetbuttcap%
\pgfsetroundjoin%
\definecolor{currentfill}{rgb}{1.000000,0.498039,0.054902}%
\pgfsetfillcolor{currentfill}%
\pgfsetfillopacity{0.500000}%
\pgfsetlinewidth{1.003750pt}%
\definecolor{currentstroke}{rgb}{1.000000,0.498039,0.054902}%
\pgfsetstrokecolor{currentstroke}%
\pgfsetstrokeopacity{0.500000}%
\pgfsetdash{}{0pt}%
\pgfpathmoveto{\pgfqpoint{1.516702in}{1.182395in}}%
\pgfpathcurveto{\pgfqpoint{1.527752in}{1.182395in}}{\pgfqpoint{1.538351in}{1.186785in}}{\pgfqpoint{1.546165in}{1.194599in}}%
\pgfpathcurveto{\pgfqpoint{1.553978in}{1.202412in}}{\pgfqpoint{1.558369in}{1.213011in}}{\pgfqpoint{1.558369in}{1.224062in}}%
\pgfpathcurveto{\pgfqpoint{1.558369in}{1.235112in}}{\pgfqpoint{1.553978in}{1.245711in}}{\pgfqpoint{1.546165in}{1.253524in}}%
\pgfpathcurveto{\pgfqpoint{1.538351in}{1.261338in}}{\pgfqpoint{1.527752in}{1.265728in}}{\pgfqpoint{1.516702in}{1.265728in}}%
\pgfpathcurveto{\pgfqpoint{1.505652in}{1.265728in}}{\pgfqpoint{1.495053in}{1.261338in}}{\pgfqpoint{1.487239in}{1.253524in}}%
\pgfpathcurveto{\pgfqpoint{1.479425in}{1.245711in}}{\pgfqpoint{1.475035in}{1.235112in}}{\pgfqpoint{1.475035in}{1.224062in}}%
\pgfpathcurveto{\pgfqpoint{1.475035in}{1.213011in}}{\pgfqpoint{1.479425in}{1.202412in}}{\pgfqpoint{1.487239in}{1.194599in}}%
\pgfpathcurveto{\pgfqpoint{1.495053in}{1.186785in}}{\pgfqpoint{1.505652in}{1.182395in}}{\pgfqpoint{1.516702in}{1.182395in}}%
\pgfpathclose%
\pgfusepath{stroke,fill}%
\end{pgfscope}%
\begin{pgfscope}%
\pgfpathrectangle{\pgfqpoint{0.557786in}{0.402778in}}{\pgfqpoint{4.393603in}{2.631944in}}%
\pgfusepath{clip}%
\pgfsetbuttcap%
\pgfsetroundjoin%
\definecolor{currentfill}{rgb}{1.000000,0.498039,0.054902}%
\pgfsetfillcolor{currentfill}%
\pgfsetfillopacity{0.500000}%
\pgfsetlinewidth{1.003750pt}%
\definecolor{currentstroke}{rgb}{1.000000,0.498039,0.054902}%
\pgfsetstrokecolor{currentstroke}%
\pgfsetstrokeopacity{0.500000}%
\pgfsetdash{}{0pt}%
\pgfpathmoveto{\pgfqpoint{1.516702in}{1.182395in}}%
\pgfpathcurveto{\pgfqpoint{1.527752in}{1.182395in}}{\pgfqpoint{1.538351in}{1.186785in}}{\pgfqpoint{1.546165in}{1.194599in}}%
\pgfpathcurveto{\pgfqpoint{1.553978in}{1.202412in}}{\pgfqpoint{1.558369in}{1.213011in}}{\pgfqpoint{1.558369in}{1.224062in}}%
\pgfpathcurveto{\pgfqpoint{1.558369in}{1.235112in}}{\pgfqpoint{1.553978in}{1.245711in}}{\pgfqpoint{1.546165in}{1.253524in}}%
\pgfpathcurveto{\pgfqpoint{1.538351in}{1.261338in}}{\pgfqpoint{1.527752in}{1.265728in}}{\pgfqpoint{1.516702in}{1.265728in}}%
\pgfpathcurveto{\pgfqpoint{1.505652in}{1.265728in}}{\pgfqpoint{1.495053in}{1.261338in}}{\pgfqpoint{1.487239in}{1.253524in}}%
\pgfpathcurveto{\pgfqpoint{1.479425in}{1.245711in}}{\pgfqpoint{1.475035in}{1.235112in}}{\pgfqpoint{1.475035in}{1.224062in}}%
\pgfpathcurveto{\pgfqpoint{1.475035in}{1.213011in}}{\pgfqpoint{1.479425in}{1.202412in}}{\pgfqpoint{1.487239in}{1.194599in}}%
\pgfpathcurveto{\pgfqpoint{1.495053in}{1.186785in}}{\pgfqpoint{1.505652in}{1.182395in}}{\pgfqpoint{1.516702in}{1.182395in}}%
\pgfpathclose%
\pgfusepath{stroke,fill}%
\end{pgfscope}%
\begin{pgfscope}%
\pgfpathrectangle{\pgfqpoint{0.557786in}{0.402778in}}{\pgfqpoint{4.393603in}{2.631944in}}%
\pgfusepath{clip}%
\pgfsetbuttcap%
\pgfsetroundjoin%
\definecolor{currentfill}{rgb}{1.000000,0.498039,0.054902}%
\pgfsetfillcolor{currentfill}%
\pgfsetfillopacity{0.500000}%
\pgfsetlinewidth{1.003750pt}%
\definecolor{currentstroke}{rgb}{1.000000,0.498039,0.054902}%
\pgfsetstrokecolor{currentstroke}%
\pgfsetstrokeopacity{0.500000}%
\pgfsetdash{}{0pt}%
\pgfpathmoveto{\pgfqpoint{1.516702in}{1.182395in}}%
\pgfpathcurveto{\pgfqpoint{1.527752in}{1.182395in}}{\pgfqpoint{1.538351in}{1.186785in}}{\pgfqpoint{1.546165in}{1.194599in}}%
\pgfpathcurveto{\pgfqpoint{1.553978in}{1.202412in}}{\pgfqpoint{1.558369in}{1.213011in}}{\pgfqpoint{1.558369in}{1.224062in}}%
\pgfpathcurveto{\pgfqpoint{1.558369in}{1.235112in}}{\pgfqpoint{1.553978in}{1.245711in}}{\pgfqpoint{1.546165in}{1.253524in}}%
\pgfpathcurveto{\pgfqpoint{1.538351in}{1.261338in}}{\pgfqpoint{1.527752in}{1.265728in}}{\pgfqpoint{1.516702in}{1.265728in}}%
\pgfpathcurveto{\pgfqpoint{1.505652in}{1.265728in}}{\pgfqpoint{1.495053in}{1.261338in}}{\pgfqpoint{1.487239in}{1.253524in}}%
\pgfpathcurveto{\pgfqpoint{1.479425in}{1.245711in}}{\pgfqpoint{1.475035in}{1.235112in}}{\pgfqpoint{1.475035in}{1.224062in}}%
\pgfpathcurveto{\pgfqpoint{1.475035in}{1.213011in}}{\pgfqpoint{1.479425in}{1.202412in}}{\pgfqpoint{1.487239in}{1.194599in}}%
\pgfpathcurveto{\pgfqpoint{1.495053in}{1.186785in}}{\pgfqpoint{1.505652in}{1.182395in}}{\pgfqpoint{1.516702in}{1.182395in}}%
\pgfpathclose%
\pgfusepath{stroke,fill}%
\end{pgfscope}%
\begin{pgfscope}%
\pgfpathrectangle{\pgfqpoint{0.557786in}{0.402778in}}{\pgfqpoint{4.393603in}{2.631944in}}%
\pgfusepath{clip}%
\pgfsetbuttcap%
\pgfsetroundjoin%
\definecolor{currentfill}{rgb}{0.172549,0.627451,0.172549}%
\pgfsetfillcolor{currentfill}%
\pgfsetfillopacity{0.500000}%
\pgfsetlinewidth{1.003750pt}%
\definecolor{currentstroke}{rgb}{0.172549,0.627451,0.172549}%
\pgfsetstrokecolor{currentstroke}%
\pgfsetstrokeopacity{0.500000}%
\pgfsetdash{}{0pt}%
\pgfpathmoveto{\pgfqpoint{4.458145in}{2.840703in}}%
\pgfpathcurveto{\pgfqpoint{4.469195in}{2.840703in}}{\pgfqpoint{4.479794in}{2.845094in}}{\pgfqpoint{4.487608in}{2.852907in}}%
\pgfpathcurveto{\pgfqpoint{4.495422in}{2.860721in}}{\pgfqpoint{4.499812in}{2.871320in}}{\pgfqpoint{4.499812in}{2.882370in}}%
\pgfpathcurveto{\pgfqpoint{4.499812in}{2.893420in}}{\pgfqpoint{4.495422in}{2.904019in}}{\pgfqpoint{4.487608in}{2.911833in}}%
\pgfpathcurveto{\pgfqpoint{4.479794in}{2.919646in}}{\pgfqpoint{4.469195in}{2.924037in}}{\pgfqpoint{4.458145in}{2.924037in}}%
\pgfpathcurveto{\pgfqpoint{4.447095in}{2.924037in}}{\pgfqpoint{4.436496in}{2.919646in}}{\pgfqpoint{4.428683in}{2.911833in}}%
\pgfpathcurveto{\pgfqpoint{4.420869in}{2.904019in}}{\pgfqpoint{4.416479in}{2.893420in}}{\pgfqpoint{4.416479in}{2.882370in}}%
\pgfpathcurveto{\pgfqpoint{4.416479in}{2.871320in}}{\pgfqpoint{4.420869in}{2.860721in}}{\pgfqpoint{4.428683in}{2.852907in}}%
\pgfpathcurveto{\pgfqpoint{4.436496in}{2.845094in}}{\pgfqpoint{4.447095in}{2.840703in}}{\pgfqpoint{4.458145in}{2.840703in}}%
\pgfpathclose%
\pgfusepath{stroke,fill}%
\end{pgfscope}%
\begin{pgfscope}%
\pgfpathrectangle{\pgfqpoint{0.557786in}{0.402778in}}{\pgfqpoint{4.393603in}{2.631944in}}%
\pgfusepath{clip}%
\pgfsetbuttcap%
\pgfsetroundjoin%
\definecolor{currentfill}{rgb}{0.172549,0.627451,0.172549}%
\pgfsetfillcolor{currentfill}%
\pgfsetfillopacity{0.500000}%
\pgfsetlinewidth{1.003750pt}%
\definecolor{currentstroke}{rgb}{0.172549,0.627451,0.172549}%
\pgfsetstrokecolor{currentstroke}%
\pgfsetstrokeopacity{0.500000}%
\pgfsetdash{}{0pt}%
\pgfpathmoveto{\pgfqpoint{3.501347in}{1.760342in}}%
\pgfpathcurveto{\pgfqpoint{3.512397in}{1.760342in}}{\pgfqpoint{3.522996in}{1.764732in}}{\pgfqpoint{3.530810in}{1.772546in}}%
\pgfpathcurveto{\pgfqpoint{3.538624in}{1.780360in}}{\pgfqpoint{3.543014in}{1.790959in}}{\pgfqpoint{3.543014in}{1.802009in}}%
\pgfpathcurveto{\pgfqpoint{3.543014in}{1.813059in}}{\pgfqpoint{3.538624in}{1.823658in}}{\pgfqpoint{3.530810in}{1.831471in}}%
\pgfpathcurveto{\pgfqpoint{3.522996in}{1.839285in}}{\pgfqpoint{3.512397in}{1.843675in}}{\pgfqpoint{3.501347in}{1.843675in}}%
\pgfpathcurveto{\pgfqpoint{3.490297in}{1.843675in}}{\pgfqpoint{3.479698in}{1.839285in}}{\pgfqpoint{3.471884in}{1.831471in}}%
\pgfpathcurveto{\pgfqpoint{3.464071in}{1.823658in}}{\pgfqpoint{3.459680in}{1.813059in}}{\pgfqpoint{3.459680in}{1.802009in}}%
\pgfpathcurveto{\pgfqpoint{3.459680in}{1.790959in}}{\pgfqpoint{3.464071in}{1.780360in}}{\pgfqpoint{3.471884in}{1.772546in}}%
\pgfpathcurveto{\pgfqpoint{3.479698in}{1.764732in}}{\pgfqpoint{3.490297in}{1.760342in}}{\pgfqpoint{3.501347in}{1.760342in}}%
\pgfpathclose%
\pgfusepath{stroke,fill}%
\end{pgfscope}%
\begin{pgfscope}%
\pgfpathrectangle{\pgfqpoint{0.557786in}{0.402778in}}{\pgfqpoint{4.393603in}{2.631944in}}%
\pgfusepath{clip}%
\pgfsetbuttcap%
\pgfsetroundjoin%
\definecolor{currentfill}{rgb}{0.172549,0.627451,0.172549}%
\pgfsetfillcolor{currentfill}%
\pgfsetfillopacity{0.500000}%
\pgfsetlinewidth{1.003750pt}%
\definecolor{currentstroke}{rgb}{0.172549,0.627451,0.172549}%
\pgfsetstrokecolor{currentstroke}%
\pgfsetstrokeopacity{0.500000}%
\pgfsetdash{}{0pt}%
\pgfpathmoveto{\pgfqpoint{2.691105in}{0.513463in}}%
\pgfpathcurveto{\pgfqpoint{2.702155in}{0.513463in}}{\pgfqpoint{2.712754in}{0.517854in}}{\pgfqpoint{2.720568in}{0.525667in}}%
\pgfpathcurveto{\pgfqpoint{2.728381in}{0.533481in}}{\pgfqpoint{2.732772in}{0.544080in}}{\pgfqpoint{2.732772in}{0.555130in}}%
\pgfpathcurveto{\pgfqpoint{2.732772in}{0.566180in}}{\pgfqpoint{2.728381in}{0.576779in}}{\pgfqpoint{2.720568in}{0.584593in}}%
\pgfpathcurveto{\pgfqpoint{2.712754in}{0.592406in}}{\pgfqpoint{2.702155in}{0.596797in}}{\pgfqpoint{2.691105in}{0.596797in}}%
\pgfpathcurveto{\pgfqpoint{2.680055in}{0.596797in}}{\pgfqpoint{2.669456in}{0.592406in}}{\pgfqpoint{2.661642in}{0.584593in}}%
\pgfpathcurveto{\pgfqpoint{2.653829in}{0.576779in}}{\pgfqpoint{2.649438in}{0.566180in}}{\pgfqpoint{2.649438in}{0.555130in}}%
\pgfpathcurveto{\pgfqpoint{2.649438in}{0.544080in}}{\pgfqpoint{2.653829in}{0.533481in}}{\pgfqpoint{2.661642in}{0.525667in}}%
\pgfpathcurveto{\pgfqpoint{2.669456in}{0.517854in}}{\pgfqpoint{2.680055in}{0.513463in}}{\pgfqpoint{2.691105in}{0.513463in}}%
\pgfpathclose%
\pgfusepath{stroke,fill}%
\end{pgfscope}%
\begin{pgfscope}%
\pgfpathrectangle{\pgfqpoint{0.557786in}{0.402778in}}{\pgfqpoint{4.393603in}{2.631944in}}%
\pgfusepath{clip}%
\pgfsetbuttcap%
\pgfsetroundjoin%
\definecolor{currentfill}{rgb}{0.172549,0.627451,0.172549}%
\pgfsetfillcolor{currentfill}%
\pgfsetfillopacity{0.500000}%
\pgfsetlinewidth{1.003750pt}%
\definecolor{currentstroke}{rgb}{0.172549,0.627451,0.172549}%
\pgfsetstrokecolor{currentstroke}%
\pgfsetstrokeopacity{0.500000}%
\pgfsetdash{}{0pt}%
\pgfpathmoveto{\pgfqpoint{1.742729in}{0.541702in}}%
\pgfpathcurveto{\pgfqpoint{1.753780in}{0.541702in}}{\pgfqpoint{1.764379in}{0.546092in}}{\pgfqpoint{1.772192in}{0.553905in}}%
\pgfpathcurveto{\pgfqpoint{1.780006in}{0.561719in}}{\pgfqpoint{1.784396in}{0.572318in}}{\pgfqpoint{1.784396in}{0.583368in}}%
\pgfpathcurveto{\pgfqpoint{1.784396in}{0.594418in}}{\pgfqpoint{1.780006in}{0.605017in}}{\pgfqpoint{1.772192in}{0.612831in}}%
\pgfpathcurveto{\pgfqpoint{1.764379in}{0.620645in}}{\pgfqpoint{1.753780in}{0.625035in}}{\pgfqpoint{1.742729in}{0.625035in}}%
\pgfpathcurveto{\pgfqpoint{1.731679in}{0.625035in}}{\pgfqpoint{1.721080in}{0.620645in}}{\pgfqpoint{1.713267in}{0.612831in}}%
\pgfpathcurveto{\pgfqpoint{1.705453in}{0.605017in}}{\pgfqpoint{1.701063in}{0.594418in}}{\pgfqpoint{1.701063in}{0.583368in}}%
\pgfpathcurveto{\pgfqpoint{1.701063in}{0.572318in}}{\pgfqpoint{1.705453in}{0.561719in}}{\pgfqpoint{1.713267in}{0.553905in}}%
\pgfpathcurveto{\pgfqpoint{1.721080in}{0.546092in}}{\pgfqpoint{1.731679in}{0.541702in}}{\pgfqpoint{1.742729in}{0.541702in}}%
\pgfpathclose%
\pgfusepath{stroke,fill}%
\end{pgfscope}%
\begin{pgfscope}%
\pgfpathrectangle{\pgfqpoint{0.557786in}{0.402778in}}{\pgfqpoint{4.393603in}{2.631944in}}%
\pgfusepath{clip}%
\pgfsetbuttcap%
\pgfsetroundjoin%
\definecolor{currentfill}{rgb}{0.172549,0.627451,0.172549}%
\pgfsetfillcolor{currentfill}%
\pgfsetfillopacity{0.500000}%
\pgfsetlinewidth{1.003750pt}%
\definecolor{currentstroke}{rgb}{0.172549,0.627451,0.172549}%
\pgfsetstrokecolor{currentstroke}%
\pgfsetstrokeopacity{0.500000}%
\pgfsetdash{}{0pt}%
\pgfpathmoveto{\pgfqpoint{0.895083in}{0.542598in}}%
\pgfpathcurveto{\pgfqpoint{0.906133in}{0.542598in}}{\pgfqpoint{0.916732in}{0.546988in}}{\pgfqpoint{0.924546in}{0.554802in}}%
\pgfpathcurveto{\pgfqpoint{0.932360in}{0.562616in}}{\pgfqpoint{0.936750in}{0.573215in}}{\pgfqpoint{0.936750in}{0.584265in}}%
\pgfpathcurveto{\pgfqpoint{0.936750in}{0.595315in}}{\pgfqpoint{0.932360in}{0.605914in}}{\pgfqpoint{0.924546in}{0.613727in}}%
\pgfpathcurveto{\pgfqpoint{0.916732in}{0.621541in}}{\pgfqpoint{0.906133in}{0.625931in}}{\pgfqpoint{0.895083in}{0.625931in}}%
\pgfpathcurveto{\pgfqpoint{0.884033in}{0.625931in}}{\pgfqpoint{0.873434in}{0.621541in}}{\pgfqpoint{0.865620in}{0.613727in}}%
\pgfpathcurveto{\pgfqpoint{0.857807in}{0.605914in}}{\pgfqpoint{0.853416in}{0.595315in}}{\pgfqpoint{0.853416in}{0.584265in}}%
\pgfpathcurveto{\pgfqpoint{0.853416in}{0.573215in}}{\pgfqpoint{0.857807in}{0.562616in}}{\pgfqpoint{0.865620in}{0.554802in}}%
\pgfpathcurveto{\pgfqpoint{0.873434in}{0.546988in}}{\pgfqpoint{0.884033in}{0.542598in}}{\pgfqpoint{0.895083in}{0.542598in}}%
\pgfpathclose%
\pgfusepath{stroke,fill}%
\end{pgfscope}%
\begin{pgfscope}%
\pgfpathrectangle{\pgfqpoint{0.557786in}{0.402778in}}{\pgfqpoint{4.393603in}{2.631944in}}%
\pgfusepath{clip}%
\pgfsetbuttcap%
\pgfsetroundjoin%
\definecolor{currentfill}{rgb}{0.172549,0.627451,0.172549}%
\pgfsetfillcolor{currentfill}%
\pgfsetfillopacity{0.500000}%
\pgfsetlinewidth{1.003750pt}%
\definecolor{currentstroke}{rgb}{0.172549,0.627451,0.172549}%
\pgfsetstrokecolor{currentstroke}%
\pgfsetstrokeopacity{0.500000}%
\pgfsetdash{}{0pt}%
\pgfpathmoveto{\pgfqpoint{0.796111in}{0.542586in}}%
\pgfpathcurveto{\pgfqpoint{0.807161in}{0.542586in}}{\pgfqpoint{0.817760in}{0.546977in}}{\pgfqpoint{0.825574in}{0.554790in}}%
\pgfpathcurveto{\pgfqpoint{0.833388in}{0.562604in}}{\pgfqpoint{0.837778in}{0.573203in}}{\pgfqpoint{0.837778in}{0.584253in}}%
\pgfpathcurveto{\pgfqpoint{0.837778in}{0.595303in}}{\pgfqpoint{0.833388in}{0.605902in}}{\pgfqpoint{0.825574in}{0.613716in}}%
\pgfpathcurveto{\pgfqpoint{0.817760in}{0.621529in}}{\pgfqpoint{0.807161in}{0.625920in}}{\pgfqpoint{0.796111in}{0.625920in}}%
\pgfpathcurveto{\pgfqpoint{0.785061in}{0.625920in}}{\pgfqpoint{0.774462in}{0.621529in}}{\pgfqpoint{0.766648in}{0.613716in}}%
\pgfpathcurveto{\pgfqpoint{0.758835in}{0.605902in}}{\pgfqpoint{0.754445in}{0.595303in}}{\pgfqpoint{0.754445in}{0.584253in}}%
\pgfpathcurveto{\pgfqpoint{0.754445in}{0.573203in}}{\pgfqpoint{0.758835in}{0.562604in}}{\pgfqpoint{0.766648in}{0.554790in}}%
\pgfpathcurveto{\pgfqpoint{0.774462in}{0.546977in}}{\pgfqpoint{0.785061in}{0.542586in}}{\pgfqpoint{0.796111in}{0.542586in}}%
\pgfpathclose%
\pgfusepath{stroke,fill}%
\end{pgfscope}%
\begin{pgfscope}%
\pgfpathrectangle{\pgfqpoint{0.557786in}{0.402778in}}{\pgfqpoint{4.393603in}{2.631944in}}%
\pgfusepath{clip}%
\pgfsetbuttcap%
\pgfsetroundjoin%
\definecolor{currentfill}{rgb}{0.172549,0.627451,0.172549}%
\pgfsetfillcolor{currentfill}%
\pgfsetfillopacity{0.500000}%
\pgfsetlinewidth{1.003750pt}%
\definecolor{currentstroke}{rgb}{0.172549,0.627451,0.172549}%
\pgfsetstrokecolor{currentstroke}%
\pgfsetstrokeopacity{0.500000}%
\pgfsetdash{}{0pt}%
\pgfpathmoveto{\pgfqpoint{0.796697in}{0.542586in}}%
\pgfpathcurveto{\pgfqpoint{0.807747in}{0.542586in}}{\pgfqpoint{0.818346in}{0.546977in}}{\pgfqpoint{0.826159in}{0.554790in}}%
\pgfpathcurveto{\pgfqpoint{0.833973in}{0.562604in}}{\pgfqpoint{0.838363in}{0.573203in}}{\pgfqpoint{0.838363in}{0.584253in}}%
\pgfpathcurveto{\pgfqpoint{0.838363in}{0.595303in}}{\pgfqpoint{0.833973in}{0.605902in}}{\pgfqpoint{0.826159in}{0.613716in}}%
\pgfpathcurveto{\pgfqpoint{0.818346in}{0.621529in}}{\pgfqpoint{0.807747in}{0.625920in}}{\pgfqpoint{0.796697in}{0.625920in}}%
\pgfpathcurveto{\pgfqpoint{0.785646in}{0.625920in}}{\pgfqpoint{0.775047in}{0.621529in}}{\pgfqpoint{0.767234in}{0.613716in}}%
\pgfpathcurveto{\pgfqpoint{0.759420in}{0.605902in}}{\pgfqpoint{0.755030in}{0.595303in}}{\pgfqpoint{0.755030in}{0.584253in}}%
\pgfpathcurveto{\pgfqpoint{0.755030in}{0.573203in}}{\pgfqpoint{0.759420in}{0.562604in}}{\pgfqpoint{0.767234in}{0.554790in}}%
\pgfpathcurveto{\pgfqpoint{0.775047in}{0.546977in}}{\pgfqpoint{0.785646in}{0.542586in}}{\pgfqpoint{0.796697in}{0.542586in}}%
\pgfpathclose%
\pgfusepath{stroke,fill}%
\end{pgfscope}%
\begin{pgfscope}%
\pgfpathrectangle{\pgfqpoint{0.557786in}{0.402778in}}{\pgfqpoint{4.393603in}{2.631944in}}%
\pgfusepath{clip}%
\pgfsetbuttcap%
\pgfsetroundjoin%
\definecolor{currentfill}{rgb}{0.172549,0.627451,0.172549}%
\pgfsetfillcolor{currentfill}%
\pgfsetfillopacity{0.500000}%
\pgfsetlinewidth{1.003750pt}%
\definecolor{currentstroke}{rgb}{0.172549,0.627451,0.172549}%
\pgfsetstrokecolor{currentstroke}%
\pgfsetstrokeopacity{0.500000}%
\pgfsetdash{}{0pt}%
\pgfpathmoveto{\pgfqpoint{0.796696in}{0.542586in}}%
\pgfpathcurveto{\pgfqpoint{0.807746in}{0.542586in}}{\pgfqpoint{0.818345in}{0.546977in}}{\pgfqpoint{0.826159in}{0.554790in}}%
\pgfpathcurveto{\pgfqpoint{0.833972in}{0.562604in}}{\pgfqpoint{0.838363in}{0.573203in}}{\pgfqpoint{0.838363in}{0.584253in}}%
\pgfpathcurveto{\pgfqpoint{0.838363in}{0.595303in}}{\pgfqpoint{0.833972in}{0.605902in}}{\pgfqpoint{0.826159in}{0.613716in}}%
\pgfpathcurveto{\pgfqpoint{0.818345in}{0.621529in}}{\pgfqpoint{0.807746in}{0.625920in}}{\pgfqpoint{0.796696in}{0.625920in}}%
\pgfpathcurveto{\pgfqpoint{0.785646in}{0.625920in}}{\pgfqpoint{0.775047in}{0.621529in}}{\pgfqpoint{0.767233in}{0.613716in}}%
\pgfpathcurveto{\pgfqpoint{0.759420in}{0.605902in}}{\pgfqpoint{0.755029in}{0.595303in}}{\pgfqpoint{0.755029in}{0.584253in}}%
\pgfpathcurveto{\pgfqpoint{0.755029in}{0.573203in}}{\pgfqpoint{0.759420in}{0.562604in}}{\pgfqpoint{0.767233in}{0.554790in}}%
\pgfpathcurveto{\pgfqpoint{0.775047in}{0.546977in}}{\pgfqpoint{0.785646in}{0.542586in}}{\pgfqpoint{0.796696in}{0.542586in}}%
\pgfpathclose%
\pgfusepath{stroke,fill}%
\end{pgfscope}%
\begin{pgfscope}%
\pgfpathrectangle{\pgfqpoint{0.557786in}{0.402778in}}{\pgfqpoint{4.393603in}{2.631944in}}%
\pgfusepath{clip}%
\pgfsetbuttcap%
\pgfsetroundjoin%
\definecolor{currentfill}{rgb}{0.172549,0.627451,0.172549}%
\pgfsetfillcolor{currentfill}%
\pgfsetfillopacity{0.500000}%
\pgfsetlinewidth{1.003750pt}%
\definecolor{currentstroke}{rgb}{0.172549,0.627451,0.172549}%
\pgfsetstrokecolor{currentstroke}%
\pgfsetstrokeopacity{0.500000}%
\pgfsetdash{}{0pt}%
\pgfpathmoveto{\pgfqpoint{0.796696in}{0.542586in}}%
\pgfpathcurveto{\pgfqpoint{0.807746in}{0.542586in}}{\pgfqpoint{0.818345in}{0.546977in}}{\pgfqpoint{0.826159in}{0.554790in}}%
\pgfpathcurveto{\pgfqpoint{0.833972in}{0.562604in}}{\pgfqpoint{0.838363in}{0.573203in}}{\pgfqpoint{0.838363in}{0.584253in}}%
\pgfpathcurveto{\pgfqpoint{0.838363in}{0.595303in}}{\pgfqpoint{0.833972in}{0.605902in}}{\pgfqpoint{0.826159in}{0.613716in}}%
\pgfpathcurveto{\pgfqpoint{0.818345in}{0.621529in}}{\pgfqpoint{0.807746in}{0.625920in}}{\pgfqpoint{0.796696in}{0.625920in}}%
\pgfpathcurveto{\pgfqpoint{0.785646in}{0.625920in}}{\pgfqpoint{0.775047in}{0.621529in}}{\pgfqpoint{0.767233in}{0.613716in}}%
\pgfpathcurveto{\pgfqpoint{0.759419in}{0.605902in}}{\pgfqpoint{0.755029in}{0.595303in}}{\pgfqpoint{0.755029in}{0.584253in}}%
\pgfpathcurveto{\pgfqpoint{0.755029in}{0.573203in}}{\pgfqpoint{0.759419in}{0.562604in}}{\pgfqpoint{0.767233in}{0.554790in}}%
\pgfpathcurveto{\pgfqpoint{0.775047in}{0.546977in}}{\pgfqpoint{0.785646in}{0.542586in}}{\pgfqpoint{0.796696in}{0.542586in}}%
\pgfpathclose%
\pgfusepath{stroke,fill}%
\end{pgfscope}%
\begin{pgfscope}%
\pgfpathrectangle{\pgfqpoint{0.557786in}{0.402778in}}{\pgfqpoint{4.393603in}{2.631944in}}%
\pgfusepath{clip}%
\pgfsetbuttcap%
\pgfsetroundjoin%
\definecolor{currentfill}{rgb}{0.172549,0.627451,0.172549}%
\pgfsetfillcolor{currentfill}%
\pgfsetfillopacity{0.500000}%
\pgfsetlinewidth{1.003750pt}%
\definecolor{currentstroke}{rgb}{0.172549,0.627451,0.172549}%
\pgfsetstrokecolor{currentstroke}%
\pgfsetstrokeopacity{0.500000}%
\pgfsetdash{}{0pt}%
\pgfpathmoveto{\pgfqpoint{0.796696in}{0.542586in}}%
\pgfpathcurveto{\pgfqpoint{0.807746in}{0.542586in}}{\pgfqpoint{0.818345in}{0.546977in}}{\pgfqpoint{0.826159in}{0.554790in}}%
\pgfpathcurveto{\pgfqpoint{0.833972in}{0.562604in}}{\pgfqpoint{0.838363in}{0.573203in}}{\pgfqpoint{0.838363in}{0.584253in}}%
\pgfpathcurveto{\pgfqpoint{0.838363in}{0.595303in}}{\pgfqpoint{0.833972in}{0.605902in}}{\pgfqpoint{0.826159in}{0.613716in}}%
\pgfpathcurveto{\pgfqpoint{0.818345in}{0.621529in}}{\pgfqpoint{0.807746in}{0.625920in}}{\pgfqpoint{0.796696in}{0.625920in}}%
\pgfpathcurveto{\pgfqpoint{0.785646in}{0.625920in}}{\pgfqpoint{0.775047in}{0.621529in}}{\pgfqpoint{0.767233in}{0.613716in}}%
\pgfpathcurveto{\pgfqpoint{0.759419in}{0.605902in}}{\pgfqpoint{0.755029in}{0.595303in}}{\pgfqpoint{0.755029in}{0.584253in}}%
\pgfpathcurveto{\pgfqpoint{0.755029in}{0.573203in}}{\pgfqpoint{0.759419in}{0.562604in}}{\pgfqpoint{0.767233in}{0.554790in}}%
\pgfpathcurveto{\pgfqpoint{0.775047in}{0.546977in}}{\pgfqpoint{0.785646in}{0.542586in}}{\pgfqpoint{0.796696in}{0.542586in}}%
\pgfpathclose%
\pgfusepath{stroke,fill}%
\end{pgfscope}%
\begin{pgfscope}%
\pgfsetbuttcap%
\pgfsetroundjoin%
\definecolor{currentfill}{rgb}{0.000000,0.000000,0.000000}%
\pgfsetfillcolor{currentfill}%
\pgfsetlinewidth{0.803000pt}%
\definecolor{currentstroke}{rgb}{0.000000,0.000000,0.000000}%
\pgfsetstrokecolor{currentstroke}%
\pgfsetdash{}{0pt}%
\pgfsys@defobject{currentmarker}{\pgfqpoint{0.000000in}{-0.048611in}}{\pgfqpoint{0.000000in}{0.000000in}}{%
\pgfpathmoveto{\pgfqpoint{0.000000in}{0.000000in}}%
\pgfpathlineto{\pgfqpoint{0.000000in}{-0.048611in}}%
\pgfusepath{stroke,fill}%
}%
\begin{pgfscope}%
\pgfsys@transformshift{0.874136in}{0.402778in}%
\pgfsys@useobject{currentmarker}{}%
\end{pgfscope}%
\end{pgfscope}%
\begin{pgfscope}%
\pgftext[x=0.874136in,y=0.305556in,,top]{\rmfamily\fontsize{10.000000}{12.000000}\selectfont \(\displaystyle 10^{-12}\)}%
\end{pgfscope}%
\begin{pgfscope}%
\pgfsetbuttcap%
\pgfsetroundjoin%
\definecolor{currentfill}{rgb}{0.000000,0.000000,0.000000}%
\pgfsetfillcolor{currentfill}%
\pgfsetlinewidth{0.803000pt}%
\definecolor{currentstroke}{rgb}{0.000000,0.000000,0.000000}%
\pgfsetstrokecolor{currentstroke}%
\pgfsetdash{}{0pt}%
\pgfsys@defobject{currentmarker}{\pgfqpoint{0.000000in}{-0.048611in}}{\pgfqpoint{0.000000in}{0.000000in}}{%
\pgfpathmoveto{\pgfqpoint{0.000000in}{0.000000in}}%
\pgfpathlineto{\pgfqpoint{0.000000in}{-0.048611in}}%
\pgfusepath{stroke,fill}%
}%
\begin{pgfscope}%
\pgfsys@transformshift{1.352500in}{0.402778in}%
\pgfsys@useobject{currentmarker}{}%
\end{pgfscope}%
\end{pgfscope}%
\begin{pgfscope}%
\pgftext[x=1.352500in,y=0.305556in,,top]{\rmfamily\fontsize{10.000000}{12.000000}\selectfont \(\displaystyle 10^{-11}\)}%
\end{pgfscope}%
\begin{pgfscope}%
\pgfsetbuttcap%
\pgfsetroundjoin%
\definecolor{currentfill}{rgb}{0.000000,0.000000,0.000000}%
\pgfsetfillcolor{currentfill}%
\pgfsetlinewidth{0.803000pt}%
\definecolor{currentstroke}{rgb}{0.000000,0.000000,0.000000}%
\pgfsetstrokecolor{currentstroke}%
\pgfsetdash{}{0pt}%
\pgfsys@defobject{currentmarker}{\pgfqpoint{0.000000in}{-0.048611in}}{\pgfqpoint{0.000000in}{0.000000in}}{%
\pgfpathmoveto{\pgfqpoint{0.000000in}{0.000000in}}%
\pgfpathlineto{\pgfqpoint{0.000000in}{-0.048611in}}%
\pgfusepath{stroke,fill}%
}%
\begin{pgfscope}%
\pgfsys@transformshift{1.830863in}{0.402778in}%
\pgfsys@useobject{currentmarker}{}%
\end{pgfscope}%
\end{pgfscope}%
\begin{pgfscope}%
\pgftext[x=1.830863in,y=0.305556in,,top]{\rmfamily\fontsize{10.000000}{12.000000}\selectfont \(\displaystyle 10^{-10}\)}%
\end{pgfscope}%
\begin{pgfscope}%
\pgfsetbuttcap%
\pgfsetroundjoin%
\definecolor{currentfill}{rgb}{0.000000,0.000000,0.000000}%
\pgfsetfillcolor{currentfill}%
\pgfsetlinewidth{0.803000pt}%
\definecolor{currentstroke}{rgb}{0.000000,0.000000,0.000000}%
\pgfsetstrokecolor{currentstroke}%
\pgfsetdash{}{0pt}%
\pgfsys@defobject{currentmarker}{\pgfqpoint{0.000000in}{-0.048611in}}{\pgfqpoint{0.000000in}{0.000000in}}{%
\pgfpathmoveto{\pgfqpoint{0.000000in}{0.000000in}}%
\pgfpathlineto{\pgfqpoint{0.000000in}{-0.048611in}}%
\pgfusepath{stroke,fill}%
}%
\begin{pgfscope}%
\pgfsys@transformshift{2.309227in}{0.402778in}%
\pgfsys@useobject{currentmarker}{}%
\end{pgfscope}%
\end{pgfscope}%
\begin{pgfscope}%
\pgftext[x=2.309227in,y=0.305556in,,top]{\rmfamily\fontsize{10.000000}{12.000000}\selectfont \(\displaystyle 10^{-9}\)}%
\end{pgfscope}%
\begin{pgfscope}%
\pgfsetbuttcap%
\pgfsetroundjoin%
\definecolor{currentfill}{rgb}{0.000000,0.000000,0.000000}%
\pgfsetfillcolor{currentfill}%
\pgfsetlinewidth{0.803000pt}%
\definecolor{currentstroke}{rgb}{0.000000,0.000000,0.000000}%
\pgfsetstrokecolor{currentstroke}%
\pgfsetdash{}{0pt}%
\pgfsys@defobject{currentmarker}{\pgfqpoint{0.000000in}{-0.048611in}}{\pgfqpoint{0.000000in}{0.000000in}}{%
\pgfpathmoveto{\pgfqpoint{0.000000in}{0.000000in}}%
\pgfpathlineto{\pgfqpoint{0.000000in}{-0.048611in}}%
\pgfusepath{stroke,fill}%
}%
\begin{pgfscope}%
\pgfsys@transformshift{2.787591in}{0.402778in}%
\pgfsys@useobject{currentmarker}{}%
\end{pgfscope}%
\end{pgfscope}%
\begin{pgfscope}%
\pgftext[x=2.787591in,y=0.305556in,,top]{\rmfamily\fontsize{10.000000}{12.000000}\selectfont \(\displaystyle 10^{-8}\)}%
\end{pgfscope}%
\begin{pgfscope}%
\pgfsetbuttcap%
\pgfsetroundjoin%
\definecolor{currentfill}{rgb}{0.000000,0.000000,0.000000}%
\pgfsetfillcolor{currentfill}%
\pgfsetlinewidth{0.803000pt}%
\definecolor{currentstroke}{rgb}{0.000000,0.000000,0.000000}%
\pgfsetstrokecolor{currentstroke}%
\pgfsetdash{}{0pt}%
\pgfsys@defobject{currentmarker}{\pgfqpoint{0.000000in}{-0.048611in}}{\pgfqpoint{0.000000in}{0.000000in}}{%
\pgfpathmoveto{\pgfqpoint{0.000000in}{0.000000in}}%
\pgfpathlineto{\pgfqpoint{0.000000in}{-0.048611in}}%
\pgfusepath{stroke,fill}%
}%
\begin{pgfscope}%
\pgfsys@transformshift{3.265955in}{0.402778in}%
\pgfsys@useobject{currentmarker}{}%
\end{pgfscope}%
\end{pgfscope}%
\begin{pgfscope}%
\pgftext[x=3.265955in,y=0.305556in,,top]{\rmfamily\fontsize{10.000000}{12.000000}\selectfont \(\displaystyle 10^{-7}\)}%
\end{pgfscope}%
\begin{pgfscope}%
\pgfsetbuttcap%
\pgfsetroundjoin%
\definecolor{currentfill}{rgb}{0.000000,0.000000,0.000000}%
\pgfsetfillcolor{currentfill}%
\pgfsetlinewidth{0.803000pt}%
\definecolor{currentstroke}{rgb}{0.000000,0.000000,0.000000}%
\pgfsetstrokecolor{currentstroke}%
\pgfsetdash{}{0pt}%
\pgfsys@defobject{currentmarker}{\pgfqpoint{0.000000in}{-0.048611in}}{\pgfqpoint{0.000000in}{0.000000in}}{%
\pgfpathmoveto{\pgfqpoint{0.000000in}{0.000000in}}%
\pgfpathlineto{\pgfqpoint{0.000000in}{-0.048611in}}%
\pgfusepath{stroke,fill}%
}%
\begin{pgfscope}%
\pgfsys@transformshift{3.744319in}{0.402778in}%
\pgfsys@useobject{currentmarker}{}%
\end{pgfscope}%
\end{pgfscope}%
\begin{pgfscope}%
\pgftext[x=3.744319in,y=0.305556in,,top]{\rmfamily\fontsize{10.000000}{12.000000}\selectfont \(\displaystyle 10^{-6}\)}%
\end{pgfscope}%
\begin{pgfscope}%
\pgfsetbuttcap%
\pgfsetroundjoin%
\definecolor{currentfill}{rgb}{0.000000,0.000000,0.000000}%
\pgfsetfillcolor{currentfill}%
\pgfsetlinewidth{0.803000pt}%
\definecolor{currentstroke}{rgb}{0.000000,0.000000,0.000000}%
\pgfsetstrokecolor{currentstroke}%
\pgfsetdash{}{0pt}%
\pgfsys@defobject{currentmarker}{\pgfqpoint{0.000000in}{-0.048611in}}{\pgfqpoint{0.000000in}{0.000000in}}{%
\pgfpathmoveto{\pgfqpoint{0.000000in}{0.000000in}}%
\pgfpathlineto{\pgfqpoint{0.000000in}{-0.048611in}}%
\pgfusepath{stroke,fill}%
}%
\begin{pgfscope}%
\pgfsys@transformshift{4.222683in}{0.402778in}%
\pgfsys@useobject{currentmarker}{}%
\end{pgfscope}%
\end{pgfscope}%
\begin{pgfscope}%
\pgftext[x=4.222683in,y=0.305556in,,top]{\rmfamily\fontsize{10.000000}{12.000000}\selectfont \(\displaystyle 10^{-5}\)}%
\end{pgfscope}%
\begin{pgfscope}%
\pgfsetbuttcap%
\pgfsetroundjoin%
\definecolor{currentfill}{rgb}{0.000000,0.000000,0.000000}%
\pgfsetfillcolor{currentfill}%
\pgfsetlinewidth{0.803000pt}%
\definecolor{currentstroke}{rgb}{0.000000,0.000000,0.000000}%
\pgfsetstrokecolor{currentstroke}%
\pgfsetdash{}{0pt}%
\pgfsys@defobject{currentmarker}{\pgfqpoint{0.000000in}{-0.048611in}}{\pgfqpoint{0.000000in}{0.000000in}}{%
\pgfpathmoveto{\pgfqpoint{0.000000in}{0.000000in}}%
\pgfpathlineto{\pgfqpoint{0.000000in}{-0.048611in}}%
\pgfusepath{stroke,fill}%
}%
\begin{pgfscope}%
\pgfsys@transformshift{4.701047in}{0.402778in}%
\pgfsys@useobject{currentmarker}{}%
\end{pgfscope}%
\end{pgfscope}%
\begin{pgfscope}%
\pgftext[x=4.701047in,y=0.305556in,,top]{\rmfamily\fontsize{10.000000}{12.000000}\selectfont \(\displaystyle 10^{-4}\)}%
\end{pgfscope}%
\begin{pgfscope}%
\pgfsetbuttcap%
\pgfsetroundjoin%
\definecolor{currentfill}{rgb}{0.000000,0.000000,0.000000}%
\pgfsetfillcolor{currentfill}%
\pgfsetlinewidth{0.602250pt}%
\definecolor{currentstroke}{rgb}{0.000000,0.000000,0.000000}%
\pgfsetstrokecolor{currentstroke}%
\pgfsetdash{}{0pt}%
\pgfsys@defobject{currentmarker}{\pgfqpoint{0.000000in}{-0.027778in}}{\pgfqpoint{0.000000in}{0.000000in}}{%
\pgfpathmoveto{\pgfqpoint{0.000000in}{0.000000in}}%
\pgfpathlineto{\pgfqpoint{0.000000in}{-0.027778in}}%
\pgfusepath{stroke,fill}%
}%
\begin{pgfscope}%
\pgfsys@transformshift{0.624009in}{0.402778in}%
\pgfsys@useobject{currentmarker}{}%
\end{pgfscope}%
\end{pgfscope}%
\begin{pgfscope}%
\pgfsetbuttcap%
\pgfsetroundjoin%
\definecolor{currentfill}{rgb}{0.000000,0.000000,0.000000}%
\pgfsetfillcolor{currentfill}%
\pgfsetlinewidth{0.602250pt}%
\definecolor{currentstroke}{rgb}{0.000000,0.000000,0.000000}%
\pgfsetstrokecolor{currentstroke}%
\pgfsetdash{}{0pt}%
\pgfsys@defobject{currentmarker}{\pgfqpoint{0.000000in}{-0.027778in}}{\pgfqpoint{0.000000in}{0.000000in}}{%
\pgfpathmoveto{\pgfqpoint{0.000000in}{0.000000in}}%
\pgfpathlineto{\pgfqpoint{0.000000in}{-0.027778in}}%
\pgfusepath{stroke,fill}%
}%
\begin{pgfscope}%
\pgfsys@transformshift{0.683776in}{0.402778in}%
\pgfsys@useobject{currentmarker}{}%
\end{pgfscope}%
\end{pgfscope}%
\begin{pgfscope}%
\pgfsetbuttcap%
\pgfsetroundjoin%
\definecolor{currentfill}{rgb}{0.000000,0.000000,0.000000}%
\pgfsetfillcolor{currentfill}%
\pgfsetlinewidth{0.602250pt}%
\definecolor{currentstroke}{rgb}{0.000000,0.000000,0.000000}%
\pgfsetstrokecolor{currentstroke}%
\pgfsetdash{}{0pt}%
\pgfsys@defobject{currentmarker}{\pgfqpoint{0.000000in}{-0.027778in}}{\pgfqpoint{0.000000in}{0.000000in}}{%
\pgfpathmoveto{\pgfqpoint{0.000000in}{0.000000in}}%
\pgfpathlineto{\pgfqpoint{0.000000in}{-0.027778in}}%
\pgfusepath{stroke,fill}%
}%
\begin{pgfscope}%
\pgfsys@transformshift{0.730134in}{0.402778in}%
\pgfsys@useobject{currentmarker}{}%
\end{pgfscope}%
\end{pgfscope}%
\begin{pgfscope}%
\pgfsetbuttcap%
\pgfsetroundjoin%
\definecolor{currentfill}{rgb}{0.000000,0.000000,0.000000}%
\pgfsetfillcolor{currentfill}%
\pgfsetlinewidth{0.602250pt}%
\definecolor{currentstroke}{rgb}{0.000000,0.000000,0.000000}%
\pgfsetstrokecolor{currentstroke}%
\pgfsetdash{}{0pt}%
\pgfsys@defobject{currentmarker}{\pgfqpoint{0.000000in}{-0.027778in}}{\pgfqpoint{0.000000in}{0.000000in}}{%
\pgfpathmoveto{\pgfqpoint{0.000000in}{0.000000in}}%
\pgfpathlineto{\pgfqpoint{0.000000in}{-0.027778in}}%
\pgfusepath{stroke,fill}%
}%
\begin{pgfscope}%
\pgfsys@transformshift{0.768011in}{0.402778in}%
\pgfsys@useobject{currentmarker}{}%
\end{pgfscope}%
\end{pgfscope}%
\begin{pgfscope}%
\pgfsetbuttcap%
\pgfsetroundjoin%
\definecolor{currentfill}{rgb}{0.000000,0.000000,0.000000}%
\pgfsetfillcolor{currentfill}%
\pgfsetlinewidth{0.602250pt}%
\definecolor{currentstroke}{rgb}{0.000000,0.000000,0.000000}%
\pgfsetstrokecolor{currentstroke}%
\pgfsetdash{}{0pt}%
\pgfsys@defobject{currentmarker}{\pgfqpoint{0.000000in}{-0.027778in}}{\pgfqpoint{0.000000in}{0.000000in}}{%
\pgfpathmoveto{\pgfqpoint{0.000000in}{0.000000in}}%
\pgfpathlineto{\pgfqpoint{0.000000in}{-0.027778in}}%
\pgfusepath{stroke,fill}%
}%
\begin{pgfscope}%
\pgfsys@transformshift{0.800036in}{0.402778in}%
\pgfsys@useobject{currentmarker}{}%
\end{pgfscope}%
\end{pgfscope}%
\begin{pgfscope}%
\pgfsetbuttcap%
\pgfsetroundjoin%
\definecolor{currentfill}{rgb}{0.000000,0.000000,0.000000}%
\pgfsetfillcolor{currentfill}%
\pgfsetlinewidth{0.602250pt}%
\definecolor{currentstroke}{rgb}{0.000000,0.000000,0.000000}%
\pgfsetstrokecolor{currentstroke}%
\pgfsetdash{}{0pt}%
\pgfsys@defobject{currentmarker}{\pgfqpoint{0.000000in}{-0.027778in}}{\pgfqpoint{0.000000in}{0.000000in}}{%
\pgfpathmoveto{\pgfqpoint{0.000000in}{0.000000in}}%
\pgfpathlineto{\pgfqpoint{0.000000in}{-0.027778in}}%
\pgfusepath{stroke,fill}%
}%
\begin{pgfscope}%
\pgfsys@transformshift{0.827777in}{0.402778in}%
\pgfsys@useobject{currentmarker}{}%
\end{pgfscope}%
\end{pgfscope}%
\begin{pgfscope}%
\pgfsetbuttcap%
\pgfsetroundjoin%
\definecolor{currentfill}{rgb}{0.000000,0.000000,0.000000}%
\pgfsetfillcolor{currentfill}%
\pgfsetlinewidth{0.602250pt}%
\definecolor{currentstroke}{rgb}{0.000000,0.000000,0.000000}%
\pgfsetstrokecolor{currentstroke}%
\pgfsetdash{}{0pt}%
\pgfsys@defobject{currentmarker}{\pgfqpoint{0.000000in}{-0.027778in}}{\pgfqpoint{0.000000in}{0.000000in}}{%
\pgfpathmoveto{\pgfqpoint{0.000000in}{0.000000in}}%
\pgfpathlineto{\pgfqpoint{0.000000in}{-0.027778in}}%
\pgfusepath{stroke,fill}%
}%
\begin{pgfscope}%
\pgfsys@transformshift{0.852247in}{0.402778in}%
\pgfsys@useobject{currentmarker}{}%
\end{pgfscope}%
\end{pgfscope}%
\begin{pgfscope}%
\pgfsetbuttcap%
\pgfsetroundjoin%
\definecolor{currentfill}{rgb}{0.000000,0.000000,0.000000}%
\pgfsetfillcolor{currentfill}%
\pgfsetlinewidth{0.602250pt}%
\definecolor{currentstroke}{rgb}{0.000000,0.000000,0.000000}%
\pgfsetstrokecolor{currentstroke}%
\pgfsetdash{}{0pt}%
\pgfsys@defobject{currentmarker}{\pgfqpoint{0.000000in}{-0.027778in}}{\pgfqpoint{0.000000in}{0.000000in}}{%
\pgfpathmoveto{\pgfqpoint{0.000000in}{0.000000in}}%
\pgfpathlineto{\pgfqpoint{0.000000in}{-0.027778in}}%
\pgfusepath{stroke,fill}%
}%
\begin{pgfscope}%
\pgfsys@transformshift{1.018138in}{0.402778in}%
\pgfsys@useobject{currentmarker}{}%
\end{pgfscope}%
\end{pgfscope}%
\begin{pgfscope}%
\pgfsetbuttcap%
\pgfsetroundjoin%
\definecolor{currentfill}{rgb}{0.000000,0.000000,0.000000}%
\pgfsetfillcolor{currentfill}%
\pgfsetlinewidth{0.602250pt}%
\definecolor{currentstroke}{rgb}{0.000000,0.000000,0.000000}%
\pgfsetstrokecolor{currentstroke}%
\pgfsetdash{}{0pt}%
\pgfsys@defobject{currentmarker}{\pgfqpoint{0.000000in}{-0.027778in}}{\pgfqpoint{0.000000in}{0.000000in}}{%
\pgfpathmoveto{\pgfqpoint{0.000000in}{0.000000in}}%
\pgfpathlineto{\pgfqpoint{0.000000in}{-0.027778in}}%
\pgfusepath{stroke,fill}%
}%
\begin{pgfscope}%
\pgfsys@transformshift{1.102373in}{0.402778in}%
\pgfsys@useobject{currentmarker}{}%
\end{pgfscope}%
\end{pgfscope}%
\begin{pgfscope}%
\pgfsetbuttcap%
\pgfsetroundjoin%
\definecolor{currentfill}{rgb}{0.000000,0.000000,0.000000}%
\pgfsetfillcolor{currentfill}%
\pgfsetlinewidth{0.602250pt}%
\definecolor{currentstroke}{rgb}{0.000000,0.000000,0.000000}%
\pgfsetstrokecolor{currentstroke}%
\pgfsetdash{}{0pt}%
\pgfsys@defobject{currentmarker}{\pgfqpoint{0.000000in}{-0.027778in}}{\pgfqpoint{0.000000in}{0.000000in}}{%
\pgfpathmoveto{\pgfqpoint{0.000000in}{0.000000in}}%
\pgfpathlineto{\pgfqpoint{0.000000in}{-0.027778in}}%
\pgfusepath{stroke,fill}%
}%
\begin{pgfscope}%
\pgfsys@transformshift{1.162139in}{0.402778in}%
\pgfsys@useobject{currentmarker}{}%
\end{pgfscope}%
\end{pgfscope}%
\begin{pgfscope}%
\pgfsetbuttcap%
\pgfsetroundjoin%
\definecolor{currentfill}{rgb}{0.000000,0.000000,0.000000}%
\pgfsetfillcolor{currentfill}%
\pgfsetlinewidth{0.602250pt}%
\definecolor{currentstroke}{rgb}{0.000000,0.000000,0.000000}%
\pgfsetstrokecolor{currentstroke}%
\pgfsetdash{}{0pt}%
\pgfsys@defobject{currentmarker}{\pgfqpoint{0.000000in}{-0.027778in}}{\pgfqpoint{0.000000in}{0.000000in}}{%
\pgfpathmoveto{\pgfqpoint{0.000000in}{0.000000in}}%
\pgfpathlineto{\pgfqpoint{0.000000in}{-0.027778in}}%
\pgfusepath{stroke,fill}%
}%
\begin{pgfscope}%
\pgfsys@transformshift{1.208498in}{0.402778in}%
\pgfsys@useobject{currentmarker}{}%
\end{pgfscope}%
\end{pgfscope}%
\begin{pgfscope}%
\pgfsetbuttcap%
\pgfsetroundjoin%
\definecolor{currentfill}{rgb}{0.000000,0.000000,0.000000}%
\pgfsetfillcolor{currentfill}%
\pgfsetlinewidth{0.602250pt}%
\definecolor{currentstroke}{rgb}{0.000000,0.000000,0.000000}%
\pgfsetstrokecolor{currentstroke}%
\pgfsetdash{}{0pt}%
\pgfsys@defobject{currentmarker}{\pgfqpoint{0.000000in}{-0.027778in}}{\pgfqpoint{0.000000in}{0.000000in}}{%
\pgfpathmoveto{\pgfqpoint{0.000000in}{0.000000in}}%
\pgfpathlineto{\pgfqpoint{0.000000in}{-0.027778in}}%
\pgfusepath{stroke,fill}%
}%
\begin{pgfscope}%
\pgfsys@transformshift{1.246375in}{0.402778in}%
\pgfsys@useobject{currentmarker}{}%
\end{pgfscope}%
\end{pgfscope}%
\begin{pgfscope}%
\pgfsetbuttcap%
\pgfsetroundjoin%
\definecolor{currentfill}{rgb}{0.000000,0.000000,0.000000}%
\pgfsetfillcolor{currentfill}%
\pgfsetlinewidth{0.602250pt}%
\definecolor{currentstroke}{rgb}{0.000000,0.000000,0.000000}%
\pgfsetstrokecolor{currentstroke}%
\pgfsetdash{}{0pt}%
\pgfsys@defobject{currentmarker}{\pgfqpoint{0.000000in}{-0.027778in}}{\pgfqpoint{0.000000in}{0.000000in}}{%
\pgfpathmoveto{\pgfqpoint{0.000000in}{0.000000in}}%
\pgfpathlineto{\pgfqpoint{0.000000in}{-0.027778in}}%
\pgfusepath{stroke,fill}%
}%
\begin{pgfscope}%
\pgfsys@transformshift{1.278400in}{0.402778in}%
\pgfsys@useobject{currentmarker}{}%
\end{pgfscope}%
\end{pgfscope}%
\begin{pgfscope}%
\pgfsetbuttcap%
\pgfsetroundjoin%
\definecolor{currentfill}{rgb}{0.000000,0.000000,0.000000}%
\pgfsetfillcolor{currentfill}%
\pgfsetlinewidth{0.602250pt}%
\definecolor{currentstroke}{rgb}{0.000000,0.000000,0.000000}%
\pgfsetstrokecolor{currentstroke}%
\pgfsetdash{}{0pt}%
\pgfsys@defobject{currentmarker}{\pgfqpoint{0.000000in}{-0.027778in}}{\pgfqpoint{0.000000in}{0.000000in}}{%
\pgfpathmoveto{\pgfqpoint{0.000000in}{0.000000in}}%
\pgfpathlineto{\pgfqpoint{0.000000in}{-0.027778in}}%
\pgfusepath{stroke,fill}%
}%
\begin{pgfscope}%
\pgfsys@transformshift{1.306141in}{0.402778in}%
\pgfsys@useobject{currentmarker}{}%
\end{pgfscope}%
\end{pgfscope}%
\begin{pgfscope}%
\pgfsetbuttcap%
\pgfsetroundjoin%
\definecolor{currentfill}{rgb}{0.000000,0.000000,0.000000}%
\pgfsetfillcolor{currentfill}%
\pgfsetlinewidth{0.602250pt}%
\definecolor{currentstroke}{rgb}{0.000000,0.000000,0.000000}%
\pgfsetstrokecolor{currentstroke}%
\pgfsetdash{}{0pt}%
\pgfsys@defobject{currentmarker}{\pgfqpoint{0.000000in}{-0.027778in}}{\pgfqpoint{0.000000in}{0.000000in}}{%
\pgfpathmoveto{\pgfqpoint{0.000000in}{0.000000in}}%
\pgfpathlineto{\pgfqpoint{0.000000in}{-0.027778in}}%
\pgfusepath{stroke,fill}%
}%
\begin{pgfscope}%
\pgfsys@transformshift{1.330611in}{0.402778in}%
\pgfsys@useobject{currentmarker}{}%
\end{pgfscope}%
\end{pgfscope}%
\begin{pgfscope}%
\pgfsetbuttcap%
\pgfsetroundjoin%
\definecolor{currentfill}{rgb}{0.000000,0.000000,0.000000}%
\pgfsetfillcolor{currentfill}%
\pgfsetlinewidth{0.602250pt}%
\definecolor{currentstroke}{rgb}{0.000000,0.000000,0.000000}%
\pgfsetstrokecolor{currentstroke}%
\pgfsetdash{}{0pt}%
\pgfsys@defobject{currentmarker}{\pgfqpoint{0.000000in}{-0.027778in}}{\pgfqpoint{0.000000in}{0.000000in}}{%
\pgfpathmoveto{\pgfqpoint{0.000000in}{0.000000in}}%
\pgfpathlineto{\pgfqpoint{0.000000in}{-0.027778in}}%
\pgfusepath{stroke,fill}%
}%
\begin{pgfscope}%
\pgfsys@transformshift{1.496501in}{0.402778in}%
\pgfsys@useobject{currentmarker}{}%
\end{pgfscope}%
\end{pgfscope}%
\begin{pgfscope}%
\pgfsetbuttcap%
\pgfsetroundjoin%
\definecolor{currentfill}{rgb}{0.000000,0.000000,0.000000}%
\pgfsetfillcolor{currentfill}%
\pgfsetlinewidth{0.602250pt}%
\definecolor{currentstroke}{rgb}{0.000000,0.000000,0.000000}%
\pgfsetstrokecolor{currentstroke}%
\pgfsetdash{}{0pt}%
\pgfsys@defobject{currentmarker}{\pgfqpoint{0.000000in}{-0.027778in}}{\pgfqpoint{0.000000in}{0.000000in}}{%
\pgfpathmoveto{\pgfqpoint{0.000000in}{0.000000in}}%
\pgfpathlineto{\pgfqpoint{0.000000in}{-0.027778in}}%
\pgfusepath{stroke,fill}%
}%
\begin{pgfscope}%
\pgfsys@transformshift{1.580737in}{0.402778in}%
\pgfsys@useobject{currentmarker}{}%
\end{pgfscope}%
\end{pgfscope}%
\begin{pgfscope}%
\pgfsetbuttcap%
\pgfsetroundjoin%
\definecolor{currentfill}{rgb}{0.000000,0.000000,0.000000}%
\pgfsetfillcolor{currentfill}%
\pgfsetlinewidth{0.602250pt}%
\definecolor{currentstroke}{rgb}{0.000000,0.000000,0.000000}%
\pgfsetstrokecolor{currentstroke}%
\pgfsetdash{}{0pt}%
\pgfsys@defobject{currentmarker}{\pgfqpoint{0.000000in}{-0.027778in}}{\pgfqpoint{0.000000in}{0.000000in}}{%
\pgfpathmoveto{\pgfqpoint{0.000000in}{0.000000in}}%
\pgfpathlineto{\pgfqpoint{0.000000in}{-0.027778in}}%
\pgfusepath{stroke,fill}%
}%
\begin{pgfscope}%
\pgfsys@transformshift{1.640503in}{0.402778in}%
\pgfsys@useobject{currentmarker}{}%
\end{pgfscope}%
\end{pgfscope}%
\begin{pgfscope}%
\pgfsetbuttcap%
\pgfsetroundjoin%
\definecolor{currentfill}{rgb}{0.000000,0.000000,0.000000}%
\pgfsetfillcolor{currentfill}%
\pgfsetlinewidth{0.602250pt}%
\definecolor{currentstroke}{rgb}{0.000000,0.000000,0.000000}%
\pgfsetstrokecolor{currentstroke}%
\pgfsetdash{}{0pt}%
\pgfsys@defobject{currentmarker}{\pgfqpoint{0.000000in}{-0.027778in}}{\pgfqpoint{0.000000in}{0.000000in}}{%
\pgfpathmoveto{\pgfqpoint{0.000000in}{0.000000in}}%
\pgfpathlineto{\pgfqpoint{0.000000in}{-0.027778in}}%
\pgfusepath{stroke,fill}%
}%
\begin{pgfscope}%
\pgfsys@transformshift{1.686862in}{0.402778in}%
\pgfsys@useobject{currentmarker}{}%
\end{pgfscope}%
\end{pgfscope}%
\begin{pgfscope}%
\pgfsetbuttcap%
\pgfsetroundjoin%
\definecolor{currentfill}{rgb}{0.000000,0.000000,0.000000}%
\pgfsetfillcolor{currentfill}%
\pgfsetlinewidth{0.602250pt}%
\definecolor{currentstroke}{rgb}{0.000000,0.000000,0.000000}%
\pgfsetstrokecolor{currentstroke}%
\pgfsetdash{}{0pt}%
\pgfsys@defobject{currentmarker}{\pgfqpoint{0.000000in}{-0.027778in}}{\pgfqpoint{0.000000in}{0.000000in}}{%
\pgfpathmoveto{\pgfqpoint{0.000000in}{0.000000in}}%
\pgfpathlineto{\pgfqpoint{0.000000in}{-0.027778in}}%
\pgfusepath{stroke,fill}%
}%
\begin{pgfscope}%
\pgfsys@transformshift{1.724739in}{0.402778in}%
\pgfsys@useobject{currentmarker}{}%
\end{pgfscope}%
\end{pgfscope}%
\begin{pgfscope}%
\pgfsetbuttcap%
\pgfsetroundjoin%
\definecolor{currentfill}{rgb}{0.000000,0.000000,0.000000}%
\pgfsetfillcolor{currentfill}%
\pgfsetlinewidth{0.602250pt}%
\definecolor{currentstroke}{rgb}{0.000000,0.000000,0.000000}%
\pgfsetstrokecolor{currentstroke}%
\pgfsetdash{}{0pt}%
\pgfsys@defobject{currentmarker}{\pgfqpoint{0.000000in}{-0.027778in}}{\pgfqpoint{0.000000in}{0.000000in}}{%
\pgfpathmoveto{\pgfqpoint{0.000000in}{0.000000in}}%
\pgfpathlineto{\pgfqpoint{0.000000in}{-0.027778in}}%
\pgfusepath{stroke,fill}%
}%
\begin{pgfscope}%
\pgfsys@transformshift{1.756764in}{0.402778in}%
\pgfsys@useobject{currentmarker}{}%
\end{pgfscope}%
\end{pgfscope}%
\begin{pgfscope}%
\pgfsetbuttcap%
\pgfsetroundjoin%
\definecolor{currentfill}{rgb}{0.000000,0.000000,0.000000}%
\pgfsetfillcolor{currentfill}%
\pgfsetlinewidth{0.602250pt}%
\definecolor{currentstroke}{rgb}{0.000000,0.000000,0.000000}%
\pgfsetstrokecolor{currentstroke}%
\pgfsetdash{}{0pt}%
\pgfsys@defobject{currentmarker}{\pgfqpoint{0.000000in}{-0.027778in}}{\pgfqpoint{0.000000in}{0.000000in}}{%
\pgfpathmoveto{\pgfqpoint{0.000000in}{0.000000in}}%
\pgfpathlineto{\pgfqpoint{0.000000in}{-0.027778in}}%
\pgfusepath{stroke,fill}%
}%
\begin{pgfscope}%
\pgfsys@transformshift{1.784505in}{0.402778in}%
\pgfsys@useobject{currentmarker}{}%
\end{pgfscope}%
\end{pgfscope}%
\begin{pgfscope}%
\pgfsetbuttcap%
\pgfsetroundjoin%
\definecolor{currentfill}{rgb}{0.000000,0.000000,0.000000}%
\pgfsetfillcolor{currentfill}%
\pgfsetlinewidth{0.602250pt}%
\definecolor{currentstroke}{rgb}{0.000000,0.000000,0.000000}%
\pgfsetstrokecolor{currentstroke}%
\pgfsetdash{}{0pt}%
\pgfsys@defobject{currentmarker}{\pgfqpoint{0.000000in}{-0.027778in}}{\pgfqpoint{0.000000in}{0.000000in}}{%
\pgfpathmoveto{\pgfqpoint{0.000000in}{0.000000in}}%
\pgfpathlineto{\pgfqpoint{0.000000in}{-0.027778in}}%
\pgfusepath{stroke,fill}%
}%
\begin{pgfscope}%
\pgfsys@transformshift{1.808975in}{0.402778in}%
\pgfsys@useobject{currentmarker}{}%
\end{pgfscope}%
\end{pgfscope}%
\begin{pgfscope}%
\pgfsetbuttcap%
\pgfsetroundjoin%
\definecolor{currentfill}{rgb}{0.000000,0.000000,0.000000}%
\pgfsetfillcolor{currentfill}%
\pgfsetlinewidth{0.602250pt}%
\definecolor{currentstroke}{rgb}{0.000000,0.000000,0.000000}%
\pgfsetstrokecolor{currentstroke}%
\pgfsetdash{}{0pt}%
\pgfsys@defobject{currentmarker}{\pgfqpoint{0.000000in}{-0.027778in}}{\pgfqpoint{0.000000in}{0.000000in}}{%
\pgfpathmoveto{\pgfqpoint{0.000000in}{0.000000in}}%
\pgfpathlineto{\pgfqpoint{0.000000in}{-0.027778in}}%
\pgfusepath{stroke,fill}%
}%
\begin{pgfscope}%
\pgfsys@transformshift{1.974865in}{0.402778in}%
\pgfsys@useobject{currentmarker}{}%
\end{pgfscope}%
\end{pgfscope}%
\begin{pgfscope}%
\pgfsetbuttcap%
\pgfsetroundjoin%
\definecolor{currentfill}{rgb}{0.000000,0.000000,0.000000}%
\pgfsetfillcolor{currentfill}%
\pgfsetlinewidth{0.602250pt}%
\definecolor{currentstroke}{rgb}{0.000000,0.000000,0.000000}%
\pgfsetstrokecolor{currentstroke}%
\pgfsetdash{}{0pt}%
\pgfsys@defobject{currentmarker}{\pgfqpoint{0.000000in}{-0.027778in}}{\pgfqpoint{0.000000in}{0.000000in}}{%
\pgfpathmoveto{\pgfqpoint{0.000000in}{0.000000in}}%
\pgfpathlineto{\pgfqpoint{0.000000in}{-0.027778in}}%
\pgfusepath{stroke,fill}%
}%
\begin{pgfscope}%
\pgfsys@transformshift{2.059101in}{0.402778in}%
\pgfsys@useobject{currentmarker}{}%
\end{pgfscope}%
\end{pgfscope}%
\begin{pgfscope}%
\pgfsetbuttcap%
\pgfsetroundjoin%
\definecolor{currentfill}{rgb}{0.000000,0.000000,0.000000}%
\pgfsetfillcolor{currentfill}%
\pgfsetlinewidth{0.602250pt}%
\definecolor{currentstroke}{rgb}{0.000000,0.000000,0.000000}%
\pgfsetstrokecolor{currentstroke}%
\pgfsetdash{}{0pt}%
\pgfsys@defobject{currentmarker}{\pgfqpoint{0.000000in}{-0.027778in}}{\pgfqpoint{0.000000in}{0.000000in}}{%
\pgfpathmoveto{\pgfqpoint{0.000000in}{0.000000in}}%
\pgfpathlineto{\pgfqpoint{0.000000in}{-0.027778in}}%
\pgfusepath{stroke,fill}%
}%
\begin{pgfscope}%
\pgfsys@transformshift{2.118867in}{0.402778in}%
\pgfsys@useobject{currentmarker}{}%
\end{pgfscope}%
\end{pgfscope}%
\begin{pgfscope}%
\pgfsetbuttcap%
\pgfsetroundjoin%
\definecolor{currentfill}{rgb}{0.000000,0.000000,0.000000}%
\pgfsetfillcolor{currentfill}%
\pgfsetlinewidth{0.602250pt}%
\definecolor{currentstroke}{rgb}{0.000000,0.000000,0.000000}%
\pgfsetstrokecolor{currentstroke}%
\pgfsetdash{}{0pt}%
\pgfsys@defobject{currentmarker}{\pgfqpoint{0.000000in}{-0.027778in}}{\pgfqpoint{0.000000in}{0.000000in}}{%
\pgfpathmoveto{\pgfqpoint{0.000000in}{0.000000in}}%
\pgfpathlineto{\pgfqpoint{0.000000in}{-0.027778in}}%
\pgfusepath{stroke,fill}%
}%
\begin{pgfscope}%
\pgfsys@transformshift{2.165225in}{0.402778in}%
\pgfsys@useobject{currentmarker}{}%
\end{pgfscope}%
\end{pgfscope}%
\begin{pgfscope}%
\pgfsetbuttcap%
\pgfsetroundjoin%
\definecolor{currentfill}{rgb}{0.000000,0.000000,0.000000}%
\pgfsetfillcolor{currentfill}%
\pgfsetlinewidth{0.602250pt}%
\definecolor{currentstroke}{rgb}{0.000000,0.000000,0.000000}%
\pgfsetstrokecolor{currentstroke}%
\pgfsetdash{}{0pt}%
\pgfsys@defobject{currentmarker}{\pgfqpoint{0.000000in}{-0.027778in}}{\pgfqpoint{0.000000in}{0.000000in}}{%
\pgfpathmoveto{\pgfqpoint{0.000000in}{0.000000in}}%
\pgfpathlineto{\pgfqpoint{0.000000in}{-0.027778in}}%
\pgfusepath{stroke,fill}%
}%
\begin{pgfscope}%
\pgfsys@transformshift{2.203103in}{0.402778in}%
\pgfsys@useobject{currentmarker}{}%
\end{pgfscope}%
\end{pgfscope}%
\begin{pgfscope}%
\pgfsetbuttcap%
\pgfsetroundjoin%
\definecolor{currentfill}{rgb}{0.000000,0.000000,0.000000}%
\pgfsetfillcolor{currentfill}%
\pgfsetlinewidth{0.602250pt}%
\definecolor{currentstroke}{rgb}{0.000000,0.000000,0.000000}%
\pgfsetstrokecolor{currentstroke}%
\pgfsetdash{}{0pt}%
\pgfsys@defobject{currentmarker}{\pgfqpoint{0.000000in}{-0.027778in}}{\pgfqpoint{0.000000in}{0.000000in}}{%
\pgfpathmoveto{\pgfqpoint{0.000000in}{0.000000in}}%
\pgfpathlineto{\pgfqpoint{0.000000in}{-0.027778in}}%
\pgfusepath{stroke,fill}%
}%
\begin{pgfscope}%
\pgfsys@transformshift{2.235128in}{0.402778in}%
\pgfsys@useobject{currentmarker}{}%
\end{pgfscope}%
\end{pgfscope}%
\begin{pgfscope}%
\pgfsetbuttcap%
\pgfsetroundjoin%
\definecolor{currentfill}{rgb}{0.000000,0.000000,0.000000}%
\pgfsetfillcolor{currentfill}%
\pgfsetlinewidth{0.602250pt}%
\definecolor{currentstroke}{rgb}{0.000000,0.000000,0.000000}%
\pgfsetstrokecolor{currentstroke}%
\pgfsetdash{}{0pt}%
\pgfsys@defobject{currentmarker}{\pgfqpoint{0.000000in}{-0.027778in}}{\pgfqpoint{0.000000in}{0.000000in}}{%
\pgfpathmoveto{\pgfqpoint{0.000000in}{0.000000in}}%
\pgfpathlineto{\pgfqpoint{0.000000in}{-0.027778in}}%
\pgfusepath{stroke,fill}%
}%
\begin{pgfscope}%
\pgfsys@transformshift{2.262869in}{0.402778in}%
\pgfsys@useobject{currentmarker}{}%
\end{pgfscope}%
\end{pgfscope}%
\begin{pgfscope}%
\pgfsetbuttcap%
\pgfsetroundjoin%
\definecolor{currentfill}{rgb}{0.000000,0.000000,0.000000}%
\pgfsetfillcolor{currentfill}%
\pgfsetlinewidth{0.602250pt}%
\definecolor{currentstroke}{rgb}{0.000000,0.000000,0.000000}%
\pgfsetstrokecolor{currentstroke}%
\pgfsetdash{}{0pt}%
\pgfsys@defobject{currentmarker}{\pgfqpoint{0.000000in}{-0.027778in}}{\pgfqpoint{0.000000in}{0.000000in}}{%
\pgfpathmoveto{\pgfqpoint{0.000000in}{0.000000in}}%
\pgfpathlineto{\pgfqpoint{0.000000in}{-0.027778in}}%
\pgfusepath{stroke,fill}%
}%
\begin{pgfscope}%
\pgfsys@transformshift{2.287339in}{0.402778in}%
\pgfsys@useobject{currentmarker}{}%
\end{pgfscope}%
\end{pgfscope}%
\begin{pgfscope}%
\pgfsetbuttcap%
\pgfsetroundjoin%
\definecolor{currentfill}{rgb}{0.000000,0.000000,0.000000}%
\pgfsetfillcolor{currentfill}%
\pgfsetlinewidth{0.602250pt}%
\definecolor{currentstroke}{rgb}{0.000000,0.000000,0.000000}%
\pgfsetstrokecolor{currentstroke}%
\pgfsetdash{}{0pt}%
\pgfsys@defobject{currentmarker}{\pgfqpoint{0.000000in}{-0.027778in}}{\pgfqpoint{0.000000in}{0.000000in}}{%
\pgfpathmoveto{\pgfqpoint{0.000000in}{0.000000in}}%
\pgfpathlineto{\pgfqpoint{0.000000in}{-0.027778in}}%
\pgfusepath{stroke,fill}%
}%
\begin{pgfscope}%
\pgfsys@transformshift{2.453229in}{0.402778in}%
\pgfsys@useobject{currentmarker}{}%
\end{pgfscope}%
\end{pgfscope}%
\begin{pgfscope}%
\pgfsetbuttcap%
\pgfsetroundjoin%
\definecolor{currentfill}{rgb}{0.000000,0.000000,0.000000}%
\pgfsetfillcolor{currentfill}%
\pgfsetlinewidth{0.602250pt}%
\definecolor{currentstroke}{rgb}{0.000000,0.000000,0.000000}%
\pgfsetstrokecolor{currentstroke}%
\pgfsetdash{}{0pt}%
\pgfsys@defobject{currentmarker}{\pgfqpoint{0.000000in}{-0.027778in}}{\pgfqpoint{0.000000in}{0.000000in}}{%
\pgfpathmoveto{\pgfqpoint{0.000000in}{0.000000in}}%
\pgfpathlineto{\pgfqpoint{0.000000in}{-0.027778in}}%
\pgfusepath{stroke,fill}%
}%
\begin{pgfscope}%
\pgfsys@transformshift{2.537465in}{0.402778in}%
\pgfsys@useobject{currentmarker}{}%
\end{pgfscope}%
\end{pgfscope}%
\begin{pgfscope}%
\pgfsetbuttcap%
\pgfsetroundjoin%
\definecolor{currentfill}{rgb}{0.000000,0.000000,0.000000}%
\pgfsetfillcolor{currentfill}%
\pgfsetlinewidth{0.602250pt}%
\definecolor{currentstroke}{rgb}{0.000000,0.000000,0.000000}%
\pgfsetstrokecolor{currentstroke}%
\pgfsetdash{}{0pt}%
\pgfsys@defobject{currentmarker}{\pgfqpoint{0.000000in}{-0.027778in}}{\pgfqpoint{0.000000in}{0.000000in}}{%
\pgfpathmoveto{\pgfqpoint{0.000000in}{0.000000in}}%
\pgfpathlineto{\pgfqpoint{0.000000in}{-0.027778in}}%
\pgfusepath{stroke,fill}%
}%
\begin{pgfscope}%
\pgfsys@transformshift{2.597231in}{0.402778in}%
\pgfsys@useobject{currentmarker}{}%
\end{pgfscope}%
\end{pgfscope}%
\begin{pgfscope}%
\pgfsetbuttcap%
\pgfsetroundjoin%
\definecolor{currentfill}{rgb}{0.000000,0.000000,0.000000}%
\pgfsetfillcolor{currentfill}%
\pgfsetlinewidth{0.602250pt}%
\definecolor{currentstroke}{rgb}{0.000000,0.000000,0.000000}%
\pgfsetstrokecolor{currentstroke}%
\pgfsetdash{}{0pt}%
\pgfsys@defobject{currentmarker}{\pgfqpoint{0.000000in}{-0.027778in}}{\pgfqpoint{0.000000in}{0.000000in}}{%
\pgfpathmoveto{\pgfqpoint{0.000000in}{0.000000in}}%
\pgfpathlineto{\pgfqpoint{0.000000in}{-0.027778in}}%
\pgfusepath{stroke,fill}%
}%
\begin{pgfscope}%
\pgfsys@transformshift{2.643589in}{0.402778in}%
\pgfsys@useobject{currentmarker}{}%
\end{pgfscope}%
\end{pgfscope}%
\begin{pgfscope}%
\pgfsetbuttcap%
\pgfsetroundjoin%
\definecolor{currentfill}{rgb}{0.000000,0.000000,0.000000}%
\pgfsetfillcolor{currentfill}%
\pgfsetlinewidth{0.602250pt}%
\definecolor{currentstroke}{rgb}{0.000000,0.000000,0.000000}%
\pgfsetstrokecolor{currentstroke}%
\pgfsetdash{}{0pt}%
\pgfsys@defobject{currentmarker}{\pgfqpoint{0.000000in}{-0.027778in}}{\pgfqpoint{0.000000in}{0.000000in}}{%
\pgfpathmoveto{\pgfqpoint{0.000000in}{0.000000in}}%
\pgfpathlineto{\pgfqpoint{0.000000in}{-0.027778in}}%
\pgfusepath{stroke,fill}%
}%
\begin{pgfscope}%
\pgfsys@transformshift{2.681467in}{0.402778in}%
\pgfsys@useobject{currentmarker}{}%
\end{pgfscope}%
\end{pgfscope}%
\begin{pgfscope}%
\pgfsetbuttcap%
\pgfsetroundjoin%
\definecolor{currentfill}{rgb}{0.000000,0.000000,0.000000}%
\pgfsetfillcolor{currentfill}%
\pgfsetlinewidth{0.602250pt}%
\definecolor{currentstroke}{rgb}{0.000000,0.000000,0.000000}%
\pgfsetstrokecolor{currentstroke}%
\pgfsetdash{}{0pt}%
\pgfsys@defobject{currentmarker}{\pgfqpoint{0.000000in}{-0.027778in}}{\pgfqpoint{0.000000in}{0.000000in}}{%
\pgfpathmoveto{\pgfqpoint{0.000000in}{0.000000in}}%
\pgfpathlineto{\pgfqpoint{0.000000in}{-0.027778in}}%
\pgfusepath{stroke,fill}%
}%
\begin{pgfscope}%
\pgfsys@transformshift{2.713492in}{0.402778in}%
\pgfsys@useobject{currentmarker}{}%
\end{pgfscope}%
\end{pgfscope}%
\begin{pgfscope}%
\pgfsetbuttcap%
\pgfsetroundjoin%
\definecolor{currentfill}{rgb}{0.000000,0.000000,0.000000}%
\pgfsetfillcolor{currentfill}%
\pgfsetlinewidth{0.602250pt}%
\definecolor{currentstroke}{rgb}{0.000000,0.000000,0.000000}%
\pgfsetstrokecolor{currentstroke}%
\pgfsetdash{}{0pt}%
\pgfsys@defobject{currentmarker}{\pgfqpoint{0.000000in}{-0.027778in}}{\pgfqpoint{0.000000in}{0.000000in}}{%
\pgfpathmoveto{\pgfqpoint{0.000000in}{0.000000in}}%
\pgfpathlineto{\pgfqpoint{0.000000in}{-0.027778in}}%
\pgfusepath{stroke,fill}%
}%
\begin{pgfscope}%
\pgfsys@transformshift{2.741233in}{0.402778in}%
\pgfsys@useobject{currentmarker}{}%
\end{pgfscope}%
\end{pgfscope}%
\begin{pgfscope}%
\pgfsetbuttcap%
\pgfsetroundjoin%
\definecolor{currentfill}{rgb}{0.000000,0.000000,0.000000}%
\pgfsetfillcolor{currentfill}%
\pgfsetlinewidth{0.602250pt}%
\definecolor{currentstroke}{rgb}{0.000000,0.000000,0.000000}%
\pgfsetstrokecolor{currentstroke}%
\pgfsetdash{}{0pt}%
\pgfsys@defobject{currentmarker}{\pgfqpoint{0.000000in}{-0.027778in}}{\pgfqpoint{0.000000in}{0.000000in}}{%
\pgfpathmoveto{\pgfqpoint{0.000000in}{0.000000in}}%
\pgfpathlineto{\pgfqpoint{0.000000in}{-0.027778in}}%
\pgfusepath{stroke,fill}%
}%
\begin{pgfscope}%
\pgfsys@transformshift{2.765702in}{0.402778in}%
\pgfsys@useobject{currentmarker}{}%
\end{pgfscope}%
\end{pgfscope}%
\begin{pgfscope}%
\pgfsetbuttcap%
\pgfsetroundjoin%
\definecolor{currentfill}{rgb}{0.000000,0.000000,0.000000}%
\pgfsetfillcolor{currentfill}%
\pgfsetlinewidth{0.602250pt}%
\definecolor{currentstroke}{rgb}{0.000000,0.000000,0.000000}%
\pgfsetstrokecolor{currentstroke}%
\pgfsetdash{}{0pt}%
\pgfsys@defobject{currentmarker}{\pgfqpoint{0.000000in}{-0.027778in}}{\pgfqpoint{0.000000in}{0.000000in}}{%
\pgfpathmoveto{\pgfqpoint{0.000000in}{0.000000in}}%
\pgfpathlineto{\pgfqpoint{0.000000in}{-0.027778in}}%
\pgfusepath{stroke,fill}%
}%
\begin{pgfscope}%
\pgfsys@transformshift{2.931593in}{0.402778in}%
\pgfsys@useobject{currentmarker}{}%
\end{pgfscope}%
\end{pgfscope}%
\begin{pgfscope}%
\pgfsetbuttcap%
\pgfsetroundjoin%
\definecolor{currentfill}{rgb}{0.000000,0.000000,0.000000}%
\pgfsetfillcolor{currentfill}%
\pgfsetlinewidth{0.602250pt}%
\definecolor{currentstroke}{rgb}{0.000000,0.000000,0.000000}%
\pgfsetstrokecolor{currentstroke}%
\pgfsetdash{}{0pt}%
\pgfsys@defobject{currentmarker}{\pgfqpoint{0.000000in}{-0.027778in}}{\pgfqpoint{0.000000in}{0.000000in}}{%
\pgfpathmoveto{\pgfqpoint{0.000000in}{0.000000in}}%
\pgfpathlineto{\pgfqpoint{0.000000in}{-0.027778in}}%
\pgfusepath{stroke,fill}%
}%
\begin{pgfscope}%
\pgfsys@transformshift{3.015829in}{0.402778in}%
\pgfsys@useobject{currentmarker}{}%
\end{pgfscope}%
\end{pgfscope}%
\begin{pgfscope}%
\pgfsetbuttcap%
\pgfsetroundjoin%
\definecolor{currentfill}{rgb}{0.000000,0.000000,0.000000}%
\pgfsetfillcolor{currentfill}%
\pgfsetlinewidth{0.602250pt}%
\definecolor{currentstroke}{rgb}{0.000000,0.000000,0.000000}%
\pgfsetstrokecolor{currentstroke}%
\pgfsetdash{}{0pt}%
\pgfsys@defobject{currentmarker}{\pgfqpoint{0.000000in}{-0.027778in}}{\pgfqpoint{0.000000in}{0.000000in}}{%
\pgfpathmoveto{\pgfqpoint{0.000000in}{0.000000in}}%
\pgfpathlineto{\pgfqpoint{0.000000in}{-0.027778in}}%
\pgfusepath{stroke,fill}%
}%
\begin{pgfscope}%
\pgfsys@transformshift{3.075595in}{0.402778in}%
\pgfsys@useobject{currentmarker}{}%
\end{pgfscope}%
\end{pgfscope}%
\begin{pgfscope}%
\pgfsetbuttcap%
\pgfsetroundjoin%
\definecolor{currentfill}{rgb}{0.000000,0.000000,0.000000}%
\pgfsetfillcolor{currentfill}%
\pgfsetlinewidth{0.602250pt}%
\definecolor{currentstroke}{rgb}{0.000000,0.000000,0.000000}%
\pgfsetstrokecolor{currentstroke}%
\pgfsetdash{}{0pt}%
\pgfsys@defobject{currentmarker}{\pgfqpoint{0.000000in}{-0.027778in}}{\pgfqpoint{0.000000in}{0.000000in}}{%
\pgfpathmoveto{\pgfqpoint{0.000000in}{0.000000in}}%
\pgfpathlineto{\pgfqpoint{0.000000in}{-0.027778in}}%
\pgfusepath{stroke,fill}%
}%
\begin{pgfscope}%
\pgfsys@transformshift{3.121953in}{0.402778in}%
\pgfsys@useobject{currentmarker}{}%
\end{pgfscope}%
\end{pgfscope}%
\begin{pgfscope}%
\pgfsetbuttcap%
\pgfsetroundjoin%
\definecolor{currentfill}{rgb}{0.000000,0.000000,0.000000}%
\pgfsetfillcolor{currentfill}%
\pgfsetlinewidth{0.602250pt}%
\definecolor{currentstroke}{rgb}{0.000000,0.000000,0.000000}%
\pgfsetstrokecolor{currentstroke}%
\pgfsetdash{}{0pt}%
\pgfsys@defobject{currentmarker}{\pgfqpoint{0.000000in}{-0.027778in}}{\pgfqpoint{0.000000in}{0.000000in}}{%
\pgfpathmoveto{\pgfqpoint{0.000000in}{0.000000in}}%
\pgfpathlineto{\pgfqpoint{0.000000in}{-0.027778in}}%
\pgfusepath{stroke,fill}%
}%
\begin{pgfscope}%
\pgfsys@transformshift{3.159831in}{0.402778in}%
\pgfsys@useobject{currentmarker}{}%
\end{pgfscope}%
\end{pgfscope}%
\begin{pgfscope}%
\pgfsetbuttcap%
\pgfsetroundjoin%
\definecolor{currentfill}{rgb}{0.000000,0.000000,0.000000}%
\pgfsetfillcolor{currentfill}%
\pgfsetlinewidth{0.602250pt}%
\definecolor{currentstroke}{rgb}{0.000000,0.000000,0.000000}%
\pgfsetstrokecolor{currentstroke}%
\pgfsetdash{}{0pt}%
\pgfsys@defobject{currentmarker}{\pgfqpoint{0.000000in}{-0.027778in}}{\pgfqpoint{0.000000in}{0.000000in}}{%
\pgfpathmoveto{\pgfqpoint{0.000000in}{0.000000in}}%
\pgfpathlineto{\pgfqpoint{0.000000in}{-0.027778in}}%
\pgfusepath{stroke,fill}%
}%
\begin{pgfscope}%
\pgfsys@transformshift{3.191856in}{0.402778in}%
\pgfsys@useobject{currentmarker}{}%
\end{pgfscope}%
\end{pgfscope}%
\begin{pgfscope}%
\pgfsetbuttcap%
\pgfsetroundjoin%
\definecolor{currentfill}{rgb}{0.000000,0.000000,0.000000}%
\pgfsetfillcolor{currentfill}%
\pgfsetlinewidth{0.602250pt}%
\definecolor{currentstroke}{rgb}{0.000000,0.000000,0.000000}%
\pgfsetstrokecolor{currentstroke}%
\pgfsetdash{}{0pt}%
\pgfsys@defobject{currentmarker}{\pgfqpoint{0.000000in}{-0.027778in}}{\pgfqpoint{0.000000in}{0.000000in}}{%
\pgfpathmoveto{\pgfqpoint{0.000000in}{0.000000in}}%
\pgfpathlineto{\pgfqpoint{0.000000in}{-0.027778in}}%
\pgfusepath{stroke,fill}%
}%
\begin{pgfscope}%
\pgfsys@transformshift{3.219597in}{0.402778in}%
\pgfsys@useobject{currentmarker}{}%
\end{pgfscope}%
\end{pgfscope}%
\begin{pgfscope}%
\pgfsetbuttcap%
\pgfsetroundjoin%
\definecolor{currentfill}{rgb}{0.000000,0.000000,0.000000}%
\pgfsetfillcolor{currentfill}%
\pgfsetlinewidth{0.602250pt}%
\definecolor{currentstroke}{rgb}{0.000000,0.000000,0.000000}%
\pgfsetstrokecolor{currentstroke}%
\pgfsetdash{}{0pt}%
\pgfsys@defobject{currentmarker}{\pgfqpoint{0.000000in}{-0.027778in}}{\pgfqpoint{0.000000in}{0.000000in}}{%
\pgfpathmoveto{\pgfqpoint{0.000000in}{0.000000in}}%
\pgfpathlineto{\pgfqpoint{0.000000in}{-0.027778in}}%
\pgfusepath{stroke,fill}%
}%
\begin{pgfscope}%
\pgfsys@transformshift{3.244066in}{0.402778in}%
\pgfsys@useobject{currentmarker}{}%
\end{pgfscope}%
\end{pgfscope}%
\begin{pgfscope}%
\pgfsetbuttcap%
\pgfsetroundjoin%
\definecolor{currentfill}{rgb}{0.000000,0.000000,0.000000}%
\pgfsetfillcolor{currentfill}%
\pgfsetlinewidth{0.602250pt}%
\definecolor{currentstroke}{rgb}{0.000000,0.000000,0.000000}%
\pgfsetstrokecolor{currentstroke}%
\pgfsetdash{}{0pt}%
\pgfsys@defobject{currentmarker}{\pgfqpoint{0.000000in}{-0.027778in}}{\pgfqpoint{0.000000in}{0.000000in}}{%
\pgfpathmoveto{\pgfqpoint{0.000000in}{0.000000in}}%
\pgfpathlineto{\pgfqpoint{0.000000in}{-0.027778in}}%
\pgfusepath{stroke,fill}%
}%
\begin{pgfscope}%
\pgfsys@transformshift{3.409957in}{0.402778in}%
\pgfsys@useobject{currentmarker}{}%
\end{pgfscope}%
\end{pgfscope}%
\begin{pgfscope}%
\pgfsetbuttcap%
\pgfsetroundjoin%
\definecolor{currentfill}{rgb}{0.000000,0.000000,0.000000}%
\pgfsetfillcolor{currentfill}%
\pgfsetlinewidth{0.602250pt}%
\definecolor{currentstroke}{rgb}{0.000000,0.000000,0.000000}%
\pgfsetstrokecolor{currentstroke}%
\pgfsetdash{}{0pt}%
\pgfsys@defobject{currentmarker}{\pgfqpoint{0.000000in}{-0.027778in}}{\pgfqpoint{0.000000in}{0.000000in}}{%
\pgfpathmoveto{\pgfqpoint{0.000000in}{0.000000in}}%
\pgfpathlineto{\pgfqpoint{0.000000in}{-0.027778in}}%
\pgfusepath{stroke,fill}%
}%
\begin{pgfscope}%
\pgfsys@transformshift{3.494193in}{0.402778in}%
\pgfsys@useobject{currentmarker}{}%
\end{pgfscope}%
\end{pgfscope}%
\begin{pgfscope}%
\pgfsetbuttcap%
\pgfsetroundjoin%
\definecolor{currentfill}{rgb}{0.000000,0.000000,0.000000}%
\pgfsetfillcolor{currentfill}%
\pgfsetlinewidth{0.602250pt}%
\definecolor{currentstroke}{rgb}{0.000000,0.000000,0.000000}%
\pgfsetstrokecolor{currentstroke}%
\pgfsetdash{}{0pt}%
\pgfsys@defobject{currentmarker}{\pgfqpoint{0.000000in}{-0.027778in}}{\pgfqpoint{0.000000in}{0.000000in}}{%
\pgfpathmoveto{\pgfqpoint{0.000000in}{0.000000in}}%
\pgfpathlineto{\pgfqpoint{0.000000in}{-0.027778in}}%
\pgfusepath{stroke,fill}%
}%
\begin{pgfscope}%
\pgfsys@transformshift{3.553959in}{0.402778in}%
\pgfsys@useobject{currentmarker}{}%
\end{pgfscope}%
\end{pgfscope}%
\begin{pgfscope}%
\pgfsetbuttcap%
\pgfsetroundjoin%
\definecolor{currentfill}{rgb}{0.000000,0.000000,0.000000}%
\pgfsetfillcolor{currentfill}%
\pgfsetlinewidth{0.602250pt}%
\definecolor{currentstroke}{rgb}{0.000000,0.000000,0.000000}%
\pgfsetstrokecolor{currentstroke}%
\pgfsetdash{}{0pt}%
\pgfsys@defobject{currentmarker}{\pgfqpoint{0.000000in}{-0.027778in}}{\pgfqpoint{0.000000in}{0.000000in}}{%
\pgfpathmoveto{\pgfqpoint{0.000000in}{0.000000in}}%
\pgfpathlineto{\pgfqpoint{0.000000in}{-0.027778in}}%
\pgfusepath{stroke,fill}%
}%
\begin{pgfscope}%
\pgfsys@transformshift{3.600317in}{0.402778in}%
\pgfsys@useobject{currentmarker}{}%
\end{pgfscope}%
\end{pgfscope}%
\begin{pgfscope}%
\pgfsetbuttcap%
\pgfsetroundjoin%
\definecolor{currentfill}{rgb}{0.000000,0.000000,0.000000}%
\pgfsetfillcolor{currentfill}%
\pgfsetlinewidth{0.602250pt}%
\definecolor{currentstroke}{rgb}{0.000000,0.000000,0.000000}%
\pgfsetstrokecolor{currentstroke}%
\pgfsetdash{}{0pt}%
\pgfsys@defobject{currentmarker}{\pgfqpoint{0.000000in}{-0.027778in}}{\pgfqpoint{0.000000in}{0.000000in}}{%
\pgfpathmoveto{\pgfqpoint{0.000000in}{0.000000in}}%
\pgfpathlineto{\pgfqpoint{0.000000in}{-0.027778in}}%
\pgfusepath{stroke,fill}%
}%
\begin{pgfscope}%
\pgfsys@transformshift{3.638194in}{0.402778in}%
\pgfsys@useobject{currentmarker}{}%
\end{pgfscope}%
\end{pgfscope}%
\begin{pgfscope}%
\pgfsetbuttcap%
\pgfsetroundjoin%
\definecolor{currentfill}{rgb}{0.000000,0.000000,0.000000}%
\pgfsetfillcolor{currentfill}%
\pgfsetlinewidth{0.602250pt}%
\definecolor{currentstroke}{rgb}{0.000000,0.000000,0.000000}%
\pgfsetstrokecolor{currentstroke}%
\pgfsetdash{}{0pt}%
\pgfsys@defobject{currentmarker}{\pgfqpoint{0.000000in}{-0.027778in}}{\pgfqpoint{0.000000in}{0.000000in}}{%
\pgfpathmoveto{\pgfqpoint{0.000000in}{0.000000in}}%
\pgfpathlineto{\pgfqpoint{0.000000in}{-0.027778in}}%
\pgfusepath{stroke,fill}%
}%
\begin{pgfscope}%
\pgfsys@transformshift{3.670219in}{0.402778in}%
\pgfsys@useobject{currentmarker}{}%
\end{pgfscope}%
\end{pgfscope}%
\begin{pgfscope}%
\pgfsetbuttcap%
\pgfsetroundjoin%
\definecolor{currentfill}{rgb}{0.000000,0.000000,0.000000}%
\pgfsetfillcolor{currentfill}%
\pgfsetlinewidth{0.602250pt}%
\definecolor{currentstroke}{rgb}{0.000000,0.000000,0.000000}%
\pgfsetstrokecolor{currentstroke}%
\pgfsetdash{}{0pt}%
\pgfsys@defobject{currentmarker}{\pgfqpoint{0.000000in}{-0.027778in}}{\pgfqpoint{0.000000in}{0.000000in}}{%
\pgfpathmoveto{\pgfqpoint{0.000000in}{0.000000in}}%
\pgfpathlineto{\pgfqpoint{0.000000in}{-0.027778in}}%
\pgfusepath{stroke,fill}%
}%
\begin{pgfscope}%
\pgfsys@transformshift{3.697961in}{0.402778in}%
\pgfsys@useobject{currentmarker}{}%
\end{pgfscope}%
\end{pgfscope}%
\begin{pgfscope}%
\pgfsetbuttcap%
\pgfsetroundjoin%
\definecolor{currentfill}{rgb}{0.000000,0.000000,0.000000}%
\pgfsetfillcolor{currentfill}%
\pgfsetlinewidth{0.602250pt}%
\definecolor{currentstroke}{rgb}{0.000000,0.000000,0.000000}%
\pgfsetstrokecolor{currentstroke}%
\pgfsetdash{}{0pt}%
\pgfsys@defobject{currentmarker}{\pgfqpoint{0.000000in}{-0.027778in}}{\pgfqpoint{0.000000in}{0.000000in}}{%
\pgfpathmoveto{\pgfqpoint{0.000000in}{0.000000in}}%
\pgfpathlineto{\pgfqpoint{0.000000in}{-0.027778in}}%
\pgfusepath{stroke,fill}%
}%
\begin{pgfscope}%
\pgfsys@transformshift{3.722430in}{0.402778in}%
\pgfsys@useobject{currentmarker}{}%
\end{pgfscope}%
\end{pgfscope}%
\begin{pgfscope}%
\pgfsetbuttcap%
\pgfsetroundjoin%
\definecolor{currentfill}{rgb}{0.000000,0.000000,0.000000}%
\pgfsetfillcolor{currentfill}%
\pgfsetlinewidth{0.602250pt}%
\definecolor{currentstroke}{rgb}{0.000000,0.000000,0.000000}%
\pgfsetstrokecolor{currentstroke}%
\pgfsetdash{}{0pt}%
\pgfsys@defobject{currentmarker}{\pgfqpoint{0.000000in}{-0.027778in}}{\pgfqpoint{0.000000in}{0.000000in}}{%
\pgfpathmoveto{\pgfqpoint{0.000000in}{0.000000in}}%
\pgfpathlineto{\pgfqpoint{0.000000in}{-0.027778in}}%
\pgfusepath{stroke,fill}%
}%
\begin{pgfscope}%
\pgfsys@transformshift{3.888321in}{0.402778in}%
\pgfsys@useobject{currentmarker}{}%
\end{pgfscope}%
\end{pgfscope}%
\begin{pgfscope}%
\pgfsetbuttcap%
\pgfsetroundjoin%
\definecolor{currentfill}{rgb}{0.000000,0.000000,0.000000}%
\pgfsetfillcolor{currentfill}%
\pgfsetlinewidth{0.602250pt}%
\definecolor{currentstroke}{rgb}{0.000000,0.000000,0.000000}%
\pgfsetstrokecolor{currentstroke}%
\pgfsetdash{}{0pt}%
\pgfsys@defobject{currentmarker}{\pgfqpoint{0.000000in}{-0.027778in}}{\pgfqpoint{0.000000in}{0.000000in}}{%
\pgfpathmoveto{\pgfqpoint{0.000000in}{0.000000in}}%
\pgfpathlineto{\pgfqpoint{0.000000in}{-0.027778in}}%
\pgfusepath{stroke,fill}%
}%
\begin{pgfscope}%
\pgfsys@transformshift{3.972556in}{0.402778in}%
\pgfsys@useobject{currentmarker}{}%
\end{pgfscope}%
\end{pgfscope}%
\begin{pgfscope}%
\pgfsetbuttcap%
\pgfsetroundjoin%
\definecolor{currentfill}{rgb}{0.000000,0.000000,0.000000}%
\pgfsetfillcolor{currentfill}%
\pgfsetlinewidth{0.602250pt}%
\definecolor{currentstroke}{rgb}{0.000000,0.000000,0.000000}%
\pgfsetstrokecolor{currentstroke}%
\pgfsetdash{}{0pt}%
\pgfsys@defobject{currentmarker}{\pgfqpoint{0.000000in}{-0.027778in}}{\pgfqpoint{0.000000in}{0.000000in}}{%
\pgfpathmoveto{\pgfqpoint{0.000000in}{0.000000in}}%
\pgfpathlineto{\pgfqpoint{0.000000in}{-0.027778in}}%
\pgfusepath{stroke,fill}%
}%
\begin{pgfscope}%
\pgfsys@transformshift{4.032323in}{0.402778in}%
\pgfsys@useobject{currentmarker}{}%
\end{pgfscope}%
\end{pgfscope}%
\begin{pgfscope}%
\pgfsetbuttcap%
\pgfsetroundjoin%
\definecolor{currentfill}{rgb}{0.000000,0.000000,0.000000}%
\pgfsetfillcolor{currentfill}%
\pgfsetlinewidth{0.602250pt}%
\definecolor{currentstroke}{rgb}{0.000000,0.000000,0.000000}%
\pgfsetstrokecolor{currentstroke}%
\pgfsetdash{}{0pt}%
\pgfsys@defobject{currentmarker}{\pgfqpoint{0.000000in}{-0.027778in}}{\pgfqpoint{0.000000in}{0.000000in}}{%
\pgfpathmoveto{\pgfqpoint{0.000000in}{0.000000in}}%
\pgfpathlineto{\pgfqpoint{0.000000in}{-0.027778in}}%
\pgfusepath{stroke,fill}%
}%
\begin{pgfscope}%
\pgfsys@transformshift{4.078681in}{0.402778in}%
\pgfsys@useobject{currentmarker}{}%
\end{pgfscope}%
\end{pgfscope}%
\begin{pgfscope}%
\pgfsetbuttcap%
\pgfsetroundjoin%
\definecolor{currentfill}{rgb}{0.000000,0.000000,0.000000}%
\pgfsetfillcolor{currentfill}%
\pgfsetlinewidth{0.602250pt}%
\definecolor{currentstroke}{rgb}{0.000000,0.000000,0.000000}%
\pgfsetstrokecolor{currentstroke}%
\pgfsetdash{}{0pt}%
\pgfsys@defobject{currentmarker}{\pgfqpoint{0.000000in}{-0.027778in}}{\pgfqpoint{0.000000in}{0.000000in}}{%
\pgfpathmoveto{\pgfqpoint{0.000000in}{0.000000in}}%
\pgfpathlineto{\pgfqpoint{0.000000in}{-0.027778in}}%
\pgfusepath{stroke,fill}%
}%
\begin{pgfscope}%
\pgfsys@transformshift{4.116558in}{0.402778in}%
\pgfsys@useobject{currentmarker}{}%
\end{pgfscope}%
\end{pgfscope}%
\begin{pgfscope}%
\pgfsetbuttcap%
\pgfsetroundjoin%
\definecolor{currentfill}{rgb}{0.000000,0.000000,0.000000}%
\pgfsetfillcolor{currentfill}%
\pgfsetlinewidth{0.602250pt}%
\definecolor{currentstroke}{rgb}{0.000000,0.000000,0.000000}%
\pgfsetstrokecolor{currentstroke}%
\pgfsetdash{}{0pt}%
\pgfsys@defobject{currentmarker}{\pgfqpoint{0.000000in}{-0.027778in}}{\pgfqpoint{0.000000in}{0.000000in}}{%
\pgfpathmoveto{\pgfqpoint{0.000000in}{0.000000in}}%
\pgfpathlineto{\pgfqpoint{0.000000in}{-0.027778in}}%
\pgfusepath{stroke,fill}%
}%
\begin{pgfscope}%
\pgfsys@transformshift{4.148583in}{0.402778in}%
\pgfsys@useobject{currentmarker}{}%
\end{pgfscope}%
\end{pgfscope}%
\begin{pgfscope}%
\pgfsetbuttcap%
\pgfsetroundjoin%
\definecolor{currentfill}{rgb}{0.000000,0.000000,0.000000}%
\pgfsetfillcolor{currentfill}%
\pgfsetlinewidth{0.602250pt}%
\definecolor{currentstroke}{rgb}{0.000000,0.000000,0.000000}%
\pgfsetstrokecolor{currentstroke}%
\pgfsetdash{}{0pt}%
\pgfsys@defobject{currentmarker}{\pgfqpoint{0.000000in}{-0.027778in}}{\pgfqpoint{0.000000in}{0.000000in}}{%
\pgfpathmoveto{\pgfqpoint{0.000000in}{0.000000in}}%
\pgfpathlineto{\pgfqpoint{0.000000in}{-0.027778in}}%
\pgfusepath{stroke,fill}%
}%
\begin{pgfscope}%
\pgfsys@transformshift{4.176324in}{0.402778in}%
\pgfsys@useobject{currentmarker}{}%
\end{pgfscope}%
\end{pgfscope}%
\begin{pgfscope}%
\pgfsetbuttcap%
\pgfsetroundjoin%
\definecolor{currentfill}{rgb}{0.000000,0.000000,0.000000}%
\pgfsetfillcolor{currentfill}%
\pgfsetlinewidth{0.602250pt}%
\definecolor{currentstroke}{rgb}{0.000000,0.000000,0.000000}%
\pgfsetstrokecolor{currentstroke}%
\pgfsetdash{}{0pt}%
\pgfsys@defobject{currentmarker}{\pgfqpoint{0.000000in}{-0.027778in}}{\pgfqpoint{0.000000in}{0.000000in}}{%
\pgfpathmoveto{\pgfqpoint{0.000000in}{0.000000in}}%
\pgfpathlineto{\pgfqpoint{0.000000in}{-0.027778in}}%
\pgfusepath{stroke,fill}%
}%
\begin{pgfscope}%
\pgfsys@transformshift{4.200794in}{0.402778in}%
\pgfsys@useobject{currentmarker}{}%
\end{pgfscope}%
\end{pgfscope}%
\begin{pgfscope}%
\pgfsetbuttcap%
\pgfsetroundjoin%
\definecolor{currentfill}{rgb}{0.000000,0.000000,0.000000}%
\pgfsetfillcolor{currentfill}%
\pgfsetlinewidth{0.602250pt}%
\definecolor{currentstroke}{rgb}{0.000000,0.000000,0.000000}%
\pgfsetstrokecolor{currentstroke}%
\pgfsetdash{}{0pt}%
\pgfsys@defobject{currentmarker}{\pgfqpoint{0.000000in}{-0.027778in}}{\pgfqpoint{0.000000in}{0.000000in}}{%
\pgfpathmoveto{\pgfqpoint{0.000000in}{0.000000in}}%
\pgfpathlineto{\pgfqpoint{0.000000in}{-0.027778in}}%
\pgfusepath{stroke,fill}%
}%
\begin{pgfscope}%
\pgfsys@transformshift{4.366685in}{0.402778in}%
\pgfsys@useobject{currentmarker}{}%
\end{pgfscope}%
\end{pgfscope}%
\begin{pgfscope}%
\pgfsetbuttcap%
\pgfsetroundjoin%
\definecolor{currentfill}{rgb}{0.000000,0.000000,0.000000}%
\pgfsetfillcolor{currentfill}%
\pgfsetlinewidth{0.602250pt}%
\definecolor{currentstroke}{rgb}{0.000000,0.000000,0.000000}%
\pgfsetstrokecolor{currentstroke}%
\pgfsetdash{}{0pt}%
\pgfsys@defobject{currentmarker}{\pgfqpoint{0.000000in}{-0.027778in}}{\pgfqpoint{0.000000in}{0.000000in}}{%
\pgfpathmoveto{\pgfqpoint{0.000000in}{0.000000in}}%
\pgfpathlineto{\pgfqpoint{0.000000in}{-0.027778in}}%
\pgfusepath{stroke,fill}%
}%
\begin{pgfscope}%
\pgfsys@transformshift{4.450920in}{0.402778in}%
\pgfsys@useobject{currentmarker}{}%
\end{pgfscope}%
\end{pgfscope}%
\begin{pgfscope}%
\pgfsetbuttcap%
\pgfsetroundjoin%
\definecolor{currentfill}{rgb}{0.000000,0.000000,0.000000}%
\pgfsetfillcolor{currentfill}%
\pgfsetlinewidth{0.602250pt}%
\definecolor{currentstroke}{rgb}{0.000000,0.000000,0.000000}%
\pgfsetstrokecolor{currentstroke}%
\pgfsetdash{}{0pt}%
\pgfsys@defobject{currentmarker}{\pgfqpoint{0.000000in}{-0.027778in}}{\pgfqpoint{0.000000in}{0.000000in}}{%
\pgfpathmoveto{\pgfqpoint{0.000000in}{0.000000in}}%
\pgfpathlineto{\pgfqpoint{0.000000in}{-0.027778in}}%
\pgfusepath{stroke,fill}%
}%
\begin{pgfscope}%
\pgfsys@transformshift{4.510686in}{0.402778in}%
\pgfsys@useobject{currentmarker}{}%
\end{pgfscope}%
\end{pgfscope}%
\begin{pgfscope}%
\pgfsetbuttcap%
\pgfsetroundjoin%
\definecolor{currentfill}{rgb}{0.000000,0.000000,0.000000}%
\pgfsetfillcolor{currentfill}%
\pgfsetlinewidth{0.602250pt}%
\definecolor{currentstroke}{rgb}{0.000000,0.000000,0.000000}%
\pgfsetstrokecolor{currentstroke}%
\pgfsetdash{}{0pt}%
\pgfsys@defobject{currentmarker}{\pgfqpoint{0.000000in}{-0.027778in}}{\pgfqpoint{0.000000in}{0.000000in}}{%
\pgfpathmoveto{\pgfqpoint{0.000000in}{0.000000in}}%
\pgfpathlineto{\pgfqpoint{0.000000in}{-0.027778in}}%
\pgfusepath{stroke,fill}%
}%
\begin{pgfscope}%
\pgfsys@transformshift{4.557045in}{0.402778in}%
\pgfsys@useobject{currentmarker}{}%
\end{pgfscope}%
\end{pgfscope}%
\begin{pgfscope}%
\pgfsetbuttcap%
\pgfsetroundjoin%
\definecolor{currentfill}{rgb}{0.000000,0.000000,0.000000}%
\pgfsetfillcolor{currentfill}%
\pgfsetlinewidth{0.602250pt}%
\definecolor{currentstroke}{rgb}{0.000000,0.000000,0.000000}%
\pgfsetstrokecolor{currentstroke}%
\pgfsetdash{}{0pt}%
\pgfsys@defobject{currentmarker}{\pgfqpoint{0.000000in}{-0.027778in}}{\pgfqpoint{0.000000in}{0.000000in}}{%
\pgfpathmoveto{\pgfqpoint{0.000000in}{0.000000in}}%
\pgfpathlineto{\pgfqpoint{0.000000in}{-0.027778in}}%
\pgfusepath{stroke,fill}%
}%
\begin{pgfscope}%
\pgfsys@transformshift{4.594922in}{0.402778in}%
\pgfsys@useobject{currentmarker}{}%
\end{pgfscope}%
\end{pgfscope}%
\begin{pgfscope}%
\pgfsetbuttcap%
\pgfsetroundjoin%
\definecolor{currentfill}{rgb}{0.000000,0.000000,0.000000}%
\pgfsetfillcolor{currentfill}%
\pgfsetlinewidth{0.602250pt}%
\definecolor{currentstroke}{rgb}{0.000000,0.000000,0.000000}%
\pgfsetstrokecolor{currentstroke}%
\pgfsetdash{}{0pt}%
\pgfsys@defobject{currentmarker}{\pgfqpoint{0.000000in}{-0.027778in}}{\pgfqpoint{0.000000in}{0.000000in}}{%
\pgfpathmoveto{\pgfqpoint{0.000000in}{0.000000in}}%
\pgfpathlineto{\pgfqpoint{0.000000in}{-0.027778in}}%
\pgfusepath{stroke,fill}%
}%
\begin{pgfscope}%
\pgfsys@transformshift{4.626947in}{0.402778in}%
\pgfsys@useobject{currentmarker}{}%
\end{pgfscope}%
\end{pgfscope}%
\begin{pgfscope}%
\pgfsetbuttcap%
\pgfsetroundjoin%
\definecolor{currentfill}{rgb}{0.000000,0.000000,0.000000}%
\pgfsetfillcolor{currentfill}%
\pgfsetlinewidth{0.602250pt}%
\definecolor{currentstroke}{rgb}{0.000000,0.000000,0.000000}%
\pgfsetstrokecolor{currentstroke}%
\pgfsetdash{}{0pt}%
\pgfsys@defobject{currentmarker}{\pgfqpoint{0.000000in}{-0.027778in}}{\pgfqpoint{0.000000in}{0.000000in}}{%
\pgfpathmoveto{\pgfqpoint{0.000000in}{0.000000in}}%
\pgfpathlineto{\pgfqpoint{0.000000in}{-0.027778in}}%
\pgfusepath{stroke,fill}%
}%
\begin{pgfscope}%
\pgfsys@transformshift{4.654688in}{0.402778in}%
\pgfsys@useobject{currentmarker}{}%
\end{pgfscope}%
\end{pgfscope}%
\begin{pgfscope}%
\pgfsetbuttcap%
\pgfsetroundjoin%
\definecolor{currentfill}{rgb}{0.000000,0.000000,0.000000}%
\pgfsetfillcolor{currentfill}%
\pgfsetlinewidth{0.602250pt}%
\definecolor{currentstroke}{rgb}{0.000000,0.000000,0.000000}%
\pgfsetstrokecolor{currentstroke}%
\pgfsetdash{}{0pt}%
\pgfsys@defobject{currentmarker}{\pgfqpoint{0.000000in}{-0.027778in}}{\pgfqpoint{0.000000in}{0.000000in}}{%
\pgfpathmoveto{\pgfqpoint{0.000000in}{0.000000in}}%
\pgfpathlineto{\pgfqpoint{0.000000in}{-0.027778in}}%
\pgfusepath{stroke,fill}%
}%
\begin{pgfscope}%
\pgfsys@transformshift{4.679158in}{0.402778in}%
\pgfsys@useobject{currentmarker}{}%
\end{pgfscope}%
\end{pgfscope}%
\begin{pgfscope}%
\pgfsetbuttcap%
\pgfsetroundjoin%
\definecolor{currentfill}{rgb}{0.000000,0.000000,0.000000}%
\pgfsetfillcolor{currentfill}%
\pgfsetlinewidth{0.602250pt}%
\definecolor{currentstroke}{rgb}{0.000000,0.000000,0.000000}%
\pgfsetstrokecolor{currentstroke}%
\pgfsetdash{}{0pt}%
\pgfsys@defobject{currentmarker}{\pgfqpoint{0.000000in}{-0.027778in}}{\pgfqpoint{0.000000in}{0.000000in}}{%
\pgfpathmoveto{\pgfqpoint{0.000000in}{0.000000in}}%
\pgfpathlineto{\pgfqpoint{0.000000in}{-0.027778in}}%
\pgfusepath{stroke,fill}%
}%
\begin{pgfscope}%
\pgfsys@transformshift{4.845048in}{0.402778in}%
\pgfsys@useobject{currentmarker}{}%
\end{pgfscope}%
\end{pgfscope}%
\begin{pgfscope}%
\pgfsetbuttcap%
\pgfsetroundjoin%
\definecolor{currentfill}{rgb}{0.000000,0.000000,0.000000}%
\pgfsetfillcolor{currentfill}%
\pgfsetlinewidth{0.602250pt}%
\definecolor{currentstroke}{rgb}{0.000000,0.000000,0.000000}%
\pgfsetstrokecolor{currentstroke}%
\pgfsetdash{}{0pt}%
\pgfsys@defobject{currentmarker}{\pgfqpoint{0.000000in}{-0.027778in}}{\pgfqpoint{0.000000in}{0.000000in}}{%
\pgfpathmoveto{\pgfqpoint{0.000000in}{0.000000in}}%
\pgfpathlineto{\pgfqpoint{0.000000in}{-0.027778in}}%
\pgfusepath{stroke,fill}%
}%
\begin{pgfscope}%
\pgfsys@transformshift{4.929284in}{0.402778in}%
\pgfsys@useobject{currentmarker}{}%
\end{pgfscope}%
\end{pgfscope}%
\begin{pgfscope}%
\pgftext[x=2.754587in,y=0.123861in,,top]{\rmfamily\fontsize{10.000000}{12.000000}\selectfont $E(k)$}%
\end{pgfscope}%
\begin{pgfscope}%
\pgfsetbuttcap%
\pgfsetroundjoin%
\definecolor{currentfill}{rgb}{0.000000,0.000000,0.000000}%
\pgfsetfillcolor{currentfill}%
\pgfsetlinewidth{0.803000pt}%
\definecolor{currentstroke}{rgb}{0.000000,0.000000,0.000000}%
\pgfsetstrokecolor{currentstroke}%
\pgfsetdash{}{0pt}%
\pgfsys@defobject{currentmarker}{\pgfqpoint{-0.048611in}{0.000000in}}{\pgfqpoint{0.000000in}{0.000000in}}{%
\pgfpathmoveto{\pgfqpoint{0.000000in}{0.000000in}}%
\pgfpathlineto{\pgfqpoint{-0.048611in}{0.000000in}}%
\pgfusepath{stroke,fill}%
}%
\begin{pgfscope}%
\pgfsys@transformshift{0.557786in}{1.264793in}%
\pgfsys@useobject{currentmarker}{}%
\end{pgfscope}%
\end{pgfscope}%
\begin{pgfscope}%
\pgftext[x=0.172561in,y=1.216576in,left,base]{\rmfamily\fontsize{10.000000}{12.000000}\selectfont \(\displaystyle 10^{-4}\)}%
\end{pgfscope}%
\begin{pgfscope}%
\pgfsetbuttcap%
\pgfsetroundjoin%
\definecolor{currentfill}{rgb}{0.000000,0.000000,0.000000}%
\pgfsetfillcolor{currentfill}%
\pgfsetlinewidth{0.803000pt}%
\definecolor{currentstroke}{rgb}{0.000000,0.000000,0.000000}%
\pgfsetstrokecolor{currentstroke}%
\pgfsetdash{}{0pt}%
\pgfsys@defobject{currentmarker}{\pgfqpoint{-0.048611in}{0.000000in}}{\pgfqpoint{0.000000in}{0.000000in}}{%
\pgfpathmoveto{\pgfqpoint{0.000000in}{0.000000in}}%
\pgfpathlineto{\pgfqpoint{-0.048611in}{0.000000in}}%
\pgfusepath{stroke,fill}%
}%
\begin{pgfscope}%
\pgfsys@transformshift{0.557786in}{2.327492in}%
\pgfsys@useobject{currentmarker}{}%
\end{pgfscope}%
\end{pgfscope}%
\begin{pgfscope}%
\pgftext[x=0.172561in,y=2.279275in,left,base]{\rmfamily\fontsize{10.000000}{12.000000}\selectfont \(\displaystyle 10^{-3}\)}%
\end{pgfscope}%
\begin{pgfscope}%
\pgfsetbuttcap%
\pgfsetroundjoin%
\definecolor{currentfill}{rgb}{0.000000,0.000000,0.000000}%
\pgfsetfillcolor{currentfill}%
\pgfsetlinewidth{0.602250pt}%
\definecolor{currentstroke}{rgb}{0.000000,0.000000,0.000000}%
\pgfsetstrokecolor{currentstroke}%
\pgfsetdash{}{0pt}%
\pgfsys@defobject{currentmarker}{\pgfqpoint{-0.027778in}{0.000000in}}{\pgfqpoint{0.000000in}{0.000000in}}{%
\pgfpathmoveto{\pgfqpoint{0.000000in}{0.000000in}}%
\pgfpathlineto{\pgfqpoint{-0.027778in}{0.000000in}}%
\pgfusepath{stroke,fill}%
}%
\begin{pgfscope}%
\pgfsys@transformshift{0.557786in}{0.521999in}%
\pgfsys@useobject{currentmarker}{}%
\end{pgfscope}%
\end{pgfscope}%
\begin{pgfscope}%
\pgfsetbuttcap%
\pgfsetroundjoin%
\definecolor{currentfill}{rgb}{0.000000,0.000000,0.000000}%
\pgfsetfillcolor{currentfill}%
\pgfsetlinewidth{0.602250pt}%
\definecolor{currentstroke}{rgb}{0.000000,0.000000,0.000000}%
\pgfsetstrokecolor{currentstroke}%
\pgfsetdash{}{0pt}%
\pgfsys@defobject{currentmarker}{\pgfqpoint{-0.027778in}{0.000000in}}{\pgfqpoint{0.000000in}{0.000000in}}{%
\pgfpathmoveto{\pgfqpoint{0.000000in}{0.000000in}}%
\pgfpathlineto{\pgfqpoint{-0.027778in}{0.000000in}}%
\pgfusepath{stroke,fill}%
}%
\begin{pgfscope}%
\pgfsys@transformshift{0.557786in}{0.709131in}%
\pgfsys@useobject{currentmarker}{}%
\end{pgfscope}%
\end{pgfscope}%
\begin{pgfscope}%
\pgfsetbuttcap%
\pgfsetroundjoin%
\definecolor{currentfill}{rgb}{0.000000,0.000000,0.000000}%
\pgfsetfillcolor{currentfill}%
\pgfsetlinewidth{0.602250pt}%
\definecolor{currentstroke}{rgb}{0.000000,0.000000,0.000000}%
\pgfsetstrokecolor{currentstroke}%
\pgfsetdash{}{0pt}%
\pgfsys@defobject{currentmarker}{\pgfqpoint{-0.027778in}{0.000000in}}{\pgfqpoint{0.000000in}{0.000000in}}{%
\pgfpathmoveto{\pgfqpoint{0.000000in}{0.000000in}}%
\pgfpathlineto{\pgfqpoint{-0.027778in}{0.000000in}}%
\pgfusepath{stroke,fill}%
}%
\begin{pgfscope}%
\pgfsys@transformshift{0.557786in}{0.841903in}%
\pgfsys@useobject{currentmarker}{}%
\end{pgfscope}%
\end{pgfscope}%
\begin{pgfscope}%
\pgfsetbuttcap%
\pgfsetroundjoin%
\definecolor{currentfill}{rgb}{0.000000,0.000000,0.000000}%
\pgfsetfillcolor{currentfill}%
\pgfsetlinewidth{0.602250pt}%
\definecolor{currentstroke}{rgb}{0.000000,0.000000,0.000000}%
\pgfsetstrokecolor{currentstroke}%
\pgfsetdash{}{0pt}%
\pgfsys@defobject{currentmarker}{\pgfqpoint{-0.027778in}{0.000000in}}{\pgfqpoint{0.000000in}{0.000000in}}{%
\pgfpathmoveto{\pgfqpoint{0.000000in}{0.000000in}}%
\pgfpathlineto{\pgfqpoint{-0.027778in}{0.000000in}}%
\pgfusepath{stroke,fill}%
}%
\begin{pgfscope}%
\pgfsys@transformshift{0.557786in}{0.944889in}%
\pgfsys@useobject{currentmarker}{}%
\end{pgfscope}%
\end{pgfscope}%
\begin{pgfscope}%
\pgfsetbuttcap%
\pgfsetroundjoin%
\definecolor{currentfill}{rgb}{0.000000,0.000000,0.000000}%
\pgfsetfillcolor{currentfill}%
\pgfsetlinewidth{0.602250pt}%
\definecolor{currentstroke}{rgb}{0.000000,0.000000,0.000000}%
\pgfsetstrokecolor{currentstroke}%
\pgfsetdash{}{0pt}%
\pgfsys@defobject{currentmarker}{\pgfqpoint{-0.027778in}{0.000000in}}{\pgfqpoint{0.000000in}{0.000000in}}{%
\pgfpathmoveto{\pgfqpoint{0.000000in}{0.000000in}}%
\pgfpathlineto{\pgfqpoint{-0.027778in}{0.000000in}}%
\pgfusepath{stroke,fill}%
}%
\begin{pgfscope}%
\pgfsys@transformshift{0.557786in}{1.029035in}%
\pgfsys@useobject{currentmarker}{}%
\end{pgfscope}%
\end{pgfscope}%
\begin{pgfscope}%
\pgfsetbuttcap%
\pgfsetroundjoin%
\definecolor{currentfill}{rgb}{0.000000,0.000000,0.000000}%
\pgfsetfillcolor{currentfill}%
\pgfsetlinewidth{0.602250pt}%
\definecolor{currentstroke}{rgb}{0.000000,0.000000,0.000000}%
\pgfsetstrokecolor{currentstroke}%
\pgfsetdash{}{0pt}%
\pgfsys@defobject{currentmarker}{\pgfqpoint{-0.027778in}{0.000000in}}{\pgfqpoint{0.000000in}{0.000000in}}{%
\pgfpathmoveto{\pgfqpoint{0.000000in}{0.000000in}}%
\pgfpathlineto{\pgfqpoint{-0.027778in}{0.000000in}}%
\pgfusepath{stroke,fill}%
}%
\begin{pgfscope}%
\pgfsys@transformshift{0.557786in}{1.100179in}%
\pgfsys@useobject{currentmarker}{}%
\end{pgfscope}%
\end{pgfscope}%
\begin{pgfscope}%
\pgfsetbuttcap%
\pgfsetroundjoin%
\definecolor{currentfill}{rgb}{0.000000,0.000000,0.000000}%
\pgfsetfillcolor{currentfill}%
\pgfsetlinewidth{0.602250pt}%
\definecolor{currentstroke}{rgb}{0.000000,0.000000,0.000000}%
\pgfsetstrokecolor{currentstroke}%
\pgfsetdash{}{0pt}%
\pgfsys@defobject{currentmarker}{\pgfqpoint{-0.027778in}{0.000000in}}{\pgfqpoint{0.000000in}{0.000000in}}{%
\pgfpathmoveto{\pgfqpoint{0.000000in}{0.000000in}}%
\pgfpathlineto{\pgfqpoint{-0.027778in}{0.000000in}}%
\pgfusepath{stroke,fill}%
}%
\begin{pgfscope}%
\pgfsys@transformshift{0.557786in}{1.161807in}%
\pgfsys@useobject{currentmarker}{}%
\end{pgfscope}%
\end{pgfscope}%
\begin{pgfscope}%
\pgfsetbuttcap%
\pgfsetroundjoin%
\definecolor{currentfill}{rgb}{0.000000,0.000000,0.000000}%
\pgfsetfillcolor{currentfill}%
\pgfsetlinewidth{0.602250pt}%
\definecolor{currentstroke}{rgb}{0.000000,0.000000,0.000000}%
\pgfsetstrokecolor{currentstroke}%
\pgfsetdash{}{0pt}%
\pgfsys@defobject{currentmarker}{\pgfqpoint{-0.027778in}{0.000000in}}{\pgfqpoint{0.000000in}{0.000000in}}{%
\pgfpathmoveto{\pgfqpoint{0.000000in}{0.000000in}}%
\pgfpathlineto{\pgfqpoint{-0.027778in}{0.000000in}}%
\pgfusepath{stroke,fill}%
}%
\begin{pgfscope}%
\pgfsys@transformshift{0.557786in}{1.216167in}%
\pgfsys@useobject{currentmarker}{}%
\end{pgfscope}%
\end{pgfscope}%
\begin{pgfscope}%
\pgfsetbuttcap%
\pgfsetroundjoin%
\definecolor{currentfill}{rgb}{0.000000,0.000000,0.000000}%
\pgfsetfillcolor{currentfill}%
\pgfsetlinewidth{0.602250pt}%
\definecolor{currentstroke}{rgb}{0.000000,0.000000,0.000000}%
\pgfsetstrokecolor{currentstroke}%
\pgfsetdash{}{0pt}%
\pgfsys@defobject{currentmarker}{\pgfqpoint{-0.027778in}{0.000000in}}{\pgfqpoint{0.000000in}{0.000000in}}{%
\pgfpathmoveto{\pgfqpoint{0.000000in}{0.000000in}}%
\pgfpathlineto{\pgfqpoint{-0.027778in}{0.000000in}}%
\pgfusepath{stroke,fill}%
}%
\begin{pgfscope}%
\pgfsys@transformshift{0.557786in}{1.584698in}%
\pgfsys@useobject{currentmarker}{}%
\end{pgfscope}%
\end{pgfscope}%
\begin{pgfscope}%
\pgfsetbuttcap%
\pgfsetroundjoin%
\definecolor{currentfill}{rgb}{0.000000,0.000000,0.000000}%
\pgfsetfillcolor{currentfill}%
\pgfsetlinewidth{0.602250pt}%
\definecolor{currentstroke}{rgb}{0.000000,0.000000,0.000000}%
\pgfsetstrokecolor{currentstroke}%
\pgfsetdash{}{0pt}%
\pgfsys@defobject{currentmarker}{\pgfqpoint{-0.027778in}{0.000000in}}{\pgfqpoint{0.000000in}{0.000000in}}{%
\pgfpathmoveto{\pgfqpoint{0.000000in}{0.000000in}}%
\pgfpathlineto{\pgfqpoint{-0.027778in}{0.000000in}}%
\pgfusepath{stroke,fill}%
}%
\begin{pgfscope}%
\pgfsys@transformshift{0.557786in}{1.771830in}%
\pgfsys@useobject{currentmarker}{}%
\end{pgfscope}%
\end{pgfscope}%
\begin{pgfscope}%
\pgfsetbuttcap%
\pgfsetroundjoin%
\definecolor{currentfill}{rgb}{0.000000,0.000000,0.000000}%
\pgfsetfillcolor{currentfill}%
\pgfsetlinewidth{0.602250pt}%
\definecolor{currentstroke}{rgb}{0.000000,0.000000,0.000000}%
\pgfsetstrokecolor{currentstroke}%
\pgfsetdash{}{0pt}%
\pgfsys@defobject{currentmarker}{\pgfqpoint{-0.027778in}{0.000000in}}{\pgfqpoint{0.000000in}{0.000000in}}{%
\pgfpathmoveto{\pgfqpoint{0.000000in}{0.000000in}}%
\pgfpathlineto{\pgfqpoint{-0.027778in}{0.000000in}}%
\pgfusepath{stroke,fill}%
}%
\begin{pgfscope}%
\pgfsys@transformshift{0.557786in}{1.904602in}%
\pgfsys@useobject{currentmarker}{}%
\end{pgfscope}%
\end{pgfscope}%
\begin{pgfscope}%
\pgfsetbuttcap%
\pgfsetroundjoin%
\definecolor{currentfill}{rgb}{0.000000,0.000000,0.000000}%
\pgfsetfillcolor{currentfill}%
\pgfsetlinewidth{0.602250pt}%
\definecolor{currentstroke}{rgb}{0.000000,0.000000,0.000000}%
\pgfsetstrokecolor{currentstroke}%
\pgfsetdash{}{0pt}%
\pgfsys@defobject{currentmarker}{\pgfqpoint{-0.027778in}{0.000000in}}{\pgfqpoint{0.000000in}{0.000000in}}{%
\pgfpathmoveto{\pgfqpoint{0.000000in}{0.000000in}}%
\pgfpathlineto{\pgfqpoint{-0.027778in}{0.000000in}}%
\pgfusepath{stroke,fill}%
}%
\begin{pgfscope}%
\pgfsys@transformshift{0.557786in}{2.007588in}%
\pgfsys@useobject{currentmarker}{}%
\end{pgfscope}%
\end{pgfscope}%
\begin{pgfscope}%
\pgfsetbuttcap%
\pgfsetroundjoin%
\definecolor{currentfill}{rgb}{0.000000,0.000000,0.000000}%
\pgfsetfillcolor{currentfill}%
\pgfsetlinewidth{0.602250pt}%
\definecolor{currentstroke}{rgb}{0.000000,0.000000,0.000000}%
\pgfsetstrokecolor{currentstroke}%
\pgfsetdash{}{0pt}%
\pgfsys@defobject{currentmarker}{\pgfqpoint{-0.027778in}{0.000000in}}{\pgfqpoint{0.000000in}{0.000000in}}{%
\pgfpathmoveto{\pgfqpoint{0.000000in}{0.000000in}}%
\pgfpathlineto{\pgfqpoint{-0.027778in}{0.000000in}}%
\pgfusepath{stroke,fill}%
}%
\begin{pgfscope}%
\pgfsys@transformshift{0.557786in}{2.091734in}%
\pgfsys@useobject{currentmarker}{}%
\end{pgfscope}%
\end{pgfscope}%
\begin{pgfscope}%
\pgfsetbuttcap%
\pgfsetroundjoin%
\definecolor{currentfill}{rgb}{0.000000,0.000000,0.000000}%
\pgfsetfillcolor{currentfill}%
\pgfsetlinewidth{0.602250pt}%
\definecolor{currentstroke}{rgb}{0.000000,0.000000,0.000000}%
\pgfsetstrokecolor{currentstroke}%
\pgfsetdash{}{0pt}%
\pgfsys@defobject{currentmarker}{\pgfqpoint{-0.027778in}{0.000000in}}{\pgfqpoint{0.000000in}{0.000000in}}{%
\pgfpathmoveto{\pgfqpoint{0.000000in}{0.000000in}}%
\pgfpathlineto{\pgfqpoint{-0.027778in}{0.000000in}}%
\pgfusepath{stroke,fill}%
}%
\begin{pgfscope}%
\pgfsys@transformshift{0.557786in}{2.162878in}%
\pgfsys@useobject{currentmarker}{}%
\end{pgfscope}%
\end{pgfscope}%
\begin{pgfscope}%
\pgfsetbuttcap%
\pgfsetroundjoin%
\definecolor{currentfill}{rgb}{0.000000,0.000000,0.000000}%
\pgfsetfillcolor{currentfill}%
\pgfsetlinewidth{0.602250pt}%
\definecolor{currentstroke}{rgb}{0.000000,0.000000,0.000000}%
\pgfsetstrokecolor{currentstroke}%
\pgfsetdash{}{0pt}%
\pgfsys@defobject{currentmarker}{\pgfqpoint{-0.027778in}{0.000000in}}{\pgfqpoint{0.000000in}{0.000000in}}{%
\pgfpathmoveto{\pgfqpoint{0.000000in}{0.000000in}}%
\pgfpathlineto{\pgfqpoint{-0.027778in}{0.000000in}}%
\pgfusepath{stroke,fill}%
}%
\begin{pgfscope}%
\pgfsys@transformshift{0.557786in}{2.224506in}%
\pgfsys@useobject{currentmarker}{}%
\end{pgfscope}%
\end{pgfscope}%
\begin{pgfscope}%
\pgfsetbuttcap%
\pgfsetroundjoin%
\definecolor{currentfill}{rgb}{0.000000,0.000000,0.000000}%
\pgfsetfillcolor{currentfill}%
\pgfsetlinewidth{0.602250pt}%
\definecolor{currentstroke}{rgb}{0.000000,0.000000,0.000000}%
\pgfsetstrokecolor{currentstroke}%
\pgfsetdash{}{0pt}%
\pgfsys@defobject{currentmarker}{\pgfqpoint{-0.027778in}{0.000000in}}{\pgfqpoint{0.000000in}{0.000000in}}{%
\pgfpathmoveto{\pgfqpoint{0.000000in}{0.000000in}}%
\pgfpathlineto{\pgfqpoint{-0.027778in}{0.000000in}}%
\pgfusepath{stroke,fill}%
}%
\begin{pgfscope}%
\pgfsys@transformshift{0.557786in}{2.278866in}%
\pgfsys@useobject{currentmarker}{}%
\end{pgfscope}%
\end{pgfscope}%
\begin{pgfscope}%
\pgfsetbuttcap%
\pgfsetroundjoin%
\definecolor{currentfill}{rgb}{0.000000,0.000000,0.000000}%
\pgfsetfillcolor{currentfill}%
\pgfsetlinewidth{0.602250pt}%
\definecolor{currentstroke}{rgb}{0.000000,0.000000,0.000000}%
\pgfsetstrokecolor{currentstroke}%
\pgfsetdash{}{0pt}%
\pgfsys@defobject{currentmarker}{\pgfqpoint{-0.027778in}{0.000000in}}{\pgfqpoint{0.000000in}{0.000000in}}{%
\pgfpathmoveto{\pgfqpoint{0.000000in}{0.000000in}}%
\pgfpathlineto{\pgfqpoint{-0.027778in}{0.000000in}}%
\pgfusepath{stroke,fill}%
}%
\begin{pgfscope}%
\pgfsys@transformshift{0.557786in}{2.647397in}%
\pgfsys@useobject{currentmarker}{}%
\end{pgfscope}%
\end{pgfscope}%
\begin{pgfscope}%
\pgfsetbuttcap%
\pgfsetroundjoin%
\definecolor{currentfill}{rgb}{0.000000,0.000000,0.000000}%
\pgfsetfillcolor{currentfill}%
\pgfsetlinewidth{0.602250pt}%
\definecolor{currentstroke}{rgb}{0.000000,0.000000,0.000000}%
\pgfsetstrokecolor{currentstroke}%
\pgfsetdash{}{0pt}%
\pgfsys@defobject{currentmarker}{\pgfqpoint{-0.027778in}{0.000000in}}{\pgfqpoint{0.000000in}{0.000000in}}{%
\pgfpathmoveto{\pgfqpoint{0.000000in}{0.000000in}}%
\pgfpathlineto{\pgfqpoint{-0.027778in}{0.000000in}}%
\pgfusepath{stroke,fill}%
}%
\begin{pgfscope}%
\pgfsys@transformshift{0.557786in}{2.834529in}%
\pgfsys@useobject{currentmarker}{}%
\end{pgfscope}%
\end{pgfscope}%
\begin{pgfscope}%
\pgfsetbuttcap%
\pgfsetroundjoin%
\definecolor{currentfill}{rgb}{0.000000,0.000000,0.000000}%
\pgfsetfillcolor{currentfill}%
\pgfsetlinewidth{0.602250pt}%
\definecolor{currentstroke}{rgb}{0.000000,0.000000,0.000000}%
\pgfsetstrokecolor{currentstroke}%
\pgfsetdash{}{0pt}%
\pgfsys@defobject{currentmarker}{\pgfqpoint{-0.027778in}{0.000000in}}{\pgfqpoint{0.000000in}{0.000000in}}{%
\pgfpathmoveto{\pgfqpoint{0.000000in}{0.000000in}}%
\pgfpathlineto{\pgfqpoint{-0.027778in}{0.000000in}}%
\pgfusepath{stroke,fill}%
}%
\begin{pgfscope}%
\pgfsys@transformshift{0.557786in}{2.967301in}%
\pgfsys@useobject{currentmarker}{}%
\end{pgfscope}%
\end{pgfscope}%
\begin{pgfscope}%
\pgftext[x=0.117005in,y=1.718750in,,bottom,rotate=90.000000]{\rmfamily\fontsize{10.000000}{12.000000}\selectfont Max Sprungstelle}%
\end{pgfscope}%
\begin{pgfscope}%
\pgfsetrectcap%
\pgfsetmiterjoin%
\pgfsetlinewidth{0.803000pt}%
\definecolor{currentstroke}{rgb}{0.000000,0.000000,0.000000}%
\pgfsetstrokecolor{currentstroke}%
\pgfsetdash{}{0pt}%
\pgfpathmoveto{\pgfqpoint{0.557786in}{0.402778in}}%
\pgfpathlineto{\pgfqpoint{0.557786in}{3.034722in}}%
\pgfusepath{stroke}%
\end{pgfscope}%
\begin{pgfscope}%
\pgfsetrectcap%
\pgfsetmiterjoin%
\pgfsetlinewidth{0.803000pt}%
\definecolor{currentstroke}{rgb}{0.000000,0.000000,0.000000}%
\pgfsetstrokecolor{currentstroke}%
\pgfsetdash{}{0pt}%
\pgfpathmoveto{\pgfqpoint{4.951389in}{0.402778in}}%
\pgfpathlineto{\pgfqpoint{4.951389in}{3.034722in}}%
\pgfusepath{stroke}%
\end{pgfscope}%
\begin{pgfscope}%
\pgfsetrectcap%
\pgfsetmiterjoin%
\pgfsetlinewidth{0.803000pt}%
\definecolor{currentstroke}{rgb}{0.000000,0.000000,0.000000}%
\pgfsetstrokecolor{currentstroke}%
\pgfsetdash{}{0pt}%
\pgfpathmoveto{\pgfqpoint{0.557786in}{0.402778in}}%
\pgfpathlineto{\pgfqpoint{4.951389in}{0.402778in}}%
\pgfusepath{stroke}%
\end{pgfscope}%
\begin{pgfscope}%
\pgfsetrectcap%
\pgfsetmiterjoin%
\pgfsetlinewidth{0.803000pt}%
\definecolor{currentstroke}{rgb}{0.000000,0.000000,0.000000}%
\pgfsetstrokecolor{currentstroke}%
\pgfsetdash{}{0pt}%
\pgfpathmoveto{\pgfqpoint{0.557786in}{3.034722in}}%
\pgfpathlineto{\pgfqpoint{4.951389in}{3.034722in}}%
\pgfusepath{stroke}%
\end{pgfscope}%
\begin{pgfscope}%
\pgfsetbuttcap%
\pgfsetmiterjoin%
\definecolor{currentfill}{rgb}{1.000000,1.000000,1.000000}%
\pgfsetfillcolor{currentfill}%
\pgfsetfillopacity{0.800000}%
\pgfsetlinewidth{1.003750pt}%
\definecolor{currentstroke}{rgb}{0.800000,0.800000,0.800000}%
\pgfsetstrokecolor{currentstroke}%
\pgfsetstrokeopacity{0.800000}%
\pgfsetdash{}{0pt}%
\pgfpathmoveto{\pgfqpoint{0.655008in}{2.334500in}}%
\pgfpathlineto{\pgfqpoint{1.516119in}{2.334500in}}%
\pgfpathquadraticcurveto{\pgfqpoint{1.543897in}{2.334500in}}{\pgfqpoint{1.543897in}{2.362278in}}%
\pgfpathlineto{\pgfqpoint{1.543897in}{2.937500in}}%
\pgfpathquadraticcurveto{\pgfqpoint{1.543897in}{2.965278in}}{\pgfqpoint{1.516119in}{2.965278in}}%
\pgfpathlineto{\pgfqpoint{0.655008in}{2.965278in}}%
\pgfpathquadraticcurveto{\pgfqpoint{0.627230in}{2.965278in}}{\pgfqpoint{0.627230in}{2.937500in}}%
\pgfpathlineto{\pgfqpoint{0.627230in}{2.362278in}}%
\pgfpathquadraticcurveto{\pgfqpoint{0.627230in}{2.334500in}}{\pgfqpoint{0.655008in}{2.334500in}}%
\pgfpathclose%
\pgfusepath{stroke,fill}%
\end{pgfscope}%
\begin{pgfscope}%
\pgfsetbuttcap%
\pgfsetroundjoin%
\definecolor{currentfill}{rgb}{0.121569,0.466667,0.705882}%
\pgfsetfillcolor{currentfill}%
\pgfsetfillopacity{0.500000}%
\pgfsetlinewidth{1.003750pt}%
\definecolor{currentstroke}{rgb}{0.121569,0.466667,0.705882}%
\pgfsetstrokecolor{currentstroke}%
\pgfsetstrokeopacity{0.500000}%
\pgfsetdash{}{0pt}%
\pgfpathmoveto{\pgfqpoint{0.821675in}{2.807292in}}%
\pgfpathcurveto{\pgfqpoint{0.832725in}{2.807292in}}{\pgfqpoint{0.843324in}{2.811682in}}{\pgfqpoint{0.851137in}{2.819496in}}%
\pgfpathcurveto{\pgfqpoint{0.858951in}{2.827309in}}{\pgfqpoint{0.863341in}{2.837908in}}{\pgfqpoint{0.863341in}{2.848958in}}%
\pgfpathcurveto{\pgfqpoint{0.863341in}{2.860008in}}{\pgfqpoint{0.858951in}{2.870607in}}{\pgfqpoint{0.851137in}{2.878421in}}%
\pgfpathcurveto{\pgfqpoint{0.843324in}{2.886235in}}{\pgfqpoint{0.832725in}{2.890625in}}{\pgfqpoint{0.821675in}{2.890625in}}%
\pgfpathcurveto{\pgfqpoint{0.810625in}{2.890625in}}{\pgfqpoint{0.800025in}{2.886235in}}{\pgfqpoint{0.792212in}{2.878421in}}%
\pgfpathcurveto{\pgfqpoint{0.784398in}{2.870607in}}{\pgfqpoint{0.780008in}{2.860008in}}{\pgfqpoint{0.780008in}{2.848958in}}%
\pgfpathcurveto{\pgfqpoint{0.780008in}{2.837908in}}{\pgfqpoint{0.784398in}{2.827309in}}{\pgfqpoint{0.792212in}{2.819496in}}%
\pgfpathcurveto{\pgfqpoint{0.800025in}{2.811682in}}{\pgfqpoint{0.810625in}{2.807292in}}{\pgfqpoint{0.821675in}{2.807292in}}%
\pgfpathclose%
\pgfusepath{stroke,fill}%
\end{pgfscope}%
\begin{pgfscope}%
\pgftext[x=1.071675in,y=2.812500in,left,base]{\rmfamily\fontsize{10.000000}{12.000000}\selectfont 16384}%
\end{pgfscope}%
\begin{pgfscope}%
\pgfsetbuttcap%
\pgfsetroundjoin%
\definecolor{currentfill}{rgb}{1.000000,0.498039,0.054902}%
\pgfsetfillcolor{currentfill}%
\pgfsetfillopacity{0.500000}%
\pgfsetlinewidth{1.003750pt}%
\definecolor{currentstroke}{rgb}{1.000000,0.498039,0.054902}%
\pgfsetstrokecolor{currentstroke}%
\pgfsetstrokeopacity{0.500000}%
\pgfsetdash{}{0pt}%
\pgfpathmoveto{\pgfqpoint{0.821675in}{2.610921in}}%
\pgfpathcurveto{\pgfqpoint{0.832725in}{2.610921in}}{\pgfqpoint{0.843324in}{2.615312in}}{\pgfqpoint{0.851137in}{2.623125in}}%
\pgfpathcurveto{\pgfqpoint{0.858951in}{2.630939in}}{\pgfqpoint{0.863341in}{2.641538in}}{\pgfqpoint{0.863341in}{2.652588in}}%
\pgfpathcurveto{\pgfqpoint{0.863341in}{2.663638in}}{\pgfqpoint{0.858951in}{2.674237in}}{\pgfqpoint{0.851137in}{2.682051in}}%
\pgfpathcurveto{\pgfqpoint{0.843324in}{2.689864in}}{\pgfqpoint{0.832725in}{2.694255in}}{\pgfqpoint{0.821675in}{2.694255in}}%
\pgfpathcurveto{\pgfqpoint{0.810625in}{2.694255in}}{\pgfqpoint{0.800025in}{2.689864in}}{\pgfqpoint{0.792212in}{2.682051in}}%
\pgfpathcurveto{\pgfqpoint{0.784398in}{2.674237in}}{\pgfqpoint{0.780008in}{2.663638in}}{\pgfqpoint{0.780008in}{2.652588in}}%
\pgfpathcurveto{\pgfqpoint{0.780008in}{2.641538in}}{\pgfqpoint{0.784398in}{2.630939in}}{\pgfqpoint{0.792212in}{2.623125in}}%
\pgfpathcurveto{\pgfqpoint{0.800025in}{2.615312in}}{\pgfqpoint{0.810625in}{2.610921in}}{\pgfqpoint{0.821675in}{2.610921in}}%
\pgfpathclose%
\pgfusepath{stroke,fill}%
\end{pgfscope}%
\begin{pgfscope}%
\pgftext[x=1.071675in,y=2.616130in,left,base]{\rmfamily\fontsize{10.000000}{12.000000}\selectfont 65536}%
\end{pgfscope}%
\begin{pgfscope}%
\pgfsetbuttcap%
\pgfsetroundjoin%
\definecolor{currentfill}{rgb}{0.172549,0.627451,0.172549}%
\pgfsetfillcolor{currentfill}%
\pgfsetfillopacity{0.500000}%
\pgfsetlinewidth{1.003750pt}%
\definecolor{currentstroke}{rgb}{0.172549,0.627451,0.172549}%
\pgfsetstrokecolor{currentstroke}%
\pgfsetstrokeopacity{0.500000}%
\pgfsetdash{}{0pt}%
\pgfpathmoveto{\pgfqpoint{0.821675in}{2.414551in}}%
\pgfpathcurveto{\pgfqpoint{0.832725in}{2.414551in}}{\pgfqpoint{0.843324in}{2.418941in}}{\pgfqpoint{0.851137in}{2.426755in}}%
\pgfpathcurveto{\pgfqpoint{0.858951in}{2.434569in}}{\pgfqpoint{0.863341in}{2.445168in}}{\pgfqpoint{0.863341in}{2.456218in}}%
\pgfpathcurveto{\pgfqpoint{0.863341in}{2.467268in}}{\pgfqpoint{0.858951in}{2.477867in}}{\pgfqpoint{0.851137in}{2.485681in}}%
\pgfpathcurveto{\pgfqpoint{0.843324in}{2.493494in}}{\pgfqpoint{0.832725in}{2.497884in}}{\pgfqpoint{0.821675in}{2.497884in}}%
\pgfpathcurveto{\pgfqpoint{0.810625in}{2.497884in}}{\pgfqpoint{0.800025in}{2.493494in}}{\pgfqpoint{0.792212in}{2.485681in}}%
\pgfpathcurveto{\pgfqpoint{0.784398in}{2.477867in}}{\pgfqpoint{0.780008in}{2.467268in}}{\pgfqpoint{0.780008in}{2.456218in}}%
\pgfpathcurveto{\pgfqpoint{0.780008in}{2.445168in}}{\pgfqpoint{0.784398in}{2.434569in}}{\pgfqpoint{0.792212in}{2.426755in}}%
\pgfpathcurveto{\pgfqpoint{0.800025in}{2.418941in}}{\pgfqpoint{0.810625in}{2.414551in}}{\pgfqpoint{0.821675in}{2.414551in}}%
\pgfpathclose%
\pgfusepath{stroke,fill}%
\end{pgfscope}%
\begin{pgfscope}%
\pgftext[x=1.071675in,y=2.419759in,left,base]{\rmfamily\fontsize{10.000000}{12.000000}\selectfont 262144}%
\end{pgfscope}%
\end{pgfpicture}%
\makeatother%
\endgroup%

    \caption{Feste Anzahl der Schritte mit verschieden Iterationszahlen; Vergleich von globalem Fehler und größter Sprungstelle}
    \label{fig:error_disc}
\end{figure}

