Der Kerngedanke des Parareal Algorithmus ist die Kombination eines groben Lösers \(\mathcal{G}\) mit einem feinen \(\mathcal{F}\) und wurde erstmals in~\cite{Lions:2001} beschrieben. \(\mathcal{F}\) erfüllt die ursprünglich geforderten Anforderungen an die Genauigkeit, während \(\mathcal{G}\) diese nicht erfüllen muss, jedoch eine deutlich kürzere Rechenzeit erfordern soll. Erreicht werden kann dies durch die Verwendung von einer geringeren Anzahl an Zeitschritten, oder einem weniger komplexen Diskretisierungsverfahren. Die Parallelität entsteht durch Gebietszerlegung, also dem Lösen der Gleichung an mehreren Zeitpunkten zur gleichen Zeit. Um dies zu organisieren, wird die Simulationszeit \(\left(t_0, T\right] = \bigcup_{i=0}^{P-1} \left(t_i, t_{i+1}\right]\) in \(P\) Unterintervalle aufgeteilt. Die zur Lösung der Unterintervalle notwendige Rechenzeit sollte möglichst ähnlich sein. Gleichzeitig ist es geschickt, die Schnittstellen \(\left\{t_i \,\middle|\, 1 \leq i \leq P-1\right\}\) jeweils so zu wählen, dass der zu erwartende lokale Fehler möglichst gering ist. Da jedes Unterintervall parallel simuliert werden soll, werden Startwerte benötigt. Zu diesem Zweck wird \(\mathcal{G}\) auf \(\left(t_0, T\right]\) ausgeführt.
\begin{displaymath}
    y_{j}^{0} = \mathcal{G}\!\left(y(t_0), t_0, t_{j}\right), \quad j \in \left\{i \,\middle|\, 0 \leq i < P\right\}
\end{displaymath}
Der grobe Löser wird hier durch eine ternäre Funktion \(\mathcal{G}\!\left(y, t_{A}, t_{\Omega}\right) \approx y(t_\Omega)\) abstrahiert. Das erste Argument enthält den Startwert, das zweite den Startzeitpunkt und das letzte den Endzeitpunkt der Berechnung. Der Wert der Funktion ist eine Approximation für die Lösung der Differentialgleichung zum Zeitpunkt \(t_\Omega\). \(y_j^k\) bezeichnet den Startwert für das Unterintervall \(j\) in der Iteration \(k\). In jeder Iteration wird, ausgehend vom gemeinsamen approximierten Startwert, sowohl \(\mathcal{G}\), als auch \(\mathcal{F}\) ausgeführt. Erlangt wird der Startwert durch die Kombination der Ergebnisse der Löser im zeitlich direkt vorhergehenden Unterintervall.
\begin{displaymath}
    y_{j+1}^{k+1} = \mathcal{G}\!\!\left(y_j^{k+1}, t_j, t_{j+1}\right) + \mathcal{F}\!\!\left(y_j^k, t_j, t_{j+1}\right) - \mathcal{G}\!\!\left(y_j^k, t_j, t_{j+1}\right)
\end{displaymath}
Offensichtlich würde \(y_{j+1}^{k+1} = \mathcal{F}\!\!\left(y_j^{k+1}, t_j, t_{j+1}\right)\) zu einem optimalen Startwert führen, jedoch ließen sich hier keine Gewinne durch Parallelisierung realisieren. Die Idee ist aber dieses Ergebnis möglichst genau vorherzusagen. Dazu wird die Auswirkung des neuen Startwerts auf das Ergebnis von \(\mathcal{F}\) durch die Auswirkung auf \(\mathcal{G}\) abgeschätzt. So kann der systematische Fehler von \(\mathcal{G}\) in eingeschänktem Maß isoliert werden. Diese Vorgehensweise kann anhand von Abbildung~\ref{fig:merge} nachvollzogen werden.
\begin{figure}[ht]
    \centering
        \begin{tikzpicture}[scale=2.2]
            % \documentclass[]{minimal}
% \usepackage{tikz}
% \usetikzlibrary{positioning}
% \usetikzlibrary{arrows}
% \usetikzlibrary{patterns}

% \begin{document}

% \pagestyle{empty}

% \begin{tikzpicture}[scale=2.2]


% coordinates
\draw[->] (0,0) -- (0,3) node[pos=1,below left] {y};

\draw[-] (0,0) -- (.25,0);
\draw[dotted] (.25,0) -- (.5,0);
\draw[-] (.5,0) -- (5.5,0);
\draw[dotted] (5.5,0) -- (5.75,0);
\draw[->] (5.75,0) -- (6,0) node[pos=1,below left] {x};

% threads
\draw[loosely dashed] (1,0) -- (1,3.3);
\draw[loosely dashed] (3,0) -- (3,3.3);
\draw[loosely dashed] (5,0) -- (5,3.3);
\node[below] at (1,0) {$t_{i-1}$};
\node[below] at (3,0) {$t_{i}$};
\node[below] at (5,0) {$t_{i+1}$};

% exact
\draw[thick] (.5,-.3) parabola  bend +(5,2.6) (5.5,2.3);

\begin{scope}[very thick]

% pCoarse
\begin{scope}[loosely dotted, color=blue]
\draw (1,.8) to[out=45,in=215] (3,2.6);
\draw (3,2.15) to[out=39,in=192] (5,2.95);
\end{scope}

% pFine
\begin{scope}[densely dotted, color=blue]
\draw (1,.8) to[out=44,in=212] (3,2.35);
\draw (3,2.15) to[out=28,in=186] (5,2.75);
\end{scope}

% coarse
\begin{scope}[loosely dotted, color=orange]
\draw (1,.5) to[out=45,in=215] (3,2.2);
\draw (3,1.95) to[out=32,in=188] (5,2.7);
\end{scope}
\end{scope}

% merges
\begin{scope}[red, ultra thick]
\draw[->, dashed] (2.96,2.6) -- (2.96,2.2);
\draw[->] (3.04,2.35) -- (3.04,1.95);

\draw[->, dashed] (4.96,2.95) -- (4.96,2.7);
\draw[->] (5.04,2.75) -- (5.04,2.5);
\end{scope}

%legend
\draw[] (3.25,.25) rectangle (4.75, 1.65);

\begin{scope}[anchor=west]
    \node[] at (3.8,1.45) {fine};
    \node[] at (3.8,1.25) {coarse};
    \node[] at (3.8,.95) {$i_k$};
    \node[] at (3.8,.75) {$i_{k+1}$};
    \node[] at (3.8,.45) {correction};
\end{scope}

\draw[-, dotted,very thick] (3.4,1.45) -- (3.7,1.45);
\draw[-, loosely dotted,very thick] (3.4,1.25) -- (3.7,1.25);
\draw[pattern color=blue,pattern=north east lines] (3.4,.9) rectangle (3.7, 1);
\draw[pattern color=orange,pattern=north east lines] (3.4,.7) rectangle (3.7, .8);
\draw[->,red, ultra thick] (3.4,.45) -- (3.7,.45);

% \end{tikzpicture}

% \end{document}
        \end{tikzpicture}
    \caption{Entstehung der Korrektur}
    \label{fig:merge}
\end{figure}
Mit jeder Iteration \(i_k\) entspricht ein weiteres Unterintervall dem Ergebnis von \(\mathcal{F}\) auf dem Gesamtintervall, da sich der ursprüngliche Anfangswert propagiert. Somit müssen die ersten \(k\) Unterintervalle in \(i_k\) nicht mehr berechnet werden. Dort ist er Fixpunkt lokal bereits erreicht. Dadurch ergeben sich auch die Startwerte für eine neue Iteration.
\begin{displaymath}
    y_{j+1}^{k+1} = \mathcal{F}\!\!\left(y_{j}^{k}, t_j, t_{j+1}\right) = \mathcal{F}\!\!\left(y_0, t_0, t_{j+1}\right), \quad j = k
\end{displaymath}

\begin{figure}[ht]
    \centering
        \documentclass[]{minimal}
\usepackage{tikz}
\usetikzlibrary{positioning}
\usetikzlibrary{arrows}

\begin{document}

\pagestyle{empty}

\begin{tikzpicture}[scale=2]
    \newcommand\Threads{4}
    \newcommand\Threadsmm{3}

    % Y-Axis
    \draw[->] (-.3,.3) -- (-.3,-1*\Threads-1.7) node[pos=1,below left, sloped] {Iteration};
    \foreach \x in {0,...,\Threads} {
        \draw (-.4, -1.1*\x) -- (-.2, -1.1*\x) node [pos=0, left] {\x};
    }

    % X-Axis
    \draw[->] (-.3,.3) -- (.75*\Threads+1.4,.3) node[pos=1,above left] {Thread};
    \foreach \x in {0,...,\Threads} {
        \draw(.75*\x, .2) rectangle (.5+0.75*\x, .4);
        \node at (0.75*\x+.25,.5) {\x};
        }


    % Computatations
    \foreach \y in {0,...,\Threads} {
        \foreach \x in {\y,...,\Threads} {
            \draw[fill=red] (.75*\x,-\y-.1-.1*\x) rectangle (.5+0.75*\x, -\y-.1*\x);
            \draw[fill=green] (.75*\x,-\y-1-.1*\x) rectangle (.5+0.75*\x, -\y-.1-.1*\x);
        }
    }

    % Communication
    % init
    \foreach \x in {0,...,\Threadsmm} {
        \draw[->] (.5+0.75*\x,-.1*\x-.1) -- (.75+0.75*\x,-.1*\x-.1);
    }
    % first of every iteration
    \foreach \x in {0,...,\Threadsmm} {
        \draw[->] (.5+0.75*\x,-1.1*\x-1) -- (.75+0.75*\x,-1.1*\x-1.1);
    }
    % merges
    \foreach \y in {1,...,\Threadsmm} {
        \foreach \x in {\y,...,\Threadsmm} {
        % previous coarse
        \draw[shorten >=0.08cm] (.5+0.75*\x,-\y-.1*\x+.9) -- (.71+0.75*\x,-\y-.1-.1*\x);
        % fine
        \draw[ shorten >=0.09cm] (.5+0.75*\x,-\y-.1*\x) -- (.71+0.75*\x,-\y-.1-.1*\x);
        % coarse
        \draw[-o, shorten >=0.0cm] (.5+0.75*\x,-\y-.1*\x-.1) -- (.75+0.75*\x,-\y-.1-.1*\x);
        }
    }



\end{tikzpicture}


\end{document}
    \caption{Auslastung und Kommunikation der Threads}
\end{figure}
