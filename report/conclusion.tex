Mit dieser Arbeit wird die Grundlage für eine Integration vom Parareal Algorithmus in einen vollwertigen Simulator gelegt. Die Implementierung im Python kann für weitere Experimente dienen. Im Kontext der Arbeit wurden damit verschiende Abbruchkriterien evaluiert. Dabei hat sich die Größe der Sprungstellen in der Lösung einer Iteration, als vielversprechend hervorgetan. Darüber hinaus kann sie auch als Richtwert für das Verhalten des Schaltungssimulators verwendet werden. Bei der Erweiterung des Experimentalsimulators Tictac, konnte eine Strategie zur Application von Parareal gefunden werden. Die Korrekturen finden nur auf einem Teil des übermittelten Zustandes statt. Dabei entsteht eine Inkonsistenz, welche durch erneutes Anwenden des Newton Verfahrens aufgelöst wird.\\

Um zu einer abschließenden Lösung zu gelangen, sind jedoch noch einige Fragen zu beantworten. Zunächst gilt es das Verhalten des Simulators mit größeren und komplexeren Schaltungen zu überprüfen. Auch weitere Bauteilmodelle sind hier sinnvoll. Nachdem die Schrittweite typischerweise nicht konstant ist, wäre es wichtig dies bei der Wahl der Unterintervallgrenzen zu berücksichtigen. Dadurch kann eine möglichst gleichmäßige Lastverteilung auf die Prozesse gewährleistet werden. Es besteht jedoch ein Zielkonflikt mit dem Wunsch die Grenzen zu möglichst stabilen Zeitpunkten zu setzen. Um eine nutzerfreundliche Interaktion zu garantieren, sollten noch einige Aufgaben automatisiert werden. Das Abbruchkriterien sollte auch im Simulator implementiert und getestet werden. Auch sollte eine Möglichkeit zur automatischen Konfiguration des groben Lösers gefunden werden. Schließlich ist steht auch noch die tatsächliche Parallelisierung von Tictac aus. Zwar sollten die im Rahmen der Arbeit vorgenommenen Änderungen dies mit relativ geringem Aufwand ermöglichen, der Beweis hierfür steht jedoch aus.