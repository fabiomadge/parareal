\documentclass[12pt,a4paper]{article}
\usepackage[utf8]{inputenc}
\usepackage{cite}
\usepackage[ngerman]{babel}
\title{Projektbeschreibung\\
    \large Parallele Integration für die Schaltungssimulation}
\author{}
\date{}

\begin{document}
\maketitle

In der Physik lassen sich viele Vorgänge durch gewöhnliche Differentialgleichungen beschreiben. Im Allgemeinen existiert aber keine exakte geschlossene Lösung für eine solche Gleichung, oder ist nicht sinnvoll berechenbar, dadurch wird der Einsatz numerischer Verfahren notwendig. Die populären Verfahren zur Lösung des Anfangswertproblems (AWP) haben alle gemein, dass sie sich als Rekursionsgleichung in der Zeit darstellen lassen. Der Ansatz ist hier das gewünschte Ergebnis nicht direkt zu Berechnen, sondern das Problem in kleinere Probleme zu teilen und auf diese die gleiche Strategie anzuwenden, bis es trivial lösbar wird. Dabei ergibt sich folglich eine Abhängigkeitshierarchie, welche eine weitgehend sequenzielle Berechnung notwendig macht. Abhängig von der Größe des Problems und der gewünschten Genauigkeit, kann so eine Berechnung viele Stunden dauern. Diese Laufzeiten haben sich in den letzten Jahren auch nicht wesentlich verbessert, nachdem sich weder die Taktfrequenzen der Prozessoren, noch der numerischen Algorithmen wesentlich verbessert haben. Um das Hardwareproblem auszugeichen, haben die Hersteller begonnen mehrere Prozessoren in einem Paket zu bündeln.\\
Mehrere Prozessoren auszunutzen ist im Kontext eines sequenziellen Programms im Allgemeinen sehr schwierig. Die Programme müssen einzeln auf potenziell nebenläufige Abschnitte untersucht werden. Bei den Algorithmen zur Lösung des Anfangswertproblems ist auch dieser naive Ansatz aufgrund der starken Abhängigkeiten nicht ausreichend. Ein Ansatz für eine Parallelisierung ist "`parallel in time"', also Unterteilung des zu berechnenden Zeitraums in mehrere abhängige AWPs. Die Idee des Parareal Algorithmus ist es einen groben mit einem feinen Lösungsverfahren zu kombinieren. Zunächst wird des grobe Verfahren verwendet um Abschätzungen für die Anfangswerte der Zeitabschnitte zu gewinnen. In einer Fixpunktiteration werden die Ergebnisse der beiden Verfahren genommen um die Qualität der Schätzung mit jeder Iteration zu steigern.\\
Die Arbeit soll sich mit den Anwendung in der analogen Schaltungssimulation beschäftigen. Die einzelnen Bauteile einer analogen Schaltung lassen sich durch gleichungen Darstellen. Zusammen mit der Anordnung der Elemente lässt sich ein Gleichungssystem aufstellen. Sobald dynamische Elemente vorkommen, müssen auch Differentialgleichungen gelöst werden. Ziel ist es, in diesem Kontext die Eignung des Parareal Algorithmus zu untersuchen, um schneller, größere Schaltungen zu analysieren.

\subsection*{Meilensteine}
\subsubsection*{Literaturstudie}
Beginnend mit~\cite{Gander:2015} betreibt der Student eine Literaturstudie. Relevante Themen sind hier Konvergenz und Effizienz. Des weiteren soll sich der Student auch mit der analogen Schaltungssimulation vertraut machen. Letztlich soll er sich einen Überblick über die aktuelle Forschungslage im Bereich der AWPs verschaffen.
\subsubsection*{Prototyping}
Um eine Vergleichsgrundlage zu schaffen und das Verhalten des Parareal Algorithmus näher kennenzulernen, beginnt der Student seine praktische Arbeit, indem er einige Methoden zum Lösen von AWPs implementiert und an verschiedenen Beispielen, wie der Logistischen Funktion testen. Diese Methoden kombiniert er anschließend im Parareal Algorithmus. Zunächst kann die Implementation sequenziell sein, sollte aber anschließend echt parallel werden.
\subsubsection*{Umsetzung}
Ein einfacher Simulator, welcher "`SPICE"'-Netzlisten verarbeiten und einige Analysen durchführen kann, soll erweitert werden. Er ist in C\texttt{++} geschrieben und soll um "`parallel in time"' Integration erweitert werden. Die vorhandene Methode zur Lösung der AWPs, arbeitet mit dynamischer Schrittweite. Der Student verwendet diese Lösungsmethode mit unterschiedlichen Toleranzen um die grobe und die feine Methode zu erhalten. Das Ergebnis evaluiert er anhand verschiedener Beispielnetzlisten.

\subsection*{Zeitplan}

\begin{tabular}{ r l }
    09.04.2018 & Beginn MA2304\\
    31.07.2018 & Prüfung MA2304\\
    01.09.2018 & Beginn Projektarbeit: Literaturstudie\\
    15.09.2018 & Beginn Prototyp\\
    01.01.2019 & Beginn Umsetzung\\
    01.03.2019 & Abschluss Praktischer Teil\\
    01.04.2019 & Dokumentation\\
    15.04.2019 & Präsentation\\
  \end{tabular}

\subsection*{Vorlesung}

Um für die Literaturstudie vorzubereiten, besucht der Student die Vorlesung: Numerik gewöhnlicher Differentialgleichungen [MA2304]. Einen großen Teil der Vorlesung stellt die Vorstellung und Bewertung verschiedener Methoden zum numerischen Lösen von AWPs dar. Diese Methoden werden im Parareal Algorithmus kombiniert.

\bibliographystyle{abbrv}
\bibliography{cites}

\end{document}